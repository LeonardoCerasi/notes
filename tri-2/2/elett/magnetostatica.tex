\section{Magnetostatica}

\subsection{Forza di Lorentz}

Consideriamo una carica elettrica $ q $ con velocità $ \vec{v} $ immersa in un campo magnetico (o campo di induzione magnetica, u.d.m. tesla $ \text{T} = \text{N}\cdot\text{s}\cdot\text{C}^{-1}\cdot\text{m}^{-1} $) $ \vec{B} $: sperimentalmente, si osserva che la forza agente sulla carica è pari a $ \vec{F_B} = q \vec{v}\times\vec{B} $.\\
%
Se consideriamo anche la presenza di un campo elettrico $ \vec{E} $, otteniamo l'espressione della forza di Lorentz:
\begin{equation}
	\vec{F}_L = q ( \vec{E} + \vec{v} \times \vec{B} )
	\label{eq:f-lotentz}
\end{equation}
Per misurare $ \vec{B} $ sono necessarie almeno due misure di velocità e forza, poiché, data una forza ed una velocità, esistono infinite combinazioni di $ \vec{B} $ e $ \theta $ che danno lo stesso prodotto vettoriale, e inoltre se $ \vec{B} $ genera una determinata forza, anche $ \vec{B} + k\vec{v} $ genera la stessa forza; con due coppie di misure $ (\vec{v}_1, \vec{F}_1 $ e $ (\vec{v}_2, \vec{F}_2 $ si ottiene:
\begin{equation}
	\vec{B} = \displaystyle\frac{1}{qv_1^2} \left[ \vec{F}_1 \times \vec{v}_1 + \displaystyle\frac{(\vec{F}_2 \times \vec{v}_2) \cdot \vec{v}_1}{v_2^2} \vec{v_1} \right]
	\label{eq:}
\end{equation}
%
È importante notare che la forza magnetica non compie lavoro: infatti, essendo essa sempre perpendicolare a $ \vec{v} $, si ha che $ dW_B = \vec{F_B} \cdot d\vec{l} = \vec{F_B} \cdot \vec{v} \, dt = 0$; pertanto, essa modificherà soltanto la direzione della velocità della carica, ma non il suo modulo.\\
%
Inoltre, sempre dato che $ \vec{F}_B \perp \vec{v} $, essa darà luogo ad un moto a spirale attorno alla linea di campo magnetico, dato dalla somma del moto rettilineo uniforme di $ \vec{v}_{\parallel} $ e di quello centripeto di $ \vec{v}_{\perp} $: questo moto a spirale è detto moto di ciclotrone. Possiamo calcolare il raggio di girazione $ r_g $ dell'orbita e la rispettiva frequenza di ciclotrone $ \nu_c $ imponendo $ F_B = F_c $, ottenendo:
\begin{equation}
	r_g = \displaystyle\frac{m v_{\perp}}{q B} \qquad \nu_c = \displaystyle\frac{q B}{2\pi m}
	\label{eq:eq-ciclotrone}
\end{equation}

\subsubsection{Forza magnetica su correnti elettriche}

Consideriamo un filo conduttore di sezione $ S $ in cui scorre una corrente $ i $, immerso per un tratto $ L $ in un campo magnetico uniforme $ \vec{B} $ perpendicolare al filo. \\ 
%
Se nel conduttore abbiamo $ n $ cariche per unità di volume, allora in un tratto $ dl $  si avranno $ n \, S \, dl $ cariche, dunque la forza totale esercitata sul tratto dal campo elettrico sarà $ d\vec{F_B} = q \, n \, S \, dl \, \vprod{v}{B} $; d'altra parte, si ha che $ q \, n = \rho $ densità di carica e $ \rho \vec{v} = \vec{J} $ densità di corrente, dunque $ d\vec{F_B} = S \, dl \, \vprod{J}{B} $, ma $ S \, J = i $, quindi possiamo assegnare il verso di $ \vec{J} $ a $ dl $, ottenendo:
\begin{equation}
	d\vec{F}_B = i \, d\vprod{l}{B}
	\label{eq:forza-filo}
\end{equation}
Dato che la forza magnetica viene esercitata sulle cariche in moto nel conduttore, ma queste interagiscono meccanicamente con la sua struttura atomica, la forza agisce sull'intero conduttore, provocando lo spostamento del filo. \\ 
%
Prendendo una distribuzione di carica in un campo elettrico uniforme $ \vec{B} $, assumendo che le cariche si muovano tutte con la stessa velocità $ \vec{v} $, si ha $ d\vec{F}_B = dq \, \vprod{v}{B} = \rho \, dV \, \vprod{v}{B} = \vprod{J}{B} \, dV $, quindi possiamo definire la densità di forza magnetica come:
\begin{equation}
	\displaystyle\frac{d\vec{F}_B}{dV} = \vprod{J}{B}
	\label{eq:dens-f-mag}
\end{equation}

\paragraph{Densità di corrente}

Possiamo generalizzare il concetto di densità di corrente a superfici e linee: considerando un conduttore di percorso da una densità di corrente $ \vec{J} $, se prendiamo una superficie infinitesima approssimativamente rettangolare avremo $ dS = ds \, dl $ e $ di = \vec{J}\cdot d\vec{S} = \rho \vec{v}\cdot\hat{n} \, ds \, dl $. \\ 
%
Facendo tendere $ ds \rightarrow 0 $ si ha $ \rho \, ds \rightarrow \sigma $ densità superficiale di carica e $ di = \sigma \vec{v} \cdot \hat{n} \, dl $, quindi possiamo definire la densità superficiale di corrente $ \vec{K} \equiv \sigma\vec{v} $ (u.d.m. $ \text{C} \cdot \text{s}^{-1} \cdot \text{m}^{-1} = \text{A}  \cdot \text{m}^{-1} $): la corrente che scorre attraverso una superficie arbitraria attraverso una data linea è:
\begin{equation}
	i = \int_{\gamma} \vec{K}\cdot d\vec{l}
	\label{eq:dens-sup-corr}
\end{equation}
%
Se invece facciamo tendere anche $ dl \rightarrow 0 $ si ha $ \rho \, ds \, dl \rightarrow \lambda $ densità lineare di carica e $ i = \lambda \vec{v} \cdot \hat{n} $, quindi definiamo la densità lineare di corrente $ \vec{I} \equiv \lambda \vec{v} $ (u.d.m. $ \text{A} $: dato che rimane una sola dimensione, $ \vec{I} $ e $ \hat{n} $ sono sempre paralleli, quindi il modulo di $ \vec{I} $ corrisponde di fatto alla corrente. \\ 
%
A seconda delle varie distribuzioni di corrente, la forza magnetica agente è data da:
\begin{equation}
	\vec{F}_B = \iiint_V \vprod{J}{B} \, dV \qquad \vec{F}_B = \oiint_S \vprod{K}{B} \, dS \qquad \vec{F}_B = \oint_{\gamma} \vprod{I}{B} \, dl
	\label{eq:distr-corr}
\end{equation}

\subsection{Legge di Biot-Savart}

A differenza della legge di Coulomb, la forza di Lorentz non dice nulla sull'ontologia dei generatori del campo magnetico: date le osservazioni sperimentali, è ragionevole supporre che la sorgente del campo magnetico sia legata alla corrente elettrica, ma la questione è più complessa (ad esempio i magneti permanenti, che richiedono una teoria quantistica della materia). \\ 
%
Consideriamo un filo percorso da una corrente stazionaria $ i $ ed un suo elemento infinitesimo $ dl $; empiricamente si trova che il campo magnetico $ d\vec{B} $ generato da $ dl $ in un punto a distanza $ \vec{r} $ da esso è dato dalla legge di Biot-Savart:
\begin{equation}
	d\vec{B} = \displaystyle\frac{\mu_0}{4\pi} \displaystyle\frac{i \, d\vec{l} \times \hat{r}}{r^2}
	\label{eq:biot-savart}
\end{equation}
dove $ d\vec{l} $ ha il verso di $ i $ e $ \mu_0 = 4\pi \cdot 10^{-7} \, \text{T} \, \text{m} \, \text{A}^{-1} $ è detta permeabilità magnetica del vuoto. \\ 
%
Questa legge vale solo in condizioni stazionarie (corrente continua): in queste condizioni $ \rho $ non dipende dal tempo, e dato che $ \dive\vec{J} = -\frac{\partial \rho}{\partial t} $, la legge di Biot-Savart vale quando $ \dive\vec{J} = 0 $.

\subsubsection{Filo percorso da corrente}

Consideriamo un filo abbastanza lungo percorso da una corrente stazionaria $ i $ e concentriamoci nella porzione lontana dalle estremità, così da poterlo approssimare come infinito: in questo caso, è evidente che il campo è indipendente dalla posizione lungo il filo, dunque possiamo fissare un sistema di assi con origine lungo il filo ed asse $ x $ lungo esso. \\ 
%
Dato che il sistema gode di evidente simmetri cilindrica, calcoliamo il campo magnetico in un punto $ P $ ad una generica distanza $ a $ dal filo, WLOG lungo l'asse $ y $: prendendo il generico elemento di filo dal quale $ P $ dista $ \vec{r} $, si ha $ d\vec{l} = dl \, \hat{e}_x \equiv dx \hat{e}_x $ e $ \hat{r} = -\cos{\theta} \, \hat{e}_x + \sin{\theta} \, \hat{e}_y $, quindi $ d\vec{l} \times \hat{r} = dl \sin{\theta} \, \hat{e}_z $; dato che $ x = r \cos{(\pi - \theta)} = - r \cos{\theta} $ e $ r = a / \sin{(\pi - \theta)} = a / \sin{\theta} $, si ha $ x = - a \cot{\theta} $, dunque $ dx = a \csc^2{\theta} \, d\theta $ e possiamo scrivere:
\begin{equation}
	dB = \displaystyle\frac{\mu_0}{4\pi} \displaystyle\frac{i\, dx \, \sin{\theta}}{r^2} = \displaystyle\frac{\mu_0}{4\pi} \displaystyle\frac{i}{a} \, \sin{\theta} \, d\theta
	\label{eq:filo-calc}
\end{equation}
Integrando su tutto il filo ($ \theta \in [0, \pi] $) e passando alle coordinate cillindriche per meglio esprimere il fatto che le linee di campo sono circonferenze centrate nel filo:
\begin{equation}
	\vec{B}(\vec{r}) = \displaystyle\frac{\mu_0 i}{2\pi r} \hat{e}_{\phi}
	\label{eq:mag-filo-corr}
\end{equation}

\subsubsection{Spira circolare percorsa da corrente}

Consideriamo ora una spira circolare di raggio $ a $ percorsa da una corrente stazionaria $ i $ e fissiamo un sistema di riferimento con origine nel centro della spira ed asse $ z $ ortogonale ad essa. \\ 
%
Vogliamo calcolare il campo magnetico in un punto lungo l'asse della spira, ovvero $ \vec{r}_2 = z \, \hat{e}_z $: dato che il generico elemento $ d\vec{l} $ ha posizione $ d\vec{r}_1 = a (\cos{\theta} \, \hat{e}_x + \sin{\theta} \, \hat{e}_y) $, si ha $ d\vec{l} = \frac{d\vec{r}_1}{d\theta}d\theta $ e quindi:
\begin{equation}
	\begin{split}
		d\vec{l} \times (\vec{r}_2 - \vec{r}_1) &= (-a \sin{\theta} \, \hat{e}_x + a \cos{\theta} \, \hat{e}_y) \times (z \, \hat{e}_z - a \sin{\theta} \, \hat{e}_x - a \sin{\theta} \hat{e}_y) \\ 
							&= \left[ az \cos{\theta} \, \hat{e}_x + az \sin{\theta} \, \hat{e}_y + a^2 (\sin^2{\theta} + \cos^2{\theta}) \, \hat{e}_z \right] d\theta \\ 
							&= (z \cos{\theta} \, \hat{e}_x + z \sin{\theta} \, \hat{e}_y + a \, \hat{e}_z) \, a \, d\theta
	\end{split}
	\label{eq:}
\end{equation}
Inoltre, abbiamo che $ \abs{\vec{r}_2 - \vec{r}_1}^3 = (a^2 + z^2)^{3/2} $, dunque:
\begin{equation}
	d\vec{B} = \displaystyle\frac{\mu_0 i a}{4\pi} \displaystyle\frac{z \cos{\theta} \, \hat{e}_x + z \sin{\theta} \, \hat{e}_y + a \, \hat{e}_z}{(a^2 + z^2)^{3/2}} d\theta
	\label{eq:}
\end{equation}
Integrando su tutta la spira ($ \theta \in [0, 2\pi] $) si ottiene:
\begin{equation}
	\vec{B}(z) = \displaystyle\frac{\mu_0 i a^2}{2(a^2 + z^2)^{3/2}}
	\label{eq:mag-spir}
\end{equation}
Il calcolo del campo in posizioni diverse dall'asse della spira è complesso e richiede l'utilizzo di integrali ellittici: qualitativamente, le linee di campo sono linee chiuse che avvolgono la spira, diventando sempre più circolari via via che si avvicinano ad essa.

\subsubsection{Monopoli magnetici}

Al contrario del campo elettrico, il campo magnetico non ha sorgenti definite, ma è associato alla presenza di correnti elettriche (anche nel caso dei magneti permanenti, associati alle correnti atomiche): fino a prova sperimentale contraria, possiamo affermare che il campo magnetico non ha sorgenti, ovvero che i monopoli magnetici non esistono. \\ 
%
Di conseguenza, per la legge di Gauss, ci aspettiamo che il campo magnetico sia solenoidale, ed è proprio così. \\ 
%
Consideriamo una distribuzione di corrente $ \vec{J}(\vec{r}) $: dato che $ \vec{i} = \vec{J} S $ e che $ \vec{J} \parallel d\vec{l} $, possiamo scrivere la legge di Biot-Savart come:
\begin{equation}
	d\vec{B} = \displaystyle\frac{\mu_0}{4\pi} \displaystyle\frac{\vec{J}(\vec{r}_1) \times (\vec{r}_2 - \vec{r}_1)}{\abs{\vec{r}_2 - \vec{r}_1}^3} S \, dl 
	\label{eq:bio-savart-2}
\end{equation}
dunque considerando superfici normali al percorso della corrente, possiamo trasformare l'integrale di linea in $ S\,dl $ in un integrale di volume in $ dV $ su tutto il volume del conduttore in cui scorre corrente, ottenendo la forma integrale della legge di Biot-Savart:
\begin{equation}
	\vec{B}(\vec{r}) = \displaystyle\frac{\mu_0}{4\pi} \iiint_V \displaystyle\frac{\vec{J}(\vec{r}\,') \times (\vec{r} - \vec{r}\,')}{\abs{\vec{r} - \vec{r}\,'}^3} \, dV'
	\label{eq:bio-savart-int}
\end{equation}
Calcoliamo esplicitamente la divergenza di $ \vec{B} $:
\begin{equation}
	\begin{split}
		\dive\vec{B}(\vec{r}) &= \displaystyle\frac{\mu_0}{4\pi} \iiint_V \nabla_r \cdot \displaystyle\frac{\vec{J}(\vec{r}\,') \times (\vec{r} - \vec{r}\,')}{\abs{\vec{r} - \vec{r}\,'}^3} \, dV' \\ 
				      & \qquad\qquad \dive (\vec{A}\times\vec{B}) = (\dive\vec{A})\cdot\vec{B} - \vec{A}\cdot (\dive\vec{B}) \\ 
				      & \qquad\qquad \nabla_r \times\vec{J}(\vec{r}\,') = 0 \\ 
				      &= -\displaystyle\frac{\mu_0}{4\pi} \iiint_V \vec{J}(\vec{r}\,') \cdot \nabla_r \times \left(\displaystyle\frac{\vec{r} - \vec{r}\,'}{\abs{\vec{r} - \vec{r}\,'}^3}\right) dV' = 0
	\end{split}
	\label{eq:}
\end{equation}
che è proprio quello che avevamo previsto col teorema di Gauss.

\subsection{Legge di Ampère}

Per cercare una relazione tra campo magnetico e correnti elettriche, dato che le linee di campo sono sempre chiuse, ragioniamo sulla circuitazione di $ \vec{B} $. \\ 
%
Consideriamo un filo infinito percorso da corrente; abbiamo visto che le linee di campo sono circonferenza attorno al filo, dunque prendiamo come loop per calcolare la circuitazione il rettangolo curvilinee delimitato tra due linee di campo, a distanze $ \vec{r}_1 $ e $ \vec{r}_2 $, determinato da un angolo $ \Delta\theta $:
\begin{equation}
	\begin{split}
		\Gamma &\equiv \oint_{\gamma} \vec{B} \cdot d\vec{l} = \int_{\gamma_1} \vec{B} \cdot d\vec{l} + \int_{\gamma_{12}} \vec{B} \cdot d\vec{l} + \int_{\gamma_2} \vec{B} \cdot d\vec{l} + \int_{\gamma_{21}} \vec{B} \cdot d\vec{l} \\ 
		       & \qquad\qquad \vec{B}\cdot d\vec{l}_{12} = \vec{B}\cdot d\vec{l}_{21} = 0 \\ 
		       &= - \displaystyle\frac{\mu_0 i}{2\pi r_1} r_1 \Delta\theta + \displaystyle\frac{\mu_0 i}{2\pi r_2} r_2 \Delta\theta = 0
	\end{split}
	\label{eq:circ-1-filo-inf}
\end{equation}
Se invece prendiamo un loop circolare centrato nel filo:
\begin{equation}
	\Gamma = \displaystyle\frac{\mu_0 i}{2\pi r} 2\pi r = \mu_0 i
	\label{eq:circ-2-filo-inf}
\end{equation}
La differenza tra i due casi è che il primo loop non ha nessuna corrente concatenata ad esso. \\ 
%
Dato che qualsiasi loop generico può essere approssimato con tratti radiali ed archi di circonferenze, la validità di quanto appena mostrato è generale e si esprime tramite la legge di Ampère: la circuitazione del campo magnetico lungo un loop di $ N $ avvolgimenti attorno a $ M $ correnti stazionarie concatenate è:
\begin{equation}
	\Gamma = N \mu_0 \displaystyle\sum_{k = 1}^{M} i_k
	\label{eq:ampere}
\end{equation}
Possiamo generalizzare ad una distribuzione di corrente $ \vec{J} $ stazionaria ($ \div\vec{J} = 0 $):
\begin{equation}
	\Gamma = \mu_0 \oiint_S \vec{J} \cdot d\vec{S}
	\label{eq:ampere-int}
\end{equation}
Utilizzando il teorema di Stokes, otteniamo la forma differenziale della legge di Ampère:
\begin{equation}
	\rot\vec{B} = \mu_0 \vec{J}
	\label{eq:ampere-diff}
\end{equation}
Nel calcolo del rotore, analogamente a quello della divergenza, dobbiamo prestare attenzione alle singolarotà derivanti da oggetti che matematicamente sono monodimensionali (fili ideali) ma nella realtà non lo sono: per trattare queste singolarità, si fa di nuovo ricorso alla $ \delta $ di Dirac. \\ 
%
Possiamo anche ottene esplicitamente la forma differenziale della legge di Ampère a partire da quella di Biot-Savart:
\begin{equation}
	\begin{split}
	 	\rot\vec{B}(\vec{r}) &= \displaystyle\frac{\mu_0}{4\pi} \iiint_V \nabla_r \times \left( \displaystyle\frac{\vec{J}(\vec{r}\,') \times (\vec{r} - \vec{r}\,')}{\abs{\vec{r} - \vec{r}\,'}^3} \right) dV' \\
				     & \qquad\qquad \rot\vprod{A}{B} = (\vec{A}\cdot\nabla)\vec{B} - (\vec{B}\cdot\nabla)\vec{A} + \vec{A}(\dive\vec{B}) - \vec{B}(\dive\vec{A}) \\ 
				     &= \displaystyle\frac{\mu_0}{4\pi} \left[ \iiint_V \left( \vec{J}(\vec{r}\,') \cdot \nabla_r \right) \frac{\vec{r} - \vec{r}\,'}{\abs{\vec{r} - \vec{r}\,'}^3} \, dV' + \iiint_V \vec{J}(\vec{r}\,') \left( \nabla_r \cdot \frac{\vec{r} - \vec{r}\,'}{\abs{\vec{r} - \vec{r}\,'}^3} \right) dV' \right] \\ 
				     & \qquad\qquad \nabla_r \frac{\vec{r} - \vec{r}\,'}{\abs{\vec{r} - \vec{r}\,'}^3} = - \nabla_{r'} \frac{\vec{r} - \vec{r}\,'}{\abs{\vec{r} - \vec{r}\,'}^3} \\ 
				     &= \displaystyle\frac{\mu_0}{4\pi} \left[ -\displaystyle\sum_{k = 1}^{3} \iiint_V \left( \vec{J}(\vec{r}\,') \cdot \nabla_r \right) \frac{x_k - x'_k}{\abs{\vec{r} - \vec{r}\,'}^3} \, dV' + \iiint_V \vec{J}(\vec{r}\,') \, 4\pi \delta(\vec{r} - \vec{r}\,') \,  dV' \right] \\
				     &=  \displaystyle\frac{\mu_0}{4\pi} \left[ -\displaystyle\sum_{k = 1}^{3} \iiint_V \left[ \nabla_{r'} \left( \vec{J}(\vec{r}\,') \frac{x_k - x'_k}{\abs{\vec{r} - \vec{r}\,'}^3} \right) -  \frac{x_k - x'_k}{\abs{\vec{r} - \vec{r}\,'}^3} \nabla_{r'} \cdot \vec{J}(\vec{r}\,') \right] dV' + 4\pi\vec{J}(\vec{r}) \right] \\ 
				     & \qquad\qquad \dive\vec{J} = 0 \\ 
				     &= - \displaystyle\frac{\mu_0}{4\pi} \displaystyle\sum_{k = 1}^{3} \oiint_S \left( \vec{J}(\vec{r}\,') \frac{x_k - x'_k}{\abs{\vec{r} - \vec{r}\,'}^3} \right) \cdot d\vec{S}\,' + \mu_0 \vec{J}(\vec{r}) = \mu_0 \vec{J}(\vec{r})
	\end{split}
	\label{eq:}
\end{equation}
dove abbiamo usato il fatto che l'ultimo integrale si annulla considerando una superficie abbastanza grande poiché dipende da $ \abs{\vec{r} - \vec{r}\,'}^{-3} $.

\subsection{Potenziale Vettore}

Così come la conservaività del campo elettrico implica l'esistenza di un potenziale scalare, il potenziale elettrico ($ \rot\vec{E} = 0 \Rightarrow \vec{E} = - \nabla \phi $), anche  il fatto che il campo magnetico sia solenoidale implica l'esistenza di un potenziale, questa volta vettoriale, detto potenziale vettore: $ \dive\vec{B} = 0 \Rightarrow \vec{B} = \rot\vec{A} $. \\ 
%
Analogamente al potenziale scalare, definito a meno di una costante, anche il potenziale vettore è definito a meno del gradiente di una funzione scalare ($ \rot\nabla = 0 $). \\ 
%
Determiniamo il potenziale vettore a partire dala legge di Ampère:
\begin{equation}
	\rot\vec{B} = \rot\rot\vec{A} = \nabla(\dive\vec{A}) - \lap\vec{A} = \mu_0 \vec{J}
	\label{eq:}
\end{equation}
Si dimostra che si può sempre scegliere un'opportuna funzione scalare $ f : \dive(\vec{A} + \nabla f) = 0 $, dunque otteniamo un'equazione di Poisson (in realtà tre) anche per il potenziale vettore:
\begin{equation}
	\lap\vec{A} = - \mu_0 \vec{J}
	\label{eq:eq-pot-vett}
\end{equation}
La condizione di esistenza della soluzione è che $ \vec{J} $ all'infinito tenda a $ 0 $ più velocemente di $ r^{-2} $, ed in tal caso una soluzione particolare è:
\begin{equation}
	\vec{A}(\vec{r}) = \displaystyle\frac{\mu_0}{4\pi} \iiint_V \displaystyle\frac{\vec{J}(\vec{r}\,')}{\abs{\vec{r} - \vec{r}\,'}} dV'
	\label{eq:sol-eq-pot-vett}
\end{equation}
Nella maggior parte dei casi d'interesse le correnti non dipendono o dipendono poco dal campo magnetico: in tal caso, non vi sono particolari condizioni al contorno da porre, dunque la soluzione particolare \ref{eq:sol-eq-pot-vett} è la soluzione finale del problema.

\subsubsection{Potenziale di un campo magnetico uniforme}

Consideriamo il campo magnetico uniforme $ \vec{B} = B_0 \hat{e}_z $: l'equazione di Poisson diventa $ \partial_y A_x - \partial_x A_y = B_0 $, che non ha un'unica soluzione; una possibile soluzione è $ \vec{A} = \frac{B_0}{2} (y\hat{e}_x - x\hat{e}_y) $. \\ 
%
Si può dimostrare che se $\vec{B} $ è uniforme ed ha un'orientazione generica, un possibile potenziale vettore è:
\begin{equation}
	\vec{A}(\vec{r}) = \displaystyle\frac{1}{2} \vec{B}\times\vec{r}
	\label{eq:pot-b-unif}
\end{equation}

\subsubsection{Potenziale di un filo rettilineo}

Consideriamo un filo di raggio $ a $ percorso da una corrente stazionaria $ i $ e densità di corrente $ \vec{J} = \frac{i}{\pi a^2} \hat{e}_z $: il potenziale vettore avrà solo la componente $ A_z $ data da:
\begin{equation}
	\begin{split}
		A_z &= \displaystyle\frac{\mu_0}{4\pi} \iiint_V \displaystyle\frac{\vec{J}(\vec{r}\,')}{\abs{\vec{r} - \vec{r}\,'}} dS' dr' = \displaystyle\frac{\mu_0 i}{4\pi} \int_{\mathbb{R}} \displaystyle\frac{dz'}{\abs{\vec{r} - \vec{r}\,'}} \\ 
		    & \qquad\qquad \rho^2 \equiv x^2 + y^2 \\ 
		    &= \displaystyle\frac{\mu_0 i}{4\pi} \int_{\mathbb{R}} \displaystyle\frac{dz'}{\sqrt{\rho^2 + (z - z')^2}} = \displaystyle\frac{\mu_0 i}{4\pi} 2\int_0^{\infty} \displaystyle\frac{d\zeta}{\sqrt{\rho^2 + \zeta^2}}
	\end{split}
	\label{eq:pot-filo-rett}
\end{equation}
L'integrale ha primitiva $ \ln{\frac{\sqrt{\rho^2 + \zeta^2} + \zeta}{\rho}} $, la quale diverge a $ +\infty $; possiamo ovviare al problema integrando su un filo di lunghezza finita $ l $:
\begin{equation}
	A_z (\rho) = \displaystyle\frac{\mu_0 i}{2\pi} \ln\left[ \displaystyle\frac{l}{\rho} \left( \sqrt{1 + \displaystyle\frac{\rho^2}{l^2}} + 1 \right) \right]
	\label{eq:pot-filo-rett-fin}
\end{equation}
Nel limite $ \rho \ll l $ esso può essere approssimato da $ A_z (\rho) = \displaystyle\frac{\mu_0 i}{2\pi} \ln{\frac{2l}{\rho}} $.

\subsubsection{Potenziale di una spira}

Consideriamo una spira $ \varsigma $ di forma generica percorsa da una corrente stazionaria $ i $: il potenziale vettore sarà dato da:
\begin{equation}
	\begin{split}
		\vec{A}(\vec{r}) &= \displaystyle\frac{\mu_0 i}{4\pi} \oint_{\varsigma} \displaystyle\frac{d\vec{r}\,'}{\abs{\vec{r} - \vec{r}\,'}} = \displaystyle\frac{\mu_0 i}{4\pi} \oint_{\varsigma} \displaystyle\frac{d\vec{r}\,'}{\sqrt{r^2 + r'^2 - 2\sprod{r}{r}\,'}} = \displaystyle\frac{\mu_0 i}{4\pi r} \oint_{\varsigma} \left( 1 + \displaystyle\frac{r'^2}{r^2} - 2 \displaystyle\frac{\sprod{r}{r}\,'}{r^2} \right) d\vec{r}\,' \\ 
				 & \qquad\qquad r \gg r' \Rightarrow \abs{\vec{r} - \vec{r}\,'} \approx \frac{1}{r} \left(1 + \frac{\sprod{r}{r}\,'}{r^2}\right) \\ 
				 &= \displaystyle\frac{\mu_0 i}{4\pi r} \left[ \oint_{\varsigma} d\vec{r}\,' + \displaystyle\frac{1}{r^2} \oint_{\varsigma} \left( \sprod{r}{r}\,' \right) d\vec{r}\,' \right] = \displaystyle\frac{\mu_0 i}{4\pi r^3} \oint_{\varsigma} \left( \sprod{r}{r}\,' \right) d\vec{r}\,' \\ 
				 & \qquad\qquad \left(\sprod{r}{r}\,'\right) d\vec{r}\,' = \vec{r}\,' \left(\vec{r}\cdot d\vec{r}\,'\right) + \left(\vec{r}\,'\times d\vec{r}\,'\right)\times\vec{r} \qquad \qquad \qquad \text{integrale lungo } \vec{r}\,' \\ 
				 & \qquad\qquad \left(\vec{r}\cdot\vec{r}\,'\right) d\vec{r}\,' = d\left[\vec{r}\,'\left(\sprod{r}{r}\,'\right)\right] - \vec{r}\,'d\left(\sprod{r}{r}\,'\right) = d\left[\vec{r}\,'\left(\sprod{r}{r}\,'\right)\right] - \vec{r}\,'\left(\vec{r}\cdot d\vec{r}\,'\right) \\ 
				 & \qquad \qquad \Rightarrow \left(\sprod{r}{r}\,'\right)d\vec{r}\,' = \frac{1}{2} \left[ d\left[ \vec{r}\,' \left(\sprod{r}{r}\,'\right)\right] + \left(\vec{r}\,' \times d\vec{r}\,'\right) \times\vec{r} \right] \\ 
				 &= \displaystyle\frac{\mu_0 i}{4\pi r^3} \displaystyle\frac{1}{2} \left[ \oint_{\varsigma} d\left[ \vec{r}\,' \left(\sprod{r}{r}\,'\right)\right] + \oint_{\varsigma} \left(\vec{r}\,' \times d\vec{r}\,'\right) \times\vec{r} \right] = \displaystyle\frac{\mu_0}{4\pi r^3} \displaystyle\frac{i}{2} \oint_{\varsigma} \left(\vec{r}\,' \times d\vec{r}\,'\right) \times\vec{r} 
	\end{split}
	\label{eq:pot-vett-sp}
\end{equation}
Definendo il momento di dipolo magnetico $ \vec{m} \equiv \frac{i}{2} \oint_{\varsigma} \left(\vec{r}\,'\times d\vec{r}\,'\right) $ possiamo scrivere il potenziale vettore lontano dalla spira come:
\begin{equation}
	\vec{A}(\vec{r}) = \displaystyle\frac{\mu_0}{4\pi r^2} \, \vec{m}\times\hat{r}
	\label{eq:pot-dip}
\end{equation}
Si può notare che il momento di monopolo magnetico $ \oint_{\varsigma} d\vec{r} = 0  $ è sempre nullo, a differenza di quello elettrico. \\ 
%
La definizione di dipolo magnetico è generalizzabile a qualsiasi distribuzione di corrente $ \vec{J} \parallel \vec{r} $:
\begin{equation}
	\vec{m} = \frac{1}{2} \iiint_V \vprod{r}{J}(\vec{r}) \, dV
	\label{eq:dip-mag-def}
\end{equation}
%
Consideriamo ora il caso di una spira circolare di raggio $ a $: $ \vec{r} = a (\cos\theta\,\hat{e}_x + \sin\theta\,\hat{e}_y) $ e $ d\vec{r} = \frac{d\vec{r}}{d\phi} \,d\phi = a (-\sin\theta\,\hat{e}_x + \cos\theta\,\hat{e}_y) \,d\phi $, quindi $ \vec{r}\times d\vec{r} = a^2 d\phi \, \hat{e}_z $ e, integrando sull'intera spira ($ \phi\in[0,2\pi] $):
\begin{equation}
	\vec{m} = i \pi a^2 \, \hat{e}_z
	\label{eq:mom-sp-circ}
\end{equation}
generalizzabile ad una generica spira piana di area $ S $ come $ \vec{m} = i S \,\hat{e}_z $. \\ 
%
Avremo dunque:
\begin{equation}
	\vec{A}(\vec{r}) = \displaystyle\frac{\mu_0}{4\pi r^3} \, \vprod{m}{r} = \displaystyle\frac{\mu_0 m}{4\pi r^3} \, \hat{e}_z \times\left(x\,\hat{e}_x + y\,\hat{e}_y + z\,\hat{e}_z\right) = \displaystyle\frac{\mu_0 m}{4\pi r^2} \displaystyle\frac{-y \, \hat{e}_x + x \, \hat{e}_y}{r}
	\label{eq:calc-po-sp-circ}
\end{equation}
Passando in coordinate sferiche, ricordando che $ \hat{e}_{\phi} = \frac{1}{\rho} (-y \, \hat{e}_x + x \, \hat{e}_y) $:
\begin{equation}
	\vec{A}(\vec{r}) = \displaystyle\frac{\mu_0 m}{4\pi r^2} \sin\theta \,\hat{e}_{\phi}
	\label{eq:po-sp-circ}
\end{equation}
Le linee di campo di $ \vec{A} $ sono delle circonferenze concentriche e parallele alla spira. \\ 
%
Possiamo così calcolare il campo magnetico lontano dalla spira:
\begin{equation}
	\begin{split}
		\vec{B}(\vec{r}) &= \rot\vec{A}(\vec{r}) \\
				 &= \displaystyle\frac{1}{r\sin\theta} \left( \displaystyle\frac{\partial}{\partial\theta} \left( A_{\phi} \sin\theta \right) - \displaystyle\frac{\partial A_{\theta}}{\partial\phi} \right) \hat{e}_r + \displaystyle\frac{1}{r} \left( \displaystyle\frac{1}{\sin\theta} \displaystyle\frac{\partial A_r}{\partial\phi} - \displaystyle\frac{\partial}{\partial r} \left(r A_{\phi} \right)\right) \hat{e}_{\theta} + \\ 
				 & \qquad\qquad\qquad\qquad\qquad\qquad\qquad\qquad\qquad\qquad\qquad\qquad + \displaystyle\frac{1}{r} \left(\displaystyle\frac{\partial}{\partial r} \left( r A_{\theta} \right) - \displaystyle\frac{\partial A_r}{\partial\theta} \right) \hat{e}_{\phi} \\ 
				 &= \displaystyle\frac{1}{r\sin\theta}\displaystyle\frac{\partial}{\partial\theta} \left(A_{\phi} \sin\theta\right) \hat{e}_r - \displaystyle\frac{1}{r} \displaystyle\frac{\partial}{\partial r} \left( r A_{\phi} \right) \hat{e}_{\theta} \\ 
				 &= \displaystyle\frac{\mu_0 m}{4\pi r^3} \left(2\cos\theta \,\hat{e}_r + \sin\theta \,\hat{e}_{\theta}\right)
	\end{split}
	\label{eq:}
\end{equation}
Si nota che questa espressione è matematicamente uguale a quella per il campo elettrico di un dipolo (lontano da esso) $ \vec{E}(\vec{r}) = \frac{p}{4\pi\epsilon_0 r^3} \left(2\cos\theta \,\hat{e}_r + \sin\theta \,\hat{e}_{\theta} \right) $: i campi dei dipoli elettrici e magnetici hanno lo stesso andamento lontano dak dipolo, mentre vicino ad esso hanno andamenti completamente differenti.

\subsection{Forze su Fili e Spire}

\subsubsection{Forza tra due fili percorsi da corrente}

Consideriamo due fili paralleli lunghi $ l $, distanti $ a $ e percorsi da correnti stazionarie $ i_1 $ e $ i_2 $ concordi. Ponendo $ a \ll l $ così da poter approssimare come infiniti i due fili, si ha che i loro campi magnetici sono dati da $ \vec{B}(\vec{r}) = \frac{\mu_0 i}{2\pi r} \hat{e}_{\phi} $, dunque i campi magnetici subiti rispettivamente dai due fili (assumendoli paralleli all'asse $ z $ lungo l'asse $ x $) sono $ \vec{B}_{1,2} = \pm \frac{\mu_0 i_{2,1}}{2\pi a} \hat{e}_{\phi} $ (segno dalla seconda regola della mano destra): la forza magnetica agente su un filo percorso da corrente è $ \vec{F}_B = l \vprod{i}{B} $, quindi nel nostro caso si ha:
\begin{equation}
	\vec{F}_{1,2} = \pm \frac{\mu_0 i_1 i_2 l}{2 \pi a} \hat{e}_x
	\label{eq:forza-fili}
\end{equation}
che è quello che si osserva sperimentalmente: se le correnti sono concordi ii fili si attraggono, se sono discordi si respingono.

\subsubsection{Spira in un campo magnetico uniforme}

Consideriamo una spira quadrata $ \varsigma $ di lato $ l $ percorsa da una corrente stazionaria $ i $ ed immersa in un campo magnetico uniforme $ \vec{B} $, formando un angolo $ \theta $ con esso e potendo ruotare rispetto all'asse $ x $. \\ 
%
Fissiamo un sistema di riferimento con asse $ z $ lungo $ \vec{B} $, la forza agente sulla spira sarà:
\begin{equation}
	\begin{split}
		\vec{F} &= i \oint_{\varsigma} d\vprod{l}{B} = i B \oint_{\varsigma} d\vec{l}\times\hat{e}_z \\ 
			&= iB\int_{x_-} dl \left(\sin\theta\,\hat{e}_y + \cos\theta\,\hat{e}_z\right) \times \hat{e}_z + iB\int_{x_+} -dl \left(\sin\theta\,\hat{e}_y + \cos\theta\,\hat{e}_z\right) \times \hat{e}_z + \\ 
			& \qquad\qquad\qquad\qquad\qquad\qquad\qquad\qquad\qquad\qquad + iB\int_{z_+} dx \, \hat{e}_x \times \hat{e}_z + iB\int_{z_-} -dx \,\hat{e}_x \times \hat{e}_z \\ 
			&= i\,lB\sin\theta\,\hat{e}_x - i\,lb\sin\theta\,\hat{e}_x - i\,lB\,\hat{e}_y + i\,lB\,\hat{e}_y = 0
	\end{split}
	\label{eq:}
\end{equation}
Dunque la spira non trasla, bensì ruota a causa del momento torcente dovuto alle forze espresse dagli ultimi due integrali (asse di rotazione è $ x $):
\begin{equation}
	\begin{split}
		\vec{\tau} &= \vec{\tau_+} + \vec{\tau_-} = \vec{r}_+ \times \vec{F}_+ + \vec{r}_- \times \vec{F}_- \\ 
			   &= -\frac{l}{2} \left(\cos\theta\,\hat{e}_y + \sin\theta\,\hat{e}_z\right) \times \left(-ilB\,\hat{e}_y\right) + \frac{l}{2} \left(\cos\theta\,\hat{e}_y + \sin\theta\,\hat{e}_z\right) \times \left(ilB\,\hat{e}_y\right) \\ 
			   &= i \, l^2 B \sin\theta \,\hat{e}_x = m B \sin\theta\,\hat{e}_x
	\end{split}
	\label{eq:calc-mom-sp}
\end{equation}
Si riconosce un prodotto vettoriale, ottenendo la formula generale (per spire piane):
\begin{equation}
	\vec{\tau} = \vprod{m}{B}
	\label{eq:mom-sp}
\end{equation}
Possiamo anche calcolare il lavoro necessario a far ruotare la spira di un angolo $ \theta $:
\begin{equation}
	W = \int_0^{\theta} \vec{\tau}_{\text{ext}} \cdot d\vec{\theta} = - \int_0^{\theta} \tau \, d\theta = - i \, l^2 B \int_0^{\theta} \sin\theta \,d\theta = - i \, l^2 B \left(1 - \cos\theta\right)
	\label{eq:}
\end{equation}
quindi possiamo definire l'energia potenziale della spira (a meno di sostante $ il^2 B $):
\begin{equation}
	U(\theta) = -\sprod{m}{B}
	\label{eq:pot-sp}
\end{equation}

\subsubsection{Spira in campo magnetico non uniforme}

Consideriamo di nuovo una spira quadrata di lato $ l $ parallela al piano $ xy $, questa volta in un campo magnetico non uniforme $ \vec{B} $ ortogonale ad essa. \\ 
%
Assumiamo per semplificare che $ \vec{B} = \vec{B}(x) $: le forze sui lati $ y_+ $ e $ y_- $ sono continuamente bilanciate, quindi possiamo concentrarci su quelle sui lati $ x_+ $ e $ x_- $: $ \vec{F}_- = -i\,lB_- \,\vec{e}_x $ e $ \vec{F}_+ = i\,lB_+ \,\vec{e}_x $. Il lavoro effettuato per portare la spira da una regione a campo nullo $ x_0 $ alle posizioni finali $ x_- $ e $ x_+ $ è:
\begin{equation}
	\begin{split}
		W &= W_- + W_+ = -\int_{x_0}^{x_+} \vec{F}_+ \cdot d\vec{l} - \int_{x_0}^{x_-} \vec{F}_- \cdot d\vec{l} = -i\,l\int_{x_0}^{x_+} B(x) \,dx + i\,l\int_{x_0}^{x_+} B(x) \,dx \\ 
		  &= -i\,l\int_{x_-}^{x_+} B(x) \,dx
	\end{split}
	\label{eq:}
\end{equation}
Supponendo che la spira sia piccola rispetto alla scala di distanze su cui varia il campo magnetico, possiamo assumere che su tutta la spira e per tutto il moto il campo magnetico sulla spira assuma il valore che ha al suo centro, quindi $ W \approx -i\,l^2 B(x_c) = -m\,B(x_c) $; inoltre, dato che $ \vec{F} = i\,l\left(\vec{B}(x_+) - \vec{B}(x_-)\right) $, sviluppando al prim'ordine otteniamo la forza totale agente sulla spira:
\begin{equation}
	\vec{F} = m \displaystyle\frac{\partial B}{\partial x} \hat{e}_x
	\label{eq:}
\end{equation}

Consideriamo ora il caso più generale $ \vec{B} = \vec{B}(\vec{r}) $, ponendo ora la spira nel piano $ yz $: approssimiamo comunque il valore del campo magnetico su ciascun lato al valore che esso assume nei rispettivi punti medi:
\begin{equation}
	\begin{split}
		\vec{F} &= i\oint_{\varsigma} d\vprod{l}{B} = i \left[ \int_{z_-}^{(1)} d\vprod{l}{B} + \int_{y_+}^{(2)} d\vprod{l}{B} + \int_{z_+}^{(3)} d\vprod{l}{B} + \int_{z_-}^{(4)} d\vprod{l}{B} \right] \\ 
			&= i \left[ \int_0^l dy \, \hat{e}_y \times (\vec{B}_1 - \vec{B}_3) + \int_0^d dz \, \hat{e}_z \times (\vec{B}_2 - \vec{B}_4) \right] \\ 
			& \qquad \text{sviluppo rispetto al centro della spira} \\
			&= i\,l\left[ - \int_0^l dy\, \hat{e}_y \times \displaystyle\frac{\partial\vec{B}}{\partial z} + \int_0^l dz\, \hat{e}_z \times \displaystyle\frac{\partial\vec{B}}{\partial y} \right] = m_c \left[ - \hat{e}_y \times \displaystyle\frac{\partial\vec{B}}{\partial z} + \hat{e}_z \times \displaystyle\frac{\partial\vec{B}}{\partial y} \right] \\ 
			&= m_c \left[ - \displaystyle\frac{\partial B_z}{\partial z} \hat{e}_x + \displaystyle\frac{\partial B_x}{\partial z} \hat{e}_z - \displaystyle\frac{\partial B_y}{\partial y} \hat{e}_x + \displaystyle\frac{\partial B_x}{\partial y} \hat{e}_y \right] \\ 
			& \qquad\qquad \dive\vec{B} = 0 \Rightarrow \displaystyle\frac{\partial B_x}{\partial x} = - \displaystyle\frac{\partial B_y}{\partial y} - \displaystyle\frac{\partial B_z}{\partial z} \\ 
			&= m_0 \left[ \displaystyle\frac{\partial B_x}{\partial x} \hat{e}_x + \displaystyle\frac{\partial B_y}{\partial y} \hat{e}_y + \displaystyle\frac{\partial B_z}{\partial z} \hat{e}_z \right] = m_0 \nabla B_x
	\end{split}
	\label{eq:}
\end{equation}
Si nota che il dipolo orientato lungo l'asse x seleziona la componente $ B_x $ per calcolare la forza totale sulla spira; si può generalizzare ad un'orientazione generica del dipolo:
\begin{equation}
	\vec{F} = \nabla(\sprod{m}{B})
	\label{eq:for-sp-non-uni}
\end{equation}

\subsubsection{Strato di corrente uniforme}

Consideriamo uno strato infinito di spessore $ 2a $ (spessore lungo asse $ x $) percorso da una densità di corrente $ \vec{J} = J\,\hat{e}_z $ uniforme: per la legge di Biot-Savart $ \vec{B}\perp\vec{J} $, quindi $ B_z = 0 $; inoltre, se per assurdo fosse $ B_x \neq 0 $, allora invertendo $ \vec{J} $ si dovrebbe invertire anche il segno di $ B_x $, ma, dato che lo strato è infinito, invertire $ \vec{J} $ significa invertire l'asse $ x $, e non ha senso fisico che il campo magnetico dipenda dal sistema di riferimento, quindi bisogna avere anche $ B_x = 0 $, quindi per simmetria $ \vec{B} = B(x) \,\hat{e}_y $; infine, sempre per la legge di Biot-Savart, ed in particolare per il termine $ \vec{J}\times(\vec{r} - \vec{r}\,') $, si deve avere $ B(0) = 0 $ e $ \sgn{B(x)} = \sgn{x} $. \\ 
%
Poniamoci nel caso $ x > 0 $ e distinguiamo tra $ x < a $ e $ x > a $, considerando sempre un loop $ \gamma $ rettangolare sul piano $ xy $ di lunghezza $ l $ sull'asse $ y $: nel caso $ x < a $ il loop è completamente contenuto nello strato, quindi la legge di Amère dà:
\begin{equation}
	\oint_{\gamma} \vec{B} \cdot d\vec{l} = B(x)l - B(0)l = B(x)l = \mu_0 i = \mu_0 \oiint_S \vec{J} \cdot d\vec{S} = \mu_0 J x l \,\,\Longrightarrow\,\, B(x) = \mu_0 J x
	\label{eq:}
\end{equation}
Per $ x > a $, invece, il loop intercetta la corrente solo fino ad $ x = a $, dunque si trova che $ B(x) = \mu_0 J a $; in definitiva, l'espressione del campo magnetico è:
\begin{equation}
	\vec{B}(\vec{r}) = 
	\begin{cases}
		\mu_0 J x \, \hat{e}_y & \abs{x} < a \\ 
		\mu_0 J a \,\text{sgn}(x) \,\hat{e}_y & \abs{x} \ge a
	\end{cases}
	\label{eq:}
\end{equation}

\subsubsection{Guscio sferico carico rotante}

Consideriamo un guscio sferico uniformemente carico di raggio $ R $ e carica superficiale $ \sigma $, in rotazione con velocità angolare $ \vec{\Omega} $, e fissiamo gli assi in modo che l'asse $ z $ sia allineato a $ \vec{r} $ e $ \vec{\Omega} $ giaccia sul piano $ xy $: poiché la carica è in movimento, si avrà una densità superficiale di corrente $ \vec{K}(\vec{r}\,') = \sigma\vec{v}\,' = \sigma\vprod{\Omega}{r}\,' $. \\ 
%
Detto $ \psi $ l'angolo tra $ \Omega $ e $ \hat{e}_z $ e posto $ \vec{r} = r\,\hat{e}_z $, si ha:
\begin{equation}
	\begin{split}
		\vprod{\Omega}{r}\,' &= (\omega \sin\psi \,\hat{e}_x + \omega \cos\psi \,\hat{e}_z) \times R (\sin\theta'\cos\phi' \,\hat{e}_x + \sin\theta'\sin\theta' \,\hat{e}_y + \cos\theta' \,\hat{e}_z) \\ 
				     &= R\omega \left[-\cos\psi\sin\theta'\sin\phi' \,\hat{e}_x + (\cos\psi\sin\theta'\cos\phi' - \sin\psi\cos\theta')\,\hat{e}_y + \sin\psi\sin\theta'\sin\phi' \,\hat{e}_z \right]
	\end{split}
	\label{eq:}
\end{equation}
\begin{equation}
	\abs{\vec{r} - \vec{r}\,'} = \sqrt{r^2 + r'^2 - 2\sprod{r}{r}\,'} = \sqrt{r^2 + R^2 - 2rR\cos\theta'}
	\label{eq:}
\end{equation}
Il potenziale vettore è dato da $ \vec{A}(\vec{r}) = \frac{\mu_0 \sigma}{4\pi} \oiint_S \frac{\vec{K}(\vec{r}\,')}{\abs{\vec{r}-\vec{r}\,'}} dS' = \frac{\mu_0 \sigma}{4\pi} \int_0^{\pi} \int_0^{2\pi} \frac{\vprod{\Omega}{r}\,'}{\abs{\vec{r}-\vec{r}\,'}} R^2 \sin\theta' d\theta' d\phi' $, ma il denominatore non dipende da $ \phi' $, mentre il numeratore vi dipende solo tramite funzioni trigonometriche semplici che hanno integrale nullo su $ [0,2\pi] $, quindi:
\begin{equation}
	\begin{split}
		\vec{A}(\vec{r}) &= \displaystyle\frac{\mu_0 \sigma R^3 \omega}{4\pi} 2\pi\sin\psi \int_0^{\pi} \displaystyle\frac{-\cos\theta'}{\sqrt{r^2 + R^2 -2rR\cos\theta'}} \sin\theta' d\theta' \,\hat{e}_y \\ 
				 & \qquad\qquad \cos\theta' = x \\ 
				 &= \displaystyle\frac{\mu_0}{2} \sigma R^3 \omega \sin\psi \int_{-1}^{1} \displaystyle\frac{x}{\sqrt{r^2 + R^2 - 2rRx}} dx \\ 
				 &= \displaystyle\frac{\mu_0}{2} \sigma R^3 \omega \sin\psi \left[ -\displaystyle\frac{1}{3R^2r^2}\left( (r^2 + R^2 + rR)\abs{R - r} - (r^2 + R^2 - rR)(R + r)\right)\right]
	\end{split}
	\label{eq:}
\end{equation}
Semplificando, otteniamo:
\begin{equation}
	\vec{A}(\vec{r}) =
	\begin{cases}
		-\frac{\mu_0}{3} \sigma r R \omega \sin\psi \,\hat{e}_y & r < R \\ 
		-\frac{\mu_0}{3} \sigma \displaystyle\frac{R^4}{r^2} \omega \sin\psi \,\hat{e}_y & r > R
	\end{cases}
	\,=
	\begin{cases}
		\frac{\mu_0}{3} \sigma R \vprod{\Omega}{r} & r < R \\ 
		\frac{\mu_0}{3} \sigma \displaystyle\frac{R^4}{r^3} \vprod{\Omega}{r} & r > R
	\end{cases}
	\label{eq:po-sf}
\end{equation}
Si nota che per $ r > R $ il potenziale ha la stessa forma matematica del potenziale di dipolo \ref{eq:pot-dip}, ponendo $ \vec{m} = \frac{4}{3} \pi R^4 \vec{\Omega} $. \\ 
%
Ruotiamo ora il sistema di riferimento in modo che l'asse $ z $ sia lungo $ \vec{\Omega} $: ora $ \vec{\Omega} = \omega \hat{e}_z = \omega (\cos\theta \,\hat{e}_r - \sin\theta \,\hat{e}_{\theta}) $ e $ \vec{r} = r \,\hat{e}_r $, quindi $ \vprod{\Omega}{r} = \omega r \sin\theta \,\hat{e}_{\phi} $ e:
\begin{equation}
	\vec{A}(\vec{r}) =
	\begin{cases}
		-\frac{\mu_0}{3} \sigma r R \omega \sin\theta \,\hat{e}_{\phi} & r < R \\ 
		-\frac{\mu_0}{3} \sigma \displaystyle\frac{R^4}{r^2} \omega \sin\theta \,\hat{e}_{\phi} & r > R
	\end{cases}
	\label{eq:}
\end{equation}
Per calcolare il campo magnetico, ricordiamo il rotore in coordinate sferiche:
\begin{equation}
	\begin{split}
		\vec{B}(\vec{r}) &= \rot\vec{A} \\ 
				 &= \displaystyle\frac{1}{r\sin\theta} \left( \displaystyle\frac{\partial}{\partial\theta} \left( A_{\phi} \sin\theta \right) - \displaystyle\frac{\partial A_{\theta}}{\partial\phi} \right) \hat{e}_r + \displaystyle\frac{1}{r} \left( \displaystyle\frac{1}{\sin\theta} \displaystyle\frac{\partial A_r}{\partial\phi} - \displaystyle\frac{\partial}{\partial r} \left(r A_{\phi} \right)\right) \hat{e}_{\theta} + \\ 
				 & \qquad\qquad\qquad\qquad\qquad\qquad\qquad\qquad\qquad\qquad\qquad\qquad + \displaystyle\frac{1}{r} \left(\displaystyle\frac{\partial}{\partial r} \left( r A_{\theta} \right) - \displaystyle\frac{\partial A_r}{\partial\theta} \right) \hat{e}_{\phi} \\ 
				 &= \displaystyle\frac{1}{r\sin\theta}\displaystyle\frac{\partial}{\partial\theta}(A_{\phi} \sin\theta) \,\hat{e}_r - \displaystyle\frac{1}{r}\displaystyle\frac{\partial}{\partial r} (rA_{\phi}) \,\hat{e}_{\theta}
	\end{split}
	\label{eq:}
\end{equation}
Nel caso $ r < R $:
\begin{equation}
	B_r = \displaystyle\frac{\mu_0}{3} \displaystyle\frac{\sigma r R \omega}{r \sin\theta} \displaystyle\frac{\partial}{\partial\theta} \sin^2\theta = \displaystyle\frac{2\mu_0}{3} \sigma R \omega \cos\theta
	\label{eq:}
\end{equation}
\begin{equation}
	B_{\theta} = -\displaystyle\frac{\mu_0}{3} \sigma R \omega \displaystyle\frac{\partial}{\partial r} (r^2 \sin\theta) = -\displaystyle\frac{2\mu_0}{3} \sigma R \omega \sin\theta
	\label{eq:}
\end{equation}
\begin{equation}
	\vec{B}(\vec{r}) = \displaystyle\frac{2}{3} \mu_0 \sigma\omega R\,\hat{e}_z
	\label{eq:}
\end{equation}
Nel caso $ r > R $:
\begin{equation}
	B_r = \displaystyle\frac{1}{r \sin\theta} \displaystyle\frac{\mu_0}{3} \displaystyle\frac{\sigma R^4 \omega}{r^2} \displaystyle\frac{\partial}{\partial\theta} \sin^2\theta = \displaystyle\frac{2\mu_0}{3} \displaystyle\frac{\sigma R^4 \omega}{r^3} \cos\theta
	\label{eq:}
\end{equation}
\begin{equation}
	B_{\theta} = -\displaystyle\frac{1}{r} \displaystyle\frac{\mu_0}{3} \sigma R^4 \omega \sin\theta \displaystyle\frac{\partial}{\partial r} \left(\displaystyle\frac{1}{r}\right) = \displaystyle\frac{\mu_0}{3} \displaystyle\frac{\sigma R^4 \omega}{r^3} \sin\theta
	\label{eq:}
\end{equation}
\begin{equation}
	\vec{B}(\vec{r}) = \displaystyle\frac{\mu_0 \sigma R^4 \omega}{3 r^3} \left(2\cos\theta \,\hat{e}_r + \sin\theta \,\hat{e}_{\theta}\right)
	\label{eq:}
\end{equation}
Quindi in definitiva:
\begin{equation}
	\vec{B}(\vec{r}) =
	\begin{cases}
		\frac{2}{3} \mu_0 \sigma\omega R\,\hat{e}_z & r < R \\ 
		\displaystyle\frac{\mu_0 \sigma R^4 \omega}{3 r^3} \left(2\cos\theta \,\hat{e}_r + \sin\theta \,\hat{e}_{\theta}\right) & r > R
	\end{cases}
	\label{eq:mag-sf}
\end{equation}
Si noti che all'interno del guscio il campo magnetico è uniforme e diretto lungo l'asse di rotazione, mentre all'esterno è equivalente a quello di un dipolo magnetico con $ \mu = \frac{4}{3} \pi R^4 \vec{\Omega} $.
