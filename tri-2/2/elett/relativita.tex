\section{Relatività}

\subsection{Relatività Ristretta}
La relatività ristretta si basa su due postulati:
\begin{enumerate}
	\item tutte le leggi fisiche sono le stesse in tutti i sistemi di riferimento inerziali, non esiste un sistema di riferimento privilegiato;
	\item la velocità della luce nello spazio vuoto ha lo stesso valore in tutti i sistemi di riferimento inerziali.
\end{enumerate}
Il secondo postulato è in evidente contrasto con la relatività galileiana ma ha evidenze sia di natura teorica che sperimentale.

\subsubsection{Equivalenza dei sistemi di riferimento}

Consideriamo il seguente Gedankenexperiment: una persona è in un vagone che viaggia a velocità $ \vec{v} $ e si guarda in uno specchio posto a distanza $ l $ in direzione opposta al verso del moto: se la velocità della luce risentisse del sistema di riferimento dell'osservatore, si avrebbe:
\begin{equation}
	\Delta t = \Delta t_{\text{andata}} + \Delta t_{\text{ritorno}} = \frac{l}{c - v} + \frac{l}{c + v} = \frac{2l}{c} \left( 1 - \frac{v^2}{c^2}\right)^{-1}
	\label{eq:1}
\end{equation}
quindi il viaggiatore potrebbe così misurare la velocità del vagone ad ogni istante, distinguendo tra sistemi inerziali diversi e quindi violando il primo postulato.

\subsubsection{Equivalenza delle leggi fisiche}

Uno dei principali risultati della teoria di Maxwell è stato l'unificazione dei fenomeni elettrici e magnetici nel campo elettromagnetico, predicendo l'esistenza di onde elettromagnetiche che si propagano nel vuoto alla velocità costante $ c \equiv \frac{1}{\sqrt{\epsilon_0\mu_0}} $: poiché le leggi fisiche devono essere le stesse in ogni sistema inerziale, $ c $ non può dipendere dal sistema di riferimento.

\subsubsection{Esperimento di Michelson-Morley}

Al tempo della loro teorizzazione e scoperta, era ragionevole pensare che le onde elettromagnetiche si propagassero in un mezzo trasparente e non interagente con la materia, detto etere. \\ 
%
Per rilevare la presenza di questo etere, si cercò di evidenziare la differenza di tempo impiegata da due raggi luminosi per viaggiare per la stessa distanza in due direzioni perpendicolare e parallela al vento d'etere, assumendo che questo viaggi a $ v \ll c $:
\begin{equation}
	\Delta t_{\parallel} = \frac{l}{c-v} + \frac{l}{c+v} = \frac{2l}{c}\frac{1}{1 - \frac{v^2}{c^2}} \approx \frac{2l}{c}\left(1 + \frac{v^2}{c^2}\right)
	\label{eq:2}
\end{equation}
Per quanto riguarda la direzione ortogonale al vento d'etere, bisogna contare che mentre il raggio percorre $ c\Delta t' $ l'intero sistema trasla ortogonalmente ad esso di $ v\Delta t' $, quindi per il teorema di Pitagora $ c^2 \Delta t'^2 = v \Delta t'^2 + l^2 $, ovvero:
\begin{equation}
	\Delta t_{\perp} = 2 \Delta t' = \frac{2l}{\sqrt{c^2 - v^2}} = \frac{2l}{c} \frac{1}{\sqrt{1 - \frac{v^2}{c^2}}} \approx \frac{2l}{c} \left(1 + \frac{1}{2}\frac{v^2}{c^2}\right)
	\label{eq:3}
\end{equation}
Si vede quindi che la differenza tra i due tempi è $ \frac{l}{c}\frac{v^2}{c^2} $, che per la velocità di rivoluzione della Terra è dell'ordine di $ 10^{-7} \s $. Invece di misurarla direttamente, Michelson e Morley misurarono il diagramma di interferenza generato dai due fasci, ruotando l'apparato in varie direzioni per vedere in quale i diagrammi risultassero diversi: contrariamente all'ipotesi dell'esistenza dell'etere, non si evidenziò alcuna variazione di tali diagrammi.

% \subsubsection{Simultaneità relativistica}

% Consideriamo due sistemi di riferimento $ \mathcal{S} $ ed $ \mathcal{S}' $, inizialmente coincidenti e con $ \mathcal{S}' $ in moto rettilineo uniforme con velocità $ \vec{v} $. \\ 
%
% Ponendo due sorgenti sonore equidistanti dall'origine $ O \equiv O' $ inizialmente comune dei due sistemi, l'osservatore in $ O $ registra i suoni allo stesso istante di tempo e giudica i due eventi simultanei, mentre per l'osservatore in $ O' $ i suoni si muovono a velocità $ c_s \pm v $, dunque non sono simultanei. \\ 
%
% Se invece consideriamo due sorgenti luminose, entrambi gli osservatori vedranno simultaneamente le due luci

\subsection{Trasformazioni di Lorentz}

Consideriamo due sistemi di riferimento $ \mathcal{S} $ ed $ \mathcal{S}' $, inizialmente coincidenti e con $ \mathcal{S}' $ in moto rettilineo uniforme con velocità $	\vec{v} = v\,\hat{e}_x $. \\ 
%
Innanzitutto, le trasformazioni che legano le coordinate tra i due sistemi devono essere lineari, altrimenti ad esempio si avrebbe che la lunghezza di un segmento dipenderebbe dalla sua posizione, contraddicendo il principio di omogeneità ed isotropia dello spazio; possiamo poi notare che il moto interessa soltanto la direzione $ x $, dunque $ (y',z') $ possono dipendere solo da $ (y,z) $, e dato che all'istante iniziale le origini coincidono si ha $ y' = \alpha y $, $ z' = \beta z $. \\ 
%
Consideriamo una sbarra di lunghezza $ l $ lungo l'asse $ y $: l'osservatore in $ \mathcal{S}' $ la misurerà lunga $ l' = \alpha l $; viceversa, se la sbarra si trova lungo $ y' $, l'osservatore in $ \mathcal{S} $ la misurerà lunga $ l = \frac{1}{\alpha} l' $. Dato che i due sistemi di riferimento devono essere equivalenti, si deve avere $ l = l' $ (altrimenti sarebbero distinguibili), quindi $ \alpha = 1 $, ovvero $ y' = y $ ed analogamente $ z = z' $. \\ 
%
Consideriamo ora che, nel sistema $ \mathcal{S} $, $ O' = (vt, 0, 0, t) $, dunque per $ x' = \alpha x + \beta y + \gamma z + \delta t $ vale il vincolo $ 0 = \alpha vt + \beta y + \gamma z + \delta t  \,\,\forall y,z\in\mathbb{R} \,\Rightarrow\, \beta = \gamma = 0 \,\wedge\, \delta = -v\alpha $, quindi $ x' = \alpha (x - vt) $; inoltre, dato il principio di isotropia dello spazio, $ t' $ può dipendere solo da $ (x,t) $, ovvero $ t' = \beta x + \gamma t $. \\ 
%
Dal secondo postulato sappiamo che una sorgente luminosa puntiforme si espande come simmetria sferica in entrambi i sistemi ed alla stessa velocità $ c $, quindi possiamo scrivere $ R^2 = x^2 + y^2 + z^2 = c^2 t^2 $ e $ R'^2 = x'^2 + y'^2 + z'^2 = c^2 t'^2 $ e otteniamo:
\begin{equation}
	(x^2- x'^2) + (y^2 - y'^2) + (z^2 - z'^2) - c^2 (t^2 - t'^2) = 0
	\label{eq:4}
\end{equation}
Sostituendo le espressioni trovate prima e svolgendo i calcoli, otteniamo il seguente sistema di equazioni:
\begin{equation}
	\begin{cases}
		1 - \alpha^2 + c^2 \beta^2 = 0 \\ 
		c^2 \delta^2 - \alpha^2 v^2 - c^2 = 0 \\ 
		\alpha^2 v + c^2 \beta\delta = 0
	\end{cases}
	\label{eq:5}
\end{equation}
che ha come soluzione $ \alpha = \delta = \frac{1}{\sqrt{1 - \frac{v^2}{c^2}}} $ e $ \beta = -\frac{v}{c^2}\frac{1}{\sqrt{1 - \frac{v^2}{c^2}}} $. \\ 
%
Definendo il fattore di Lorentz $ \gamma \equiv \frac{1}{\sqrt{1 - \frac{v^2}{c^2}}} \in [1, \infty)$, otteniamo quindi l'espressione delle trasformazioni di Lorentz:
\begin{equation}
	\begin{cases}
		x' = \gamma \left(x - vt\right) \\ 
		y' = y \\ 
		z' = z \\ 
		t' = \gamma \left(t - \frac{v}{c^2} x\right)
	\end{cases}
	\qquad\Longleftrightarrow\qquad
		\begin{cases}
		x = \gamma \left(x' + vt'\right) \\ 
		y = y' \\ 
		z = z' \\ 
		t = \gamma \left(t' + \frac{v}{c^2} x'\right)
	\end{cases}
	\label{eq:6}
\end{equation}

\subsubsection{Contrazione delle lunghezze}

Consideriamo una barra in moto rettilineo uniforme con $ \vec{v} = v\,\hat{e}_x $: nel sistema di riferimento solidale alla barra essa ha coordinate $ x'_1 = 0 $ e $ x'_2 = l' $, mentre, assumendo di effettuare la misura a $ t = 0 $, per un osservatore essa ha estremi $ x_1 = 0 $ e $ x_2 = l $: per le trasformazioni di Lorentz, sappiamo che $ x' = \gamma (x - vt) $, quindi osserviamo il fenomeno della contrazione delle lunghezze (poiché $ \gamma \ge 1 $):
\begin{equation}
	l = \frac{l'}{\gamma}
	\label{eq:7}
\end{equation}
ovvero un osservatore in quiete misura una lunghezza minore di un oggetto in movimento rispetto ad un osservatore solidale all'oggetto. Ciò non è un effetto apparente, ma è una vera contrazione dello spazio stesso, il quale diventa appunto relativo.

\subsubsection{Dilatazione dei tempi}

Consideriamo ora un orologio in $ \mathcal{S}' $ in una posizione fissata $ x'_1 $ e immaginiamo che un osservatore solidale a $ \mathcal{S}' $ faccia due misure di tempi $ t'_1 $ e $ t'_2 $: allora per le trasformazioni di Lorentz un osservatore in $ \mathcal{S} $ misurerà $ t_2 = \gamma \left(t'_2 + \frac{v}{c^2} x'_1\right) $ e $ t_1 = \gamma \left(t'_1 + \frac{v}{c^2} x'_1\right) $, dunque possiamo osservare il fenomeno della dilatazione dei tempi:
\begin{equation}
	\Delta t = \gamma \Delta t'
	\label{eq:8}
\end{equation}
ovvero un osservatore in quiete misura dei tempi maggiori per un evento rispetto ad un osservatore solidale a tale evento: il tempo non è più un'entità assoluta. \\ 
%
È possibile ricavare la dilatazione dei tempi anche senza trasformazioni di Lorentz: consideriamo un orologio costituito da un segnale luminoso che rimbalza tra due specchi in $ \mathcal{S}' $ posti a distanza $ l' $: evidentemente $ \Delta t' = \frac{2l'}{c} $. Per l'osservatore in $ \mathcal{S} $ il segnale luminoso compie un moto a zig-zag rimbalzando tra gli specchi in moto, percorrendo in un rimbalzo una distanza $ l \,:\, l^2 = l'^2 + \frac{v^2 \Delta t^2}{4} $, con $ \Delta t $ il tempo che il segnale compie per completare un periodo (ovvero due rimbalzi): dato che, d'altra parte, anche in $ \mathcal{S} $ la luce viaggia a $ c $, si ha $ 4\left(l'^2 + \frac{v^2 \Delta t^2}{4}\right) = c^2 \Delta t^2 \, \Rightarrow \, \Delta t^2 = \frac{4l'^2}{c^2} \gamma^2 $, ovvero proprio la dilatazione dei tempi $ \Delta t = \gamma \Delta t' $.

\subsection{Quadrivettori e Tensore Metrico}

Come suggerito dalle trasformazioni di Lorentz, spazio e tempo non sono indipendenti ma legati tra loro, quindi adottiamo un punto di vista quadri-dimensionale definendo $ x^0 \equiv ct $, $ x^1 \equiv x $, $ x^2 \equiv y $, $ x^3 \equiv z $; possiamo scrivere le trasformazioni di Lorentz come:
\begin{equation}
	\begin{pmatrix}
		x'^0 \\ x'^1 \\ x'^2 \\ x'^3
	\end{pmatrix}
	=
	\begin{bmatrix}
		\gamma & -\gamma\beta & 0 & 0 \\ 
		-\gamma\beta & \gamma & 0 & 0 \\ 
		0 & 0 & 1 & 0 \\ 
		0 & 0 & 0 & 1 \\ 
	\end{bmatrix}
	\begin{pmatrix}
		x^0 \\ x^1 \\ x^2 \\ x^3 
	\end{pmatrix}
	\label{eq:9}
\end{equation}
dove è stato definito $ \beta\equiv\frac{v}{c} $. \\ 
%
Adottando la convezione di sommatoria sottintesa di Einstein, definiamo il quadrivettore controvariante $ x^{\mu} \equiv (ct, x, y, z) $ e il suo corrispettivo covariante $ x_{\mu} \equiv (ct, -x, -y, -z) $: le trasformazioni di Lorentz lasciano invariata la quantità $ x^{\mu}x_{\mu} \equiv x\cdot x$, detto invariante di Lorentz. \\ 
%
Per passare dalle componenti controvarianti a quelle covarianti e viceversa si utilizza il tensore metrico $ g_{\mu\nu} = g^{\mu\nu} = \text{diag}(1,-1,-1,-1) $: $ x_{\mu} = g_{\mu\nu}x^{\mu} $. \\ 
Possiamo scrivere le trasformazioni di Lorentz come:
\begin{equation}
	x'^{\mu} = \Lambda^{\mu}_{\nu} x^{\nu}
	\label{eq:10}
\end{equation}
dove la matrice di trasformazione $ \Lambda^{\mu}_{\nu} $ è data da \ref{eq:9}.

\subsubsection{Cinematica relativistica}

Dato che le trasformazioni di Lorentz sono lineari, anche il differenziale di un quadrivettore gode di tutte le proprietà di un quadrivettore; se prendiamo il suo modulo:
\begin{equation}
	dx^{\mu}dx_{\mu} = c^2dt^2 - \abs{d\vec{r}}^2 \equiv d\tau^2
	\label{eq:11}
\end{equation}
Possiamo dunque definire il tempo proprio (ha però le dimensioni di una distanza):
\begin{equation}
	d\tau = c \,dt \sqrt{1 - \displaystyle\frac{1}{c^2}\left(\displaystyle\frac{\abs{d\vec{r}}}{dt}\right)^2} = \displaystyle\frac{c}{\gamma}\,dt
	\label{eq:12}
\end{equation}
invariante rispetto alle trasformazioni di Lorentz (ovvero lo stesso in tutti i sistemi di riferimento inerziali). \\ 
%
Vediamo come trasforma la velocità $ \vec{u} $ di un punto $ P \in \mathcal{S} $ in un sistema di riferimento $ \mathcal{S}' $ in moto relativo rettilineo uniforme rispetto ad esso con velocità $ \vec{v} $; dalle trasformazioni di Lorentz:
\begin{equation}
	\begin{split}
		dx^0 &= \gamma (dx'^0 + \beta dx'^1) \qquad dx^2 = dx'^2 \\ 
		dx^1 &= \gamma (\beta dx'^0 + dx'^1) \qquad dx^3 = dx'^3
	\end{split}
	\label{eq:13}
\end{equation}
Quindi:
\begin{equation}
	\begin{split}
		u^1 &= c \displaystyle\frac{dx^1}{dx^0} = c \displaystyle\frac{\beta dx'^0 + dx'^1}{dx'^0 + \beta dx'^1} = \displaystyle\frac{v + u'^1}{1 + \frac{vu'^1}{c^2}} \\ 
		u^{2,3} &= c \displaystyle\frac{dx^{2,3}}{dx^0} = \displaystyle\frac{u'^{2,3}}{\gamma (1 + \frac{vu'^1}{c^2})}
	\end{split}
	\label{eq:14}
\end{equation}
ovvero, generalizzando:
\begin{equation}
	u_{\parallel} = \displaystyle\frac{v + u'_{\parallel}}{1 + \frac{\sprod{v}{u}\,'}{c^2}} \qquad\qquad u_{\perp} = \displaystyle\frac{1}{\gamma}\displaystyle\frac{u'_{\perp}}{1 + \frac{\sprod{v}{u}\,'}{c^2}}
	\label{eq:15}
\end{equation}
Si possono notare due importanti conseguenze: innanzitutto se $ \abs{u'} = c $ allora anche $ \abs{u} = c $, in accordo con il secondo postulato; inoltre, se la quantità di moto è conservata in un sistema di riferimento, in generale essa non sarà conservata anche nell'altro. \\ 
Da queste leggi di trasformazione notiamo che la velocità non è un quadrivettore; per avere una velocità che abbia le proprietà dei quadrivettori, dobbiamo definire la quadrivelocità:
\begin{equation}
	\eta^{\mu} \equiv \displaystyle\frac{dx^{\mu}}{d\tau} = \gamma \, (1, \vec{\beta})
	\label{eq:16}
\end{equation}
Si nota che $ \eta^2 = 1 $, ovvero che $ \eta^{\mu} $ è una quantità adimensionale. \\ 
%
Analogamente, definiamo il quadrimomento:
\begin{equation}
	p^{\mu} \equiv m_0 c^2 \eta^{\mu} = \gamma m_0 c^2 (1, \vec{\beta})
	\label{eq:17}
\end{equation}
dove $ m_0 $ è la massa a riposo del corpo. \\ 
Si nota che $ p^2 = m_0^2 \, c^4 $, ovvero che $ p^{\mu} $ ha le dimensioni di un'energia.

\subsubsection{Dinamica relativistica}

Riprendiamo le tre leggi di Newton:
\begin{enumerate}
	\item prima legge: viene inglobata nel primo postulato di Einstein;
	\item seconda legge: vale anche in relatività ristretta come $ \vec{F} = \frac{d\vec{p}}{dt} $, con $ \vec{p} = \gamma m_0 \vec{v} $;
	\item terza legge: non vale in relatività ristretta, poiché si basa sulla simultaneità delle forze, concetto che cambia in relatività rispetto alla meccanica classica.
\end{enumerate}

Consideriamo una particella soggetta ad una forza costante $ \vec{F} $ e consideriamo un sistema di riferimento con asse $ x $ orientato lungo $ \vec{F} $, facendo l'ulteriore semplificazione che $ \vec{p}(0) = 0 \,\Rightarrow\, \vec{p}(t) \parallel \vec{F} \,\,\forall t\in\mathbb{R}^+$, così da poter scrivere:
\begin{equation}
	F = \displaystyle\frac{dp}{dt} \qquad \Longrightarrow \qquad p(t) = F t = \displaystyle\frac{m_0 v(t)}{1 - \frac{v^2(t)}{c^2}}
	\label{eq:18}
\end{equation}
Prendendo il modulo quadro:
\begin{equation}
	\displaystyle\frac{m_0^2 \, c^2 \beta^2}{1 - \beta^2} = F^2 t^2 \qquad \Longrightarrow \qquad v^2(t) = c^2 \displaystyle\frac{F^2 t^2}{m_0^2 \, c^2 + F^2 t^2}
	\label{eq:19}
\end{equation}
Si evidenziano due comportamenti asintotici:
\begin{itemize}
	\item limite non-relativistico: $ Ft \ll m_0 c \,\,\Rightarrow\,\, v(t) \approx \frac{F}{m_0}t$;
	\item limite ultra-relativistico: $ Ft \gg m_0 c \,\,\Rightarrow\,\, v(t) \rightarrow c $.
\end{itemize}
Possiamo determinare la traiettoria integrando:
\begin{equation}
	\begin{split}
		x(t) &= c \int_o^t \displaystyle\frac{Ft'}{\sqrt{{m_0^2 \, c^2 + F^2 t'^2}}} dt' = c \int_0^t \displaystyle\frac{\frac{Ft'}{m_0 c}}{\sqrt{1 + (\frac{F t'}{m_0 c})^2}} \\ 
		     & \qquad\qquad y \equiv \frac{F t'}{m_0 c} \\ 
		     &= \frac{m_0 c^2}{F} \int_0^{\frac{Ft}{m_0 c}} \displaystyle\frac{y}{\sqrt{1 + y^2}} dy + x_0 = \displaystyle\frac{m_0 c^2}{F} \left(\sqrt{1 + \left(\frac{F t}{m_0 c}\right)^2} - 1\right) + x_0
	\end{split}
	\label{eq:20}
\end{equation}
Nel limite non relativistico $ \frac{Ft}{m_0c} \ll 1 $, quindi $ x(t) \approx x_0 + \frac{1}{2}\frac{F}{m}t^2 $, che è proprio quello che ci aspetteremmo dalla meccanica classica: la soluzione classica è un ramo di parabola, mentre quella realtivistica è un ramo di iperbole. \\ 
%
Vediamo ora come diventa il teorema dell'energia cinetica in ambito relativistico:
\begin{equation}
	\begin{split}
		W &= \int \vec{F}\cdot d\vec{l} = \int_0^t \displaystyle\frac{d\vec{p}}{dt'}\cdot \displaystyle\frac{d\vec{l}}{dt'} dt' = \int_0^t \left[ \displaystyle\frac{m_0}{\sqrt{1 - \frac{v^2}{c^2}}} \displaystyle\frac{d\vec{v}}{dt'} + \frac{1}{2}\displaystyle\frac{m_0 2\frac{v}{c^2}\frac{dv}{dt'}}{\left(1 - \frac{v^2}{c^2}\right)^{3/2}} \right] \cdot\vec{v}\, dt' \\ 
		  &= \int_0^t \displaystyle\frac{m_0}{\sqrt{1-\frac{v^2}{c^2}}} v \displaystyle\frac{dv}{dt'} \left[ 1 + \displaystyle\frac{\frac{v^2}{c^2}}{1 - \frac{v^2}{c^2}}\right] dt' = \int_0^t \displaystyle\frac{m_0 v}{\left(1-\frac{v^2}{c^2}\right)^{3/2}} \displaystyle\frac{dv}{dt'} dt' = \int_0^t \displaystyle\frac{d}{dt'}\left(\gamma m_0 c^2\right)dt'
	\end{split}
	\label{eq:21}
\end{equation}
Dato che $ W = K $, abbiamo le espressioni per l'energia totale, l'energia cinetica e l'energia a riposo della particella:
\begin{equation}
	E = \gamma m_0 c^2 \qquad\qquad K = m_0 c^2 (\gamma-1) \qquad\qquad E_0 = m_0 c^2
	\label{eq:22}
\end{equation}
Nel limite non relativistico $ \gamma \approx 1 + \frac{1}{2}\frac{v^2}{c^2} $, quindi si ottiene giustamente $ K \approx \frac{1}{2} m_0 v^2 $. \\ 
Infine, possiamo vedere come trasforma la forza $ \vec{F} $ in un sistema di riferimento in moto relativo con $ \vec{v} = -v\,\hat{e}_x $:
\begin{equation}
	F'_{y,z} = \displaystyle\frac{dp'_{y,z}}{dt'} = \displaystyle\frac{dp_{y,z}}{\gamma (dt - \frac{\beta}{c}dx)} = \displaystyle\frac{F_{y,z}}{\gamma (1 - \frac{\beta}{c} v_x)}
	\label{eq:23}
\end{equation}
Dato che $ p^{\mu} = (E, c\vec{p}) $, si ha $ dp'_x = \frac{dp'^1}{c} = \gamma (-\beta \frac{dE}{c} + dp_x) $, quindi:
\begin{equation}
	F'_x = \displaystyle\frac{dp'_x}{dt'} = \displaystyle\frac{\gamma (-\beta \frac{dE}{c} + dp_x)}{\gamma (dt - \frac{\beta}{c} dx)} = \displaystyle\frac{F_x - \frac{\beta}{c}\frac{dE}{dt}}{1 - \frac{\beta}{c}v_x} = \displaystyle\frac{F_x - \frac{\beta}{c}\sprod{F}{v}}{1 - \frac{\beta}{c} v_x}
	\label{eq:24}
\end{equation}
dove abbiamo ripreso $ \frac{dE}{dt} = \frac{d}{dt}(\gamma m_0 c^2) = \frac{d\vec{p}}{dt}\cdot\vec{v}=\sprod{F}{v} $. Generalizzando:
\begin{equation}
	F_{\parallel} = \displaystyle\frac{F_{\parallel} - \frac{\beta}{c}\sprod{F}{v}}{1 - \frac{\beta}{c} \abs{\vec{v}}} \qquad\qquad F_{\perp} = \displaystyle\frac{F_{\perp}}{\gamma (1 - \frac{\beta}{c} \abs{\vec{v}})}
	\label{eq:25}
\end{equation}
Si può anche definire un quadrivettore forza, detto forza di Minkowski:
\begin{equation}
	f^{\mu} \equiv \displaystyle\frac{dp^{\mu}}{d\tau} = \gamma \left(\displaystyle\frac{1}{c}\displaystyle\frac{dE}{dt}, \vec{F}\right)
	\label{eq:26}
\end{equation}

\subsection{Campo Elettromagnetico}

I pilastri dell'elettrostatica sono la legge di Coulomb e quella di Gauss: il linea di principio, non ci sarebbe nessun motivo per cui esse debbano valere anche per cariche in moto, ed infatti la legge di Coulomb vale solo per cariche ferme (o, in maniera approssimativa, se $ v \ll c $); d'altra parte, sperimentalmente si osserva che la legge di Gauss vale anche per cariche in moto, quindi in un certo senso essa è una legge più generale di quella di Coulomb. \\ 
%
Una conseguenza di ciò è che la carica è un invariante relativistico: essa è indipendente dal sistema di riferimento, e dunque dalla velocità (a differenza, ad esempio, della massa, per la quale vale $ m = \gamma m_0 $): intuitivamente, ciò viene evidenziato, ad esempio, considerando un conduttore neutro e notando che aumentandone, la temperatura, la velocità degli elettroni aumenta più della velocità degli ioni, quindi se la carica dipendesse dalla temperatura si potrebbe caricare il conduttore semplicemente scaldandolo. \\ 
%
Se la carica è invariante, si vede che le densità di carica non lo sono. \\ 
%
Consideriamo una superficie piana con densità di carica $ \sigma = \frac{q}{S} $: se essa si muove con velocità $ \vec{v} $ nella direzione di uno dei due lati, il lato parallelo a $ \vec{v} $ subirà la contrazione di un fattore $ \gamma^{-1} $ per un osservatore esterno, dunque $ S' = \gamma^{-1} S $:
\begin{equation}
	\sigma' = \gamma\sigma
	\label{eq:27}
\end{equation}

\subsubsection{Trasformazioni di Lorentz del campo elettrico}

Consideriamo un condensatore a facce piane parallele con densità di carica $ \pm\sigma $ a riposo in un sistema di riferimento $ \mathcal{S} $ e calcoliamo il campo elettrico in un sistema $ \mathcal{S}' $ in moto con $ \vec{v} = -v\,\hat{e}_x $: il campo elettrico tra le armature in $ \mathcal{S} $ è $ \vec{E} = \frac{\sigma}{\epsilon_0}\,\hat{e}_y $, ma a priori non sappiamo come esso sia visto da $ \mathcal{S}' $, quindi assumiamo che abbia un componente lungo $ \hat{e}_x $ (ignoriamo per il momento il campo magnetico). \\ 
%
Se prendiamo il campo tra le due armature, dato che le densità di carica sono opposte in segno anche le componenti $ E_x $ dei due campi elettrici saranno opposte in segno, quindi si elideranno tra loro: tra le due armature il campo elettrico rimane ortogonale ad esse.
Se prendiamo come superficie gaussiana un parallelepipedo di sezione orizzontale $ S $ che intersechi una delle due armature (WLOG quella con $ +\sigma $), la legge di Gauss ci dà $ \Phi(E') = E'S' = \frac{\sigma'S'}{\epsilon_0} $, dunque ricaviamo la legge di trasformazione:
\begin{equation}
	E'_{\perp} = \gamma E_{\perp}
	\label{eq:28}
\end{equation}
Applicando la stessa analisi ad un condensatore che invece si muova parallelamente al campo elettrico, ad esempio $ \vec{v} = v \,\hat{e}_z $, si trova:
\begin{equation}
	E'_{\parallel} = E_{\parallel}
	\label{eq:29}
\end{equation}
Queste trasformazioni sono generali e valgono per qualsiasi configurazione delle sorgenti.

\subsubsection{Campo elettrico di una carica in moto}

Consideriamo una carica $ q $ a riposo in un sistema di riferimento $	\mathcal{S} $: il suo campo elettrico è dato da:
\begin{equation}
	\vec{E}(\vec{r}) = \frac{q}{4\pi\epsilon_0} \frac{x\,\hat{e}_x + y\,\hat{e}_y}{(x^2+y^2)^{3/2}} \qquad\Longrightarrow\qquad \frac{E_y}{E_x} = \frac{y}{x} \quad (\text{campo radiale})
	\label{eq:30}
\end{equation} 
Consideriamo ora un sistema di riferimento $ \mathcal{S}' $ in moto con $ \vec{v} = -v\,\hat{e}_x $ ($ O' $ vede la carica muoversi con $ +v\,\hat{e}_x $) tale che le origini coincidano a $ t = 0 $: in generale $ x = \gamma (x' + \beta ct') $ e $ t = \gamma (t' - \frac{\beta}{c} x') $, quindi all'istante iniziale si ha $ x = \gamma x' $ e $ y = y' $, quindi:
\begin{equation}
	\vec{E}'(\vec{r}\,') = E'_x\,\hat{e}_x + E'_y\,\hat{e}_y = E_x\,\hat{e}_x + \gamma E_y\,\hat{e}_y = \frac{q}{4\pi\epsilon_0} \frac{\gamma x' \,\hat{e}_x + \gamma y' \,\hat{e}_y}{(\gamma^2x^2+y^2)^{3/2}}
	\label{eq:31}
\end{equation}
Quindi il campo continua ad essere radiale ($ \frac{E'_y}{E'_x} = \frac{y'}{x'} $). Per quando riguarda il modulo, svolgendo i calcoli si trova:
\begin{equation}
	E'(\vec{r}\,') = \frac{q}{4\pi\epsilon_0 r'^2} \frac{1-\beta^2}{(1 - \beta^2\sin^2\theta)^{3/2}}
	\label{eq:32}
\end{equation}
dove $ \theta $ è l'angolo tra $ \vec{r}\,' $ e $ \vec{v} $. \\ 
%
Tenendo $ r' $ fissato, si vede subito che $ E'(\frac{\pi}{2}) = \gamma E \ge E = E'(0) $, ovvero il campo elettrico di una carica in moto è maggiore nella direzione perpendicolare allo spostamento: di conseguenza, il campo elettrico non è più conservativo, poiché $ \oint_{\gamma}\vec{E}'\cdot d\vec{l} \neq 0 $ se si considera ad esempio un arco di corona tra $ [r'_1,r'_2] $ e $ [0,\frac{\pi}{2}] $, ovvero $ \rot\vec{E}' \neq 0 $. \\ 
%
Dato che il campo elettrico varia in modulo in base alla direzione e che col passare del tempo la carica si muove, se fissiamo un punto nello spazio e misuriamo il campo elettrico lì ad ogni istante misureremo un campo diverso: una carica in moto genera un campo elettrico variabile nel tempo. \\ 
%
Consideriamo ora una carica inizialmente ferma in $ x = 0 $: il campo elettrico è radiale e non dipende dalla direzione. \\Se la carica si mette in moto, nelle sue immediate vicinanze il campo elettrico è quello di una carica in movimento, appena determinato; lontano dalla carica, invece, e più precisamente all'esterno di una sfera centrata in $ x = 0 $ e di raggio $ ct $, l'informazione che la carica si è messa in moto non è ancora arrivata ed il campo è ancora quello elettrostatico: si viene quindi a creare un guscio sferico all'interfaccia tra queste due regioni, viaggiante a $ c $, in cui le linee di campo si devono raccordare, dunque non sono radiali ma essenzialmente tangenti alla sfera (lo spessore di questo guscio dipende dall'accelerazione iniziale della carica). \\ 
L'accelerazione della carica ha quindi generato un impulso viaggiante alla velocità della luce in cui il campo elettrico è praticamente perpendicolare alla direzione di propagazione dell'impulso: si vedrà che questo impulso è associato ad un campo magnetico e che il fronte della perturbazione è un'onda elettromagnetica che si propaga. \\ 
Supponiamo ora che la carica si fermi bruscamente in un istante che ridefiniremo come $ t = 0 $ ed in una nuova posizione $ x = 0 $: in ogni istante successivo a $ t = 0 $ il campo interno alla sfera centrata in $ x = 0 $ e di raggio $ ct $ è quello elettrostatico di una carica puntiforme, mentre, all'esterno di questa sfera l'informazione ancora non è arrivata ed il campo è ancora quello di una carica in moto che si trovasse dopo il punto in cui la carica si è fermata, ovvero nel punto $ x = vt $; all'interfaccia tra le due ragioni si viene a creare di nuovo un guscio sferico analogo al precedente. \\ 
%
Consideriamo una linea di campo singola (WLOG nel caso di decellerazione) e cerchiamo la relazione tra il suo tratto all'interno del guscio sferico e all'esterno di esso, caratterizzati da due angoli $ \theta_0 $ e $ \phi_0 $ (dentro e fuori) rispetto alla direzione di $ \vec{v} $ (WLOG $ \hat{e}_x $): per farlo, consideriamo il loop in figura. \\ 
%
%
%
\hbox{} \\ INSERIRE LA FIGURA \\ \hbox{} \\ 
%
%
%
In particolare, consideriamo i punti $ A $ ed $ F $ tali che essi si trovino ad una distanza fissata $ r $ rispettivamente $ x = 0 $ e da $ x = vt $ e prendiamo la superficie gaussiana data dalla rotazione del loop attorno all'asse $ x $: per il teorema di Gauss il flusso totale del campo elettrico attraverso questa superficie deve esse nullo, ed inoltre si nota che gli unici tratti del loop che contribuiscono al flusso sono i due archi all'interno ed all'esterno del guscio, poiché in tutti gli altri tratti il campo è parallelo alla superficie. Questi flussi sono:
\begin{equation}
	\Phi_{\text{interno}} = \int_{\text{arco interno}} \vec{E}\cdot d\vec{S} = -\frac{q}{4\pi\epsilon_0 r^2} 2\pi r^2 \int_0^{\theta_0} \sin\theta \,d\theta = -\frac{q}{2\epsilon_0} (1-\cos\theta_0)
	\label{eq:33}
\end{equation}
\begin{equation}
	\Phi_{\text{esterno}} = \int_{\text{arco esterno}} \vec{E}\cdot d\vec{S} = \frac{q}{4\pi\epsilon_0 r^2} 2\pi r^2 \int_0^{\phi_0} \displaystyle\frac{1 - \beta^2}{(1 - \beta^2 \sin^2\phi)^{3/2}} \sin\phi \,d\phi
	\label{eq:34}
\end{equation}
Risolvendo l'integrale si trova il flusso totale $ \Phi = \Phi_{\text{interno}} + \Phi_{\text{esterno}} $:
\begin{equation}
	\Phi = -\frac{q}{2\epsilon_0} (1-\cos\theta_0) + \frac{q}{2\epsilon_0} \left[ 1 - \frac{\sqrt{2}\cos\phi_0}{\sqrt{2 - \beta^2 + \beta^2 \cos2\phi_0}} \right] = 0
	\label{eq:35}
\end{equation}
da cui:
\begin{equation}
	\cos\theta_0 = \frac{\sqrt{2}\cos\phi_0}{\sqrt{2(1 - \beta^2\sin^2\phi_0)}} \qquad\Longrightarrow\qquad \cos^2\theta_0 = \frac{\cos^2\phi_0}{1 - \beta^2\sin^2\phi_0}
	\label{eq:36}
\end{equation}
Semplificando:
\begin{equation}
	\tan^2\theta_0 = \sin^2\theta_0 \frac{1 - \beta^2\sin^2\phi_0}{\cos^2\phi} = \left[1 - \frac{\cos^2\phi_0}{1 - \beta^2\sin^2\phi_0} \right] \frac{1 - \beta^2\sin^2\phi_0}{\cos^2\phi_0} = \frac{1 - \cos^2\phi_0}{\cos^2\phi_0} - \beta^2\tan^2\phi_0
	\label{eq:37}
\end{equation}
Dato che $ 1 - \cos^2\phi_0 = \sin^2\phi_0 $ e $ 1 - \beta^2 = \gamma^{-2} $, otteniamo la relazione semplificata:
\begin{equation}
	\tan\phi_0 = \gamma\tan\theta_0
	\label{eq:38}
\end{equation}

\subsubsection{Trasformazioni di $ \vec{E} $ e $ \vec{B} $}

Consideriamo di nuovo un condensatore a facce piane parallele che si muove con velocità $ \vec{u} = u \,\hat{e}_x $ in un sistema di riferimento $ \mathcal{S} $, nel quale i piani hanno densità di carica $ \pm\sigma $ e corrente $ \vec{K} = \pm\sigma\vec{u} $, le quali generano un campo elettrico $ \vec{E} = \frac{\sigma}{\epsilon_0} \,\hat{e}_y $ ed uno magnetico $ \vec{B} = \mu_0 \sigma u\,\hat{e}_z $ nello spazio fra i piani. \\ 
%
Consideriamo ora un sistema di riferimento $ \mathcal{S}' $ in moto rispetto ad $ \mathcal{S} $ con $ \vec{v} = v \,\hat{e}_x $: in esso il condensatore avrà velocità:
\begin{equation}
	u' = \frac{u - v}{1 - \frac{uv}{c^2}} = c \frac{\beta_u - \beta}{1 - \beta_u\beta}
	\label{eq:39}
\end{equation}
Di conseguenza:
\begin{equation}
	\gamma_{u'}^2 = \frac{1}{1 - \frac{u'^2}{c^2}} = \frac{(1 - \beta_u\beta)^2}{(1-\beta_u\beta)^2 - (\beta_u-\beta)^2} = \frac{(1-\beta_u\beta)^2}{1 + \beta_u^2\beta^2-\beta_u^2-\beta^2} = \frac{1}{1-\beta_u^2}\frac{1}{1-\beta^2}(1-\beta_u\beta)^2
	\label{eq:40}
\end{equation}
ovvero:
\begin{equation}
	\gamma_{u'} = \gamma_u\gamma(1-\beta_u\beta)
	\label{eq:41}
\end{equation}
Per quanto riguarda le sorgenti, nel sistema di riferimento solidale al condensatore si ha $ \sigma_0 = \frac{\sigma}{\gamma_u} $, mentre in $ \mathcal{S}' $ si ha $ \sigma' = \sigma_0\gamma_{u'} = \sigma\gamma(1-\beta_u\beta) $ e $ \vec{K'} = \sigma'u'\,\hat{e}_{x'} = \sigma\gamma c(\beta_u-\beta)\,\hat{e}_{x'} $. \\ 
%
Possiamo quindi scrivere:
\begin{equation}
	E'_{y'} = \frac{\sigma'}{\epsilon_0} = \gamma \frac{\sigma}{\epsilon_0} - \gamma \frac{\gamma}{\epsilon_0}\beta_u\beta = \gamma E_y - \gamma \frac{\sigma}{\epsilon_0 c^2} uv = \gamma E_y - \gamma \sigma \mu_0 uv = \gamma (E_y - vB_z)
	\label{eq:42}
\end{equation}
\begin{equation}
	B'_{z'} = \mu_0 K' = \mu_0 \sigma \gamma c \beta_u - \mu_0 \sigma \gamma c \beta = \gamma \mu_0 \sigma u - \frac{\sigma}{\epsilon_0 c^2} \sigma \gamma v = \gamma (B_z - \frac{v}{c^2} E_y)
	\label{eq:43}
\end{equation}
ovvero:
\begin{equation}
	E'_{y'} = \gamma (E_y - \beta cB_z) \qquad\qquad cB'_{z'} = \gamma (cB_z - \beta E_y)
	\label{eq:44}
\end{equation}
che sono simili a $ x' = \gamma (x - \beta ct) $ e $ ct' = \gamma (ct - \beta x) $. \\ 
%
Facendo lo stesso ragionamento su un sistema con piani paralleli al piano $ xy $ e che si muove sempre lungo $ \vec{e}_x $ si ricava:
\begin{equation}
	E'_{z'} = \gamma(E_z + \beta cB_y) \qquad\qquad cB'_{y'} = \gamma (cB_y + \beta E_z)
	\label{eq:45}
\end{equation}
%
Consideriamo ora un solenoide a riposo in $ \mathcal{S} $ con $ n $ spire per unità di lunghezza ed in cui scorre una corrente $ i $: il suo campo magnetico è $ \vec{B} = \mu_0 n i \,\hat{e}_x $ (direzione dell'asse del solenoide). \\ 
%
In un sistema di riferimento $ \mathcal{S}' $ in moto con $ \vec{v} = -v \,\hat{e}_x $ il campo magnetico è sempre diretto lungo $ x' $ e, per la contrazione delle lunghezze, $ n' = \gamma n $, mentre $ i' = \frac{dq}{dt'} = \gamma^{-1} i $, quindi il campo rimane invariato: $ B_x = B'_{x'} $ (analogamente al campo elettrico). \\ 
%
Possiamo dunque scrivere le leggi di trasformazione per il campo elettrico e per quello magnetico:
\begin{equation}
	\begin{split}
		\vec{E}'_{\parallel} &= \vec{E}_{\parallel} \qquad \vec{E}'_{\perp} = \gamma (\vec{E}_{\perp} + \vprod{v}{B}_{\perp}) \\ 
		\vec{B}'_{\parallel} &= \vec{B}_{\parallel} \qquad \vec{B}'_{\perp} = \gamma (\vec{B}_{\perp} - \tfrac{1}{c^2} \vprod{v}{E}_{\perp})
	\end{split}
	\label{eq:46}
\end{equation}
Queste trasformazioni suggeriscono che i campi elettrico e magnetico non siano indipendenti, ma manifestazioni di una stessa grandezza fisica: il campo elettromagnetico. \\ 
%
Notiamo due casi particolari:
\begin{itemize}
	\item $ \vec{B} = 0 $: se a riposo non è presente un campo magnetico, in un sistema di riferimento generico è presente $ \vec{B}' = \vec{B}'_{\perp} = -\gamma \frac{1}{c^2}\vprod{v}{E}_{\perp} = -\frac{1}{c^2}\vprod{v}{E}'_{\perp} $, ma $ \vprod{v}{E}_{\parallel} = 0 $, quindi vale $ \vec{B} = -\frac{1}{c^2}\vprod{v}{E}' $;
	\item $ \vec{E} = 0 $: se a riposo non è presente un campo eletttrico, in un sistema di riferimento generico è presente $ \vec{E}' = \vec{E}'_{\perp} = \gamma \vprod{v}{B}_{\perp} = \vprod{v}{B}'_{\perp} $, ma $ \vprod{v}{B}'_{\parallel} = 0 $, quindi vale $ \vec{E}' = \vprod{v}{B}' $.
\end{itemize}
Si può inoltre dimostrare che ci sono due invarianti: $ \sprod{E}{B} $ e $ E^2 - c^2B^2 $. Di conseguenza:
\begin{itemize}
	\item Se in un sistema inerziale uno dei due campi è nullo, allora in ogni altro sistema inerziale $ \vec{E}\perp\vec{B} $ (poiché $ \sprod{E}{B} = 0 $ cost.);
	\item Se in un sistema inerziale $ \sprod{E}{B} = 0 $, allora ci sono tre casi:
	\begin{enumerate}
		\item $ E^2 - c^2B^2 > 0 $: esiste un sistema $ \mathcal{S}' $ tale che $ \vec{B}' = 0 $ e il modulo della sua velocità relativa è $ v = c^2 \frac{B}{E} $;
		\item $ E^2 - c^2B^2 < 0 $: esiste un sistema $ \mathcal{S}' $ tale che $ \vec{E}' = 0 $ e il modulo della sua velocità relativa è $ v = \frac{E}{B} $;
		\item $ E^2 - c^2B^2 = 0 $: in ogni sistema inerziale vale $ E = c B $ ed è un'onda elettromagnetica che si propaga a velocità $ c $.
	\end{enumerate}
\end{itemize}

\subsubsection{Carica in moto rettilineo uniforme}

Consideriamo una carica $ q $ a riposo in un sistema di riferimento $ \mathcal{S} $ ed un sistema $ \mathcal{S}' $ in moto relativo con $ \vec{v} = -v\,\hat{e}_x $ (vede la carica muoversi con $ \vec{v}\,' = +v\,\hat{e}_x $): in $ \mathcal{S} $ si ha $ \vec{B} = 0 $. \\ 
In $ \mathcal{S}' $ vale $ \vec{B}' = \frac{1}{c^2} \vec{v}\,'\times\vec{E}' $, ma anche (fissato $ r' $):
\begin{equation}
	\vec{E}'(\theta') = \frac{1}{4\pi\epsilon_0} \frac{q}{r'^2} \frac{1 - \beta^2}{1 - \beta^2\sin^2\theta'} \,\hat{r}' \equiv E'(\theta') \,\hat{r}'
	\label{eq:47}
\end{equation}
quindi:
\begin{equation}
	\vec{B}' = E'(\theta')\frac{1}{c^2}\frac{v}{r'} (z\,\hat{e}_y - y\,\hat{e}_z)
	\label{eq:48}
\end{equation}
In $ \mathcal{S}' $ non solo il campo magnetico aumenta di intensità in direzione ortogonale al moto, ma è anche presente un campo magnetico sempre ortogonale al moto e con linee di campo circolari centrare sull'asse della direzione del moto. \\ 
Nel limite non-relativistico $ \beta \rightarrow 0 $ si ha $ \vec{B}' \rightarrow \frac{\mu_0}{4\pi}\frac{q}{r'^2}\vec{v}\,'\times\hat{r}' $, che, ricordando $ q\vec{v}\,' = i \vec{l} $, non è altro che la legge di Biot-Savart.

\subsubsection{Barretta conduttrice in moto rettilineo uniforme}

Consideriamo una barretta conduttrice a riposo nel sistema di riferimento $ \mathcal{S} $ (barretta sul piano $ xy $) immersa in un campo magnetico costante, omogeneo ed uniforme $ \vec{B} = B\,\hat{e}_z $ e con $ \vec{E} = 0 $. \\ 
%
Supponiamo ora che la barretta si muova con $ \vec{v} = v \,\vec{e}_y $: per effetto del movimento, si sviluppa una forza $ \vec{F} = F \,\hat{e}_x $ sulle cariche della barretta, quindi si crea uno sbilanciamento di cariche ed un campo elettrico $ \vec{E} $ che riporta le cariche in una situazione stazionaria; l'equilibrio tra forza di Lorentz e campo elettrico indotto porta a $ \vec{E}_{\text{int}} = -\vprod{v}{B} $: il campo elettrico esterno così generato non è né omogeneo né uniforme (vedere figura). \\ 
%
%
%
\hbox{} \\ INSERIRE LA FIGURA \\ \hbox{} \\ 
%
%
%
Mettiamoci ora in un sistema di riferimento $ \mathcal{S}' $ per cui $ \mathcal{S} $ si muove a $ \vec{v}' = -v\,\hat{e}_{y'} $ rispetto ad esso: in questo sistema il campo magnetico è $ \vec{B}' = \gamma B \,\hat{e}_z $, ma non ha effetto sulle cariche poiché in questo sistema sono ferme; d'altra parte, in $ \mathcal{S}' $ vi è un campo elettrico $ \vec{E} = \vprod{v}{B}' = \gamma vB \,\hat{e}_{x'} $, dunque il sistema è puramente elettrostatico: questo campo elettrico fa muovere le cariche che si dispongono sulla superficie della barretta, modificando il campo elettrico esterno e rendendolo nullo all'interno. \\ 
%
%
%
\hbox{} \\ INSERIRE LA FIGURA \\ \hbox{} \\ 
%
%
%
Ricapitolando:
\begin{itemize}
	\item in $ \mathcal{S} $: la barretta si muove in un campo magnetico uniforme e la forza di Lorentz fa muovere le cariche verso la superficie, generando un campo elettrico che all'interno del conduttore va ad annullare la forza totale;
	\item in $ \mathcal{S}' $: la barretta è ferma in un campo magnetico uniforme (più intenso di prima) ed è presente un campo elettrostatico dato da $ \vec{E}' = \vprod{v}{B}' $, il quale fa muovere le cariche verso la superficie generando un campo elettrico indotto che annulla il campo totale all'interno e si va a sommare a quello esterno.
\end{itemize}
In entrambi i sistemi la barretta è sottoposta allo stesso effetto fisico, ma i campi in gioco sono diversi.
