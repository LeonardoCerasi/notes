\selectlanguage{italian}

\section{Modello di Yukawa}

Yukawa nel 1934 propose il primo modello per spiegare l'interazione tra nucleoni in termini di uno scambio di bosoni, ricalcando quello per l'interazione elettromagnetica.\\
In particolare, la teoria di campo proposta da Yukawa si basa sull'ipotesi che i nucleoni stessi siano le sorgenti del campo bosonico attraverso il quale interagiscono. Il campo d'interazione proposto deve avere le seguenti proprietà:
\begin{itemize}
	\item l'interazione deve avere un raggio d'azione $ a \sim 1\fm $;
	\item deve essere indipendente dalla carica elettrica (simmetria d'isospin);
	\item deve dipendere dallo stato di spin del sistema di nucleoni.
\end{itemize}
La terza condizione non era inclusa nel modello originale di Yukawa, e per semplicità si considera un potenziale a simmetria sferica. In questo caso, il campo bosonico può essere descritto dall'equazione di Klein-Gordon, che nel limite statico a simmetria sferica diventa:
\begin{equation*}
	\frac{1}{r^2} \frac{d}{dr} \left( r^2 \frac{d\phi(r)}{dr} \right) - \frac{mc}{\hbar} \phi(r) = 0
	\quad \Rightarrow \quad
	\phi(r) = \frac{\mathcal{N}}{r} e^{- r / a}\,,\,\, a = \frac{\hbar}{mc}
\end{equation*}
Si trova quindi che il campo di Yukawa, centrato in ciascun nucleone, è:
\begin{equation}
	V_{\text{Y}}(r) = \frac{g_s}{r} e^{-r / a}\,,\quad a = \frac{\hbar}{mc}
	\label{eq:9.1}
\end{equation}
dove $ g_s $ è la coupling constant dell'interazione. Dato che il raggio d'azione dell'interazione tra nucleoni è $ a \sim 1\mev $, si trova che il bosone mediatore, detto \textit{mesone} ($ \virgolette{stato intermedio} $) deve avere massa $ m \sim 200\mev/c^2 $: questo è in accordo col principio d'indeterminazione di Heisenberg, poiché detto $ \Delta t $ il tempo d'esistenza del mesone, si ha $ a = c \Delta t \sim \frac{\hbar c}{\Delta E} = \frac{\hbar}{mc} $.\\
Ad oggi, in realtà, è noto che il mesone di Yukawa non è una particella elementare, bensì composta da una coppia quark-antiquark: in particolari, l'interazione tra nucleoni a lunga distanza è mediata da mesoni $ \pi $ ($ 140\gev/c^2 $), a distanza ottimale ($ \approx 0.8\fm $) da mesoni $ \sigma $ ($ 500\mev/c^2 $) e a corta distanza, quando la forza diventa repulsiva, da mesoni $ \omega $ e $ \rho $ ($ 784\mev/c^2 $).\\
Inoltre, la teoria fondamentale alla base delle interazioni tra nucleoni è quella dell'interazione forte, la Cromodinamica Quantistica (QCD): questa descrive le interazioni tra quarks mediate da gluoni, non direttamente quelle tra nucleoni. Sebbene l'interazione tra nucleoni tramite scambio di mesoni sia ricavabile dalla QCD, essa va modificata quando avviene in presenza di altri nucleoni, come nel caso di un nucleo atomico: questo è un sistema quantistico a molti corpi estremamente complesso e per essere descritto richiede un modello d'interazione effettiva. Un tale esempio è la Lattice QCD, la quale può essere incorporata nella Chiral Effective Field Theory: questa teoria semplifica il modello del nucleo atomico considerando i nucleoni come gradi di libertà del sistema (al posto di quarks e gluoni).

\subsection{Simmetria di isospin}

Dato che il protone ed il neutrone non vengono distinti dall'interazione forte e che $ \left( m_n - m_p \right) / m_n \approx 10^{-3} $, Heisenberg propose di considerarli come due stati distinti di una stessa particella, il nucleone. In analogia allo spin, egli introdusse l'\textit{isospin} (o spin isotropico) $ I $, una grandezza adimensionale che si comporta matematicamente in maniera identica allo spin, assegnando $ I = \frac{1}{2} $ al nucleone e distinguendo protone e neutrone in base alla terza componente dell'isospin (analogo alla componente $ z $ dello spin): $ I_3 = +\frac{1}{2} $ per il protone e $ I_3 = -\frac{1}{2} $ per il neutrone.\\
L'hamiltoniana associata all'interazione forte è invariante per tutte le operazioni nello spazio astratto dell'isospin. Trascurando l'interazione elettromagnetica e quella debole, dunque, i livelli energetici del sistema sono degeneri e possono essere classificati secondo l'isospin totale $ I $: essendo analogo allo spin, i suoi possibili valori sono interi e semi-interi, e ad ogni suo valore corrisponde un multipletto di $ (2I + 1) $ autostati con la stessa energia ma con valori diversi di $ I_3 \in \left[ -I, I \right] $.\\
Mentre l'isospin totale si comporta come lo spin, la terza componente dell'isospin si comporta come la carica elettrica: mentre l'interazione forte, indipendente da $ I_3 $ e $ Q $ e dipendente solo da $ I $, conserva l'isospin totale, l'interazione elettromagnetica conserva solo $ I_3 $ e l'interazione debole non conserva l'isospin. L'indipendenza dell'interazione forte da $ I_3 $ e $ Q $ si può vedere, a livello nucleare, osservando che $ \ch{^7Li} $ e $ \ch{^7Be} $ hanno la stessa binding energy: questi sono \textit{mirror nuclei}, dunque hanno stesso spin e parità; formano inoltre un doppietto d'isospin con $ I = \frac{1}{2} $, dove $ \ch{^7Li} $ ha $ I_3 = -\frac{1}{2} $ e $ \ch{^7Be} $ ha $ I_3 = +\frac{1}{2} $. La simmetria di carica è verificata poiché ti trova che la forza protone-protone è uguale a quella neutrone-neutrone. Un'altra evidenza sperimentale deriva dalla seguente reazione:
\begin{equation*}
	\ch{^2H} + \ch{^2H} \rightarrow \ch{^4He} + \pi^0
\end{equation*}
L'isospin totale non è conservato, poiché $ 0 + 0 \neq 0 + 1 $, mentre $ I_3 $ lo è ($ 0 + 0 = 0 + 0 $), dunque la reazione è proibita dall'interazione forte ma amessa da quella elettromagnetica: sperimentalmente, si trova che la sezione d'urto della reazione è quella elettromagnetica e non quella forte, confermando l'ipotesi.\\
Anche i pioni formano un tripletto d'isospin: essi hanno proprietà identiche, eccetto la carica elettrica, e infatti per l'interazione forte essi sono come tre stati degeneri di una stessa particella. Dato che la degenerazione è di grado 3, il pione deve avere $ I = 1 $: $ \pi^+ $ ha $ I_3 = +1 $, $ \pi^0 $ ha $ I_3 = 0 $ e $ \pi^- $ ha $ I_3 = -1 $.\\
In generale, la relazione tra $ I_3 $, carica elettrica $ Q $ e numero barionico $ B \equiv \frac{1}{3} \left( n_q - n_{\bar{q}} \right) $ è dato dalla \textit{formula di Gell-Mann-Nishijima}:
\begin{equation}
	Q = I_3 + \frac{1}{2} B
	\label{eq:9.2}
\end{equation}
Questa formula considera solo quark di prima generazione ($ u,d $): nel caso si includano tutte le tre generazioni fermioniche, vanno aggiunti ulteriori termini.

\paragraph{Selection rules}

La conservazione dell'isospin porta a delle selection rules per i processi ad interazione forte. Ad esempio, si considerino le seguenti reazioni:
\begin{equation*}
	p + p \rightarrow \ch{^2H} + \pi^+
\end{equation*}
\begin{equation*}
	p + n \rightarrow \ch{^2H} + \pi^0
\end{equation*}
Entrambe hanno uno stato finale con $ I = 1 $; la prima reazione ha uno stato iniziale con $ I = 1 $, mentre la seconda ha uno stato iniziale che è una sovrapposizione di uno stato con $ I = 0 $ (50\%) ed uno con $ I = 1 $ (50\%): di conseguenza, dato che entrambe le reazioni sono dovute all'interazione forte, la prima procede interamente, mentre la seconda solo nel 50\% dei casi. Si deve quindi avere $ \sigma(pp \rightarrow d\pi^+) / \sigma(pn \rightarrow d\pi^0) = 2 $, che è proprio ciò che si osserva sperimentalmente.

\paragraph{Isospin dei quarks}

A partire da come sono costituiti $ p \left( uud \right) $ ed $ n \left( udd \right) $, si ha che i quarks $ u $ e $ d $ formano un doppietto di isospin $ I = \frac{1}{2} $, con $ I_3 = +\frac{1}{2} $ per $ u $ e $ I_3 = -\frac{1}{2} $ per $ d $. I rispettivi antiquarks hanno il segno di $ I_3 $ invertito, mentre gli altri quarks hanno $ I = 0 $.\\
Al giorno d'oggi si pensa che la simmetria di isospin sia dovuta alla quasi uguaglianza tra quark up e down ($ m_u \approx m_d $).

\subsection{Esperimenti di Conversi, Pancini e Piccioni}

Essendo nel giusto range di massa, inizialmente si pensò che il muone, allora noto come mesotrone, fosse il mesone di Yukawa (teorizzato nel 1934, mesotrone scoperto nel 1936). Con i loro esperimenti, però, Conversi, Pancini e Piccioni dimostrarono che questa identificazione era errata.\\
Il loro metodo sperimentale si basa sulle fast delayed coincidences, così da poter misurare con suddificiente precisioni i tempi di decadimento dell'ordine dei microsecondi. L'apparato sperimentale permette di studiare la penetrazione e l'assorbimento di mesotroni cosmici in diversi materiali ed in base al loro segno: mesotroni negativi dovrebbero essere catturati nell'orbita idrogenoide K degli atomi e poi, se mediatori dell'interazione forte, interagire col nucleo atomico prima di decadere; mesotroni positivi, invece, dovrebbero essere respinti dal nucleo e decadere nello spazio vuoto tra gli atomi del materiale assorbitore.\\
Sperimentando con un assorbitore di grafite, nel 1946 CPP trovarono che i mesotroni positivi e negativi si comportavano in maniera praticamente identica e con vite medie troppo lunghe per essere effettivamente i mediatori dell'interazione forte: Fermi, Teller e Weisskopf calcolarono che la probabilità d'assorbimento di un mesotrone negativo in quiete è di un fattore $ 10^{12} $ inferiore a quella prevedibile per un mesone di Yukawa. Nel 1947 fu osservato il pione, che sarebbe poi stato identificato come il mesone di Yukawa.










