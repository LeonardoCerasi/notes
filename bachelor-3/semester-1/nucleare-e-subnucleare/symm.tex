\selectlanguage{italian}

Il Modello Standard della fisica delle particelle comprende il modello a quarks e il modello elettrodebole: quest'ultimo è stato introdotto a partire dallo studio delle violazioni di varie simmetrie.

\section{Simmetrie}

Oltre al modello a quarks, per mettere ordine nel particle zoo è necessario introdurre delle simmetrie che si osservano in varie reazioni.\\
Il concetto di simmetria in fisica delle particelle è stato introdotto da Weyl e Feynman: in generale essa è presente quando, a seguito dell'applicazione di una certa trasformazione ad un oggetto, esso rimane invariato; si parla di violazione di simmetria quando invece esso risulta modificato. Come messo in luce da Noether, il concetto di simmetria è estremamente importante in fisica poiché è legato alle leggi di conservazione di determinate quantità.\\
Il Modello Standard presenta naturalmente tre \textit{quasi-simmetrie}:
\begin{itemize}
	\item C-symmetry (carica): particella $ \mapsto $ antiparticella;
	\item P-symmetry (parità): $ \ve{r} \mapsto -\ve{r} $;
	\item T-symmetry (inversione temporale): $ t \mapsto -t $.
\end{itemize}
Queste sono dette quasi-simmetrie perché nell'Universo attuale ciascuna di esse è violata: ad esempio, l'interazione debole viola la C-symmetry e la P-symmetry (vedere esperimenti di Goldhaber e di Madame Wu), mentre il secondo principio della termodinamica asserisce che l'universo evolve in una direzione temporale ben precisa, violando la T-symmetry; inoltre, il decadimento dei kaoni neutri viola la CP-symmetry. Il Modello Standard, però, predice che la CPT-symmetry debba essere una simmetria: Ludens, Pauli e Schwinger (indipendentemente) hanno dimostrato che la LOrentz invariance implica la CPT-symmetry, dunque non c'è nessuna teoria consistente che permetta la violazione di tale simmetria. Una conseguenza della CPT-symmetry è che particelle e antiparticelle hanno stessa massa, stessa vita media e carica elettrica e momento magnetico uguali in modulo e di segno opposto.

\subsection{Esperimento di Madame Wu}

Nel 1956 Lee e Yang, con una literature review, notarono che i dati sperimentali non confermavano la P-symmetry nell'interazione debole. Rivolgendosi a Madame Wu, esperta di spettroscopia $ \beta $, proposero esperimenti per testare la P-symmetry nel decadimento $ \beta $.\\
L'esperimento di Madame Wu studia il decadimento $ \ch{^{60}Co} \rightarrow \ch{^{60}Ni}^* + e^- + \bar{\nu}_e $, con successiva emissione di $ 2\gamma $ per diseccitare il Nichel: gli osservabili sono dunque i fotoni e l'elettrone. In linea teorica, considerando un nucleo di cobalto nel suo rest-frame con spin $ \ve{s} $ e supponendo che l'elettrone venga emesso con momento $ \ve{p} $, l'operatore di parità agisce come $ \hat{\mathcal{P}} \ve{s} = \ve{s} $ e $ \hat{\mathcal{P}} \ve{p} = - \ve{p} $ (il momento angolare, essendo $ \ve{L} = \ve{r} \times \ve{p} $, è conservato dalla parità), dunque un modo per testare la P-symmetry è studiare la direzione d'emissione degli elettroni dal campione. In coordinate sferiche con $ \ve{s} \parallel \ve{e}_z $, basta studiare l'angolo d'emissione $ \theta $: l'operatore parità porta $ \theta \mapsto \pi - \theta $, dunque un'asimmetria nell'emissione dei elettroni dal campione agli angoli $ \theta $ e $ \pi - \theta $ indicherebbe una violazione della P-symmetry. Per quanto riguarda la distribuzione dei raggi $ \gamma $, essa è invariante per operatore di parità, poiché una volta fissato l'asse dello spin essa è determinata solo dal $ \Delta \ell $ della transizione (nel caso del $ \ch{^{60}Ni}^*(4^+) \rightarrow \ch{^{60}Ni}(0^+) $ sono due transizioni $ \Delta \ell = 2 $, dunque quadripolari).

\subsubsection{Apparato sperimentale}

\begin{figure}
	\centering
	\includegraphics[width = 0.70\textwidth]{wu-exp.png}
	\caption{Madame Wu's experimental setup.}
	\label{wu-exp}
\end{figure}

L'apparato sperimentale di Madame Wu è schematizzato in Fig. \ref{wu-exp}. La sorgente di cobalto è inserita in misura opportuna per favorire il raffreddamento del sistema: infatti, una volta che il sistema è stato magnetizzato, per mantenere l'allineamento degli spin è necessario raffreddarlo a temperature di $ \sim 10\,\text{mK} $, poiché l'agitazione termica tende a rompere l'allineamento.\\
Per controllare che il sistema mantenga l'allineamento degli spin per un periodo sufficiente di tempo, sono presenti due scintillatori posti a 90° per monitorare la distribuzione dei raggi $ \gamma $: questa dipende dall'asse di spin, con una forte differenza di emissione tra il piano equatoriale e quello polare. Come si può vedere in Fig. \ref{wu-data}, il sistema mantiene correttamente la polarizzazione per circa $ 6 - 8\,\text{min} $.

\begin{figure}
	\centering
	\includegraphics[width = 0.70\textwidth]{wu-data.png}
	\caption{Data from Madame Wu's experiment.}
	\label{wu-data}
\end{figure}

Sempre in Fig. \ref{wu-data} si può vedere che, per tutto il tempo in cui la direzione di riferimento data dagli spin polarizzati è ben definita, le distribuzioni di elettroni forward ($ \theta $) e backward ($ \pi - \theta $) sono decisamente asimmetriche: c'è una direzione preferenziale di emissione, che è quella backward rispetto alla direzione dello spin, dunque si evince che il decadimento $ \beta $ effettivamente viola la P-symmetry.

\subsection{Esperimento di Goldhaber}

A seguito dell'esperimento di Madame Wu sulla violazione della P-symmetry, nel 1957 Goldhaber mostrò che l'interazione debole violava anche la C-symmetry.\\
Innanzitutto, si definisce l'\textit{helicity} di una particella di momento $ \ve{p} $ e spin $ \ve{s} $ come:
\begin{equation}
	h \defeq \frac{\ve{s} \cdot \ve{p}}{\norm{\ve{s}} \norm{\ve{p}}}
	\label{eq:10.1}
\end{equation}
A seconda che il momento della particella sia parallelo o antiparallelo allo spin, l'helicity assume i valori $ +1 $ (right-handed) o $ -1 $ (left-handed). Inoltre, si vede che $ \hat{\mathcal{P}}h = -h $.\\
Con un esperimento molto sofisticato, Goldhaber riuscì a dimostrare che neutrini ed antineutrini hanno helicity differente. Per studiare un processo debole senza incorrere nel problema che il decadimento $ \beta $ è un decadimento a tre corpi, si considera la cattura elettronica che, essendo un processo a due corpi, permette di controllare energia e momento angolare e studiare le leggi di conservazione. In particolare, si consideri l'electron K-capture $ \ch{^{152}Eu}(0^-) + e^- \rightarrow \ch{^{152}Sm}^*(1^-) + \nu_e $, seguita da $ \ch{^{152}Sm}^*(1^-) \rightarrow \ch{^{152}Sm}(0^-) + \gamma $: essendo processi a due corpi, si ha $ E_{\nu_e} = 840\kev $ ed $ E_{\gamma} = 961\kev $. Si dimostra che $ h(\nu_e) = h(\ch{^{152}Sm}^*) = h(\gamma) $, dunque per trovare l'helicity del neutrino basta misurare quella del fotone. Quest'ultima è deducibile dal Compton scattering: la Compton cross-section dipende dall'orientazione relativa degli spin del fotone e dell'elettrone, ed in particolare si osserva un minor assorbimento se lo spin del fotone è parallelo a quello dell'elettrone. Dalle misure sperimentali di Goldhaber, si trova $ h(\nu_e) = -1.0 \pm 0.3 $, ovvero i neutrini sono sempre left-handed, mentre gli antineutrini sempre right-handed: di conseguenza, non c'è simmetria per scambio particella-antiparticella.

\subsection{Decadimento dei kaoni}

Il kaone neutro presenta un comportamento strano: esso infatti può decadere sia in due che in tre pioni, con tempi di decadimento molto diversi. Si parla di kaone short-lived $ K^0_S $, il quale decade come $ K^0_S \rightarrow \pi^+ \pi^- $ o $ K^0_S \rightarrow 2\pi^0 $ con $ \tau \approx 8.95\cdot10^{-11}\,\text{s} $, e di kaone long-lived $ K^0_L $, il quale decade come $ K^0_L \rightarrow \pi^+ \pi^- \pi^0 $ o $ K^0_L \rightarrow 3\pi^0 $ con $ \tau \approx 5.12\cdot10^{-8}\,\text{s} $.\\
Dato che i mesoni hanno sempre parià (intrinseca) negativa (quarks ed antiquarks hanno parità opposte), il decadimento del $ K^0_S $ è un'ulteriore dimostrazione della violazione della P-symmetry nei processi deboli.
Il decadimento del kaone neutro, però, può anche mettere in luce come l'interazione debole violi anche la CP-symmetry: con un esperimento del 1963, Cronin e Fitch dimostrarono che occasionalmente un kaone $ K^0_L $ può decadere in due pioni, violando anch'esso la P-symmetry.

\subsubsection{Esperimento di Cronin e Fitch}

L'esperimento consiste in un fascio di kaoni neutri all'interno di un tubo di $ 17\,\text{m} $, all'interno del quale vengono rilevati i pioni prodotti dal decadimento dei kaoni. Data la differenza di 3 ordini di grandezza nelle vite medie, ci si aspetterebbe di osservare soltanto i decadimenti a tre pioni del $ K^0_L $ all'estremità opposta del tubo: vengono invece osservati anche dei decadimenti a due pioni, con una frequenza di circa 1 ogni 500 decadimenti.\\
I decadimenti a doppio pione così osservati non sono attribuibili al decadimento del $ K^0_S $: data la half-life $ t_{1/2} = 5.5\cdot10^{-10}\,\text{s} $ ed assumendo $ v = 0.98c $, nel rest-frame la popolazione diventa $ 1/500 $ in appena $ 17\,\text{cm} $, ovvero $ 85\,\text{cm} $ nel frame del laboratorio ($ \gamma \approx 5 $); per qualsiasi velocità fisica, nessun $ K^0_S $ riesce a decadere all'estremità del tubo a $ 17\,\text{m} $.\\
L'unica spiegazione è dunque che in 1 decadimento ogni 500 il $ K^0_L $ decade in due pioni al posto di tre, violando dunque la CP-symmetry. Ciò può essere illustrato col modello a quarks: ricordando che $ \ket{K^0} = \ket{d\bar{s}} $ e $ \bar{K}^0 = \ket{\bar{d}s} $, si trova che:
\begin{equation*}
	\ket{K^0_S} = \frac{1}{\sqrt{2}} \left( \ket{K^0} + \ket{\bar{K}^0} \right)
	\qquad \qquad
	\ket{K^0_L} = \frac{1}{\sqrt{2}} \left( \ket{K}^0 - \ket{\bar{K}^0} \right)
\end{equation*}
Questi sono entrambi autostati di $ \hat{\mathcal{C}}\hat{\mathcal{P}} $, con rispettivi autovalori $ +1 $ e $ -1 $: si vede dunque che il decadimento del $ K^0_L $ secondo il canale del $ K^0_S $ viola la CP-symmetry.

\paragraph{Bariogenesi}

Il modello di bariogenesi proposto da Sakharov richiede la violazione della CP-symmetry per spiegare l'asimmetria tra materia ed antimateria, dunque lo studio di tale simmetria è di fondamentale importanza. Ad oggi, la violazione della CP-symmetry è stata osservata solo per l'interazione debole e non ci sono ancora evidenze che sussista anche per l'interazione forte.

\subsection{T-symmetry}

Data la violazione della CP-symmetry, se la CPT-symmetry deve sempre essere preservata, allora è necessario che i processi che violano la CP-symmetry violino anche la T-symmetry, ovvero che la probabilità del processo sia $ P(t) \neq P(-t) $.\\
Ad oggi, nessuna evidenza sperimentale di violazioni della T-symmetry è stata ancora osservata. Un esempio potrebbe essere la misura di un momento di dipolo elettrico non-nullo per il neutrone.

\subsubsection{Neutron electric dipole moment}

Il momento di dipolo elettrico del neutrone (nEDM), indicato con $ d_n $, dà un'indicazione sulle distribuzioni di cariche elettriche positive e negative all'interno del neutrone: in particolare, $ d_n \neq 0 $  indicherebbe che i centri di tali distribuzioni non coincidono. Ad oggi, la stima migliore non permette di dirimere la questione: $ d_n = (0.0 \pm 1.1) \cdot 10^{-26} e \, \text{cm} $.\\
Un momento di dipolo elettrico permanente nel neutrone violerebbe sia la P-symmetry che la T-symmetry: infatti, $ \hat{\mathcal{P}} d_n = - d_n $ e $ \hat{\mathcal{P}} \mu_n = \mu_n $, mentre $ \hat{\mathcal{T}} d_n = d_n $ e $ \hat{\mathcal{T}} \mu_n = - \mu_n $, dunque sia applicando $ \hat{\mathcal{P}} $ che applicando $ \hat{\mathcal{T}} $ si avrebbe una configurazione per $ d_n $ e $ \mu_n $ diversa da quella iniziale, violando la rispettiva simmetria.

\paragraph{Atomic EDM}

Un'alternativa per ricerche sulla violazione della T-symmetry è la ricerca di momenti di dipolo elettrico permanenti in nuclidi con deformazioni statiche permanenti (dunque non vibrazioni), in particolare nuclei con octupole deformations ($ \virgolette{pear-shaped} $). Negli anni sono stati individuati almeno tre nuclei pesanti con asimmetria rigida di carica ($ \ch{^{222}Ra} $, $ \ch{^{224}Ra} $ e $ \ch{^{226}Ra} $) e la ricerca al momento è incentrata sul come collegare tali misure a quelle di dipolo elettrico permanente nelle particelle.


























\section{Modello elettrodebole}

Negli anni '50, a seguito della trattazione rigorosa dell'Elettrodinamica Quantistica (QED) e la formulazione da parte di Yang e Mills della teoria di gauge $ \SUn{2} $ per l'isospin, i fisici teorici iniziarono a notare una certa difficoltà nello svolgere i calcoli relativi all'interazione forte, dunque l'interesse si spostò sull'interazione debole: in particolare, si iniziò a pensare che ad alte energie l'interazione debole e quella elettromagnetica potrebbero essere unificate in un'unica \textit{interazione elettrodebole}.
Un'eventuale teoria elettrodebole, però, deve subire uno spontaneous symmetry breaking a basse energie, poiché i fotoni sono masless e raggio d'azione infinito, mentre i bosoni che mediano l'interazione debole sono molto massivi e short range.

\subsection{Interazione debole}

Sperimentalmente, si trova che le dimensionless coupling constants delle interazioni del Modello Standard sono legate da:
\begin{equation}
	\alpha_s : \alpha_e : \alpha_w = 1 : 10^{-2} : 10^{-6}
	\label{eq:10.8454243}
\end{equation}
L'interazione debole può essere descritta tramite un modello a scambio di particelle, come le altre due interazioni. In particolare, essa è mediata da tre bosoni massivi di spin 1 (come il fotone, solo le proiezioni $ m_s = \pm 1 $ sono ammesse): $ W^{\pm} $ e $ Z^0 $. Sperimentalmente, questi sono stati osservati nel 1983 da Rubbia e Van der Meer: essi sono bosoni estremamente massivi, con $ m_{W^{\pm}} = 80.4\gev/c^2 $ e $ m_{Z^0} = 91.2\gev/c^2 $. Di conseguenza, dato che il principio d'indeterminazione impone un range $ \Delta x \sim \frac{\hbar}{m c} $, si trova che l'interazione debole ha il range più corto tra le interazioni: $ \Delta x \sim 10^{-3} \fm $.\\
I tempi d'interazione sono più lunghi rispetto all'interazione forte: per i decadimenti deboli, si ha $ \tau \sim 10^{-8} - 10^{-13}\,\text{s} $. Inoltre, a differenza del bosone $ Z^0 $, i bosoni $ W^{\pm} $ possono cambiare il flavour dei quarks e leptoni con cui interagiscono: di conseguenza, l'interazione debole non conserva i numeri quantici associati alle generazioni di quarks (isospin, stranezza, etc.).\\
Un esempio di decadimento debole è il decadimento del neutrone $ n \rightarrow p e^- \bar{\nu}_e $: un quark down nel neutrone diventa un quark up, rendendo il nucleone un protone, emettendo un bosone $ W^- $ virtuale, il quale decade entro $ 10^{-18}\,\text{m} $ in una coppia elettrone-antineutrino elettronico.

\subsection{Teoria GWS}

Un modello di unificazione dell'interazione elettromagnetica e di quella debole fu proposto a fine anni '70 da Glashow, Salam e Weinberg. Nella teoria GWS, ad alte energie l'interazione elettrodebole è mediata da quattro bosoni massless di spin 1; scendendo a basse energie, tre di questi bosoni acquistano massa per rottura spontanea di simmetria. Il processo di spontaneous symmetry breaking è legato all'\textit{angolo di Weinberg} (o weak mixing angle):
\begin{equation}
	\cos \theta_W \defeq \frac{m_{W^{\pm}}}{m_{Z^0}}
	\label{eq:10.2315454}
\end{equation}
Dai dati sperimentali, si trova $ \sin^2 \theta_W = 0.22305 $. Questo processo porta ad un coupling dei bosoni $ W^{\pm} $ e $ Z^0 $ col campo di Higgs (mediato dal bosone di Higgs, strength proporzionale alle masse), il quale è responsabile dell'acquisizione di massa delle particelle.










