\selectlanguage{italian}

\section{Rilevatori di radiazione}

A seconda del tipo di particella interagente da rilevare, c'è bisogno di un differente tipo di rilevatore, data la diversa natura delle particelle $ \alpha $, $ \beta $ e $ \gamma $:
\begin{itemize}
	\item rilevatori a gas, usati principalmente per particelle $ \alpha $ e protoni (es.: contatore Geiger-Müller);
	\item rilevatori a scintillazione, usati principalmente per radiazione $ \gamma $ (divisi in organici ed inorganici);
	\item rilevatori a stato solido, basati sui semiconduttori;
\end{itemize}
Per i neutroni il discorso è diverso, in quanto sono particelle neutre: essi interagiscono più facilmente con i protoni, dunque si utilizza come materiale frenante l'acqua, materiali a basso $ Z $ o materiali ad alta cross-section per cattura neutronica.

\section{Applicazioni}

Le tecniche usate in Fisica Nucleare trovano numerose applicazioni al di fuori di tale ambito. Alcuni esempi sono:
\begin{itemize}
	\item analisi della composizione dei materiali, in cui le tecniche nucleari sono preferite a quelle chimiche, in quanto non distruttive;
	\item spettrometria di massa, usata per separare ioni di uguale carica ma massa diversa;
	\item datazione radiometrica, in particolare con l'uso del $ \ch{^{14} C} $ (sebbene diventerà non applicabile nei prossimi millenni, a causa delle emissioni da utilizzo di combustibili fossili);
	\item medicina nucleare, in particolare la \textit{Positron Emission Tomography} (PET) per quanto riguarda la diagnosi e l'adroterapia per la cura (si usano protoni poiché la maggior parte dell'energia è rilasciata nel picco di Bragg, e non all'inizio come per i raggi X).
\end{itemize}
