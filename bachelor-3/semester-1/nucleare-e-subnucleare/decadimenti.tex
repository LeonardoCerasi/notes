\selectlanguage{italian}

\section{Radioattività}

I primi indizi verso la radioattività sono derivati dalla rilevazione dei raggi X: questi sono onde elettromagnetiche, ovvero fotoni, generate da transizioni di elettroni tra livelli energetici ($ \Delta E_e = \hbar \omega $); questi dunque non sono processi nucleari, dato che avvengono all'interno della nube elettronica dell'atomo.\\
Quando si parla di radioattività in senso stretto, però, si fa riferimento ai decadimenti dei nuclidi.

\subsection{Decadimenti radioattivi}

I decadimenti radioattivi sono processi in cui un nuclide stabile raggiunge una configurazione con energia più bassa emettendo spontaneamente radiazione.\\
Utilizzando un campo magnetico, Rutherford riuscì a distinguere tre tipologie di radiazione, e dunque di decadimenti radioattivi:
\begin{enumerate}
  \item decadimento $ \alpha $, in cui la radiazione è costituita da un nucleo di $ \ch{^4_2 He} $ ed è poco penetrante, coinvolge l'interazione elettromagnetica e quella forte;
  \item decadimento $ \beta^{\pm} $, in cui la radiazione è costituita da un $ e^{\pm} $ ed è mediamente penetrante, coinvolge l'interazione debole:
    \begin{itemize}
      \item decadimento $ \beta^- $: $ n \rightarrow p^+ + e^- + \bar{\nu}_e $;
      \item decadimento $ \beta^+ $: $ p^+ \rightarrow n + e^+ + \nu_e $;
    \end{itemize}
  \item decadimento $ \gamma $, in cui la radiazione è costituita da un fotone con energia dell'ordine delle decine di MeV, dunque estremamente penetrante.
\end{enumerate}
A questi si aggiunge la cattura elettronica, un decadimento in cui un nuclide proton-rich cattura un elettrone dalle shell interne dell'atomo, seguendo la reazione $ p^+ + e^- \rightarrow n + \nu_e $, emettendo raggi X a seguito del rimpiazzo dell'elettrone interno con uno dalle shell esterne.

\subsection{Energy balance}

Un decadimento radioattivo può essere visto come un caso particolare di reazione nucleare; è quindi possibile definire il $ Q $-value del decadimento come la differenza di energia a riposo (massa) tra reagenti (nuclide instabile) e prodotti, così da poter stabilire qualora esso sia possibile, ovvero spontaneo, con la condizione $ Q > 0 $.\\
In particolare, si definiscono i $ Q $-values dei seguenti decadimenti:
\begin{enumerate}
  \item decadimento $ \alpha $: $ Q_{\alpha} \equiv \left[ M(Z,A) - M(Z-2,A-4) - m(\ch{^4_2 He}) \right] c^2 $;
  \item decadimento $ \beta^- $: $ Q_{\beta^-} \equiv \left[ M(Z,A) - M(Z+1,A) - m_e \right] c^2 $;
  \item decadimento $ \beta^+ $: $ Q_{\beta^+} \equiv \left[ M(Z,A) - M(Z-1,A) - m_e \right] c^2 $;
  \item electron capture: $ Q_{e} \equiv \left[ M(Z,A) + m_e - M(Z-1,A) \right] c^2 $;
\end{enumerate}
Si vede subito che $ Q_e > Q_{\beta^+} $: di conseguenza, nell'electron capture i prodotti di decadimento hanno maggior energia cinetica disponibile, inoltre ci sono dei casi in cui può avvenire l'electron capture ma non il decadimento $ \beta^+ $.

\subsection{Radioactive decay law}

Il processo di decadimento ha natura aleatoria, dunque va trattato in maniera statistica.\\
Il numero di decadimenti al secondo è proporzionale al numero di nuclidi radioattivi:
\begin{equation}
	- \frac{dN}{dt} = \lambda N(t)
	\label{eq:2.1}
\end{equation}
Si trova quindi la legge di decadimento esponenziale:
\begin{equation}
	N(t) = N_0 e^{-\lambda t}
	\label{eq:2.2}
\end{equation}
Si definisce inoltre il decay rate (o activity) come $ A(t) \defeq \lambda N(t) $, misurato in Bequerel $ 1\,\text{Bq} = 1 \,\text{decay}/\text{s} $ o in Curie $ 1\,\text{Ci} = 3.7\cdot10^{10}\,\text{Bq} $ (activity di $ 1\,\text{g} $ di radio). Si definiscono inoltre la half-life $ t_{1/2} \equiv \frac{\ln 2}{\lambda} $ e la vita media $ \tau = \frac{1}{\lambda} $ rispettivamente come il tempo dopo il quale il campione si è ridotto di $ \frac{1}{2} $ e di $ \frac{1}{e} $: si ha $ t_{1/2} \approx 0.693 \tau < \tau $.\\
Per i decay rates si trova facilmente che, definendo $ A_0 \equiv \lambda N_0 $:
\begin{equation}
	A((n+1)t) = A(t) \left( \frac{A(t)}{A_0} \right)^n
	\label{eq:2.3}
\end{equation}
Nel caso di una miscela di radioisotopi, è possibile risalire alle singole costanti di decadimento nel caso in cui le vite medie siano molto diverse, poiché quando la specie con la $ \tau $ più corta è completamente decaduta si può misurare direttamente la $ \lambda $ dell'altra specie, per poi risalire a quella della prima tramite la differenza delle activities.

\subsubsection{Decay branches}

Può capitare che lo stesso nuclide radioattivo possa decadere in due o più modi differenti, detti decay branches: detta $ \lambda_k $ la costante di decadimento parziale della $ k $-esima branch, nel caso di $ n $ branches si ha:
\begin{equation}
	\lambda \equiv \lambda_1 + \dots + \lambda_n
	\label{eq:2.4}
\end{equation}
e questa costante totale è l'unica che si osserva, anche quando si rileva una sola delle branches. Si definiscono i branching ratios come $ B_k \defeq \frac{\lambda_k}{\lambda} $.

\subsubsection{Decay chains}

Spesso, in un decadimento radioattivo, capita che anche i prodotti siano radiaottivi: ciò dà vita ad una catena di decadimenti $ N_1 \xrightarrow{\lambda_1} N_2 \xrightarrow{\lambda_2} N_3 \dots $, dove le $ \lambda_k $ sono diverse tra loro.\\
Una decay chain è descritto da un sistema di coupled differential equations. Nel caso, ad esempio, di un doppio decadimento (quindi con prodotto $ N_3 $ stabile):
\begin{equation}
	\begin{cases}
		\dot{N}_1 = - \lambda_1 N_1 \\
		\dot{N}_2 = \lambda_1 N_1 - \lambda_2 N_2 \\
		\dot{N}_3 = \lambda_2 N_2
	\end{cases}
	\quad\Longrightarrow\quad
	\begin{cases}
		N_1(t) = N_0 e^{-\lambda_1 t} \\
		N_2(t) = \frac{\lambda_1}{\lambda_1 + \lambda_2} N_0 \left( e^{-\lambda_1 t} - e^{-\lambda_2 t} \right) \\
		N_3(t) = \frac{\lambda_1 \lambda_2}{\lambda_1 + \lambda_2} N_0 \left( \frac{1 - e^{-\lambda_1 t}}{\lambda_1} - \frac{1 - e^{-\lambda_2 t}}{\lambda_2} \right)
	\end{cases}
	\label{eq:2.5}
\end{equation}
La soluzione generale al caso di $ n $ decadimenti è dato dall'\textit{equazione di Bateman}:
\begin{equation}
	N_k(t) = \sum_{i = 1}^{k} \left[ N_0^{(i)} \left( \prod_{j = 1}^{k-1} \lambda_j \right) \left( \sum_{j = i}^{k} \frac{e^{-\lambda_j t}}{\prod_{p=i, p\neq j}^{k} (\lambda_p - \lambda_j)} \right) \right]
	\label{eq:2.6}
\end{equation}

\paragraph{Equilibrio radioattivo} Si parla di equilibrio radioattivo quando la specie radioattiva madre e quella figlia hanno la stessa attività, ovverosia quando la specie figlia decade allo stesso rate a cui è prodotta. In una decay chain, l'equilibrio radioattivo può instaurarsi tra ciascuna coppia di nuclidi della catena: la condizione generale da soddisfarre è che $ (t_{1/2})_{\text{madre}} > (t_{1/2})_{\text{figlia}} $.\\
In particolare, si parla di:
\begin{enumerate}
	\item equilibrio transiente: si ha quando $ (t_{1/2})_{\text{madre}} \approx (t_{1/2})_{\text{figlia}} $, dunque, dopo un periodo di transienza iniziale, l'activity della specie madre e quella della specie figlia diventano uguali;
	\item equilibrio secolare: si ha quando $ (t_{1/2})_{\text{madre}} \rightarrow \infty $ (comparabile all'età della Terra), dunque la sua activity è praticamente costante; di conseguenza, dopo un certo periodo di tempo, anche l'activity della specie figlia diventerà costante e pari a quella della specie madre (si vede analiticamente dall'Eq. 2.5 ponendo $ \lambda_1 \ll \lambda_2 $).
\end{enumerate}

\paragraph{Serie radioattive naturali}

Ci sono 4 serie decay chains naturali principali, tutte composte da decadimenti $ \alpha $ e $ \beta^- $:
\begin{enumerate}
  \item serie del torio: inizia col $ \ch{^{232} Th} $ e termina col $ \ch{^{208} Pb} $, il decadimento col tempo di dimezzamento più lungo è $ \ch{^{232} Th} \rightarrow \ch{^{228} Ra} $ con $ t_{1/2} = 14 \,\text{Gy} $;
  \item serie dell'uranio: inizia col $ \ch{^{238} U} $ e termina col $ \ch{^{206} Pb} $, il decadimento col tempo di dimezzamento più lungo è $ \ch{^{238} U} \rightarrow \ch{^{234} Th} $ con $ t_{1/2} = 4.5 \,\text{Gy} $;
  \item serie del plutonio: inizia col $ \ch{^{239} Pu} $ e termina col $ \ch{^{207} Pb} $, il decadimento col tempo di dimezzamento più lungo è $ \ch{^{235} U} \rightarrow \ch{^{231} Th} $ con $ t_{1/2} = 0.71 \,\text{Gy} $;
  \item serie del nettunio: inizia col $ \ch{^{237} Np} $ e termina col $ \ch{^{209} Bi} $, il decadimento col tempo di dimezzamento più lungo è $ \ch{^{237} Np} \rightarrow \ch{^{233} Pa} $ con $ t_{1/2} = 2.3 \,\text{My} $;
\end{enumerate}
Quest'ultima serie non è più osservabile in natura poiché la sua vita media non è comparabile con l'età della Terra, a differenza delle altre tre.

\paragraph{Carbon dating}

Il $ \ch{^{14}C} $ è un isotopo radioattivo del carbonio con $ t_{1/2} = 5730\,\text{y} $; nonostante questa vita media relativamente corta, esso è prodotto continuamente grazie ai raggi cosmici in atmosfera tramite neutron capture: $ \ch{^{14}N} + n \rightarrow \ch{^{14}C} + p^+ $; una volta assorbito dai sistemi biologici, esso decade per decadimento $ \beta^- $: $ \ch{^{14}C} \rightarrow \ch{^{14}N} + e^- + \bar{\nu}_e $.\\
Misurata la specific activity $ a $ del campione da datare, si può risalire al time since death comparandola alla standard specific activity $ a_0 = 0.266\,\text{Bq}/\text{g} $: $ T = \frac{t_{1/2}}{\ln 2} \ln \frac{a}{a_0} = - 8033\,\text{y} \cdot \ln \frac{a}{a_0} $.\\
L'assunzione fondamentale di questo metodo di datazione è che la concentrazione naturale di $ \ch{^{14}C} $ rimanga costante nel tempo: con la Guerra Fredda questo è diventato un problema, poiché i test nucleari hanno fatto raddoppiare per un periodo tale concentrazione, e solo ultimamente si sta tornando ai livelli precedenti. Inoltre, va notato che il carbon dating è efficacie solo per oggetti non più vecchi di 50'000 anni: per epoche precedenti, sono necessari altre coppie isotopiche, come ad esempio $ \ch{^{40}K} $ - $ \ch{^{40}Ar} $, $ \ch{^{235}U} $ - $ \ch{^{207}Pb} $ o $ \ch{^{238}U} $ - $ \ch{^{208}Pb} $.










