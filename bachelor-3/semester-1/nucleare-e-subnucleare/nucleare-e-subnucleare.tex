% !TeX TS-program = lualatex

\documentclass[a4paper, 12pt, openany]{book}
%\usepackage[utf8]{inputenc}
\usepackage[italian]{babel}

\usepackage[]{csvsimple}
\usepackage{float}

\usepackage{ragged2e}
\usepackage[left=25mm, right=25mm, top=15mm]{geometry}
\geometry{a4paper}
\usepackage{graphicx}
\usepackage{booktabs}
\usepackage{paralist}
\usepackage{subfig} 
\usepackage{fancyhdr}
\usepackage{amsmath}
\usepackage{amssymb}
\usepackage{amsfonts}
\usepackage{amsthm}
\usepackage{mathtools}
\usepackage{enumitem}
\usepackage{titlesec}
\usepackage{braket}
\usepackage{gensymb}
\usepackage{url}
\usepackage{hyperref}
\usepackage{csquotes}
\usepackage{multicol}
\usepackage{graphicx}
\usepackage{wrapfig}
\usepackage{caption}

\usepackage{esint}

\captionsetup{font=small}
\pagestyle{fancy}
\renewcommand{\headrulewidth}{0pt}
\lhead{}\chead{}\rhead{}
\lfoot{}\cfoot{\thepage}\rfoot{}
\usepackage{sectsty}
\usepackage[nottoc,notlof,notlot]{tocbibind}
\usepackage[titles,subfigure]{tocloft}
\renewcommand{\cftsecfont}{\rmfamily\mdseries\upshape}
\renewcommand{\cftsecpagefont}{\rmfamily\mdseries\upshape}

\let\oldsection\section% Store \section
\renewcommand{\section}{% Update \section
	\renewcommand{\theequation}{\thesection.\arabic{equation}}% Update equation number
	\oldsection}% Regular \section
\let\oldsubsection\subsection% Store \subsection
\renewcommand{\subsection}{% Update \subsection
	\renewcommand{\theequation}{\thesubsection.\arabic{equation}}% Update equation number
	\oldsubsection}% Regular \subsection

\newcommand{\abs}[1]{\left\lvert#1\right\rvert}
\newcommand{\norm}[1]{\left\lVert#1\right\rVert}

\newcommand{\g}{\text{g}}
\newcommand{\m}{\text{m}}
\newcommand{\cm}{\text{cm}}
\newcommand{\mm}{\text{mm}}
\newcommand{\s}{\text{s}}
\newcommand{\N}{\text{N}}
\newcommand{\Hz}{\text{Hz}}

\newcommand{\virgolette}[1]{``\text{#1}"}
\newcommand{\tildetext}{\raise.17ex\hbox{$\scriptstyle\mathtt{\sim}$}}

\renewcommand{\arraystretch}{1.2}

\addto\captionsenglish{\renewcommand{\figurename}{Fig.}}
\addto\captionsenglish{\renewcommand{\tablename}{Tab.}}

\DeclareCaptionLabelFormat{andtable}{#1~#2  \&  \tablename~\thetable}

\setlength{\parindent}{0pt}

\graphicspath{{./images/}}

% feynman diagrams
\usepackage[compat=1.1.0]{tikz-feynman}

\author{Leonardo Cerasi%
	\thanks{\scriptsize\href{mailto:leonardo.cerasi@studenti.unimi.it}{leo.cerasi@pm.me}}%
	, Lucrezia Bioni\\
	\small GitHub repository: \href{https://github.com/LeonardoCerasi/notes}{LeonardoCerasi/notes}}

\title{\Huge\textbf{Fisica Nucleare e Subnucleare} \\ \large Prof.ssa S. Leoni, a.a. 2024-25}

\begin{document}

\frontmatter

\maketitle

\tableofcontents
\pagestyle{indice}

\mainmatter

\chapter*{Introduzione}
\pagestyle{introd}
\addcontentsline{toc}{chapter}{Introduzione}
\markboth{Introduzione}{}
\selectlanguage{italian}

\paragraph{Scale di grandezza}

Nello studio della fisica dei nuclei e delle particelle subatomiche, le scale di grandezza tipiche sono estremamente piccole: la scala tipica delle dimensioni di un atomo è $ 1\ang = 10^{-10}\m  $, mentre quella del nucleo è di $ 4 $ ordini di grandezza minore ($ 10^{-14}\m = 10\fm $); per un singolo nucleone, invece, le dimensioni sono dell'ordine di $ 1\fm = 10^{-15}\m $, e il range tipico delle interazioni deboli è $ 10^{-18}\m $.\\
Per quanto riguarda la scala di energie, i processi atomici hanno energie solitamente attorno a $ 1\ev = 1.602\cdot10^{-19}\,\text{J} $, mentre quelli nucleari arrivano anche a $ 10\mev $; per le interazioni ad alte energie studiate dalla fisica particellare, i moderni accelleratori arrivano a scale di $ 1\tev $.\\
Per studiare la struttura dei costituenti della materia a vari livelli, è necessario utilizzare fasci di particelle (fotoni, elettroni, etc.) con determinate lunghezze d'onda (relazioni di de Broglie $ \lambda = \frac{h}{p} $), corrispondenti a determinate energie: per sondare i nuclei atomici sono necessari $ \lambda \sim 10\fm $ ed $ E \sim 1\mev $; per evidenziare la struttura a molti corpi del nucleo atomico servono $ \lambda \sim 1\fm $ ed $ E \sim 10\mev $; se si vogliono studiare gli stati eccitati dei singoli nucleoni occorrono $ \lambda \sim 10^{-3}\fm $ ed $ E \sim 1\gev $; infine, se si vuole mettere in luce la struttura composta da quark dei nucleoni, bisogna raggiungere $ \lambda < 10^{-4}\fm $ ed $ E > 200\gev $.

\paragraph{Interazioni fondamentali}

I vari costituenti della materia interagiscono tramite $ 4 $ interazioni fondamentali:
\begin{enumerate}
  \item interazione elettromagnetica: mediata dal fotone ($ m_{\gamma} = 0 $), con coupling constant $ \alpha \approx 1/137 $ e raggio d'azione infinito (essendo il fotone massless);
  \item interazione debole: mediata dai bosoni $ \w^{\pm} $ e $ \z^0 $ ($ m_W = 80.4\gev $, $ m_Z = 90.1\gev $), con coupling constant $ G_F \approx 1\cdot10^{-5} $) e raggio d'azione $ < 10^{-3}\fm $, dovuto al fatto che i bosoni bosoni $ \w^{\pm} $ e $ \z^0 $ sono molto pesanti e dunque, per il principio d'indeterminazione ($ \Delta E \Delta t \ge \frac{\hbar}{2} $), possono essere prodotti solo come particelle virtuali in processi di scattering per periodi di tempo estremamente brevi;
  \item interazione forte: mediata dai gluoni ($ m_g = 0 $), con coupling constant $ \alpha_s \approx 1 $ e raggio d'azione $ \approx 1\fm $, dovuto al fatto che i gluoni, sebbene massless, possono interagire tra loro;
  \item interazione gravitazionale: mediata dall'ipotetico gravitone ($ m_G = 0 $), con coupling constant $ G_N \approx 6\cdot10^{-39} $ e raggio d'azione infinito.
\end{enumerate}
Come si può notare, la gravità ha un'intensità di decine di ordini di grandezza inferiore alle altre interazioni fondamentali, per questo in ambito atomico, nucleare e particellare può essere trascurata.

\paragraph{Esperimenti}

A differenza della fisica atomica, che è descritta completamente dalla QED (Quantum Electrodynamics), la fisica nucleare non ha un'unica teoria coerente: la teoria fondamentale dell'interazione forte, la QCD (Quantum Chromodynamics), descrive le interazioni tra quark (mediate da gluoni), non quelle tra nucleoni (mediate da mesoni virtuali); inoltre, in ambito atomico le energie che entrano in gioco nei decadimenti ($ \sim 10\mev $) sono meno dello $ 0.1\% $ della massa del nucleo (espressa in unità naturali), dunque gli effetti relativistici possono essere ignorati, mentre per quanto riguarda i processi tra nucleoni le energie possono essere anche $ 100 $ volte la massa equivalente del protone, rendendo necessario l'utilizzo della meccanica quantistica relativistica; infine, bisogna considerare che sia il nucleo atomico che i nucleoni sono sistemi complessi a molti corpi, dunque, anche avendo una teoria dell'interazione tra singole coppie di costituenti, è estremamente difficile sviluppare modelli teorici per descrivere questi sistemi, e la trattazione è principalmente di natura fenomenologica, con tante teorie dei singoli processi che vengono sviluppate a partire dai dati sperimentali.\\
Gli esperimenti in fisica nucleare (utilizzati anche per studiare gli adroni in generale) sono principalmente di due tipi:
\begin{enumerate}
  \item scattering: un fascio di particelle, di cui si conoscono energia e momento lineare, è diretto verso l'oggetto bersagio da studiare e, attraverso le variazioni di quantità cinematiche misurabili, è possibile studiare le proprietà dell'interazioni e la struttura del bersaglio (risoluzione data dalla relazione di de Broglie);
  \item spettroscopia: nucleoni (o anche mesoni e barioni) vengono eccitati e si studiano i prodotti del decadimento di questi stati eccitati, inferendo le proprietà degli stati eccitati e le interazioni tra i prodotti di decadimento.
\end{enumerate}
Sia esperimenti di scattering che esperimenti spettroscopici necessitano di energie di ordini di grandezza simili.\\
Nel caso dello scattering è importante studiare la sezione d'urto d'interazione (cross section), ovvero la probabilità che avvenga una determinata reazione: in base all'angolo solido $ \Delta\Omega $ del rilevatore, alla cross section $ \frac{d\sigma}{s\Omega} $, all'intensità $ I_0 $ del fascio incidente e alla densità numerica di particelle $ n_0 $ che attraversano lo spessore $ dz $ del rilevatore, si può calcolare il numero di particelle rilevate in funzione dell'angolo d'emissione:
\begin{equation}
  \frac{dn(\theta)}{dt} = I_0 n_0 dz \frac{d\sigma}{d\Omega} \Delta\Omega
  \label{eq:1}
\end{equation}
La cross section è un'area geometrica (l'area effettiva di collisione) ed è solitamente misurata in barn: $ 1\barn = 100\fm^2 $; questa sezione d'urto è in realtà molto grande e misure più tipiche sono espresse in microbarn.

\thispagestyle{introd}


\part{Fisica Nucleare}
\pagestyle{body}

\chapter{Nuclidi}
\pagestyle{body}
\selectlanguage{italian}
\section{Definizioni e nomenclatura}

Un nuclide (o nucleo) è una specifica combinazione di protoni e neutroni: si definiscono il numero atomico $ Z $ come il numero di protoni, il numero di neutroni $ N $ ed il numero di massa $ A = Z + N $ come il numero di nucleoni. In un atomo neutro, $ Z $ è anche il numero totale di elettroni negli orbitali.\\
Il simbolo completo di un nuclide è $ ^A_Z \text{X}_N $, dove $ \text{X} $ è il simbolo della specie chimica: tale scrittura è però ridondante, poiché la specie chimica definisce di per sé il numero di protoni nel nuclide, dunque è sufficiente scrivere $ ^A \text{X} $.\\
Nuclidi con lo stesso $ Z $ sono detti isotopi, con lo stesso $ A $ isobari e con lo stesso $ N $ isotoni.

\subsection{Unità di misura}

Nell'ambito della fisica nucleare e particellare è sconveniente utilizzare le unità di misura del Sistema Internazionale: unità di misura tipiche sono il fermi $ 1\fm = 10^{-15}\m $, l'elettronvolt $ 1\ev = 1.602\cdot10^{-19}\,\text{J} $ e l'unità di massa atomica $ 1\,\text{u} = 1.6606\cdot10^{-27}\,\text{kg} = 931.502 \mev/c^2 $ (definita come $ 1/12 $ della massa di un atomo di $ \ch{^{12}C} $).\\
Per semplificare le equazioni, è utile porre le costanti fondamentali $ c = \hbar = 1 $: questo sistema di misura è detto Sistema Naturale e in esso massa, momento lineare, energia, lunghezza$ ^{-1} $ e tempo$ ^{-1} $ hanno la stessa unità di misura, poiché le equazioni di Einstein, Plank e de Broglie diventano rispettivamente $ E^2 = m_0^2 + p^2 $, $ E = 2\pi \nu $ e $ \lambda = \frac{2\pi}{p} $.

\subsubsection{Masse e costanti}

Nel SI, è utile ricordare i seguenti valori approssimati delle costanti fondamentali:
\begin{equation*}
    \begin{split}
	  &c = 2.99792458\cdot10^8 \m/\text{s} \approx 3\cdot10^8 \m/\text{s}\\
	  &\hbar = 6.58211928\cdot10^{-22} \mev\,\text{s} \approx \frac{2}{3}\cdot10^{-21} \mev\,\text{s}\\
	  &\hbar c = 197.3269718 \mev\fm \approx 200 \mev\fm
    \end{split}
\end{equation*}

Si possono quindi esprimere le masse dei nucleoni e dell'elettrone in varie unità di misura:
\begin{equation*}
	\begin{split}
		&m_p = 1.673\cdot10^{-27}\,\text{kg} = 1.00728\,\text{u} = 938.279 \mev/c^2\\
		&m_n = 1.675\cdot10^{-27}\,\text{kg} = 1.00867\,\text{u} = 939.573 \mev/c^2\\
		&m_e = 9.110\cdot10^{-31}\,\text{kg} = 0.511\mev/c^2
	\end{split}
\end{equation*}

\subsection{La tavola di Segré}

Al pari delle specie chimiche nella tavola periodica, anche i nuclidi possono essere messi in una tabella, tipicamente in un piano $ Z - N $ (Fig. \ref{segre-chart}): questa viene detta tavola di Segré e permette di tracciare facilmente i vari decadimenti radioattivi dei nuclidi, visualizzando efficaciemente le decay chains.\\
Come si vede in Fig. \ref{drip-lines}, è possibile distinguere la tavola dei nuclidi in due regioni separate da due linee: queste sono dette nuclear driplines e distinguono tra configurazioni di protoni e neutroni che possono effettivamente formare dei nuclidi (sia stabili che instabili, ovvero radioattivi) e configurazioni nelle quali invece l'interazione forte non riesce a mantenere insieme i nucleoni per formare un nucleo. Si stima che possano esistere oltre $ 7000 $ nuclidi nell'Universo, ma di questi solo circa $ 3000 $ sono stati effettivamente scoperti (di cui solo $ 251 $ nuclidi stabili): si parla in questo caso di $ \virgolette{Terra incognita} $ per indicare il teoricamente alto numero di nuclidi ancora ignoti; in particolare, è stata teoricamente prevista un'$ \virgolette{isola} $ di elementi super-pesanti attorno a $ Z = 114 $ ed oltre, con nuclidi con vite medie dell'ordine di minuti o giorni: sebbene non ancora osservati, si pensa che la chimica degli elementi super-pesanti con $ Z > 118 $ sia di natura relativistica, dunque incomparabile a quella degli elementi fin'ora scoperti.

\begin{figure}
  \centering
  \includegraphics[width = 0.75\textwidth]{segre-chart.png}
  \caption{Tavola di Segré.}
  \label{segre-chart}
\end{figure}
\begin{figure}
  \centering
  \includegraphics[width = 0.55\textwidth]{drip-lines.png}
  \caption{Nuclear driplines.}
  \label{drip-lines}
\end{figure}

\section{Evidenze sperimentali}

Le prime evidenze sperimentali dell'esistenza del nucleo atomico si devono al gruppo di ricerca di Rutherford, Geiger e Marsden: prima di loro, Thomson era riuscito ad estrarre delle cariche negative dall'atomo, identificando l'elettrone, ed aveva di conseguenza formulato la sua teoria della struttura atomica come sfera carica positivamente in cui sono immersi gli elettroni ($ \virgolette{plum pudding} $ model).\\
Con i loro esperimenti, Rutherford et al. dimostrarono invece che le cariche positive erano concentrate in una regione piccola al centro dell'atomo.

\subsection{Scattering di Rutherford}

L'esperimento condotto da Rutherford et al. consiste nell'irradiare una lamina sottile di oro con un fascio di particelle $ \alpha $ (nuclei di $ \ch{^{4} He} $): a livello puramente cinematico (ignorando la natura dell'interazione tra beam e target), essendo la velocità delle particelle $ \alpha $ $ v_0 \sim 0.1c $, è possibile trattare il problema come un urto elastico non-relativistico:
\begin{equation}
	\begin{cases}
	  m_{\alpha} \ve{v}_0 = m_{\alpha} \ve{v}_f + m_t \ve{v}_t \\
	  m_{\alpha} v_0^2 = m_{\alpha} v_f^2 + m_t v_t^2 \\
	\end{cases}
	\label{eq:2}
\end{equation}
Combinando le due equazioni e definendo $ \theta $ l'angolo tra $ \ve{v}_f $ e $ \ve{v}_t $:
\begin{equation}
	\cos \theta = \frac{1}{2} \frac{v_t}{v_f} \left(1 - \frac{m_t}{m_{\alpha}}\right)
	\label{eq:3}
\end{equation}
Si possono distinguere due principali casi:
\begin{enumerate}
	\item $ m_t \ll m_{\alpha} $: $ \cos \theta > 0 $, dunque si parla di forward scattering, poiché non sono possibili grossi valori di $ \theta $;
	\item $ m_t \gg m_{\alpha} $: $ \cos \theta < 0 $, dunque diventano possibili anche angoli prossimi $ \pi $.
\end{enumerate}
Il modello di Thomson rientra nella prima casistica, poiché in tal caso all'interno dell'atomo lo scattering può avvenire solo con gli elettroni, che hanno $ m_e \ll m_{\alpha} \approx 4m_p $.\\
Ciò che Rutherford et al. osservarono, però, è che occasionalmente delle particelle $ \alpha $ vengono riflesse dalla lamina d'oro: questo risultato è incompatibile con lo scattering con elettroni o con una carica positiva diffusa, dunque fu confermato che la carica positiva nell'atomo è concentrata in un unico punto massivo, il nucleo atomico.

\subsubsection{Cross-section di Rutherford}

Nella trattazione cinematica è stata ignorata l'interazione tra particelle $ \alpha $ e nucleo atomico, che è ciò che effettivamente determina lo scattering: essa può essere modellata, in forma approssimativa (in particolare per parametro d'urto compreso tra il raggio nucleare e l'orbita elettronica più interna), dal potenziale coulombiano; dette $ Z $ il numero atomico dell'atomo target e $ Z' $ quello degli atomi del beam (nel caso specifico dello scattering di Rutherford $ Z = Z_{\ch{Au}} = 79 $ e $ Z' = Z_{\ch{He}} = 4 $), il potenziale d'interazione è:
\begin{equation}
	V(\ve{r}) = \frac{ZZ' e^2}{r}
	\label{eq:4}
\end{equation}
Dalla meccanica classica è possibile legare il parametro d'urto $ b $ all'angolo di scattering $ \theta $:
\begin{equation}
	b = \frac{ZZ' e^2}{2 E_0} \cot \frac{\theta}{2}
	\label{eq:5}
\end{equation}
dove $ E_0 $ è l'energia della particella incidente.\\
È possibile stimare quanto vicino al nucleo atomico si possono spingere le particelle $ \alpha $ tramite la distanza di closest approach $ a $, definia dalla condizione $ V(a) = E_0 $ ed esprimibile anche in funzione di $ b $ e $ \theta $ tramite $ \tan \frac{\theta}{2} = \frac{a}{2b} $: le particelle $ \alpha $ usate da Rutherford avevano $ E_{\alpha} \approx 5\mev $, dunque fu in grado di sondare il nucleo atomico poiché $ a \approx 45\fm $.\\
Per calcolare la cross-section dello scattering di Rutherford, si consideri un fascio incidente monoenergetico con energia $ E_0 $ e $ N_0 $ particelle incidenti per unità di area e di tempo: facendo variare il parametro d'urto tra $ b $ e $ b + db $, dunque variando l'angolo di scattering tra $ \theta $ e $ \theta - d \theta $, si avranno $ 2\pi N_0 b \,db $ particelle incidenti per unità di tempo (data la sezione d'urto $ \Delta\sigma = 2\pi b \,db $, Fig. \ref{rutherford}).
\begin{figure}
	\centering
	\includegraphics[width=0.95\textwidth]{rutherford.png}
	\caption{Sezione d'urto dello scattering di Rutherford.}
	\label{rutherford}
\end{figure}

Va notato che è possibile ignorare il fatto che la lamina target ha un numero elevato di atomi target e che ogni partiella nel beam incidente ha un diverso parametro d'urto relativo a ciascuno di essi, poiché la lastra è considerata così sottile da rendere improbabili collisioni multiple della stessa particella incidente; inoltre, dal modello atomico di Rutherford ricaviamo che i nuclei atomici si trovano a distanze grandi rispetto alle loro dimensioni, rendendo significative solo le traiettorie con parametro d'urto vicino al nucleo atomico.\\
Nel caso di un potenziale d'interazione generico, $ \Delta\sigma $ può avere anche dipendenza azimuthale:
\begin{equation}
	\Delta\sigma(\theta,\phi) = b \,db\,d\phi = - \frac{d\sigma}{d\Omega} (\theta,\phi) \,d\Omega= - \frac{d\sigma}{d\Omega} (\theta,\phi) \sin \theta \,d\theta\,d\phi
	\label{eq:6}
\end{equation}
dov'è stata utilizzata la differential cross-section $ \frac{d\sigma}{d\Omega} $ e dove si è tenuto conto che un aumento di $ b $ porta ad una diminuzione di $ \theta $ tramite il segno negativo.\\
Essendo il potenziale coulombiano un potenziale centrale a simmetria sferica, è possibile semplificare il calcolo grazie alla simmetria azimuthale, ottenendo:
\begin{equation}
	\frac{d\sigma}{d\Omega} (\theta) = - \frac{b}{\sin \theta} \frac{db}{d\theta}
	\label{eq:7}
\end{equation}
Lo scattering di Rutherford può essere quindi completamente caratterizzato utilizzando l'Eq. \ref{eq:5}:
\begin{equation}
	\frac{d\sigma}{d\Omega} (\theta) = \left( \frac{ZZ' e^2}{4 E_0} \right)^2 \frac{1}{\sin^2 \frac{\theta}{2}}
	\label{eq:8}
\end{equation}
È anche possibile definire la sezione d'urto totale $ \sigma_{\text{tot}} $ come:
\begin{equation}
	\sigma_{\text{tot}} = \int_{\Omega} \frac{d\sigma}{d\Omega} (\theta,\phi) \,d\Omega
	\label{eq:9}
\end{equation}
Essa rappresenta una sorta di area di scattering effettiva che la sorgente del potenziale determina a tutti i possibili valori del parametro d'urto.\\
Nel caso dello scattering di Rutherford:
\begin{equation}
	\begin{split}
		\sigma_{\text{tot}} &= \int_0^{2\pi} \int_0^{\pi} \frac{d\sigma}{d\Omega} (\theta,\phi) \sin \theta \,d\theta\,d\phi = 2\pi \int_0^{\pi} \frac{d\sigma}{d\Omega} (\theta) \sin \theta \,d\theta \\
				    &= 8\pi \left( \frac{ZZ' e^2}{4 E_0} \right)^2 \int_0^1 \frac{1}{\sin^3 \frac{\theta}{2}} d\left( \sin \frac{\theta}{2} \right) \longrightarrow \infty
	\end{split}
	\label{eq:10}
\end{equation}
Questo risultato divergente è coerente con l'interpretazione data della sezione d'urto totale: il potenziale coulombiano è associato all'interazione elettromagnetica, la quale ha un range infinito, dunque anche l'area efficace di scattering sarà infita.\\
In maniera realistica, però, si può considerare che dopo un determinato valore di cutoff $ b_0 $ lo scattering non abbia più effetti osservabili sulla particella incidente, dunque la sezione d'urto totale osservabile si ottiene integrando la differential cross-section tra $ 0 $ e $ \theta_0 < \pi$, ottenendo dunque un valore finito.\\
La formula di Rutherford \ref{eq:8} cessa di essere valida quando $ E_0 $ diventa troppo alta, in particolare quando la particella $ \alpha $ riesce a penetrare nel nucleo, poiché a quel punto subentrano l'interazione nucleare fin'ora ignorata: studiando sperimentalmente a quale energia (a differenti angoli di scattering) si iniziano a manifestare le deviazioni dalla cross-section di Rutherford, è possibile stimare il raggio nucleare tramite la distanza di closest approach $ a $, trovando il fit sperimentale $ R = R_0 A^{1/3} $ con $ R_0 = 1.4\fm $.

\subsection{Scattering elettronico}

Data la dualità onda-particella che risulta da una descrizione quanto-meccanica della materia, la sezione d'urto da scattering non sarà determinata soltanto dall'interazione coulombiana ma anche da effetti diffrattivi, evidenziati dal pattern di diffrazione in Fig. \ref{diffraction}: l'analogo ottico è la diffrazione da disco opaco, poiché il nucleo atomico assorbe nucleoni, con la dovuta differenza che la superficie del nucleo ha una determinata diffusività, la quale determina dei minimi non-nulli nello spettro di diffrazione.
\begin{figure}
	\centering
	\includegraphics[width=0.95\textwidth]{diffraction.png}
	\caption{Sezione d'urto dello scattering elettronico.}
	\label{diffraction}
\end{figure}\\
Per evitare che all'interazione coulombiana si sovrapponga anche quella tra nucleoni, per studiare la struttura del nucleo atomico le sonde migliori sono gli elettroni: poiché per sondare una scala di lunghezze $ \Delta x $ è necessaria una lunghezza d'onda $ \lambda \sim \Delta x $, dalla relazione di de Broglie si ha che la quantità di moto degli elettroni incidenti deve essere $ p \sim h / \Delta x $. La struttura del nucleo atomico risulta visibile su scale di $ 1\fm $, dunque sono necessari elettroni con momento lineare $ p \approx 200 \mev/c $; nel caso si voglia studiare la struttura interna dei nucleoni, la stima aumenta di almeno un ordine di grandezza.\\
Va ricordato che gli elettroni, in quanto particelle cariche, quando percorrono una traiettoria curva irraggiano, dunque per questo tipo di scattering sono necessari accelleratori lineari.
Ricordando che $ E^2 = p^2 c^2 + m_0^2 c^4 $ ed $ E = K + m_0 c^2 $, considerando che $ m_e = 0.511 \mev/c^2 $, si vede che gli elettroni utilizzati per sondare il nucleo atomico sono in regime ultra-relativistico, dunque nel calcolo della cross-section sono da tenere in conto effetti relativistici legati allo spin ed il nuclear recoil (ovverosia il rinculo subito dal nucleo atomico a seguito dello scattering): trascurando quest'ultimo, si può applicare una correzione alla cross-section di Rutherford, detta cross-section di Mott:
\begin{equation}
	\left(\frac{d\sigma}{d\Omega}\right)_{\text{Mott}} = \left(\frac{d\sigma}{d\Omega}\right)_{\text{Ruth}} \left(1 - \beta^2 \sin^2 \frac{\theta}{2}\right)
	\label{eq:11}
\end{equation}
Queste sezioni d'urto, però, considerano scattering tra oggetti puntiformi, ma mentre un elettrone può effettivamente essere considerato tale, lo stesso non si può dire per il nucleo atomico, il quale avrà una certa distribuzione di carica $ \rho(\ve{r}) $ estesa nello spazio (essa coincide con la distribuzione di massa solo per nuclidi stabili); per considerare anche questo fatto, nel calcolo della cross-section si applica un'ulteriore correzione:
\begin{equation}
	\frac{d\sigma}{d\Omega} = \left(\frac{d\sigma}{d\Omega}\right)_{\text{Mott}} \abs{F(q)}^2
	\label{eq:12}
\end{equation}
dove $ \abs{F(q)}^2 $ è detto form factor e $ \ve{q} \defeq \frac{1}{\hbar} \abs{\ve{p}' - \ve{p}} $, con $ \ve{p} $ e $ \ve{p}' $ momento iniziale e finale dell'elettrone, esprime la variazione di quantità di moto dell'elettrone.\\
Dato che si sta considerando uno scattering elastico, si ha $ p = p' $ e dunque $ q = \frac{1}{\hbar} 2p \sin \frac{\theta}{2} $, con $ \theta $ angolo di scattering.\\
Nel caso dell'approssimazione di Born (elettroni come onde piane $ \psi(\ve{r}) = \frac{1}{\sqrt{V}} e^{i \ve{p}\cdot\ve{x} / \hbar} $) e trascurando il nuclear recoil, il form factor è la trasformata di Fourier della distribuzione di carica $ \rho(\ve{r}) $:
\begin{equation}
	F(q) = \int_{r \le R} e^{i \ve{p}\cdot\ve{r} / \hbar} \rho(\ve{r}) d^3\ve{r}
	\label{eq:13}
\end{equation}
con $ R $ raggio nucleare ed opportuna normalizzazione $ F(0) = 1 $.\\
In questa approssimazione, quindi, una misura della cross-section di scattering elettronico può dare informazioni sulla distribuzione di carica nel nucleo atomico; inoltre, è possibile dare una stima del raggio nucleare dallo studio dello spettro di diffrazione evidenziato dalla cross-section, poiché il primo minimo di diffrazione soddisfa la relazione $ \frac{q}{\hbar} \approx \frac{4.5}{R} $; se invece si considerano due isotopi, dal fatto che la separazione angolare tra due minimi è $ \Delta\theta = \frac{\hbar}{p R} $ si evince che all'aumentare del numero di massa aumenta anche il raggio nucleare (vedere Fig. \ref{diffraction}): ciò in generale è valido solo per nuclidi nella cosiddetta $ \virgolette{valle di stabilità} $, poiché per essi la distribuzione di neutroni segue quella di protoni, ovvero le distribuzioni ci carica e massa vanno a coincidere, dunque un nuclide con più nucleoni avrà un raggio nucleare maggiore; la difficoltà principale nella misura della distribuzione di nucleoni sta nel fatto che i neutroni sono trasparenti agli esperimenti di scattering, poiché non interagiscono né tramite interazione elettromagnetica né tramite interazione nucleare forte.

\subsubsection{Distribuzione di carica nucleare}

Compiendo esperimenti di scattering elettronico e raccogliendo dati relativi a vari nuclei atomici, si è giunti alla conclusione che i nuclidi non sono sfere con un confine ben delineato, ma al loro interno la densità di carica si mantiene approssimativamente costante, mentre verso la superficie essa si riduce su un intervallo radiale relativamente ampio; la distribuzione di carica nucleare può dunque essere approssimata da una distribuzione di Fermi:
\begin{equation}
	\rho(\ve{r}) = \frac{\rho_0}{1 + e^{(r - c) / a}}
	\label{eq:14}
\end{equation}
dove i parametri empirici valgono (per nuclei pesanti) $ c = 1.07\fm \cdot A^{1/3} $ e $ a = 0.54\fm $: $ c $ è il raggio nucleare a mezza altezza della distribuzione di carica, mentre $ a $ è la diffusività (ciò che rende la distribuzione smooth piuttosto che sharp).\\
Una volta nota la distribuzione di carica, è possibile calcolare il raggio quadratico medio: per nuclei medi e pesanti, si ha $ \sqrt{\langle r^2 \rangle} = 0.94\fm \cdot A^{1/3} $. Se si approssima il nucleo come una sfera uniformemente carica, il suo raggio, definito come raggio nucleare, è dato da $ R^2 = \frac{5}{3} \langle r^2 \rangle $, ovvero:
\begin{equation}
	R = 1.21\fm \cdot A^{1/3}
	\label{eq:14-bis}
\end{equation}
che è la definizione più diffusa di raggio nucleare.\\
È anche possibile definire una skin depth (o surface thickness) $ t $ come lo spessore del guscio sferico in cui la densità di carica diminuisce dal $ 90\% $ al $ 10\% $ del suo valore massimo: per nuclei pesanti, si trova $ t = 2a\ln9 \approx 4.4 a $.\\
Come si può vedere in Fig. \ref{charge-distr}, la densità di carica centrale $ \rho_0 $ diminuisce leggermente all'aumentare del numero di massa; se però viene considerata la presenza di neutroni e si moltiplica per un fattore $ A / Z $, si trova un valore quasi identico per tutti i nuclidi (questo è coerente con $ R \sim A^{1/3} $, poiché così $ \rho_0 \sim A / \text{Vol} $ rimane costante): questo corrisponde alla densità che teoricamente avrebbe della materia nucleare infinitamente estesa, pari a $ \rho_n \approx 0.17 \,\text{nucleoni} / \text{fm}^3 $, che corrisponde a $ c = 1.12\fm \cdot A^{1/3} $.
\begin{figure}[!ht]
	\centering
	\includegraphics[width=0.95\textwidth]{charge-distribution.png}
	\caption{Distribuzione di carica in vari nuclidi.}
	\label{charge-distr}
\end{figure}

Sperimentalmente si trova che alcuni nuclidi (ad esempio i lantanidi) non hanno una forma sferica ma assumono deformazioni ellissoidali: queste forme non possono essere studiate con scattering elettronico, poiché esso evidenzia soltanto una superficie molto diffusa.\\
Va infine notato che nuclidi leggeri come $ \ch{^{6,7}Li} $, $ \ch{^{9}Be} $ e soprattutto $ \ch{^{4}He} $ costituiscono dei casi speciali: essi non presentano una densità di carica centrale costante, ma il suo andamento è approssimativamente gaussiano.

\subsubsection{Nuclidi instabili}

Quanto riportato fin'ora vale solo per nuclidi nella valle di stabilità (vedere Fig. \ref{segre-chart}), ovverosia l'insieme di nuclei stabili che non decadono radioattivamente (tipicamente per decadimento $ \beta $), i quali sono osservati in abbondanza sulla Terra.\\
Se, partendo dalla valle di stabilità, si percorre una catena isotopica, i nuclidi diventano via via più neutron-rich, fino a quando non si raggiungono le driplines, oltre le quali i nuclei non sono più sistemi legati a causa dell'interazione nucleare forte.\\
Lo spostamento verso nuclidi esotici neutron-rich causa un mutamento drastico rispetto ai loro isotopi stabili: in un lavoro pionieristico del 1985 di Tanihata et al., fu misurata la cross-section d'interazione tra nuclei di $ \ch{^{11}Li} $ (isotopo instabile) e dei nuclidi stabili target, la quale può essere teoricamente calcolata dal modello di Glauber per lo scattering relativistico di nuclidi (traiettorie iniziali e finali rettilinee) come $ \sigma_I \sim \pi \left( R_p^2 + R_t^2 \right) $, dove $ R_p $ è il raggio del nucleo proiettile ed $ R_t $ quello del nucleo target; dai dati fu possibile ricavare il raggio nucleare del $ \ch{^{11}Li} $, il quale dovrebbe essere dell'ordine di $ \sim 2.4\fm $, trovando un valore cinque volte maggiore (comparabile con quello del $ \ch{^{208}Pb} $): questa fu la prima verifica sperimentale dell'esistenza di sistemi ad alone, nuclidi con un corpo centrale sferico circondati da uno o due neutroni o protoni (si parla di neutron skin o proton skin), i quali vanno a formare un halo attorno al nucleo aumentandone considerevolmente le dimensioni osservate.\\
Condurre esperimenti di scattering elettronico su nuclei instabili è estremamente difficile, poiché pochi di essi hanno delle vite medie abbastanza lunghe, dunque non è possibile predisporre un target composto del materiale da studiare: in alcuni laboratori si è riusciti a costruire delle trappole con le quali i nuclidi instabili sono accellerati e confinati, così da poter essere bombardati da fasci elettronici.

\paragraph{Luminosità}

Nello studio dei rilevatori di scattering un importante parametro è la luminosità:
\begin{equation}
	\mathcal{L} = N_b N_t
	\label{eq:15}
\end{equation}
dove $ N_b $ è il numero di particelle incidenti per unità di tempo e $ N_t $ il numero di nuclidi target per unità d'area.\\
Questo parametro è legato all'efficienza del rilevatore, misurata dal numero di eventi rilevati nell'unità di tempo $ \dot{N} $:
\begin{equation}
	\dot{N} (E, \theta) = \mathcal{L}\, \frac{d\sigma}{d\Omega} (E,\theta) \,\Delta\Omega
	\label{eq:16}
\end{equation}
Nel caso dello scattering elettronico, la cross-section è molto piccola, dunque sono necessari collisori dall'elevata luminosità (tecnicamente difficile): si va da $ \mathcal{L}_{\text{min}} \sim 10^{26} \,\text{cm}^{-2} \,\text{s}^{-1} $ per sondare nuclidi con $ Z \sim 80 $ a $ \mathcal{L}_{\text{min}} \sim 10^{31} \,\text{cm}^{-2}\,\text{s}^{-1} $ per $ Z \sim 10 $.\\
Inoltre, la luminosità varia anche in base al target considerato: per nuclidi stabili si raggiungono luminosità per scattering elettronico di $ 10^{33} \,\text{cm}^{-2}\,\text{s}^{-1} $ (es: STABLE), mentre su nuclidi instabili si arriva a $ 10^{27} \,\text{cm}^{-2}\,\text{s}^{-1} $ (es: SCRIT).

\section{Proprietà dei nuclidi}

\subsection{Masse nucleari e binding energy}

Preso un atomo di una specie chimica $ ^A_Z \text{X}_N $, se si sommano le masse degli $ N $ neutroni, dei $ Z $ protoni e dei $ Z $ elettroni, si trova una massa minore di quella misurata per l'atomo: questo avviene poiché, essendo l'atomo uno stato legato, parte della massa dei suoi costituenti viene convertita in energia di legame (positiva, poiché stato legato), ovvero:
\begin{equation}
	M_{\text{atom}} = N m_n + Z \left( m_p + m_e \right) - \frac{1}{c^2} \left( B_{\text{atom}} + B_{\text{nucleus}} \right)
	\label{eq:1.17}
\end{equation}
dove si è distinto tra binding energy atomica e nucleare: queste ultime sono conosciute con incertezze maggiori, dato che $ (\delta m / m)_{\text{atom}} \sim 10^{-10} $ e $ (\delta m / m)_{\text{nucleus}} \sim 10^{-7} $.\\
Si trova quindi che la binding energy di un nuclide può essere espressa come:
\begin{equation}
	B(A,Z) = \left[ Z m({\ch{^{1}H}}) + N m_n - M(A,Z) \right] c^2
	\label{eq:1.18}
\end{equation}
dove si sono usate le masse atomiche, poiché misurate con più precisione (si ricordi la conversione $ 1u \approx 931.494\mev/c^2 $).\\
La misura delle masse atomiche è importante poiché permette di fare misure indirette sulle masse nucleari, e dunque studiare l'abbondanza isotopica di un elemento, e di stimare le binding energies, le quali sono legate alla natura della forza nucleare.\\
Ci sono varie tecniche per misurare le masse atomiche: spettrometri di massa, cinematica delle reazioni (solitamente quando i primi non si possono utilizzare), trappole (ad oggi gli strumenti più precisi) ed anelli di accumulazione (storage rings).

\subsubsection{Spettrometri di massa}

Uno spettrometro di massa permette di misurare le masse di ioni di un determinato elemento.
\begin{figure}[!b]
	\centering
	\includegraphics[width=0.75\textwidth]{spectrometer.png}
	\caption{Schematizzazione di uno spettrometro di massa.}
	\label{spettr}
\end{figure}

Schematicamente (Fig. \ref{spettr}), esso è costituito da:
\begin{itemize}
	\item una sorgente di ioni;
	\item un selettore di velocità;
	\item uno spettrometro magnetico;
	\item un focal plane detector.
\end{itemize}
La sorgente di ioni ionizza gli atomi da studiare, li accellera e li collima in un fascio; questo fascio viene diretto in un selettore di velocità: questo è solitamente un filtro di Wien, composto da un campo elettrico e un campo magnetico incrociati che determinano traiettorie rettilinee solo se $ \ve{F}_E + \ve{F}_B = \ve{0} $, ovvero, assumendo un fascio già collimato nella direzione desiderata, se $ v = E / B $.\\
L'elemento ottico di questo setup è lo spettrometro magnetico: in esso le traiettorie degli ioni curvano in base al loro momento e alla loro carica, dato che il raggio di girazione è $ r = \frac{vm}{qB} $: di conseguenza, sperimentalmente vanno misurati sia il raggio di girazione che la carica per poter determinare la massa dello ione, e ciò è proprio quello che fa il focal plane detector.\\
Va notato che, in generale, le misure assolute di massa non permettono una grande precisione, perciò si preferisce effettuare misurazioni rispetto a ioni di cui già di conosce bene la massa: in particolare, le precisioni maggiori si raggiungono sui mass ratios tra ioni di egual carica, poiché in tal caso $ m_1 / m_2 = r_1 / r_2 $. Il campione di riferimento solitamente utilizzato è il carbonio, poiché dati i suoi numerosi composti permette di effettuare misure su un vasto range di ioni.

\paragraph{Abbondanze isotopiche}

Tendenzialmente, un elemento avrà più di un isotopo stabile (if any at all), dunque con una spettrografia di massa risulterà una distribuzione di massa con picchi di abbondanza relativa corrispondenti agli isotopi stabili: ciò permette di misurarne le masse $ m_i $ e l'abbondanza percentuale nel campione $ w_i $, così da poter stimare la massa atomica dell'elemento come media ponderata $ m = \sum_{i} w_i m_i $.\\
In generale, le abbondanze isotopiche sono diverse nell'Universo rispetto alla Terra, ed anche sulla Terra possono avere forti fluttuazioni tra una zona geografica e l'altra; ad esempio, si considerino due campioni di $ \ch{Xe} $, uno prelevato da una roccia metamorfica vecchia di $ 2.7 \text{Gy} $ ed uno dall'atmosfera: le relative analisi spettrografiche (riportate in Fig. \ref{iso-distr}) mostrano delle diverse abbondanze isotopiche nei due campioni, dovute alla presenza nella roccia metamorfica di prodotti della fissione nucleare spontanea dell'uranio.
\begin{figure}[!b]
	\centering
	\includegraphics[width=0.55\textwidth]{isotop-distr.png}
	\caption{Distribuzioni isotopiche di due campioni di $ \ch{Xe} $, il primo prelevato da una roccia metamorfica, il secondo dall'atmosfera.}
	\label{iso-distr}
\end{figure}

\begin{figure}[!ht]
	\centering
	\includegraphics[width=0.75\textwidth]{isotop-distr-univ.png}
	\caption{Distribuzioni isotopiche dei principali elementi nel Sistema Solare, normalizzati all'abbondanza del $ \ch{Si} $: l'abbondanza del $ \ch{Li} $ è uno dei problemi nella ricostruzione dell'origine dell'Universo.}
	\label{iso-distr-univ}
\end{figure}
Lo studio delle abbondanze relative nell'Universo (Fig. \ref{iso-distr-univ}) permette di studiarne l'evoluzione: secondo gli attuali modelli, la sintesi di deuterio ed elio è avvenuta all'origine dell'Universo dalla fusione nucleare dell'idrogeno, mentre i nuclidi fino al $ \ch{^{56}Fe} $ vengono sintetizzati dalla fusione nucleare nei nuclei stellari; per quanto riguarda i nuclei pesanti, invece, la loro sintesi avviene nelle esplosioni di stelle molto massive.

\subsubsection{Cinematica delle reazioni}

Qualora il tempo di volo tra il generatore di ioni ed il piano focale fosse maggiore della vita media dello ione che si vuole studiare, al posto della spettroscopia di massa è possibile calcolare la massa dello ione studiandone le sue reazioni.\\
Si consideri una reazione del tipo:
\begin{equation}
	a + \text{X} \longrightarrow b
	\label{eq:1.19}
\end{equation}
dove $ a $ è la particella proiettile, $ \text{X} $ lo ione target e $ b $ rappresenta l'insieme di prodotti della reazione.\\
Assumendo uno scattering elastico, la conservazione dell'energia ci dà:
\begin{equation}
	m_a c^2 + T_a + m_{\text{X}} c^2 + T_{\text{X}} = m_b c^2 + T_b
	\label{eq:1.20}
\end{equation}
È possibile definire il cosiddetto Q-value della reazione:
\begin{equation}
	Q \defeq T_{\text{final}} - T_{\text{initial}} = T_b - T_a - T_{\text{X}}
	\label{eq:1.21}
\end{equation}
così da poter calcolare la massa dell'isotopo $ \text{X} $ come:
\begin{equation}
	m_{\text{X}} = \frac{1}{c^2} Q + m_b - m_a
	\label{eq:1.22}
\end{equation}
Con questo metodo, è possibile raggiungere una precisione $ \delta m / m \sim 10^{-6} $.












\chapter{Decadimenti}
\pagestyle{body}
\selectlanguage{italian}

\section{Radioattività}

I primi indizi verso la radioattività sono derivati dalla rilevazione dei raggi X: questi sono onde elettromagnetiche, ovvero fotoni, generate da transizioni di elettroni tra livelli energetici ($ \Delta E_e = \hbar \omega $); questi dunque non sono processi nucleari, dato che avvengono all'interno della nube elettronica dell'atomo.\\
Quando si parla di radioattività in senso stretto, però, si fa riferimento ai decadimenti dei nuclidi.

\subsection{Decadimenti radioattivi}

I decadimenti radioattivi sono processi in cui un nuclide stabile raggiunge una configurazione con energia più bassa emettendo spontaneamente radiazione.\\
Utilizzando un campo magnetico, Rutherford riuscì a distinguere tre tipologie di radiazione, e dunque di decadimenti radioattivi:
\begin{enumerate}
  \item decadimento $ \alpha $, in cui la radiazione è costituita da un nucleo di $ \ch{^4_2 He} $ ed è poco penetrante, coinvolge l'interazione elettromagnetica e quella forte;
  \item decadimento $ \beta^{\pm} $, in cui la radiazione è costituita da un $ e^{\pm} $ ed è mediamente penetrante, coinvolge l'interazione debole:
    \begin{itemize}
      \item decadimento $ \beta^- $: $ n \rightarrow p^+ + e^- + \bar{\nu}_e $;
      \item decadimento $ \beta^+ $: $ p^+ \rightarrow n + e^+ + \nu_e $;
    \end{itemize}
  \item decadimento $ \gamma $, in cui la radiazione è costituita da un fotone con energia dell'ordine delle decine di MeV, dunque estremamente penetrante.
\end{enumerate}
A questi si aggiunge la cattura elettronica, un decadimento in cui un nuclide proton-rich cattura un elettrone dalle shell interne dell'atomo, seguendo la reazione $ p^+ + e^- \rightarrow n + \nu_e $, emettendo raggi X a seguito del rimpiazzo dell'elettrone interno con uno dalle shell esterne.

\subsection{Energy balance}

Un decadimento radioattivo può essere visto come un caso particolare di reazione nucleare; è quindi possibile definire il $ Q $-value del decadimento come la differenza di energia a riposo (massa) tra reagenti (nuclide instabile) e prodotti, così da poter stabilire qualora esso sia possibile, ovvero spontaneo, con la condizione $ Q > 0 $.\\
In particolare, si definiscono i $ Q $-values dei seguenti decadimenti:
\begin{enumerate}
  \item decadimento $ \alpha $: $ Q_{\alpha} \equiv \left[ M(Z,A) - M(Z-2,A-4) - m(\ch{^4_2 He}) \right] c^2 $;
  \item decadimento $ \beta^- $: $ Q_{\beta^-} \equiv \left[ M(Z,A) - M(Z+1,A) - m_e \right] c^2 $;
  \item decadimento $ \beta^+ $: $ Q_{\beta^+} \equiv \left[ M(Z,A) - M(Z-1,A) - m_e \right] c^2 $;
  \item electron capture: $ Q_{e} \equiv \left[ M(Z,A) + m_e - M(Z-1,A) \right] c^2 $;
\end{enumerate}
Si vede subito che $ Q_e > Q_{\beta^+} $: di conseguenza, nell'electron capture i prodotti di decadimento hanno maggior energia cinetica disponibile, inoltre ci sono dei casi in cui può avvenire l'electron capture ma non il decadimento $ \beta^+ $.

\subsection{Radioactive decay law}

Il processo di decadimento ha natura aleatoria, dunque va trattato in maniera statistica.\\
Il numero di decadimenti al secondo è proporzionale al numero di nuclidi radioattivi:
\begin{equation}
	- \frac{dN}{dt} = \lambda N(t)
	\label{eq:2.1}
\end{equation}
Si trova quindi la legge di decadimento esponenziale:
\begin{equation}
	N(t) = N_0 e^{-\lambda t}
	\label{eq:2.2}
\end{equation}
Si definisce inoltre il decay rate (o activity) come $ A(t) \defeq \lambda N(t) $, misurato in Bequerel $ 1\,\text{Bq} = 1 \,\text{decay}/\text{s} $ o in Curie $ 1\,\text{Ci} = 3.7\cdot10^{10}\,\text{Bq} $ (activity di $ 1\,\text{g} $ di radio). Si definiscono inoltre la half-life $ t_{1/2} \equiv \frac{\ln 2}{\lambda} $ e la vita media $ \tau = \frac{1}{\lambda} $ rispettivamente come il tempo dopo il quale il campione si è ridotto di $ \frac{1}{2} $ e di $ \frac{1}{e} $: si ha $ t_{1/2} \approx 0.693 \tau < \tau $.\\
Per i decay rates si trova facilmente che, definendo $ A_0 \equiv \lambda N_0 $:
\begin{equation}
	A((n+1)t) = A(t) \left( \frac{A(t)}{A_0} \right)^n
	\label{eq:2.3}
\end{equation}
Nel caso di una miscela di radioisotopi, è possibile risalire alle singole costanti di decadimento nel caso in cui le vite medie siano molto diverse, poiché quando la specie con la $ \tau $ più corta è completamente decaduta si può misurare direttamente la $ \lambda $ dell'altra specie, per poi risalire a quella della prima tramite la differenza delle activities.

\subsubsection{Decay branches}

Può capitare che lo stesso nuclide radioattivo possa decadere in due o più modi differenti, detti decay branches: detta $ \lambda_k $ la costante di decadimento parziale della $ k $-esima branch, nel caso di $ n $ branches si ha:
\begin{equation}
	\lambda \equiv \lambda_1 + \dots + \lambda_n
	\label{eq:2.4}
\end{equation}
e questa costante totale è l'unica che si osserva, anche quando si rileva una sola delle branches. Si definiscono i branching ratios come $ B_k \defeq \frac{\lambda_k}{\lambda} $.

\subsubsection{Decay chains}

Spesso, in un decadimento radioattivo, capita che anche i prodotti siano radiaottivi: ciò dà vita ad una catena di decadimenti $ N_1 \xrightarrow{\lambda_1} N_2 \xrightarrow{\lambda_2} N_3 \dots $, dove le $ \lambda_k $ sono diverse tra loro.\\
Una decay chain è descritto da un sistema di coupled differential equations. Nel caso, ad esempio, di un doppio decadimento (quindi con prodotto $ N_3 $ stabile):
\begin{equation}
	\begin{cases}
		\dot{N}_1 = - \lambda_1 N_1 \\
		\dot{N}_2 = \lambda_1 N_1 - \lambda_2 N_2 \\
		\dot{N}_3 = \lambda_2 N_2
	\end{cases}
	\quad\Longrightarrow\quad
	\begin{cases}
		N_1(t) = N_0 e^{-\lambda_1 t} \\
		N_2(t) = \frac{\lambda_1}{\lambda_1 + \lambda_2} N_0 \left( e^{-\lambda_1 t} - e^{-\lambda_2 t} \right) \\
		N_3(t) = \frac{\lambda_1 \lambda_2}{\lambda_1 + \lambda_2} N_0 \left( \frac{1 - e^{-\lambda_1 t}}{\lambda_1} - \frac{1 - e^{-\lambda_2 t}}{\lambda_2} \right)
	\end{cases}
	\label{eq:2.5}
\end{equation}
La soluzione generale al caso di $ n $ decadimenti è dato dall'\textit{equazione di Bateman}:
\begin{equation}
	N_k(t) = \sum_{i = 1}^{k} \left[ N_0^{(i)} \left( \prod_{j = 1}^{k-1} \lambda_j \right) \left( \sum_{j = i}^{k} \frac{e^{-\lambda_j t}}{\prod_{p=i, p\neq j}^{k} (\lambda_p - \lambda_j)} \right) \right]
	\label{eq:2.6}
\end{equation}

\paragraph{Equilibrio radioattivo} Si parla di equilibrio radioattivo quando la specie radioattiva madre e quella figlia hanno la stessa attività, ovverosia quando la specie figlia decade allo stesso rate a cui è prodotta. In una decay chain, l'equilibrio radioattivo può instaurarsi tra ciascuna coppia di nuclidi della catena: la condizione generale da soddisfarre è che $ (t_{1/2})_{\text{madre}} > (t_{1/2})_{\text{figlia}} $.\\
In particolare, si parla di:
\begin{enumerate}
	\item equilibrio transiente: si ha quando $ (t_{1/2})_{\text{madre}} \approx (t_{1/2})_{\text{figlia}} $, dunque, dopo un periodo di transienza iniziale, l'activity della specie madre e quella della specie figlia diventano uguali;
	\item equilibrio secolare: si ha quando $ (t_{1/2})_{\text{madre}} \rightarrow \infty $ (comparabile all'età della Terra), dunque la sua activity è praticamente costante; di conseguenza, dopo un certo periodo di tempo, anche l'activity della specie figlia diventerà costante e pari a quella della specie madre (si vede analiticamente dall'Eq. 2.5 ponendo $ \lambda_1 \ll \lambda_2 $).
\end{enumerate}

\paragraph{Serie radioattive naturali}

Ci sono 4 serie decay chains naturali principali, tutte composte da decadimenti $ \alpha $ e $ \beta^- $:
\begin{enumerate}
  \item serie del torio: inizia col $ \ch{^{232} Th} $ e termina col $ \ch{^{208} Pb} $, il decadimento col tempo di dimezzamento più lungo è $ \ch{^{232} Th} \rightarrow \ch{^{228} Ra} $ con $ t_{1/2} = 14 \,\text{Gy} $;
  \item serie dell'uranio: inizia col $ \ch{^{238} U} $ e termina col $ \ch{^{206} Pb} $, il decadimento col tempo di dimezzamento più lungo è $ \ch{^{238} U} \rightarrow \ch{^{234} Th} $ con $ t_{1/2} = 4.5 \,\text{Gy} $;
  \item serie del plutonio: inizia col $ \ch{^{239} Pu} $ e termina col $ \ch{^{207} Pb} $, il decadimento col tempo di dimezzamento più lungo è $ \ch{^{235} U} \rightarrow \ch{^{231} Th} $ con $ t_{1/2} = 0.71 \,\text{Gy} $;
  \item serie del nettunio: inizia col $ \ch{^{237} Np} $ e termina col $ \ch{^{209} Bi} $, il decadimento col tempo di dimezzamento più lungo è $ \ch{^{237} Np} \rightarrow \ch{^{233} Pa} $ con $ t_{1/2} = 2.3 \,\text{My} $;
\end{enumerate}
Quest'ultima serie non è più osservabile in natura poiché la sua vita media non è comparabile con l'età della Terra, a differenza delle altre tre.

\paragraph{Carbon dating}

Il $ \ch{^{14}C} $ è un isotopo radioattivo del carbonio con $ t_{1/2} = 5730\,\text{y} $; nonostante questa vita media relativamente corta, esso è prodotto continuamente grazie ai raggi cosmici in atmosfera tramite neutron capture: $ \ch{^{14}N} + n \rightarrow \ch{^{14}C} + p^+ $; una volta assorbito dai sistemi biologici, esso decade per decadimento $ \beta^- $: $ \ch{^{14}C} \rightarrow \ch{^{14}N} + e^- + \bar{\nu}_e $.\\
Misurata la specific activity $ a $ del campione da datare, si può risalire al time since death comparandola alla standard specific activity $ a_0 = 0.266\,\text{Bq}/\text{g} $: $ T = \frac{t_{1/2}}{\ln 2} \ln \frac{a}{a_0} = - 8033\,\text{y} \cdot \ln \frac{a}{a_0} $.\\
L'assunzione fondamentale di questo metodo di datazione è che la concentrazione naturale di $ \ch{^{14}C} $ rimanga costante nel tempo: con la Guerra Fredda questo è diventato un problema, poiché i test nucleari hanno fatto raddoppiare per un periodo tale concentrazione, e solo ultimamente si sta tornando ai livelli precedenti. Inoltre, va notato che il carbon dating è efficacie solo per oggetti non più vecchi di 50'000 anni: per epoche precedenti, sono necessari altre coppie isotopiche, come ad esempio $ \ch{^{40}K} $ - $ \ch{^{40}Ar} $, $ \ch{^{235}U} $ - $ \ch{^{207}Pb} $ o $ \ch{^{238}U} $ - $ \ch{^{208}Pb} $.












\chapter{Radiazioni}
\pagestyle{body}
\selectlanguage{italian}

L'interazione delle radiazioni con il mezzo in cui si propagano dipende dalla natura delle particelle di cui esse sono composte:
\begin{itemize}
	\item particelle cariche, che hanno un'interazione continua con gli elettroni del mezzo a causa della forza coulombiana:
	\begin{itemize}
		\item particelle cariche pesanti (range tipico $ \sim 10^{-5} \m $);
		\item elettroni veloci (range tipico $ \sim 10^{-3} \m $);
	\end{itemize}
	\item patricelle neutre, che invece non subiscono l'interazione coulombiana ma agiscono in maniera discreta, causando un grande scambio energetico per ogni singola interazione (spesso con i nuclei):
	\begin{itemize}
		\item raggi X e $ \gamma $ (range tipico $ 10^{-1} \m $);
		\item neutroni (range tipico $ 10^{-1} \m $).
	\end{itemize}
\end{itemize}
Per quanto riguarda i principali tipi di radiazioni, che rispecchiano ciascuna delle casistiche appena evidenziate:
\begin{enumerate}
	\item raggi $ \alpha $: percorrono qualche centimetro in aria, vengono arrestati da un foglio di carta;
	\item raggi $ \beta $: percorrono qualche metro in aria, vengono arrestati da qualche millimetro di alluminio;
	\item raggi $ \gamma $: a seconda dell'energia, percorrono fino a molte centinaia di metri in aria, vengono arrestati da notevoli spessori di cemento o piombo;
	\item neutroni: la penetrazione dipende dall'energia, vengono arrestati da notevoli spessori di cemento, acqua o paraffina.
\end{enumerate}

\section{Particelle cariche}

Il tipo di interazione dominante tra radiazione e materia sono le collisioni inelastiche con gli elettroni del mezzo: ciò induce vari fenomeni, principalmente l'eccitazione degli atomi, la loro ionizzazione, la fluorescenza e la fosforescenza del materiale (questi ultimi sono causati dal tempo di diseccitazione molto lento degli stati molecolari, ceh genera l'emissione di luce visibile).\\
Dato che le interazioni avvengono principalmente con gli elettroni atomici, si può calcolare il rapporto delle sezioni d'urto coulombiane:
\begin{equation}
	\frac{\sigma_{\text{nucl}}}{\sigma_{\text{atom}}} \sim \frac{\pi R^2}{\pi a_Z^2} \approx \frac{A^{2/3} \cdot 10^{-26} \,\text{cm}^2}{10^{16} \,\text{cm}^2} = A^{2/3} \cdot 10^{-10}
	\label{eq:3.1}
\end{equation}
Il range tipico per questo rapporto è dunque tra $ 10^{-7} $ e $ 10^{-8} $.

\subsection{Scattering e straggling}

A livello cinematico, considerando una particella di massa $ m \gg m_e $ e velocità $ v_0 $ incidente su un elettrone a riposo:
\begin{equation*}
	\begin{cases}
		\frac{1}{2} m v_0^2 = \frac{1}{2} m v_f^2 + \frac{1}{2} m_e v_e^2 \\
		m v_0 = m v_f + m_e v_e
	\end{cases}
	\quad\Rightarrow\quad
	\begin{cases}
		v_f = \frac{m - m_e}{m + m_e} v_0 \\
		v_e = \frac{2m}{m + m_e} v_0
	\end{cases}
\end{equation*}
Dato che $ v_e \approx 2v_0 $, si ha che:
\begin{equation}
	K_e = 4 \frac{m_e}{m} K_0
	\label{eq:3.2}
\end{equation}
Se ad esempio si considera un protone ($ m_p \approx 2000 m_e $), si vede che $ \Delta K_p = \frac{1}{500} K_0 $, quindi in circa 500 collisione esso viene completamente arrestato: questo avviene in pochi micrometri di materiale.\\
Si vede dunque che c'è una perdita di energia nel corso di numerose interazioni: il processo è graduale. Inoltre, al diminuire dell'energia le particelle incidenti tendono a deviare sempre di più dal percorso rettilineo originario a causa degli scattering, andando a creare uno sciame di particelle: questo fenomeno, detto \textit{straggling}, è sia energetico che posizionale, in quanto ci sarà una regione del materiale nel quale è presente la concentrazione più alta di particelle di radiazione.\\
La proiezione della traiettoria di una particella lungo la direzione di propagazione iniziale è detto \textit{range} ed ha un comportamento stocastico: il range sarà distribuito (diffuso) sia posizionalmente (dipendenza dal materiale) che energeticamente (dipendenza dalla particella), infatti si parla di range straggling. Quantitativamente, detto $ d\ve{s} $ il displacement tra uno scattering e l'altro e $ \theta_s $ l'angolo tra $ d\ve{s} $ e la direzione iniziale di propagazione, si definisce il range come:
\begin{equation}
	R(E) \defeq \int ds \braket{\cos \theta_s} = \int_0^E dE \left( \frac{dE}{ds} \right)^{-1} \braket{\cos \theta_s}
	\label{eq:}
\end{equation}
dove $ \frac{dE}{ds} $ è detto \textit{stopping power} ed esprime la perdita di energia per unità di lunghezza percorsa. Si vede subito che il range è più piccolo dell'effettiva traiettoria percorsa: $ \int ds \ge R(E) $.

\subsection{Modello fenomenologico}

È possibile formulare dei modelli fenomenologici per esprimere lo stopping power in funzione delle caratteristiche della particella e del materiale.\\
Un primo modello fu proposto da Lindhardt, che con la sua teoria dell'elettrone descrisse la perdita di energia causata dalla ionizzazione del materiale:
\begin{equation}
	- \frac{1}{\rho} \frac{dE}{ds} \approx \frac{4\pi Z_p^2 e^4}{m_e v^2} Z_t \frac{1}{2} \ln \left[ \frac{2m_e v^2}{I} \right]
	\label{eq:3.4}
\end{equation}
dove $ \rho $ è la densità atomica del materiale, $ Z_t $ il numero atomico degli atomi target, $ Z_p $ quello della particella incidente, $ v $ la sua velocità e $ I $ è la cosiddetta \textit{ionizing energy}, un parametro empirico che rappresenta l'energia d'eccitazione media degli elettroni nel materiale. Si vede che il fattore dominante è il primo, mentre il logaritmo diventa importante per alte velocità.\\
Un modello che descrive meglio le particelle massive ($ m \gg m_e $) anche in regime relativistico è quello dato dall'\textit{equazione di Bethe-Bloch}, che esprime l'average energy loss:
\begin{equation}
	-\frac{1}{\rho} \frac{dE}{ds} \approx 0.1535 \frac{Z_p^2}{\beta^2} \frac{Z_t}{A_t} \left[ 2 \ln \left( \frac{2m_e c^2}{I (1 - \beta^2)} \right) - 2\beta^2 \right] \text{MeV} \text{cm}^2 \text{g}^{-1}
	\label{eq:3.5}
\end{equation}
Empiricamente, si trova:
\begin{equation}
	I = h \braket{\nu_e} =
	\begin{cases}
		(12 Z_t + 7) \ev & Z_t < 13 \\
		(9.76 Z_t + 58.8 Z_t^{-0.19}) \ev & Z_t \ge 13
	\end{cases}
	\label{eq:3.6}
\end{equation}
È importante ricordare gli andamenti dello stopping power in funzione dei vari parametri:
\begin{itemize}
	\item $ \sim \rho Z_t $, maggiore per materiali densi;
	\item $ \sim Z_p^2 $, maggiore per raggi di ioni pesanti;
	\item $ \sim v^{-2} $, maggiore per particelle lente.
\end{itemize}
In particolare, l'andamento $ \sim v^{-2} $ è equivalente a $ \sim K_0^{-1} $: la proporzionalità inversa all'energia della particella incidente è seguita sperimentalmente con buon accordo fino al minimo di ionizzazione del materiale, ovvero all'energia per cui c'è probabilità minore di interazione tra radiazione e mezzo (dunque minore stopping power). Per energie elevate, al limite relativistiche, i termini $ \beta^2 $ diventano dominanti; inoltre, è necessario apportare delle correzioni per considerare le perdite radiative da Bremsstrahlung ed eventuali effetti $ \rho $-dependent nei plasmi.\\
In Fig. \ref{silicon} è possibile osservare come, fissato il mezzo, lo stopping power aumenti come all'aumentare di $ Z_p $ e diminuisca all'aumentare di $ K_0 $; inoltre, si può vedere come gli ioni più pesanti, per raggiungere lo stesso range, necessitino di maggiore energia.

\begin{figure}[!b]
	\centering
	\includegraphics[width=0.45\textwidth]{silicon-sp.png}
	\includegraphics[width=0.495\textwidth]{silicon-r.png}
	\caption{Stopping power and range in Silicon.}
	\label{silicon}
\end{figure}

\paragraph{Particle identification}

Se si ha uno strumento in grado di misurare sia la perdita di energia di una particella carica che la sua energia totale\footnote{Un esempio potrebbe essere un rilevatore al silicio con due strati: uno sottile per misurare la perdita di energia ed uno spesso per fermare la particella, così da poter sommare le due rilevazioni ed ottenere l'energia totale.}, è possibile correlarle ed ottenere dei rami d'iperbole descritti da $ \frac{dE}{ds} E \propto Z_p^2 $. In questo modo si può distinguere tra i vari fasci di particelle ed identificarle, senza però poterne estrarre le masse.

\paragraph{Bragg peak}

La graduale perdita di energia nel range è ciò che sta alla base delle tecniche di adroterapia, ovvero la cura di tumori tramite bombardamento con particelle cariche (soprattutto protoni). Per fare ciò, è necessario conoscere con elevata precisione la perdita di energia specifica (perdita di energia per unità di massa del target) ed il range: quando una particella carica attraversa il corpo umano, essa ionizza tutto il materiale che percorre.\\
Per avere la massima efficacia, sarebbe ideale che la maggior parte dell'energia venisse dissipata in un intervallo specifico del range, così da poter attaccare accuratamente il tumore, ma questo è proprio ciò che avviene: più spazio viene percorso, più la particella perde energia e, di conseguenza, rallenta, quindi la maggior parte dell'energia viene rilasciata alla fine del range, nel cosiddetto \textit{picco di Bragg}. In questo modo, si può regolare il fascio di ioni in modo da far rilasciare la maggior parte dell'energia nel punto in cui si trova la massa tumorale.\\
Per avere un metro di misura della perdita di energia specifica si definisce la dose assorbita $ \frac{dE}{dm} $, misurata in $ \text{gray} \equiv \text{J} / \text{m} $, dove $ dE $ è l'energia rilasciata nella massa $ dm $ del mezzo dalla radiazione ionizzante. In Fig. \ref{bragg} è possibile vedere come la presenza del picco di Bragg renda i fasci di particelle cariche massive particolarmente efficaci nella cura di tumori.

\begin{figure}[!t]
	\centering
	\includegraphics[width=1.00\textwidth]{bragg.png}
	\caption{Absorbed dose for different kinds of radiation.}
	\label{bragg}
\end{figure}

\section{Raggi \texorpdfstring{$ \gamma $}{TEXT}}

A differenza delle particelle cariche, i fotoni agiscono in maniera discreta col materiale e la ionizzazioni da essi causati avviene in regioni limitate del mezzo.\\
L'interazione dei fotoni col mezzo può avvenire in tre modi (vedere Fig. \ref{ph-i-p}):
\begin{enumerate}
	\item effetto fotoelettrico, dominante per $ E_{\gamma} < 500 \kev $;
	\item Compton scattering, dominante per $ 500 \kev \le E_{\gamma} \le 2 \mev $;
	\item pair production, dominante per $ E_{\gamma} > 2 \mev $.
\end{enumerate}

\begin{figure}[!b]
	\centering
	\includegraphics[width=0.70\textwidth]{lin-att-coeff.png}
	\caption{Linear attenuation coefficient for photon interaction.}
	\label{ph-i-p}
\end{figure}












\chapter{Interazione tra Nucleoni}
\pagestyle{body}
\selectlanguage{italian}

\section{Interazioni nucleone-nucleone}

\subsection{Bosoni e fermioni}

Si ricordi che un generico operatore di momento angolare soddisfa la condizione di commutazione:
\begin{equation}
	[\hat{L}_i,\hat{L}_j] = i\hbar \sum_{k = 1}^{3} \epsilon_{ijk} \hat{L}_k
	\label{eq:4.1}
\end{equation}
Diagonalizzando $ \hat{L}^2 $ e $ \hat{L}_z $:
\begin{equation}
	\hat{L}^2 \ket{\psi} = \hbar \ell (\ell + 1) \ket{\psi}
	\label{eq:4.2}
\end{equation}
\begin{equation}
	\hat{L}_z \ket{\psi} = \hbar m \ket{\psi}
	\label{eq:4.3}
\end{equation}
con $ m = -\ell, \dots, \ell $. Il momento angolare può essere sia orbitale che di spin (intrinseco): questo permette di distinguere particelle fermioniche e bosoniche.\\
Le particelle con spin semi-intero sono dette \textit{fermioni}, quelle con spin intero \textit{bosoni}: in maniera grossolana, si può dire che i fermioni costituiscono la materia, mentre i bosoni mediano le interazioni, ma ci sono delle eccezioni. Una delle principali proprietà è che, in sistemi di particelle identiche, le funzioni d'onda bosoniche sono simmetriche per scambio di particelle, mentre quelle fermioniche sono antisimmetriche: una conseguenza importante è il \textit{principio di esclusione di Pauli}, il quale afferma che due fermioni identici non possono avere gli stessi numeri quantici (ovvero avere lo stesso stato).\\
Si ricordi che, in generale, quando due momenti angolari interagiscono, i numeri quantici si compongono secondo:
\begin{equation}
	\abs{\ell_1 - \ell_2} \le j \le \ell_1 + \ell_2
	\label{eq:4.4}
\end{equation}
Ad esempio, un sistema di due fermioni può avere spin 0 o 1.

\subsection{Caratteristiche dell'interazione}

Non si conosce molto bene l'interazione tra nucleoni. Dallo studio della binding energy (vedere Fig. \ref{bind-en}) si evince che la binding energy per nucleone ($ B(A,Z)/A $) satura a circa $ 8\mev $: questo è una conseguenza del fatto che l'interazione tra nucleoni è a corto range. Infatti, le interazioni a lungo range scalano come $ E \sim \frac{1}{2} A (A - 1) $, come avviene ad esempio per quella coulombiana, mentre le interazioni a corto range come $ E \sim A $, dato che sono le particelle immediatamente vicine interagiscono tra loro.\\
Una stima cruda del range dell'interazioni tra nucleoni è data dalle dimensioni della particella $ \alpha $, che ha un daimetro di circa $ 1.5\fm $: si evince che il range dell'interazioni è circa tra $ 1\fm $ e $ 2\fm $.\\
In maniera più precisa, si può determinare il range a partire dal mediatore dell'interazione: dato che durante un'interazione la particella mediatrice deve esistere per un tempo $ \Delta t = \Delta r / c $, dove $ \Delta r $ è il range, dal principio d'indeterminazione si ha che $ \Delta t \Delta E \sim \hbar $, dunque se il mediatore ha massa $ m $:
\begin{equation}
	\Delta r \sim \frac{\hbar}{m c}
	\label{eq:4.5}
\end{equation}
Per l'interazione elettromagnetica, ad esempio, dato che il mediatore è il fotone, che ha massa nulla, il range è infinito. L'interazione tra nucleoni, invece, è mediata da mesoni: il mesone più leggero è il pione, la cui massa $ m_{\pi} \approx 135\mev $ determina un range $ \Delta r \approx 1.4 \fm $, in accordo con la stima iniziale: mesoni più pesanti medieranno l'interazione con range più corti.\\
Si ricordi, inoltre, che i nucleoni non sono oggetti indivisibili, bensì hanno una struttura a quark: per questo, l'interazione tra nucleoni non è altro che una risultante dell'interazione tra quark, sebbene questa sia ancora eccessivamente complessa da studiare. L'unica conclusione qualitativa che si può ottenere è che il potenziale d'interazione tra nucleoni è analogo al potenziale inter-molecolare (potenziale di Lennard-Jones), il quale può essere modellato come una buca di potenziale sferica.

\section{Il deutone}

Il deutone è il nucleo di deuterio, ovvero il nuclide $ \ch{^2_1 H} $: questo è il sistema nucleonico più semplice, composto da un neutrone ed un protone.\\
La sua binding energy è pari a $ B = 2.22457\mev $, calcolabile tramite l'energia del fotone nella reazione $ n + p^+ \rightarrow d^+ + \gamma $: questo è un sistema debolmente legato, dato che $ B / A \approx 1 \mev $, poiché l'interazione non raggiunge la saturazione (essendoci solo due nucleoni).\\
Tramite electron scattering è possibile stimare la distanza tra i due nucleoni, trovando il valore medio $ \braket{r^2}^{1/2} = 1.963 \pm 0.04 \fm $: dato che protone e neutrone sono larghi circa $ 0.6\fm $ ed il range d'interazione è circa $ 1.4\fm $, non solo i due nucleoni sono poco legati, ma sono anche abbastanza lontani tra loro.\\
Il sistema ha inoltre momento angolare totale $ J^{\pi} = 1^+ $ ed isospin $ I_3 = 0 $: è interessante notare che lo stato fondamentale del deutone ha momento angolare orbitale $ \ell = 0 $, mentre lo spin è $ s = 1 $ e non ci sono stati eccitati. Se però si considera che protone e neutrone hanno spin $ \frac{1}{2} $, si vede che i valori possibili di spin per il deutone dovrebbero essere $ s = 0 $ (singoletto, $ m_s = 0 $) ed $ s = 1 $ (tripletto, $ m_s = -1,0,1 $): ciò evidenzia come l'interazione tra nucleoni sia spin-dependent, in modo da essere maggiormente attrattiva nella configurazione $ s = 1 $.\\
Infine, si trovano il momento di dipolo magnetico $ \mu_d = 0.857406 \mu_N $ ed il momento di quadrupolo elettrico $ Q_d = 0.2859 e \fm^2 $: quest'ultimo indica che i due nucleoni non orditano in maniera sferica, poiché altrimenti si avrebbe solo momento di dipolo.

\subsection{Modello semplificato}

È possibile schematizzare i due nucleoni in orbita tra loro come una singola particella di massa ridotta $ \mu = (m_p^{-1} + m_n^{-1})^{-1} $ in orbita attorno al centro di massa del sistema e soggetta ad un potenziale d'interazione, il quale può essere studiato come una buca di potenziale sferica. In particolare:
\begin{equation}
	V(r) =
	\begin{cases}
		0 & r < a \\
		-V_0 & r > a
	\end{cases}
	\label{eq:4.6}
\end{equation}
Essendo questo un potenziale centrale a simmetria sferica, la funzione d'onda del sistema può essere scritta come:
\begin{equation}
	\psi(r,\theta,\phi) = \frac{u(r)}{r} Y_{\ell,m}(\theta,\phi)
	\label{eq:4.7}
\end{equation}
L'equazione di Schrödinger diventa quindi:
\begin{equation}
	- \frac{\hbar^2}{2\mu} \frac{d^2 u(r)}{dr^2} + \left[ V(r) + \frac{\hbar^2 \ell (\ell + 1)}{2\mu r^2} \right] u(r) = \varepsilon u(r)
	\label{eq:4.8}
\end{equation}
dove $ \varepsilon $ è l'energia dello stato considerato. Risolvendo si trova:
\begin{equation}
	u(r) =
	\begin{cases}
		A \sin (k_1 r) + B \cos (k_2 r) & r < a \\
		C e^{-k_2 r} + D e^{k_2 r} & r > a
	\end{cases}
	\qquad
	k_1 = \sqrt{\frac{2\mu}{\hbar^2} (\varepsilon - V_0)}, \, k_2 = \sqrt{\frac{2\mu \abs{\varepsilon}}{\hbar^2}}
	\label{eq:4.9}
\end{equation}
Dato che $ u(r) $ deve tendere a $ 0 $ per $ r \rightarrow 0 $ e $ r \rightarrow \infty $, si ha $ B = D = 0 $. La continua derivabilità in $ r = a $ impone invece la condizione:
\begin{equation}
	k_1 \cot (k_1 a) = - k_2
	\label{eq:4.10}
\end{equation}
Inserendo in questa equazione la binding energy ed il raggio del deutone trovati sperimentalmente, si ottiene il valore $ V_0 \approx -35\mev $: questa è ovviamente solo una stima, dato che il modello è molto semplice, ed è efficace solo per il caso $ s = 1 $, dato che per $ s = 0 $ non ci sono stati legati.

\subsection{Momento di dipolo magnetico}

Il momento magnetico del deutone può essere espresso come:
\begin{equation}
	\ve{\mu}_s = g_p \mu_N \ve{S}_p + g_n \mu_N \ve{S}_n \equiv g_s \mu_N \ve{S}
	\label{eq:4.11}
\end{equation}
Proiettando nella direzione dello spin totale:
\begin{equation*}
	\begin{split}
		g_s S^2
		&= g_s s (s + 1) = g_p \ve{S}_p \cdot \ve{S} + g_n \ve{S}_n \cdot \ve{S} = g_p (S_p^2 + \ve{S}_p \cdot \ve{S}_n) + g_n (S_n^2 + \ve{S}_n \cdot \ve{S}_p)\\
		&= g_p (s_p (s_p + 1) + s_p s_n) + g_n (s_n (s_n + 1) + s_n s_p) = (g_p s_p + g_n s_n) (s_p + s_n + 1)
	\end{split}
\end{equation*}
Nello stato fondamentale $ s_p = s_n = \frac{1}{2} $ e $ s = 1 $, dunque, ricordando i valori $ g_p = +5.585691 $ e $ g_n = -3.826084 $:
\begin{equation}
	g_s = \frac{g_p + g_n}{2} = +0.879804
	\label{eq:4.12}
\end{equation}
Il valore trovato considerando solo l'accoppiamento degli spin non corrisponde a quello sperimentale ($ g_s = +0.857406 $): evidentemente anche il momento angolare orbitale contribuisce al momento di dipolo magnetico e ciò è un'altra indicazione della non sfericità dell'orbita.\\
Classicamente si avrebbe $ \ve{\mu} = \frac{q}{2m} \ve{L} $, mentre quanto-meccanicamente $ \ve{\mu} = \frac{q}{2m} \frac{\hbar}{2} \ve{L} = \frac{1}{2} \mu_N \ve{L} $, quindi:
\begin{equation}
	\ve{\mu} = g_s \mu_N \ve{S} + \frac{1}{2} \mu_N \ve{L} \equiv g_d \mu_N \ve{J}
	\label{eq:4.13}
\end{equation}
Proiettando lungo il momento angolare totale:
\begin{equation*}
	\begin{split}
		g_d J^2
		&= g_d j (j + 1) = \frac{1}{2} \ve{L} \cdot (\ve{L} + \ve{S}) + g_s \ve{S} \cdot (\ve{L} + \ve{S}) = \frac{1}{2} L^2 + g_s S^2 + \left( \frac{1}{2} + g_s \right) \ve{L} \cdot \ve{S}
	\end{split}
\end{equation*}
Ricordando che $ \ve{L}\cdot\ve{S} = \frac{1}{2} (J^2 - L^2 - S^2) $:
\begin{equation}
	g_d = \frac{1}{2} \left( \frac{1}{2} + g_s \right) + \frac{1}{2} \frac{\ell (\ell + 1) - s (s + 1)}{j (j + 1)} \left( \frac{1}{2} - g_s \right)
	\label{eq:4.14}
\end{equation}
Dato che sperimentalmente si trova che lo stato fondamentale del deutone ha $ J^{\pi} = 1^+ $ ed in generale $ \abs{\ell - s} \le j \le \ell + s $ e $ \pi = (-1)^{\ell} $, lo stato fondamentale è sovrapposizione di soli due stati, esprimibili come $ \ket{\ell,s,j} = \ket{0,1,1} $ e $ \ket{\ell,s,j} = \ket{2,1,1} $: per il primo $ g_0 = g_s = +0.879804 $, mentre per il secondo $ g_2 = \frac{3}{4} - \frac{1}{2} g_s = +0.310098 $. Si ha quindi:
\begin{equation}
	\ket{\psi_d} = a_0 \ket{0,1,1} + a_2 \ket{2,1,1}
	\label{eq:4.15}
\end{equation}
È possibile trovare i due coefficienti grazie alla normalizzazione e all'expectation value di $ g_d $ (sperimentale):
\begin{equation*}
	\begin{cases}
		\abs{a_0}^2 + \abs{a_2}^2 = 1 \\
		g_0 \abs{a_0}^2 + g_2 \abs{a_2}^2 = g_d
	\end{cases}
	\quad\Longrightarrow\quad
	\abs{a_0}^2 = 0.96,\, \abs{a_2}^2 = 0.04
\end{equation*}
Lo stato fondamentale del deutone è una sovrapposizione: al 96\% è sferico, mentre al 4\% è allungato.

\subsection{Momenti di multipolo elettrici}

In generale, il momento di $ n $-polo elettrico è definito come integrale pesato della densità di carica:
\begin{equation}
	K_n \defeq \int_V d^3\ve{r} \, r^{n-1} P_{n-1}(\cos \theta) \rho_(\ve{r})
	\label{eq:4.16}
\end{equation}
dove $ P_n(\cos \theta) $ sono i polinomi di Legendre.\\
Sperimentalmente, si trova che se un sistema quantistico ha un parità definita (ovvero è un numero quantico, dunque la densità non cambia quando si invertono gli assi) il momento di dipolo è nullo: c'è molta ricerca in corso per verificare se ciò è vero per ogni particella elementare (ad esempio tramite misure del momento di dipolo del neutrone).\\
Un parametro importante è il momento di quadrupolo:
\begin{equation}
	Q = \int_V d^3\ve{r} \, r^2 (3\cos^2 \theta - 1) \rho(\ve{r})
	\label{eq:4.17}
\end{equation}
Si ha infatti che, considerando una carica totale $ q = Ze $ distruibuita come un ellissoide di semiasse maggiore (lungo $ z $) $ b $ e sezione sul piano $ xy $ circolare di raggio $ a $, il momento di quadrupolo vale:
\begin{equation}
	Q = \frac{2}{5} q (b^2 - a^2)
	\label{eq:4.18}
\end{equation}
Il segno di $ Q $ permette quindi di determinare se la distribuzione è prolata (allungata, $ b > a $ e $ Q > 0 $) o oblata (schiacciata, $ b < a $ e $ Q < 0 $) lungo $ z $. Sperimentalmente, per il deutone si trova $ Q_d = 0.2859 e\fm^2 $, dunque esso ha forma prolata; è anche possibile mostrare che tale valore $ Q_d $ si ottiene con la stessa combinazione di $ \ket{0,1,1} $ e $ \ket{2,1,1} $ trovata per il momento magnetico.

\subsection{Forze non-centrali}

Se la forza agente sui nucleoni fosse puramente centrale, il momento angolare sarebbe un buon numero quantico poiché le forze centrali lo conservano. Invece, si è visto che lo stato fondamentale del deutone è una sovrapposizione di due stati ad $ \ell $ distinto: ciò suggerisce la presenza di una forza non-centrale, detta \textit{forza tensoriale}: questa è l'equivalente forte dell'interazione elettromagnetica tra due dipoli elettrici o magnetici (naturalmente molto più intensa). Il potenziale tensoriale dipende da un coefficiente $ S_{1,2} $ definito come:
\begin{equation}
	S_{1,2} \defeq 3 \frac{(\ve{S}_1 \cdot \ve{r}) (\ve{S}_2 \cdot \ve{r})}{r^2} - \ve{S}_1 \cdot \ve{S}_2
	\label{eq:4.19}
\end{equation}
dove $ \ve{r} $ è il vettore di separazione tra i due nucleoni; si nota che $ S_{1,2} $ dipende soltanto dall'orientazione di $ \ve{r} $ e non da $ r $, ed inoltre la sua media su tutte le possibili orientazioni (dunque su $ \mathbb{S}^2 $) è nulla. Nella configurazione prolata ($ \ve{r} \parallel \hat{\ve{z}},\ve{S}_1,\ve{S}_2 $) $ S_{1,2} = \frac{1}{2} $, mentre in quella oblata ($ \ve{r} \perp \hat{\ve{z}},\ve{S}_1,\ve{S}_2 $) $ S_{1,2} = - \frac{1}{4} $: ciò indica che la configurazione prolata è quella favorita, e ciò è confermato dal momento di quadrupolo elettrico.\\
È presente anche una forza di tipo nucleare, detta \textit{forza spin-orbita}, visibile sia tramite esperimenti di scattering che nell'ordinamento dei libelli nel modello a shell, con andamento:
\begin{equation}
	V_{LS} \sim \ve{L} \cdot \ve{S}
	\label{eq:4.20}
\end{equation}
In conclusione, lo stato fondamentale del deutone ha principalmente $ \ell = 0 $, con una componente $ \ell = 2 $ che spiega il momento di quadrupolo elettrico e il momento di dipolo magnetico, avendo comunque $ s = 1 $ e $ J^{\pi} = 1^+ $: la simmetria della funzione d'onda non viola il principio di esclusione di Pauli, poiché neutrone e protone sono distinguibili; il principio non permette invece di avere funzioni d'onda simmetriche per stati neutrone-neutrone e protone-protone, dunque non possono esistere stati legati di questo tipo.












\chapter{Nuclear Models}
\pagestyle{body}
\selectlanguage{italian}

Il nucleo atomico è un sistema particolarmente complicato da descrivere, essendo un sistema a molti corpi quantistico legato dall'interazione forte, la quale non ha un'espressione ben definita come la forza coulombiana (non essendo neanche una forza che agisce a coppie di particelle).\\
In particolare, è possibile dare una duplice descrizione del nucleo, tenendo conto di diverse osservazioni sperimentali: da un lato è possibile vede il nucleo come un insieme di particelle singole, esprimendo gli stati nucleari in funzione delle singole eccitazioni dei vari nucleoni; dall'altro è invece possibile descrivere il nucleo come una sorta di $ \virgolette{liquid drop} $ (riprendendo il modello di Weiszäcker), così da poter tener conto dei moti collettivi nucleari che si osservano.

\section{Nuclear shell model}

La descrizione del nucleo come insieme di particelle singole porta ad un modello analogo al modello atomico a shell, sotto l'ipotesi che i nucleoni si muovano indipendentemente gli uni dagli altri in un potenziale a simmetria sferica: questo però è vero solo nell'intorno di determinati valori di $ Z $ e $ N $, mentre in realtà la maggior parte dei nuclei sono deformati.

\subsection{Atomic physics}

Il modello atomico a shell descrive gli elettroni dell'atomo come disposti su una serie di livelli energetici, detti shells, posti a determinate distanza (sia spaziali che energetiche) tra loro.\\
In particolare, gli elettroni sono descritti da quattro quantum numbers $ \ket{n,\ell,m,s} $: quello principale $ n \in \N $, quello orbitale $ \ell \in [0,n-1] $, quello magnetico $ m \in [-\ell,\ell] $ e quello di spin $ s = \pm \frac{1}{2} $.\\
Nell'atomo più semplice, ovvero l'atomo di idrogeno $ \ch{^1H} $, si ha una degenerazione delle shells a diverso momento angolare:
\begin{equation}
	E_{n,\ell} = \alpha^2 \frac{m_e c^2}{2 (n + \ell)^2}
	\label{eq:5.1}
\end{equation}
dove $ \alpha = \frac{e^2}{4\pi \epsilon_0 \hbar c} $ è la costante di struttura fine. In presenza di deviazioni dal potenziale puro $ \sim \frac{1}{r} $ questa degenerazione viene rimossa: ciò avviene considerando la presenza di altri elettroni, che modificano il potenziale, e l'interazione spin-orbita, che porta al cosiddetto splitting fine.\\
Il modello atomico a shell può essere giustificato sia a livello teorico che sperimentale. Innanzitutto il potenziale può essere assunto come quello Coulombiano a simmetria sferica poiché $ R_{\text{nucleo}} \ll R_{\text{atomo}} $, ed inoltre il centro del potenziale è ben definito poiché $ M_{\text{nucleo}} \gg M_{\text{elettroni}} $: una volta determinato in questo modo il potenziale, tutto deriva dalle leggi di quantizzazione del momento angolare e dal principio d'esclusione di Pauli. A livello sperimentale, invece, esso può essere confermato studiando l'andamento di varie proprietà atomiche; ad esempio, in Fig. \ref{atom-prop} sono riportati il raggio atomico e l'energia di prima ionizzazione dei vari atomi: si può notare che, in corrispondenza delle varie shell closures, si evidenziano notevoli discontinuità nell'andamento altrimenti piuttosto monotono di tali quantità. Le shell closures altro non sono che il completo riempimento delle varie subshells magnetiche di ciascuna shell a dato momento angolare: infatti, una shell associata ad un determinato $ \ell $ ha $ 2\ell + 1 $ subshells degeneri a causa di $ m $, con un ulteriore fattore $ 2 $ dovuto ad $ s $; dunque, fissato il livello energetico $ n $, il numero di elettroni che possono occupare tale livello è:
\begin{equation}
	Z_n = \sum_{\ell = 0}^{n - 1} 2(2\ell + 1) = 2n^2
	\label{eq:5.2}
\end{equation}
Ordinando gli orbitali $ n\ell $ in base alla loro energia e seguendo la \textit{regola dell'Aufbau} (o regola $ n + \ell $), si trovano le shell closures in corrispondenza di $ Z = 2,10,18,36,\dots $, ovvero quelli che nel sistema periodico vengono chiamati elementi nobili: questi sono caratterizzati da un momento angolare totale nullo $ J = 0 $, un'elevata energia di legame ed una bassa reattività.

\begin{figure}[!t]
	\centering
	\includegraphics[width=0.45\textwidth]{atom-rad.png}
	\includegraphics[width=0.52\textwidth]{atomic-ion-en.png}
	\caption{Atomic radius and ionization energy as functions of atomic mass number.}
	\label{atom-prop}
\end{figure}

\subsection{Evidenze sperimentali}

La semplicità del modello atomico a shell non è così immediata da traslare alla fisica nucleare: infatti, mentre gli elettroni nell'atomo sono soggette al campo coulombiano generato dal nucleo atomico, nel nucleo i nucleoni sono soggetti ad un campo che essi stessi generano, rendendo la trattazione analitica estremamente più complicata; inoltre, i nucleoni hanno un raggio relativamente grande rispetto al raggio del nucleo, a differenza degli elettroni che possono essere approssimati come point particles all'interno dell'atomo: ciò rende il concetto di orbite spaziali ben definite difficile da applicare al nucleo, dove il moto dei nucleoni può essere disturbato da urti con altri nucleoni.\\
Nonostante ciò, ci sono varie evidenze sperimentali che suggeriscono l'esistenza di shell nucleari.\\
Innanzitutto, se si plotta la binding energy per nucleone (Fig. \ref{magic-bind-en}), si notano delle consistenti deviazioni dalla formula di Weizsäcker per il liquid drop model in corrispodenza di precisi valori (uguali) di $ N $ e $ Z $, detti magic numbers: 2, 8, 20, 28, 50, 82, 126. Ciò significa che i nuclidi con determinati numeri di neutroni e/o protoni sono più legati, e dunque più stabili, dei nuclidi ad essi adiacenti. Per completezza, si ricordi che il liquid drop model non si applica per $ A < 20 $, poiché tali nuclidi presentano una struttura nucleare più complessa.

\begin{figure}[!t]
	\centering
	\includegraphics[width=0.60\textwidth]{magic-bind-en.png}
	\caption{Binding energy per nucleon, with $ N $ (dark grey) and $ Z $ (light grey) highlighted.}
	\label{magic-bind-en}
\end{figure}

\begin{figure}[!ht]
	\centering
	\includegraphics[width=0.60\textwidth]{2-n-b-e.png}
	\caption{2-neutron binding energy as a function of $ N $ for various isotopic chains.}
	\label{2-n-b-e}
\end{figure}

\begin{figure}[!t]
	\centering
	\includegraphics[width=0.60\textwidth]{magic-n-cap.png}
	\caption{$ n $-capture cross-section ad a funzion of.}
	\label{n-cap-cr}
\end{figure}

Gli stessi magic numbers si notano studiando la cosiddetta \textit{2-neutron binding energy}, ovvero la quantità $ S_{2n}(N,Z) \equiv B(N,Z) - B(N-2,Z) $, come si può vedere in Fig. \ref{2-n-b-e}: ciò può essere interpretato come una maggiore stabilità data dal completamento di shell neutroniche.\\
Anche l'abbondanza di isotopi stabili nel sistema solare mostra dei picchi in corrispondenza dei magic numbers, evidenziando la particolare stabilità dei nuclidi con shell closures.\\
Inoltre, la cross-section per la cattura neutronica subisce delle riduzioni di un paio di ordini di grandezza in corrispondenza dei magic numbers (Fig. \ref{n-cap-cr}: ciò è sia conseguenza diretta delle shell closures, sia conseguenza della riduzione del raggio nucleare determinato dalle suddette (dato che $ \sigma \sim \pi R^2 $).
Infine, per nuclei vicini ai magic numbers si osservano degli stati eccitati relativamente long-lived ($ \tau > 10^{-6}\,\text{s} $) detti \textit{isomeri}: i magic numbers determinano delle cosiddette $ \virgolette{islands of isomerism} $.\\
Si deve aggiungere poi che il liquid drop model non consente una descrizione di vari proprietà nucleari come lo spin, la parità, i momenti magnetici etc. Inoltre, esso non predice la densità nucleare e tutti i suoi coefficienti sono puramente empirici.

\subsection{Modello a particelle indipendenti}

Il primo modello a shell in grado di riprodurre i magic numbers e le proprietà dei magic nuclei fu formulato nel 1949 da Mayer e Jensen (entrambi Nobel nel 1963).\\
Il cosiddetto extreme single-particle shell model, detto anche shell model a particelle indipendenti, ha come assunzione di base che i nucleoni possano muoversi nel nucleo per la maggior parte del tempo senza interagire con altri nucleoni: ciò equivale a dire che il libero cammino medio di un nucleone è maggiore delle dimensioni del nucleo. Ovviamente questa è solo una semplificazione, dato che l'interazione tra nucleoni ha effetti importanti.\\
Rispetto allo shell model atomico, quello nucleare deve tener conto di alcune complicazioni: innanzitutto la forma del potenziale non è nota e, qualora lo si volesse assumere centrale, non ha avrebbe un centro definito poiché ciascun nucleone è sorgente del campo; inoltre, i nucleoni occupano il  nucleo in modo continuo, dunque non è ovvio come estendere il concetto di orbitale in questo contesto. Quest'ultima difficoltà è in parte ridotta dal principio d'esclusione di Pauli e dal modello a gas di Fermi\footnote{Un gas di Fermi è un modello ideale di un ensamble di molti fermioni non-interagenti in equilibrio termico, descritto dalla statistica di Fermi-Dirac.}: se il gas di nucleoni è fortemente degenere, ciascun nucleone è in uno stato quantistico e non scattera con un altro nucleone se non con un meccanismo di scambio, il che permette di impostare un'equazione del moto per il singolo nucleone (da qui il nome di modello a particelle indipendenti).\\
Per rendere l'equazione di Schrödinger risolvibile per un singolo nucleone, si applica l'\textit{approssimazione di campo medio}:
\begin{equation*}
	\hat{\mathcal{H}} = \sum_{j = 1}^{A} \frac{\hat{p}_j^2}{2m_j} + \sum_{j \neq k}^{A} \hat{V}_k(r_j) = \sum_{j = 1}^{A} \left[ \frac{\hat{p}_j^2}{2m_j} + \hat{V}(r_j) \right] + \sum_{j = 1}^{A} \underbrace{\left[ \sum_{k \neq j}^{A} \hat{V}_k(r_j) - \hat{V}(r_j) \right]}_{0}
\end{equation*}
dove l'approssimazione consiste nell'ultima semplificazione. Ciò permette di sostituire il termine d'interazione (che è a 2 corpi solo in prima approssimazione) con un potenziale centrale medio, ignorando l'interazione residua. Dato che il potenziale ha simmetria sferica, è possibile separare la funzione d'onda come:
\begin{equation}
	\psi(\ve{r}) = \frac{u_{\ell}}{r} Y_{\ell,m}(\theta,\phi) X_s
	\label{eq:5.3}
\end{equation}
recuperando così gli ususali numeri quantici. Rimane da determinare il potenziale; i tre potenziali usuali sono quello coulombiano, la buca di potenziale e quello armonico: il potenziale coulombiano non è adatto poiché l'interazione nucleare è a breve raggio e poiché non può riprodurre la saturazione delle forze nucleari; la buca di potenziale non è realistica poiché non può riprodurre l'energia cinetica e potenziale dei nucleoni; il potenziale armonico non va a 0 all'infinito, dunque non potrebbe descrivere la fuoriuscita di nucleoni dal nucleo, cosa che avviene se ad esempio viene aggiunto un nucleone ad un nuclide debolmente legato. Il potenziale che meglio rappresenta l'interazione nucleare è il \textit{potenziale di Woods-Saxon}:
\begin{equation}
	V_{\text{WS}}(r) = \frac{-V_0}{1 + e^{(r - R_0) / a}}
	\label{eq:5.4}
\end{equation}
dove $ V_0 \approx 50\mev $ (verso le driplines varia drasticamente), $ R_0 \approx 1.25\fm \cdot A^{1/3} $ è il raggio nucleare medio e $ a \approx 0.524\fm $ è la skin thickness del nucleo (distanza in cui $ V(r) $ passa da $ 0.9 V_0 $ a $ 0.1 V_0 $); esso è plottato in Fig. \ref{woods-saxon}.

\begin{figure}[!b]
	\centering
	\includegraphics[width=0.70\textwidth]{woods-saxon.png}
	\caption{Woods-Saxon potential.}
	\label{woods-saxon}
\end{figure}

Il potenziale di Woods-Saxon prevede sia gli stati legati, con energia negativa, sia gli stati del continuo, con energia positiva. Gli stati sono giustamente quantizzati con tre numeri quantici: quello principale $ n $, quello angolare orbitale $ \ell $ e quello angolare totale $ j $ (relativo a $ \ve{J} = \ve{L} + \ve{S} $). Gli orbitali nucleari vengono quindi identificati come $ n\ell_j $, dove $ \ell $ è indicato con una lettera: $ s $ per $ \ell = 0 $, $ p $ per $ \ell = 1 $, $ d $ per $ \ell = 2 $, $ f $ per $ \ell = 3 $ e poi in ordine alfabetico.\\
Come si può vedere in Fig. \ref{ws-sp}, il solo potenziale di Woods-Saxon predice correttamente soltanto i primi tre magic numbers. È quindi necessario modificare tale potenziale con un termine di nuovo mutuato dalla fisica atomica: un termine d'interazione spin-orbita. Nel caso atomico, l'interazione spin-orbita avviene poiché il momento magnetico dell'elettrone interagisce col campo magnetico generato dal suo moto attorno al nucleo; nel caso nucleare, essa non deriva dall'interazione elettromagnetica ma dalla natura spin-dependent dell'interazione tra nucleoni.\\
Ricordando l'Eq. \ref{eq:4.20}, si può scrivere il \textit{potenziale d'interazione spin-orbita} come:
\begin{equation}
	V_{\ell,s}(r) \ve{L}\cdot\ve{S} = - \lambda \frac{1}{r} \frac{dV_{\text{WS}}}{dr} \ve{L}\cdot\ve{S}
	\label{eq:5.5}
\end{equation}
dove $ \lambda > 0 $ non ha particolare interesse fisico. Questo termine deriva dalla descrizione relativistica di un singolo nucleone nel nucleo ed è particolarmente importante sulla superficie nucleare (quando $ V_{\text{WS}}(r) $ varia maggiormente).

\begin{figure}[!b]
	\centering
	\includegraphics[width=0.60\textwidth]{ws-spectrum.png}
	\caption{Energy levels for the Woods-Saxon potential, without and with spin-orbit interaction.}
	\label{ws-sp}
\end{figure}

Si può vedere come l'aggiunta di questo termine splitti gli orbitali in modo da ottenere i corretti magic numbers. Ricordando che un nucleone ha $ s = \frac{1}{2} $, si vede che i possibili valori per $ j $ sono $ j = \ell + \frac{1}{2} $ e $ j = \ell - \frac{1}{2} $; inoltre, si può lavorare sul termine $ \ve{L}\cdot\ve{S} $:
\begin{equation*}
	\braket{\ve{L}\cdot\ve{S}} = \frac{1}{2} \braket{\ve{J}^2 - \ve{L}^2 - \ve{S}^2} = \frac{\hbar^2}{2} \left[ j (j + 1) - \ell (\ell + 1) - s (s + 1) \right]
\end{equation*}
A questo punto, si deve considerare che i due possibili valori di $ j $ per ogni $ \ell $ portano ad uno split degli orbitali considerati: ad esempio, un orbitale $ 1f $ (con $ \ell = 3 $) viene splittato in $ 1f_{5/2} $ e $ 1f_{7/2} $. Ciascun orbitale così ottenuto ha degenerazione $ 2j + 1 $ determinata da $ m_j $ (con l'interazione spin-orbita, $ m_{\ell} $ ed $ m_s $ non vanno più bene come numeri quantici poiché non sono ben definiti): in questo modo si ottiene comunque la corretta degenerazione totale, poiché $ 2(\ell + \frac{1}{2}) + 1 + 2(\ell - \frac{1}{2}) + 1 = 2(2\ell + 1) $, solo che viene redistribuita asimmetricamente tra due orbitali.\\
Questo splitting porta ad uno splitting anche energetico proporzionale ad $ \ell $, poiché:
\begin{equation}
	\braket{\ve{L}\cdot\ve{S}}_{j = \ell + \frac{1}{2}} - \braket{\ve{L}\cdot\ve{S}}_{j = \ell - \frac{1}{2}} = \frac{\hbar^2}{2} (2\ell + 1)
	\label{eq:5.6}
\end{equation}
Dato che $ V_{\ell,s}(r) < 0 $ (dall'Eq. \ref{eq:5.5}), l'orbitale con $ j $ maggiore ha un'energia più bassa: come si vede in Fig. \ref{ws-sp}, ciò permette di ottenere i corretti magic numbers ed anche di predirne uno nuovo ancora mai osservato: 184.\\
Un'importante conseguenza di questo splitting è che orbitali con diverso $ n $ ed $ \ell $ vengono scambiati di posto nella ladder degli stati: di conseguenza, dato che la parità di un orbitale è $ \pi = (-1)^{\ell} $, orbitali di parità diversa vengono mescolati nella ladder; inoltre, dato che l'interazione forte preserva la parità, orbitali di parità diversa sono ben distinti tra loro e possono essere considerati degli stati puri.

\subsubsection{Nucleoni di valenza}

Il modello a shell, nella sua versione più semplice, ascrive tutte le proprietà di un nuclide dispari (ovvero con $ A $ dispari) al singolo nucleone spaiato nella shell più esterna: se esso si trova nell'orbitale $ n\ell_j $, il ground state del nuclide avrà spin $ j $ e parità $ (-1)^{\ell} $.\\
Nonostante l'estrema semplicità, questo modello predice correttamente il $ J^{\pi} $ di praticamente tutti i nuclidi dispari nel range di masse in cui il modello a shell è valido (ossia $ A < 150 $ e $ 190 < A < 220 $).
Un calcolo più preciso deve ovviamente tener conto almeno di tutti i nucleoni sulla shell di valenza: in tal caso, ciascuno stato eccitato può essere ottenuto mediante varie eccitazioni di nucleoni diversi (sovrapposizione di stati), dette particle-hole eccitations: passando da una shell di energia minore ad una di energie maggiore, il nucleone lascia un buco nella shell minore e questo processo richiede una notevole energia, specialmente se nei pressi degli shell gaps. Anche in questo caso, si ha un buon accordo per nuclidi dispari, specialmente per quelli esprimibili come un doubly magic nuclei + uno o più (pochi) nucleoni (es.: elio, ossigeno, calcio, etc.).\\
In generale (quindi anche per nuclidi pari), ogni orbitale completamente pieno di nucleoni non contribuisce allo spin nucleare, dato che essi si accoppiano in coppie di nucleoni identici con spin opposti (configurazione energeticamente più favorevole).

\subsubsection{Momento magnetico}

Dall'Eq. \ref{eq:4.14}, considerando un $ g_{\ell} $ generico (e non il caso specifico $ g_{\ell} = \frac{1}{2} $), si ha:
\begin{equation}
	g_j = \frac{1}{2} (g_{\ell} + g_s) + \frac{1}{2} \frac{\ell (\ell + 1) - s (s + 1)}{j (j + 1)} (g_{\ell} - g_s)
	\label{eq:5.7}
\end{equation}
I momenti magnetici nucleari possono dunque essere calcolati come:
\begin{equation}
	\braket{\mu} =
	\begin{cases}
		\left[ g_{\ell} (j - \frac{1}{2}) + \frac{1}{2} g_s \right] \mu_N & j = \ell + \frac{1}{2} \\
		\frac{j}{j + 1} \left[ g_{\ell} (j + \frac{3}{2}) - \frac{1}{2} g_s \right] \mu_N & j = \ell - \frac{1}{2}
	\end{cases}
	\label{eq:5.8}
\end{equation}
Queste sono le cosiddette \textit{linee di Schmidt} e, se moltiplicate per un fattore di scala di circa $ 0.60 $, riproducono il trend seguito dai dati sperimentali, come si può vedere in Fig. \ref{schmidt}.

\begin{figure}[!t]
	\includegraphics[width=0.60\textwidth]{schmidt-1.png}
	\includegraphics[angle=90,width=0.35\textwidth]{schmidt-2.png}
	\caption{Schmidt lines.}
	\label{schmidt}
\end{figure}

Lo scatter dei dati rispetto alle linee di Schmidt indica che il calcolo del momento magnetico considerando i nucleoni come indipendenti è una semplificazione eccessiva; per un calcolo più preciso, si dovrebbero considerare degli effetti non-banali come l'influenza reciproca dei nucleoni, il fatto che lo spin pairing non sia perfetto, le nubi mesoniche che circondano i nucleoni (che contengono pioni $ \pi^{\pm} $ carichi) e la non sfericità dei nuclei.

\subsubsection{Interazione residua}

Per dare una descrizione unificata di tutti i nuclidi, è necessario aggiungere allo shell model la descrizione dell'interazione residua che viene ignorata nel modello. Ad esempio, per nuclei leggeri è necessario considerare interazioni nucleari a tre o più corpi, problemi difficoltosi da trattare in QCD.\\
Ad oggi, un modo computazionalmente efficiente di calcolare l'interazione residua è dividere il nuclide in un core composto dal più vicino doubly-magic nuclei ($ \ch{^4He} $, $ \ch{^{16}O} $, $ \ch{^{40}Ca} $, $ \ch{^{132}Sn} $ e $ \ch{^{208}Pb} $), il quale rimane freezato e non viene considerato nelle interazioni, e le rimanenti shell d'interazione. Dato che il numero di modi di distribuire $ k $ nucleoni su $ n $ orbitali è pari a $ \binom{n}{k} $, che aumenta fattorialmente all'aumentare dei nucleoni, per snellire ulteriormente il calcolo dal punto di vista computazionale si può ulterioremente restringere l'insieme dei nucleoni sui quali calcolare l'interazione residua alla sola shell di valenza.\\
Anche in questo caso ci sono però delle limitazioni, prime su tutti il considerare solo i nucleoni di valenza e l'ignorare completamente le eccitazioni del core.\\
I principali filoni di ricerca a livello computazionale riguardano da un lato la risoluzione numerica della many-body hamiltonian, dall'altro lo studio delle interazioni nucleari.












\chapter{Reazioni Nucleari}
\pagestyle{body}
\selectlanguage{italian}

\section{Proprietà generali}

La forma tipica di una reazione nucleare è la seguente:
\begin{equation*}
	\mathrm{a} + \mathrm{X} \rightarrow \mathrm{Y} + \mathrm{b}
\end{equation*}
dove $ \mathrm{X} $ e $ \mathrm{Y} $ sono nuclidi target-like, $ \mathrm{a} $ e $ \mathrm{b} $ projectile-like. Una scrittura alternativa è: $ \mathrm{X} \left( \mathrm{a},\mathrm{b} \right) \mathrm{Y} $.\\
In linea generale, le reazioni nucleari soddisfano alcune leggi di conservazione, sebbene alcune di esse non in maniera sempre esatta: energia totale, momento lineare totale, momento angolare totale, carica elettrica totale, parità, numero atomico. In base all'energia per nucleone possono presentarsi comportamenti aggiuntivi:
\begin{enumerate}
	\item low-energy ($ E \lesssim 10\mev $ per nucleone): non ci sono processi legati all'interazione forte, dunque si conservano anche il numero di protoni e quello di neutroni;
	\item medium-energy ($ 100\mev \lesssim E \lesssim 1\gev $ per nucleone): subentrano processi di meson production, dunque protoni e neutroni possono trasformarsi gli uni negli altri;
	\item high-energy ($ E \gtrsim 10\gev $ per nucleone); possono essere prodotte molte particelle esotiche, arrivando a riarrangiare anche i quarks nei nucleoni.
\end{enumerate}
Inoltre, dato che affinché avvenga la reazione si deve superare la barriera coulombiana nei nuclei, è necessario che la reazione abbia un certo $ Q $-value. Dalla conservazione dell'energia totale si ha:
\begin{equation*}
	M_{\mathrm{a}} c^2 + T_{\mathrm{a}} + M_{\mathrm{X}} c^2 T_{\mathrm{X}} = M_{\mathrm{Y}} c^2 + T_{\mathrm{Y}} + M_{\mathrm{b}} c^2 + T_{\mathrm{b}}
	\quad \Rightarrow \quad
	Q = M_{\mathrm{a}} c^2 + M_{\mathrm{X}} c^2 - \left( M_{\mathrm{Y}} c^2 + M_{\mathrm{b}} c^2 \right)
\end{equation*}
Se $ Q > 0 $ si parla di \textit{reazione esoergonica} (o esotermica), nella quale parte della massa nucleare o della binding energy viene liberata sotto forma di energia cinetica, mentre se $ Q < 0 $ di \textit{reazione endoergonica} (o endotermica), nella quale parte dell'energia cinetica è convertita in massa nucleare o binding energy. Nel caso in cui $ Q = 0 $ si parla di \textit{collisione elastica}.\\
Una reazione è sempre possibile se $ Q \ge 0 $, mentre se $ Q < 0 $ è necessario che $ T_{\mathrm{a}} $ superi una certa threshold energy:
\begin{equation}
	T_{\text{th}} = -Q \frac{M_{\mathrm{Y}} + M_{\mathrm{b}}}{M_{\mathrm{Y}} + M_{\mathrm{b}} - M_{\mathrm{a}}}
	\label{eq:6.1}
\end{equation}
Dalla conservazione dell'impulso, invece, considerando il laboratory frame in cui il target $ \mathrm{X} $ è a riposo e definendo $ \theta $ e $ \xi $ gli angoli tra i momenti di $ \mathrm{Y} $ e $ \mathrm{b} $ e quello di $ \mathrm{a} $, si ha:
\begin{equation*}
	\begin{cases}
		p_{\mathrm{a}} = p_{\mathrm{b}} \cos \theta + p_{\mathrm{Y}} \cos \xi \\
		0 = p_{\mathrm{b}} \sin \theta - p_{\mathrm{Y}} \sin \xi
	\end{cases}
\end{equation*}
Si hanno tre equazioni in quattro incognite ($ T_{\mathrm{b}}, T_{\mathrm{Y}}, \theta, \xi $), dunque non c'è soluzione unica. Dato che è più facile osservare $ \mathrm{b} $ rispetto ad $ \mathrm{Y} $, dato che quest'ultimo può rimanere all'interno dello strato target, conviene eliminare le osservabili relative ad $ \mathrm{Y} $, trovando:
\begin{equation}
	\sqrt{T_{\mathrm{b}}} = \frac{1}{M_{\mathrm{Y}} + M_{\mathrm{b}}} \left[ \sqrt{M_{\mathrm{a}} M_{\mathrm{b}} T_{\mathrm{a}}} \cos \theta \pm \sqrt{M_{\mathrm{a}} M_{\mathrm{b}} T_{\mathrm{a}} \cos^2 \theta + \left( M_{\mathrm{Y}} + M_{\mathrm{b}} \right) \left( M_{\mathrm{Y}} Q + (M_{\mathrm{Y}} - M_{\mathrm{a}}) T_{\mathrm{a}} \right)} \right]
	\label{eq:6.2}
\end{equation}
Si può quindi calcolare $ T_{\mathrm{b}} $ in base alla misura dell'angolo $ \theta $.\\
In Fig. \ref{reac-graph} è riportato il plot di $ T_{\mathrm{b}} $ rispetto a $ T_{\mathrm{a}} $ a vari angoli $ \theta $ per la reazione $ \ch{^3H} \left( p,n \right) \ch{^3He} $, la quale ha $ Q = -763.75\kev $: si può notare che quasi ovunque c'è una corrispondenza biunivoca tra $ \theta $ e $ T_{\mathrm{b}} $; fa eccezione una regione tra $ 1.019\mev $ e $ 1.147\mev $, in cui invece per ogni valore di $ \theta $ sono possibili due valori di $ T_{\mathrm{b}} $. In generale, questa double-valued region è presente solo in reazioni con $ Q < 0 $ per $ T_{\text{th}} < T_{\mathrm{a}} < T_{\text{d}} $, con:
\begin{equation}
	T_{\text{d}} = - Q \frac{M_{\mathrm{Y}}}{M_{\mathrm{Y}} - M_{\mathrm{a}}}
	\label{eq:6.3}
\end{equation}

\begin{figure}[!b]
	\centering
	\includegraphics[width=0.70\textwidth]{reac-graph.png}
	\caption{$ T_{\mathrm{b}} $ vs $ T_{\mathrm{b}} $ at various outgoing angles for $ \ch{^3H} \left( p,n \right) \ch{^3He} $.}
	\label{reac-graph}
\end{figure}

\paragraph{Stati eccitati}

È possibile che il nuclide $ \mathrm{Y} $ sia prodotto in uno stato eccitato $ \mathrm{Y}^* $; in tal caso, si vede che il $ Q $-value diminuisce a causa dell'excitation energy $ E_{\text{ex}} $:
\begin{equation}
	Q = Q_0 - E_{\text{ex}}
	\label{eq:6.4}
\end{equation}
dove $ Q_0 $ è relativo alla reazione che produce $ \mathrm{Y} $ nel ground state. È quindi possibile ricostruire lo spettro della specie $ \mathrm{Y} $ calcolando il $ Q $-value a partire da $ T_{\mathrm{a}} $, $ \theta $ (fissati) e $ T_{\mathrm{b}} $ (misurato), il che è possibile invertendo l'Eq. \ref{eq:6.2}:
\begin{equation}
	Q = \left( 1 + \frac{M_{\mathrm{b}}}{M_{\mathrm{Y}}} \right) T_{\mathrm{b}} - \left( 1 - \frac{M_{\mathrm{a}}}{M_{\mathrm{Y}}} \right) T_{\mathrm{a}} - 2 \sqrt{\frac{M_{\mathrm{a}}}{M_{\mathrm{Y}}} \frac{M_{\mathrm{b}}}{M_{\mathrm{Y}}} T_{\mathrm{a}} T_{\mathrm{b}}} \cos \theta
	\label{eq:6.5}
\end{equation}
Solitamente si raggiunge una sufficiente accuratezza utilizzando i numeri atomici al posto delle masse nucleari, specialmente per $ \theta \approx \frac{\pi}{2} $ (si annulla l'ultimo termine).

\paragraph{Cross-section}

Si consideri un projectile beam d'intensità $ \Phi_0 $ (particelle al secondo) diretto su uno strato sottile di nuclei target con spessore $ s $: a causa delle interazioni coi targets, il projectile beam risulta attenuato a seguito del passaggio nello strato sottile, con un'intensità risultante $ \Phi_{\text{f}} $. La frazione di particelle incidenti che reagiscono è funzione della densità numerica di targets $ n_{\text{t}} $ e dalla cross-section $ \sigma $ della reazione:
\begin{equation}
	d\Phi = - \Phi n_{\text{t}} \sigma dx
	\quad \Rightarrow \quad
	\Phi_0 - \Phi_{\text{f}} = \Phi_0 \left( 1 - e^{- n_{\text{t}} \sigma s} \right) \approx \Phi_0 n_{\text{t}} \sigma s
	\label{eq:6.6}
\end{equation}

\paragraph{Trattazione semiclassica}

Classicamente, si può pensare ai nuclidi come sfere rigide che urtano, con vari risultati a seconda del parametro d'impatto $ b $: per $ b \approx R_{\mathrm{X}} $ si ha un urto elastico, per $ b \ll R_{\mathrm{X}} $ ci può essere uno scambio di nucleoni tra i nuclidi, per $ b \approx 0 $ si può arrivare alla formazione di un unico nucleo composto dal nucleo target e da quello incidente.

\section{Reazioni di diffusione e scattering}

Si dicono reazioni di diffusione o di scattering quelle reazioni in cui $ \mathrm{b} = \mathrm{a} $ (o suoi stati eccitati). Si parla di \textit{scattering elastico} per le reazioni $ \mathrm{A} \left( \mathrm{a},\mathrm{a} \right) \mathrm{A} $, di \textit{scattering inelastico} negli altri casi.













\part{Fisica Subnucleare}
\pagestyle{body}

\chapter{Fisica delle Particelle}
\selectlanguage{italian}

La fisica delle particelle studia i costituenti fondamentali della materia e le interazioni fra loro. La materia è costituita da particelle e campi, i quali sono associati alle interazioni e causano le forze attraverso le quali interagiscono le particelle: le interazioni sono mediate da particolari particelle, dette \textit{bosoni di gauge}. Le particelle fondamentali, come i quark e l'elettrone, possono essere considerati puntiformi ($ \lesssim 10^{-16}\,\text{cm} $, mentre un nuclide $ \sim 10^{-13}\,\text{cm} $).

\paragraph{Particelle}

Solo quattro particelle sono stabili, di cui sono due sono fondamentali: il protone, il neutrone, l'elettrone ed il neutrino. Le altre particelle decadono rapidamente verso le particelle stabili, ed infatti sono tendenzialmente più massive di quest'ultime. A causa dell'instabilità, queste particelle non si trovano in natura, ma possono essere prodotte sperimentalmente usando fasci di particelle incidenti con sufficiente energia (superiore alla rest energy delle particelle da produrre).\\
Attualmente, le particelle fondamentali più massive scoperte sono il bosone $ W $ ($ 80.4\gev/c^2 $), il bosone $ Z $ ($ 91.2\gev/c^2 $), il bosone di Higgs ($ 125\gev/c^2 $) ed il quark top ($ 173\gev/c^2 $). Ciascuna di queste particelle è circa un centinaio di volte più pesante del protone; per produrle sono necessarie altissime energie, mentre per studiarle è necessario sondare distanze piccolissime: dal principio di Heisenberg, se $ \Delta x \ll 1\fm $, allora $ \Delta p \gg 200\mev/c $.

\paragraph{Interazioni}

Tutte le interazioni tra particelle sono dovute allo scambio di particelle mediatrici. Ad oggi si conoscono solo quattro interazioni fondamentali: quella gravitazionale, quella debole, quella elettromagnetica e quella forte. Si pensa, però, che nelle prime frazioni di secondo dell'universo queste fossero unificate (almeno quelle forte, debole ed elettromagnetica): si parla di \textit{Grand Unification Theory} (GUT), secondo la quale, col trascorrere del tempo ed il diminuire della temperatura dell'universo, le interazioni si siano gradualmente separate, fino ad avere quelle che osserviamo oggi. Le predizioni della GUT, però, rimangono tuttora inosservate sperimentalmente: le principali sono il decadimento del protone su scale lunghissime, l'esistenza di monopoli magnetici e l'esistenza di dipoli elettrici fondamentali, ovvero particelle che, sebbene puntiformi, presentano un'asimmetria di carica.

\paragraph{Modello Standard}

All'inizio del Novecento si avevano evidenze sperimentali soltanto dell'elettrone, del fotone e dei nuclei atomici, ma nel giro di un secolo si è arrivati a sviluppare un modello che spiegasse la formazione di tutte le particelle della materia a partire da quelle fondamentali: il \textit{Modello Standard}. Questo è appunto un modello, e non una teoria: mentre la seconda dà spiegazioni esatte dei dati sperimentali e ne predice di nuovi, un modello è soltanto un'approssimazione desunta dagli esperimenti, dunque deve essere integrato man mano che questi procedono. Il Modello Standard è uno dei modelli più solidi mai sviluppati.

\section{Antimateria}

Una delle caratteristiche fondamentali delle particelle, lo spin (momento angolare intrinseco), permette di distinguerle in due classi: i bosoni, con spin intero, ed i fermioni, con spin semi-intero. I fermioni rispondono al principio d'esclusione di Pauli, il quale postula che non ci possono essere due fermioni occupanti lo stesso stato quantistico, mentre ciò è possibile per i bosoni: ciò dà luce a fenomeni particolari come la superconduttività e la superliquidità.\\
Se si ignora lo spin, l'equazione di Schrödinger può essere facilmente espressa in forma relativistica; partendo dall'energia $ E^2 = m^2 c^4 + p^2 c^2 $ e ricordando le regole di quantizzazione $ E \mapsto i\hbar \frac{\pa}{\pa t}, p \mapsto -i\hbar \nabla $, si ottiene l'\textit{equazione di Klein-Gordon}:
\begin{equation}
	\frac{1}{c^2} \frac{\pa^2}{\pa t^2} \phi(\ve{x},t) - \lap \phi(\ve{x},t) + \frac{m^2 c^2}{\hbar^2} \phi(\ve{x},t) = 0
	\label{eq:7.1}
\end{equation}
dove $ \phi(\ve{x},t) $ è un campo scalare. Per trattare particelle dotate di spin, è necessario trovare altre equazioni. Per particelle con $ s = \frac{1}{2} $, descritte da uno spinore $ \psi $ a quattro componenti, si trova l'\textit{equazione di Dirac}:
\begin{equation}
	i \gamma^{\mu} \pa_{\mu} \psi - \frac{mc}{\hbar} \psi = 0
	\label{eq:7.2}
\end{equation}
dove le matrici $ \gamma $ sono legate alle matrici di Pauli:
\begin{equation*}
	\gamma^0 \equiv
	\begin{bmatrix}
		\tens{I}_2 & 0 \\ 0 & \tens{I}_2
	\end{bmatrix}
	\qquad \qquad
	\gamma^k \equiv
	\begin{bmatrix}
		0 & \sigma_k \\ -\sigma_k & 0
	\end{bmatrix}
\end{equation*}
La prima grande conseguenza dell'equazione di Dirac è che, risolvendola per elettroni liberi, si trovano due soluzioni: una con energia positiva ed una con energia negativa. Dato che nel mondo reale tutte le energie sono positive, essendoci un'energia minima per le particelle costituita dalla loro massa a riposo, Dirac spiegò la soluzione ad energia negativa tramite l'idea del \textit{Fermi sea}: oltre agli stati ad energia positiva, esistono anche stati speculari ad energia negativa che sono normalmente tutti occupati, così che le particelle reali possano avere solo energie positive. Può capitare però che, con eccitazioni sufficientemente energetiche, una particella nel mare di Fermi venga portata in uno stato ad energia positiva: il vuoto lasciato nel mare di Fermi è interpretato come una antiparticella, ovvero come una copia della particella reale ma con carica elettrica opposta. Questo è proprio il principio della pair production: un fotone sufficientemente energetico ($ E_{\gamma} > 2m_e = 1022\kev $) può fornire l'eccitazione sufficiente a produrre una coppia elettrone-positrone.\\
Un'interpretazione diversa dell'antimateria sarà poi data da Feynman: le antiparticelle non sono più viste come buchi nel Fermi sea, ma come particelle che si muovono in maniera opposta lungo la direzione temporale: ciò comporta che le antiparticelle hanno la stessa massa delle particelle, ma carica elettrica opposta.

\subsection{Evidenze sperimentali}

Le prime evidenze sperimentali dell'esistenza dell'antimateria si ebbero con lastre fotografiche che, mostrando tracce di ionizzazione, rilevavano fenomeni di pair production.

\paragraph{Positrone}

Il positrone fu osservato per la prima volta da Skobeltsyn nel 1929 durante lo studio di raggi cosmici in una cloud chamber e nel 1932 da Anderson con lo stesso metodo. La scoperta fu confermata nel 1932 da Blacket ed Occhialini in laboratorio. A posteriori, altri sperimentatori in precedenza si sarebbero potuti accorgere della sua esistenza: ad esempio, i coniugi Curie lo avevano osservato, ma senza dargli attenzione lo avevano classificato come un protone.\\
Il positrone è una particella stabile come l'elettrone, ma la sua vita media è comunque estremamente breve ($ \tau \sim 10^{-10}\,\text{s} $) poiché forma immediatamente un sistema con un elettrone nella materia, detto positronio, il quale decade per annichilazione.\\
Un'importante applicazione del positrone è la \textit{Positron-Electron Tomography} (PET), un esame diagnostico per tumori: dato che il positronio decade in due fotoni con $ E_{\gamma} = 511\kev $ emessi in direzioni opposte, facendo accumulare una concentrazione abbastanza alta di un positron-emitting radionuclide in una massa tumorale, è possibile tracciarne con precisione la posizione nel corpo, dato che essa deve trovarsi lungo la linea che congiunge i due fotoni rilevati. Un metodo utilizzato è l'accumulo di $ \ch{^{18}F} $ (decade $ \beta^+ $) misto a glucosio: le cellule tumorali hanno un consumo di zuccheri dieci volte superiore alle cellule normali.

\paragraph{Antiprotone}

L'antiprotone fu osservato nel 1955 al Bevatron, un sincrotrone per protoni a Berkley, da Chamberlain e Segrè: ciò confermo che ogni particella aveva una corrispondente antiparticella identica ad essa ma di carica elettrica opposta. Il delay di 20 anni tra le due scoperte fu dovuto al fatto che la massa del protone è quasi 2000 volte quella dell'elettrone, dunque anche le energie necessarie alla produzione di antiprotoni lo sono.\\
Anche gli antiprotoni sono stabili ma short-lived, sempre a causa dell'annichilimento da contatto con materia ordinaria.\\
Poco dopo la conferma dell'esistenza dell'antiprotone, nel 1956 sempre al Bevatron fu osservato l'antineutrone.












\chapter{Leptoni}
\selectlanguage{italian}

I leptoni sono le particelle fondamentali più leggere, se comparate ai sistemi formati da quarks (mesoni e barioni); fa eccezione il tauone ($ 1.78\gev/c^2 $). Sono divisi in tre generazioni, detti anche flavours, ciascuna composta da una particella di carica elettrica $ q = -1e $ ed un neutrino elettricamente neutro; inoltre, i leptoni non sono soggeti all'interazione forte.

\section{Raggi cosmici}

I \textit{raggi cosmici} (o astroparticelle) sono particelle o gruppi di particelle ad alte energie che si muovono nello spazio a velocità relativistiche: principalmente si tratta di protoni e nuclei atomici. Essi possono avere origine solare, galattica o anche extragalattica.\\
All'impatto con l'atmosfera, i raggi cosmici producono delle \textit{showers} di patricelle secondarie a causa dell'interazione con le molecole atmosferiche: parte di queste showers riesce ad arrivare sulla superficie terrestre, risultando dunque rilevabile a terra, mentre la maggior parte viene deflessa nello spazio dalla magnetosfera.\\
Lo studio approfondito dei raggi cosmici è importante principalmente perché costituiscono una fonte di pericolo per missioni spaziali ed extraplanetarie: essi causano danni ai sistemi elettronici e biologici non protetti da un'atmosfera o una magnetosfera, dunque sono una complicazione  per missioni lunari e marziane. In maniera non secondaria, gli ultra-high-energy cosmic rays (UHECRs) possono raggiungere energie fino a $ 3\cdot10^{20}\ev $ (massimo finora osservato), di svariati ordini di grandezza superiori a quelle raggiungibili negli accelleratori odierni: si parla infatti di accelleratori cosmici.

\subsection{Osservazioni}

Le prime osservazioni di raggi cosmici furono svolte da Hess et al. nel 1912 con un pallone aerostatico: misurando la radiazione inonizzante in funzione dell'altezza, si scoprì che essa aumentava man mano che si allontanava dalla superficie terrestre. La conclusione fu che la sorgente di questa radiazione doveva essere extraplanetaria: la causa della ionizzazione dell'atmosfera sono appunto i raggi cosmici. La maggior parte della radiazione dovuta ai raggi cosmici è assorbita dall'atmosfera o deflessa nello spazio, ma una parte riesce ad arrivare alla superficie terrestre.

\subsubsection{Primary Cosmic Rays}

I raggi cosmici primari sono quelli che collidono direttamente con l'atmosfera terrestre: sono costituiti principalmente da protoni e raggi $ \alpha $ (nuclei di idrogeno ed elio), ma una frazione relativamente piccola è costituita dai cosiddetti \textit{ioni HZE} (high atomic number and energy), i quali sono nuclei di atomi pesanti ($ Z > 2 $) con cariche elettriche maggiori di $ +3e $. Quest'ultimi sono particolarmente pericolosi per i cosmonauti, poiché contribuiscono in maniera significativa alla loro dose assorbita di radiazione.\\
In Fig. \ref{flux-cr} è riportato il flusso incidente di di raggi cosmici: si può vedere che l'andamento non è liscio ma presenta tre picchi, detti knees ed ankle: si pensa che i due knees siano dovuti al fatto che gli accelleratori di raggi cosmici galattici abbiamo un massimo energetico oltre il quale non possono accellerare particelle, mentre l'ankle dovrebbe corrispondere ad una zona dominata dai raggi cosmici extragalattici, sebbene ci sia ancora molta incertezza su questi dati.

\begin{figure}
	\centering
	\includegraphics[width=0.70\textwidth]{cosmic-rays.png}
	\caption{Flux of cosmic rays.}
	\label{flux-cr}
\end{figure}

\subsubsection{Secondary Cosmic Rays}

Interagendo con le molecole nell'atmosfera, i raggi cosmici primari producono showers di particelle, dette raggi cosmici secondari: questi sono principalmente pioni, ma una frazione minore è composta da kaoni, protoni, neutroni e rispettive antiparticelle. La componente mesonica (pioni e kaoni) decade principalmente producendo muoni e neutrini:

\begin{equation*}
	\pi^0 \rightarrow \gamma + \gamma \qquad \pi^+ \rightarrow \mu^+ + \bar{\nu}_{\mu} \qquad \pi^- \rightarrow \mu^- + \nu_{\mu}
\end{equation*}
\begin{equation*}
	K^{\pm} \rightarrow \pi^{\pm} + \pi^0 \qquad K^+ \rightarrow \pi^+ + \bar{\nu}_{\mu} \qquad K^- \rightarrow \mu^- + \nu_{\mu}
\end{equation*}

Importanti studi sui raggi cosmici secondari sono stati svolti da Bruno Rossi: nel 1932, si accorse della presenza di due componenti nelle showers, un \textit{soft component} ed un \textit{hard component}. Il soft component, circa il 30\% della radiazione secondaria, è costituita dalle electromagnetic showers, dovute principalmente al decadimento del $ \pi^0 $ e costituite da coppie elettrone-positrone e fotoni ad alta energia: il nome è dovuto al fatto che queste particelle vengono bloccate da pochi millimetri di materiale assorbente. L'hard component, invece, costituisce il 70\% della radiazione secondaria ed è costituito dalle hardonic showers: queste comprendono princiaplmente muoni e, se sufficientemente energetici, possono penetrare vari metri di materiale assorbente.

\paragraph{Pierre Auger Observatory}

Il Pierre Auger Observatory è un osservatorio internazionale di raggi cosmici secondari situato in Argentina: il suo obbiettivo è la rilevazione di UHECRs con energie $ \gtrsim 10^18\ev $. Questi hanno un flusso stimato di $ 1\,\text{km}^{-2} / 100\,\text{y} $, dunque è necessaria una vasta area d'osservazione: essendo composto da 200 water tanks con detection area di $ 12\,\text{km}^2 $, il PAO ha una detection area complessiva di oltre $ 3000\,\text{km}^2 $, comparabile alla superficie del Lussemburgo.\\
In particolare, l'osservazione precisa delle showers da parte di numerose water tanks permette di estrapolare informazioni sulla direzione del cosmic ray e sulla posizione approssimativa della sua sorgente.

\subsection{Muoni}

Il muone è stata la prima particella fondamentale instabile scoperta: esso ha $ m_{\mu} = 105.7\mev/c^2 $, $ \tau = 2.196\,\mu\text{s} $, $ q = -1e $ e $ s = \frac{1}{2} $. Inizialmete si era pensato che fosse il mesone di Yukawa, ovvero il mediatore dell'interazione forte (all'epoca detto mesotrone), ma studiando il suo decadimento si notò che non c'erano legami tra il muone ed i nucleoni:
\begin{equation*}
	\mu^- \rightarrow e^- + \bar{\nu}_e + \nu_{\mu}
\end{equation*}
Questo è un esempio di conservazione del numero leptonico: a ciascuna famiglia leptonica è associato un numero leptonico, il quale deve essere conservato nei decadimenti: ai leptoni è associato $ +1 $, mentre agli antileptoni $ -1 $, dunque nel decadimento del muone devono rimanere costanti $ L_e = 0 $ ed $ L_{\mu} = +1 $, ergo la presenza di $ \bar{\nu}_e $ e $ \nu_{\mu} $.\\
Il muone costituisce anche un perfetto esempio di cinematica relativistica: assumendo che vengano prodotti muoni a $ 10\,\text{km} $ dalla superficie terrestre a causa di raggi cosmici e che essi si muovano a velocità $ 0.98c $, essi impiegherebbero $ t = 34\,\mu\text{s} $ per arrivare a terra; data la vita media del muone $ t_{1/2} = 1.56\,\mu\text{s} $, a terra dovrebbe giungere una frazione del flusso iniziale pari a $ 2^{-34/1.56} = 0.27\cdot10^{-6} $, praticamente nulla. Non si sta però contando la dilatazione relativistica dei tempi: per il muone $ \gamma \approx 5 $, dunque la vita media nel suo rest-frame è sempre $ 1.56\,\mu\text{s} $, ma per un osservatore a riposo rispetto alla Terra essa è $ t'_{1/2} = \gamma t_{1/2} = 7.8\,\mu\text{s} $: la frazione di flusso osservata nel frame della Terra è quindi $ 0.049 $, che è quella effettivamente misurata. ALternativamente, si può anche svolgere il calcolo nel rest-frame dei muoni, considerando dunque la contrazione delle lunghezze: nel loro frame, i muoni non percorrono $ 10\,\text{km} $, bensì $ 2\,\text{km} $.\\
I muoni costituiscono la principale fonte di rumore nella misura degli eventi rari a terra, con un flusso medio al livello del mare di $ 1\,\text{cm}^{-2} / \text{min} $: una soluzione per l'eliminazione del muon background è quella di costruire laboratori sotterranei, specialmente sotto montagne rocciose alte (es.: il Gran Sasso).

\paragraph{Scoperta}

Nel 1936 Anderson e Neddermeyer, studiando la radiazione cosmica al Caltech, si accorsero di particelle che, in presenza di un campo magnetico, curvavano in maniera simile agli elettroni ma con raggio diverso: in particolare, la curva era più larga di quella degli elettroni, ma più stretta di quella dei protoni, dunque si evinse l'esistenza di una particella simile all'elettrone ma con massa intermedia tra quella dell'elettrone e quella del fotone.\\
Per quanto riguarda invece la misura della vita media del muone, il principale limite è posto dagli effetti relativistici che entrano in gioco per i muoni cosmici. per questo motivo, Rasetti e Rossi usarono il coincidence method ideato da Bothe: un circuito elettronico integrato in uno scintillatore di grosse dimensioni segna il tempo di arrivo del muone in esso ed il tempo in cui avviene il suo decadimento (emissione di elettrone o positrone). Con un circuito sufficientemente raffinato, si riuscì a misurare la vita media del muone.

\paragraph{Tauone}

Oltre all'elettrone ed al muone, la terza famiglia leptonica è quella del tauone e del neutrino tauonico: il primo fu inizialmente teorizzato da Yung-su Tsai nel 1971 e successivamente osservato da Perl nel 1974, mentre il secondo fu osservato solo nei primi anni 2000. Un tentativo italiano di rilevare il tauone fu fatto da Zichichi negli anni '60 col collisore di elettroni e positroni ADONE a Frascati, la le energie raggiunte non erano sufficienti per la reazione $ e^- + e^+ \rightarrow \tau^- + \tau^+ $.\\
Il tauone è analogo all'elettrone ed al muone, con una massa molto più alta $ m_{\tau} = 1.777\gev/c^2 $. I suoi principali decay branches sono:
\begin{equation*}
	\tau^- \rightarrow e^- + \bar{\nu}_e + \nu_{\tau}
	\qquad
	\tau^- \rightarrow \mu^- + \bar{\nu}_{\mu} + \nu_{\tau}
\end{equation*}

\section{Neutrini}

Essendo elettricamente neutri, i neutrini interagiscono solo tramite interazione debole: dato che la loro interaction cross-section è praticamente nulla, non si riesce ad osservarli direttamente con rilevatori e le loro proprietà vanno inferite in maniera indiretta.\\
Le prime evidenze indirette dell'esistenza dei neutrini si ebbero con lo studio del decadimento $ \beta $: per spiegare l'apparente violazione della conservazione dell'energia, Pauli postulò che esso fosse un decadimento a tre corpi, dove la terza particella era estremamente difficile da osservare. Con il successivo sviluppo della teoria di Fermi del decadimento $ \beta $ è stato possibile iniziare a stimare i parametri dei neutrini, in particolare del neutrino elettronico: dallo studio dei Fermi-Kurie plots e notando che l'energia osservata dell'elettrone prodotto dal decadimento è vincolata da $ m_e \le E_e \le \Delta M - m_{\nu_e} $, si può porre un porre dei limiti per i valori di $ m_{\nu_e} $. Ad oggi, la stima migliore è data dall'analisi del decadimento del trizio $ \ch{^3H} \rightarrow \ch{^3He} + e^- + \bar{\nu}_e $: $ m_{\nu_e} \le 0.8\ev/c^2 $.

\subsection{Neutrino detection}

Solitamente si misura l'interazione di una particella in un mezzo con dei rilevatori che identificano il segnale prodotto dall'interazione, il quale dipende dalla natura della particella: ad esempio, particelle cariche portano ad una ionizzazione del mezzo, creando coppie elettrone-ione, dunque un rilevatore potrebbe misurare la corrente elettrica prodotta; i fotoni interagiscono per effetto Compton oppure con pair production; i neutroni danno luogo a reazioni nucleari nel mezzo che inducono la produzione di particelle cariche.\\
I neutrini, però, interagiscono troppo debolmente con la materia e danno luogo a segnali troppo deboli: un neutrino necessita di $ \sim 10^{13}\,\text{km} $ (un anno luce) di piombo per essere stoppato con una probabilità del 50\%, e anche con un rilevatore grande come la Terra soltanto un neutrino su 100 miliardi interagirebbe. Lo studio dei neutrini è altresì molto importante, poiché sono particelle onnipresenti nell'Universo: la densità media di neutrini è $ \sim 10^8 \,\text{m}^{-3} $ (relic neutrinos), dunque giocano un ruolo cruciale in numerosi processi astrofisici.\\
I principali processi che producono neutrini sono di due tipi:
\begin{enumerate}
	\item reazioni nucleari a bassa energia: fusione nucleare nel Sole (sotto $ 100\kev $ per nucleone) e fissione nucleare nei reattori (neutroni termici che decadono $ \beta^- $ in $ \tau \approx 15\,\text{min} $);
	\item collisioni ad alta energia: collisori di particelle e cosmic ray showers (pioni che decadono in muoni e neutrini).
\end{enumerate}
La conferma sperimentale dell'esistenza dei neutrini è stata ottenuta nel 1956 da Cowan e Reines con un detector presso una centrale nucleare in South Carolina, la quale produce un flusso di neutrini di $ \sim 10^{19} \,\text{s}^{-1} $ (flusso effettivo nel rilevatore $ 5\cdot10^{13}\,\text{cm}^{-2}\text{s}^{-1} $). In particolare, sfruttarono la reazione del decadimento $ \beta $ inverso:
\begin{equation*}
	\bar{\nu}_e + p \rightarrow n + e^+
\end{equation*}
L'antineutrino proveniente dalla centrale inizialmente converte un protone in un neutrone, producendo un positrone: quest'ultimo si annichila con gli elettroni atomici presenti nel mezzo, generando due fotoni energetici che, per effetto Compton, generano fotoni rilevabili dal detector; mentre ciò avviene in maniera praticamente istantanea, con un delay di $ \sim 1\,\text{ms} $ il neutrone viene catturato da un atomo di $ \ch{^{113}Cd} $ nel mezzo, producendo altri fotoni. In questo modo, fu possibile misurare la sezione d'urto d'interazione degli antineutrini elettronici: $ 6.3\cdot10^{-44} \,\text{cm}^2 $, in perfetto accordo col valore teorico.\\
A seguito della scoperta del muone, si teorizzò l'esistenza del neutrino muonico, il quale fu effettivamente osservato nel 1962 da Lederman, Schwartz e Steinberger con un apparato che, producendo muoni a partire da un fascio protonico (tramite pioni e kaoni), è in grado di rilevare direttamente neutrini muonici. In maniera analoga si procedette alla teorizzazione e scoperta del neutrino tauonico.

\subsection{Neutrini solari}

Le reazioni di fusione nucleare nel Sole producono neutrini. In particolare, ci sono due principali cicli di reazioni che producono energia nel Sole (Fig. \ref{sol-cyc}): il ciclo pp ed il ciclo CNO.

\begin{figure}[!b]
	\centering
	\includegraphics[width=0.70\textwidth]{solar-cycles.png}
	\caption{Nuclear fusion in the Sun via pp-cycle and CNO-cycle: neutrino production highlighted.}
	\label{sol-cyc}
\end{figure}

Il \textit{ciclo pp} costituisce la sorgente di circa il 99\% dell'output energetico solare: esso è costituito da una serie di reazioni di fusione nucleare e decadimenti deboli. Il ciclo pp è dominato dal ciclo pp-I, il quale ha un bilancio netto di reazione dato da:
\begin{equation*}
	6p \rightarrow \ch{^4He} + 2p + 2e^+ + 2\nu_e + 2\gamma + 24.68\mev
\end{equation*}
Il restante 1\% dell'output solare proviene dal \textit{ciclo CNO}, anch'esso costituito da reazioni di fusione nucleare e decadimenti deboli. Il ciclo CNO è dominato dal ciclo CNO-I, con bilancio netto:
\begin{equation*}
	\ch{^{12}C} + 4p \rightarrow \ch{^4He} + \ch{^{12}C} + 2e^+ + 2\nu_e + 3\gamma + 26.73\mev
\end{equation*}
Si vede dunque che, in entrambi i cicli, data la presenza di decadimenti deboli c'è anche produzione di neutrini elettronici: con la misura del flusso di neutrini solari si possono dunque sondare direttamente le reazioni che avvengono nel nucleo del Sole\footnote{Si ricordi che le emissioni solari rilevate direttamente provengono dalla fotosfera, non dal nucleo.}.\\
Essendo l'energia totale emessa dal Sole facilmente misurabile, è possibile stimare con precisione il flusso di neutrini solari emessi e quello incidente sulla Terra: esso è $ \approx 66\cdot10^9 \,\text{cm}^{-2}\text{s}^{-1} $. Di questi, circa il 99\% è prodotto dal ciclo pp, ma, a seconda dell'energia, i vari canali di reazione producono neutrini con sezioni d'urto diverse: in Fig. \ref{s-n-en-sp} sono riportati gli spettri energetici dei vari canali di neutrino production, mentre in Fig. \ref{s-n-fl} è riportato il flusso di neutrini solari misurato dall'esperimento Borexino ai LNGS.

\begin{figure}
	\centering
	\includegraphics[width=0.70\textwidth]{sol-neutrino-en-sp.png}
	\caption{Solar neutrino energy spectra.}
	\label{s-n-en-sp}
\end{figure}
\begin{figure}
	\centering
	\includegraphics[width=0.70\textwidth]{sol-neutrino-flux.png}
	\caption{Solar neutrino flux measured by Borexino.}
	\label{s-n-fl}
\end{figure}

Si vede che, all'aumentare dell'energia, i canali di neutrino production hanno sezioni d'urto via via più basse: ciò determina una certa difficoltà, poiché misurare neutrini a basse energie è complicato, essendo necessario un background bassissimo. Ciò è stato l'obbiettivo dell'esperimento Borexino, situato presso i Laboratori Nazionali del Gran Sasso proprio per avere un setup il più radiopuro possibile: l'apparato consisteva in uno scintillatore sferico di notevoli dimensioni, per aumentare la probabilità d'interazione dei neutrini col liquido scintillante, contornata da fototubi per la rilevazione della radiazione Čerenkov\footnote{Emissione di luce in forma conica dovuta al fatto che la velocità della sorgente è maggiore di quella della luce nel mezzo, analogamente al moto supersonico.} prodotta dalle interazioni dei neutrini.\\
Storicamente, le prime misurazioni di neutrini solari furono svolte dal Homestake Solar Neutrino Observatory (HNSO), in South Dakota: questo era costituito da un enorme barilotto di 600 tonnellate di tetracloroetilene $ \ch{Cl_2 C = C Cl_2} $ volto alla misurazione del flusso di neutrini elettronici solari tramite il decadimento $ \beta $ inverso, secondo la reazione:
\begin{equation*}
	\nu_e + \ch{^{37}Cl} \rightarrow \ch{^{37}Ar} + e^-
\end{equation*}
Dalla misura radiometrica dell'argon fu possibile determinare che il flusso di neutrini solari era 1/3 di quello calcolato teoricamente. Questo risultato fu confermato dai rilevatori Kamiokande e SuperKamiokande in Giappone: quest'ultimo è il più grande rilevatore Čerenkov al mondo, costituito da uno scintillatore sferico di 50'000 tonnellate di acqua pura e 13'000 fotomoltiplicatori (PMTs); in particolare, la radiazione Čerenkov è emessa sia quando un neutrino interagisce con un elettrone atomico, facendolo fuoriuscire dal suo atomo, sia quando invece interagisce con un nucleone, producendo un leptone relativistico: il passaggio di queste particelle a velocità relativistiche in acqua causa l'effetto Čerenkov, il quale permette di risalire con sufficiente accuratezza alla direzione d'incidenza del neutrino.

\subsubsection{Neutrino oscillations}

Il problema dei neutrini solari rilevato dal HSNO può essere risolto supponendo che un neutrino elettronico prodotto dalle reazioni nel Sole non mantenga la sua identità lungo il tragitto fino alla Terra, ma oscilli tra le tre famiglie leptoniche: così facendo, il flusso di neutrini elettronici incidente sulla Terra sarà minore di quello previsto in precedenza. La teoria delle neutrino oscillations fu inizialmente sviluppata da Pontecorvo e poi espansa da Maki, Nakagawa e Sakata: in particolare, le neutrino oscillations sono possibili soltanto se si ammette che essi abbiano massa.\\
Si suppone che gli stati osservabili dei neutrini (elettronico, muonico, tauonico) non siano gli stessi nei quali si trovano i neutrini mentre viaggiano nello spazio: si parla di flavour eigenstates $ \ket{\nu_i} $ e mass eigenstates $ \ket{\nu_{\alpha}} $, legati tra loro dalla matrice PMNS:
\begin{equation}
	\ket{\nu_{\alpha}} = \sum_{j = 1}^3 U_{\alpha j} \ket{\nu_j}
	\label{eq:8.1}
\end{equation}
dove $ \alpha $ indica la famiglia leptonica ed $ j $ il mass eigenstate. L'evoluzione temporale di un mass eigenstate con energia $ E_{\nu} $, espressa in funzione della distanza $ s $ percorsa, è data da:
\begin{equation}
	\ket{\nu_j(s)} = e^{-i \frac{m_j^2}{2E_{\nu}} s} \ket{\nu_j(0)}
	\label{eq:8.2}
\end{equation}
Mass eigenstates diversi acquisteranno dunque fattori di fase diversi. Dato che, dall'Eq. \ref{eq:8.1}, i flavour eigenstates sono sovrapposizioni di mass eigenstates, un neutrino emesso con una certa identità leptonica muovendosi nello spazio avrà probabillità diverse ed oscillanti di essere rilevato con altre identità leptoniche: un esempio di oscillazioni risultanti è riportato in Fig. \ref{neutrino-oscillations}.

\begin{figure}
	\centering
	\includegraphics[width=0.60\textwidth]{neutrino-electron.png}
	\includegraphics[width=0.60\textwidth]{neutrino-muon.png}
	\includegraphics[width=0.60\textwidth]{neutrino-tau.png}
	\caption{Neutrino oscillations: $ \nu_e $ in black, $ \nu_{\mu} $ in blue, $ \nu_{\tau} $ in red.}
	\label{neutrino-oscillations}
\end{figure}

L'esistenza delle neutrino oscillations è stata confermata da SuperKamiokande e dal Sudbury Neutrino Observatory (SNO), valendo il Nobel del 2015 a Takaaki Kajita e Arthur Bruce McDonald. Il SNO, locato in Canada, è stato uno scintillatore sferico a base di acqua pesante (deuterio), la quale è sensibile a tutti e tre i flavours, focalizzatosi principalmente sul canale $ \ch{^8B} $ di neutrino production.\\
Il principale obbiettivo della fisica dei neutrini al momento è la misura del mass ordering: a tal fine, sono programmati gli esperimenti DUNE (USA) e JUNO (Cina). Per quanto riguarda la neutrino-based astronomy, è attivo l'IceCube Telescope in Antartide, mentre nel Mar Mediterraneo è in costruzione il KM3NeT/ORCA Telescope. Un'altra linea di ricerca è quella sull'esistenza di un eventuale quarto neutrino sterile, sebbene l'esperimento STEREO a Grenoble ha escluso questa ipotesi.












\chapter{Quarks}
\selectlanguage{italian}

\section{Modello di Yukawa}

Yukawa nel 1934 propose il primo modello per spiegare l'interazione tra nucleoni in termini di uno scambio di bosoni, ricalcando quello per l'interazione elettromagnetica.\\
In particolare, la teoria di campo proposta da Yukawa si basa sull'ipotesi che i nucleoni stessi siano le sorgenti del campo bosonico attraverso il quale interagiscono. Il campo d'interazione proposto deve avere le seguenti proprietà:
\begin{itemize}
	\item l'interazione deve avere un raggio d'azione $ a \sim 1\fm $;
	\item deve essere indipendente dalla carica elettrica (simmetria d'isospin);
	\item deve dipendere dallo stato di spin del sistema di nucleoni.
\end{itemize}
La terza condizione non era inclusa nel modello originale di Yukawa, e per semplicità si considera un potenziale a simmetria sferica. In questo caso, il campo bosonico può essere descritto dall'equazione di Klein-Gordon, che nel limite statico a simmetria sferica diventa:
\begin{equation*}
	\frac{1}{r^2} \frac{d}{dr} \left( r^2 \frac{d\phi(r)}{dr} \right) - \frac{mc}{\hbar} \phi(r) = 0
	\quad \Rightarrow \quad
	\phi(r) = \frac{\mathcal{N}}{r} e^{- r / a}\,,\,\, a = \frac{\hbar}{mc}
\end{equation*}
Si trova quindi che il campo di Yukawa, centrato in ciascun nucleone, è:
\begin{equation}
	V_{\text{Y}}(r) = \frac{g_s}{r} e^{-r / a}\,,\quad a = \frac{\hbar}{mc}
	\label{eq:9.1}
\end{equation}
dove $ g_s $ è la coupling constant dell'interazione. Dato che il raggio d'azione dell'interazione tra nucleoni è $ a \sim 1\mev $, si trova che il bosone mediatore, detto \textit{mesone} ($ \virgolette{stato intermedio} $) deve avere massa $ m \sim 200\mev/c^2 $: questo è in accordo col principio d'indeterminazione di Heisenberg, poiché detto $ \Delta t $ il tempo d'esistenza del mesone, si ha $ a = c \Delta t \sim \frac{\hbar c}{\Delta E} = \frac{\hbar}{mc} $.\\
Ad oggi, in realtà, è noto che il mesone di Yukawa non è una particella elementare, bensì composta da una coppia quark-antiquark: in particolari, l'interazione tra nucleoni a lunga distanza è mediata da mesoni $ \pi $ ($ 140\gev/c^2 $), a distanza ottimale ($ \approx 0.8\fm $) da mesoni $ \sigma $ ($ 500\mev/c^2 $) e a corta distanza, quando la forza diventa repulsiva, da mesoni $ \omega $ e $ \rho $ ($ 784\mev/c^2 $).\\
Inoltre, la teoria fondamentale alla base delle interazioni tra nucleoni è quella dell'interazione forte, la Cromodinamica Quantistica (QCD): questa descrive le interazioni tra quarks mediate da gluoni, non direttamente quelle tra nucleoni. Sebbene l'interazione tra nucleoni tramite scambio di mesoni sia ricavabile dalla QCD, essa va modificata quando avviene in presenza di altri nucleoni, come nel caso di un nucleo atomico: questo è un sistema quantistico a molti corpi estremamente complesso e per essere descritto richiede un modello d'interazione effettiva. Un tale esempio è la Lattice QCD, la quale può essere incorporata nella Chiral Effective Field Theory: questa teoria semplifica il modello del nucleo atomico considerando i nucleoni come gradi di libertà del sistema (al posto di quarks e gluoni).

\subsection{Simmetria di isospin}

Dato che il protone ed il neutrone non vengono distinti dall'interazione forte e che $ \left( m_n - m_p \right) / m_n \approx 10^{-3} $, Heisenberg propose di considerarli come due stati distinti di una stessa particella, il nucleone. In analogia allo spin, egli introdusse l'\textit{isospin} (o spin isotropico) $ I $, una grandezza adimensionale che si comporta matematicamente in maniera identica allo spin, assegnando $ I = \frac{1}{2} $ al nucleone e distinguendo protone e neutrone in base alla terza componente dell'isospin (analogo alla componente $ z $ dello spin): $ I_3 = +\frac{1}{2} $ per il protone e $ I_3 = -\frac{1}{2} $ per il neutrone.\\
L'hamiltoniana associata all'interazione forte è invariante per tutte le operazioni nello spazio astratto dell'isospin. Trascurando l'interazione elettromagnetica e quella debole, dunque, i livelli energetici del sistema sono degeneri e possono essere classificati secondo l'isospin totale $ I $: essendo analogo allo spin, i suoi possibili valori sono interi e semi-interi, e ad ogni suo valore corrisponde un multipletto di $ (2I + 1) $ autostati con la stessa energia ma con valori diversi di $ I_3 \in \left[ -I, I \right] $.\\
Mentre l'isospin totale si comporta come lo spin, la terza componente dell'isospin si comporta come la carica elettrica: mentre l'interazione forte, indipendente da $ I_3 $ e $ Q $ e dipendente solo da $ I $, conserva l'isospin totale, l'interazione elettromagnetica conserva solo $ I_3 $ e l'interazione debole non conserva l'isospin. L'indipendenza dell'interazione forte da $ I_3 $ e $ Q $ si può vedere, a livello nucleare, osservando che $ \ch{^7Li} $ e $ \ch{^7Be} $ hanno la stessa binding energy: questi sono \textit{mirror nuclei}, dunque hanno stesso spin e parità; formano inoltre un doppietto d'isospin con $ I = \frac{1}{2} $, dove $ \ch{^7Li} $ ha $ I_3 = -\frac{1}{2} $ e $ \ch{^7Be} $ ha $ I_3 = +\frac{1}{2} $. La simmetria di carica è verificata poiché ti trova che la forza protone-protone è uguale a quella neutrone-neutrone. Un'altra evidenza sperimentale deriva dalla seguente reazione:
\begin{equation*}
	\ch{^2H} + \ch{^2H} \rightarrow \ch{^4He} + \pi^0
\end{equation*}
L'isospin totale non è conservato, poiché $ 0 + 0 \neq 0 + 1 $, mentre $ I_3 $ lo è ($ 0 + 0 = 0 + 0 $), dunque la reazione è proibita dall'interazione forte ma amessa da quella elettromagnetica: sperimentalmente, si trova che la sezione d'urto della reazione è quella elettromagnetica e non quella forte, confermando l'ipotesi.\\
Anche i pioni formano un tripletto d'isospin: essi hanno proprietà identiche, eccetto la carica elettrica, e infatti per l'interazione forte essi sono come tre stati degeneri di una stessa particella. Dato che la degenerazione è di grado 3, il pione deve avere $ I = 1 $: $ \pi^+ $ ha $ I_3 = +1 $, $ \pi^0 $ ha $ I_3 = 0 $ e $ \pi^- $ ha $ I_3 = -1 $.\\
In generale, la relazione tra $ I_3 $, carica elettrica $ Q $ e numero barionico $ B \equiv \frac{1}{3} \left( n_q - n_{\bar{q}} \right) $ è dato dalla \textit{formula di Gell-Mann-Nishijima}:
\begin{equation}
	Q = I_3 + \frac{1}{2} B
	\label{eq:9.2}
\end{equation}
Questa formula considera solo quark di prima generazione ($ u,d $): nel caso si includano tutte le tre generazioni fermioniche, vanno aggiunti ulteriori termini.

\paragraph{Selection rules}

La conservazione dell'isospin porta a delle selection rules per i processi ad interazione forte. Ad esempio, si considerino le seguenti reazioni:
\begin{equation*}
	p + p \rightarrow \ch{^2H} + \pi^+
\end{equation*}
\begin{equation*}
	p + n \rightarrow \ch{^2H} + \pi^0
\end{equation*}
Entrambe hanno uno stato finale con $ I = 1 $; la prima reazione ha uno stato iniziale con $ I = 1 $, mentre la seconda ha uno stato iniziale che è una sovrapposizione di uno stato con $ I = 0 $ (50\%) ed uno con $ I = 1 $ (50\%): di conseguenza, dato che entrambe le reazioni sono dovute all'interazione forte, la prima procede interamente, mentre la seconda solo nel 50\% dei casi. Si deve quindi avere $ \sigma(pp \rightarrow d\pi^+) / \sigma(pn \rightarrow d\pi^0) = 2 $, che è proprio ciò che si osserva sperimentalmente.

\paragraph{Isospin dei quarks}

A partire da come sono costituiti $ p \left( uud \right) $ ed $ n \left( udd \right) $, si ha che i quarks $ u $ e $ d $ formano un doppietto di isospin $ I = \frac{1}{2} $, con $ I_3 = +\frac{1}{2} $ per $ u $ e $ I_3 = -\frac{1}{2} $ per $ d $. I rispettivi antiquarks hanno il segno di $ I_3 $ invertito, mentre gli altri quarks hanno $ I = 0 $.\\
Al giorno d'oggi si pensa che la simmetria di isospin sia dovuta alla quasi uguaglianza tra quark up e down ($ m_u \approx m_d $).

\subsection{Esperimenti di Conversi, Pancini e Piccioni}

Essendo nel giusto range di massa, inizialmente si pensò che il muone, allora noto come mesotrone, fosse il mesone di Yukawa (teorizzato nel 1934, mesotrone scoperto nel 1936). Con i loro esperimenti, però, Conversi, Pancini e Piccioni dimostrarono che questa identificazione era errata.\\
Il loro metodo sperimentale si basa sulle fast delayed coincidences, così da poter misurare con suddificiente precisioni i tempi di decadimento dell'ordine dei microsecondi. L'apparato sperimentale permette di studiare la penetrazione e l'assorbimento di mesotroni cosmici in diversi materiali ed in base al loro segno: mesotroni negativi dovrebbero essere catturati nell'orbita idrogenoide K degli atomi e poi, se mediatori dell'interazione forte, interagire col nucleo atomico prima di decadere; mesotroni positivi, invece, dovrebbero essere respinti dal nucleo e decadere nello spazio vuoto tra gli atomi del materiale assorbitore.\\
Sperimentando con un assorbitore di grafite, nel 1946 CPP trovarono che i mesotroni positivi e negativi si comportavano in maniera praticamente identica e con vite medie troppo lunghe per essere effettivamente i mediatori dell'interazione forte: Fermi, Teller e Weisskopf calcolarono che la probabilità d'assorbimento di un mesotrone negativo in quiete è di un fattore $ 10^{12} $ inferiore a quella prevedibile per un mesone di Yukawa. Nel 1947 fu osservato il pione, che sarebbe poi stato identificato come il mesone di Yukawa.

\section{Mesoni}

I mesoni sono particelle composte da una coppia quark-antiquark: essi sono bosoni, poiché hanno spin intero. Mesoni liberi possono essere prodotti dalle collisioni tra nucleoni, decadendo rapidamente in mesoni più leggeri, fotoni o leptoni: i tempi di decadimento sono tipicamente $ 10^{-20} - 10^{-23}\,\text{s} $ per quelli forti, $ 10^{-16} - 10^{-18}\,\text{s} $ per quelli elettromagnetici e $ 10^{-8} - 10^{-10}\,\text{s} $ per quelli deboli.\\
I mesoni sono soggetti a tutte le interazioni fondamentali; inoltre, non esistono mesoni stabili.

\subsection{Pioni}

I mesoni $ \pi $ sono i mesoni più leggeri, responsabili per la maggior parte dei legami tra nucleoni. Essi possono avere carica elettrica $ \pm e $ o 0, dunque vengono indicati con $ \pi^{\pm} $ e $ \pi^0 $; inoltre, hanno isospin $ I = 1 $, con $ I_3 = \pm 1 $ per $ \pi^{\pm} $ ed $ I = 0 $ per $ \pi^0 $. I pioni carichi sono uno l'antiparticella dell'altro, mentre il pione neutro è l'antiparticella di sé stesso.

\subsubsection{Scoperta}

La scoperta del mesone $ \pi $ deriva dallo studio dei raggi cosmici secondari, in particolare utilizzando lastre fotografiche (metodo approntato da Powell): come si vede in Fig. \ref{pion-det}, in accordo con l'equazione di Bethe-Bloch (Eq. \ref{eq:3.5}), la ionizzazione del mezzo aumenta man mano che la particella rallenta (poiché perde più energia). In queste emulsioni, i pioni sono stati identificati da tracce di ionizzazione molto più dense di quelle degli elettroni.\\
I pioni così scoperti fittano molto bene nel modello di Yukawa: il potenziale risultante riproduce gli effetti osservati delle forze nucleari a grande distanza. Su scale comparabili alle dimensioni dei nucleoni, invece, bisogna considerare la struttura interna degli adroni (mesoni e barioni), e il modello a scambio di pioni risulta inadeguato. Gli effetti di spin-orbita e potenziale tensoriale che si manifestano a corto range sono modellati con scambio di mesoni $ \sigma $ e $ \omega,\rho $.

\begin{figure}
	\centering
	\includegraphics[width=0.70\textwidth]{pions.png}
	\caption{Emulsion of photographic plates showing pions decaying in muons, and then muons decaying in electrons.}
	\label{pion-det}
\end{figure}

\subsubsection{Massa}

La massa del pione negativo può essere determinata con molta precisione dall'energia dei raggi X emessi quando un $ \pi^- $ è catturato in un'orbita atomica e cade verso il nucleo, emettendo fotoni nel processo (analogamente a quelli emessi dalle transizioni di elettroni).\\
In particolare, l'apparato sperimentale utilizzato da Lu et al. produce pioni come particelle secondarie quando un fascio di protoni ad alta energia, proveniente da un sincrotrone, colpisce un target spesso. I pioni così prodotti vengono rallentati e stoppati nel materiale desiderato, dove i pioni negativi sono catturati in orbite atomiche e, decadendo di orbitale in orbitale (come stati eccitati), emettono fotoni. La differenza con gli elettroni è che i pioni si muovono molto più vicini al nucleo ($ r = \frac{4\pi \epsilon_0 h^2}{me^2} $, ma $ m_{\pi} \approx 270m_e $), dunque c'è una probabilità crescente per il processo di cattura $ \pi^- p \rightarrow n $: bisogna quindi studiare i raggi X da stati con $ n = 3,4,5 $, poiché il pione non sopravvive fino a quelli con $ n = 1,2 $. Per gli atomi di fosforo ($ Z = 15 $) e titanio ($ Z = 22 $) si trovano le transition energies $ \Delta E_{4\rightarrow3}(\ch{P}) = 40.49\kev $ e $ \Delta E_{5\rightarrow4}(\ch{Ti}) = 40.39\kev $, dunque dallo spettro energetico idrogenoide $ E_n = - \frac{m Z^2 e^4}{8 \epsilon_0^2 h^2} \frac{1}{n^2} $ si trova la massa del pione negativo $ m_{\pi^-} = 139.57\mev/c^2 $.\\
Questo metodo non si applica al pione positivo. Dallo studio del suo decadimento $ \pi^+ \rightarrow \mu^+\nu_{\mu} $ si trova $ m_{\pi^+} = 139.57\mev/c^2 $.\\
Per quanto riguarda la massa del pione neutro, essa può essere calcolata con la tecnica della massa invariante. La \textit{massa invariante} di un sistema è l'energia del sistema nel frame del centro di massa, che per un sistema di particelle è:
\begin{equation}
	\left( Mc^2 \right)^2 = \left( \sum E \right)^2 - \norm{\sum \ve{p} c}^2
	\label{eq:9.3}
\end{equation}
Per un sistema di due particelle, in unità naturali si trova:
\begin{equation*}
	M^2 = E_1^2 + E_2^2 + 2E_1 E_2 - p_1^2 - p_2^2 - 2\ve{p}_1 \cdot \ve{p}_2 = m_1^2 + m_2^2 + 2\left( E_1 E_2 - p_1 p_2 \cos \theta \right)
\end{equation*}
Nel caso in cui le masse a riposo delle particelle siano trascurabili, si ha:
\begin{equation*}
	M^2 = 2 E_1 E_2 \left( 1 - \cos \theta \right)
\end{equation*}
Questo è proprio il caso del decadimento $ \pi^0 \rightarrow \gamma\gamma $: plottando le misure di massa invariante per tutte le possibili coppie di fotoni, la distribuzione risulta avere un picco centrato in $ m_{\pi^0} $ con larghezza dipendente dalla risoluzione sperimentale e dalla decay width $ \Gamma \equiv \frac{\hbar}{\tau} $ ($ \sim 150\mev $ per decadimenti forti). In questo modo, si trova $ m_{\pi^0} = 134.97\mev/c^2 $.

\subsubsection{Decadimenti}

Essendo il mesone più leggero, il pione è la più leggera particella soggetta all'interazione forte: i suoi decadimenti sono dunque deboli o elettromagnetici, risultando in vite medie più lunghe rispetto agli altri mesoni.\\
In Fig. \ref{pion-det} è mostrato il principale decadimento dei pioni carichi, con branching ratio $ \approx 99.99 $\% e $ \tau = 2.6\cdot10^{-8}\,\text{s} $:
\begin{equation*}
	\pi^+ \rightarrow \mu^+ + \nu_{\mu}
\end{equation*}
con successivo decadimento $ \mu^+ \rightarrow e^+ \nu_e \bar{\nu}_{\mu} $.\\
Il pione neutro, invece, decade elettromagneticamente con branching ratio $ \approx 98.8 $\% e $ \tau = 0.8\cdot10^{-16}\,\text{s} $:
\begin{equation*}
	\pi^0 \rightarrow \gamma + \gamma
\end{equation*}
È possibile anche un Dalitz decay mode, con branching ratio $ \approx 1.2 $\%:
\begin{equation*}
	\pi^0 \rightarrow \gamma + e^+ + e^-
\end{equation*}

\subsubsection{Spin e parità}

Il fatto che il pione, in quanto mesone, debba avere spin intero è confermato dal decadimento $ \pi^0 \rightarrow \gamma \gamma $ e dalla reazione di produzione $ p p \rightarrow p n \pi^+ $.\\
Per studiare nel dettaglio lo spin del pione positivo, si sfruttano le due reazioni forti:
\begin{equation*}
	p + p \rightarrow d + \pi^+
	\qquad \qquad
	\pi^+ + d \rightarrow p + p
\end{equation*}
dove $ d $ indica il deuterone. L'interazione forte gode di simmetria per inversione temporale, a meno di un fattore cinematico: si parla di \textit{principio di bilancio dettagliato}, secondo il quale:
\begin{equation}
	\sigma \propto \frac{g}{k^2}
	\label{eq:9.4}
\end{equation}
dove $ p = \hbar k $ e $ g $ è un fattore statistico dipendente dallo spin. Per le reazioni considerate, si trova:
\begin{equation*}
	\frac{\sigma(pp \rightarrow d\pi^+)}{\sigma(d\pi^+ \rightarrow pp)} = \frac{g(pp \rightarrow d\pi^+)}{g(d\pi^+ \rightarrow pp)} \frac{k_{\pi^+}^2}{k_p^2} = \frac{(2s_{\pi^+} + 1)(2s_d + 1)}{\frac{1}{2} (2s_p + 1)^2} \frac{k_{\pi^+}^2}{k_p^2} = \frac{3}{2} \left( 2s_{\pi^+} + 1 \right) \frac{k_{\pi^+}^2}{k_p^2}
\end{equation*}
Si noti il fattore di $ \frac{1}{2} $ al denominatore, derivante dal principio d'esclusione che dimezza i possibili stati iniziali $ pp $. Confrontando i dati sperimantali con vari valori di $ s_{\pi^+} $, si trova che $ s_{\pi^+} = 0 $.\\
Per quanto riguarda il pione neutro, invece, si considera il decadimento $ \pi^0 \rightarrow \gamma \gamma $: nel rest frame di $ \pi^0 $, i due fotoni sono emessi in direzioni opposte; dato che $ s_{\gamma} = 1 $ e necessariamente si deve avere $ m_s = \pm 1 $ lungo la direzione del loro moto (dal fatto che le onde elettromagnetiche sono trasversali), si può avere $ s_{\pi^0} = 0,2 $: in analogia ai pioni carichi, si ha quindi $ s_{\pi^0} = 0 $.\\
La parità intrinseca del pione negativo può essere dedotta dalla seguente reazione:
\begin{equation*}
	\pi^- + d \rightarrow n + n
\end{equation*}
La parità iniziale è $ P_{\text{i}} = P_{\pi^-} P_d \left( -1 \right)^{\ell_{\text{i}}} $, dunque, selezionando $ \ell_{\text{i}} = 0 $ e ricordando che $ P_d = +1 $, si ha $ P_{\text{i}} = P_{\pi^-} $. Lo stato finale ha invece $ P_{\text{f}} = P_n P_n \left( -1 \right)^{\ell_{\text{f}}} = \left( -1 \right)^{\ell_{\text{f}}} $, dato che $ P_n = +1 $. Per lo stato iniziale si ha $ \ve{j}_{\text{i}} = \ve{s}_{\pi^-} + \ve{s}_d + \ve{\ell}_{\text{i}} $, ma $ s_{\pi^-} = 0 $, $ s_d = 1 $ e si è scelto $ \ell_{\text{i}} = 1 $, dunque $ j_{\text{i}} = 1 $. Per lo stato finale, invece, essendo i neutroni dei fermioni, la sua funzione d'onda totale deve essere antisimmetrica per scambio di neutroni: se la parte di spin della funzione d'onda è simmetrica ($ s_{n_1} + s_{n_2} = 1 $), la parte spaziale deve essere antisimmetrica ($ \ell_{\text{f}} $ dispari), mentre se $ s_{n_1} + s_{n_2} = 0 $ si deve avere $ \ell_{\text{f}} $ pari. La seconda opzione è esclusa dal fatto che $ j_{\text{i}} = 1 $, poiché se $ s_{n_1} + s_{n_2} = 0 $ l'unico modo per avere $ j_{\text{f}} = 1 $ è $ \ell_{\text{f}} = 1 $, che non è pari. L'unica opzione rimanente è che $ \ell_{\text{f}} $ sia dispari, il che implica $ P_{\pi^-} = -1 $.\\
Con un procedimento analogo si trova $ P_{\pi^+} = -1 $, considerando la reazione $ \pi^+ d \rightarrow p p $.\\
Per trovare $ P_{\pi^0} = -1 $, invece, si può studiare la polarizzazione degli elettroni nel decadimento $ \pi^0 \rightarrow e^+ e^+ e^- e^- $, oppure si può ricavarlo direttamente dalla reazione $ \pi^- d \rightarrow n n \pi^0 $.

\subsubsection{Produzione}

Le principali reazioni di produzione di pioni sono indotte dai raggi cosmici:
\begin{equation*}
	\begin{split}
		p + p
		&\rightarrow p + n + \pi^+ \\
		&\rightarrow p + p + \pi^0 \\
		&\rightarrow p + p + \pi^+ + \pi^-
	\end{split}
\end{equation*}
Queste reazioni possono essere riprodotte in accelleratori con frequenza anche maggiore, ma i raggi cosmici possono produrre energie maggiori. Si noti che viene rispettata la conservazione del numero barionico, poiché fermioni; per i mesoni, invece, in quanto bosoni, non vale nessuna legge di conservazione del numero, dunque le reazioni nucleone-nucleone possono produrre un numero arbitrario di mesoni, finché permesso dal bilancio energetico e dalla conservazione della carica.\\
Per le reazioni di produzione di due pioni, dall'Eq. \ref{eq:6.1} si trova la threshold energy $ T_{\text{th}} = 4m_{\pi} \approx 600\mev $, poiché $ Q = 2m_{\pi} $. Pioni vengono prodotti anche da raggi $ \gamma $ incidenti su nucleoni, secondo le reazioni:
\begin{equation*}
	\gamma + p \rightarrow n + \pi^+
	\qquad \qquad
	\gamma + p \rightarrow p + \pi^0
	\qquad \qquad
	\gamma + n \rightarrow p + \pi^0
\end{equation*}
Nelle cosiddette meson factories, si fanno incidere raggi $ \gamma $ su target solidi a basso $ Z $, producendo pioni con una threshold energy $ T_{\text{th}} \approx 150\mev $.

\subsubsection{Reazioni}

A partire da fasci di pioni prodotti in laboratorio, si possono studiare tre tipi di reazioni:
\begin{enumerate}
	\item scattering elastico: ad esempio $ \pi^+ + p \rightarrow p + \pi^+ $;
	\item scattering inelastico: ad esempio $ \pi^+ + p \rightarrow n + \pi^+ + \pi^+ $;
	\item scambio di carica: ad esempio $ \pi^- + p \rightarrow n + \pi^0 $;
\end{enumerate}
Studiando le reazioni $ \pi^{\pm} p $ si osservano delle grosse risonanze nelle cross-sections: si trova che queste risonanze pione-nucleone hanno energia, vita media, psin, parità e decay modes ben definiti, dunque sono delle strutture ben definite al pari di altre particelle come protoni e neutroni, sebbene estremamente short-lived.
In Fig. \ref{pion-scatt} si può notare la risonanza a $ T_{\pi} \approx 200\mev $, presente sia in $ \pi^+ p $ che in $ \pi^- p $ (sia per scattering elastico che per scambio di carica): questa risonanza, corrispondente ad una center-of-mass energy di $ 1232\mev $, è nota come risonanza $ \Delta(1232) $; le risonanze meno pronunciate sono risonanze $ N $.

\begin{figure}[!hb]
	\centering
	\includegraphics[width=0.70\textwidth]{pion-scatt.png}
	\caption{Cross-sections for $ \pi^- p $ reactions.}
	\label{pion-scatt}
\end{figure}

Le risonanze $ N $ e $ \Delta $ possono essere considerate come stati eccitati del nucleone: le risonanze $ \Delta $ esistono in quartetti di carica $ Q = +2,+1,0,-1 $, mentre le risonanze $ N $ in doppietti di carica $ Q = +1,0 $; a queste risonanze viene dunque assegnato un isospin $ I = \frac{3}{2} $ per le $ \Delta $ e $ I = \frac{1}{2} $ per le $ N $ (in accordo col coupling di nucleone con $ I = \frac{1}{2} $ e pione con $ I = 1 $). In assenza di interazioni elettromagnetiche, i membri di un multipletto avrebbero tutti la stessa massa: nella realtà, si osserva uno splitting di pochi MeV$ /c^2 $. Queste risonanze nucleoniche sono consistenti con quelle osservate in altri processi di scattering, come $ pp $, $ \gamma p $ e $ e^- p $.

\subsection{Risonanze}

Essendo i pioni i mesoni pià leggeri, è possibile produtte mesoni più pesanti aumentando l'energia incidente nelle reazioni $ pp $ o $ \pi p $. Tutti i mesoni più pesanti del pione hanno masse superiori a $ 2m_{\pi} $, dunque, non essendoci leggi di conservazione sul numero mesonico, i loro decadimenti possono produrre due o più pioni tramite interazione forte, con $ \tau \sim 10^{-23}\,\text{s} $. Sebbene tali tempi di decadimento impediscano l'osservazione diretta, è possibile inferire l'esistenza di queste risonanze mesoniche dallo studio dei loro prodotti di decadimento: dalla distribuzione energetica di tali prodotti è possibile risalire alla decay width della risonanza, e dunque alla sua vita media.

\begin{figure}[!h]
	\centering
	\includegraphics[width = 0.80\textwidth]{meson-reson.png}
	\caption{Invariant mass distribution for $ \pi^+ \pi^0 $ and $ \pi^+ \pi^- $, produced respectively by $ \pi^+ p \rightarrow p \pi^+ \pi^0 $ at $ 2.08\gev/c^2 $ incident momentum and $ \gamma p \rightarrow p \pi^+ \pi^- $ with $ E_{\gamma} = 2.8\gev $.}
	\label{meson-reson}
\end{figure}

In Fig. \ref{meson-reson} si possono vedere gli invariant mass plots per le reazioni $ \pi^+ p \rightarrow p \pi^+ \pi^0 $ e $ \gamma p \rightarrow p \pi^+ \pi^- $, nei quali è possibile identificare delle risonanze a circa $ 770\mev $, di larghezza $ 150\mev $ tipica dei decadimenti forti: questi sono identificabili come due mesoni, rispettivamente il mesone $ \pi^+ $ e quello $ \rho^0 $, i quali decadono in maniera forte come $ \rho^+ \rightarrow \pi^+ \pi^0 $ e $ \rho^0 \rightarrow \pi^+ \pi^- $.\\
Il mesone $ \rho $, così come il mesone $ \pi $, presenta un tripletto d'isospin $ I = 1 $: $ \rho^+ $ con $ I_3 = +1 $, $ \rho^0 $ con $ I_3 = 0 $ e $ \rho^- $ con $ I_3 = -1 $. Studiando i pioni prodotti dal decadimento, si trova che il mesone $ \rho $ ha spin $ s_{\rho} = 1 $ e parità $ P_{\rho} = -1 $.\\
Altre importanti risonanze mesoniche sono il mesone $ \omega $ a $ 783\mev $ e il mesone $ \eta $ a $ 549\mev $: il mesone $ \omega $ ha spin 1, isospin 0 e parità $ -1 $, mentre il mesone $ \eta $ spin 0, isospin 0 e parità $ -1 $, ed entrambi formano un singoletto elettricamente neutro. Il principale decay mode del mesone $ \omega $ è quello forte $ \omega \rightarrow \pi^+ \pi^- \pi^0 $, ma sono anche possibili quelli elettromagnetici $ \omega \rightarrow \pi^0 + \gamma $ e $ \omega \rightarrow \pi^+ \pi^- $ (non si conserva l'isospin): la decay width è $ \Gamma \approx 10\mev $, dunque la vita media del mesone $ \omega $ è circa un ordine di grandezza più lunga di quella del mesone $ \eta $. Il mesone $ \eta $, invece, decade principalmente in maniera elettromagnetica come $ \eta \rightarrow \gamma \gamma $, $ \eta \rightarrow \pi^0 \pi^0 \pi^0 $ ed $ \eta \rightarrow \pi^+ \pi^- \pi^0 $ (proibiti dalla conservazione dell'isospin): la decay width risultante è $ \Gamma \approx 0.8\kev $, corrispondente ad una vita media $ \tau \sim 10^{-18} \,\text{s} $, molto maggiore di quella dei mesoni $ \omega $ e $ \rho $ che decadono forte.\\
Infine, si deve tener conto che le risonanze mesoniche possono essere prodotte anche da collisioni $ e^+ e^- $ altamente energetiche, oltre che da quelle di particelle che interagiscono forte.

\subsection{Mesoni strani}

Negli anni '50 furono osservate delle particelle che, seppur prodotte da reazioni forti, decadevano con tempi lunghi caratteristici dei decadimenti deboli. Per questo motivo, furono chiamate particelle strane, e successivamente sarebbero state associate alla presenza di strange quarks nella loro composizione.
Alle paricelle strane è associato un numero quantico di stranezza, il quale è conservato dalle interazioni forte ed elettromagnetica, mentre non lo è da quella debole.\\
La particella strana più leggera è il mesone $ K $, con massa di circa $ 500\mev/c^2 $, spin 0 e parità $ -1 $, e facente parte di un doppietto di isospin $ I = \frac{1}{2} $: $ K^+ $ con $ I_3 = \frac{1}{2} $ e $ K^0 $ con $ I_3 = -\frac{1}{2} $; le rispettive antiparticelle sono $ K^- $, con $ I_3 = -\frac{1}{2} $, e $ \bar{K}^0 $, con $ I_3 = \frac{1}{2} $. Similmente alla produzione di pioni, i kaoni possono essere prodotti da reazioni del tipo:
\begin{equation*}
	\pi^- + p \rightarrow n + K^- + K^+
\end{equation*}
Convenzionalmente si assegna stranezza $ S = +1 $ a $ K^+ $, dunque, essendo questa una reazone forte, si deve avere $ S = -1 $ per $ K^- $ per avere la conservazione della stranezza; inoltre, si trova $ S = +1 $ per $ K^0 $ ed $ S = -1 $ per $ \bar{K}^0 $. Una conseguenza del fatto che l'interazione forte conserva la stranezza è l'\textit{associated production}: le particelle strane sono sempre prodotte in coppie o set, mai sole, così da conservare la stranezza nulla iniziale. Essendo la particella strana più leggera, il kaone può decadere solo in particelle con stranezza nulla, quindi solo debolmente, risultando così in una vita media relativamente lunga $ \tau \sim 10^{-8}\,\text{s} $. I principali decay modes per il kaone positivo sono $ K^+ \rightarrow \mu^+ \nu_{\mu} $, $ K^+ \rightarrow \pi^+ \pi^0 $ e $ K^+ \rightarrow \pi^+ \pi^+ \pi^- $.\\
Ci sono molte risonanze mesoniche strane, le quali solitamente decadono in mesoni strani più leggeri senza violare la conservazione della stranezza. Un esempio è il mesone $ K^* $, risonanza a $ 892\mev $, il quale decade secondo $ K^* \rightarrow K \pi $: questo è un decadimento forte, ed infatti risulta $ \tau \sim 10^{-23}\,\text{s} $.

\section{Modello a quarks}

Per spiegare il particle zoo scoperto tra gli anni '50 e '60, fu proposto che tutti i mesoni ed i barioni fossero composte di particelle fondamentali, dette quarks. Ad oggi, si pensa che queste siano particelle fondamentali puntiformi, senza substruttura: in particolare, i quarks sono fermioni con spin $ \frac{1}{2} $, numero barionico $ B = \frac{1}{3} $ e carica elettrica frazionaria, soggetti a tutte le interazioni fondamentali (sono le uniche particelle ad interagire forte).\\
Il modello a quark originario, psoposto da Gell-Mann e Zweig nel 1962, comprendeva soltanto tre quarks: up, down e strange. Nei decenni successivi sono stati scoperti altre quarks: charm, bottom e top. In particolare, essi sono organizzati in tre generazioni analogamente ai leptoni, ciascuna con un quark di carica elettrica $ Q = \frac{2}{3} e $ ed uno con $ Q = -\frac{1}{3} e $; inoltre, la prima generazione di quarks è l'unica ad avere isospin non-nullo e forma un doppietto con $ I = \frac{1}{2} $: $ u $ ha $ I_3 = +\frac{1}{2} $ e $ d $ ha $ I_3 = -\frac{1}{2} $. A ciascuno dei quarks della seconda e terza generazione è associato un numero quantico: strangeness $ S $, charm $ C $, bottomness $ B' $ e topness $ T $.\\
Nel modello a quark, i mesoni sono composti da coppie quark-antiquark, mentre i barioni da tre quarks. Negli ultimi anni sono state osservate risonanze esotiche composte da un numero superiore di quarks: nel 2013 l'esperimento BESS III in Cina ha osservato dei tetraquarks, mentre nel 2015 LHCb presso il CERN dei pentaquarks. Inoltre, finora non sono mai stati osservati adroni composti da top quarks.

\paragraph{Evidenze sperimentali}

Il modello a quarks presenta una visione relativamente semplice della struttura interna delle particelle subatomiche: con pochi parametri liberi relativi ai quarks, esso è in grado di predire accuratamente le proprietà delle varie particelle composite (massa, vita media, momento magnetico, etc.). Ad oggi non sono ancora state trovate contraddizioni a tale modello.\\
Una delle evidenze a supporto del modello è l'esistenza di eccitazioni nucleoniche: il nucleone rappresenta un doppietto (protone e neutrone) con isospin $ I = \frac{1}{2} $ e splitting $ \Delta m = 1.3\mev $, e può essere eccittato alla risonanza $ \Delta (1232) $, la quale costituisce un quadrupletto con isospin $ I = \frac{3}{2} $ e splitting $ \Delta m = 5\mev $. Ciò suggerisce che il nucleone abbia una struttura interna: il doppietto del nucleone è costituito da $ p (uud) $ e $ n (udd) $, mentre quello del barione $ \Delta (1232) $ da $ \Delta^{++} (uuu) $, $ \Delta^+ (uud) $, $ \Delta^0 (udd) $ e $ \Delta^- (ddd) $.\\
Sempre studiando i nucleoni, il fatto che il neutrone abbia un momento magnetico implica che esso sia costituito da particelle cariche la cui carica elettrica totale sia nulla. Per quanto riguarda il protone, invece, se esso fosse una particella elementare la sua carica sarebbe uniformemente distribuita o nel suo volume (sfera dielettrica) o sulla sua superficie (sfera conduttrice): entrambi questi modelli sono smentiti dagli esperimenti di scattering, che mostrano una distribuzione di carica elettrica più complessa.\\
Inoltre, le collisioni inelastiche ad alte energie mostrano un andamento diverso da quello calcolato assumendo l'assenza di una struttura interna, mentre il modello a quark riesce a spiegare molto bene la formazione di nuove particelle dallo scattering.

\subsection{Adroni}

Le simmetrie degli adroni possono essere messe facilmente in luce dal modello a quark, in particolare plottango i multipletti di mesoni e barioni con un dato spin su un grafico $ I_3 - S $. Come si può vedere in Figg. \ref{baryons}-\ref{mesons}, i barioni di spin $ \frac{1}{2} $ si dispongono in un ottetto e quelli di spin $ \frac{3}{2} $ in un decupletto, mentre i mesoni di spin 0 e 1 si dispongono entrambi in un nonetto. Un'importante conferma del modello a quark si ebbe con la scoperta del barione $ \Omega^- $: esso non era ancora stato osservato al momento della formulazione della teoria, la quale però lo prevede per completare il decupletto barionico.\\
Ricordando che $ I_3 = \frac{1}{2} \left( n_u - n_d \right) $ e che $ S = - \left( n_s - n_{\bar{s}} \right) $, dai plots si può ricavare facilmente la composizione in quark dei vari adroni: ad esempio, $ \ket{\pi^0} = \frac{1}{\sqrt{2}} \left( \ket{u\bar{u}} - \ket{d\bar{d}} \right) $. Si possono anche vedere gli allineamenti degli spin dei quarks: ad esempio, $ \ket{p^+} = \ket{u^+ u^- d^+} $ e $ \ket{{\Delta^0}^+} = \ket{u^+ d^+ d^+} $, dove gli apici si riferiscono al segno di $ m_s $.\\
Si vede che i mesoni, essendo composti da una coppia di quark, devono avere spin intero, mentre i barioni semi-intero. Inoltre, va notato che i quarks, in quanto fermioni, sono soggetti al principio d'esclusione di Pauli: il loro ket di stato è determinato dal ket spaziale, dal ket di flavour e dal ket di spin.

\begin{figure}
	\centering
	\includegraphics[width = 0.49\textwidth]{baryon-octet.png}
	\includegraphics[width = 0.49\textwidth]{baryon-decuplet.png}
	\caption{Baryon octet ($ s = \frac{1}{2} $) and decuplet ($ s = \frac{3}{2} $).}
	\label{baryons}
\end{figure}
\begin{figure}
	\centering
	\includegraphics[width = 0.49\textwidth]{meson-nonet-1.png}
	\includegraphics[width = 0.49\textwidth]{meson-nonet-2.png}
	\caption{Meson nonets ($ s = 0 $ and $ s = 1 $).}
	\label{mesons}
\end{figure}

\subsection{Colore}

Si può notare che il modello a quarks postula che dei quarks identici occupino lo stesso stato quantistico all'interno di alcuni barioni: ad esempio, nei barioni $ \Delta^{++}(uuu) $, $ \Delta^- (ddd) $ ed $ \Omega^-(sss) $ tre quarks identici si trovano tutti in uno stato con $ \ell = 0 $ e $ s = \frac{1}{2} $, risultando simmetricamente scambiabili tra loro. Per non violare il principio d'esclusione è dunque necessario introdurre un nuovo numero quantico, al fine di distinguere gli stati di questi quarks altrimenti identici: il \textit{colore}. I numero quantico di colore ha tre possibili valori: blu ($ \mathsf{b} $), rosso ($ \mathsf{r} $) e verde ($ \mathsf{g} $), ai quali sono associati anti-blu ($ \bar{\mathsf{b}} $), anti-rosso ($ \bar{\mathsf{r}} $) ed anti-verde ($ \bar{\mathsf{g}} $) per gli antiquarks.\\
Una proprietà fondamentale del colore è che solo particelle incolori possono essere osservate: queste sono particelle in cui sono combinati colore e rispettivo anticolore (mesoni), oppure tutti e tre i colori in egual misura (barioni); le altre combinazioni non sono permesse. Questo è il motivo per cui non si osservano quark singoli: essi vengono istantaneamente confinati in adroni.\\
Il colore, così come gli altri numeri quantici associati ai quarks (isospin, stanezza, charm, bottomness e topness) sono conservati dall'interazione forte, ma non da quella debole. Risulta quindi che la maggior parte degli adroni decade forte, con vite medie $ \tau \sim 10^{-23}\,\text{s} $; gli adroni con le masse più piccole per ciascun set di numeri quantici (adroni strani, adroni charm, etc.), invece, non possono decadere forte poiché non ci sono particelle più leggere con gli stessi numeri quantici: di conseguenza, hanno vite medie più lunghe poiché decadono tramite interazione debole ($ \tau \sim 10^{-7} - 10^{-13}\,\text{s} $) o elettromagnetica ($ \tau \sim 10^{-16} - 10^{-21}\,\text{s} $).

\subsubsection{Cromodinamica Quantistica}

La natura della forza forte è legata al colore: si parla infatti di \textit{carica di colore}. Il force carrier del color field (o strong field) è il \textit{gluone}, un bosone massless e con spin 1.\\
Essendo mediatori dell'interazione tra quarks, i gluoni devono essere portatori di due colori, e nello specifico di un colore e di un anti-colore: ad esempio, la reazione $ u_{\mathsf{r}} + d_{\mathsf{b}} \rightarrow u_{\mathsf{b}} + d_{\mathsf{r}} $ è mediata da un gluone $ \mathsf{r}\bar{\mathsf{b}} $ emesso dal quark up ed assorbito dal quark down. Le possibili combinazioni colore - anti-colore sono 9, ma quelle colorless $ \mathsf{b}\bar{\mathsf{b}} $, $ \mathsf{r}\bar{\mathsf{r}} $ e $ \mathsf{g}\bar{\mathsf{g}} $ vanno opportunamente combinate secondo le regole di simmetria del color field:
\begin{equation*}
	\frac{1}{\sqrt{2}} \left( \mathsf{r}\bar{\mathsf{r}} - \mathsf{g}\bar{\mathsf{g}} \right)
	\qquad
	\frac{1}{\sqrt{6}} \left( \mathsf{r}\bar{\mathsf{r}} + \mathsf{g}\bar{\mathsf{g}} - 2\mathsf{b}\bar{\mathsf{b}} \right)
	\qquad
	\frac{1}{\sqrt{3}} \left( \mathsf{r}\bar{\mathsf{r}} + \mathsf{b}\bar{\mathsf{b}} + \mathsf{g}\bar{\mathsf{g}} \right)
\end{equation*}
Soltanto le prime due di queste tre combinazioni possono effettivamente trasmettere colore, dunque in totale ci sono 8 gluoni.\\
I quarks all'interno degli adroni sono legati dal color field scambiandosi gluoni, dunque cambiano continuamente colore, ma sempre preservando la neutralità overall dell'adrone. I quarks non possono esistere in maniera individuale poiché la color force scresce d'intensità all'aumentare della distanza: man mano che due quarks si allontanano tra loro, l'energia d'interazione aumenta fino al punto che è sufficiente a creare una nuova coppia quark-antiquark, formando quindi due particelle overall sempre colorless. Si vede dunque che se ad un adrone viene data sufficiente energia, ad esempio da scattering ad alte energie, non si riesce mai ad estrarre un singolo quark da esso, ma il risultato sarà sempre la produzione di altri adroni (tendenzialmente mesoni).\\
Le interazioni tra quarks danno luogo alle forze nucleari: sebbene i protoni nel nucleo si respingono tra loro per via dell'interazione elettromagnetica, l'interazione forte tra i rispettivi quark ha un'intensità maggiore, dunque i due nucleoni rimangono legati.














\chapter{Modello Standard}
\selectlanguage{italian}

Il Modello Standard della fisica delle particelle comprende il modello a quarks e il modello elettrodebole: quest'ultimo è stato introdotto a partire dallo studio delle violazioni di varie simmetrie.

\section{Simmetrie}

Oltre al modello a quarks, per mettere ordine nel particle zoo è necessario introdurre delle simmetrie che si osservano in varie reazioni.\\
Il concetto di simmetria in fisica delle particelle è stato introdotto da Weyl e Feynman: in generale essa è presente quando, a seguito dell'applicazione di una certa trasformazione ad un oggetto, esso rimane invariato; si parla di violazione di simmetria quando invece esso risulta modificato. Come messo in luce da Noether, il concetto di simmetria è estremamente importante in fisica poiché è legato alle leggi di conservazione di determinate quantità.\\
Il Modello Standard presenta naturalmente tre \textit{quasi-simmetrie}:
\begin{itemize}
	\item C-symmetry (carica): particella $ \mapsto $ antiparticella;
	\item P-symmetry (parità): $ \ve{r} \mapsto -\ve{r} $;
	\item T-symmetry (inversione temporale): $ t \mapsto -t $.
\end{itemize}
Queste sono dette quasi-simmetrie perché nell'Universo attuale ciascuna di esse è violata: ad esempio, l'interazione debole viola la C-symmetry e la P-symmetry (vedere esperimenti di Goldhaber e di Madame Wu), mentre il secondo principio della termodinamica asserisce che l'universo evolve in una direzione temporale ben precisa, violando la T-symmetry; inoltre, il decadimento dei kaoni neutri viola la CP-symmetry. Il Modello Standard, però, predice che la CPT-symmetry debba essere una simmetria: Ludens, Pauli e Schwinger (indipendentemente) hanno dimostrato che la Lorentz invariance implica la CPT-symmetry, dunque non c'è nessuna teoria consistente che permetta la violazione di tale simmetria. Una conseguenza della CPT-symmetry è che particelle e antiparticelle hanno stessa massa, stessa vita media e carica elettrica e momento magnetico uguali in modulo e di segno opposto.

\subsection{Esperimento di Madame Wu}

Nel 1956 Lee e Yang, con una literature review, notarono che i dati sperimentali non confermavano la P-symmetry nell'interazione debole. Rivolgendosi a Madame Wu, esperta di spettroscopia $ \beta $, proposero esperimenti per testare la P-symmetry nel decadimento $ \beta $.\\
L'esperimento di Madame Wu studia il decadimento $ \ch{^{60}Co} \rightarrow \ch{^{60}Ni}^* + e^- + \bar{\nu}_e $, con successiva emissione di $ 2\gamma $ per diseccitare il Nichel: gli osservabili sono dunque i fotoni e l'elettrone. In linea teorica, considerando un nucleo di cobalto nel suo rest-frame con spin $ \ve{s} $ e supponendo che l'elettrone venga emesso con momento $ \ve{p} $, l'operatore di parità agisce come $ \hat{\mathcal{P}} \ve{s} = \ve{s} $ e $ \hat{\mathcal{P}} \ve{p} = - \ve{p} $ (il momento angolare, essendo $ \ve{L} = \ve{r} \times \ve{p} $, è conservato dalla parità), dunque un modo per testare la P-symmetry è studiare la direzione d'emissione degli elettroni dal campione. In coordinate sferiche con $ \ve{s} \parallel \ve{e}_z $, basta studiare l'angolo d'emissione $ \theta $: l'operatore parità porta $ \theta \mapsto \pi - \theta $, dunque un'asimmetria nell'emissione dei elettroni dal campione agli angoli $ \theta $ e $ \pi - \theta $ indicherebbe una violazione della P-symmetry. Per quanto riguarda la distribuzione dei raggi $ \gamma $, essa è invariante per operatore di parità, poiché una volta fissato l'asse dello spin essa è determinata solo dal $ \Delta \ell $ della transizione (nel caso del $ \ch{^{60}Ni}^*(4^+) \rightarrow \ch{^{60}Ni}(0^+) $ sono due transizioni $ \Delta \ell = 2 $, dunque quadripolari).

\subsubsection{Apparato sperimentale}

\begin{figure}
	\centering
	\includegraphics[width = 0.70\textwidth]{wu-exp.png}
	\caption{Madame Wu's experimental setup.}
	\label{wu-exp}
\end{figure}

L'apparato sperimentale di Madame Wu è schematizzato in Fig. \ref{wu-exp}. La sorgente di cobalto è inserita in misura opportuna per favorire il raffreddamento del sistema: infatti, una volta che il sistema è stato magnetizzato, per mantenere l'allineamento degli spin è necessario raffreddarlo a temperature di $ \sim 10\,\text{mK} $, poiché l'agitazione termica tende a rompere l'allineamento.\\
Per controllare che il sistema mantenga l'allineamento degli spin per un periodo sufficiente di tempo, sono presenti due scintillatori posti a 90° per monitorare la distribuzione dei raggi $ \gamma $: questa dipende dall'asse di spin, con una forte differenza di emissione tra il piano equatoriale e quello polare. Come si può vedere in Fig. \ref{wu-data}, il sistema mantiene correttamente la polarizzazione per circa $ 6 - 8\,\text{min} $.

\begin{figure}
	\centering
	\includegraphics[width = 0.70\textwidth]{wu-data.png}
	\caption{Data from Madame Wu's experiment.}
	\label{wu-data}
\end{figure}

Sempre in Fig. \ref{wu-data} si può vedere che, per tutto il tempo in cui la direzione di riferimento data dagli spin polarizzati è ben definita, le distribuzioni di elettroni forward ($ \theta $) e backward ($ \pi - \theta $) sono decisamente asimmetriche: c'è una direzione preferenziale di emissione, che è quella backward rispetto alla direzione dello spin, dunque si evince che il decadimento $ \beta $ effettivamente viola la P-symmetry.

\subsection{Esperimento di Goldhaber}

A seguito dell'esperimento di Madame Wu sulla violazione della P-symmetry, nel 1957 Goldhaber mostrò che l'interazione debole violava anche la C-symmetry.\\
Innanzitutto, si definisce l'\textit{helicity} di una particella di momento $ \ve{p} $ e spin $ \ve{s} $ come:
\begin{equation}
	h \defeq \frac{\ve{s} \cdot \ve{p}}{\norm{\ve{s}} \norm{\ve{p}}}
	\label{eq:10.1}
\end{equation}
A seconda che il momento della particella sia parallelo o antiparallelo allo spin, l'helicity assume i valori $ +1 $ (right-handed) o $ -1 $ (left-handed). Inoltre, si vede che $ \hat{\mathcal{P}}h = -h $.\\
Con un esperimento molto sofisticato, Goldhaber riuscì a dimostrare che neutrini ed antineutrini hanno helicity differente. Per studiare un processo debole senza incorrere nel problema che il decadimento $ \beta $ è un decadimento a tre corpi, si considera la cattura elettronica che, essendo un processo a due corpi, permette di controllare energia e momento angolare e studiare le leggi di conservazione. In particolare, si consideri l'electron K-capture $ \ch{^{152}Eu}(0^-) + e^- \rightarrow \ch{^{152}Sm}^*(1^-) + \nu_e $, seguita da $ \ch{^{152}Sm}^*(1^-) \rightarrow \ch{^{152}Sm}(0^-) + \gamma $: essendo processi a due corpi, si ha $ E_{\nu_e} = 840\kev $ ed $ E_{\gamma} = 961\kev $. Si dimostra che $ h(\nu_e) = h(\ch{^{152}Sm}^*) = h(\gamma) $, dunque per trovare l'helicity del neutrino basta misurare quella del fotone. Quest'ultima è deducibile dal Compton scattering: la Compton cross-section dipende dall'orientazione relativa degli spin del fotone e dell'elettrone, ed in particolare si osserva un minor assorbimento se lo spin del fotone è parallelo a quello dell'elettrone. Dalle misure sperimentali di Goldhaber, si trova $ h(\nu_e) = -1.0 \pm 0.3 $, ovvero i neutrini sono sempre left-handed, mentre gli antineutrini sempre right-handed: di conseguenza, non c'è simmetria per scambio particella-antiparticella.

\subsection{Decadimento dei kaoni}

Il kaone neutro presenta un comportamento strano: esso infatti può decadere sia in due che in tre pioni, con tempi di decadimento molto diversi. Si parla di kaone short-lived $ K^0_S $, il quale decade come $ K^0_S \rightarrow \pi^+ \pi^- $ o $ K^0_S \rightarrow 2\pi^0 $ con $ \tau \approx 8.95\cdot10^{-11}\,\text{s} $, e di kaone long-lived $ K^0_L $, il quale decade come $ K^0_L \rightarrow \pi^+ \pi^- \pi^0 $ o $ K^0_L \rightarrow 3\pi^0 $ con $ \tau \approx 5.12\cdot10^{-8}\,\text{s} $.\\
Dato che i mesoni hanno sempre parità (intrinseca) negativa (quarks ed antiquarks hanno parità opposte), il decadimento del $ K^0_S $ è un'ulteriore dimostrazione della violazione della P-symmetry nei processi deboli.
Il decadimento del kaone neutro, però, può anche mettere in luce come l'interazione debole violi anche la CP-symmetry: con un esperimento del 1963, Cronin e Fitch dimostrarono che occasionalmente un kaone $ K^0_L $ può decadere in due pioni, violando anch'esso la P-symmetry.

\subsubsection{Esperimento di Cronin e Fitch}

L'esperimento consiste in un fascio di kaoni neutri in un tubo di $ 17\,\text{m} $, all'interno del quale vengono rilevati i pioni prodotti dal decadimento dei kaoni. Data la differenza di 3 ordini di grandezza nelle vite medie, ci si aspetterebbe di osservare soltanto i decadimenti a tre pioni del $ K^0_L $ all'estremità opposta del tubo: vengono invece osservati anche dei decadimenti a due pioni, con una frequenza di circa 1 ogni 500 decadimenti.\\
I decadimenti a doppio pione così osservati non sono attribuibili al decadimento del $ K^0_S $: data la half-life $ t_{1/2} = 5.5\cdot10^{-10}\,\text{s} $ ed assumendo $ v = 0.98c $, nel rest-frame la popolazione diventa $ 1/500 $ in appena $ 17\,\text{cm} $, ovvero $ 85\,\text{cm} $ nel frame del laboratorio ($ \gamma \approx 5 $); per qualsiasi velocità fisica, nessun $ K^0_S $ riesce a decadere all'estremità del tubo a $ 17\,\text{m} $.\\
L'unica spiegazione è dunque che in 1 decadimento ogni 500 il $ K^0_L $ decade in due pioni al posto di tre, violando dunque la CP-symmetry. Ciò può essere illustrato col modello a quarks: ricordando che $ \ket{K^0} = \ket{d\bar{s}} $ e $ \bar{K}^0 = \ket{\bar{d}s} $, si trova che:
\begin{equation*}
	\ket{K^0_S} = \frac{1}{\sqrt{2}} \left( \ket{K^0} + \ket{\bar{K}^0} \right)
	\qquad \qquad
	\ket{K^0_L} = \frac{1}{\sqrt{2}} \left( \ket{K}^0 - \ket{\bar{K}^0} \right)
\end{equation*}
Questi sono entrambi autostati di $ \hat{\mathcal{C}}\hat{\mathcal{P}} $, con rispettivi autovalori $ -1 $ e $ +1 $: si vede dunque che il decadimento del $ K^0_L $ secondo il canale del $ K^0_S $ viola la CP-symmetry.

\paragraph{Bariogenesi}

Il modello di bariogenesi proposto da Sakharov richiede la violazione della CP-symmetry per spiegare l'asimmetria tra materia ed antimateria, dunque lo studio di tale simmetria è di fondamentale importanza. Ad oggi, la violazione della CP-symmetry è stata osservata solo per l'interazione debole e non ci sono ancora evidenze che sussista anche per l'interazione forte.

\subsection{T-symmetry}

Data la violazione della CP-symmetry, se la CPT-symmetry deve sempre essere preservata, allora è necessario che i processi che violano la CP-symmetry violino anche la T-symmetry, ovvero che la probabilità del processo sia $ P(t) \neq P(-t) $.\\
Ad oggi, nessuna evidenza sperimentale di violazioni della T-symmetry è stata ancora osservata. Un esempio potrebbe essere la misura di un momento di dipolo elettrico non-nullo per il neutrone.

\subsubsection{Neutron electric dipole moment}

Il momento di dipolo elettrico del neutrone (nEDM), indicato con $ d_n $, dà un'indicazione sulle distribuzioni di cariche elettriche positive e negative all'interno del neutrone: in particolare, $ d_n \neq 0 $  indicherebbe che i centri di tali distribuzioni non coincidono. Ad oggi, la stima migliore non permette di dirimere la questione: $ d_n = (0.0 \pm 1.1) \cdot 10^{-26} e \, \text{cm} $.\\
Un momento di dipolo elettrico permanente nel neutrone violerebbe sia la P-symmetry che la T-symmetry: infatti, $ \hat{\mathcal{P}} d_n = - d_n $ e $ \hat{\mathcal{P}} \mu_n = \mu_n $, mentre $ \hat{\mathcal{T}} d_n = d_n $ e $ \hat{\mathcal{T}} \mu_n = - \mu_n $, dunque sia applicando $ \hat{\mathcal{P}} $ che applicando $ \hat{\mathcal{T}} $ si avrebbe una configurazione per $ d_n $ e $ \mu_n $ diversa da quella iniziale, violando la rispettiva simmetria.

\paragraph{Atomic EDM}

Un'alternativa per ricerche sulla violazione della T-symmetry è la ricerca di momenti di dipolo elettrico permanenti in nuclidi con deformazioni statiche permanenti (dunque non vibrazioni), in particolare nuclei con octupole deformations ($ \virgolette{pear-shaped} $). Negli anni sono stati individuati almeno tre nuclei pesanti con asimmetria rigida di carica ($ \ch{^{222}Ra} $, $ \ch{^{224}Ra} $ e $ \ch{^{226}Ra} $) e la ricerca al momento è incentrata sul come collegare tali misure a quelle di dipolo elettrico permanente nelle particelle.

\section{Modello elettrodebole}

Negli anni '50, a seguito della trattazione rigorosa dell'Elettrodinamica Quantistica (QED) e della formulazione da parte di Yang e Mills della teoria di gauge $ \SUn{2} $ per l'isospin, i fisici teorici iniziarono a notare una certa difficoltà nello svolgere i calcoli relativi all'interazione forte, dunque l'interesse si spostò sull'interazione debole: in particolare, si iniziò a pensare che ad alte energie l'interazione debole e quella elettromagnetica potrebbero essere unificate in un'unica \textit{interazione elettrodebole}.
Un'eventuale teoria elettrodebole, però, deve subire uno spontaneous symmetry breaking a basse energie, poiché i fotoni sono masless e con raggio d'azione infinito, mentre i bosoni che mediano l'interazione debole sono molto massivi ed a short range.

\subsection{Interazione debole}

Sperimentalmente, si trova che le dimensionless coupling constants delle interazioni del Modello Standard sono legate da:
\begin{equation}
	\alpha_s : \alpha_e : \alpha_w = 1 : 10^{-2} : 10^{-6}
	\label{eq:10.8454243}
\end{equation}
L'interazione debole può essere descritta tramite un modello a scambio di particelle, come le altre due interazioni. In particolare, essa è mediata da tre bosoni massivi di spin 1 (come il fotone, solo le proiezioni $ m_s = \pm 1 $ sono ammesse): $ W^{\pm} $ e $ Z^0 $. Sperimentalmente, questi sono stati osservati nel 1983 da Rubbia e van der Meer: essi sono bosoni estremamente massivi, con $ m_{W^{\pm}} = 80.4\gev/c^2 $ e $ m_{Z^0} = 91.2\gev/c^2 $. Di conseguenza, dato che il principio d'indeterminazione impone un range $ \Delta x \sim \frac{\hbar}{m c} $, si trova che l'interazione debole ha il range più corto tra le interazioni: $ \Delta x \sim 10^{-3} \fm $.\\
I tempi d'interazione sono più lunghi rispetto all'interazione forte: per i decadimenti deboli, si ha $ \tau \sim 10^{-8} - 10^{-13}\,\text{s} $. Inoltre, a differenza del bosone $ Z^0 $, i bosoni $ W^{\pm} $ possono cambiare il flavour dei quarks e leptoni con cui interagiscono: di conseguenza, l'interazione debole non conserva i numeri quantici associati alle generazioni di quarks (isospin, stranezza, etc.).\\
Un esempio di decadimento debole è il decadimento del neutrone $ n \rightarrow p e^- \bar{\nu}_e $: un quark down nel neutrone diventa un quark up, rendendo il nucleone un protone, emettendo un bosone $ W^- $ virtuale, il quale decade entro $ 10^{-18}\,\text{m} $ in una coppia elettrone-antineutrino elettronico: confrontando tale distanza con le dimensioni del nucleone, si può dire che il decadimento del bosone $ W^- $ avvenga praticamente nel punto d'emissione stesso.

\subsection{Teoria GWS}

Un modello di unificazione dell'interazione elettromagnetica e di quella debole fu proposto a fine anni '70 da Glashow, Salam e Weinberg. Nella teoria GWS, ad alte energie ($ \sim 100\gev $) l'interazione elettrodebole è mediata da quattro bosoni massless di spin 1; scendendo a basse energie, tre di questi bosoni acquistano massa per rottura spontanea di simmetria. Il processo di spontaneous symmetry breaking è legato all'\textit{angolo di Weinberg} (o weak mixing angle):
\begin{equation}
	\cos \theta_W \defeq \frac{m_{W^{\pm}}}{m_{Z^0}}
	\label{eq:10.2315454}
\end{equation}
Dai dati sperimentali, si trova $ \sin^2 \theta_W = 0.22305 $. Questo processo porta ad un coupling dei bosoni $ W^{\pm} $ e $ Z^0 $ col campo di Higgs (mediato dal bosone di Higgs, strength proporzionale alle masse), il quale è responsabile dell'acquisizione di massa delle particelle.

\subsubsection{Scoperta dei bosoni \texorpdfstring{$ W^{\pm} $}{TEXT} e \texorpdfstring{$ Z^0 $}{TEXT}}

Il valore dell'angolo di Weinberg fu misurato alla fine degli anni '70 in esperimenti di diffusione di neutrini ad alta energia, ricavando che la massa dei bosoni $ W^{\pm} $ e $ Z^0 $ doveva essere $ M \sim 80 - 90 \gev $. Per osservare particelle talmente massive, è necessaria un'energia al centro di massa $ E_{\text{cm}} \equiv \sqrt{s} $ ben al di sopra delle possibilità degli accelleratori esistenti all'epoca. In particolare, il protosincrotrone (SPS) del CERN (di cui Rubbia supervisionava il detector UA1), essendo un accelleratore a bersaglio fisso\footnote{Considerando una collisione $ p + N \rightarrow N' + X $, dove $ N,N' $ sono nucleoni ed $ X $ è la particella da produrre, la soglia di produzione è $ s = (p_p + p_N)^2 = (m_{N'} + M_X)^2 $ (pura massa, no energia cinetica). Per un collisore a bersaglio fisso $ p_p = (E_p,\ve{p}_p) $ e $ p_N = (E_N, \ve{0}) $, dunque:
\begin{equation*}
	s = (p_p + p_N)^2 = m_p^2 + m_N^2 + 2m_N E_p
	\quad \Rightarrow \quad
	E_p = \frac{(m_{N'} + M_X)^2 - m_p^2 - m_N^2}{2m_N} \sim \frac{M_X^2}{2m_p}
\end{equation*}
Per due fasci simmetrici, invece, $ p_p = (E_p,\ve{p}) $ e $ p_{N} = (E_p, -\ve{p}) $, quindi il valore di soglia è diverso:
\begin{equation*}
	s = (p_p + p_N)^2 = (E_p + E_N)^2 = 4E_p^2
	\quad \Rightarrow \quad
	E_p = \frac{m_{N'} + M_X}{2} \sim \frac{M_X}{2}
\end{equation*}
Con un collisore di fasci, a parità di energia del fascio, si producono energie nel centro di massa più elevate.}, avrebbe dovuto accelerare protoni a $ E_p > M^2 / 2m_p \approx 4000\gev $, un ordine di grandezza al di sopra del possibile.\\
Per questo morivo, Rubbia propose di convertire SPS in un anello di collisione protone-antiprotone, così da raggiungere le energie necessarie. La produzione di bosoni $ W^{\pm} $ e $ Z^0 $ in collisioni $ p\bar{p} $ avviene tramite il \textit{processo di Drell-Yan}: quando due adroni ad alte energie scatterano, una coppia quark-antiquark si annichila producendo un bosone virtuale, il quale a sua volta decade in una coppia leptone-antileptone. Esempi di reazioni possibili sono le seguenti:
\begin{equation*}
	u + \bar{d} \rightarrow W^+ \rightarrow e^+ + \nu_e
	\qquad
	\bar{u} +  d \rightarrow W^-\rightarrow e^- + \bar{\nu}_e
	\qquad
	u + \bar{u} \rightarrow Z^0 \rightarrow e^+ + e^- \,/\, \mu^+ + \mu^-
\end{equation*}
I decadimenti in due particelle sono utili poiché permettono di risalire all'energia della particella di partenza col metodo della massa invariante, anche se per quelli che involvono un neutrino vanno utilizzate altre tecniche.\\
Il problema di mantenere sotto controllo le dispersioni dei fasci di protoni ed antiprotoni fu risolto da van der Meer tramite lo \textit{stochastic cooling}: questa tecnica mantiene il fascio ben collimato tramite correzione automatica della traiettoria. In particolare, lungo l'anello sono poste due stazioni di controllo (pick-up probe e kicker): nella prima un sistema di rilevatori permette di misurare la posizione del fascio e la sua eventuale deviazione dalla traiettoria desiderata, inviando un segnale di correzione amplificato alla seconda stazione, la quale corregge la traiettoria.

\paragraph{Collisori di adroni e di leptoni}

Gli adroni, essendo più massivi, permettono di salire facilmente di energia nel centro di massa nei processi di scattering: gli accelleratori di adroni vengono usati principalmente per sondare settori inesplorati di energia per fare scoperte. D'altro canto i leptoni, non avendo struttura interna, permettono misure estremamente precise (sebbene ad energie minori) poiché si conosce perfettamente la particella incidente: gli accelleratori di leptoni servono dunque a misurare con precisione maggiore le proprietà delle particelle scoperte dagli accelleratori di adroni.

\paragraph{LHC e Higgs boson}

A seguito della scoperta dei bosoni mediatori dell'interazione debole, l'unica particella del Modello Standard non ancora scoperta era il bosone di Higgs. Per rilevarlo, sempre al CERN il LEP, in servizio fino al 2000, è stato riconvertito in un accelleratore di protoni, l'LHC, che ha permesso la scoperta del bosone di Higgs nel 2012. In particolare, lungo i $ 26.7\,\text{km} $ dell'LHC sono presenti quattro esperimenti: ATLAS e CMS, impiegati nella ricerca del bosone di Higgs, ALICE, per studi sulla materia nei tempi primordiali dopo il Big Bang, ed LHCb, per studi sulle particelle strane.












\part{Applicazioni}
\pagestyle{body}

\chapter{Astrofisica Nucleare}
\selectlanguage{italian}

\section{Nucleosintesi}

Le abbondanze relative degli elementi nell'Universo (Fig. \ref{iso-distr-univ}) sono correlate alle loro proprietà nucleari. Il processo di formazione di questi elementi è detto \textit{nucleosintesi}.

\subsection{Nucleosintesi del Big Bang}

Si stima che il primo evento di nucleosintesi risalga a $ \sim 3\,\text{min} $ dopo il Big Bang. In particolare, il primo secondo di vita dell'Universo dovrebbe essere stato caratterizzato da un equilibrio tra protoni e neutroni attraverso l'interazione debole:
\begin{equation*}
	n + e^+ \leftrightarrow p + \bar{\nu}_e
	\qquad
	p + e^- \leftrightarrow n + \nu_e
\end{equation*}
I successivi $ 3\,\text{min} $, invece, hanno visto il disaccoppiamento dei neutrini tramite i decadimenti $ \beta^{\pm} $ e la fotodissociazione del deuterio:
\begin{equation*}
	p + n \rightarrow \ch{^2 H} + \gamma \quad (Q = 2.2 \mev)
\end{equation*}
Per i successivi $ 17\,\text{min} $, lasso di tempo nel quale la temperatura dell'Univerto scese da $ T \sim \kev $ a $ T \sim 10\kev $, i fotoni non ebbero più energia sufficiente per portare avanti la reazione di fotodissociazione, dunque le reazioni nucleari divennero efficaci. Questa è la \textit{Big Bang nucleosynthesis} (BBN), al termine della quale essenzialmente tutti i neutroni furono catturati in nuclei atomici: l'Universo a quel punto era composto da $ \ch{H} (75\%) $, $ \ch{He} (25\%) $ e tracce di $ \ch{Li} $ e $ \ch{Be} $.

\subsection{Nucleosintesi stellare}

Se la fonte di energia del Sole fosse esclusivamente la sua energia potenziale gravitazionale, dal teorema del viriale si avrebbe:
\begin{equation*}
	E = -\frac{1}{2} \Omega = - \frac{3}{4} \frac{GM_\odot^2}{R_\odot} \sim 3 \cdot 10^{41} \,\text{J}
	\quad \Rightarrow \quad
	\Delta t = \frac{E}{L_\odot} \sim 3 \cdot 10^{15} \,\text{s} = 24 \,\text{My}
\end{equation*}
Questa stima, però, è in disaccordo con l'età stimata della Terra $ \sim 4.5 \,\text{Gy} $. Negli anni 1920s si capì che la fonte energetica del Sole devono essere invece le reazioni di fusione nucleare: nonostante fosse noto che la temperatura del Sole non è abbastanza alta da permettere questo tipo di reazioni, a causa della barriera Coulombiana, il problema fu risolto da Gamow grazie all'effetto tunnel quantistico (idem alla sua teoria del decadimento $ \alpha $, si veda Sec. \ref{sec-gamow}). Negli anni successivi, Bethe e Weizsäcker studiarono i principali cicli di reazioni nel nucleo solare: la catena pp ed il ciclo CNO (Fig. \ref{sol-cyc}). In particolare, il ciclo CNO è responsabile delle tracce di elementi più pesanti presenti specialmente in stelle di seconda generazione (formate da gas espulso da esplosioni stellari) e diventa dominante alle alte temperature ($ \gtrsim 18 \,\text{MK} $ nel nucleo) a causa della barriera Coulombiana, specialmente in stelle più grandi del Sole: in quest'ultimo, solo l'1\% della produzione di energia è dovuta al ciclo CNO. La prova definitiva della nucleosintesi stellare avvenne con l'osservazione nel 1952 del tecnezio ($ T_{1/2} = 4.2 \,\text{My} $) nelle stelle, a dimostrazione della recente nucleosintesi.

\subsubsection{Bruciamento dell'idrogeno}

Il bruciamento dell'idrogeno avviene in stelle con $ T \gtrsim 10^7 \,\text{K} $ e $ M \gtrsim 0.08 M_\odot $. La principale reazione di fusione dell'idrogeno è:
\begin{equation*}
	4 \ch{^1 H} \rightarrow \ch{^4_2 He} + 2 e^+ + 2 \nu_e + 26.7\mev
\end{equation*}
Assumendo che tutta la produzione energetica solare sia dovuta a questa reazione, si possono calcolare il rate di fusione ed il tempo scala stimato per il bruciamento dell'idrogeno nel Sole:
\begin{equation*}
	N = \frac{L_\odot}{26.7\mev} \sim 10^{38} \,\text{s}^{-1}
	\quad \Rightarrow \quad
	M_{4\ch{^1 H}} = 4 m_p \cdot N \sim 6.4 10^{14} \,\text{g} \,\text{s}^{-1}
	\quad \Rightarrow \quad
	\Delta t = \frac{10\% M_\odot}{M_{4\ch{^1 H}}} \sim 10^{10} \,\text{y}
\end{equation*}
Questo valore è in accordo con l'età stimata della Terra.

\subsubsection{Bruciamento dell'elio}

Il bruciamento dell'elio avviene in stelle con $ T \gtrsim 10^8 \,\text{K} $ e $ M \gtrsim 0.4 M_\odot $, tipicamente a seguito della contrazione (e aumento della temperatura) dovuta al completo bruciamento del nucleo d'idrogeno, che a questo punto è un nucleo d'elio.\\
La fusione dell'elio è data da $ \ch{^4_2 He} + \ch{^4_2 He} \rightarrow \ch{^8_4 Be} $: quest'ultimo è un nuclide instabile che decade $ 2\alpha $ ($ T_{1/2} \sim 10^{-16}\,\text{s} $), dunque si viene a creare una concentrazione d'equilibrio (dinamico) di $ \ch{^8_2 Be} $. Inoltre, può avvenire un processo di cattura $ \ch{^8_4 Be} + \ch{^4_2 He} \rightarrow \ch{^{12}_6 C} $ in cui $ \ch{^{12}_6 C} $ ha un'energia prossima allo stato risonante a $ 7.656\mev $ e $ J^\pi = 0^+ $ detto \textit{stato di Hoyle}: questo stato si trova $ 285\kev $ al di sopra della configurazione $ \ch{^8 Be} + \alpha $ e decade principalmente $ \alpha $, sebbene abbia una bassa probabilità ($ \sim 0.04\% $) di decadere nel ground state. L'energia $ 285\kev $ corrisponde alla temperatura stellare $ \approx 2.5 \cdot 10^8 \,\text{K} $: il $ \ch{^{12} C} $ è un nucleo fortemente legato e fondamentale in quanto punto di partenza per la nucleosintesi di nuclei più pesanti.

\subsubsection{Stadi stellari avanzati}
\label{sec-stell-av}

Dopo bruciamento dell'$ \ch{He} $, la stella ha un nucleo di $ \ch{C} $ e $ \ch{O} $ (formato come $ \ch{^{12}C} \left( \alpha, \gamma \right) \ch{^{16}O} $, dunque la stella collassa e, se $ T \gtrsim 0.85 \cdot 10^9 \,\text{K} $ e $ M \gtrsim 8 M_\odot $, può innescare la combustione del carbonio, seguita dal bruciamento di neon e ossigeno (se $ T \gtrsim 1.5 \cdot 10^9 \,\text{K} $).\\
A seguito di ciò, gli elementi più abbondanti sono $ \ch{^{28}Si} $ e $ \ch{^{32}S} $: la barriera Coulombiana per la fusione di $ \ch{^{28}Si} $ è troppo alta, dunque diventa efficacie prima la fotodissociazione (se $ T \gtrsim 3 \cdot 10^9 \,\text{K} $): poiché non tutti i nuclei di $ \ch{^{28}Si} $ si dissociano, i rimanenti catturano le particelle leggere prodotte e formano nuclei più pesanti.\\
Le reazioni di fusione nucleare non procedono oltre il $ \ch{^{56}Fe} $, picco di stabilità per la average binding energy per nucleon, poiché il processo sarebbe endotermico: i nuclidi con $ A > 60 $ possono essere formati tramite cattura neutronica e successivo decadimento $ \beta $:
\begin{equation*}
	\ch{^A X} + n \rightarrow \ch{^{A+1} X} + \gamma
	\qquad
	\ch{^{A+1} X} + n \rightarrow \ch{^{A+1} Y} + e^- + \bar{\nu}_e
\end{equation*}
Detti $ \lambda_n $ e $ \lambda_\beta $ il tasso di cattura neutronica e di decadimento $ \beta $:
\begin{itemize}
	\item $ \lambda_n \ll \lambda_\beta $: il nucleo $ \ch{^{A+1} X} $ decade prima che ci sia tempo di catturare un secondo neutrone, poiché la cattura neutronica avviene con bassa frequenza e si parla di \textit{s-process} (slow);
	\item $ \lambda_n \gg \lambda_\beta $: sono possibili catture neutroniche multiple che spingono il nuclide sempre più lontano dalla valle di stabilità e si parla di \textit{r-process} (rapid).
\end{itemize}

\section{Esperimenti}

\subsection{Picco di Gamow}

Si consideri una reazione nucleare del tipo $ 0 + 1 \rightarrow 2 + 3 $. Il \textit{rate di reazione termonucleare} è definito come:
\begin{equation}
	r_{0,1} \defeq N_0 N_1 v \sigma(v)
	\label{eq:13.1}
\end{equation}
dove $ N_j $ è la densità numerica della $ j $-esima specie nucleare, $ v $ è la velocità relativa di $ 0 $ e $ 1 $ e $ \sigma(v) $ la sezione d'urto della reazione. $ r_{0,1} $ esprime un numero di reazioni per unità di tempo e di volume. Nelle stelle, le particelle sono in equilibrio termodinamico e seguono la distribuzione di Maxwell-Boltzmann:
\begin{equation}
	P(v) dv = \left( \frac{m_{0,1}}{2\pi k T} \right)^{3/2} e^{-m_{0,1} v^2 / 2kT} 4\pi v^2 dv
	\label{eq:13.2}
\end{equation}
Il rate di reazione termonucleare sarà quindi:
\begin{equation}
	r_{0,1} = N_0 N_1 \int_0^\infty v P(v) \sigma(v) dv \equiv N_0 N_1 \braket{\sigma v}_{0,1}
	\label{eq:13.3}
\end{equation}
Convertendo dalla velocità relativa all'energia del centro di massa:
\begin{equation*}
	P(E) dE = \frac{2}{\sqrt{\pi}} (kT)^{-3/2} \sqrt{E} e^{-E / kT} dE
	\quad \Rightarrow \quad
	\braket{\sigma v}_{0,1} = \sqrt{\frac{8}{\pi m_{0,1}}} (kT)^{-3/2} \int_0^\infty E \sigma(E) e^{-E / kT} dE
\end{equation*}
Detta $ E_c = Z_0 Z_1 \frac{e^2}{R} \sim 1 \mev $ il potenziale di repulsione Coulombiana, la sezione d'urto per $ E < E_c $ ha una dipendenza energetica $ \sim E^{-1} $. Inoltre, per effetto tunneling (analogo al decadimento $ \alpha $), si ha il coefficiente di trasmissione:
\begin{equation}
	\hat{T} \approx \exp \left[ - \frac{2\pi}{\hbar} \sqrt{\frac{m}{2E}} Z_0 Z_1 e^2 \right] \equiv e^{-2\pi \eta}
	\label{eq:13.4}
\end{equation}
La sezione d'urto è solitamente espressa come:
\begin{equation}
	\sigma(E) \equiv \frac{1}{E} e^{-2\pi \eta} S(E)
	\label{eq:13.5}
\end{equation}
dove $ S(E) $ è una funzione che varia lentamente con l'energia per una reazione non risonante, mettendo in risalto la forza d'interazione nucleare e gli effetti della struttura nucleare. Il rate di reazione termonucleare risulta essere dunque:
\begin{equation}
	r_{0,1} = \sqrt{\frac{8}{\pi m_{0,1}}} \frac{N_0 N_1}{(kT)^{3/2}} \int_0^\infty S(E) e^{-2\pi \eta} e^{-E/kT} dE
	\label{eq:13.6}
\end{equation}

\begin{figure}
	\centering
	\includegraphics[width = 0.60 \textwidth]{gamow-peak.png}
	\caption{Gamow peak.}
	\label{gamow-peak}
\end{figure}

Fig. \ref{gamow-peak} mostra l'andamento di questa funzione dell'energia: risulta evidente che le reazioni avvengono efficaciemente soltanto in un certo intervallo di energie, detto \textit{picco di Gamow}. Si vede, inoltre, che la probabilità di tunneling diminuisce esponenzialmente al diminuire dell'energia, e con essa la sezione d'urto.

\subsection{Tecniche di laboratorio}

Un setup sperimentale tipico è costituito da:
\begin{enumerate}
	\item un accelleratore in grado di generare un fascio ad alta intensità e stabilità e con un basso spread di energia;
	\item un target ad elevata purezza e stabile sotto irraggiamento;
	\item un rilevatore con il miglior compromesso tra efficienza e risoluzione energetica, capace di rilevare prodotti di reazione tipici come $ \gamma $, $ p $, $ n $ ed $ \alpha $.
\end{enumerate}
Tipicamente, l'accelleratore è in grado di generare un flusso di proiettili $ N_p \sim 10^{14} \,\text{s}^{-1} $ (per correnti $ i \sim 0.1 \,\text{mA} $, il target (usually solid-state) presenta una densità di $ N_t \sim 10^{18} \,\text{cm}^{-2} $ e la sezione d'urto del processo è $ \sigma \sim 1 \,\text{pbarn} = 10^{-36}  \,\text{cm}^2 $, dunque, detta $ \varepsilon \in \left( 0.1 , 1 \right) $ l'efficienza del rilevatore, il rate di conteggi dell'esperimento è:
\begin{equation*}
	C = N_p N_t \sigma \varepsilon \sim 0.004 - 0.4 \,\text{h}^{-1}
\end{equation*}
Le principali sorgenti di rumore nelle misurazioni sono:
\begin{enumerate}
	\item radioattività ambientale: catene di $ \ch{^{235} U} $, $ \ch{^{238} U} $, $ \ch{^{232} Th} $ e $ \ch{^{40} K} $;
	\item raggi cosmici: principalmente muoni al livello del mare;
	\item neutroni: prodotti in reazioni $ \left( \alpha, n \right) $ e di spallazione.
\end{enumerate}
In particolare, i radioisotopi influiscono sulle misure a basse energie, mentre raggi cosmici e neutroni su quelle ad alte energie. Per migliorare il rapporto segnale/rumore, si può:
\begin{enumerate}
	\item aumentare il segnale: aumentando la corrente del fascio (ma degrada il target), aumentando la densità del bersaglio (ma il fascio perde energia ed avviene lo straggling) o migliorando l'efficienza del rilevatore;
	\item ridurre il fondo di rumore: con schermature attive o passive, con tecniche di soppressione del fondo o ponendo gli esperimenti underground.
\end{enumerate}
In particolare, i laboratori sotterranei costituiscono un ottimo metodo per ridurre il fondo di rumore, poiché le rocce riescono a sopprimere efficacemente il flusso di raggi cosmici. Rimane il problema dei radionuclidi, il quale può essere risolto prestando attenzione ai materiali scelti per la costruzione degli apparati (es.: piombo romano). L'Abruzzo ospita i più grandi laboratori sotterranei di fisica al mondo, i Laboratori Nazionali del Gran Sasso, con all'interno vari esperimenti: tra questi, l'esperimento LUNA (Laboratory for Underground Nuclear Astrophysics), che si occupa di studiare le reazioni attive in stelle massicce ed ambienti esplosivi.












\end{document}
