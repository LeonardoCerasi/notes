\selectlanguage{italian}

Il Modello Standard è una teoria incompleta, in quanto non riesce a spiegare, ad esempio, i valori delle masse delle particelle, le quali rimangono dei parametri liberi del modello, oltre a non includere affatto la gravità.

\section{Materia oscura}

Le osservazioni astronomiche indicano che la materia barionica costituisce soltanto il 4\% della materia totale nell'Universo: il restante 96\% è diviso tra materia oscura (22\%) ed energia oscura (74\%).

\subsection{Evidenze sperimentali}

Le prime evidenze dell'esistenza della materia oscura si ebbero negli anni 1930s, quando Zwicky e Smith osservarono le velocità delle galassie all'interno di clusters e si resero conto che queste erano $ 10-100 $ volte maggiori a quelle calcolate dalla massa visibile, in particolare ai bordi dei clusters, suggerendo l'esistenza di ulteriore massa per spiegare l'eccessiva forza gravitazionale; questa, però, non era un'evidenza abbastanza convincente, in quanto misure di questo tipo erano prone ad errori sistematici dovuti alla difficoltà di distinguere le galassie interne al cluster da quelle esterne e da quelle sullo sfondo.\\
Si ricordi che un sistema many-body gravitazionalmente legato ubbidisce al \textit{teorema del viriale}:
\begin{equation}
	2 \braket{K} = - \braket{U_g}
	\quad \Rightarrow \quad
	\braket{v^2} \approx G M \braket{r^{-1}}
	\label{eq:11.1}
\end{equation}
Inoltre, la velocità di una galassia rispetto alla Terra può essere misurata dal Doppler shift delle sia linee spettrali atomiche, note dalle misure in laboratorio:
\begin{equation}
	z \defeq \frac{\lambda_{\text{observed}} - \lambda_{\text{emitted}}}{\lambda_{\text{emitted}}} = \sqrt{\frac{1 + v/c}{1 - v/c}} - 1 \approx \frac{v}{c}
	\label{eq:11.2}
\end{equation}
dove l'approssimazione è valida per velocità non-relativistiche $ v \ll c $. L'effetto Doppler genera un blueshift $ z < 0 $ per oggetti che si avvicinano ed un redshift $ z > 0 $ per oggetti che si allontanano.\\
Evidenze empiriche più solide dell'esistenza della materia oscura si ebbero negli anni 1970s, quando Vera Rubin e Ken Freeman osservarono le rotation curves di varie galassie e notarono che le velocità delle stelle ai bordi delle galassie non potevano essere spiegate dalla sola materia osservata: l'andamento funzionale previsto dal modello Kepleriano è $ v^2 = \frac{GM}{r} $ ($ F_c = F_g $), ma ciò non è quello che si evince dai osservati, i quali invece riportano delle rotation curve essenzialmente piatte. L'unica spiegazione possibile, senza modificare la teoria della gravità, è che ciascuna galassai sia avvolta da un dark matter halo molto più esteso della galassia stessa e 10 volte più massivo.

\subsubsection{Gravitational lensing}

Il gravitational lensing è un fenomeno previsto dalla Relatività Generale ed è causato dalla deformazione della luce proveniente da una sorgente lontana ad opera un oggetto massimo lungo il suo percorso: infatti, analogamente alla deflessione di corpi massivi, anche la luce viene deviata dalla presenza di campi gravitazionali. Il gravitational lensing, dunque, può essere utilizzato per studiare la distribuzione di massa dell'oggetto che causa la deflessione della luce.\\
Un caso particolare di gravitational lensing è l'\textit{Einstein ring}, fenomeno che avviene col perfetto allineamento di sorgente, deflettore e Terra: l'immagine distorta della sorgente prenderà la forma di un anello attorno al corpo massivo che causa il lensing. In particolare, c'è una relazione tra la massa $ M $ di tale corpo massivo e la dimensione angolare $ \theta $ (in radianti) dell'Einstein ring:
\begin{equation}
	\theta = \sqrt{\frac{4GM}{c^2} \frac{D_{LS}}{D_S D_L}}
	\label{eq:11.3}
\end{equation}
dove $ D_L $, $ D_S $ e $ D_{SL} $ sono le angular diameter distances\footnote{distanza definita a partire dalla dimensione fisica $ x $ di un oggetto e dalla sua dimensione angolare $ \theta $ come $ d = \frac{x}{\theta} $.} rispettivamente tra Terra e lente, tra Terra e sorgente e tra lente e sorgente.\\
A partire da questo fenomeno, nel 1937 Zwicky ipotizzò la possibilità di misurare la distribuzione di materia oscura in un galaxy cluster tramite lo studio delle galassie in background che subiscono lensing dal cluster stesso: questo metodo d'osservazione è stato realizzato per la prima volta a metà anni 1990s. Anche studiando le collisioni tra galaxy clusters si è trovato che il centro di massa misurato dal gravitational lensing non corrisponde a quello misurato per la materia visibile, andando a confermare l'ipotesi dell'esistenza della materia oscura.

\subsection{Ipotesi teoriche}

Tutte i dati sperimentali finora rilevati sulla materia oscura sono di natura gravitazionale, dunque non delineano la sua natura a livello particellare: in particolare, questi dati sono fittati consistentemente da modelli che prevedono costituenti di materia oscura con masse da $ 10^{-22} \ev/c^2 $ a $ 100 M_\odot $.\\
Negli anni, si sono susseguiti e si susseguono numerosi esperimenti per sondare questi 80 ordini di grandezza, e molte ipotesi sono state ruled out: un esempio è l'ipotesi dei neutrini massivi ($ \sim 10-100 \ev $), smentita dall'esperimento KATRIN. Ad oggi, le principali ipotesi, denominate \textit{cold dark matter}, sono due:
\begin{enumerate}
	\item decadimento del neutrone in un dark sector, hintato da una discrepanza di $ 8\,\text{s} $ in misure del tempo di decadimento del neutrone, sebbene poco probabile per le conseguenze sulle proprietà delle stelle di neutroni;
	\item WIMPs (weakly-interacting massive particles), considerate l'ipotesi più probabile.
\end{enumerate}
I principali metodi sperimentali utilizzati per la misura delle proprietà della materia oscura sono: esperimenti di direct detection (es.: XENON1T at LNGS, DUNE in USA), indirect detection (es.: FERMI satellite, LIGO/VIRGO/KAGRA) e collisori di particelle (es.: LHC at CERN). Ad oggi, per quanto riguarda le WIMPs, si hanno soltanto dei limiti superiori alla loro cross-section in vari range di massa, come si vede in Fig. \ref{dark-matter}: il \textit{neutroni floor} costituisce un limite inferiore teorico per i modelli di WIMPs, interpretato comunemente come il punto in cui il segnale da materia oscura si confonde col simile background neutrinico.

\begin{figure}
	\centering
	\includegraphics[width = 0.70 \textwidth]{dark-matter.png}
	\caption{Lower limit of WIMPs' interaction cross-section.}
	\label{dark-matter}
\end{figure}

\section{Energia oscura}

In cosmologia, l'energia oscura è una forma di energia proposta come causa dell'espansione accelerata dell'Universo. Questa energia oscura dovrebbe avere effetti solo su larga scala e si stima che abbia una densità di $ 6 \cdot 10^{-10} \,\text{J} \,\text{m}^{-3} \approx 7 \cdot 10^{-30} \,\text{g} \,\text{cm}^{-3} $, molto inferiore alla densità della materia barionica o della materia oscura: nonostante ciò, essendo distribuita uniformemente nell'Universo, essa è globalmente dominante.\\
Dopo aver formulato la teoria della Relatività Generale, Einstein si accorse che le soluzioni alle equazioni di campo prevedevano un Universo in espansione o in collasso, sebbene all'epoca si supponesse che l'Universo fosse statico: di conseguenza, introdusse un termine costante, denominato \textit{costante cosmologica} ed interpretata come la densità d'energia del vuoto, tale da cancellare gli effetti gravitazionali su larga scala.\\
Nel 1929, però, Hubble osservò invece che l'Universo sembra essere proprio in espansione: in particolare, dalle misure di redshift ricavò la \textit{legge di Hubble} per la velocità di allontanamento delle galassie in funzione della loro distanza dalla Terra:
\begin{equation}
	v = H_0 d
	\label{eq:11.4}
\end{equation}
dove $ H_0 \approx 70 \,\text{km}/\text{s} / \text{Mpc} $ è nota come costante di Hubble. Si vede dunque che più una galassia è lontana, più si allontana velocemente: questo è indice del fatto che l'espansione dell'Universo omogenea ed isotropa, in accordo col \textit{principio cosmologico}. Un'importante conseguenza dell'osservazione dell'espansione dell'Universo è che, ad un certo punto nel tempo, esso deve essere stato confinato in un singolo punto: questo viene detto Big Bang.\\
A seguito di questa scoperta, Einstein ritrattò sulla costante cosmologica, giudicata un errore. Negli anni 1960s, però, calcoli di QFT mostrarono che la densità di energia del vuoto non è zero, ma dovrebbe avere un valore estremamente grande e fisicamente insensato, a causa delle fluttuazioni quantistiche del vuoto: questo è tuttora noto come il problema della costante cosmologica.

\subsection{Modello di Friedmann}

L'Universo in espansione è modellato dalle equazioni di Friedmann, ricavate da quelle di Einstein, e le soluzioni sono di tre tipi:
\begin{enumerate}
	\item curvatura positiva ($ k = +1 $): l'Universo è un universo chiuso e ad un certo punto nel tempo inizierà a contrarsi, fino a ridursi di nuovo ad un punto;
	\item curvatura nulla ($ k = 0 $): l'Universo continuerà ad espandersi, ma l'espansione decellererà fino ad arrestarsi in un tempo infinito;
	\item curvatura negativa ($ k = -1 $): l'Universo continuerà ad espandersi, accellerando sempre di più.
\end{enumerate}
Il parametro di curvatura dell'Universo è legato al valore della densità media dell'Universo rispetto alla sua densità critica $ \rho_c \approx 9.47 \cdot 10^{-27} \,\text{kg} \, \text{m}^{-3} $: $ k = +1 $ se $ \rho > \rho_c $,  $ k = 0 $ se $ \rho = \rho_c $ e $ k = -1 $ se $ \rho < \rho_c $. È utile definire il \textit{parametro di densità} $ \Omega \defeq \rho / \rho_c $, che si può scrivere come:
\begin{equation}
	\Omega = \Omega_m + \Omega_r + \Omega_\Lambda
	\label{eq:11.5}
\end{equation}
$ \Omega_m $ è il contributo della materia barionica e di quella oscura, $ \Omega_r $ quello delle particelle relativistiche (fotoni e neutrini) e $ \Omega_\Lambda $ quello dovuto all'energia oscura. Le attuali stime danno $ \Omega = 1.02 \pm 0.02 $, dunque l'Universo dovrebbe essere prossimo alla densità critica.

\subsection{Evidenze sperimentali}

Le evidenze che suggeriscono l'esistenza dell'energia oscura sono indirette, ma sono tutte indipendenti tra loro:
\begin{enumerate}
	\item misure di wave patterns su larga scala della densità di massa nell'Universo;
	\item misure di distanza con supernovae di tipo Ia\footnote{sistemi binari con una consistente peak luminosity, dunque in grado di misurare con precisione le distanze tramite redshift (fino a $ 1000 \,\text{Mpc} $). Con questo metodo si è osservata l'espansione accellerata dell'Universo, valsa il Nobel nel 2011.};
	\item assenza di curvatura globale dalla CMB.
\end{enumerate}
In particolare, l'assenza di curvatura, ovverosia la globale piattezza dell'Universo, non può essere spiegata dalle sole materia barionica e oscura, le quali forniscono soltanto $ \approx 1/3 $ dell'energia totale richiesta. Per spiegare ciò e l'espansione accellerata dell'Universo, i cosmologi hanno proposto l'esistenza dell'energia oscura come legata alla costante cosmologica, in quanto omogenea ed isotropa nell'Universo.










