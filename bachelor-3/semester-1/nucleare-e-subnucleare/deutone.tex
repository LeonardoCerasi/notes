\selectlanguage{italian}

\section{Interazioni nucleone-nucleone}

\subsection{Bosoni e fermioni}

Si ricordi che un generico operatore di momento angolare soddisfa la condizione di commutazione:
\begin{equation}
	[\hat{L}_i,\hat{L}_j] = i\hbar \sum_{k = 1}^{3} \epsilon_{ijk} \hat{L}_k
	\label{eq:4.1}
\end{equation}
Diagonalizzando $ \hat{L}^2 $ e $ \hat{L}_z $:
\begin{equation}
	\hat{L}^2 \ket{\psi} = \hbar \ell (\ell + 1) \ket{\psi}
	\label{eq:4.2}
\end{equation}
\begin{equation}
	\hat{L}_z \ket{\psi} = \hbar m \ket{\psi}
	\label{eq:4.3}
\end{equation}
con $ m = -\ell, \dots, \ell $. Il momento angolare può essere sia orbitale che di spin (intrinseco): questo permette di distinguere particelle fermioniche e bosoniche.\\
Le particelle con spin semi-intero sono dette \textit{fermioni}, quelle con spin intero \textit{bosoni}: in maniera grossolana, si può dire che i fermioni costituiscono la materia, mentre i bosoni mediano le interazioni, ma ci sono delle eccezioni. Una delle principali proprietà è che, in sistemi di particelle identiche, le funzioni d'onda bosoniche sono simmetriche per scambio di particelle, mentre quelle fermioniche sono antisimmetriche: una conseguenza importante è il \textit{principio di esclusione di Pauli}, il quale afferma che due fermioni identici non possono avere gli stessi numeri quantici (ovvero avere lo stesso stato).\\
Si ricordi che, in generale, quando due momenti angolari interagiscono, i numeri quantici si compongono secondo:
\begin{equation}
	\abs{\ell_1 - \ell_2} \le j \le \ell_1 + \ell_2
	\label{eq:4.4}
\end{equation}
Ad esempio, un sistema di due fermioni può avere spin 0 o 1.

\subsection{Caratteristiche dell'interazione}

Non si conosce molto bene l'interazione tra nucleoni. Dallo studio della binding energy (vedere Fig. \ref{bind-en}) si evince che la binding energy per nucleone ($ B(A,Z)/A $) satura a circa $ 8\mev $: questo è una conseguenza del fatto che l'interazione tra nucleoni è a corto range. Infatti, le interazioni a lungo range scalano come $ E \sim \frac{1}{2} A (A - 1) $, come avviene ad esempio per quella coulombiana, mentre le interazioni a corto range come $ E \sim A $, dato che sono le particelle immediatamente vicine interagiscono tra loro.\\
Una stima cruda del range dell'interazioni tra nucleoni è data dalle dimensioni della particella $ \alpha $, che ha un daimetro di circa $ 1.5\fm $: si evince che il range dell'interazioni è circa tra $ 1\fm $ e $ 2\fm $.\\
In maniera più precisa, si può determinare il range a partire dal mediatore dell'interazione: dato che durante un'interazione la particella mediatrice deve esistere per un tempo $ \Delta t = \Delta r / c $, dove $ \Delta r $ è il range, dal principio d'indeterminazione si ha che $ \Delta t \Delta E \sim \hbar $, dunque se il mediatore ha massa $ m $:
\begin{equation}
	\Delta r \sim \frac{\hbar}{m c}
	\label{eq:4.5}
\end{equation}
Per l'interazione elettromagnetica, ad esempio, dato che il mediatore è il fotone, che ha massa nulla, il range è infinito. L'interazione tra nucleoni, invece, è mediata da mesoni: il mesone più leggero è il pione, la cui massa $ m_{\pi} \approx 135\mev $ determina un range $ \Delta r \approx 1.4 \fm $, in accordo con la stima iniziale: mesoni più pesanti medieranno l'interazione con range più corti.\\
Si ricordi, inoltre, che i nucleoni non sono oggetti indivisibili, bensì hanno una struttura a quark: per questo, l'interazione tra nucleoni non è altro che una risultante dell'interazione tra quark, sebbene questa sia ancora eccessivamente complessa da studiare. L'unica conclusione qualitativa che si può ottenere è che il potenziale d'interazione tra nucleoni è analogo al potenziale inter-molecolare (potenziale di Lennard-Jones), il quale può essere modellato come una buca di potenziale sferica.

\section{Il deutone}

Il deutone è il nucleo di deuterio, ovvero il nuclide $ \ch{^2_1 H} $: questo è il sistema nucleonico più semplice, composto da un neutrone ed un protone.\\
La sua binding energy è pari a $ B = 2.22457\mev $, calcolabile tramite l'energia del fotone nella reazione $ n + p^+ \rightarrow d^+ + \gamma $: questo è un sistema debolmente legato, dato che $ B / A \approx 1 \mev $, poiché l'interazione non raggiunge la saturazione (essendoci solo due nucleoni).\\
Tramite electron scattering è possibile stimare la distanza tra i due nucleoni, trovando il valore medio $ \braket{r^2}^{1/2} = 1.963 \pm 0.04 \fm $: dato che protone e neutrone sono larghi circa $ 0.6\fm $ ed il range d'interazione è circa $ 1.4\fm $, non solo i due nucleoni sono poco legati, ma sono anche abbastanza lontani tra loro.\\
Il sistema ha inoltre momento angolare totale $ J^{\pi} = 1^+ $ ed isospin $ I_3 = 0 $: è interessante notare che lo stato fondamentale del deutone ha momento angolare orbitale $ \ell = 0 $, mentre lo spin è $ s = 1 $ e non ci sono stati eccitati. Se però si considera che protone e neutrone hanno spin $ \frac{1}{2} $, si vede che i valori possibili di spin per il deutone dovrebbero essere $ s = 0 $ (singoletto, $ m_s = 0 $) ed $ s = 1 $ (tripletto, $ m_s = -1,0,1 $): ciò evidenzia come l'interazione tra nucleoni sia spin-dependent, in modo da essere maggiormente attrattiva nella configurazione $ s = 1 $.\\
Infine, si trovano il momento di dipolo magnetico $ \mu_d = 0.857406 \mu_N $ ed il momento di quadrupolo elettrico $ Q_d = 0.2859 e \fm^2 $: quest'ultimo indica che i due nucleoni non orditano in maniera sferica, poiché altrimenti si avrebbe solo momento di dipolo.

\subsection{Modello semplificato}

È possibile schematizzare i due nucleoni in orbita tra loro come una singola particella di massa ridotta $ \mu = (m_p^{-1} + m_n^{-1})^{-1} $ in orbita attorno al centro di massa del sistema e soggetta ad un potenziale d'interazione, il quale può essere studiato come una buca di potenziale sferica. In particolare:
\begin{equation}
	V(r) =
	\begin{cases}
		0 & r < a \\
		-V_0 & r > a
	\end{cases}
	\label{eq:4.6}
\end{equation}
Essendo questo un potenziale centrale a simmetria sferica, la funzione d'onda del sistema può essere scritta come:
\begin{equation}
	\psi(r,\theta,\phi) = \frac{u(r)}{r} Y_{\ell,m}(\theta,\phi)
	\label{eq:4.7}
\end{equation}
L'equazione di Schrödinger diventa quindi:
\begin{equation}
	- \frac{\hbar^2}{2\mu} \frac{d^2 u(r)}{dr^2} + \left[ V(r) + \frac{\hbar^2 \ell (\ell + 1)}{2\mu r^2} \right] u(r) = \varepsilon u(r)
	\label{eq:4.8}
\end{equation}
dove $ \varepsilon $ è l'energia dello stato considerato. Risolvendo si trova:
\begin{equation}
	u(r) =
	\begin{cases}
		A \sin (k_1 r) + B \cos (k_2 r) & r < a \\
		C e^{-k_2 r} + D e^{k_2 r} & r > a
	\end{cases}
	\qquad
	k_1 = \sqrt{\frac{2\mu}{\hbar^2} (\varepsilon - V_0)}, \, k_2 = \sqrt{\frac{2\mu \abs{\varepsilon}}{\hbar^2}}
	\label{eq:4.9}
\end{equation}
Dato che $ u(r) $ deve tendere a $ 0 $ per $ r \rightarrow 0 $ e $ r \rightarrow \infty $, si ha $ B = D = 0 $. La continua derivabilità in $ r = a $ impone invece la condizione:
\begin{equation}
	k_1 \cot (k_1 a) = - k_2
	\label{eq:4.10}
\end{equation}
Inserendo in questa equazione la binding energy ed il raggio del deutone trovati sperimentalmente, si ottiene il valore $ V_0 \approx -35\mev $: questa è ovviamente solo una stima, dato che il modello è molto semplice, ed è efficace solo per il caso $ s = 1 $, dato che per $ s = 0 $ non ci sono stati legati.

\subsection{Momento di dipolo magnetico}

Il momento magnetico del deutone può essere espresso come:
\begin{equation}
	\ve{\mu}_s = g_p \mu_N \ve{S}_p + g_n \mu_N \ve{S}_n \equiv g_s \mu_N \ve{S}
	\label{eq:4.11}
\end{equation}
Proiettando nella direzione dello spin totale:
\begin{equation*}
	\begin{split}
		g_s S^2
		&= g_s s (s + 1) = g_p \ve{S}_p \cdot \ve{S} + g_n \ve{S}_n \cdot \ve{S} = g_p (S_p^2 + \ve{S}_p \cdot \ve{S}_n) + g_n (S_n^2 + \ve{S}_n \cdot \ve{S}_p)\\
		&= g_p (s_p (s_p + 1) + s_p s_n) + g_n (s_n (s_n + 1) + s_n s_p) = (g_p s_p + g_n s_n) (s_p + s_n + 1)
	\end{split}
\end{equation*}
Nello stato fondamentale $ s_p = s_n = \frac{1}{2} $ e $ s = 1 $, dunque, ricordando i valori $ g_p = +5.585691 $ e $ g_n = -3.826084 $:
\begin{equation}
	g_s = \frac{g_p + g_n}{2} = +0.879804
	\label{eq:4.12}
\end{equation}
Il valore trovato considerando solo l'accoppiamento degli spin non corrisponde a quello sperimentale ($ g_s = +0.857406 $): evidentemente anche il momento angolare orbitale contribuisce al momento di dipolo magnetico e ciò è un'altra indicazione della non sfericità dell'orbita.\\
Classicamente si avrebbe $ \ve{\mu} = \frac{q}{2m} \ve{L} $, mentre quanto-meccanicamente $ \ve{\mu} = \frac{q}{2m} \frac{\hbar}{2} \ve{L} = \frac{1}{2} \mu_N \ve{L} $, quindi:
\begin{equation}
	\ve{\mu} = g_s \mu_N \ve{S} + \frac{1}{2} \mu_N \ve{L} \equiv g_d \mu_N \ve{J}
	\label{eq:4.13}
\end{equation}
Proiettando lungo il momento angolare totale:
\begin{equation*}
	\begin{split}
		g_d J^2
		&= g_d j (j + 1) = \frac{1}{2} \ve{L} \cdot (\ve{L} + \ve{S}) + g_s \ve{S} \cdot (\ve{L} + \ve{S}) = \frac{1}{2} L^2 + g_s S^2 + \left( \frac{1}{2} + g_s \right) \ve{L} \cdot \ve{S}
	\end{split}
\end{equation*}
Ricordando che $ \ve{L}\cdot\ve{S} = \frac{1}{2} (J^2 - L^2 - S^2) $:
\begin{equation}
	g_d = \frac{1}{2} \left( \frac{1}{2} + g_s \right) + \frac{1}{2} \frac{\ell (\ell + 1) - s (s + 1)}{j (j + 1)} \left( \frac{1}{2} - g_s \right)
	\label{eq:4.14}
\end{equation}
Dato che sperimentalmente si trova che lo stato fondamentale del deutone ha $ J^{\pi} = 1^+ $ ed in generale $ \abs{\ell - s} \le j \le \ell + s $ e $ \pi = (-1)^{\ell} $, lo stato fondamentale è sovrapposizione di soli due stati, esprimibili come $ \ket{\ell,s,j} = \ket{0,1,1} $ e $ \ket{\ell,s,j} = \ket{2,1,1} $: per il primo $ g_0 = g_s = +0.879804 $, mentre per il secondo $ g_2 = \frac{3}{4} - \frac{1}{2} g_s = +0.310098 $. Si ha quindi:
\begin{equation}
	\ket{\psi_d} = a_0 \ket{0,1,1} + a_2 \ket{2,1,1}
	\label{eq:4.15}
\end{equation}
È possibile trovare i due coefficienti grazie alla normalizzazione e all'expectation value di $ g_d $ (sperimentale):
\begin{equation*}
	\begin{cases}
		\abs{a_0}^2 + \abs{a_2}^2 = 1 \\
		g_0 \abs{a_0}^2 + g_2 \abs{a_2}^2 = g_d
	\end{cases}
	\quad\Longrightarrow\quad
	\abs{a_0}^2 = 0.96,\, \abs{a_2}^2 = 0.04
\end{equation*}
Lo stato fondamentale del deutone è una sovrapposizione: al 96\% è sferico, mentre al 4\% è allungato.

\subsection{Momenti di multipolo elettrici}

In generale, il momento di $ n $-polo elettrico è definito come integrale pesato della densità di carica:
\begin{equation}
	K_n \defeq \int_V d^3\ve{r} \, r^{n-1} P_{n-1}(\cos \theta) \rho_(\ve{r})
	\label{eq:4.16}
\end{equation}
dove $ P_n(\cos \theta) $ sono i polinomi di Legendre.\\
Sperimentalmente, si trova che se un sistema quantistico ha un parità definita (ovvero è un numero quantico, dunque la densità non cambia quando si invertono gli assi) il momento di dipolo è nullo: c'è molta ricerca in corso per verificare se ciò è vero per ogni particella elementare (ad esempio tramite misure del momento di dipolo del neutrone).\\
Un parametro importante è il momento di quadrupolo:
\begin{equation}
	Q = \int_V d^3\ve{r} \, r^2 (3\cos^2 \theta - 1) \rho(\ve{r})
	\label{eq:4.17}
\end{equation}
Si ha infatti che, considerando una carica totale $ q = Ze $ distruibuita come un ellissoide di semiasse maggiore (lungo $ z $) $ b $ e sezione sul piano $ xy $ circolare di raggio $ a $, il momento di quadrupolo vale:
\begin{equation}
	Q = \frac{2}{5} q (b^2 - a^2)
	\label{eq:4.18}
\end{equation}
Il segno di $ Q $ permette quindi di determinare se la distribuzione è prolata (allungata, $ b > a $ e $ Q > 0 $) o oblata (schiacciata, $ b < a $ e $ Q < 0 $) lungo $ z $. Sperimentalmente, per il deutone si trova $ Q_d = 0.2859 e\fm^2 $, dunque esso ha forma prolata; è anche possibile mostrare che tale valore $ Q_d $ si ottiene con la stessa combinazione di $ \ket{0,1,1} $ e $ \ket{2,1,1} $ trovata per il momento magnetico.

\subsection{Forze non-centrali}

Se la forza agente sui nucleoni fosse puramente centrale, il momento angolare sarebbe un buon numero quantico poiché le forze centrali lo conservano. Invece, si è visto che lo stato fondamentale del deutone è una sovrapposizione di due stati ad $ \ell $ distinto: ciò suggerisce la presenza di una forza non-centrale, detta \textit{forza tensoriale}: questa è l'equivalente forte dell'interazione elettromagnetica tra due dipoli elettrici o magnetici (naturalmente molto più intensa). Il potenziale tensoriale dipende da un coefficiente $ S_{1,2} $ definito come:
\begin{equation}
	S_{1,2} \defeq 3 \frac{(\ve{S}_1 \cdot \ve{r}) (\ve{S}_2 \cdot \ve{r})}{r^2} - \ve{S}_1 \cdot \ve{S}_2
	\label{eq:4.19}
\end{equation}
dove $ \ve{r} $ è il vettore di separazione tra i due nucleoni; si nota che $ S_{1,2} $ dipende soltanto dall'orientazione di $ \ve{r} $ e non da $ r $, ed inoltre la sua media su tutte le possibili orientazioni (dunque su $ \mathbb{S}^2 $) è nulla. Nella configurazione prolata ($ \ve{r} \parallel \hat{\ve{z}},\ve{S}_1,\ve{S}_2 $) $ S_{1,2} = \frac{1}{2} $, mentre in quella oblata ($ \ve{r} \perp \hat{\ve{z}},\ve{S}_1,\ve{S}_2 $) $ S_{1,2} = - \frac{1}{4} $: ciò indica che la configurazione prolata è quella favorita, e ciò è confermato dal momento di quadrupolo elettrico.\\
È presente anche una forza di tipo nucleare, detta \textit{forza spin-orbita}, visibile sia tramite esperimenti di scattering che nell'ordinamento dei libelli nel modello a shell, con andamento:
\begin{equation}
	V_{LS} \sim \ve{L} \cdot \ve{S}
	\label{eq:4.20}
\end{equation}
In conclusione, lo stato fondamentale del deutone ha principalmente $ \ell = 0 $, con una componente $ \ell = 2 $ che spiega il momento di quadrupolo elettrico e il momento di dipolo magnetico, avendo comunque $ s = 1 $ e $ J^{\pi} = 1^+ $: la simmetria della funzione d'onda non viola il principio di esclusione di Pauli, poiché neutrone e protone sono distinguibili; il principio non permette invece di avere funzioni d'onda simmetriche per stati neutrone-neutrone e protone-protone, dunque non possono esistere stati legati di questo tipo.










