\selectlanguage{italian}

\paragraph{Scale di grandezza}

Nello studio della fisica dei nuclei e delle particelle subatomiche, le scale di grandezza tipiche sono estremamente piccole: la scala tipica delle dimensioni di un atomo è $ 1\ang = 10^{-10}\m  $, mentre quella del nucleo è di $ 4 $ ordini di grandezza minore ($ 10^{-14}\m = 10\fm $); per un singolo nucleone, invece, le dimensioni sono dell'ordine di $ 1\fm = 10^{-15}\m $, e il range tipico delle interazioni deboli è $ 10^{-18}\m $.\\
Per quanto riguarda la scala di energie, i processi atomici hanno energie solitamente attorno a $ 1\ev = 1.602\cdot10^{-19}\,\text{J} $, mentre quelli nucleari arrivano anche a $ 10\mev $; per le interazioni ad alte energie studiate dalla fisica particellare, i moderni accelleratori arrivano a scale di $ 1\tev $.\\
Per studiare la struttura dei costituenti della materia a vari livelli, è necessario utilizzare fasci di particelle (fotoni, elettroni, etc.) con determinate lunghezze d'onda (relazioni di de Broglie $ \lambda = \frac{h}{p} $), corrispondenti a determinate energie: per sondare i nuclei atomici sono necessari $ \lambda \sim 10\fm $ ed $ E \sim 1\mev $; per evidenziare la struttura a molti corpi del nucleo atomico servono $ \lambda \sim 1\fm $ ed $ E \sim 10\mev $; se si vogliono studiare gli stati eccitati dei singoli nucleoni occorrono $ \lambda \sim 10^{-3}\fm $ ed $ E \sim 1\gev $ (solo oggetti compositi, dotati di una struttura, hanno stati eccitati); infine, se si vuole mettere in luce la struttura composta da quark dei nucleoni, bisogna raggiungere $ \lambda < 10^{-4}\fm $ ed $ E > 200\gev $.

\paragraph{Interazioni fondamentali}

I vari costituenti della materia interagiscono tramite $ 4 $ interazioni fondamentali, mediate da particelle specifiche - i bosoni di gauge:
\begin{enumerate}
  \item interazione elettromagnetica: mediata dal fotone ($ m_{\gamma} = 0 $), con coupling constant $ \alpha \approx 1/137 $ e raggio d'azione infinito (essendo il fotone massless). Agisce su tutte le particelle dotate di carica, e il campo ha un andamento $\alpha \hbar c / r$;
  \item interazione debole: mediata dai bosoni $ \w^{\pm} $ e $ \z^0 $ ($ m_W = 80.4\gev $, $ m_Z = 90.1\gev $), con coupling constant $ G_F \approx 1\cdot10^{-5} $ e raggio d'azione $ < 10^{-3}\fm $, dovuto al fatto che i bosoni $ \w^{\pm} $ e $ \z^0 $ sono molto pesanti e dunque, per il principio d'indeterminazione ($ \Delta E \Delta t \ge \frac{\hbar}{2} $), possono essere prodotti solo come particelle virtuali in processi di scattering per periodi di tempo estremamente brevi. Agisce su leptoni e adroni, e il campo ha un andamento $g \delta(\vec{r})$;
  \item interazione forte: mediata dai gluoni ($ m_g = 0 $), con coupling constant $ \alpha_s \approx 1 $ e raggio d'azione $ \approx 1\fm $, dovuto al fatto che i gluoni, sebbene massless, possono interagire tra loro. Agisce sugli adroni, e il campo ha un andamento $\alpha_s \hbar e^{-\mu r} / r$;
  \item interazione gravitazionale: mediata dall'ipotetico gravitone ($ m_G = 0 $), con coupling constant $ G_N \approx 6\cdot10^{-39} $ e raggio d'azione infinito. Agisce su tutte le particelle dotate di massa ed energia, e il campo ha un andamento $1 / r$;
\end{enumerate}
Come si può notare, la gravità ha un'intensità di decine di ordini di grandezza inferiore alle altre interazioni fondamentali, per questo in ambito atomico, nucleare e particellare può essere trascurata. \\
Inoltre, secondo la "Teoria della Grande Unificazione" (GUT, Grand Unification Theory), la forza elettromagnetica e le interazioni forte e debole si sono manifestate come forze distinte solo da un certo punto della vita dell'universo in poi: prima, infatti, si pensa che fossero concentrate in una forza sola. La medesima teoria prevede anche il decadimento dei protoni, l'esistenza di monopoli magnetici, etc. Tuttavia, fino ad oggi non sono state trovate evidenze sperimentali che confermino le predizioni di tale modello.

\paragraph{Esperimenti}

A differenza della fisica atomica, che è descritta completamente dalla QED (Quantum Electrodynamics), la fisica nucleare non ha un'unica teoria coerente: la teoria fondamentale dell'interazione forte, la QCD (Quantum Chromodynamics), descrive le interazioni tra quark (mediate da gluoni), non quelle tra nucleoni (mediate da mesoni virtuali); inoltre, in ambito atomico le energie che entrano in gioco nei decadimenti ($ \sim 10\mev $) sono meno dello $ 0.1\% $ della massa del nucleo (espressa in unità naturali), dunque gli effetti relativistici possono essere ignorati, mentre per quanto riguarda i processi tra nucleoni le energie possono essere anche $ 100 $ volte la massa equivalente del protone, rendendo necessario l'utilizzo della meccanica quantistica relativistica; infine, bisogna considerare che sia il nucleo atomico che i nucleoni sono sistemi complessi a molti corpi, dunque, anche avendo una teoria dell'interazione tra singole coppie di costituenti, è estremamente difficile sviluppare modelli teorici per descrivere questi sistemi, e la trattazione è principalmente di natura fenomenologica, con tante teorie dei singoli processi che vengono sviluppate a partire dai dati sperimentali.\\
Gli esperimenti in fisica nucleare (utilizzati anche per studiare gli adroni in generale) sono principalmente di due tipi:
\begin{enumerate}
  \item scattering: un fascio di particelle, di cui si conoscono energia e momento lineare, è diretto verso l'oggetto bersaglio da studiare e, attraverso le variazioni di quantità cinematiche misurabili, è possibile studiare le proprietà dell'interazioni e la struttura del bersaglio (risoluzione data dalla relazione di de Broglie). Il raggio nucleare può essere misurato con fasci di elettroni di circa $10^8 \text{eV}$, il raggio del protone con $10^9 \text{eV}$;
  \item spettroscopia: nucleoni (o anche mesoni e barioni) vengono eccitati e si studiano i prodotti del decadimento di questi stati eccitati, inferendo le proprietà degli stati eccitati e le interazioni tra i prodotti di decadimento. Le energie richieste per produrre stati eccitati sono simili a quelle necessarie agli esperimenti di scattering.
\end{enumerate}
Nel caso dello scattering è importante studiare la sezione d'urto d'interazione (cross section), ovvero la probabilità che avvenga una determinata reazione: in base all'angolo solido $ \Delta\Omega $ del rilevatore, alla cross section $ \frac{d\sigma}{d\Omega} $, all'intensità $ I_0 $ del fascio incidente e alla densità numerica di particelle $ n_0 $ che attraversano lo spessore $ dz $ del rilevatore, si può calcolare il numero di particelle rilevate in funzione dell'angolo d'emissione:
\begin{equation}
  \frac{dn(\theta)}{dt} = I_0 n_0 dz \frac{d\sigma}{d\Omega} \Delta\Omega
  \label{eq:1}
\end{equation}
La cross section è un'area geometrica (l'area effettiva di collisione) ed è solitamente misurata in barn: $ 1\barn = 100\fm^2 $; questa sezione d'urto è in realtà molto grande e misure più tipiche sono espresse in microbarn.

\thispagestyle{introd}
