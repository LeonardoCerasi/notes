\selectlanguage{italian}

\section{Rappresentazioni}

\begin{definition}
	Dati due gruppi $ (A,*_A) $ e $ (B,*_B) $, un mapping $ f : A \rightarrow B $ si definisce \textit{omomorfismo} tra $ A $ e $ B $ se $ f(a_1 *_A a_2) = f(a_1) *_B f(a_2) \,\forall a_1,a_2 \in A $.
\end{definition}

\begin{definition}
	Dato un omomorfismo $ f : A \rightarrow B $, si definiscono la sua \textit{immagine} o \textit{range} $ \ran f \defeq \{b \in B : b = f(a), a \in A\} $ ed il suo \textit{nucleo} o \textit{kernel} $ \ker f \defeq \{a \in A : f(a) = e_B\} $.
\end{definition}

\begin{proposition}
	$ \ker f \trianglelefteq A $.
\end{proposition}
\begin{proof}
	Dato $ a \in \ker f $, $ f(gag^{-1}) = f(g) f(a) f(g^{-1}) = f(g) f(g^{-1}) = f(e_A) = e_B $.
\end{proof}

\begin{example}
	$ f : D^4 \rightarrow \Z_2 $ definita da $ f(e) = f(c^2) = f(b) = f(bc^2) = +1 $ e $ f(c) = f(c^3) = f(bc) = f(bc^3) = -1 $ è un omomorfismo.
\end{example}

\begin{definition}
	Dato un gruppo $ G $, si definisce una sua \textit{rappresentazione lineare} su uno spazio vettoriale finito $ V(\K) $ un omomorfismo $ \rho : G \rightarrow \mathrm{GL}(V) $.
\end{definition}

\begin{proposition}
	$ \rho(e) = \mathrm{id}_V $.
\end{proposition}

Nella maggior parte delle applicazioni, piuttosto che considerare le applicazioni lineari $ f \in \mathrm{GL}(V) $ si prendono direttamente le loro matrici associate $ M_f \in \GL{n}{\K} $, dove $ n = \dim_{\K}{V} $ (detto \textit{grado} della rappresentazione).

\begin{definition}
	Si definisce \textit{isomorfismo} un omomorfismo biiettivo.
\end{definition}

\begin{definition}
	Una rappresentazione si dice \textit{fedele} se è un isomorfismo.
\end{definition}

\begin{example}
	$ \On{n} \cong \SOn{n} \otimes \{\tens{I}_n, -\tens{I}_n\} $.
\end{example}

\begin{definition}
	Dato un gruppo $ G $, due sue rappresentazioni $ \rho_1,\rho_2 $ sugli spazi vettoriali $ V,W $ si dicono \textit{equivalenti} (o \textit{isomorfe} o \textit{simili}) se esiste un isomorfismo $ \varphi : V \rightarrow W $ tale che:
	\begin{equation}
		\varphi \circ \rho_1(g) = \rho_2 (g) \circ \varphi \quad \forall g \in G
		\label{eq:8.1}
	\end{equation}
\end{definition}

In termini matriciali $ \rho_1 \sim \rho_2 \,\Leftrightarrow\, \exists S \in \GL{n}{\K} : \rho_2(g) = S \rho_1(g) S^{-1} \,\,\forall g \in G $.\\
Si vede subito che la similitudine preserva la struttura di gruppo:
\begin{equation*}
	\rho_2(g_1 g_2) = S \rho_1(g_1 g_2) S^{-1} = S \rho_1(g_1) S^{-1} S \rho_1(g_2) S^{-1} = \rho_2(g_1) \rho_2(g_2)
\end{equation*}

\begin{definition}
	Dato un gruppo $ G $, una sua \textit{rappresentazione unitaria} è una rappresentazione $ \rho : G \rightarrow \Un{n} $.
\end{definition}

Si può dare una struttura topologica ad un gruppo.

\begin{definition}
	Si definisce \textit{gruppo topologico} $ G $ un gruppo dotato di una struttura di spazio topologico di Hausdorff, rispetto alla quale le applicazioni $ (a,b) \mapsto ab $ e $ a \mapsto a^{-1} $ sono continue.
\end{definition}

\begin{definition}
	Un gruppo topologico $ G $ è detto \textit{gruppo di Lie} se le applicazioni $ (a,b) \mapsto ab $ e $ a \mapsto a^{-1} $ sono mappe lisce.
\end{definition}

In maniera informale, si possono vedere i gruppi di Lie sia come gruppi che come varietà differenziali (propriamente, c'è un diffeomorfismo tra i due).

\begin{example}
	$ \SOn{2} $ corrisponde a $ \mathbb{S}^1 $.
\end{example}
\begin{example}
	$ \SOn{3} $ corrisponde a $ \mathbb{D}^3 / \tildetext $, dove $ \sim $ è la relazione d'equivalenza che associa ad ogni punto il suo antipodale.
\end{example}

Una volta data la struttura topologica, si può stabilire la compattezza di un gruppo: ad esempio, vedere Def. \ref{def-m-comp}. Nel caso dei gruppi di Lie, la compattezza può essere determinata studiando la varietà differenziale associata.

\begin{proposition}
	Per i gruppi di Lie compatti, ogni rappresentazione è equivalente ad una rappresentazione unitaria.
\end{proposition}

\section{Rappresentazioni riducibili ed irriducibili}

\begin{definition}
	Dati un gruppo $ G $ ed un insieme $ \Omega $, si dice che $ G $ \textit{agisce} su $ \Omega $ se viene fissata una mappa $ \Omega \times G \rightarrow \Omega : (\alpha,g) \mapsto \alpha^g $ tale che $ \alpha^{st} = (\alpha^t)^s \,\forall \alpha \in \Omega, s,t \in G $ ed $ \alpha^e = \alpha \,\forall \alpha \in \Omega $.
\end{definition}
Si noti che si usa la convenzione di azione a sinistra.\\
Si vede che una rappresentazione $ \rho $ di $ G $ suo spazio vettoriale $ V $ determina un'azione di $ G $ su $ V $ definita da $ \rho_g(v) = \rho(g) v \,\forall v \in V, g \in G $, dove nel lato destro si è usata la rappresentazione matriciale.

\begin{definition}
	Dati un gruppo $ G $ ed una rappresentazione $ \rho $ su uno spazio vettoriale $ V $, un sottospazio $ W \subset V $ si dice \textit{invariante} per l'azione di $ G $ se $ \rho_g(w) \in W \,\forall w \in W, g \in G $.
\end{definition}

Se esiste un sottospazio $ G $-invariante $ W $, allora $ \rho $ induce due rappresentazioni, una su $ W $ ed una sul quoziente $ V / W $: la prima è semplicemente la restrizione di $ \rho $ a $ W $, mentre la seconda è determinata da $ \rho_g(v + W) = \rho_g(v) + W \,\forall v \in V/W $ (ben definita per l'invarianza di $ W $). Queste sono dette \textit{rappresentazioni costituenti}.

\begin{definition}
	Dato un gruppo $ G $, una sua rappresentazione $ \rho $ su $ V $ si dice \textit{riducibile} se esiste un sottospazio $ W \subset V $ invariante per l'azione di $ G $, altrimenti è \textit{irriducibile}. Nel caso in cui $ V $ si spezzi in somma diretta di sottospazi invarianti irriducibili, $ \rho $ si dice \textit{completamente riducibile}.
\end{definition}

Si vede che una rappresentazione irriducibile risulta essere completamente riducibile.\\
In termini matriciali, $ \rho $ è riducibile se può essere scritta come:
\begin{equation}
	\rho(g) =
	\begin{bmatrix}
		\rho_1(g) & 0 \\
		\tau(g) & \rho_2(g)
	\end{bmatrix}
	\label{eq:8.2}
\end{equation}
dove $ \rho_1(g) $ è l'azione di $ G $ su $ W $ e $ \rho_2(g) $ su $ V/W $.\\
Si vede che $ \rho_1 $ e $ \rho_2 $ sono rappresentazioni ristrette ai sottospazi:
\begin{equation*}
	\rho_g(w + v) =
	\begin{bmatrix}
		\rho_1(g) & 0 \\
		\tau(g) & \rho_2(g)
	\end{bmatrix}
	\begin{pmatrix}
		w \\ v
	\end{pmatrix}
	=
	\begin{bmatrix}
		\rho_1(g) w & 0 \\
		\tau(g) w & \rho_2(g) v
	\end{bmatrix}
	=
	\begin{bmatrix}
		\rho_g(w) & 0 \\
		\tau(g) w & \rho_g(v)
	\end{bmatrix}
\end{equation*}
dove $ w \in W $ e $ v \in V/W $.\\
Il termine $ \tau(g) $ è ciò che rende il sottospazio $ V/W $ non invariante, ed infatti non costituisce una rappresentazione:
\begin{equation*}
	\rho(g_1) \rho(g_2) =
	\begin{bmatrix}
		\rho_1(g_1) & 0 \\
		\tau(g_1) & \rho_2(g_1)
	\end{bmatrix}
	\begin{bmatrix}
		\rho_1(g_2) & 0 \\
		\tau(g_2) & \rho_2(g_2)
	\end{bmatrix}
	=
	\begin{bmatrix}
		\rho_1(g_1) \rho_2(g_2) & 0 \\
		\tau(g_1) \rho_1(g_2) + \rho_2(g_1) \tau(g_2) & \rho_2(g_1) \rho_2(g_2)
	\end{bmatrix}
\end{equation*}
Si dimostra che, per gruppi compatti, si può sempre trovare una rappresentazione equivalente in cui $ \tau = 0 $, rendendo la rappresentazione completamente riducibile.\\
Se $ \rho $ è completamente riducibile, essa può essere scritta come:
\begin{equation}
	\rho(g) =
	\begin{bmatrix}
		\rho_1(g) & \dots & 0 \\
		\vdots & \ddots & \vdots \\
		0 & \dots & \rho_m(g)
	\end{bmatrix}
	\label{eq:8.3}
\end{equation}
dove $ \rho_j $ è la restrizione di $ \rho $ su $ W_j $ sottospazio invariante irriducibile. In questo caso si ha:
\begin{equation}
	V = \bigoplus_{j = 1}^{m} W_j
	\qquad \qquad
	\rho = \bigoplus_{j = 1}^{m} \rho
	\label{eq:8.4}
\end{equation}
dove la \textit{somma diretta} di rappresentazioni (su sottospazi disgiunti) è definita come:
\begin{equation}
	(\rho_i \oplus \rho_j)_g (u + v) = \rho_i(g) u + \rho_j(g) v
	\label{eq:8.5}
\end{equation}
con $ u \in W_i $ e $ v \in W_j $.

\begin{example}
	$ \SOn{2} $ rappresentato su $ \R^3 $ è completamente riducibile sui sottospazi invarianti determinati dal piano $ xy $ e dall'asse $ z $, dato che la sua azione può essere espressa come:
	\begin{equation}
		\tens{R}(\theta) =
		\begin{bmatrix}
			\cos \theta & - \sin \theta & 0 \\
			\sin \theta & \cos \theta & 0 \\
			0 & 0 & 1
		\end{bmatrix}
		\equiv
		\begin{bmatrix}
			\rho_{xy}(\theta) & 0 \\
			0 & \rho_z(\theta)
		\end{bmatrix}
	\end{equation}
\end{example}

\paragraph{Spazi vettoriali}

Questi concetti sono mutuati dalla teoria degli spazi vettoriali. Dato uno spazio vettoriale $ n $-dimensionale $ V(\K) $, un endomorfismo $ f \in \End{V} $ è rappresentabile con una matrice $ M_f \in \GL{n}{\K} $:
\begin{equation*}
	f(\ve{v}) = M_f \ve{v} \quad \Longrightarrow \quad v'_i = \sum_{j = 1}^{n} M_{ij} v_j \quad \lor \quad \ve{e}'_i = \sum_{j = 1}^{n} M_{ji} \ve{e}_j
\end{equation*}
Si consideri un'altra base $ \{\tilde{\ve{e}}_i\}_{i = 1, \dots, n} $: dato che il cambio di base è un isomorfismo (endomorfismo biunivoco), detta $ S $ la matrice che lo rappresenta, per essa valgono le stesse proprietà.\\
Si può determinare l'azione di $ f $ sulla nuova base:
\begin{equation*}
	\begin{split}
		\tilde{\ve{v}}'
		&= S \ve{v}' = S M_f \ve{v}\\
		&= \tilde{M}_f \tilde{\ve{v}} = \tilde{M}_f S \ve{v}
	\end{split}
	\qquad \Longrightarrow \qquad
	\tilde{M}_f = S M_f S^{-1}
\end{equation*}
Un cambiamento di rappresentazione, dunque, equivale ad un cambio di base nello spazio vettoriale.

\section{Teorema di Maschke}

\begin{definition}
	Dato uno spazio vettoriale $ V $ ed un suo sottospazio $ W \subset V $, si definisce \textit{proiettore} su $ W $ una mappa $ \pi : V \rightarrow W : \pi(v) = v_W $, dove $ v_W $ è la componente di $ v \in V $ su $ W $.
\end{definition}

\begin{definition}
	Dato uno spazio vettoriale $ V $ ed un suo sottospazio $ W \subset V $, detto $ \pi $ un proiettore su di esso, esso definisce un suo \textit{complementare} (o \textit{complemento ortogonale}) $ W^{\perp} \defeq \ker{\pi} $.
\end{definition}

C'è una corrispondenza biunivoca tra proiettori e complementari. Solitamente, si considera il proiettore tale per cui $ W^{\perp} = V / W $.

\begin{proposition}
	Data $ f \in \End{V}  $ e $ W \subset V $ sottospazio, se $ \im{f} = W \land f(w) = w \,\forall w \in W $, allora $ V = W \oplus \ker{f} $.
\end{proposition}

\begin{propcorollary}
	$ V = W \oplus W^{\perp} $.
\end{propcorollary}

\begin{theorem}[Maschke]\label{th-maschke}
	Dato un gruppo finito $ G $ il cui ordine non è divisibile dalla caratteristica del campo, ogni rappresentazione di $ G $ è completamente riducibile.
\end{theorem}
\begin{proof}
	Presa $ \rho $ su $ V $, se $ \rho $ è irriducibile non c'è nulla da dimostrare.
	Se $ \rho $ è riducibile, allora esiste $ W \subset V $ sottospazio $ G $-invariante: basta dimostrare che esiste sempre un suo complementare $ G $-invariante.
	Preso $ \pi : V \rightarrow W $ proiettore su $ W $, si definisce:
	\begin{equation*}
		\pi^{\circ} \defeq \frac{1}{\abs{G}} \sum_{g \in G} \rho_g \circ \pi \circ \rho_{g^{-1}}
	\end{equation*}
	Si vede che $ \pi^{\circ} $ è un proiettore su $ W $: dati $ v \in V $, $ \rho_g \circ \pi \circ \rho_{g^{-1}} (v) = \rho_g (w) \in W $ per invarianza di $ W $, dove $ w \equiv \pi(\rho_{g^{-1}}(v)) \in W $; dato $ w \in W $, $ \rho_g \circ \pi \circ \rho_{g^{-1}}(w) = \rho_g \circ \pi(\rho_{g^{-1}}(w)) = \rho_g \circ \rho_{g^{-1}}(w) = w $ per invarianza di $ W $ e poiché $ \pi(w) = w \,\forall w \in W $, da cui segue che $ \pi^{\circ}(w) = \frac{1}{\abs{G}} \abs{G} w = w $; dunque, $ \pi^{\circ} $ è un proiettore di $ V $ su $ W $. Esso definisce allora un complementare $ W^{\perp} = \ker \pi^{\circ} $: per dimostrare che esso è $ G $-invariante, basta mostrare che, dato $ v \in V $, $ \pi^{\circ}(v) = 0 \,\Rightarrow\, \pi^{\circ}(\rho_g(v)) = 0 \,\forall g \in G $, ovverosia che $ \pi^{\circ} $ permuta con l'azione di $ G $ (dato che in tal caso $ \pi \circ \rho_g (v) = \rho_g \circ \pi(v) = \rho_g (0) = 0 $):
	\begin{equation*}
		\begin{split}
			\rho_g \circ \pi^{\circ} \circ \rho_{g^{-1}} 
			&= \rho_g \circ \left[ \frac{1}{\abs{G}} \sum_{t \in G} \rho_t \circ \pi \circ \rho_{t^{-1}} \right] \circ \rho_{g^{-1}} = \frac{1}{\abs{G}} \sum_{t \in G} \rho_g \circ \rho_t \circ \pi \circ \rho_{t^{-1}} \circ \rho_{g^{-1}}\\
			&= \frac{1}{\abs{G}} \sum_{t \in G} \rho_{gt} \circ \pi \circ \rho_{(gt)^{-1}} = \frac{1}{\abs{G}} \sum_{s \in G} \rho_s \circ \pi \circ \rho_{s^{-1}} = \pi^{\circ}
		\end{split}
	\end{equation*}
	Si vede allora che $ \rho = \rho_1 \oplus \rho_2 $, dove $ \rho_1,\rho_2 $ sono le restrizioni di $ \rho $ a $ W,W^{\perp} $: se queste sono irriducibili, si è dimostrata la tesi, altrimenti si procede iterativamente fino a decomporre $ V $ in sottospazi invarianti irriducibili.
\end{proof}

L'ipotesi sulla caratteristica $ p $ del campo è necessaria in quanto $ \frac{1}{\abs{G}} $ perderebbe di senso nel caso in cui $ p $ dividesse $ \abs{G} $. È anche possibile estendere il teorema di Maschke ai gruppi compatti, equipaggiandoli con un'appropriata misura d'integrazione (per rendere convergente $ \sum_{g} \rightarrow \int d\mu(g) $).

\subsection{Rappresentazioni unitarie}

Nel caso in cui $ \K = \C $, si può dare un'altra formulazione del Teorema di Maschke.

\begin{definition}
	Dato uno spazio vettoriale $ V(\C) $, si definisce \textit{forma hermitiana} un'applicazione $ V \times V \rightarrow \C $ tale che, per ogni $ u,v,w \in V $, $ \alpha,\beta \in C $:
	\begin{enumerate}
		\item $ (u,v) = \overline{(v,u)} $;
		\item $ (\alpha u + \beta v, w) = \alpha (u,w) + \beta (v,w) $;
		\item $ (v,v) \in \R^+_0 \land (v,v) = 0 \,\Leftrightarrow\, v = 0 $.
	\end{enumerate}
\end{definition}

\begin{definition}
	Dato un gruppo $ G $ ed una sua rappresentazione $ \rho $ su $ V(\C) $, una forma hermitiana su $ V $ si dice $ G $\textit{-invariante} se:
	\begin{equation}
		(\rho_g(u),\rho_g(v)) = (u,v) \quad\forall u,v \in V, g \in G
		\label{eq:8.7}
	\end{equation}
\end{definition}

\begin{proposition}\label{herm-inv}
	Data una forma hermitiana, si può sempre definire una forma hermitiana $ G $\textit{-invariante} come:
	\begin{equation}
		(u|v) \defeq \sum_{g \in G} (\rho_g(u), \rho_g(v))
		\label{eq:8.8}
	\end{equation}
\end{proposition}
\begin{proof}
	Dato $ t \in G $, $ (\rho_t(u)|\rho_t(v)) = \sum_{g} (\rho_{gt}(u),\rho_{gt}(v)) = \sum_{s} (\rho_s(u),\rho_s(v)) = (u|v) $.
\end{proof}

In termini matriciali, la condizione in Eq. \ref{eq:8.7} equivale a dire che $ \rho(g) $ è una matrice unitaria:
\begin{equation*}
	(\rho_g(u),\rho_g(v)) = (\rho(g) u)^{\dagger} (\rho(g) v) = u^{\dagger} \rho(g)^{\dagger} \rho(g) v = (u,v) = u^{\dagger} v \quad\Longleftrightarrow\quad \rho(g)^{\dagger} \rho(g) = \tens{I}
\end{equation*}

\begin{definition}
	Dato un gruppo $ G $ ed una sua rappresentazione $ \rho $ su $ V(\C) $, questa si dice \textit{rappresentazione unitaria} se esiste una forma hermitiana $ G $-invariante su $ V $.
\end{definition}

\begin{proposition}\label{uni-rid}
	Una rappresentazione unitaria è completamente riducibile.
\end{proposition}
\begin{proof}
	Basta dimostrare che, dato un sottospazio $ G $-invariante $ W \subset V $, il suo complemento ortogonale $ W^{\perp} \defeq \{v \in V : (w,v) = 0 \,\forall w \in W\} $ è $ G $-invariante, dato che $ V = W \oplus W^{\perp} $.\\
	Dati $ v \in W^{\perp} $ e $ g \in G $, $ (\rho_g(w),v) = 0 \,\forall w \in W $ per invarianza di $ W $, dunque:
	\begin{equation*}
		0 = (\rho_g(w),v) = (\rho_g(w),\rho_g \circ \rho_{g^{-1}}(v)) = (w,\rho_{g^{-1}}(v)) \quad\Rightarrow\quad \rho_{g^{-1}}(v) \in W^{\perp}
	\end{equation*}
	$ W^{\perp} $ è quindi $ G $-invariante e $ V = W \oplus W^{\perp} $; se questi non sono irriducibili, si procede iterativamente.
\end{proof}

\begin{theorem}[Maschke]
	Dato un gruppo finito $ G $, ogni sua rappresentazione complessa è completamente riducibile.
\end{theorem}
\begin{proof}
	Per la Prop. \ref{herm-inv} è sempre possibile rendere la rappresentazione unitaria, ovvero completamente riducibile per la Prop. \ref{uni-rid}.
\end{proof}

Anche in questo caso è possibile generalizzare a gruppi compatti con adeguata misura d'integrazione.

\section{Ortogonalità}

Per dimostrare il cosiddetto grande teorema di ortogonalità è necessario prima dimostrare due lemmi, detti primo e secondo lemma di Schur.

\begin{lemma}[Schur]
	Dato un gruppo $ G $ ed una sua rappresentazione irriducibile $ \rho $ su $ V(\C) $, se $ \varphi \in \End V $ commuta con l'azione di $ G $, allora $ \varphi = \lambda \id_V $ con $ \lambda \in \C $.
\end{lemma}
\begin{proof}
	Sia $ \lambda \in \C $ un autovalore di $ \varphi $: dato $ v \in V $ un autovettore associato a $ \lambda $, per ogni $ g \in G $ si ha $ \varphi \circ \rho_g(v) = \rho_g \circ \varphi(v) = \rho_g(\lambda v) = \lambda \rho_g(v) $, dunque anche $ \rho_g(v) $ è un autovettore di $ \varphi $ associato allo stesso $ \lambda $.
	Gli autovettori associati ad ogni autovalore di $ \varphi $ formano quindi dei sottospazi $ G $-invarianti, ma $ \rho $ è irriducibile: gli unici sottospazi $ G $-invarianti possibili sono $ V $ e $ \{0\} $. Il secondo caso non è possibile, dato che ogni endomorfismo  in campo complesso ha almeno un autovalore (il suo polinomio caratteristico ha almeno una radice complessa per il teorema fondamentale dell'algebra), dunque tutti i vettori in $ V $ sono associati allo stesso autovalore $ \lambda $: di conseguenza, $ \varphi = \lambda \id_V $.
\end{proof}

\begin{lemma}[Schur]
	Dato un gruppo $ G $ e due sue rappresentazioni irriducibili non-equivalenti $ \rho $ su $ V(\K) $ e $ \rho' $ su $ V'(\K) $, se $ \varphi : V \rightarrow V' $ lineare commuta con l'azione di $ G $, ovvero $ \varphi \circ \rho_g = \rho'_g \circ \varphi \,\forall g \in G $, allora $ \varphi = 0_V $.
\end{lemma}
\begin{proof}
	Sia $ \varphi \neq 0_V $: dato $ w \in \ker \varphi $, $ \varphi(\rho_g(w)) = \rho'_g(\varphi(w)) = \rho'_g(0) = 0 \,\forall g \in G $, dunque $ \ker \varphi $ è un sottospazio $ G $-invariante di $ V $; essendo $ \rho $ irriducibile, $ \ker \varphi = V $ oppure $ \ker \varphi = \{0\} $. Il primo caso è escluso da $ \varphi \neq 0_V $, quindi $ \ker \varphi = \{0\} $, ovvero $ \varphi $ è iniettiva. Dato $ v \in V $, $ \rho'_g(\varphi(v)) = \varphi(\rho_g(v)) \in \im \varphi \,\forall g \in G $, dunque $ \im \varphi $ è un sottospazio $ G $-invariante di $ V' $; essendo $ \rho' $ irriducibile, $ \im \varphi = V' $ oppure $ \im \varphi = \{0\} $. Il secondo caso è escluso da $ \varphi \neq 0_V $, quindi $ \im \varphi = V' $, ovvero $ \varphi $ è suriettiva. Di conseguenza, $ \varphi $ è biunivoca, dunque un isomorfismo: ciò implica l'equivalenza di $ \rho $ e $ \rho' $, il che è escluso per ipotesi, dunque deve essere $ \varphi = 0_V $.
\end{proof}

\begin{theorem}
	Dato un gruppo $ G $ ed una famiglia di rappresentazioni irriducibili non-equivalenti $ \rho_{\mu} $ su spazi vettoriali $ V_{\mu}(\C) $ di dimensione $ n_{\mu} $, si ha che:
	\begin{equation}
		\sum_{g \in G} \left[ \rho_{\mu}(g) \right]_{ir} \left[ \rho_{\nu}(g^{-1}) \right]_{sj} = \frac{\abs{G}}{n_{\mu}} \delta_{\mu \nu} \delta_{ij} \delta_{rs}
		\label{eq:8.9}
	\end{equation}
\end{theorem}
\begin{proof}
	Fissati $ \mu,\nu $, si consideri $ \varphi : V_{\nu} \rightarrow V_{\mu} $ lineare e si costruisca la mappa $ \psi : V_{\nu} \rightarrow V_{\mu} $ definita come:
	\begin{equation*}
		M_{\psi} \defeq \sum_{g \in G} \rho_{\mu}(g) M_{\varphi} \rho_{\nu}(g^{-1})
	\end{equation*}
	Preso $ h \in G $, si ha che:
	\begin{equation*}
		\rho_{\mu}(h) M_{\psi} = \sum_{g \in G} \rho_{\mu}(hg) M_{\varphi} \rho_{\nu}(g^{-1}) = \sum_{t \in G} \rho_{\mu}(t) M_{\varphi} \rho_{\nu}(t^{-1}h) = M_{\psi} \rho_{\nu}(h)
	\end{equation*}
	Per i lemmi di Schur, se $ \mu \neq \nu $ allora $ \psi = 0_{V_{\mu}} $, mentre se $ \mu = \nu $ si ha $ \psi = \lambda_{\mu} \id_{V_{\mu}} $ ovvero $ M_{\psi} = \lambda^{(\mu)} \tens{I} $. Quindi, scegliendo $ \left[ M_{\varphi} \right]_{lm} = \delta_{lr}\delta_{ms} $:
	\begin{equation*}
		\delta_{\mu \nu} \lambda^{(\mu)}_{rs} \delta_{ij} = \sum_{g \in G} \left[ \rho_{\mu}(g) M_{\varphi} \rho_{\nu}(g^{-1}) \right]_{ij} = \sum_{g \in G} \sum_{l,m = 1}^{n_{\mu},n_{\nu}} \left[ \rho_{\mu}(g) \right]_{il} \delta_{lr} \delta_{ms} \left[ \rho_{\nu}(g^{-1}) \right]_{mj} = \sum_{g \in G} \left[ \rho_{\mu}(g) \right]_{ir} \left[ \rho_{\nu}(g^{-1}) \right]_{sj}
	\end{equation*}
	Bisogna solo mostrare che $ \lambda^{(\mu)}_{rs} = \frac{\abs{G}}{n_{\mu}} \delta_{rs} $; si prenda l'espressione precedente con $ \mu = \nu $ e $ i = j $ e sommando su $ i $ (varia tra $ 1 $ e $ n_{\mu} $):
	\begin{equation*}
		\lambda^{(\mu)}_{rs} n_{\mu} = \sum_{g \in G} \left[ \rho_{\mu}(g^{-1}) \rho_{\mu}(g) \right]_{sr} = \sum_{g \in G} \delta_{sr} = \abs{G} \delta_{sr}
	\end{equation*}
\end{proof}

\begin{corollary}\label{cond-n-mu}
	Se le rappresentazioni sono unitarie, $ \sum_{\mu} n_{\mu}^2 \le \abs{G} $.
\end{corollary}
\begin{proof}
	Fissati $ \mu = \nu $ e $ i = j $ ed $ r = s $, l'Eq. \ref{eq:8.9} può essere riscritta come:
	\begin{equation*}
		\frac{\abs{G}}{n_{\mu}} = \sum_{g \in G} \left[ \rho_{\mu}(g) \right]_{ir} \left[ \rho_{\mu}(g)^{\dagger} \right]_{ri} = \sum_{g \in G} \left[ \rho_{\mu}(g) \right]_{ir} \overline{\left[ \rho_{\mu}(g) \right]}_{ir}
	\end{equation*}
	la quale può essere vista come il prodotto scalare di un vettore $ v^{(\mu)}_{ij} \equiv \left( [\rho_{\mu}(g_1)]_{ir}, \dots, [\rho_{\mu}(g_{\abs{G}})]_{ir} \right) $ con sé stesso; in totale, ci sono $ \sum_{\mu} n_{\mu}^2 $ di questi vettori, e l'Eq. \ref{eq:8.9} determina la loro mutua ortogonalità. Dato che in uno spazio vettoriale di dimensione $ n $ un sistema ortogonale può avere al massimo $ n $ vettori, si deve avere $ \sum_{\mu} n_{\mu}^2 \le \abs{G} $, dato che i $ v^{(\mu)}_{ir} $ sono definiti in uno spazio di dimensione $ \abs{G} $.
\end{proof}

Si può dimostrare che vale l'uguaglianza. Inoltre, dai lemmi di Schur deriva un'importante proprietà dei gruppi abeliani.

\begin{proposition}\label{irrep-unidim}
	Dato un gruppo abeliano, qualsiasi sua rappresentazione irriducibile complessa è unidimensionale.
\end{proposition}
\begin{proof}
	Dato che $ G $ è abeliano e fissato $ \mu $, $ \rho_{\mu}(g) $ commuta con $ \rho_{\mu}(h) \,\forall g,h \in G $, dunque per il primo lemma di Schur si deve avere $ \rho_{\mu}(g) = \lambda_{\mu} \tens{I} \,\forall g \in G $. Se $ n_{\mu} > 1 $, la rappresentazione è completamente riducibile poiché è già espressa in forma diagonale, quindi, dato che gli elementi sulla diagonale hanno dimensione unitaria (i relativi sottospazi invarianti hanno dimensione unitaria), ogni rappresentazione di $ G $ può essere decomposta in rappresentazioni irriducibili unidimensionali.
\end{proof}

Questa proprietà risulta particolarmente utile nella determinazione delle tavole dei caratteri.

\section{Caratteri}
\label{sec-char}

\begin{definition}
	Dato un gruppo $ G $ ed una sua rappresentazione complessa $ \rho $, si definisce il \textit{carattere} di $ \rho $ come $ \chi : G \rightarrow \C : \chi(g) = \tr{\rho(g)} $.
\end{definition}

\begin{proposition}
	Due rappresentazioni equivalenti hanno lo stesso carattere.
\end{proposition}
\begin{proof}
	Dalla ciclicità della traccia: $ \tr \left( S \rho(g) S^{-1} \right) = \tr \left( S^{-1} S \rho(g) \right) = \tr \rho(g) $.
\end{proof}

\begin{propcorollary}\label{cahr-con}
	$ \chi(tgt^{-1}) = \chi(g) \,\forall g,t \in G $.
\end{propcorollary}

\begin{proposition}\label{char-herm}
	Data una rappresentazione unitaria $ \rho $, $ \chi(g^{-1}) = \overline{\chi(g)}\,\forall g \in G $.
\end{proposition}
\begin{proof}
	$ \chi(g^{-1}) = \tr \rho(g^{-1}) = \tr \rho(g)^{\dagger} = \overline{\tr \rho(g)} = \overline{\chi(g)} $.
\end{proof}

\subsection{Ortogonalità dei caratteri}

Se nell'Eq. \ref{eq:8.9} si considerano $ i = r $ e $ j = s $ e si somma su $ i $ e $ j $, dalla Prop. \ref{char-herm} si ottiene:
\begin{equation}
	\frac{1}{\abs{G}} \sum_{g \in G} \chi_{\mu}(g) \overline{\chi_{\nu}(g)} = \delta_{\mu \nu}
	\label{eq:8.10}
\end{equation}
Questa espressione permette di definire il prodotto scalare tra vettori $ \chi_{\mu} \equiv (\chi_{\mu}(g_1), \dots, \chi_{\mu}(g_{\abs{G}})) $: caratteri di rappresentazioni irriducibili determinano vettori ortonormali così definiti.\\
Inoltre, dato che tutti gli elementi di uno stesso coniugio hanno lo stesso carattere (per il Cor. \ref{cahr-con}), detto $ k $ il numero delle classi di coniugazione di $ G $, $ n_c $ il numero di elementi della $ c $-esima classe e $ \chi_{\mu}(c) $ il suo carattere, l'Eq. \ref{eq:8.10} diventa:
\begin{equation}
	\braket{\chi_{\mu},\chi_{\nu}} \defeq \frac{1}{\abs{G}} \sum_{c = 1}^{k} n_c \chi_{\mu}(c) \overline{\chi_{\nu}(c)} = \delta_{\mu \nu}
	\label{eq:8.11}
\end{equation}
I vettori $ \chi_{\mu} $ così definiti formano uno spazio di dimensione $ k $, dunque, per lo stesso ragionamento usato nella dimostrazione del Cor. \ref{cond-n-mu}, detto $ r $ il numero delle rappresentazioni irriducibili, la condizione di ortonormalità impone $ r \le k $.\\
È altresì possibile dimostrare che:
\begin{equation}
	\frac{1}{\abs{G}} \sum_{\mu = 1}^{r} n_c \chi_{\mu}(c) \overline{\chi_{\mu}(d)} = \delta_{cd}
	\label{eq:8.12}
\end{equation}
dalla quale deriva che $ k \le r $. Unendo le condizioni, si trova che il numero di rappresentazioni irriducibili di un gruppo è uguale al numero delle sue classi di coniugio: $ r = k $.

\subsubsection{Rappresentazioni completamente riducibili}

Per gruppi finiti (o compatti), ogni rappresentazione è completamente riducibile in somma diretta di rappresentazioni irriducibili: ciò significa che le matrici di rappresentazione possono essere scritte in forma diagonale a blocchi (Eq. \ref{eq:8.3}). In questa scrittura, ciascuna rappresentazione irriducibile può comparire più volte; in generale:
\begin{equation}
	\rho = \bigoplus_{\mu = 1}^{r} \alpha_{\mu} \rho_{\mu}
	\label{eq:8.13}
\end{equation}
dove $ \alpha_{\mu} \in \N $ indica quante volte $ \rho_{\mu} $ compare in $ \rho $. Prendendo la traccia di questa equazione, si trova:
\begin{equation}
	\chi(g) = \sum_{\mu = 1}^{r} \alpha_{\mu} \chi_{\mu}(g)
	\label{eq:8.14}
\end{equation}
Moltiplicando per $ \chi_{\mu}(g^{-1}) $ e sommando su $ g \in G $:
\begin{equation*}
	\sum_{g \in G} \chi_{\mu}(g^{-1}) \chi(g) = \sum_{\nu = 1}^{r} \alpha_{\nu} \sum_{g \in G} \chi_{\mu}(g^{-1}) \chi_{\nu}(g) = \sum_{\nu = 1}^{r} \alpha_{\nu} \abs{G} \delta_{\nu \mu} = \abs{G} \alpha_{\mu}
\end{equation*}
È quindi possibile determinare $ \alpha_{\mu} $:
\begin{equation}
	\alpha_{\mu} = \braket{\chi,\chi_{\mu}}
	\label{eq:8.15}
\end{equation}
Si nota ancora l'analogia con gli spazi vettoriali.

\subsection{Rappresentazione regolare}

\begin{theorem}[Cayley]
	Ogni gruppo finito di ordine $ n $ è isomorfo ad un sottogruppo di $ S^n $.
\end{theorem}

Da questo teorema deriva che, dato un gruppo finito $ G $ di ordine $ n $, l'azione di $ g \in G $ su $ G $ è data da:
\begin{equation}
	gg_i = \sum_{j = 1}^{n} D_{ij}(g) g_j
	\label{eq:8.16}
\end{equation}
Le matrici $ D_{ij}(g) $, di dimensione $ n^2 $, danno una rappresentazione del gruppo $ G $, detta \textit{rappresentazione regolare}: in particolare, $ D_{ij}(g) $ ha un solo valore non-nullo in ogni riga ed in ogni colonna, in quanto deve agire come una permutazione (in particolare, $ D(e) = \tens{I} $).

\begin{example}
	In $ C^3 = \{e,c,c^2\} $, si ha:
	\begin{equation*}
		D(e) =
		\begin{bmatrix}
			1 & 0 & 0 \\
			0 & 1 & 0 \\
			0 & 0 & 1
		\end{bmatrix}
		\qquad
		D(c) =
		\begin{bmatrix}
			0 & 0 & 1 \\
			1 & 0 & 0 \\
			0 & 1 & 0
		\end{bmatrix}
		\qquad
		D(c^2) =
		\begin{bmatrix}
			0 & 1 & 0 \\
			0 & 0 & 1 \\
			1 & 0 & 0
		\end{bmatrix}
	\end{equation*}
\end{example}

Dato che $ ag = g $ ha come unica soluzione $ a = e $, si ha che l'unica $ D(g) $ con elementi diagonali non nulli e $ D(e) $: per $ g \neq e $, si ha $ D_{ii}(g) = 0 \,\forall i = 1, \dots, n $. Di conseguenza, nella rappresentazione regolare:
\begin{equation}
	\chi(g) =
	\begin{cases}
		0 & g \neq e \\
		n & g = e
	\end{cases}
	\label{eq:8.17}
\end{equation}
Dall'Eq. \ref{eq:8.15} si ricava che $ a_{\mu} = \chi_{\mu}(e) $, ovvero:
\begin{equation}
	\alpha_{\mu} = n_{\mu}
	\label{eq:8.18}
\end{equation}
Dunque, nella rappresentazione regolare ogni rappresentazione irriducibile è presente un numero di volte pari al suo grado. Inoltre, ponendo $ g = e $ in Eq. \ref{eq:8.14} si trova:
\begin{equation}
	\sum_{\mu = 1}^{r} n_{\mu}^2 = n
	\label{eq:8.19}
\end{equation}
Di conseguenza, risulta che:
\begin{equation}
	\frac{1}{\abs{G}} \sum_{\mu = 1}^{r} \chi_{\mu}(e) \chi_{\mu}(g) =
	\begin{cases}
		0 & g \neq e \\
		1 & g = e
	\end{cases}
	\label{eq:8.20}
\end{equation}

\subsection{Tavola dei caratteri}

La \textit{tavola dei caratteri} di un gruppo finito corrisponde ad una tabella in cui le righe corrispondono alle sue rappresentazioni irriducibili e le colonne alle sue classi di coniugio: i suoi elementi corrispondono dunque ai caratteri di classe $ \chi_{\mu}(c) $.\\
Per stilare la tavola dei caratteri di un gruppo, è utile tenere a mente delle proprietà fondamentali derivanti dal grande teorema di ortogonalità già dimostrate:
\begin{enumerate}
	\item i gradi delle rappresentazioni irriducibili sono vincolati da (Eq. \ref{eq:8.19}):
		\begin{equation}
			\sum_{\mu = 1}^{r} n_{\mu}^2 = \abs{G}
			\label{eq:8.21}
		\end{equation}
	\item la tavola dei caratteri è quadrata:
		\begin{equation}
			k = r
			\label{eq:8.22}
		\end{equation}
	\item le righe sono ortogonali (Eq. \ref{eq:8.11}):
		\begin{equation}
			\frac{1}{\abs{G}} \sum_{c = 1}^{k} n_c \chi_{\mu}(c) \overline{\chi_{\nu}(c)} = \delta_{\mu \nu}
			\label{eq:8.23}
		\end{equation}
	\item le colonne sono ortogonali (Eq. \ref{eq:8.12}):
		\begin{equation}
			\frac{1}{\abs{G}} \sum_{\mu = 1}^{r} n_c \chi_{\mu}(c) \overline{\chi_{\mu}(d)} = \delta_{cd}
			\label{eq:8.24}
		\end{equation}
\end{enumerate}
Risulta inoltre utile ricordare che data una rappresentazione $ \rho $ completamente riducibile, con $ \chi $ il carattere ad essa associato, e detto $ \alpha_{\mu} $ il numero di volte che la rappresentazione irriducibile $ \rho_{\mu} $ appare in $ \rho $, si ha (Eq. \ref{eq:8.15}):
\begin{equation}
	\frac{1}{\abs{G}} \sum_{c = 1}^{k} n_c \chi(c) \overline{\chi_{\mu}(c)} = \alpha_{\mu}
	\label{eq:8.25}
\end{equation}
Sommando su $ \mu $ si ottiene quindi:
\begin{equation}
	\frac{1}{\abs{G}} \sum_{c = 1}^{k} n_c \chi(c) \overline{\chi(c)} = \sum_{\mu = 1}^{r} \alpha_{\mu}^2
	\label{eq:8.26}
\end{equation}
Si noti inoltre che per una rappresentazione irriducibile di dimensione unitaria i caratteri rispecchiano la legge moltiplicativa di gruppo (dato che la traccia di uno scalare è lo scalare stesso).\\
Infine, la trattazione dei gruppi abeliana risulta semplificata dalla seguente proprietà.

\begin{proposition}\label{abel-coniug}
	Ogni elemento di un gruppo abeliano costituisce una classe di coniugio a sé.
\end{proposition}
\begin{proof}
	Dato $ G $ abeliano, $ gag^{-1} = agg^{-1} = a \,\forall a,g \in G $, dunque $ a \sim a \,\forall a \in G $.
\end{proof}

\subsubsection{Tavola dei caratteri di \texorpdfstring{$ C_3 $}{TEXT}}

Il gruppo delle rotazioni di un triangolo regolare consta di tre elementi: $ C_3 = \{e,c,c^2\} $, dove $ c $ è una rotazione di $ 120^{\circ} $ rispetto al suo centro. Essendo un gruppo abeliano, per la Prop. \ref{irrep-unidim} si ha che tutte le sue rappresentazioni irriducibili sono unidimensionali; questo poteva alternativamente essere visto dal vincolo Eq. \ref{eq:8.21}: dalla Prop. \ref{abel-coniug} si ha che $ k = \abs{G} = 3 $, ma $ r = k $, dunque ci sono tre rappresentazioni irriducibili vincolate da $ n_1^2 + n_2^2 + n_3^2 = 3 $, la cui unica soluzione è $ n_1 = n_2 = n_3 = 1 $ (poiché $ n_{\mu} \in \N $).\\
La prima rappresentazione irriducibile unidimensionale, indicata con $ \mathtt{1} $, è quella banale in cui tutti gli elementi del gruppo $ G $ sono rappresentati da $ \rho_g = \id \,\forall g \in G $ (è comune in tutti i gruppi).\\
Le altre due rappresentazioni $ \mathtt{1}' $ e $ \mathtt{1}'' $ possono essere trovate sfruttando la unidimensionalità di tali rappresentazioni (i cartteri seguono le leggi di gruppo):
\begin{equation*}
	c^3 = e \quad \Rightarrow \quad \chi(c^3) = \chi(e) \quad \Rightarrow \quad (\chi(c))^3 = 1 \quad \Rightarrow \quad \chi(c) = 1, \omega, \omega^2
\end{equation*}
dove $ \omega \equiv e^{2\pi i / 3} $ è la prima radice cubica complessa dell'unità. La soluzione $ \chi(c) = 1 $ dà la rappresentazione banale, mentre le altre due danno $ \mathtt{1}' $ e $ \mathtt{1}'' $. Da $ c^{-1} = c^2 $ si trova quindi la tavola dei caratteri di $ C_3 $:

\begin{table}[H]
	\centering
	\begin{tabular}{c|ccc}
		$ C_3 $ & $ e $ & $ c $ & $ c^2 $ \\
		\hline
		$ \mathtt{1} $ & 1 & 1 & 1 \\
		$ \mathtt{1}' $ & 1 & $ \omega $ & $ \omega^2 $ \\
		$ \mathtt{1}'' $ & 1 & $ \omega^2 $ & $ \omega $
	\end{tabular}
\end{table}

Un altro modo di vedere queste rappresentazioni per via geometrica è considerando l'azione dei vari elementi del gruppo in $ \R^3 $. In generale, una rotazione antioraria di un angolo $ \phi $ nel piano $ xy $ è rappresentata da:
\begin{equation}
	\ve{v}' =
	\begin{bmatrix}
		\cos \phi & -\sin \phi & 0 \\
		\sin \phi & \cos \phi & 0 \\
		0 & 0 & 1
	\end{bmatrix}
	\ve{v}
	\label{eq:8.27}
\end{equation}
Si definisce questa rappresentazione $ \mathtt{V} $. Si vede dunque che:
\begin{equation*}
	\rho(e) =
	\begin{bmatrix}
		1 & 0 & 0 \\
		0 & 1 & 0 \\
		0 & 0 & 1
	\end{bmatrix}
	\qquad
	\rho(c) =
	\begin{bmatrix}
		-1/2 & -\sqrt{3}/2 & 0 \\
		\sqrt{3}/2 & -1/2 & 0 \\
		0 & 0 & 1
	\end{bmatrix}
	\qquad
	\rho(c^2) =
	\begin{bmatrix}
		-1/2 & \sqrt{3}/2 & 0 \\
		-\sqrt{3}/2 & -1/2 & 0 \\
		0 & 0 & 1
	\end{bmatrix}
\end{equation*}
È possibile decomporre in due rappresentazioni: quella triviale irriducibile $ \mathtt{1} $, data dall'azione sulla componente $ z $, e una $ \Gamma $ che si può mostrare essere completamente riducibile data dall'azione sulla componente nel piano $ xy $.
Si può anche vedere la rotazione di un angolo $ \phi $ nel piano di Argand:
\begin{equation}
	z' = e^{\pm i\phi} z
	\label{eq:8.28}
\end{equation}
dove $ \pm $ dà il verso della rotazione. Si ha quindi:
\begin{equation*}
	\rho(e) = 1
	\qquad
	\rho(c) = e^{\pm 2\pi i / 3}
	\qquad
	\rho(c^2) = e^{\pm 4\pi / 3}
\end{equation*}
Risulta quindi che la rappresentazione antioraria è $ \mathtt{1}' $, mentre quella oraria è $ \mathtt{1}'' $. Si ha inoltre $ \Gamma = \mathtt{1}' \oplus \mathtt{1}'' $, dunque $ \mathtt{V} = \mathtt{1} \oplus \mathtt{1}' \oplus \mathtt{1}'' $, che è la decomposizione completa.

\paragraph{Decomposizione}

È possibile determinare la decomposizione di $ \mathtt{V} $ anche studiando i caratteri. Innanzitutto si vede che:
\begin{equation*}
	\chi_{\mathtt{V}} = (\chi_{\mathtt{V}}(e), \chi_{\mathtt{V}}(c), \chi_{\mathtt{V}}(c^2)) = (3,0,0)
\end{equation*}
I coefficienti di decomposizione $ \alpha_{\mu} $ sono dati dall'Eq. \ref{eq:8.25}:
\begin{equation*}
	\alpha_{\mathtt{1}} = \frac{1}{3} (1 \cdot 3 \cdot 1 + 1 \cdot 0 \cdot 1 + 1 \cdot 0 \cdot 1) = 1
\end{equation*}
\begin{equation*}
	\alpha_{\mathtt{1}'} = \frac{1}{3} (1 \cdot 3 \cdot 1 + 1 \cdot 0 \cdot \overline{\omega} + 1 \cdot 0 \cdot \overline{\omega^2}) = 1
\end{equation*}
\begin{equation*}
	\alpha_{\mathtt{1}'} = \frac{1}{3} (1 \cdot 3 \cdot 1 + 1 \cdot 0 \cdot \overline{\omega^2} + 1 \cdot 0 \cdot \overline{\omega}) = 1
\end{equation*}
Banalmente, si poteva anche evitare il calcolo esplicito vedendo che:
\begin{equation*}
	\alpha_{\mu} = \braket{\chi,\chi_{\mu}} = \frac{1}{3} (3 \chi_{\mu}(e)) = \chi_{\mu}(e) = 1
\end{equation*}
In ogni caso, si recupera la corretta decomposizione.

\paragraph{Rotazioni}

Dall'Eq. \ref{eq:8.27} si ottiene una rappresentazione $ \mathtt{V} $ per $ \SOn{2} $: questo, nonostante sia un gruppo continuo, essendo un gruppo compatto ha le proprietà analoghe ai gruppi finiti. In particolare, essendo abeliano, ogni elemento $ c_{\phi} $ forma una classe di coniugio a sé; inoltre, si hanno la rappresentazione banale irriducibile $ \mathtt{1} $ e la rappresentazione $ \Gamma $ ottenute da $ \mathtt{V} $ analogamente a $ C_3 $, dove $ \chi_{\Gamma}(c_{\phi}) = 2\cos \phi $. Anche in questo caso $ \Gamma $ può essere decomposta considerando le rotazioni nel piano di Argand, ritrovando le rappresentazioni irriducibili $ \mathtt{1}' $ e $ \mathtt{1}'' $:

\begin{table}[H]
	\centering
	\begin{tabular}{c|cc}
		$ \SOn{2} $ & $ e $ & $ c_{\phi} $ \\
		\hline
		$ \mathtt{1} $ & 1 & 1 \\
		$ \mathtt{1}' $ & 1 & $ e^{i \phi} $ \\
		$ \mathtt{1}'' $ & 1 & $ e^{-i \phi} $
	\end{tabular}
\end{table}

Ciò è conseguenza di un importante risultato: $ \SOn{2} \cong \Un{1} $.

\subsubsection{Tavola dei caratteri di \texorpdfstring{$ D_3 $}{TEXT}}

Il gruppo diedrale $ D_3 $ comprende le rotazioni di $ C_3 $ con l'aggiunta delle riflessioni rispetto alle mediane tre mediale: $ D_3 = \{e,c,c^2,b,bc,bc^2\} $. La rotazione $ c $ è detta \textit{3-fold} poiché $ c^3 = e $, mentre $ b $ è una rotazione \textit{2-fold} (riflessione) poiché $ b^2 = e $.\\
Si vede che le classi di coniugio sono $ K_1 \equiv \{e\} $, $ K_2 = \{c,c^2\} $ ($ c^2 = bcb $) e $ K_3 = \{b,bc,bc^2\} $, dunque sono presenti 3 rappresentazioni irriducibili: $ n_1^2 + n_2^2 + n_3^2 = 6 $, dunque $ n_1 = n_2 = 1 $ e $ n_3 = 2 $, il che determina la prima colonna della tavola poiché $ \chi_{\mu}(e) = n_{\mu} $; la prima riga è invece determinata dalla rappresentazione banale $ \mathtt{1} $.\\
Per quanto riguarda la rappresentazione $ \mathtt{1}' $, essendo monodimensionale si ha $ \chi_{\mathtt{1}'}(bc) = \chi_{\mathtt{1}'}(b) \chi_{\mathtt{1}'}(c) $, ma $ bc $ e $ b $ sono nella stessa classe di coniugio, dunque si deve avere $ \chi_{\mathtt{1}'}(c) = 1 $. Inoltre, $ b^2 = e $ implica $ \chi_{\mathtt{1}'}(b) = \pm 1 $: la soluzione positiva dà la rappresentazione banale, dunque $ \chi_{\mathtt{1}'}(b) = -1 $:

\begin{table}[H]
	\centering
	\begin{tabular}{c|ccc}
		$ D_3 $ & $ K_1 $ & $ K_2 $ & $ K_3 $ \\
		\hline
		$ \mathtt{1} $ & 1 & 1 & 1 \\
		$ \mathtt{1}' $ & 1 & 1 & -1 \\
		$ \mathtt{2} $ & 2 & $ \alpha $ & $ \beta $
	\end{tabular}
\end{table}

Per determinare $ \alpha $ e $ \beta $, si impone $ \braket{\chi_{\mathtt{2}},\chi_{\mathtt{1}}} = \braket{\chi_{\mathtt{2}},\chi_{\mathtt{1}'}} = 0 $:
\begin{equation*}
	\begin{cases}
		2 + 2\alpha + 3\beta = 0 \\
		2 + 2\alpha - 3\beta = 0
	\end{cases}
	\quad \Rightarrow \quad
	\alpha = -1,\, \beta = 0
\end{equation*}
Si determina completamente la tavola dei caratteri:

\begin{table}[H]
	\centering
	\begin{tabular}{c|ccc}
		$ D_3 $ & $ K_1 $ & $ K_2 $ & $ K_3 $ \\
		\hline
		$ \mathtt{1} $ & 1 & 1 & 1 \\
		$ \mathtt{1}' $ & 1 & 1 & -1 \\
		$ \mathtt{2} $ & 2 & -1 & 0
	\end{tabular}
\end{table}

La rappresentazione vettoriale $ \mathtt{V} $ è analoga a quella per $ C_3 $, con l'aggiunta di $ b $ (può essere assunto WLOG come una rotazione di $ 180^{\circ} $ attorno all'asse $ x $):
\begin{equation*}
	\rho(b) =
	\begin{bmatrix}
		1 & 0 & 0 \\
		0 & -1 & 0 \\
		0 & 0 & -1
	\end{bmatrix}
\end{equation*}

\paragraph{Decomposizione}

Si studia la decomposizione di $ \mathtt{V} $ a partire da:
\begin{equation*}
	\chi_{\mathtt{V}} = (\chi_{\mathtt{V}}(K_1),\chi_{\mathtt{V}}(K_2),\chi_{\mathtt{V}}(K_3)) = (3,0,-1)
\end{equation*}
Ricordando che $ \alpha_{\mu} = \braket{\chi,\chi_{\mu}} $:
\begin{equation*}
	\alpha_{\mathtt{1}} = \frac{1}{6} (1 \cdot 3 \cdot 1 + 2 \cdot 0 \cdot 1 + 3 \cdot (-1) \cdot 1) = 0
\end{equation*}
\begin{equation*}
	\alpha_{\mathtt{1}'} = \frac{1}{6} (1 \cdot 3 \cdot 1 + 2 \cdot 0 \cdot 1 + 3 \cdot (-1) \cdot (-1)) = 1
\end{equation*}
\begin{equation*}
	\alpha_{\mathtt{2}} = \frac{1}{6} (1 \cdot 3 \cdot 2 + 2 \cdot 0 \cdot (-1) + 3 \cdot (-1) \cdot 0) = 1
\end{equation*}
Si trova dunque che $ \mathtt{V} = \mathtt{1}' \oplus \mathtt{2} $.

\section{Prodotto diretto di rappresentazioni}

\begin{definition}
	Dato un gruppo $ G $, il \textit{prodotto diretto} di due sue rappresentazioni $ \rho_{\mu}, \rho_{\nu} $, indicato con $ \rho_{\mu \times \nu} \defeq \rho_{\mu} \otimes \rho_{\nu} $, è definito come:
	\begin{equation}
		[\rho_{\mu \times \nu}(g)]_{ab,ij} \defeq [\rho_{\mu}(g)]_{ab} [\rho_{\nu}(g)]_{ij}
		\label{eq:8.29}
	\end{equation}
	con $ a,b = 1,\dots,n_{\mu} $ e $ i,j = 1,\dots,n_{\nu} $.
\end{definition}

\begin{proposition}
	$ \rho_{\mu \times \nu} $ è una rappresentazione di $ G $.
\end{proposition}
\begin{proof}
	Basta mostrare che è soddisfatta la legge moltiplicativa di gruppo:
	\begin{equation*}
		\begin{split}
			\rho_{\mu \times \nu}(g) \rho_{\mu \times \nu}(h)
			&= (\rho_{\mu}(g) \otimes \rho_{\nu}(g))(\rho_{\mu}(h) \otimes \rho_{\nu}(h)) \\
			&= (\rho_{\mu}(g) \rho_{\mu}(h)) \otimes (\rho_{\nu}(g) \rho_{\nu}(h)) = \rho_{\mu \times \nu}(gh)
		\end{split}
	\end{equation*}
\end{proof}

Il prodotto diretto di due rappresentazioni di grado $ n_{\mu} $ e $ n_{\nu} $ è una rappresentazione di grado $ n_{\mu} n_{\nu} $.

\begin{proposition}
	$ \chi_{\mu \times \nu}(c) = \chi_{\mu}(c) \chi_{\nu}(c) $.
\end{proposition}
\begin{proof}
	$ \chi_{\mu \times \nu}(c) = \sum_{a = 1}^{n_{\mu}} \sum_{i = 1}^{n_{\nu}} [\rho_{\mu \times \nu}(c)]_{ai,ai} = \sum_{a = 1}^{n_{\mu}} \sum_{i = 1}^{n_{\nu}} [\rho_{\mu}(c)]_{aa} [\rho_{\nu}(c)]_{ii} = \chi_{\mu}(c) \chi_{\nu}(c) $.
\end{proof}

\begin{proposition}
	Date due rappresentazioni irriducibili $ \rho_{\mu}, \rho_{\nu} $ di un gruppo $ G $, si ha:
	\begin{equation}
		\rho_{\mu} \otimes \rho_{\nu} = \bigoplus_{\sigma = 1}^{r} \alpha_{\sigma} \rho_{\sigma}
		\label{eq:8.30}
	\end{equation}
	detta \textit{serie di Clebsch-Gordan}, dove i coefficienti sono dati da:
	\begin{equation}
		\alpha_{\sigma} = \braket{\chi_{\mu} \chi_{\nu}, \chi_{\sigma}}
		\label{eq:8.31}
	\end{equation}
\end{proposition}
\begin{proof}
	Banale ricordando l'Eq. \ref{eq:8.15}.
\end{proof}

Dunque, il prodotto diretto di due rappresentazioni irriducibile non è a priori irriducibile.

\begin{example}
	Considerando $ D_3 $, si introduce la notazione cristallografica $ \mathtt{1} \equiv \mathtt{A}_1 $, $ \mathtt{1}' \equiv \mathtt{A}_2 $ e $ \mathtt{2} \equiv \mathtt{E} $ (in generale, $ \mathtt{A} $ indica una rappresentazione irriducibile di grado 1 e $ \mathtt{E} $ una di grado 2). Si prenda $ \mathtt{E} \otimes \mathtt{E} $: dalla tavola dei caratteri si trova che il suo carattere è $ \chi = (4,1,0) $, dunque si possono calcolare i suoi coefficienti di Clebsch-Gordan dall'Eq. \ref{eq:8.31}:
	\begin{equation*}
		\alpha_{\mathtt{A}_1} = \frac{1}{6} (1 \cdot 4 \cdot 1 + 2 \cdot 1 \cdot 1 + 3 \cdot 0 \cdot 1) = 1
	\end{equation*}
	\begin{equation*}
		\alpha_{\mathtt{A}_2} = \frac{1}{6} (1 \cdot 4 \cdot 1 + 2 \cdot 1 \cdot 1 + 3 \cdot 0 \cdot (-1)) = 1
	\end{equation*}
	\begin{equation*}
		\alpha_{\mathtt{E}} = \frac{1}{6} (1 \cdot 4 \cdot 2 + 2 \cdot 1 \cdot (-1) + 3 \cdot 0 \cdot 0) = 1
	\end{equation*}
	Dunque si ha che per $ D_3 $:
	\begin{equation*}
		\mathtt{E} \otimes \mathtt{E} = \mathtt{A}_1 \oplus \mathtt{A}_2 \oplus \mathtt{E}
	\end{equation*}
\end{example}

\section{Applicazioni fisiche}

\subsection{Sistemi quantistici a più corpi}

Si consideri un sistema quantistico composto da due elettroni, descritti dalle funzioni d'onda $ \psi_a(\ve{x}) $ e $ \phi_b(\ve{x}) $. Si supponga inoltre che questi elettroni godano della simmetria rispetto ad un gruppo di trasformazioni $ G $: ad esempio, per elettroni liberi questo potrebbe essere $ \SOn{3} $, mentre per elettroni in un solido il gruppo puntuale del reticolo cristallino.\\
Ciascuna funzione d'onda trasforma sotto l'azione di un $ g \in G $ secondo una sua rappresentazione:
\begin{equation*}
	\psi'_a(\ve{x}) = [\rho_{\mu}(g)]_{ac} \psi_c(\ve{x})
\end{equation*}
\begin{equation*}
	\phi'_b(\ve{x}) = [\rho_{\nu}(g)]_{bd} \phi_d(\ve{x})
\end{equation*}
La funzione d'onda del sistema è data dal loro prodotto $ \psi_{ab}(\ve{x}) \defeq \psi_a(\ve{x}) \phi_b(\ve{x}) $, dunque essa trasforma secondo il gruppo $ G \times G $:
\begin{equation*}
	\psi'_{ab}(\ve{x}) = [\rho_{\mu}(g)]_{ac} [\rho_{\nu}(g)]_{bd} \psi_c(\ve{x}) \phi_d(\ve{x}) = [\rho_{\mu \times \nu}(g)]_{ac,bd} \psi_{cd}(\ve{x})
\end{equation*}
che è proprio il prodotto diretto delle due rappresentazioni.\\
Se in prima approssimazione si può ignorare l'interazione tra i due elettroni, allora ciascuno di essi può trasformare in maniera indipendente, ovvero il sistema è invariante per trasformazioni indipendenti $ g_1,g_2 \in G $:
\begin{equation*}
	[\rho_{\mu \times \nu}(g_1,g_2)]_{ac,bd} \defeq [\rho_{\mu}(g_1)]_{ac} [\rho_{\nu}(g_2)]_{bd}
\end{equation*}
In questo caso, se $ \rho_{\mu} $ e $ \rho_{\nu} $ sono rappresentazioni irriducibili di $ G $, anche $ \rho_{\mu \times \nu} $ lo è: se per assurdo così non fosse, esisterebbe un sottospazio proprio invariante per ogni $ g_1,g_2 \in G $, ma fissando $ g_2 = e $ ciò significa che $ \rho_{\mu} $ è riducibile, contrariamente all'ipotesi.\\
Se invece si tiene conto dell'interazione tra gli elettroni, la rotazione deve essere la stessa, dunque si recupera l'usuale prodotto diretto di rappresentazioni, che in generale è riducibile.

\subsection{Ferromagnetismo e ferroelettricità}

Dato un reticolo cristallino con una certa simmetria, ovvero con un certo gruppo puntuale $ G $, la sua capacità di mantenere una magnetizzazione permanente $ \ve{M} $ o una polarizzazione permanente $ \ve{P} $ dipende da $ G $, ed in particolare è possibile solo se questi vettori sono invarianti rispetto alle trasformazioni di $ G $ (si noti che questo è il gruppo puntuale, non quello spaziale, poiché le traslazioni microscopiche del reticolo non sono rilevanti); ciò può essere interpretato come il fatto che le proprietà fisiche del cristallo non possono cambiare sotto trasformazioni che portano il cristallo in sé stesso.\\
Essendo $ \ve{M} $ e $ \ve{P} $ vettori cartesiani, essi trasformano secondo la rappresentazione vettoriale $ \mathtt{V} $, ovvero:
\begin{equation*}
	\ve{M}' = \rho_{\mathtt{V}}(g) \ve{M}
\end{equation*}
Dunque, dato che la condizione per l'esistenza di $ \ve{M} $ e $ \ve{P} $ è $ \ve{M}' = \ve{M} $ e $ \ve{P}' = \ve{P} $, ciò significa che la rappresentazione di $ G $ deve contenere almeno una volta la rappresentazione banale $ \ve{A}_1 $: fisicamente, ciò corrisponde all'esistenza di almeno un asse fisso, lungo il quale possono puntare $ \ve{M} $ e $ \ve{P} $.\\
Va inoltre notato che, essendo $ \ve{M} $ un vettore assiale e $ \ve{P} $ un vettore polare, sotto operatore parità $ \ve{M} $ rimane invariato mentre $ \ve{P} $ cambia di segno: se $ G $ contiene delle riflessioni, il reticolo può mantenere soltanto uno tra $ \ve{M} $ e $ \ve{P} $, non entrambi.\\
Si consideri, per esempio, un reticolo con simmetria trigonale $ C_3 $: ricordando che per $ C_3 $ la rappresentazione vettoriale si decompone come $ \mathtt{V} = \mathtt{A}_1 \oplus \mathtt{A}_2 \oplus \mathtt{A}_3 $, si vede che è presente un asse invariante, fisicamente corrispondente all'asse $ z $.\\
La simemtria $ C_3 $ è quella di un triangolo con lati direzionali, dunque distingue le direzioni $ +z $ e $ -z $: questa viene persa considerando invece come gruppo di simmetria $ D_3 $, a causa delle rotazioni diedre (riflessioni). Nel caso di $ D_3 $, infatti, si ha che $ \mathtt{V} = \mathtt{A}_2 \oplus \mathtt{E} $, che non contiene la rappresentazione banale, quindi non ci sono assi invarianti ed il reticolo non può mantenere né $ \ve{M} $ né $ \ve{P} $.

\subsection{Tensore di conduttività}

In un mezzo anisotropo come un solido cristallino, la relazione tra il campo elettrico $ \ve{E} $ e la densità di corrente $ \ve{j} $ è in generale di natura tensoriale:
\begin{equation*}
	\ve{j} = \underline{\sigma} \ve{E}
\end{equation*}
dove $ \underline{\sigma} $ è il tensore di conduttività. Sotto azione del gruppo di simmetria $ G $ del reticolo:
\begin{equation*}
	\ve{j}' = \underline{\sigma}' \ve{E}'
	\quad \Rightarrow \quad
	\rho_{\mathtt{V}}(g) \ve{j} = \underline{\sigma}' \rho_{\mathtt{V}}(g) \ve{E}
	\quad \Rightarrow \quad
	\underline{\sigma} = \rho_{\mathtt{V}}(g)^{-1} \underline{\sigma}' \rho_{\mathtt{V}}(g)
\end{equation*}
Dato che la rappresentazione $ \mathtt{V} $ è reale, dal teorema di Maschke (Th. \ref{th-maschke}) si ha che $ \rho_{\mathtt{V}}(g) $ è ortogonale:
\begin{equation*}
	\underline{\sigma}' = \rho_{\mathtt{V}}(g) \underline{\sigma} \rho_{\mathtt{V}}(g)^{\intercal}
	\quad \Rightarrow \quad
	\sigma'_{ij} = [\rho_{\mathtt{V}}(g)]_{ik} [\rho_{\mathtt{V}}(g)]_{jl} \sigma_{kl}
\end{equation*}
che è proprio la legge di trasformazione di un tensore di rango 2. Si vede quindi che $ \underline{\sigma} $ trasforma secondo $ \mathtt{V} \otimes \mathtt{V} $. Inoltre, si dimostra che l'invarianza per inversione temporale implica la simmetria del tensore di conduttività.\\
In generale, un tensore di rango 2 può essere separato nelle sue parti simmetrica e antisimmetrica:
\begin{equation}
	T_{ij} = T_{\{ij\}} + T_{[ij]}
	\label{eq:8.32}
\end{equation}
\begin{equation}
	T_{\{ij\}} \defeq \frac{1}{2} \left( T_{ij} + T_{ji} \right)
	\label{eq:8.33}
\end{equation}
\begin{equation}
	T_{[ij]} \defeq \frac{1}{2} \left( T_{ij} - T_{ji} \right)
	\label{eq:8.34}
\end{equation}
Le parti simmetrica e antisimmetrica formano due sottospazi invarianti, trasformando in maniera indipendente; le relative rappresentazioni simmetrica e antisimmetrica si trovano essere:
\begin{equation}
	[\rho_{\mathtt{V} \otimes \mathtt{V}}^{\pm}(g)]_{ij,kl} \equiv \frac{1}{2} \left( [\rho_{\mathtt{V}}(g)]_{ik} [\rho_{\mathtt{V}}(g)]_{jl} \pm [\rho_{\mathtt{V}}(g)]_{il} [\rho_{\mathtt{V}}(g)]_{jk} \right)
	\label{eq:8.35}
\end{equation}
Contraendo con $ \delta_{ik} \delta_{jl} $ si trova la relazione per i caratteri:
\begin{equation}
	\chi_{\mathtt{V} \otimes \mathtt{V}}^{\pm}(c) = \frac{1}{2} \left[ (\chi_{\mathtt{V}}(c))^2 \pm \chi_{\mathtt{V}}(c^2) \right]
	\label{eq:8.36}
\end{equation}
La condizione sul tensore di conduttività è quindi che esso sia invariante rispetto alla rappresentazione $ (\mathtt{V} \otimes \mathtt{V})^+ $ del gruppo di simmetria $ G $ del reticolo cristallino.\\
Si consideri ad esempio la simmetria diedrale $ D_3 $:

\begin{table}[H]
	\centering
	\begin{tabular}{c|ccc}
		$ D_3 $ & $ \{e\} $ & $ \{c,c^2\} $ & $ \{b,bc,bc^2\} $ \\
		\hline
		$ \mathtt{V} $ & $ 3 $ & $ 0 $ & $ -1 $ \\
		$ \left( \mathtt{V} \otimes \mathtt{V} \right)^+ $ & $ 6 $ & $ 0 $ & $ 2 $ \\
		$ \left( \mathtt{V} \otimes \mathtt{V} \right)^- $ & $ 3 $ & $ 0 $ & $ -1 $
	\end{tabular}
\end{table}
Innanzitutto si può notare che $ \underline{\sigma}_{ij} = \sigma \delta_{ij} $ è invariante rispetto a tutte le rotazioni (equivale all'ortogonalità delle matrici di rotazione), mentre altre possibili espressioni per il tensore di conduttività vanno cercate studiando la decomposizione di $ \left( \mathtt{V} \otimes \mathtt{V} \right)^+ $, ed in particolare quante volte la rappresentazione banale $ \mathtt{A}_1 $ appare in essa:
\begin{equation*}
	\alpha_{\mathtt{A}_1} = \braket{\chi_{\mathtt{V} \otimes \mathtt{V}}^+, \chi_{\mathtt{A}_1}} = \frac{1}{6} (1 \cdot 6 \cdot 1 + 2 \cdot 0 \cdot 1 + 3 \cdot 2 \cdot 1) = 2
\end{equation*}
Dunque, oltre a quella già trovata, ci deve essere un'altra possibile espressione per $ \underline{\sigma} $: è facile vedere che questa è proporzionale a $ \delta_{i3} \delta_{j3} $, poiché la forma quadratica $ z^2 $ è invariante (oltre a $ x^2 + y^2 + z^2 $), quindi l'espressione più generale per il tensore di conduttività è:
\begin{equation*}
	\underline{\sigma} =
	\begin{bmatrix}
		a & 0 & 0 \\
		0 & a & 0 \\
		0 & 0 & b
	\end{bmatrix}
\end{equation*}
Se invece si considera un reticolo a simmetria cubica $ T $, si trova che $ \alpha_{\mathtt{A}_1} = 1 $, dunque l'unica espressione possibile è $ \underline{\sigma}_{ij} = \sigma \delta_{ij} $, ovvero il reticolo è macroscopicamente isotropo.

\subsection{Degenerazione energetica}

In generale, il problema agli autovalori per l'energia di un sistema quantistico è:
\begin{equation*}
	\hat{\mathcal{H}} \ket{\psi_{\alpha}} = E_{\alpha} \ket{\psi_{\alpha}}
\end{equation*}
con $ E_{\alpha} \neq E_{\beta} $ per $ \alpha \neq \beta $. Nel caso in cui, però, esista un sottoinsieme di autofunzioni $ \{\ket{\psi_a}\}_{a = 1,\dots,d} $ associate allo stesso autovalore di energia $ E $, si parla di degenerazione $ d $-fold.\\
Si supponga che il sistema goda della simmetria rispetto ad un gruppo $ G $, con trasformazioni date da una rappresentazione unitaria $ \hat{U}(g) $:
\begin{equation*}
	\hat{U}(g)^{\dagger} \hat{\mathcal{H}} \hat{U}(g) = \hat{\mathcal{H}}
	\quad \Leftrightarrow \quad
	[\hat{\mathcal{H}}, \hat{U}(g)] = 0
\end{equation*}
Da ciò si ottiene facilmente che:
\begin{equation*}
	\hat{\mathcal{H}} (\hat{U}(g) \ket{\psi_{\alpha}}) = \hat{U}(g) (\hat{\mathcal{H}} \ket{\psi_{\alpha}}) = E_{\alpha} \hat{U}(g) \ket{\psi_{\alpha}}
\end{equation*}
Dunque, gli stati trasformati $ \ket{\psi'_{\alpha}} = \hat{U}(g) \ket{\psi_{\alpha}} $ sono ancora associati allo stesso autovalore di energia: se lo stato appartiene ad un sottospazio degenere di dimensione $ d $ (degenerazione), si deve avere:
\begin{equation*}
	\ket{\psi'_a} = [U(g)]_{ab} \ket{\psi_b}
\end{equation*}
con $ a,b = 1,\dots,d $. Si trova così una rappresentazione irriducibile di $ G $ di grado $ d $ (degenerazione).\\
In generale, vale che ad ogni sottospazio degenere di dimensione $ d $ è associata una rappresentazione irriducibile di grado $ d $ del gruppo di simmetria, e viceversa. Qualora fosse presente ulteriore degenerazione, si parla di $ \virgolette{degenerazione accidentale} $, che può essere dovuta ad un'altra simmetria $ \virgolette{nascosta} $.\\
Si consideri ad esempio lo spettro energetico di una particella in un potenziale centrale: il gruppo di simmetria è $ \SOn{3} $, le cui rappresentazioni irriducibili sono etichettate dall'intero $ \ell $ (associato con la parte angolare della funzione d'onda) e sono di grado $ 2\ell + 1 $. Se il potenziale centrale è generico, allora i livelli energetici $ E_{n,\ell} $ saranno distinti; se invece si considera il potenziale coulombiano $ U(r) = - \frac{k}{r} $ si ha una degenerazione accidentale tale per cui $ E_{n,\ell} = E_n $, con $ 0 \le \ell \le n - 1 $ e degenerazione $ d = n^2 $ (dovuta anche a $ -\ell \le m \le \ell $): questa degenerazione è dovuta al fatto che l'Hamiltoniana coulombiana è invariante rispetto a $ \SOn{4} $.

\subsubsection{Splitting}

Data un'Hamiltoniana $ \mathcal{H}_0 $ invariante rispetto ad un gruppo di simmetria $ G $, si introduca un termine d'interazione $ V $ tale per cui l'Hamiltoniana $ \mathcal{H}_1 \equiv \mathcal{H}_0 + V $ risulti invariante solo per un sottogruppo $ H < G $. Di conseguenza, i livelli energetici di $ \mathcal{H}_1 $ sono descritti dalle rappresentazioni irriducibili di $ H $, dunque per studiare il loro splitting è necessario determinare come le rappresentazioni irriducibili di $ G $ si decompongono in rappresentazioni irriducibili di $ H $ (in generale, restringendo una rappresentazione irriducibile ad un sottogruppo proprio essa non rimane irriducibile, dato che possono esserci nuovi sottospazi $ H $-invarianti).\\
Un esempio è il \textit{crystal field splitting}, ovvero lo splitting dei livelli energetici degeneri che avviene quanto un atomo libero viene posto all'interno di un reticolo cristallino: in questo caso, il gruppo di simmetria si riduce da $ \SOn{3} $ al gruppo puntuale del reticolo. Ad esempio, si consideri un reticolo a simmetria tetraedra $ T $ (cubica), le cui classi di coniugio sono $ E = \{e\} $, $ 3C_2 $, $ 4C_3 $ e $ 4C_3^2 $, dove $ kC_n $ (anche indicato con $ k\Z_n $) indica una classe di coniugio composta da $ k $ rotazioni di angolo $ \frac{2\pi}{n} $ (rotazioni attorno alle diagonali del cubo):

\begin{table}[H]
	\centering
	\begin{tabular}{c|cccc}
		$ T $ & $ E $ & $ 3C_2 $ & $ 4C_3 $ & $ 4C_3^2 $ \\
		\hline
		$ \mathtt{A} $ & $ 1 $ & $ 1 $ & $ 1 $ & $ 1 $ \\
		$ \mathtt{E}_1 $ & $ 1 $ & $ 1 $ & $ \omega $ & $ \omega^2 $ \\
		$ \mathtt{E}_2 $ & $ 1 $ & $ 1 $ & $ \omega^2 $ & $ \omega $ \\
		$ \mathtt{T} $ & $ 3 $ & $ -1 $ & $ 0 $ & $ 0 $
	\end{tabular}
\end{table}
dove $ \mathtt{E} = \mathtt{E}_1 \oplus \mathtt{E}_2 $. Si dimostra inoltre che il carattere della rappresentazione irriducibile $ \ell $ di $ \SOn{3} $ è (Eq. \ref{eq:9.24}, con $ \ell $ intero):
\begin{equation}
	\chi_{\ell}(\theta) = \frac{\sin \left( \left( \ell + \frac{1}{2} \right) \theta \right)}{\sin \left( \frac{1}{2} \theta \right)} = 1 + 2 \sum_{k = 1}^{\ell} \cos k\theta
	\label{eq:8.37}
\end{equation}
Per esempio, si può mostrare lo splitting dei livelli energetici di un orbitale $ d $ ($ \ell = 2 $): dato che in $ T $ le rotazioni possibili sono solo con $ \theta = 0, \frac{2\pi}{3}, \pi, \frac{4\pi}{3} $, si ha $ \chi_2 = (5,1,-1,-1) $. Dalla tavola dei caratteri:
\begin{equation*}
	\alpha_{\mathtt{A}} = \frac{1}{12} (1 \cdot 5 \cdot 1 + 3 \cdot 1 \cdot 1 + 4 \cdot (-1) \cdot 1 + 4 \cdot (-1) \cdot 1) = 0
\end{equation*}
\begin{equation*}
	\alpha_{\mathtt{E}_1} = \frac{1}{12} (1 \cdot 5 \cdot 1 + 3 \cdot 1 \cdot 1 + 4 \cdot (-1) \cdot \overline{\omega} + 4 \cdot (-1) \cdot \overline{\omega^2}) = 1
\end{equation*}
\begin{equation*}
	\alpha_{\mathtt{E}_2} = \frac{1}{12} (1 \cdot 5 \cdot 1 + 3 \cdot 1 \cdot 1 + 4 \cdot (-1) \cdot \overline{\omega^2} + 4 \cdot (-1) \cdot \overline{\omega}) = 1
\end{equation*}
\begin{equation*}
	\alpha_{\mathtt{A}} = \frac{1}{12} (1 \cdot 5 \cdot 3 + 3 \cdot 1 \cdot (-1) + 4 \cdot (-1) \cdot 0 + 4 \cdot (-1) \cdot 0) = 1
\end{equation*}
Si ottiene quindi:
\begin{equation*}
	\rho_2 = \mathtt{E}_1 \oplus \mathtt{E}_2 \oplus \mathtt{T}
\end{equation*}
Si vede che la simmetria $ \SOn{3} $ dell'orbitale $ d $, che ha una degenerazione 5-fold, viene rotta dal campo anisotropo del reticolo cristallino: la rappresentazione irriducibile in $ \SOn{3} $ di grado 5 viene decomposta in $ T $ in rappresentazioni irriducibili di grado 1, 1 e 3 (due non degeneri e una degenere 3-fold). Si può anche notare che, in realtà, la simmetria per inversione temporale fa sì che ci sia una degenerazione accidentale, per cui i livelli energetici associati ad $ \mathtt{E}_1 $ ed $ \mathtt{E}_2 $ hanno la stessa energia, e per questo possono essere considerati come un singolo livello energetico associato a $ \mathtt{E} $: questa degenerazione viene splittata in presenza di un campo magnetico.










