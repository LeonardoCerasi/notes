\selectlanguage{italian}

\section{Principio d'azione stazionaria}

\subsection{Caso classico}

\begin{definition}
	Dato un sistema descritto da una lagrangiana $ L $, si definisce l'azione come:
	\begin{equation}
		S[\ve{x}(t)] \defeq \int_{t_1}^{t_2} L(\ve{x}, \dot{\ve{x}}) dt
		\label{eq:5.1}
	\end{equation}
\end{definition}

Il principio di minima azione afferma che la traiettoria percorsa dal sistema è un estremante dell'azione, ovvero, considerata una variazione della traiettoria $ \delta\ve{x} : \delta\ve{x}(t_1) = \delta\ve{x}(t_2) = \ve{0} $:
\begin{equation}
	\delta S = 0
	\label{eq:5.2}
\end{equation}

\subsubsection{Particella libera}

Classicamente, una particella libera è descritta da $ L = \frac{1}{2} m \dot{\ve{x}}^2 $, dunque:
\begin{equation*}
	\begin{split}
		0
		&= \delta S = \int_{t_1}^{t_2} \delta L \,dt = \int_{t_1}^{t_2} m \dot{\ve{x}} \cdot \delta\dot{\ve{x}} \,dt = \int_{t_1}^{t_2} \left[ m \frac{d}{dt} \left( \dot{\ve{x}} \cdot \delta\ve{x} \right) - m \ddot{\ve{x}} \cdot \delta\ve{x} \right] dt\\
		&= m \left[ \ddot{\ve{x}} \cdot \delta\ve{x} \right]_{t_1}^{t_2} - m \int_{t_1}^{t_2} \ddot{\ve{x}} \cdot \delta\ve{x} \,dt = -m \int_{t_1}^{t_2} \ddot{\ve{x}} \cdot \delta\ve{x} \,dt
	\end{split}
\end{equation*}
Dunque, data l'arbitrarietà di $ \delta\ve{x} $, si ha l'equazione del moto della particella libera classica:
\begin{equation}
	\ddot{\ve{x}} = \ve{0}
	\label{eq:5.3}
\end{equation}

\subsubsection{Moto in un potenziale}

In presenza di un potenziale, la lagrangiana diventa $ L = \frac{1}{2} m \dot{\ve{x}}^2 - V(\ve{x}) $, quindi, dato che $ \delta V(\ve{x}) = \nabla V(\ve{x}) \cdot \delta\ve{x} $, si ha l'equazione del moto:
\begin{equation}
	0 = \delta S = \int_{t_1}^{t_2} \left( - m\ddot{\ve{x}} - \nabla V(\ve{x}) \right) \cdot \delta\ve{x} \,dt \quad\Longrightarrow\quad m\ddot{\ve{x}} = - \nabla V(\ve{x})
	\label{eq:5.4}
\end{equation}

\subsection{Caso relativistico}

\begin{proposition}
	Una particella libera relativistica è descritta da $ L = - \frac{m c^2}{\gamma} $.
\end{proposition}
\begin{proof}
	$ p_i = \frac{\pa L}{\pa v_i} = - mc^2 \frac{\pa}{\pa v_i} \sqrt{1 - \frac{\ve{v}^2}{c^2}} = -mc^2 \left( -  \gamma \frac{v_i}{c} \right) = \gamma m v_i $, ovvero $ \ve{p} = \gamma m \ve{v} $.
\end{proof}

\subsubsection{Particella libera}

L'azione che descrive una particella libera relativistica è:
\begin{equation}
	S = - mc^2 \int_{t_1}^{t_2} \frac{dt}{\gamma} = -mc \int_{\tau_1}^{\tau_2} d\tau
	\label{eq:5.5}
\end{equation}
Ponendo $ c = 1 $, è possibile calcolare le equazioni del moto:
\begin{equation*}
	\begin{split}
		0
		&= \delta S = - \delta \int_{t_1}^{t_2} m \sqrt{1 - \dot{\ve{x}}^2} \,dt = m \int_{t_1}^{t_2} \frac{\dot{\ve{x}} \cdot \delta\dot{\ve{x}}}{\sqrt{1 - \dot{\ve{x}}^2}} dt = m \int_{t_1}^{t_2} \frac{d\ve{x}}{dt} \cdot \frac{d \delta\ve{x}}{dt} \frac{dt}{d\tau} \,dt\\
		&= m \int_{\tau_1}^{\tau_2} \frac{d\ve{x}}{dt} \cdot \frac{d \delta\ve{x}}{dt} \frac{dt}{d\tau} \frac{dt}{d\tau} \,d\tau = m \int_{\tau_1}^{\tau_2} \frac{d\ve{x}}{d\tau} \cdot \frac{d\delta\ve{x}}{d\tau} \,d\tau\\
		&= m \int_{\tau_1}^{\tau_2} \left[ \frac{d}{d\tau} \left( \dot{\ve{x}} \cdot \delta\ve{x} \right) - \frac{d^2\ve{x}}{d\tau^2} \cdot \delta\ve{x} \right] d\tau = - m \int_{\tau_1}^{\tau_2} \frac{d^2 \ve{x}}{d\tau^2} \cdot \delta\ve{x} \,d\tau
	\end{split}
\end{equation*}
Dall'arbitrarietà di $ \delta\ve{x} $ si ottiene l'equazione del moto:
\begin{equation}
	\frac{d^2 \ve{x}}{d\tau^2} = \ve{0}
	\label{eq:5.6}
\end{equation}

\subsubsection{Campo elettromagnetico}

Per descrivere il moto di una particella in un campo elettromagnetico, è necessario aggiungere un termine d'interazione alla lagrangiana:
\begin{equation}
	S = -m \int_{\tau_1}^{\tau_2} d\tau + q \int_{\ve{x}_1}^{\ve{x}_2} A_i(\ve{x}) dx^i = \int_{\tau_1}^{\tau_2} \left( -m + q A_i (\ve{x}) \frac{dx^i}{d\tau} \right) d\tau
	\label{eq:5.7}
\end{equation}
Per ottenere le equazioni del moto, dunque:
\begin{equation*}
	\begin{split}
		0
		&= \delta S = -m \int_{\tau_1}^{\tau_2} \delta\sqrt{-\eta_{ij} dx^i dx^j} + q \int_{\ve{x}_1}^{\ve{x}_2} \delta \left( A_i dx^i \right)\\
		&= -m \int_{\tau_1}^{\tau_2} \frac{1}{2} \frac{\delta \left( - \eta_{ij} dx^i dx^j \right)}{\sqrt{- \eta_{ij} dx^i dx^j}} + q \int_{\ve{x}_1}^{\ve{x}_2} \left( \delta A_i dx^i + A_i \delta dx^i \right)\\
		&= m \int_{\tau_1}^{\tau_2} \frac{1}{2} \frac{\eta_{ij} \left( \delta dx^i dx^j + dx^i \delta dx^j \right)}{\sqrt{- \eta_{ij} dx^i dx^j}} + q \int_{\tau_1}^{\tau_2} \left( \delta A_i \frac{dx^i}{d\tau} + A_i \frac{d\delta x^i}{d\tau} \right) d\tau\\
		&= \int_{\tau_1}^{\tau_2} \left( m \frac{\eta_{ij} dx^i}{\sqrt{- \eta_{ij} dx^i dx^j}} \frac{d\delta x^j}{d\tau} + q \frac{\pa A_i}{\pa x^k} \delta x^k \frac{dx^i}{d\tau} + q A_i \frac{d\delta x^i}{d\tau} \right) d\tau\\
		&= \int_{\tau_1}^{\tau_2} \left( m \eta_{ij} \frac{dx^i}{d\tau} \frac{d\delta x^j}{d\tau} + q A_{i,k} \delta x^k \frac{dx^i}{d\tau} + q A_i \frac{d\delta x^i}{d\tau} \right) d\tau\\
		&= \int_{\tau_1}^{\tau_2} \left( - m \frac{du_k}{d\tau} \delta x^k + q A_{i,k} u^i \delta x^k - q \frac{dA_k}{d\tau} \delta x^k \right) d\tau\\
		&= \int_{\tau_1}^{\tau_2} \left( -m \frac{du_k}{d\tau} + q \left( A_{i,k} - A_{k,i} \right) u^i \right) \delta x^k d\tau
	\end{split}
\end{equation*}
dove si è usata la quadrivelocità $ u_k $. Ricordando la definizione del tensore di Faraday $ F_{ij} \defeq A_{j,i} - A_{i,j} $, si trova l'espressione covariante della forza di Lorentz:
\begin{equation}
	\frac{dp_k}{d\tau} = q F_{ki} u^i
	\label{eq:5.8}
\end{equation}
Ciò conferma la scelta della lagrangiana.

\subsubsection{Campo gravitazionale}

Si consideri ora una particella su una varietà curva:
\begin{equation*}
	\begin{split}
		0
		&= \delta S = -m \int_{\tau_1}^{\tau_2} \delta d\tau = -m \int_{\tau_1}^{\tau_2} \delta \sqrt{- g_{ij} dx^i dx^j} = m \int_{\tau_1}^{\tau_2} \frac{1}{2} \frac{\delta g_{ij} dx^i dx^j + 2g_{ij} dx^i \delta dx^j}{\sqrt{- g_{ij} dx^i dx^j}}\\
		&= m \int_{\tau_1}^{\tau_2} \left( \frac{1}{2} g_{ij,k} u^i u^j \delta x^k d\tau + g_{ij} u^i \delta dx^j \right) = m \int_{\tau_1}^{\tau_2} \left( \frac{1}{2} g_{ij,k} u^i u^j \delta x^k + g_{ij} u^i \frac{d\delta x^j}{d\tau} \right) d\tau\\
		&= m \int_{\tau_1}^{\tau_2} \left( \frac{1}{2} g_{ij,k} u^i u^j \delta x^k - \frac{d}{d\tau} \left( g_{ij} u^i \right) \delta x^j \right) d\tau = m \int_{\tau_1}^{\tau_2} \left( \frac{1}{2} g_{ij,k} u^i u^j - \frac{d}{d\tau} \left( g_{ik} u^i \right) \right) \delta x^k d\tau
	\end{split}
\end{equation*}
Essendo $ \delta x^k $ arbitrario si ottiene:
\begin{equation}
	\frac{1}{2} g_{ij,k} u^i u^j - g_{ik,j} u^i u^j - g_{ik} \frac{du^i}{d\tau} = 0
	\label{eq:5.9}
\end{equation}
Moltiplicando per $ g^{rk} $:
\begin{equation}
	\frac{du^r}{d\tau} + g^{rk} \left( g_{ik,j} - \frac{1}{2} g_{ij,k} \right) u^i u^j = 0
	\label{eq:5.10}
\end{equation}
Dato che sopravvive solo la parte simmetrica rispetto a $ i,j $ del tensore moltiplicato per $ u^i u^j $:
\begin{equation}
	\frac{du^r}{d\tau} + \frac{1}{2} g^{rk} \left( g_{ik,j} + g_{jk,i} - g_{ij,k} \right) u^i u^j = 0
	\label{eq:5.11}
\end{equation}
Si vede la definizione di connessione di Levi-Civita (Eq. \ref{levi-civita}):
\begin{equation}
	\frac{d^2 x^r}{d\tau^2} + \Gamma^r_{ij} \frac{dx^i}{d\tau} \frac{dx^j}{d\tau} = 0
	\label{eq:5.12}
\end{equation}
Questa è nota come \textit{equazione geodetica} e descrive il moto di una particella libera su una varietà curva: si vede che, oltre al termine inerziale, è presente un termine di forza dovuto alla geometria stessa dello spazio; inoltre, si può osservare come la traiettoria sia indipendente dalla massa del corpo.
Sebbene la connessione di Levi-Civita non sia un tensore, si dimostra che il termine di forza annulla i termini non-omogenei, rendendo l'equazione geodesica un'equazione tensoriale.\\
In generale, le soluzioni dell'equazione geodetica sono dette geodetiche.
\begin{definition}
	Data una varietà differenziale $ \mathcal{M} $ e due punti $ p,q\in\mathcal{M} $, si dice \textit{geodetica} da $ p $ a $ q $ la curva di lunghezza minore tra $ p $ e $ q $.
\end{definition}
\begin{proposition}
	Dati una geodetica $ \gamma : I \subset \R \rightarrow \mathcal{M} $ ed il suo vettore tangente $ \ve{t}_{\gamma} $, si ha $ \frac{d\ve{t}_{\gamma}}{ds} = \ve{0} $.
\end{proposition}
\begin{proof}
	Ricordando che $ t^i = \frac{dx^i}{ds} $:
	\begin{equation*}
		\frac{d\ve{t}_{\gamma}}{ds} = t^i_{;k} \frac{dx^k}{ds} \ve{e}_i = \left( \frac{\pa t^i}{\pa x^k} + \Gamma^i_{jk} t^j \right) \frac{dx^k}{ds} \ve{e}_i = \left( \frac{d^2 x^i}{ds^2} + \Gamma^i_{jk} \frac{dx^j}{ds} \frac{dx^k}{ds} \right) \ve{e}_i = \ve{0}
	\end{equation*}
\end{proof}

\subsubsection{Campo elettromagnetico e gravitazionale}

Nel caso di una particella soggetta ad un campo elettromagnetico su una varietà curva:
\begin{equation}
	S = -m \int_{\tau_1}^{\tau_2} \sqrt{- g_{ij} dx^i dx^j} + q \int_{\tau_1}^{\tau_2} A_i \frac{dx^i}{d\tau} d\tau
	\label{eq:5.13}
\end{equation}
Svolgendo i calcoli, si trova l'equazione del moto:
\begin{equation}
	\frac{d^2 x^r}{d\tau^2} + \Gamma^r_{ij} \frac{dx^i}{d\tau} \frac{dx^j}{d\tau} - \frac{q}{m} F^r_{\,\,\,i} \frac{dx^i}{d\tau} = 0
	\label{eq:5.14}
\end{equation}
che conferma l'identificazione del termine geometrico con la forza di gravità.

\subsection{Principio d'equivalenza}

In presenza di un campo gravitazionale, dunque su una varietà curva, è possibile scegliere delle coordinate, dette free-fall coordinates, in cui, restringendosi ad una regione di spaziotempo sufficientemente piccola (così da poter trascurare localmente la curvatura, tutte le leggi fisiche hanno la stessa forma: ciò è possibile poiché, in tale regione, il tensore metrico può essere ricondotto con un cambio di RF alla metrica di Minkowski, dunque si recupera la fisica in spaziotempo piatto.\\
Perché ciò avvenga:
\begin{equation}
	\eta_{ij} = \frac{\pa x^k}{\pa y^i} \frac{\pa x^l}{\pa y^j} g_{kl} \quad\Longrightarrow\quad \eta = \tens{D}^{\intercal} g \tens{D}
	\label{eq:5.15}
\end{equation}
con $ \tens{D} = \tens{J}^{-1} $. Questa è una relazione di similitudine.\\
In tale RF, l'equazione del moto è quella della particella libera relativistica:
\begin{equation}
	\frac{d^2 y^i}{d \tau^2} = 0, \,\,\,\, d\tau^2 = - \eta_{ij} dx^i dx^j
	\label{eq:5.16}
\end{equation}
Queste coordinate sono anche dette coordiante libere o di Fermi.\\
È possibile recuperare l'equazione geodesica dall'Eq. \ref{eq:5.16}:
\begin{equation}
	0 = \frac{d}{d\tau} \left( \frac{\pa y^i}{\pa x^k} \frac{dx^k}{d\tau} \right) = \frac{\pa y^i}{\pa x^k} \frac{d^2 x^k}{d\tau^2} + \frac{\pa^2 y^i}{\pa x^l \pa x^k} \frac{dx^k}{d\tau} \frac{dx^l}{d\tau}
	\label{eq:5.17}
\end{equation}
Moltiplicando per $ \frac{\pa x^m}{\pa y^i} $:
\begin{equation}
	\frac{d^2 x^m}{d\tau^2} + \frac{\pa x^m}{\pa y^i} \frac{\pa^2 y^i}{\pa x^l \pa x^k} \frac{dx^k}{d\tau} \frac{dx^l}{d\tau}
	\label{eq:5.18}
\end{equation}
che corrisponde all'equazione geodetica ponendo:
\begin{equation}
	\Gamma^m_{kl} = \frac{\pa x^m}{\pa y^i} \frac{\pa^2 y^i}{\pa x^l \pa x^k}
	\label{eq:5.19}
\end{equation}
Questo poter approssimare localmente lo spaziotempo curvo con lo spaziotempo di Minkowski è il \textit{principio d'equivalenza di Einstein}.

\subsubsection{Principio d'equivalenza newtoniana}

Anche nella fisica newtoniana è presente un principio d'equivalenza. Si cosideri una particella massiva, descritta dall'equazione del moto:
\begin{equation}
	m_{\text{i}} \frac{d^2 \ve{x}}{dt^2} = m_{\text{g}} \ve{g} + \sum_{k} \ve{F}_k
	\label{eq:5.20}
\end{equation}
In questo caso, le coordinate free-fall sono definite da:
\begin{equation}
	\ve{y} = \ve{x} - \frac{1}{2} \ve{g} t^2
	\label{eq:5.21}
\end{equation}
In questo RF, l'equazione del moto diventa:
\begin{equation}
	m_{\text{i}} \left[ \frac{d^2\ve{y}}{dt^2} + \ve{g} \right] = m_{\text{g}} \ve{g} + \sum_{k} \ve{F}_k
	\label{eq:5.22}
\end{equation}
Il principio d'equivalenza newtoniano è un fatto sperimentale e afferma che:
\begin{equation}
	m_{\text{i}} = m_{\text{g}}
	\label{eq:5.23}
\end{equation}
Dunque, nel caso di una particella non soggetta ad altre forze, nelle coordinate free-fall l'equazione del moto diventa quella per la particella libera:
\begin{equation}
	\frac{d^2 \ve{y}}{dt^2} = \ve{0}
	\label{eq:5.24}
\end{equation}
Si vede dunque che le coordinate free-fall $ \virgolette{seguono} $ il moto del grave nel campo gravitazionale; in maniera analoga, le coordinate di Fermi individuano un RF che segue il moto della particella all'interno dello spaziotempo curvo.

\subsection{Campi gravitazionali deboli statici}

Le connessioni affini costituiscono una generalizzazione del campo di forza gravitazionale newtoniano, mentre il tensore metrico è una generalizzazione del potenziale gravitazionale newtoniano.\\
Per semplicità, si pongano le approssimazioni di campo debole (componenti spaziali trascurabili rispetto a quella temporale) e statico (derivate temporali trascurabili): l'equazione geodetica diventa:
\begin{equation}
	\frac{du^r}{d\tau} + \Gamma^r_{00} \left( u^0 \right)^2 = 0
	\label{eq:5.25}
\end{equation}
Dato che, nell'approssimazione fatta, $ g_{k0,0} = g_{0k,0} = 0 $, si ha $ \Gamma^r_{00} = -\frac{1}{2} g^{rk}g_{00,k} $. Nell'approssimazione di campo debole si può scrivere:
\begin{equation}
	g_{ij} = \eta_{ij} + h_{ij}, \quad \abs{h_{ij}} \ll 1
	\label{eq:5.26}
\end{equation}
Trascurando i termini $ o(h) $, si trova $ \Gamma^r_{00} = -\frac{1}{2} h_{00,r} $, con $ r = 1,2,3 $, quindi:
\begin{equation}
	\frac{d^2 x^r}{d\tau^2} = \frac{1}{2} \left( \frac{dx^0}{d\tau} \right)^2 h_{00,r} \quad\Longrightarrow\quad \frac{d^2 \ve{x}}{dt^2} = \frac{c^2}{2} \nabla h_{00}
	\label{eq:5.27}
\end{equation}
Confrontando con l'equazione classica $ \ve{a} = - \nabla \Phi(r) $, con $ \Phi(r) = - \frac{GM}{r} $ potenziale gravitazionale newtoniano, si trova il vincolo imposto dal caso subrelativistico:
\begin{equation}
	h_{00} = - \frac{2\Phi(r)}{c^2}
	\label{eq:5.28}
\end{equation}
$ \Phi / c^2 $ è adimensionale e vale $ \sim 10^{-39} $ sulla superficie di un protone, $ \sim 10^{-9} $ su quella terrestre, $ \sim 10^{-6} $ su quella del Sole e $ \sim 10^{-4} $ su quella di una nana bianca.

\subsubsection{Time dilation gravitazionale}

Nell'approssimazione di campo gravitazionale debole statico è possibile dare una stima dell'effetto di time dilation gravitazionale. Preso un orologio, il suo tempo proprio (tempo misurato dall'orologio) è:
\begin{equation}
	d\tau = \frac{1}{c} \sqrt{-g_{ij} dx^i dx^j} = \sqrt{-g_{00}} dt = \sqrt{ 1 - \frac{2GM}{c^2 r}} dt
	\label{eq:5.29}
\end{equation}
dove $ dt $ è il tempo misurato nello spazio vuoto. Si vede che $ d\tau < dt $.\\
Prendendo il caso di due corpi sulla Terra a distanze dal centro $ r $ ed $ r + h $:
\begin{equation}
	\frac{T_2}{T_1} = \sqrt{\frac{1 - \frac{2GM}{c^2(r+h)}}{1 - \frac{2GM}{c^2r}}} \approx 1 + \frac{gh}{c^2}
	\label{eq:5.30}
\end{equation}
dove è stato definito $ g \equiv \frac{GM}{r^2} $. Da ciò discende il redshift (o blueshift) gravitazionale, verificato sperimentalmente.

\section{Curvatura}

\begin{definition}
	Si definisce il \textit{tensore di Riemann} come:
	\begin{equation}
		R^i_{mnk} \defeq \Gamma^i_{nm,k} - \Gamma^i_{km,n} + \Gamma^i_{kj} \Gamma^j_{nm} - \Gamma^i_{nj} \Gamma^j_{km}
		\label{eq:5.31}
	\end{equation}
\end{definition}

Il tensore di Riemann può anche essere visto come commutatore di derivate covarianti.

\begin{proposition}
	$ R^i_{mnk} = \left[ \pa_k + \Gamma_k, \pa_n + \Gamma_n \right]^i_{\,\,\,m} $.
\end{proposition}
\begin{proof}
	Si considerano le connessioni affini come matrici $ [\Gamma_k]^i_{\,\,\,m} = \Gamma^i_{km} $:
	\begin{equation*}
		\begin{split}
			\left[ \pa_k + \Gamma_k, \pa_n + \Gamma_n \right]^i_{\,\,\,m}
			&= \left[ \Gamma_{n,k} - \Gamma_{k,n} + \Gamma_k \Gamma_n - \Gamma_n \Gamma_k \right]^i_{\,\,\,m}\\
			&= \Gamma^i_{nm,k} - \Gamma^i_{km,n} + \Gamma^i_{kj} \Gamma^j_{nm} - \Gamma^i_{nj} \Gamma^j_{km}
		\end{split}
	\end{equation*}

\end{proof}

Dunque, equivalentemente, si può scrivere:
\begin{equation}
	\left[ \nabla_k, \nabla_n \right] V^i = R^i_{mnk} V^m
	\label{eq:5.32}
\end{equation}
Il significato fisico del tensore di Riemann è quindi quello di fare trasporto parallelo di un vettore lungo un loop: se non ci fosse curvatura, l'operazione lascerebbe il vettore invariato, ed infatti il tensore di Riemann è detto anche \textit{tensore di curvatura}.

\begin{proposition}
	$ R^i_{mnk} = - R^i_{mkn} $.
\end{proposition}

\begin{definition}
	Si definisce \textit{tensore di Ricci} la contrazione $ R_{mk} \defeq R^n_{mnk} $.
\end{definition}

\begin{definition}
	Si definisce \textit{scalare di Ricci} la contrazione $ R \defeq g^{mk} R_{mk} $.
\end{definition}

\subsection{Curvatura della sfera}

È un utile esercizio calcolare la curvatura di una sfera.\\
Si consideri una 2-sfera di raggio $ r $ con RF $ (\theta,\phi) \in (0,\pi) \times [0,2\pi) $: le possibili connessioni di Levi-Civita sono 8, ma solo 6 sono indipendenti per simmetria negli indici inferiori. Prendendo $ \R^3 $ come embedding space:
\begin{equation*}
	\ve{p} = r \left( \sin \theta \cos \phi, \sin \theta \sin \phi, \cos \theta \right)
\end{equation*}
Calcolando i vettori base:
\begin{equation*}
	\ve{e}_{\theta} = \frac{\pa \ve{p}}{\pa \theta} = r \left( \cos \theta \cos \phi, \cos \theta \sin \phi, -\sin \theta \right) \equiv r \hat{\ve{e}}_{\theta}
\end{equation*}
\begin{equation*}
	\ve{e}_{\phi} = \frac{\pa \ve{p}}{\pa \phi} = r \sin \theta \left( - \sin \phi, \cos \theta, 0 \right) \equiv r \sin \theta \hat{\ve{e}}_{\phi}
\end{equation*}
Si trova dunque il tensore metrico della sfera:
\begin{equation*}
	g_{ij} =
	\begin{bmatrix}
		r^2 & 0 \\
		0 & r^2 \sin^2 \theta
	\end{bmatrix}
	\qquad g^{ij} =
	\begin{bmatrix}
		r^{-2} & 0 \\
		0 & r^{-2} \sin^{-2} \theta
	\end{bmatrix}
\end{equation*}
I vettori duali base risultano quindi essere $ \ve{e}^{\theta} = r^{-1} \hat{\ve{e}}_{\theta} $ e $ \ve{e}^{\phi} = r^{-1} \sin^{-1} \theta \hat{\ve{e}}_{\phi} $. Le derivate dei vettori base si calcolano essere:
\begin{equation*}
	\ve{e}_{\theta,\theta} = -r \hat{\ve{e}}_r \qquad \ve{e}_{\theta,\phi} = \ve{e}_{\phi,\theta} = r \cos \theta \hat{\ve{e}}_{\phi} \qquad \ve{e}_{\phi,\phi} = - r \sin \theta \left( \cos \phi, \sin \phi, 0 \right)
\end{equation*}
Ricordando che $ \Gamma^i_{jk} \defeq \ve{e}^i \cdot \ve{e}_{k,j} $ e che $ [\Gamma_k]^i_{\,\,\,m} = \Gamma^i_{km} $, si trovano:
\begin{equation*}
	\Gamma_{\theta} =
	\begin{bmatrix}
		0 & 0 \\
		0 & \cot \theta
	\end{bmatrix}
	\qquad \Gamma_{\phi} =
	\begin{bmatrix}
		0 & - \sin \theta \cos \theta \\
		\cot \theta & 0
	\end{bmatrix}
	\quad\Longrightarrow\quad \left[ \Gamma_{\theta},\Gamma_{\phi} \right] = - \left[ \Gamma_{\phi},\Gamma_{\theta} \right] =
	\begin{bmatrix}
		0 & \cos^2 \theta \\
		\cot^2 \theta & 0
	\end{bmatrix}
\end{equation*}
Per il calcolo del tensore di Ricci sono necessari solo 4 componenti del tensore di Riemann:
\begin{equation*}
	R^{\theta}_{\theta\theta\theta} = \left[ \pa_{\theta} + \Gamma_{\theta}, \pa_{\theta} + \Gamma_{\theta} \right]^{\theta}_{\,\,\,\theta} = 0
\end{equation*}
\begin{equation*}
	R^{\phi}_{\theta\phi\theta} = \left[ \pa_{\theta} + \Gamma_{\theta}, \pa_{\phi} + \Gamma_{\phi} \right]^{\phi}_{\,\,\,\theta} = [\Gamma_{\phi,\theta}]^{\phi}_{\,\,\,\theta} + \left[ \Gamma_{\theta},\Gamma_{\phi} \right]^{\phi}_{\,\,\,\theta} = (\cot \theta)_{,\theta} + \cot^2 \theta = -1
\end{equation*}
\begin{equation*}
	R^{\theta}_{\phi\theta\phi} = \left[ \pa_{\phi} + \Gamma_{\phi}, \pa_\theta + \Gamma_{\theta} \right]^{\theta}_{\,\,\,\phi} = - [\Gamma_{\phi,\theta}]^{\theta}_{\,\,\,\phi} = \cos^2 \theta - \sin^2 \theta - \cos^2 \theta = - \sin^2 \theta
\end{equation*}
\begin{equation*}
	R^{\phi}_{\phi\phi\phi} = \left[ \pa_{\phi} + \Gamma_{\phi}, \pa_{\phi} + \Gamma_{\phi} \right]^{\phi}_{\,\,\,\phi} = 0
\end{equation*}
Le componenti non-nulle del tensore di Ricci sono:
\begin{equation*}
	R_{\theta\theta} = R^{\theta}_{\theta\theta\theta} + R^{\phi}_{\theta\phi\theta} = -1
\end{equation*}
\begin{equation*}
	R_{\phi\phi} = R^{\theta}_{\phi\theta\phi} + R^{\phi}_{\phi\phi\phi} = - \sin^2 \theta
\end{equation*}
Dato che $ R = g^{mk} R_{mk} = g^{\theta\theta} R_{\theta\theta} + g^{\phi\phi} R_{\phi\phi} $, si trova lo scalare di curvatura di una 2-sfera:
\begin{equation}
	R = - \frac{2}{r^2}
\end{equation}
Ciò conferma il teorema di Gauss-Bonnett, che lega la curvatura $ K $ allo scalare di curvatura: per una 2-sfera $ K = \frac{1}{r^2} $, dunque $ K = - \frac{R}{2} $ (affermato dal teorema).

\section{Equazioni di Einstein}

Una particella isolata in uno spaziotempo curvo si muove secondo l'equazione geodesica. In presenza di un campo di materia, invece, la faccenda si complica.\\
Un campo di materia, nel quale si hanno flussi di materia ed energia, è descritto dall'energy-momentum tensor $ T_{ij} $, definito in modo da soddisfarre la conservazione dell'energia $ \nabla_i T^i_{\,\,\,j} = 0 $.

\begin{example}
	Per un fluido ideale isotropico a riposo di densità $ \rho $ e pressione $ p $, l'energy-momentum tensor è $ T_{ij} = p g_{ij} + (p + \rho)u_i u_j $, con $ u^i = \frac{dx^i}{d\tau} = \gamma (c, \ve{v}) $.
\end{example}

Il legame tra il campo di materia e la curvatura dello spaziotempo è dato dall'\textit{equazione di campo di Einstein}:
\begin{equation}
	R_{ij} - \frac{1}{2} g_{ij} R = - \frac{8\pi G}{c^4} T_{ij}
	\label{eq:5.34}
\end{equation}
Contraendo con $ g^{ij} $, dato che $ g^{ij}g_{ij} = \delta^i_i = 4 $, definendo la traccia dell'energy-momentum tensor $ T \equiv T^i_i $, si trova che $ R = \frac{8\pi G}{c^4} T $, dunque una forma equivalente dell'equazione di campo è:
\begin{equation}
	R_{ij} = - \frac{8\pi G}{c^4} \left( T_{ij} - \frac{T}{2} g_{ij} \right)
	\label{eq:5.35}
\end{equation}
Nello spazio vuoto e su scale piccole (spazio isotropo), trascurando la dark energy, l'equazione di campo diventa $ R_{ij} = 0 $.

\subsection{Azione di Einstein-Hilbert}

È possibile costruire un'azione da cui derivi l'equazione di campo: essa, essendo uno scalare, è invariante per trasformazioni arbitrarie di coordinate, dunque le equazioni tensoriali da esso derivate risultano invarianti per tali trasformazioni.\\
Per determinare un'azione appropriata, è necessario uno scalare dipendente dalla metrica: lo scalare di Ricci. Ricordando che la 4-form di volume è $ \sqg \,d^4x $, si ricava l'\textit{azione di Einstein-Hilbert} nello spazio vuoto:
\begin{equation}
	S_{\text{EH}} = - \frac{c^4}{16 \pi G} \int *R = - \frac{c^4}{16\pi G} \int R \sqg \,d^4x
	\label{eq:5.36}
\end{equation}

\begin{proposition}
	L'equazione di campo di Einstein deriva dall'azione di Einstein-Hilbert.
\end{proposition}
\begin{proof}
	In assenza di campi di materia, data una variazione della metrica $ \delta g_{ij} $ che si annulla all'infinito, è possibile scrivere la variazione al prim'ordine dell'azione come:
	\begin{equation*}
		\delta S_{\text{EH}} = - \frac{c^4}{16\pi G} \int \delta (R\sqg) \,d^4x = - \frac{c^4}{16\pi G} \int \left( \sqg \frac{\delta R}{\delta g^{ij}} + R \frac{\delta \sqg}{\delta g^{ij}} \right) g^{ij} d^4x
	\end{equation*}
	Per il primo termine, usando $ \delta R^i_{klm} = \nabla_l (\delta \Gamma^i_{mk}) - \nabla_m (\delta \Gamma^i_{lk}) $:
	\begin{equation*}
		\delta R = R_{ij} \delta g^{ij} + g^{ij} \delta R_{ij} = R_{ij} \delta g^{ij} + g^{ij} \delta R^k_{ikj} = R_{ij} \delta g^{ij} + g^{ij} \left[ \nabla_k (\delta \Gamma^k_{ji}) - \nabla_j (\delta \Gamma^k_{ki}) \right] = R_{ij} \delta g^{ij}
	\end{equation*}
	dove la nullità dell'ultimo termine è stata dimostrata da Gibbons, Hawking e York nel 1970. Per quando riguarda il secondo termine, si ricordi la formula di Jacobi $ \frac{\delta \det \tens{A}}{\delta A_{ij}} = (A^{-1})_{ij} \det \tens{A} $:
	\begin{equation*}
		\delta \sqg = \frac{1}{2\sqg} \delta g = \frac{1}{2\sqg} g g^{ij} \delta g_{ij} = - \frac{1}{2} \sqg g_{ij} \delta g^{ij}
	\end{equation*}
	dove nell'ultima uguaglianza si è usato $ g^{ij} \delta g_{ij} = - g_{ij} \delta g^{ij} $, che è una proprietà generale:
	\begin{equation*}
		k^{-1} \frac{\delta k}{\delta p} + \frac{\delta k^{-1}}{\delta p} k = \frac{\delta (k^-1 k)}{\delta p} = \frac{\delta 1}{\delta p} = 0 \quad\Longrightarrow\quad \delta g^{ij} = - g^{il} (\delta g_{lm}) g^{mj}
	\end{equation*}
	Di conseguenza:
	\begin{equation*}
		\delta S_{\text{EH}} = - \frac{c^4}{16\pi G} \int \left( R_{ij} - \frac{1}{2} g_{ij} R \right) \delta g^{ij} \sqg \,d^4x
	\end{equation*}
	dunque:
	\begin{equation*}
		\delta S_{\text{EH}} = 0 \quad\Longleftrightarrow\quad R_{ij} - \frac{1}{2} g_{ij} R = 0
	\end{equation*}
	In presenza di un campo di materia, si aggiunge un termine del tipo:
	\begin{equation*}
		\delta S_m = - \frac{1}{2} \int T_{ij} \delta g^{ij} \sqg \,d^4x \quad\Longleftrightarrow\quad T_{ij} = - 2 \frac{1}{\sqg} \frac{\delta S_m}{\delta g^{ij}}
	\end{equation*}
	quindi si trova l'equazione di campo di Einstein:
	\begin{equation*}
		- \frac{c^4}{16\pi G} \left( R_{ij} - \frac{1}{2} g_{ij} R \right) - \frac{1}{2} T_{ij} = 0
	\end{equation*}
\end{proof}










