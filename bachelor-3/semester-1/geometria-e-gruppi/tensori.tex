\selectlanguage{italian}

\section{Sistemi di coordinate e trasformazioni}


Le leggi fisiche sono equazioni che coinvolgono quantità invarianti rispetto a cambiamenti di sistemi di riferimento (RF): queste quantità sono espresse tramite tensori, oggetti che ubbidiscono a determinate leggi di trasformazione nel passaggio da un RF a un altro. \\ 
I tensori sono generalmente delle funzioni spaziali, ovvero dipendono da set ordinati di coordinate spaziali (quantità misurabili sperimentalmente, dunque fisiche, non solo matematiche). Le coordinate spaziali designano un punto nello spaziotempo che, di per sé, è assoluto: utilizzando diversi sistemi di riferimento, però, esso sarà descritto da set di coordinate differenti, sebbene la sua esistenza sia slegata da essi.

\begin{definition}
	Dato uno spazio o una sua regione costituita da punti, descritti da $ n $-ple ordinate di variabili continue $ \{x^k\}_{k=1,\dots,n}\subset\R, n \in \N $, si definisce \textit{trasformazione ammissibile} un insieme di funzioni $ y^k = y^k (x^1, \dots, x^n), k = 1, \dots, n $ che soddisfino le seguenti condizioni:
	\begin{enumerate}
		\item $ y^k (x^1, \dots, x^n) $ è monodroma (single-valued) e continuamente derivabile $ \forall k = 1, \dots, n $;
		\item $ \det \left[ \frac{\pa y^k}{\pa x^j} \right] \neq 0 $.
	\end{enumerate}
\end{definition}

La condizione sulla Jacobiana è necessaria per garantire l'invertibilità della trasformazione. 

\begin{definition}
	Si definisce \textit{tensore} di rango $ r+s $ un oggetto con $ r+s $ componenti dipendenti dalle coordinate $ x^k $:
	\begin{equation}
		T^{\,i_1 \dots i_r}_{j_1 \dots j_s} = T^{\,i_1 \dots i_r}_{j_1 \dots j_s} (x^1, \dots, x^n)
		\label{eq:1}
	\end{equation}
	dove ciascun indice varia in $ \{1, \dots, n\} $.
\end{definition}
Si adottano inoltre le seguenti convenzioni sugli indici:
\begin{enumerate}
	\item Einstein's summation convention: $ A^k B_k \equiv \sum_{k = 1}^{n} A^k B_k $;
	\item range convention: ogni indice libero in un'equazione varia indipendentemente in $ \{1, \dots, n\} $, quindi se in un'equazione sono presenti $ N $ indici liberi, essa rappresenta $ n^N $ equazioni distinte;
	\item in $ \frac{\pa y^k}{\pa x^j} $ l'indice $ j $ è considerato un petice.
\end{enumerate}

\begin{definition}
	Data una trasformazione ammissibile delle coordinate ed un tensore, si definisce la \textit{trasformazione indotta} delle componenti del tensore:
	\begin{equation}
		\tilde{T}^{\,i_1 \dots i_r}_{j_1 \dots j_s} (y^1, \dots, y^n) = \tilde{T}^{\,i_1 \dots i_r}_{j_1 \dots j_s} \left[ T^{1 \dots 1}_{1 \dots 1} (x^1, \dots, x^n), \dots, T^{\,n \dots n}_{n \dots n} (x^1, \dots, x^n) \right]
		\label{eq:2}
	\end{equation}
\end{definition}

È possibile dimostrare i seguenti asserti:

\begin{theorem}
	Esiste un \textnormal{isomorfismo} tra una trasformazione di coordinate e la sua trasformazione indotta.
\end{theorem}

\begin{proposition}
	Ogni trasformazione indotta è un isomorfismo.
\end{proposition}

In generale, i tensori sono definiti dalle loro leggi di trasformazione indotte, che possono essere lineari o non-lineari (ad esempio nel caso di RF curvilinei), ma sempre omogenee. \\
Nel seguito verranno considerati solo tensori che sono funzioni $ \R^n \longrightarrow \R^{n^{r+s}} $.


\subsection{Tensori assoluti}


\begin{definition}
	Si definisce \textit{scalare} un tensore $ \alpha (x^1, \dots, x^n) $ di rango $ 0 $ con legge di trasformazione:
	\begin{equation}
		\tilde{\alpha} (y^1, \dots, y^n) = \alpha (x^1, \dots, x^n)
		\label{eq:3}
	\end{equation}
\end{definition}

Il valore di un campo scalare in un punto dello spazio è (giustamente) indipendente dal RF.

\begin{definition}
	Si definisce \textit{vettore contravariante} un tensore $ a^k (x^1, \dots, x^n) $ di rango $ 1 $ con legge di trasformazione:
	\begin{equation}
		\tilde{a}^k (y^1, \dots, y^n) = \frac{\pa y^k}{\pa x^j} a^j (x^1, \dots, x^n)
		\label{eq:4}
	\end{equation}
\end{definition}

\begin{definition}
	Si definisce \textit{vettore covariante} un tensore $ a_k (x^1, \dots, x^n) $ di rango $ 1 $ con legge di trasformazione:
	\begin{equation}
		\tilde{a}_k (y^1, \dots, y^n) = \frac{\pa x^j}{\pa y^k} a_j (x^1, \dots, x^n)
		\label{eq:5}
	\end{equation}
\end{definition}

Per ricavare le due leggi di trasformazione, va ricordato l'isomorfismo canonico tra uno spazio vettoriale $ V $ ed il suo duale $ V^* $: grazie ad esso, un generico vettore $ \mathbf{v} $ può essere espresso sia sulla base canonica $ \{\mathbf{b}_i\}_{i=1,\dots,n} $ di $ V $ come $ \mathbf{v} = v^i \mathbf{b}_i $, sia sulla base duale $ \{\mathbf{b}^i\}_{i=1,\dots,n} $ di $ V^* $ come $ \mathbf{v} = v_i \mathbf{b}^i $. \\ 
%
Considerando uno spazio euclideo con RF $ (x^1, \dots, x^n) $, la base canonica dello spazio in un punto $ \mathbf{r} $ è data da $ \mathbf{e}_i = \frac{\pa \mathbf{r}}{\pa x^i} $, dunque, dato che $ \mathbf{a} = a^j \mathbf{e}_j = \tilde{a}^k \tilde{\mathbf{e}}_k $:

\begin{equation}
	\mathbf{e}_j = \frac{\pa \mathbf{r}}{\pa x^j} = \frac{\pa y^k}{\pa x^j} \frac{\pa \mathbf{r}}{\pa y^k} = \frac{\pa y^k}{\pa x^j} \tilde{\mathbf{e}}_k \quad \Longrightarrow \quad \tilde{a}^k = \frac{\pa y^k}{\pa x^j} a^j
	\label{eq:6}
\end{equation}

Se invece si esprime $ \mathbf{a} $ sulla base duale $ \mathbf{e}^i = \nabla x^i $, dato che $ \mathbf{a} = a_j \mathbf{e}^j = \tilde{a}_k \tilde{\mathbf{e}}^k $:

\begin{equation}
	\mathbf{e}^j = \frac{\pa x^j}{\pa \mathbf{r}} = \frac{\pa x^j}{\pa y^k} \frac{\pa y^k} {\pa \mathbf{r}} = \frac{\pa x^j}{\pa y^k} \tilde{\mathbf{e}}^k \quad \Longrightarrow \quad \tilde{a}_k = \frac{\pa x^j}{\pa y^k} a_j
	\label{eq:7}
\end{equation}
Si può notare che la legge di trasformazione contravariante è quella del differenziale, mentre quela covariante è quella del gradiente. \\
%
È possibile generalizzare queste leggi di trasformazioni a tensori qualsiasi.

\begin{definition}
	Dato un tensore $ T^{\,i_1 \dots i_r}_{j_1 \dots j_s} (x^1, \dots, x^n) $ di rango $ r+s $, la sua legge di trasformazione è:
	\begin{equation}
		\tilde{T}^{\,i_1 \dots i_r}_{j_1 \dots j_s} (y^1, \dots, y^n) = \frac{\pa y^{i_1}}{\pa x^{k_1}} \dots \frac{\pa y^{i_r}}{\pa x^{k_r}} \frac{\pa x^{l_1}}{\pa y^{j_1}} \dots \frac{\pa x^{l_s}}{\pa y^{l_s}} T^{k_1 \dots k_r}_{l_1 \dots l_s} (x^1, \dots, x^n)
		\label{eq:8}
	\end{equation}
\end{definition}

\begin{example}
	Un esempio interessante è il tensore di Kronecker, definito a partire dalla delta di Kronecker: $ \tilde{\delta}^i_j = \frac{\pa y^i}{\pa x^k} \frac{\pa x^l}{\pa y^j} \delta^k_l = \frac{\pa y^i}{\pa y^j} = \delta^i_l $, dunque è un tensore isotropico (indipendente dal RF).
\end{example}

\begin{example}
	Un altro esempio di tensore è il campo elettrico, definito come il gradiente di un campo scalare (anche esso tensore di rango $ 0 $): $ \mathbf{E} = -\nabla\phi $, ovvero $ E_i = - \frac{\pa \phi}{\pa x^i} $.
\end{example}

\begin{example}
	Un tensore di rango $ 2 $ può essere ottenuto, ad esempio, come prodotto diretto tra due tensori di rango $ 1 $: ad esempio $ \tens{T} = \mathbf{v} \otimes \mathbf{u} $, che può essere espresso in tre modi:
	\begin{enumerate}
		\item contravariante: $ \tens{T} = v^j \mathbf{e}_j \otimes u^k \mathbf{e}_k $, quindi $ T^{jk} = v^j u^k $;
		\item covariante: $ \tens{T} = v_j \mathbf{e}^j \otimes u_k \mathbf{e}^k $, quindi $ T_{jk} = v_j u_k $;
		\item mixed: $ \tens{T} = v_j \mathbf{e}^j \otimes u^k \mathbf{e}_k $, quindi $ T_j^k = v_j u^k $.
	\end{enumerate}
\end{example}

\begin{example}
	Anche il gradiente di un vettore è un tensore di rango $ 2 $, poiché può essere visto come prodotto diretto: $ \tens{T} = \nabla\mathbf{v} $, ovvero $ T^{\,i}_j = \frac{\pa v^i}{\pa x^j} $.
\end{example}


\subsection{Pseudotensori}


Un particolare tipo di trasformazioni di coordinate sono le rotazioni in $ \R^3 $.

\begin{definition}
	Si definisce \textit{rotazione} un'isometria di $ \R^3 $ rappresentata da una matrice ortogonale $ \tens{R} \in \Ot \defeq \{\tens{R} \in \R^{3\times3} : \tens{R}^{\intercal}\tens{R} = \tens{R}\tens{R}^{\intercal} = \tens{I}_3 \} $.
\end{definition}

È necessario fare una distinzione tra due classi di rotazioni.

\begin{definition}
	Si definisce \textit{rotazione propria} una rotazione rappresentata da una matrice ortogonale $ \tens{R} \in \SOt \defeq \{\tens{R} \in \Ot : \det\tens{R} = +1 \} $.
\end{definition}

\begin{example}
	Considerando un tensore di rango $ 2 $ definito come $ \tens{T} = \ve{v} \otimes \ve{u} $ ed una rotazione propria $ \tens{R} $: $ \tens{T} = T^i_j \ve{e}_i \otimes \ve{e}^j = \tilde{T}^i_j \tilde{\ve{e}}_i \otimes \tilde{\ve{e}}^j $, ma $ \ve{e}^j = \frac{\pa x^j}{\pa y^k} \tilde{\ve{e}}^k $, dunque $ \tilde{T}^i_j = \frac{\pa y^i}{\pa x^k} \frac{\pa x^l}{\pa y^j} T^k_l $; dato che il Jacobiano della trasformazione è proprio la matrice di rotazione $ \tens{R} $, si vede che le componenti contravarianti del tensore trasformano secondo $ \tens{R} $, mentre le componenti covarianti secondo $ \tens{R}^{-1} = \tens{R}^{\intercal} $.
\end{example}

\begin{definition}
	Si definisce \textit{rotazione impropria} una rotazione rappresentata da una matrice ortogonale $ \tens{R} \in \Ot : \det\tens{R} = -1 $.
\end{definition}

Le rotazioni improprie sono rotazioni con l'inversione di un asse, dunque non conservano l'orientazione (handedness) dello spazio.

\begin{definition}
	Si definisce \textit{pseudotensore} di rango $ r+s $ un oggetto tensoriale di rango $ r+s $ che per effetto di una rotazione trasforma secondo:
	\begin{equation}
		\tilde{T}^{\,i_1 \dots i_r}_{j_1 \dots j_s} (y^1, \dots, y^n) = \left(\det\tens{R}\right)\, \frac{\pa y^{i_1}}{\pa x^{k_1}} \dots \frac{\pa y^{i_r}}{\pa x^{k_r}} \frac{\pa x^{l_1}}{\pa y^{j_1}} \dots \frac{\pa x^{l_s}}{\pa y^{l_s}} T^{k_1 \dots k_r}_{l_1 \dots l_s} (x^1, \dots, x^n)
		\label{eq:9}
	\end{equation}
\end{definition}

Uno pseudotensore acquista un segno $ - $ sotto rotazioni improprie. \\
Lo pseduotensore per eccellenza è il tensore di Levi-Civita.

\begin{definition}\label{def-lc}
	Si definisce il simbolo di Levi-Civita nel caso tridimensionale euclideo:
	\begin{equation}
		\epsilon_{ijk} = \epsilon^{ijk} = 
		\begin{cases}
			+1 &\quad (i,j,k) = \sigma_p(1,2,3) \\
			-1 &\quad (i,j,k) = \sigma_d(1,2,3) \\
			0  &\quad \text{altrimenti}
		\end{cases}
		\label{eq:10}
	\end{equation}
	dove $ \sigma_{p,d} $ sono delle permutazioni pari/dispari.
\end{definition}

È possibile dimostrare le seguenti proposizioni.

\begin{proposition}\label{levi-civ-det}
	Data $ \tens{A} \in \R^{3\times 3} $, si ha $ \left(\det\tens{A}\right)\, \epsilon_{lmn} = A_{li} A_{mj} A_{nk} \epsilon^{ijk} $.
\end{proposition}

\begin{proposition}\label{epsilon-delta}
	$ \epsilon_{ijk}\epsilon_{klm} = \delta_{il}\delta_{jm} - \delta_{im}\delta_{jl}.$
\end{proposition}

Dalla prop. \ref{levi-civ-det} è facile ricavare la legge di trasformazione del simbolo di Levi-Civita:
\begin{equation}
	\tilde{\epsilon}_{ijk} = \frac{\pa x^l}{\pa y^i} \frac{\pa x^m}{\pa y^j} \frac{\pa x^n}{\pa y^k} \epsilon_{lmn} = \left(\det\tens{R}^{-1}\right) \epsilon_{ijk} = \left(\det\tens{R}\right) \epsilon_{ijk}
	\label{eq:11}
\end{equation}
Quindi sotto rotazioni improprie il simbolo di Levi-Civita cambia di segno.

\begin{example}
	Dato che $ B^i = \left(\nabla\times\ve{A}\right)^i = \epsilon^{ijk}\frac{\pa A_k}{\pa x^j}$, si vede che, per effetto di rotazioni improprie, il campo magnetico cambia di segno: ciò è legato al cambiamento di handedness dello spazio.
\end{example}


\section{Tensore Metrico}


Mentre nel caso euclideo la base canonica $ \ve{e}_j = \frac{\pa \ve{r}}{\pa x^j} $ ($ \ve{e}_j $ tangente alla $ j $-esima linea coordinata) coincide con la base duale $ \ve{e}^j = \nabla x^j $ ($ \ve{e}^j $ ortogonale alla $ j $-esima linea coordinata), ciò non è valido in generale.\\
In RF generici, per passare da una base alla sua duale è necessario introdurre un oggetto tensoriale detto tensore metrico.

\begin{definition}
	Data una base $ \{\ve{e}_j\}_{j=1,\dots,n} $, si definisce il \textit{tensore metrico} in tale base come $ g_{ij} \defeq \ve{e}_i \cdot \ve{e}_j \, \forall i,j = 1,\dots,n$.
\end{definition}

Ovviamente è possibile specificare le componenti controvarianti $ g^{ij} $, al posto di quelle covarianti. \\
Nel caso in cui i vettori base sono mutualmente ortogonali, il tensore metrico sarà rappresentato da una matrice diagonale: come negli spazi vettoriali si può sempre ottenere una base ortonormale da una base non-ortonormale (algoritmo di Gram-Schmidt), così si può diagonalizzare anche il tensore metrico.

\begin{proposition}
	$ ds^2 = g_{ij} dx^i dx^j $.
\end{proposition}
\begin{proof}
	$ ds^2 \equiv d\ve{r} \cdot d\ve{r} = dx^i \ve{e}_i \cdot dx^j \ve{e}_j = g_{ij} dx^i dx^j $.
\end{proof}

\begin{proposition}
	Nel caso tridimensionale $ dV = \sqg\, dx^1 dx^2 dx^3 $, con $ \tens{g} \defeq \det g_{ij} $.
\end{proposition}
\begin{proof}
	Considerando una base già ortogonalizzata tale che $ g_{ij} = h_i^2 \delta_{ij} $, dove $ h_i $ è detto fattore scala, abbiamo $ ds^2 = h_i^2 (dx^i)^2 $ (concorde col fatto che $ d\ve{r} = \frac{\pa \ve{r}}{\pa x^j} dx^j = h_j dx^j \hat{\ve{e}}_j $ per definizione di $ \hat{\ve{e}}_j $); per definizione $ dV = \abs{dx^1\ve{e}_1 \cdot (dx^2\ve{e}_2 \times dx^3\ve{e}_3)} = h_1 h_2 h_3 dx^1 dx^2 dx^3 \hat{\ve{e}}_1 \cdot (\hat{\ve{e}}_2 \times \hat{\ve{e}}_3) $, ma la base è ortogonale e per il tensore metrico si ha $ \det\tens{g} = h_1^2 h_2^2 h_3^2 $, dunque $ dV = \sqg\, dx^1 dx^2 dx^3 $.
\end{proof}
Questa proposizione è generalizzabile al caso $ n $-dimensionale non-ortogonale (per la definizione del wedge product, vedere Sec. \ref{sec-wp}).
\begin{proposition}
	In una varietà $ n $-dimensionale con tensore metrico $ g_{ij} $:
	\begin{equation}
		dV = \sqg \displaystyle\bigwedge_{k = 1}^n dx^k
	\end{equation}
\end{proposition}

Il tensore metrico è dunque fondamentale per il calcolo di distanze, aree, volumi, etc., ma ha anche un'altra importante funzione: passare dalle componenti covarianti a quelle contravarianti e viceversa.\\
Il prodotto scalare può essere espresso in vari modi: $ \ve{a}\cdot\ve{b} = a_i b^i = a^i b_i = g_{ij} a^i b^j = g^{ij} a_i b_j $, dunque si notano due importanti identità:

\begin{equation}
	\begin{split}
		g_{ij} a^i &= a_j \\
		g^{ij} a_i &= a^j
	\end{split}
	\label{eq:12}
\end{equation}

È dunque possibile passare da una base al suo duale mediante contrazione col tensore metrico.

\begin{proposition}
	$ \left[g^{ij}\right] = \left[g_{ij}\right]^{-1} $.
\end{proposition}
\begin{proof}
	$ a^i = g^{ij} a_j = g^{ij} g_{jk} a^k $, dunque $ g^{ij} g_{jk} = \delta^i_k $, ovvero $ \left[g^{ij}\right]\left[g_{ij}\right] = \tens{I}_n $.
\end{proof}

Da questa dimostrazione si nota che $ g^i_j = \delta^i_j $.\\
Data la relazione di passaggio tra componenti covarianti e contravarianti, è possibile enunciare in maniera generica la legge di trasformazione di un tensore.

\begin{definition}
	Dato un tensore $ T^{\,i_1 \dots i_r}_{j_1 \dots j_s} (x^1, \dots, x^n) $ di rango $ r+s $, la sua legge di trasformazione è:
	\begin{equation}
		\tilde{T}^{\,i_1 \dots i_r}_{j_1 \dots j_s} (y^1, \dots, y^n) = \det\left[\frac{\pa x}{\pa y}\right]^w \frac{\pa y^{i_1}}{\pa x^{k_1}} \dots \frac{\pa y^{i_r}}{\pa x^{k_r}} \frac{\pa x^{l_1}}{\pa y^{j_1}} \dots \frac{\pa x^{l_s}}{\pa y^{l_s}} T^{k_1 \dots k_r}_{l_1 \dots l_s} (x^1, \dots, x^n)
		\label{eq:13}
	\end{equation}
	dove in base al valore di $ w $ si parla di:
	\begin{itemize}
		\item $ w = 0 $: tensore assoluto;
		\item $ w = -1 $: tensore relativo (o pseudotensore);
		\item $ w = +1 $: densità tensoriale.
	\end{itemize}
\end{definition}

Si può notare che $ \det\left[\frac{\pa x}{\pa y}\right] = \frac{1}{\det\tens{J}} $, dove $ \tens{J} $ è il Jacobiano della trasformazione.

\begin{theorem}[del quoziente]
	Dati due tensori $ \tens{A} $ e $ \tens{C} $, se per un oggetto $ \tens{B} $ vale l'equazione tensoriale $ \tens{A}\tens{B} = \tens{C} $, allora $ \tens{B} $ è un tensore.
\end{theorem}

\subsection{Sistemi di riferimento mobili}

È possibile considerare SR in cui i vettori base dipendono dal punto in cui sono applicati: dato un punto $ \ve{r}(\ve{x}) $, lo spostamento infinitesimo da esso si scrive come $ d\ve{r}(\ve{x}) = \ve{e}_j(\ve{x}) dx^j $ e i vettori base sono $ \ve{e}_j (\ve{x}) = \frac{\pa \ve{r}}{\pa x^j}(\ve{x}) $.\\
Nel passaggio da un SR all'altro, essi trasformano come vettori covarianti:

\begin{equation}
	\tilde{\ve{e}}_j (\ve{y}) = \frac{\pa \ve{r}}{\pa y^j} = \frac{\pa x^k}{\pa y^j} \frac{\pa \ve{r}}{\pa x^k} = \frac{\pa x^k}{\pa y^j} \ve{e}_k (\ve{x})
	\label{eq:1.14}
\end{equation}

Questi sono anche vettori nell'embedding manifold (una varietà a dimensione maggiore, tendenzialmente euclidea, che $ \virgolette{ricopre} $ la varietà considerata: ad esempio, $ \mathbb{S}^2 $ è embedded in $ \R^3 $), il quale avrà un tensore metrico $ \eta_{ij} $.\\
Questi vettori possono essere sia non-normalizzati che non-simmetrici; inoltre, definiscono un tensore metrico sull'embedded manifold:

\begin{equation}
	g_{ij} (\ve{x}) \defeq \ve{e}_i (\ve{x}) \cdot \ve{e}_j (\ve{x})
	\label{eq:1.15}
\end{equation}

dove il prodotto scalare è quello definito sull'embedding manifold da $ \eta_{ij} $. Per varietà reali, il tensore metrico è reale e simmetrico.\\
Sotto trasformazione da un moving frame ad un altro, il tensore metrico trasforma come un tensore doppiamente covariante:

\begin{equation}
	\tilde{g}_{ij} (\ve{y}) = \tilde{\ve{e}}_i (\ve{y}) \cdot \tilde{\ve{e}}_j (\ve{y}) = \frac{\pa x^k}{\pa y^i} \ve{e}_k (\ve{x}) \cdot \frac{\pa x^l}{\pa y^j} \ve{e}_j (\ve{x}) = \frac{\pa x^k}{\pa y^i} \frac{\pa x^l}{\pa y^j} g_{ij} (\ve{x})
	\label{eq:1.16}
\end{equation}

Nel caso di sistemi ortogonali, si possono calcolare i fattori scala $ h_i \defeq \norm{\ve{e}_i(\ve{x})} $, così da ortonormalizzare il sistema: $ \hat{\ve{e}}_i = \frac{\ve{e}_i}{h_i} $. In questo caso $ g_{ij} = h_i^2 \delta_{ij} $.

\begin{proposition}
	$ \hat{\ve{e}}_i = h_i \ve{e}^i $.
\end{proposition}
\begin{proof}
	Si noti che non è sottintesa la somma su $ i $, in questo caso:
	\begin{equation*}
		\hat{\ve{e}}_i = \frac{1}{h_i} \ve{e}_i = \frac{1}{h_i} g_{ij} \ve{e}^j = \frac{1}{h_i} h_i^2 \delta_{ij} \ve{e}^j = h_i \ve{e}^i
	\end{equation*}
\end{proof}










