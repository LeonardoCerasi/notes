\selectlanguage{italian}

\section{Gruppo di Lorentz}

Considerando due sistema di riferimento $ \mathfrak{S} $ e $ \mathfrak{S}' $, dove quest'ultimo di muove con velocità $ \ve{v} $ costante rispetto al primo, le trasformazioni non-relativistiche tra i due sistemi sono le \textit{trasformazioni di Galileo}:
\begin{equation}
	\begin{cases}
		t' = t \\
		\ve{x}' = \ve{x} - \ve{v} t
	\end{cases}
	\label{eq:10.1}
\end{equation}
Tali trasformazioni preservano la distanza tra punti nello spazio:
\begin{equation}
	\left( \ve{x}'_1 - \ve{x}'_2 \right)^2 = \left( \ve{x}_1 - \ve{x}_2 \right)^2
	\label{eq:10.2}
\end{equation}
La relativa trasformazione per la velocità è:
\begin{equation}
	\ve{u}' = \ve{u} - \ve{v}
	\label{eq:10.3}
\end{equation}
Ciò però è in contraddizione col fatto che relativisticamente la velocità della luce è costante in tutti i sistemi di riferimento. Imponendo questo constraint, Einstein ricavò le \textit{trasformazioni di Lorentz}:
\begin{equation}
	\begin{cases}
		t' = \gamma \left( t - \frac{1}{c^2} \ve{x} \cdot \ve{v} \right) \\
		\ve{x}'_{\parallel} = \gamma \left( \ve{x}_{\parallel} - \ve{v} t \right) \\
		\ve{x}'_{\perp} = \ve{x}_{\perp}
	\end{cases}
	\label{eq:10.4}
\end{equation}
dove le direzioni parallela e perpendicolare sono rispetto a $ \ve{v} $ e $ \gamma \defeq \left( 1 - \ve{v}^2 / c^2 \right)^{-1/2} \in [1,\infty) $.

\begin{proposition}
	Le trasformazioni di Lorentz sono relativistiche.
\end{proposition}
\begin{proof}
	Usando la chain rule:
	\begin{equation*}
		\begin{split}
			\ve{u}'^2
			&= \ve{u}'^2_{\parallel} + \ve{u}'^2_{\perp} = \left( \frac{d\ve{x}'_{\parallel}}{dt} \frac{dt}{dt'} \right)^2 + \left( \frac{d\ve{x}'_{\perp}}{dt} \frac{dt}{dt'} \right)^2 = \left( \frac{\gamma \left( \ve{u}_{\parallel} - \ve{v} \right)}{\gamma \left( 1 - \ve{u} \cdot \ve{v} / c^2 \right)} \right)^2 + \left( \frac{\ve{u}_{\perp}}{\gamma \left( 1 - \ve{u} \cdot \ve{v} / c^2 \right)} \right)^2 \\
			&= c^2 - c^2 \left( 1 - \frac{\ve{v}^2}{c^2} \right) \left( 1 - \frac{\ve{u}^2}{c^2} \right) \left( 1 - \frac{\ve{u} \cdot \ve{v}}{c^2} \right)^{-1}
		\end{split}
	\end{equation*}
	Si ha dunque che $ \norm{\ve{u}'} \le c $ e $ \norm{\ve{u}'} = c \,\Leftrightarrow\, \norm{\ve{u}} = c $.
\end{proof}

Per le trasformazioni di Lorentz l'invariante non è più la distanza cartesiana tra punti nello spazio, ma la separazione tra due eventi nello spaziotempo:
\begin{equation}
	c^2 \left( t'_1 - t'_2 \right)^2 - \left( \ve{x}'_1 - \ve{x}'_2 \right)^2 = c^2 \left( t_1 - t_2 \right)^2 - \left( \ve{x}_1 - \ve{x}_2 \right)^2
	\label{eq:10.5}
\end{equation}

\subsection{4-vectors}

Si noti la metrica con segnatura $ \left( +,-,-,- \right) $. Si definiscono:
\begin{equation}
	ds^2 \defeq dx^2 + dy^2 + dz^2
	\label{eq:10.6}
\end{equation}
\begin{equation}
	c^2 d\tau^2 \defeq c^2 dt^2 - ds^2
	\label{eq:10.7}
\end{equation}
Trattando $ ct $ come coordinata al posto di $ t $, si definiscono i \textit{4-vectors}:
\begin{equation}
	x_{\mu} \defeq \left( x_0, \ve{x} \right) = \left( ct, \ve{x} \right)
	\qquad
	x^{\mu} \defeq \left( x_0, -\ve{x} \right) = \left( ct, -\ve{x} \right)
	\label{eq:10.8}
\end{equation}

\begin{definition}
	Si definisce il \textit{prodotto scalare} di due 4-vectors $ x_{\mu},y_{\mu} $ come:
	\begin{equation}
		x \cdot y \defeq x_{\mu} y^{\mu} = c^2 t_x t_y - \ve{x} \cdot \ve{y}
		\label{eq:10.9}
	\end{equation}
\end{definition}

\begin{proposition}
	$ c^2 \tau^2 = x_{\mu} x^{\mu} $.
\end{proposition}
\begin{proof}
	$ x_{\mu} x^{\mu} = c^2 t^2 - \ve{x} \cdot \ve{x} = c^2 t^2 - \ve{x}^2 = c^2 \tau^2 $.
\end{proof}

Dalla necessità di invarianza di Lorentz, si definisce la \textit{4-velocità} come:
\begin{equation}
	u_{\mu} \defeq \frac{dx_{\mu}}{d\tau}
	\label{eq:10.10}
\end{equation}
Dato che $ \frac{d\tau}{dt} = \frac{1}{\gamma} $, dalle trasformazioni di Lorentz si trova:
\begin{equation}
	u_{\mu} = \gamma \left( c, \ve{u} \right)
	\label{eq:10.11}
\end{equation}
Analogamente, si definisce il \textit{4-momentum} come:
\begin{equation}
	p_{\mu} \defeq m u_{\mu}
	\label{eq:10.12}
\end{equation}
Questo si scompone in:
\begin{equation}
	p_{\mu} = \left( \gamma m c, \gamma m \ve{u} \right) = \left( \tfrac{1}{c} E, \ve{p} \right)
	\label{eq:10.13}
\end{equation}
Entrambi sono quadrivettori, dato che la norma quadratica è Lorentz-invariante:
\begin{equation*}
	u^2 = \gamma^2 \left( c^2 - \ve{u}^2 \right) = c^2
	\qquad
	p^2 = m^2 c^2
\end{equation*}
Dall'Eq. \ref{eq:10.13} si trova:
\begin{equation}
	E^2 = m^2 c^4 + \ve{p}^2 c^2
	\label{eq:10.14}
\end{equation}
Nel limite sub-relativistico $ c \rightarrow \infty $, ciò si riduce alla relazione classica per l'energia cinetica:
\begin{equation*}
	E = c \sqrt{m^2 c^2 + \ve{p}^2} = mc^2 \sqrt{1 + \frac{\ve{p}^2}{m^2 c^2}} \approx mc^2 \left( 1 + \frac{\ve{p}^2}{2m^2 c^2} \right) = mc^2 + \frac{\ve{p}^2}{2m}
\end{equation*}

\subsection{Gruppo \texorpdfstring{$ \SOn{3,1} $}{TEXT}}

Oltre alle rotazioni, bisogna anche caratterizzare i boost. Si consideri un boost lungo l'asse $ z $, e si noti che è possibile riparametrizzare il fattore di Lorentz: non più in funzione della velocità $ \ve{v} $, bensì di un parametro $ \zeta $. In particolare, $ \gamma \in [1,\infty) $, dunque si può scrivere come:
\begin{equation}
	\gamma = \cosh \zeta
	\label{eq:10.15}
\end{equation}
con $ \zeta \in \R^+_0 $. Dato che $ \cosh^2 \zeta - \sinh^2 \zeta = 1 $, si ricava:
\begin{equation}
	\tanh \zeta = \frac{v}{c}
	\qquad
	\sinh \zeta = \gamma \frac{v}{c}
	\label{eq:10.16}
\end{equation}
A partire da ciò, si può dimostrare che un boost lungo $ z $ è rappresentato da:
\begin{equation}
	\begin{cases}
		x'_1 = x_1 \\
		x'_2 = x_2 \\
		x'_3 = x_3 \cosh \zeta - x_0 \sinh \zeta \\
		x'_0 = -x_3 \sinh \zeta + x_0 \cosh \zeta
	\end{cases}
	\label{eq:10.17}
\end{equation}
Anche in questo caso si preserva la norma quadratica in Eq. \ref{eq:10.5}. I boost da soli non formano un gruppo, poiché componendo due boost lungo direzioni diverse si ottiene una rotazione.\\
Dato che sia rotazioni che boost lasciano invariato $ c^2 \tau^2 $, si può considerare il gruppo formato da tutte le rotazioni e tutti i boost.

\begin{definition}
	Si definisce \textit{gruppo di Lorentz} il gruppo di tutte le trasformazioni lineari dello spaziotempo di Minkowski che preservano $ c^2 \tau^2 $.
\end{definition}

\begin{definition}
	Si definisce \textit{gruppo di Lorentz proprio}, e si indica con $ \SOn{3,1} $, il sottogruppo del gruppo di Lorentz ottenuto escludendo le inversioni.
\end{definition}

Si escludono le inversioni poiché non sono continuamente connesse all'identità, dunque non descrivibili dalla teoria di Lie. Il generico elemento di $ \SOn{3,1} $ può essere rappresentato da una matrice $ \Lambda \in \R^{4\times4} $, così che $ x' = \Lambda x $ (vettori colonna). In forma matriciale, il prodotto scalare di due 4-vectors è $ x \cdot y = x^{\intercal} \eta y $, dove $ \eta = \diag \left( 1,-1,-1,-1 \right) $ è la metrica di Minkowski, dunque l'invarianza di Lorentz implica che:
\begin{equation}
	\Lambda^{\intercal} \eta \Lambda = \eta
	\label{eq:10.18}
\end{equation}
Questa condizione è detta di pseudo-ortogonalità e generalizza quella di ortogonalità di $ \On{4} $ al gruppo $ \On{3,1} $ (tre segni negativi ed uno positivo). Prendendo il determinante dell'Eq. \ref{eq:10.18} si trova $ \left( \det \Lambda \right)^2 = 1 $: scegliendo $ \det \Lambda = 1 $ si trova il gruppo $ \SOn{3,1} $, sottogruppo di $ \On{3,1} $.

\subsubsection{Isomorfismi}

Si ricordi che le tre matrici di Pauli generano il gruppo di matrici $ \C^{2\times2} $ con determinante unitario, ovvero $ \SUn{2} $: queste matrici sono parametrizzate da un versore $ \ve{n} $ ed un angolo $ \theta $. La stessa parametrizzazione si ha per $ \SOn{3} $, dunque si può considerare un omomorfismo $ \SUn{2} \rightarrow \SOn{3} $: questo non è biunivoco, poiché il suo kernel non è triviale. In particolare, $ \SOn{3} $ è periodico di $ 2\pi $, mentre $ \SUn{2} $ di $ 4\pi $; indicando con $ \tens{U}_{\ve{n}}(\theta) $ un generico elemento di $ \SUn{2} $, nella rappresentazione fondamentale (generata dalle matrici di Pauli, dato che $ [\frac{\sigma_a}{2}, \frac{\sigma_b}{2}] = i \epsilon^{abc} \frac{\sigma_c}{2} $) si ha:
\begin{equation}
	\tens{U}_{\ve{n}}(\theta) = \exp \left( -\frac{1}{2} i \sigma \cdot \ve{n} \theta \right) = \tens{I} \cos \frac{1}{2}\theta - i \sigma \cdot \ve{n} \sin \frac{1}{2}\theta
	\label{eq:10.19}
\end{equation}
dove $ \sigma \equiv \left( \sigma_1, \sigma_2, \sigma_3 \right) $. Si vede dunque che $ \tens{U}_{\ve{n}}(2\pi) = -\tens{I} $, mentre $ \tens{R}_{\ve{n}}(2\pi) = \tens{I} $: si ha in generale che due elementi distinti di $ \SUn{2} $ vengono mappati allo stesso elemento di $ \SOn{3} $.\\
Per ristabilire un mapping biunivoco, si consideri il vettore così definito:
\begin{equation*}
	X \defeq \ve{x} \cdot \sigma = x \sigma_1 + y \sigma_2 + z \sigma_3 =
	\begin{bmatrix}
		z & x - i y \\
		x + i y & -z
	\end{bmatrix}
\end{equation*}
Si vede che $ X $ è hermitiano e traceless, ed un generico elemento $ \tens{U} \in \SUn{2} $ agisce su di esso come $ X' = \tens{U}^{\dagger} X \tens{U} $; ricordando che $ \tens{U} $ è unitaria, si ha $ \det X' = \det X $ e, dato che $ \det X = -\ve{x}^2 $, $ \tens{U} $ preserva il modulo del vettore: ciò suggerisce che questa sia una rotazione $ \tens{R} \in \SOn{3} $. Per mostrarlo esplicitamente, si consideri $ \ve{n} = (0,0,1) $:
\begin{equation*}
	\tens{U}_3(\theta) = \tens{I} \cos \frac{1}{2} \theta - i \sigma_3 \sin \frac{1}{2} \theta =
	\begin{bmatrix}
		\cos \frac{1}{2} \theta - i \sin \frac{1}{2} \theta & 0 \\
		0 & \cos \frac{1}{2} \theta + i \sin \frac{1}{2} \theta
	\end{bmatrix}
	=
	\begin{bmatrix}
		e^{-i \frac{\phi}{2}} & 0 \\ 0 & e^{i \frac{\phi}{2}}
	\end{bmatrix}
\end{equation*}
Per determinare l'azione su $ X $, basta determinare l'azione sulle matrici di Pauli:
\begin{equation*}
	\begin{split}
	\tens{U}_3^{\dagger} \sigma_1 \tens{U}_3^{\dagger} &=
	\begin{bmatrix}
		e^{i \frac{\phi}{2}} & 0 \\ 0 & e^{-i \frac{\phi}{2}}
	\end{bmatrix}
	\begin{bmatrix}
		0 & 1 \\ 1 & 0
	\end{bmatrix}
	\begin{bmatrix}
		e^{-i \frac{\phi}{2}} & 0 \\ 0 & e^{i \frac{\phi}{2}}
	\end{bmatrix}
	=
	\begin{bmatrix}
		0 & e^{i\phi} \\ e^{-i\phi} & 0
	\end{bmatrix}
	= \sigma_1 \cos \phi - \sigma_2 \sin \phi \\
	\tens{U}_3^{\dagger} \sigma_2 \tens{U}_3^{\dagger} &=
	\begin{bmatrix}
		e^{i \frac{\phi}{2}} & 0 \\ 0 & e^{-i \frac{\phi}{2}}
	\end{bmatrix}
	\begin{bmatrix}
		0 & -i \\ i & 0
	\end{bmatrix}
	\begin{bmatrix}
		e^{-i \frac{\phi}{2}} & 0 \\ 0 & e^{i \frac{\phi}{2}}
	\end{bmatrix}
	=
	\begin{bmatrix}
		0 & -i e^{i\phi} \\ i e^{-i\phi} & 0
	\end{bmatrix}
	= \sigma_1 \sin \phi + \sigma_2 \cos \phi \\
	\tens{U}_3^{\dagger} \sigma_3 \tens{U}_3^{\dagger} &=
	\begin{bmatrix}
		e^{i \frac{\phi}{2}} & 0 \\ 0 & e^{-i \frac{\phi}{2}}
	\end{bmatrix}
	\begin{bmatrix}
		1 & 0 \\ 0 & -1
	\end{bmatrix}
	\begin{bmatrix}
		e^{-i \frac{\phi}{2}} & 0 \\ 0 & e^{i \frac{\phi}{2}}
	\end{bmatrix}
	=
	\begin{bmatrix}
		1 & 0 \\ 0 & -1
	\end{bmatrix}
	= \sigma_3
	\end{split}
\end{equation*}
Si ha quindi:
\begin{equation*}
	\begin{split}
		X'
		&= \tens{U}_3^{\dagger} X \tens{U}_3^{\dagger} = \left( \sigma_1 \cos \phi - \sigma_2 \sin \phi \right) x + \left( \sigma_1 \sin \phi + \sigma_2 \cos \phi \right) y + \sigma_3 z \\
		&= \sigma_1 \left( x \cos \phi + y \sin \phi \right) + \sigma_2 \left( -x \sin \phi + y \cos \phi \right) + \sigma_3 z \equiv \sigma_1 x' + \sigma_2 y' + \sigma_3 z'
	\end{split}
\end{equation*}
Questa è proprio una rotazione di $ \phi $ attorno all'asse $ z $, dunque si è confermato che l'azione di $ \tens{U}_{\ve{n}}(\theta) \in \SUn{2} $ è analoga all'azione di $ \tens{R}_{\ve{n}}(\theta) \in \SOn{3} $.
Notando infine che $ -\tens{U} $ produce la stessa rotazione di $ \tens{U} $, si può affermare che questa è proprio l'omomorfismo $ \SUn{2} \rightarrow \SOn{3} $.\\
Dato che $ \{\tens{U},-\tens{U}\} $ vengono mappati alla stessa rotazione, topologicamente si dice che $ \SUn{2} $ ricopre due volte $ \SOn{3} $. Il kernel dell'omomorfismo è dunque $ \{\tens{I},-\tens{I}\} \triangleleft \SUn{2} $, quindi, dato che $ \{\tens{I},-\tens{I}\} \cong \Z_2 $:
\begin{equation}
	\SOn{3} \cong \SUn{2} / \Z_2
	\label{eq:10.20}
\end{equation}
In maniera analoga, $ \SOn{3,1} $ è legato al gruppo $ \SL{2}{\C} $ di matrici $ \C^{2\times2} $ con determinante unitario da un omomorfismo che determina un doppio ricoprimento. Definendo $ \sigma_{\mu} \equiv \left( 1, \sigma \right) $, preso un 4-vector $ v_{\mu} $ è possibile costruire una matrice associata come $ V = v_{\mu} \sigma^{\mu} $; ricordando $ \sigma_a \sigma_b = \tens{I} \delta_{ab} + i \epsilon^{abc} \sigma_c $, si dimostra che $ v_{\mu} = \frac{1}{2} \tr \left( \tilde{\sigma}_{\mu} V \right) $, dove $ \tilde{\sigma}_{\mu} \equiv \left( 1, -\sigma \right) $ (stesse componenti di $ \sigma^{\mu} $). Esplicitando:
\begin{equation*}
	V =
	\begin{bmatrix}
		v_0 - v_3 & -v_1 + i v_2 \\
		-v_1 - i v_2 & v_0 + v_3
	\end{bmatrix}
\end{equation*}
dunque $ \det V = v^2 $. Di conseguenza, data $ \tens{A} \in \SL{2}{\C} $, la trasformazione unitaria $ V' = \tens{A} V \tens{A}^{\dagger} $ lascia invariata la norma quadratica di $ v_{\mu} $, ovvero induce una trasformazione di Lorentz propria. Dato che $ \{\tens{A},-\tens{A}\} $ vengono mappate allo stesso elemento di $ \SOn{3,1} $, si ha di nuovo un doppio ricoprimento:
\begin{equation}
	\SOn{3,1} \cong \SL{2}{\C} / \Z_2
	\label{eq:10.21}
\end{equation}
Si può infine notare che $ \SOn{3} < \SOn{3,1} $ e $ \SUn{2} < \SL{2}{\C} $.

\subsection{Generatori}

Dato che $ \SOn{3} < \SOn{3,1} $, i generatori delle rotazioni in $ \R^3 $ sono anche generatori delle rotazioni nello spazio di Minkowski; in particolare, esse non hanno effetto sulla coordinata timelike, dunque i loro generatori sono:
\begin{equation*}
	\tens{X}_1 =
	\begin{bmatrix}
		0 & 0 & 0 & 0 \\
		0 & 0 & 0 & 0 \\
		0 & 0 & 0 & -i \\
		0 & 0 & i & 0
	\end{bmatrix}
	\qquad
	\tens{X}_2 =
	\begin{bmatrix}
		0 & 0 & 0 & 0 \\
		0 & 0 & 0 & i \\
		0 & 0 & 0 & 0 \\
		0 & -i & 0 & 0
	\end{bmatrix}
	\qquad
	\tens{X}_3 =
	\begin{bmatrix}
		0 & 0 & 0 & 0 \\
		0 & 0 & -i & 0 \\
		0 & i & 0 & 0 \\
		0 & 0 & 0 & 0
	\end{bmatrix}
\end{equation*}
Per quanto riguarda invece i boost, essi sono rappresentati dalle matrici di boost:
\begin{equation*}
	\tens{B}_1(\zeta) =
	\begin{bmatrix}
		\cosh \zeta & -\sinh \zeta & 0 & 0 \\
		-\sinh \zeta & \cosh \zeta & 0 & 0 \\
		0 & 0 & 1 & 0 \\
		0 & 0 & 0 & 1
	\end{bmatrix}
	\qquad
	\tens{B}_2(\zeta) =
	\begin{bmatrix}
		\cosh \zeta & 0 & -\sinh \zeta & 0 \\
		0 & 1 & 0 & 0 \\
		-\sinh \zeta  & 0 & \cosh \zeta & 0 \\
		0 & 0 & 0 & 1
	\end{bmatrix}
\end{equation*}
\begin{equation*}
	\tens{B}_3(\zeta) =
	\begin{bmatrix}
		\cosh \zeta & 0 & 0 & -\sinh \zeta \\
		0 & 1 & 0 & 0 \\
		0 & 0 & 1 & 0 \\
		-\sinh \zeta & 0 & 0 & \cosh \zeta
	\end{bmatrix}
\end{equation*}
Ricordando che i generatori di Lie sono definiti come $ \tens{Y}_j = i \frac{d}{d\zeta}\big\vert_{\zeta = 0} \tens{B}_j(\zeta) $, si ha:
\begin{equation*}
	\tens{Y}_1 =
	\begin{bmatrix}
		0 & -i & 0 & 0 \\
		-i & 0 & 0 & 0 \\
		0 & 0 & 0 & 0 \\
		0 & 0 & 0 & 0
	\end{bmatrix}
	\qquad
	\tens{Y}_2 =
	\begin{bmatrix}
		0 & 0 & -i & 0 \\
		0 & 0 & 0 & 0 \\
		-i & 0 & 0 & 0 \\
		0 & 0 & 0 & 0
	\end{bmatrix}
	\qquad
	\tens{Y}_3 =
	\begin{bmatrix}
		0 & 0 & 0 & -i \\
		0 & 0 & 0 & 0 \\
		0 & 0 & 0 & 0 \\
		-i & 0 & 0 & 0
	\end{bmatrix}
\end{equation*}
In maniera compatta:
\begin{equation}
	\tensor{\left[ \tens{Y}_j \right]}{_\mu^\nu} = -i \left( \eta_{\mu 0} \tensor{\eta}{_j^\nu} - \eta_{\mu j} \tensor{\eta}{_0^\nu} \right)
	\label{eq:10.22}
\end{equation}

\begin{proposition}
	Per i generatori di $ \SOn{3,1} $ valgono le seguenti relazioni di commutazione:
	\begin{equation}
		\left[ \tens{X}_i, \tens{X}_j \right] = i \epsilon^{ijk} \tens{X}_k
		\label{eq:10.23}
	\end{equation}
	\begin{equation}
		\left[ \tens{Y}_i, \tens{Y}_j \right] = - i \epsilon^{ijk} \tens{X}_k
		\label{eq:10.24}
	\end{equation}
	\begin{equation}
		\left[ \tens{X}_i, \tens{Y}_j \right] = i \epsilon^{ijk} \tens{Y}_k
		\label{eq:10.25}
	\end{equation}
\end{proposition}
\begin{proof}
	Calcoli.
\end{proof}

Si vede dunque che l'algebra $ \SOn{3,1} $ presenta la sottoalgebra $ \SOn{3} $, mentre i boost, non formando un gruppo, non formano neanche una sottoalgebra. È possibile semplificare l'algebra $ \SOn{3,1} $ definendo dei nuovi generatori.

\begin{proposition}
	Definendo i seguenti generatori di $ \SOn{3,1} $:
	\begin{equation}
		\tens{X}^{\pm}_j \defeq \frac{1}{2} \left( \tens{X}_j \pm i \tens{Y}_j \right)
		\label{eq:10.26}
	\end{equation}
	si hanno le seguenti relazioni di commutazione:
	\begin{equation}
		\left[ \tens{X}^{\pm}_i, \tens{X}^{\pm}_j \right] = i \epsilon^{ijk} \tens{X}^{\pm}_k
		\label{eq:10.27}
	\end{equation}
	\begin{equation}
		\left[ \tens{X}^{\pm}_i, \tens{X}^{\mp}_j \right] = 0
		\label{eq:10.28}
	\end{equation}
\end{proposition}
\begin{proof}
	Calcoli.
\end{proof}

Questi sei generatori sono divisi in due set tra loro indipendenti che formano due sottoalgebre chiuse: l'algebra $ \SOn{3,1} $ si spezza in due sottoalgebre $ \SUn{2} $.\\
Data tale decomposizione dell'algebra $ \SOn{3,1} $, il gruppo di Lorentz ha rappresentazioni irriducibili etichettate come $ \left( j_1,j_2 \right) $, dove questi indici indicano due rappresentazioni irriducibili di $ \SUn{2} $ di dimensione $ (2j_1 + 1) $ e $ (2j_2 + 1) $: la rappresentazione $ (j_1,j_2) $ di $ \SOn{3,1} $ avrà dunque dimensione $ (2j_1 + 1)(2j_2 + 1) $.\\
La rappresentazione $ (0,0) $ è quella banale. Le rappresentazioni $ (\frac{1}{2},0) $ e $ (0,\frac{1}{2}) $ sono dette \textit{rappresentazioni di Weyl} e agiscono su spinori bidimensionali, detti \textit{spinori di Weyl} (usati per descrivere i neutrini). Queste rappresentazioni mettono in luce la violazione della parità: ad esempio, nella rappresentazione $ (\frac{1}{2},0) $ si ha $ \rho_+(\tens{X}^+_j) = \frac{1}{2} \sigma_j $ e $ \rho_+(\tens{X}^-_j) = 0 $, dunque $ \rho_+(\tens{X}_j) = \frac{1}{2} \sigma_j $ e $ \rho_+(\tens{Y}_j) = - \frac{i}{2} \sigma_j $; analogamente, per la rappresentazione $ (0,\frac{1}{2}) $ si trova $ \rho_-(\tens{X}_j) = \frac{1}{2} \sigma_j $ e $ \rho_-(\tens{Y}_j) = \frac{i}{2} \sigma_j $. Dato che l'operatore parità è definito come $ \mathcal{P}\ve{x} = - \ve{x} $, si ha $ \mathcal{P}\tens{X}_j = \tens{X}_j $ e $ \mathcal{P}\tens{Y}_j = -\tens{Y}_j $: sotto trasformazione di parità, le due rappresentazioni si trasformano l'una nell'altra, e questa è proprio la cosiddetta \textit{parity violation}.\\
Per far fronte alla parity violation, si definisce la \textit{rappresentazione di Dirac} $ (\frac{1}{2},0) \oplus (0,\frac{1}{2}) $, la quale agisce sugli spinori a quattro componenti (usati per descrivere, ad esempio, l'elettrone): la somma diretta, infatti, tiene conto del fatto che l'operatore parità trasforma la rappresentazione $ (j_1,j_2) $ in $ (j_2,j_1) $, dunque qualsiasi rappresentazioni che contempli la conservazione della parità deve necessariamente essere simmetrica in $ j_1,j_2 $.\\
Infine, la rappresentazione $ (\frac{1}{2},\frac{1}{2}) $ è quella che $ \virgolette{definisce} $ $ \SOn{3,1} $, essendo la rappresentazione a $ 4 \times 4 $ componenti che agisce sui 4-vectors.\\
Inoltre, le rappresentazioni di dimensione finita di $ \SOn{3,1} $ non sono unitarie, dato che gli $ \tens{X}_j $ sono hermitiani, ma gli $ \tens{Y}_j $ sono antihermitiani, dunque generano matrici antihermitiane: questa è la ragione per cui il doppio ricoprimento di $ \SOn{3,1} $ è dato da $ \SL{2}{\C} $, mentre per $ \SOn{4} $ si ha:
\begin{equation}
	\SOn{4} \cong \SUn{2} \otimes \SUn{2} / \Z_2
	\label{eq:10.29}
\end{equation}
La differenza fondamentale è data dalla segnatura della metrica. Un'ulteriore differenza tra $ \SOn{3,1} $ ed $ \SOn{4} $ è che, a causa delle rotazioni parametrizzate da $ \zeta \in \R^+_0 $, $ \SOn{3,1} $ non è compatto.












