\selectlanguage{italian}

\section{Metrica}

Lo spazio di Minkowski è uno spazio piatto quadridimensionale dotato di metrica:

\begin{equation}
	\left[\eta_{kl}\right] = \left[\eta^{kl}\right] = \diag{(-1,1,1,1)}
	\label{eq:2.1}
\end{equation}

Dalla metrica, si definisce il prodotto scalare:

\begin{equation}
	(\ve{p},\ve{q}) \equiv \ve{p}\cdot\ve{q} \defeq p^k \eta_{kl} q^l
	\label{eq:2.2}
\end{equation}

dove $ \ve{p} = p^k \ve{e}_k $ e $ \ve{q} = q^l \ve{e}_l $. Per la metrica vale:

\begin{equation}
	\eta_{ij} = (\ve{e}_i,\ve{e}_j)
	\label{eq:2.3}
\end{equation}

data la proprietà di abbassamento degli indici Eq. \ref{eq:12}.\\
Da ciò si evince non solo che la base $ \{\ve{e}_j\} $ è ortonormale, ma anche che essa è ortonormale alla base duale:

\begin{equation}
	(\ve{e}^i,\ve{e}_j) = \eta^{ik} (\ve{e}_k,\ve{e}_j) = \eta^{ik} \eta_{kj} = \delta^i_j
	\label{eq:2.4}
\end{equation}

Per un dato vettore $ \ve{x} = x^k \ve{e}_k $, se si passa ad una base $ \{\tilde{\ve{e}}_j\} $, le componenti nella nuova base si trovano come:

\begin{equation}
	\tilde{x}^i = \tilde{\ve{e}}^i \cdot \ve{x} = (\tilde{\ve{e}}^i,\ve{e}_j) x^j \equiv L^i_{\,\,j} x^j
	\label{eq:2.5}
\end{equation}

dove è stata definita la trasformazione di Lorentz $ L^i_{\,\,j} \defeq (\tilde{\ve{e}}^i,\ve{e}_j) $.\\
Il prodotto scalare è invariante rispetto alla base, dunque:

\begin{equation}
	\begin{cases}
		(\ve{p},\ve{q}) = \eta_{ij} p^i q^j  \\
		(\ve{p},\ve{q}) = \eta_{kl} \tilde{p}^k \tilde{q}^l = \eta_{kl} L^k_{\,\,\,i} L^l_{\,\,j} p^i q^j
	\end{cases}
	\quad \Longleftrightarrow \quad \eta_{ij} = L^k_{\,\,\,i} \eta_{kl} L^l_{\,\,j}
	\label{eq:2.6}
\end{equation}

Scrivendo in maniera coordinate-free ($ L^k_{\,\,\,i} = (L^{\intercal})^{\,\,k}_i $):

\begin{equation}
	\eta = \tens{L}^{\intercal} \eta \tens{L}
	\label{eq:2.7}
\end{equation}

Riprendendo le leggi di trasformazione Eq. \ref{eq:4} - \ref{eq:5} si ha:

\begin{equation}
	\begin{split}
		\frac{\pa \tilde{x}^i}{\pa x^j} &= \tilde{\ve{e}}^i \cdot \ve{e}_j\\
		\frac{\pa x^i}{\pa \tilde{x}^j} &= \ve{e}^i \cdot \tilde{\ve{e}}_j
	\end{split}
	\label{eq:2.8}
\end{equation}

È possibile ricavare una relazione generale tra la legge di trasformazione contravariante e quella covariante:

\begin{equation}
	\frac{\pa \tilde{x}^i}{\pa x^j} = \tilde{\ve{e}}^i \cdot \ve{e}_j = \eta^{ik} \eta_{jl} \frac{\pa x^l}{\pa \tilde{x}^k}
	\label{eq:2.9}
\end{equation}

Dato che per la metrica di Minkowski $ \eta^{ik} \eta_{jl} = 0, \pm 1 $, i coefficienti di trasformazione possono differire al più per un segno.\\
È possibile ottenere le trasformazioni di Lorentz non solo come mappe lineari tra coordinate per un cambio di SR, ma anche come mappe lineari tra basi; si definisce:

\begin{equation}
	\tilde{\ve{e}}_i = \Lambda_{i}^{\,\,k} \ve{e}_k
	\label{eq:2.10}
\end{equation}

dunque:

\begin{equation}
	\eta_{ij} = \tilde{\ve{e}}_i \cdot \tilde{\ve{e}}_j = \Lambda_i^{\,\,k} \Lambda_j^{\,\,\,l} \ve{e}_k \cdot \ve{e}_l = \Lambda_i^{\,\,k} \eta_{kl} (\Lambda^{\intercal})_{\,\,j}^l
	\label{eq:2.11}
\end{equation}

ovvero, coordinate-free:

\begin{equation}
	\eta = \Lambda \eta \Lambda^{\intercal}
	\label{eq:2.12}
\end{equation}

Dall'invarianza del prodotto scalare deriva l'invarianza del modulo, ovvero:

\begin{equation}
	\ve{p} = x^{k} \ve{e}_k = \tilde{x}^i \tilde{\ve{e}}_i = L^i_{\,\,k} \Lambda^{\,\,j}_i x^k \ve{e}_j \quad \Longleftrightarrow \quad (\Lambda^{\intercal})^j_{\,\,\,i} L^i_{\,\,k} = \delta^j_{\,\,k}
	\label{eq:2.13}
\end{equation}

ovvero, coordinate-free:

\begin{equation}
	\Lambda^{\intercal} = \tens{L}^{-1}
	\label{eq:2.14}
\end{equation}

Moltiplicando l'Eq. \ref{eq:2.12} a destra e sinistra per $ \eta $, dato che $ \eta^2 = \tens{I}_4 $ si ha:

\begin{equation}
	\eta \Lambda \eta \Lambda^{\intercal} = \Lambda \eta \Lambda^{\intercal} \eta
	\label{eq:2.15}
\end{equation}

Dal teorema di Binet, inoltre, si ricava che:

\begin{equation}
	\det \Lambda = \pm 1
	\label{eq:2.16}
\end{equation}

\section{Relatività ristretta}

Lo spaziotempo della relatività speciale è descritto dalla metrica di Minkowski.\\
Dato che la distanza tra due punti è invariante, consideriamo due punti vicini tra loro $ p^i = x^i $ e $ q^i = x^i + dx^i $:

\begin{equation}
	(\ve{p} - \ve{q}) \cdot (\ve{p} - \ve{q}) = dx^i \ve{e}_i \cdot dx^j \ve{e}_j = \eta_{ij} dx^i dx^j = d\ve{r}^2 - (dx^0)^2
	\label{eq:2.17}
\end{equation}

Dato che $ dx^0 = c\,dt $, si ricava l'invarianza dell'elemento di linea:

\begin{equation}
	ds^2 = d\ve{r}^2 - c^2 dt^2 = (\ve{v}^2 - c^2) dt^2 \le 0
	\label{eq:2.18}
\end{equation}

dove l'uguaglianza vale solo per particelle di massa nulla alla velocità della luce.\\
Nel SR solidale $ \ve{v} = \ve{0} $, dunque si può definire il tempo proprio $ \tau $ come:

\begin{equation}
	d \tau = \frac{\sqrt{-ds^2}}{c} = \sqrt{1 - \frac{\ve{v}^2}{c^2}} dt
	\label{eq:2.19}
\end{equation}

Si deduce così la time dilation, poiché se una particella ha una vita media $ T(\ve{0}) $, in un generico SRI la sua vita media sarà:

\begin{equation}
	T(\ve{v}) = \gamma T(\ve{0})
	\label{eq:2.20}
\end{equation}

dove è stato definito il fattore di Lorentz $ \gamma \defeq \left(1 - \frac{\ve{v}}{c^2}\right)^{-1/2} \in [1, +\infty) $.

\subsection{Elettrodinamica covariante}

Le equazioni di Maxwell sono già in forma covariante, necessitano solo di un paio di piccole correzioni.\\
I potenziali scalare e vettore vengono uniti nel potenziale quadrivettoriale:

\begin{equation}
	A_i \defeq \left(-\frac{\phi}{c}, \ve{A}\right)
	\label{eq:2.21}
\end{equation}

È possibile esprimere i campi elettrico $ \ve{E} = - \nabla \phi - \dot{\ve{A}} $ e magnetico $ \ve{B} = \nabla\times\ve{A} $ in funzione del potenziale quadrivettoriale:

\begin{equation}
	\begin{split}
		B_i &= \epsilon^{ijk} \pa_j A_k\\
		E_i &= c \left(\pa_i A_0 - \pa_o A_i\right)
	\end{split}
	\quad i,j,k = 1,2,3
	\label{eq:2.22}
\end{equation}

Risulta evidente che sia possibile unire $ \ve{E} $ e $ \ve{B} $ in un'unico oggetto, un tensore di rango $ 2 $ detto tensore di Faraday:

\begin{equation}
	F_{ij} \defeq \pa_i A_j - \pa_j A_i
	\label{eq:2.23}
\end{equation}

Si vede subito che è un tensore antisimmetrico e che $ E_i = c F_{0i} $ e $ B_i = \frac{1}{2} \epsilon^{ijk} F_{jk} $, con $ i,j,k = 1,2,3 $. Usando la Prop. \ref{epsilon-delta}, si trova che $ F_{jk} = \epsilon_{ijk} B^i $, con $ i,j,k = 1,2,3 $.\\
In forma esplicita, data la metrica di segnatura $ (-,+,+,+) $, si ha:

\begin{equation}
	\left[F_{ij}\right] =
	\begin{bmatrix}
		0 & -E_1 & -E_2 & -E_3 \\
		E_1 & 0 & B_3 & -B_2 \\
		E_2 & -B_3 & 0 & B_1 \\
		E_3 & B_2 & -B_1 & 0
	\end{bmatrix}
	\label{eq:2.24}
\end{equation}

Grazie al tensore di Faraday, è possibile condensare le due equazioni di Maxwell omogenee $ \nabla\cdot\ve{B} = 0 $ e $ \nabla\times\ve{E} = - \dot{\ve{B}} $ in un'unica equazione (usando il tensore di Levi-Civita quadridimensionale):

\begin{equation}
	\epsilon^{ijkl} \pa_j F_{kl} = 0
	\label{eq:2.25}
\end{equation}

Questa viene anche detta identità di Bianchi ed esprime, tra l'altro, l'inesistenza dei monopoli magnetici: essi infatti sono dei difetti topologici, dunque una loro esistenza renderebbe l'identità di Bianchi non-nulla.\\
In particolare, $ i = 0 $ dà la legge di Gauss per $ \ve{B} $ ed $ i = 1,2,3 $ danno le componenti spaziali dell'equazione di Faraday.\\
Per quanto riguarda le equazioni di Maxwell non-omogenee $ \nabla\cdot\ve{E} = \frac{\rho}{\epsilon_0} $ e $ \nabla\times\ve{B} = \mu_0 \ve{J} + \frac{1}{c^2} \dot{\ve{E}} $, è necessario definire la quadricorrente:

\begin{equation}
	j^k \defeq \left(c\rho, \vec{J}\right)
	\label{eq:2.26}
\end{equation}

In questo modo si può scrivere:

\begin{equation}
	\pa_i F^{ki} = \mu_0 j^k
	\label{eq:2.27}
\end{equation}

È anche possibile scrivere la forza di Lorentz in forma covariante:

\begin{equation}
	\frac{dp^i}{dt} = q \left(-F^{i0} + \epsilon^{ijk} v_j B_k\right)
	\label{eq:2.28}
\end{equation}

Questa equazione si può estendere anche al caso $ j = 0 $: ricordando che $ p^0 = \frac{\mathcal{E}}{c} $, con $ \mathcal{E} $ energia, si ha $ \dot{\mathcal{E}} = q\ve{E}\cdot\ve{v} $, confermando il fatto che $ \ve{B} $ non compie lavoro.\\
L'espressione delle equazioni di Maxwell in forma tensoriale ci assicura che sia soddisfatto il primo postulato di Einstein, ovvero che le leggi della fisica sono uguali in qualsiasi SRI, poiché esse sono invarianti per trasformazioni di Lorentz.\\
Affinché l'elettrodinamica sia una teoria pienamente relativistica, vanno apportate due correzioni:
\begin{enumerate}
	\item la quantità di moto considerata deve essere il proper momentum $ \ve{p} = \gamma m \ve{v} $;
	\item l'energia totale deve includere la rest energy $ \mathcal{E}_0 = mc^2 $.
\end{enumerate}










