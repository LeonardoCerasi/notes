\selectlanguage{italian}

\section{Definizioni}

\begin{definition}
	Si definisce \textit{gruppo} un insieme $ G $ su cui è definita un'operazione binaria (rispetto alla quale $ G $ gode di chiusura) $ \left( a,b \right) \in G \times G \mapsto ab \in G $ con le seguenti proprietà:
	\begin{enumerate}
		\item associatività: $ a(bc) = (ab)c \,\forall a,b,c \in G $;
		\item esistenza dell'identità: $ \exists e \in G : ge = eg = g \,\forall g \in G $;
		\item esistenza dell'inverso: $ \forall g \in G \,\exists g^{-1} \in G : g g^{-1} = g^{-1} g = e $.
	\end{enumerate}
\end{definition}

\begin{example}
	Le traslazioni in $ \R^n $ formano un gruppo in cui l'operazione binaria è la somma vettoriale.
\end{example}
\begin{example}
	L'insieme di tutte le trasformazioni in $ \R^{3,1} $ che lasciano invariata la distanza di Minkowski $ x_1^2 + x_2^2 + x_3^2 - x_4^2 $ forma il \textit{gruppo di Lorentz} $ \SOn{3,1} $.
\end{example}
\begin{example}
	L'insieme di tutte le trasformazioni in $ \R^{3,1} $ che lasciano invariata la distanza tra due punti è il \textit{gruppo di Poincaré} (contiene il gruppo di Lorentz).
\end{example}

\begin{definition}
	Un gruppo $ G $ si definisce \textit{abeliano} se l'operazione di gruppo soddisfa la proprietà commutativa.
\end{definition}

\begin{definition}
	Dato un gruppo $ G $, la cardinalità dell'insieme $ \abs{G} $ è detta \textit{ordine} del gruppo.
\end{definition}

\begin{example}
	Si definisce $ \Z_n $ l'insieme degli interi modulo $ n $, che formano un gruppo finito abeliano di ordine $ n $.
\end{example}
\begin{example}
	Si definisce $ S^n $ il gruppo delle permutazioni di $ \{1,\dots,n\} $.
\end{example}

\begin{definition}
	Si definisce \textit{gruppo di Lie} un gruppo in cui ogni elemento dipende in maniera continua da uno o più parametri.
\end{definition}
I gruppi di Lie hanno ordine infinito.

\begin{definition}\label{def-m-comp}
	Un gruppo di matrici $ G $ si dice \textit{compatto} se la norma data dalla traccia è limitata superiormente: $ \exists m \in \R : \tr ( D^{\dagger}D ) \le m \,\forall D \in G $.
\end{definition}

\subsection{Gruppi di matrici}

Gli insiemi di matrici formano naturalmente gruppi con la moltiplicazione tra matrici.

\begin{definition}
	Si definisce \textit{gruppo lineare generale} su un campo $ \K $ il gruppo $ \GL{n}{\K} $ formato dalle matrici $ \K^{n \times n} $ non singolari.
\end{definition}

\begin{proposition}
	$ \GL{n}{\K} $ è un gruppo infinito non-abeliano e non-compatto.
\end{proposition}

\begin{definition}
	Si definisce \textit{gruppo lineare speciale} $ \SL{n}{\K} \defeq \{ \tens{A} \in \GL{n}{\K} : \det \tens{A} = 1\} $.
\end{definition}

\begin{definition}
	Si definisce \textit{gruppo ortogonale} il gruppo $ \On{n} $ delle matrici $ \R^{n \times n} $ ortogonali.
\end{definition}

\begin{definition}
	Si definisce \textit{gruppo ortogonale speciale} $ \SOn{n} \defeq \{ \tens{A} \in \On{n} : \det \tens{A} = 1\} $.
\end{definition}

$ \On{n} $ è il gruppo delle rotazioni in $ \R^n $, sia proprie che improprie, mentre $ \SOn{n} $ è il sottogruppo delle rotazioni proprie.\\
$ \On{n} $ è il gruppo delle matrici che lasciano invariata la forma quadratica reale $ x_1^2 + \dots + x_n^2 $:
\begin{equation*}
	\braket{\ve{y} , \ve{y}} = \ve{y}^{\intercal} \ve{y} = \ve{x}^{\intercal} \tens{A}^{\intercal} \tens{A} \ve{x} = \ve{x}^{\intercal} \tens{I}_n \ve{x} = \ve{x}^{\intercal} \ve{x} = \braket{\ve{x} , \ve{x}}
\end{equation*}

\begin{proposition}
	$ \tens{A} \in \On{n} \,\Rightarrow\, \det \tens{A} = \pm 1 $.
\end{proposition}
\begin{proof}
	$ \tens{A}^{\intercal} \tens{A} = \tens{I}_n \,\Rightarrow\, 1 = \det(\tens{A}^{\intercal}\tens{A}) = \det(\tens{A})^2 $.
\end{proof}

\begin{definition}
	Si definisce \textit{gruppo unitario} il gruppo $ \Un{n} $ delle matrici $ \C^{n \times n} $ unitarie.
\end{definition}

\begin{definition}
	Si definisce \textit{gruppo unitario speciale} $ \SUn{n} \defeq \{ \tens{A} \in \Un{n} : \det \tens{A} = 1\} $.
\end{definition}

\begin{example}
	$ \Un{1} $ descrive le rotazioni nel piano di Argand.
\end{example}
\begin{example}
	$ \SUn{3} $ è il gruppo di simmetria per interazione forte (QCD).
\end{example}

$ \Un{n} $ è il gruppo delle matrici che lasciano invariata la forma quadratica complessa $ \abs{x_1}^2 + \dots + \abs{x_n}^2 $.

\begin{proposition}
	$ \On{n} $, $ \SOn{n} $, $ \Un{n} $ ed $ \SUn{n} $ sono gruppi di Lie infiniti.
\end{proposition}

\begin{proposition}
	$ \SOn{2} $ è abeliano, $ \SOn{n} $ con $ n \ge 3 $ è non-abeliano.
\end{proposition}

\subsection{Cristallografia}

\begin{definition}
	Si definsce \textit{cristallo} o \textit{reticolo di Bravais} un reticolo di atomi invariante per traslazioni.
\end{definition}

\begin{definition}
	Si definisce il \textit{gruppo spaziale} di un cristallo il gruppo di trasformazioni (traslazioni, rotazioni, riflessioni ed inversioni) che lo lasciano invariato.
\end{definition}

\begin{definition}
	Si definisce il \textit{gruppo puntuale} o \textit{gruppo di simmetria} di un cristallo il sottogruppo del suo gruppo spaziale ottenuto escludendo le traslazioni.
\end{definition}

\begin{example}
	Il reticolo di Bravais cubico è quello col gruppo di simmetria più ampio: esso è invariante per tutte le permutazioni delle coordinate con tutti i segni possibili, dunque il suo gruppo puntuale ha ordine $ 2^3 3! = 48 $.
\end{example}

\begin{theorem}
	Gli unici angoli di rotazione possibili per reticoli di Bravais sono $ \varphi_n = \frac{2\pi}{n} $ con $ n = 1,2,3,4,6 $.
\end{theorem}
\begin{proof}
	Dato un punto $ A $ del reticolo su un suo asse di simmetria perpendicolare al piano considerato, si consideri un punto $ B $ a distanza $ l $ da $ A $ ottenuto tramite traslazione: essendo il reticolo invariante per traslazioni, anche $ B $ si troverà su un suo asse di simmetria perpendicolare al piano.\\
	Siano i punti $ A' $ e $ B' $ ottenuti dalla rotazione di $ A $ e $ B $ con un angolo $ \varphi_n = \frac{2\pi}{n} $ attorno rispettivamente a $ B $ ed $ A $: supponendo che la rotazione di $ \varphi_n $ sia una simmetria del reticolo, i punti $ A' $ e $ B' $ apparterranno al reticolo, dunque possono essere fatti coincidere con una traslazione.\\
	Supponendo che $ l $ sia il più piccolo periodo di traslazione del reticolo, si deve avere che $ A'B' = pl $ con $ p \in \N $, ovvero:
	\begin{equation*}
		A'B' = l + 2l \sin \left( \varphi_n - \frac{\pi}{2} \right) = a \left( 1 - 2 \cos \varphi_n \right) = p l \quad \Rightarrow \quad \cos \varphi_n = \frac{1}{2} (1 - p)
	\end{equation*}
	Dalla condizione $ \abs{\cos \varphi_n} \le 1 $ si trova che i valori possibili di $ p $ sono $ p = 0,1,2,3 $, che corrispondono a $ n = 1,2,3,4,6 $.
\end{proof}

Come conseguenza, è possibile avere, ad esempio, un reticolo di forma esagonale, ma non di forma pentagonale.\\
In realtà dei $ \virgolette{quasi-cristalli} $ con $ n = 5 $ sono stati scoperti nel 1982: questi reticoli godono di simmetria orientazionale a lungo raggio, ma non di simmetria traslazionale a lungo raggio.

\begin{definition}
	L'elemento generico del gruppo delle rotazioni $ C^n $ si indica con $ C_n^r \equiv \frac{2\pi r}{n} $.
\end{definition}

\begin{proposition}
	Ponendo $ c \equiv C_n^1 $, si ha che $ C^n = \braket{c} $.
\end{proposition}
\begin{proof}
	$ C_n^r = c^n \,\forall n = 0, \dots, n - 1 $.
\end{proof}

\begin{proposition}
	$ C^n $ è un gruppo ciclico abeliano di ordine $ n $.
\end{proposition}
\begin{proof}
	$ c^r c^s = c^{r + s} = c^s c^r $ e $ c^n = e $.
\end{proof}

\begin{proposition}
	$ C^n \cong \Z_n $.
\end{proposition}

\begin{proposition}
	Il gruppo puntuale $ D^n $ è di ordine $ 2n $ e $ C^n $ è un suo sottogruppo.
\end{proposition}

\begin{example}
	Il gruppo puntuale $ D^n $ di un poligono regolare di 4 lati è costituito dalle rotazioni attorno ad assi perpendicolati al piano descritte da $ C^4 $ e dalle rotazioni diedre attorno agli assi di simmetria nel piano, generabili a partire dalla riflessione $ b $ rispetto all'asse $ x $: $ b_x = b $, $ b_{y = -x} = bc $, $ b_y = bc^2 $, $ b_{y = x} = bc^3 $. Si vede subito che $ D^4 = \{e,c,c^2,c^3,b,bc,bc^2,bc^3\} $; inoltre, $ c^4 = b^2 = (bc)^2 = e $.
\end{example}
\begin{example}
	Generalizzando l'esempio precedente, $ D^n = \{e,c,\dots,c^{n-1},b,bc,\dots,bc^{n-1}\} $.
\end{example}

\section{Sottogruppi}

\begin{definition}
	Dato un gruppo $ G $, un suo sottoinsieme $ H $ si dice \textit{sottogruppo} se esso è un gruppo sotto la stessa operazione di $ G $, e si scrive $ H \le G $.
\end{definition}
\begin{definition}
	Si dice \textit{sottogruppo proprio} di un gruppo $ G $ un suo sottogruppo $ H $ non triviale, e si scrive $ H < G $.
\end{definition}

\begin{example}
	$ D^3 $ ha sottogruppi porpri $ B^2 = \{e,b\} $ e $ C^3 = \{e,c,c^2\} $.
\end{example}

\begin{definition}
	Dato un sottogruppo $ H $ di un gruppo $ G $, il \textit{left coset} di un elemento $ g \in G $ è definito come $ gH \defeq \{ gh \in G : h \in H\} $.
\end{definition}

\begin{definition}
	Si definisce una \textit{classe di equivalenza} di un gruppo $ G $ un insieme di elementi di $ G $ mutualmente coniugati, ovvero $ g_1 \sim g_2 \,\Leftrightarrow\, \exists f \in G : g_2 = f g f^{-1} $.
\end{definition}

\begin{example}
	In $ D^4 $ gli elementi $ b $ e $ bc^2 $ sono mutualmente coniugati poiché corrispondono a rotazioni diedre dello stesso angolo $ \pi $, i cui assi possono essere fatti coincidere mediante una rotazione: fisicamente, ciò corrisponde al fatto che queste rotazioni possono essere fatte coincidere con una rotazione del RF.
\end{example}

\begin{theorem}[Lagrange]
	Dati un gruppo finito $ G $ e $ H \le G $, si ha che $ \abs{H} $ divide $ \abs{G} $.
\end{theorem}
\begin{proof}
	Dato $ g \in G : g \notin H $, si ha $ gH \cap H = \emptyset $: per assurdo $ \exists g \in G - H : \exists h_1,h_2 \in H : g h_1 = h_2 $, quindi:
	\begin{equation*}
		g h_1 = h_2 \,\Rightarrow\, (g h_1) h_1^{-1} = h_2 h_1^{-1} \,\Rightarrow\, g = h_2 h_1^{-1} \in H
	\end{equation*}
	il che è assurdo. Inoltre, si dimostra che dati $ g_1, g_2 \in G : g_1 \neq g_2 $, allora $ g_1 H \cap g_2 H = \emptyset $. Di conseguenza, è possibile scrivere:
	\begin{equation}
		G = H \cup \bigcup_{g \notin H} gH
		\label{eq:7.1}
	\end{equation}
	dalla quale risulta la tesi.
\end{proof}

\begin{definition}
	Dato un gruppo $ G $, $ H \le G $ si dice \textit{invariante} o \textit{normale} se $ gHg^{-1} = H \,\forall g \in G $, e si scrive $ H \trianglelefteq G $ ($ H \triangleleft G $ se $ H $ è proprio).
\end{definition}

Un sottogruppo normale contiene tutta la classe di equivalenza di ogni suo elemento.

\begin{example}
	In $ D^3 $, $ \{e,b\} $ non è normale poiché la classe di equivalenza di $ b $ è $ \{b,bc,bc^2\} $, mentre $ \{e,c,c^2\} $ è normale.
\end{example}

\begin{definition}
	Dati un gruppo $ G $ e $ H \trianglelefteq G $, si definisce il \textit{gruppo quoziente} di $ G $ rispetto ad $ H $ come $ G / H \defeq \{gH \subseteq G : g \notin H\} $.
\end{definition}

\begin{proposition}
	$ G / H $ è un gruppo con legge moltiplicativa $ (g_1 h_1) (g_2 h_2) = (g_1 g_2) h_3 $.
\end{proposition}
\begin{proof}
	$ (g_1 h_1) (g_2 h_2) = g_1 (g_2 g_2^{-1}) h_1 g_2 h_2 = g_1 g_2 (g_2^{-1} h_1 g_2) h_2 = g_1 g_2 h'_1 h_2 = (g_1 g_2) h_3 $.
\end{proof}

\begin{proposition}
	$ \abs{G / H} = \abs{G} / \abs{H} $.
\end{proposition}
\begin{proof}
	Dal teorema di Lagrange.
\end{proof}

\begin{definition}
	Dati un gruppo $ G $ e $ A,B \trianglelefteq G $, si dice che $ G $ è il \textit{prodotto diretto} $ G = A \otimes B $ se $ ab = ba \,\forall a \in A,b \in B $ e $ \forall g \in G \,\exists a \in A, b \in B : g = ab $.
\end{definition}

\begin{example}
	$ \On{n} = \SOn{n} \otimes \{\tens{I}_n, -\tens{I}_n\} $.
\end{example}

\section{Gruppi di Lie}

\subsection{Rotazioni 2D}

È noto che in $ \R^2 $ una rotazione del sistema di riferimento cartesiano di un angolo $ \theta \in [0,2\pi) $ è data da $ \ve{r}' = \tens{R}(\theta) \ve{r} $, con:
\begin{equation}
	\tens{R}(\theta) =
	\begin{bmatrix}
		\cos \theta & \sin \theta \\
		-\sin \theta & \cos \theta
	\end{bmatrix}
	\label{eq:7.2}
\end{equation}
Si vede immediatamente che $ \abs{\ve{r}'}^2 = \abs{\ve{r}}^2 $; le rotazioni, infatti, possono essere caratterizzate con questa proprietà, che corrisponde ad imporre:
\begin{equation}
	\tens{R}^{\intercal}(\theta) \tens{R}(\theta) = \tens{I}_2
	\label{eq:7.3}
\end{equation}
ovverosia $ \tens{R}(\theta) \in \On{2} $. È facile vedere che l'operazione di gruppo è la composizione di rotazioni $ \tens{R}(\theta_1) \tens{R}(\theta_2) = \tens{R}(\theta_1 + \theta_2) $, con identità $ \tens{R}(0) $ ed inversa $ \tens{R}(-\theta) $.\\
È importante ricordare che $ \On{n} $ è il gruppo delle rotazioni sia proprie che improprie: le rotazioni proprie descritte in Eq. \ref{eq:7.2} hanno l'ulteriore condizione $ \det\tens{R} = 1 $ e formano il sottogruppo $ \SOn{n} $.\\
L'Eq. \ref{eq:7.2} è ricavabile sia da relazioni trigonometriche che ragionando sulla condizione in Eq. \ref{eq:7.3}: l'idea fondamentale è quella di considerare rotazioni infinitesime, ovvero con $ \theta \ll 1 $:
\begin{equation}
	\tens{R}(\theta) \simeq \tens{I}_2 + \theta \tens{A} + o(\theta^2)
	\label{eq:7.4}
\end{equation}
La condizione di ortogonalità diventa dunque:
\begin{equation*}
	\tens{R}^{\intercal}(\theta) \tens{R}(\theta) \simeq \left( \tens{I}_2 + \theta \tens{A}^{\intercal} \right) \left( \tens{I}_2 + \theta \tens{A} \right) \simeq \tens{I}_2 + \theta \left( \tens{A}^{\intercal} + \tens{A} \right) = \tens{I}_2
\end{equation*}
ovvero:
\begin{equation}
	\tens{A}^{\intercal} = - \tens{A}
	\label{eq:7.5}
\end{equation}
In $ \GL{2}{\R} $ c'è un'unica matrice che soddisfa tale condizione (a meno di moltiplicazione per scalare):
\begin{equation}
	\tens{J} =
	\begin{bmatrix}
		0 & 1 \\
		-1 & 0
	\end{bmatrix}
	\label{eq:7.6}
\end{equation}
Ciò significa che le rotazioni infinitesime hanno la forma $ (x',y') \simeq (x + \theta y, y - \theta x) $, concorde con l'Eq. \ref{eq:7.2}.

\begin{proposition}
	Per $ \theta \in [0,2\pi) $ si ha:
	\begin{equation}
		\tens{R}(\theta) = e^{\theta \tens{J}}
		\label{eq:7.7}
	\end{equation}
\end{proposition}
\begin{proof}
	Data la legge moltiplicativa del gruppo delle rotazioni:
	\begin{equation*}
		\tens{R}(\theta) = \lim_{n \rightarrow \infty} \left( \tens{R}\left( \frac{\theta}{n} \right) \right)^n = \lim_{n \rightarrow \infty} \left( \tens{I}_2 + \frac{\theta \tens{J}}{n} \right)^n = e^{\theta \tens{J}}
	\end{equation*}
\end{proof}

Si vede dunque che:
\begin{equation}
	\tens{J} = \frac{d\tens{R}(\theta)}{d\theta} \bigg\vert_{\theta = 0}
	\label{eq:7.8}
\end{equation}

\begin{proposition}
	Eq. \ref{eq:7.7} $ \Rightarrow $ Eq. \ref{eq:7.2}.
\end{proposition}
\begin{proof}
	Notando che $ \tens{J}^2 = -\tens{I}_2 $:
	\begin{equation*}
		e^{\theta \tens{J}} = \sum_{k = 0}^{\infty} \frac{\theta^k \tens{J}^k}{k!} = \sum_{k = 0}^{\infty} \left( -1 \right)^k \frac{\theta^{2k}}{(2k)!} \tens{I}_2 + \sum_{k = 0}^{\infty} \left( -1 \right)^k \frac{\theta^{2k + 1}}{(2k + 1)!} \tens{J} = \cos \theta \tens{I}_2 + \sin \theta \tens{J}
	\end{equation*}
\end{proof}

Si può notare che col metodo di Lie si è ottenuta l'espressione per le rotazioni proprie senza mai dover imporre $ \det \tens{R} = 1 $: ciò è conseguenza del fatto che il metodo di Lie si basa sulla dipendenza del gruppo da un parametro continuo, ma $ \On{n} $ è composto da due sottospazi disconnessi ($ \det\tens{R} = 1 $ e $ \det\tens{R} = -1 $) e uno dei due è disconnesso dall'identità, mentre si è definito $ \tens{R}(\theta) $ continuamente connesso a $ \tens{I}_2 $.

\subsection{Algebra di Lie}

Il metodo di Lie è un metodo generale che si applica a tutti i gruppi dipendenti da un parametro continuo. In particolare, permette di ricavare l'espressione delle rotazioni in $ \R^n : n > 2 $, per le quali l'approccio trigonometrico è infattibile: infatti, la condizione in Eq. \ref{eq:7.5} è valida in qualsiasi dimensione.\\
Nel caso $ n = 3 $ le matrici antisimmetriche base sono:
\begin{equation}
	\tens{J}_x =
	\begin{bmatrix}
		0 & 0 & 0 \\
		0 & 0 & 1 \\
		0 & -1 & 0
	\end{bmatrix}
	\qquad \tens{J}_y =
	\begin{bmatrix}
		0 & 0 & -1 \\
		0 & 0 & 0 \\
		1 & 0 & 0
	\end{bmatrix}
	\qquad \tens{J}_z =
	\begin{bmatrix}
		0 & 1 & 0 \\
		-1 & 0 & 0 \\
		0 & 0 & 0
	\end{bmatrix}
	\label{eq:7.9}
\end{equation}
Le rotazioni 3D dipendono da tre angoli (sempre in $ [0,2\pi) $):
\begin{equation}
	\tens{R}(\theta_x,\theta_y,\theta_z) = e^{\theta_x \tens{J}_x + \theta_y \tens{J}_y + \theta_z \tens{J}_z}
	\label{eq:7.10}
\end{equation}
Le matrici in Eq. \ref{eq:7.9} sono anti-hermitiane: per renderle hermitiane, si definiscono:
\begin{equation}
	J_k \defeq -i \tens{J}_k
	\label{eq:7.11}
\end{equation}
In questo modo, le rotazioni vengono espresse come:
\begin{equation}
	\tens{R}(\theta_x,\theta_y,\theta_z) = \exp \left[ i \sum_{k = 1}^{3} \theta_k J_k \right]
	\label{eq:7.12}
\end{equation}
A differenza delle rotazioni 2D, le rotazioni in $ \R^n : n > 2 $ non commutano:
\begin{equation}
	\tens{R}\tens{R}'\tens{R}^{-1} \simeq \left( \tens{I}_2 + \tens{A} \right) \left( \tens{I}_2 + \tens{B} \right) \left( \tens{I}_2 - \tens{B} \right) = \tens{I}_2 + \tens{B} + \tens{A}\tens{B} - \tens{B}\tens{A} \simeq \tens{R}' + \left[ \tens{A},\tens{B} \right]
\end{equation}
La condizione di commutazione è dunque $ [\tens{A},\tens{B}] = 0 $, dove:
\begin{equation}
	[\tens{A},\tens{B}] = - \sum_{i,j} \theta_i \theta_j \left[ J_i, J_j \right]
	\label{eq:7.14}
\end{equation}
dove il fattore negativo deriva da $ \left( i \right)^2 $ (dall'Eq. \ref{eq:7.12}).
Nel caso $ n = 3 $, svolgendo i calcoli si trova:
\begin{equation}
	\left[ J_i, J_j \right] = i \sum_{k = 1}^{3} \epsilon_{ijk} J_k
	\label{eq:7.15}
\end{equation}
Formalmente, le $ J_k $ si definiscono \textit{generatori} del \textit{gruppo di Lie}, mentre le loro relazioni di commutazione sono l'\textit{algebra di Lie} del gruppo. In generale, un gruppo di Lie può dipendere da un generico set di $ k $ parametri continui (tali per cui $ g(0,\dots,0) = e $): si dicono $ \{T_j\}_{j = 1,\dots,k} $ i generatori del gruppo:
\begin{equation}
	g(\theta_1, \dots, \theta_k) = \exp \left[ \sum_{\alpha = 1}^{k} \theta_{\alpha} T_{\alpha} \right]
	\label{eq:7.16}
\end{equation}
L'algebra di Lie è determinata da relazioni del tipo:
\begin{equation}
	\left[ T_{\alpha}, T_{\beta} \right] = \sum_{\gamma = 1}^{k} f_{\alpha \beta \gamma} T_{\gamma}
	\label{eq:7.17}
\end{equation}
dove le $ f_{\alpha \beta \gamma} $ sono dette costanti di struttura dell'algebra.

\subsection{Rotazioni generiche}

Le rotazioni in $ \R^n $ sono descritte da $ \SOn{n} $: questo è un gruppo di Lie dipendente da $ \frac{1}{2}n(n-1) $ angoli generalizzati $ \theta_{(ab)} : a = 1, \dots, n, \, b = 1, \dots, n-1 $, i quali hanno la proprietà $ \theta_{(ab)} = -\theta_{(ba)} $ (rotazione nel piano $ ab $ cambia di verso quando si prende $ ba $). I generatori di $ \SOn{n} $ sono dunque:
\begin{equation}
	J_{(ab)ij} = -i \left( \delta_{ai} \delta_{bj} - \delta_{mj} \delta_{ni} \right) =
	\begin{cases}
		-i & i = a, j = b \\
		i & i = b, j = a \\
		0 & \text{altrimenti}
	\end{cases}
	\label{eq:7.18}
\end{equation}
Una rotazione generica è quindi:
\begin{equation}
	\tens{R}(\theta) = \exp \left[ i \sum_{a = 1}^{n} \sum_{b = 1}^{n - 1} \theta_{(ab)} J_{(ab)} \right]
	\label{eq:7.19}
\end{equation}

\begin{example}
	In $ \R^3 $, le rotazioni attorno all'asse $ z $ sono generate da $ J_{(12)} $, poiché avvengono nel piano $ xy $.
\end{example}

\begin{proposition}
	L'algebra di Lie di $ \SOn{n} $ è definita da:
	\begin{equation}
		\left[ J_{(ab)},J_{(pq)} \right] = i \left( \delta_{ap} J_{(bq)} + \delta_{bq} J_{(ap)} - \delta_{bp} J_{(aq)} - \delta_{aq} J_{(bp)} \right)
		\label{eq:7.20}
	\end{equation}
\end{proposition}

\subsubsection{Generatori in forma differenziale}

È possibile esprimere i generatori delle rotazioni come operatori differenziali. Ad esempio, in $ \R^2 $:
\begin{equation*}
	\psi'(x',y') = \psi(x - \theta y, y + \theta x) \simeq \psi(x,y) + \theta \left( - y \frac{\pa}{\pa x} + x \frac{\pa}{\pa y} \right) \psi(x,y) \simeq \left( 1 + i \theta \hat{J}_z \right) \psi(x,y)
\end{equation*}
Si notino i segni dovuti al fatto che $ \hat{J}_z $ ruota l'oggetto e non il RF (si traspone).\\
Generalizzando ad $ \R^3 $:
\begin{equation}
	\hat{\ve{J}} = - i \ve{x} \times \nabla
	\label{eq:7.21}
\end{equation}
Si trova con calcolo diretto che questi 3 generatori soddisfano l'Eq. \ref{eq:7.15}. Questi non sono altro che gli operatori momento angolare della meccanica quantistica: $ \ve{L} = \hbar \ve{J} $.










