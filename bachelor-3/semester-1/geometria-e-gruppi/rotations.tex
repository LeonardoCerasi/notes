\selectlanguage{italian}

Il gruppo continuo di rotazione $ \SOn{n} $ è il gruppo che contiene tutte le rotazioni di angoli arbitrari in $ \R^n $: questo è un gruppo compatto, dato che i parametri angolari variano in $ [0,2\pi) $ (o $ [0,\pi) $).

\section{Gruppo \texorpdfstring{$ \SOn{2} $}{TEXT}}

È noto che $ \SOn{2} $ ha una rappresentazione vettoriale fedele $ \mathtt{R} $:
\begin{equation}
	\tens{R}(\phi) =
	\begin{bmatrix}
		\cos \phi & -\sin \phi \\
		\sin \phi & \cos \phi
	\end{bmatrix}
	\label{eq:9.1}
\end{equation}
Se questa rappresentazione è considerata sul campo $ \R $ allora è irriducibile, mentre su $ \C $ è riducibile:
\begin{equation*}
	\begin{pmatrix}
		z' \\ \overline{z}
	\end{pmatrix}
	=
	\begin{pmatrix}
		r' e^{i\theta'} \\ r' e^{-i\theta'}
	\end{pmatrix}
	=
	\begin{pmatrix}
		r e^{i (\theta + \phi)} \\ r e^{-i(\theta + \phi)}
	\end{pmatrix}
	\quad \Rightarrow \quad
	\begin{pmatrix}
		x' + i y' \\ x' - i y'
	\end{pmatrix}
	=
	\begin{bmatrix}
		e^{i\phi} & 0 \\ 0 & e^{-i\phi}
	\end{bmatrix}
	\begin{pmatrix}
		x + i y \\ x - i y
	\end{pmatrix}
\end{equation*}
In generale, in campo complesso $ \SOn{2} $ ha infinite rappresentazioni definite come:
\begin{equation}
	\rho_m(\phi) = e^{im\phi} \quad m\in\Z
	\label{eq:9.2}
\end{equation}
Questo riflette il fatto che il gruppo è abeliano, dunque le sue rappresentazioni complesse irriducibili sono tutte di grado unitario. Inoltre, ciò rende evidente un importante risultato.

\begin{proposition}
	$ \SOn{2} \cong \Un{1} $.
\end{proposition}
\begin{proof}
	L'omomorfismo è $ \varphi : \SOn{2} \rightarrow \Un{1} : \tens{R}(\phi) \mapsto e^{i\phi} $.
\end{proof}

Dato che $ \SOn{2} $ è un gruppo compatto con varietà associata $ \mathbb{S}^1 $, si può ridefinire il prodotto scalare in Eq. \ref{eq:8.11} sostituendo:
\begin{equation*}
	\frac{1}{\abs{G}} \sum_{g \in G} \quad \longrightarrow \quad \frac{1}{2\pi} \int_0^{2\pi} d\phi
\end{equation*}
Ciò dà la corretta ortogonalità tra caratteri:
\begin{equation*}
	\braket{\chi_n, \chi_m} = \int_0^{2\pi} \frac{d\phi}{2\pi} e^{in\phi} e^{-im\phi} = \delta_{nm}
\end{equation*}
È così possibile studiare la serie di Clebsch-Gordan per due rapppresentazioni irriducibili di $ \SOn{2} $:
\begin{equation}
	\braket{\chi_m \chi_n, \chi_k} = \delta_{m + n, k}
	\quad \Rightarrow \quad
	\rho_m \otimes \rho_n = \rho_{m + n}
	\label{eq:9.3}
\end{equation}
Si trova inoltre la decomposizione di $ \mathtt{R} $:
\begin{equation*}
	\alpha_m = \braket{\chi_{\mathtt{R}}, \chi_m} = \int_0^{2\pi} \frac{d\phi}{2\pi} 2\cos \phi e^{im\phi} = \int_0^{2\pi} \frac{d\phi}{2\phi} \left( e^{i (m+1) \phi} + e^{i (m - 1)\phi} \right) = \delta_{m,-1} + \delta_{m,1}
\end{equation*}
Si ritrova dunque la decomposizione già nota:
\begin{equation}
	\mathtt{R} = \rho_1 \oplus \rho_{-1}
	\label{eq:9.4}
\end{equation}

\subsection{Generatori infinitesimi}

Per studiare l'algebra di Lie dei generatori di $ \SOn{2} $, si possono determinare quest'ultimi espandendo la generica rotazione attorno all'origine:
\begin{equation}
	\tens{R}(\phi) = \tens{I} - i \phi \tens{X} + o(\phi^2)
	\quad \Rightarrow \quad
	\tens{X} = i \frac{d \tens{R}(\phi)}{d\phi} \bigg\vert_{\phi = 0} =
	\begin{bmatrix}
		0 & -i \\ i & 0
	\end{bmatrix}
	\eqdef \sigma_2
	\label{eq:9.5}
\end{equation}
Il fatto che il generatore infinitesimo di $ \SOn{2} $ sia hermitiano (in particolare, una matrice di Pauli) deriva dalla condizione di ortogonalità:
\begin{equation*}
	\tens{I} = \tens{R}(\phi)^{\dagger} \tens{R}(\phi) = \left( \tens{I} + i \phi \tens{X}^{\dagger} \right) \left( \tens{I} - i \phi \tens{X} \right) = \tens{I} - i \phi \left( \tens{X} - \tens{X}^{\dagger} \right) + o(\phi^2)
	\quad \Rightarrow \quad
	\tens{X}^{\dagger} = \tens{X}
\end{equation*}
Inoltre, dal fatto che le matrici di rotazione hanno determinante unitario deriva che il generatore infinitesimo deve avere traccia nulla.

\begin{proposition}
	Per le rotazioni proprie (continuamente connesse all'identità, dunque senza riflessioni) vale che:
	\begin{equation}
		\tens{R}(\phi) = \exp \left( - i \phi \tens{X} \right)
		\label{eq:9.6}
	\end{equation}
\end{proposition}
\begin{proof}
	Ricordando che $ \sigma_2^2 = \tens{I} $:
	\begin{equation*}
		\exp (-i \phi \tens{X}) = \sum_{k = 0}^{\infty} \frac{(-i \phi)^k}{k!} \tens{X}^k = \tens{I} \cos \phi -i \tens{X} \sin \phi = \tens{R}(\phi)
	\end{equation*}
\end{proof}

Per una rappresentazione complessa irriducibile $ \rho_m $, invece, il generatore è banale: $ \tens{X}_m = -m $.

\subsubsection{Meccanica quantistica}

Nel contesto della meccanica quantistica, il generatore infinitesimo è un operatore $ \hat{X} $ che agisce sulla funzione d'onda $ \psi(\ve{x}) $. Si noti che a seguito di una rotazione, la funzione d'onda ruotata deve continuare ad avere lo stesso valore nel punto ruotato: $ \psi'(\ve{x}') = \psi(\ve{x}) $. Di conseguenza, una rotazione $ \tens{R} $ definisce l'operatore unitario di trasformazione della funzione d'onda $ \hat{U}_{\tens{R}} $ come:
\begin{equation}
	\hat{U}_{\tens{R}} \psi(\ve{x}) = \psi(\tens{R}^{-1} \ve{x})
	\label{eq:9.7}
\end{equation}
Assumendo per semplicità simmetria azimuthale ed espandendo attorno all'origine:
\begin{equation*}
	(1 - i \phi \hat{X} + o(\phi^2)) \psi(r, \theta) = \psi(r, \theta - \phi)
	\quad \Rightarrow \quad
	i \hat{X} = \frac{\pa}{\pa \theta}
\end{equation*}
Scalando $ \hat{X} $ di un fattore $ \hbar $, si recupera l'usuale generatore del momento angolare lungo $ z $:
\begin{equation}
	\hat{J}_z = -i\hbar \frac{\pa}{\pa \theta}
	\label{eq:9.8}
\end{equation}
Le rappresentazioni irriducibili $ \rho_m $ di $ \SOn{2} $ avranno delle autofunzioni di base $ u(\theta) = e^{im\theta} $ che soddisfano $ \hat{X} u_m = m u_m $, dunque si ricava l'equazione agli autovalori:
\begin{equation}
	\hat{J}_z u_m = \hbar m u_m
	\label{eq:9.9}
\end{equation}

\section{Gruppo \texorpdfstring{$ \SOn{3} $}{TEXT}}

Le rotazioni in $ \SOn{2} $ sono un caso particolare delle rotazioni in $ \SOn{3} $, in particolare sono tutte e sole le rotazioni nel piano ortogonale a $ \ve{n} = (0,0,1) $. In $ \SOn{3} $, le rotazioni attorno all'asse $ z $ sono scritte come:
\begin{equation*}
	\tens{R}_3(\phi) =
	\begin{bmatrix}
		\cos \phi & -\sin \phi & 0 \\
		\sin \phi & \cos \phi & 0 \\
		0 & 0 & 1
	\end{bmatrix}
\end{equation*}
Si trova quindi il generatore corrispondente $ [\tens{X}_3]_{ij} = -i \epsilon_{ij3} $. Per una rotazione attorno all'asse $ k = 1,2,3 $:
\begin{equation}
	[\tens{X}_k]_{ij} = -i \epsilon_{ijk}
	\label{eq:9.10}
\end{equation}
Si mostra dunque che:
\begin{equation*}
	\tens{R}_1(\phi) =
	\begin{bmatrix}
		1 & 0 & 0 \\
		0 & \cos \phi & -\sin \phi \\
		0 & \sin \phi & \cos \phi
	\end{bmatrix}
	\quad
	\tens{R}_2(\phi) =
	\begin{bmatrix}
		\cos \phi & 0 & \sin \phi \\
		0 & 1 & 0 \\
		-\sin \phi & 0 & \cos \phi
	\end{bmatrix}
	\quad
	\tens{R}_3(\phi) =
	\begin{bmatrix}
		\cos \phi & -\sin \phi & 0 \\
		\sin \phi & \cos \phi & 0 \\
		0 & 0 & 1
	\end{bmatrix}
\end{equation*}
\begin{equation*}
	\tens{X}_1 =
	\begin{bmatrix}
		0 & 0 & 0 \\
		0 & 0 & -i \\
		0 & i & 0
	\end{bmatrix}
	\quad
	\tens{X}_2 =
	\begin{bmatrix}
		0 & 0 & i \\
		0 & 0 & 0 \\
		-i & 0 & 0
	\end{bmatrix}
	\quad
	\tens{X}_3 =
	\begin{bmatrix}
		0 & -i & 0 \\
		i & 0 & 0 \\
		0 & 0 & 0
	\end{bmatrix}
\end{equation*}
In maniera equivalente, è possibile studiare fisicamente come agisce la rotazione; in generale, per una rotazione infinitesima attorno ad un versore $ \ve{n} $ si ha:
\begin{equation}
	\delta \ve{r} = (\ve{n} \times \ve{r}) \phi
	\label{eq:9.11}
\end{equation}
Per determinare il generico elemento della matrice di rotazione:
\begin{equation*}
	r'_i = [\ve{r} + (\ve{n} \times \ve{r}) \phi]_i = r_i + \phi \sum_{j,k = 1}^{3} \epsilon_{ijk} n_j r_k = r_i - \phi \sum_{j,k = 1}^{3} \epsilon_{ijk} n_k r_j = \sum_{j = 1}^{3} \left[ \delta_{ij} - \phi \sum_{k = 1}^{3} \epsilon_{ijk} n_k \right] r_j
\end{equation*}
Ricordando che $ r_i = \sum_{j = 1}^{3} [\tens{R}_{\ve{n}}(\phi)]_{ij} r_j $, si trova quindi:
\begin{equation}
	[\tens{R}_{\ve{n}}(\phi)]_{ij} = \delta_{ij} - \phi \sum_{k = 1}^{3} \epsilon_{ijk} n_k
	\label{eq:9.12}
\end{equation}
Dall'Eq. \ref{eq:9.5} si ha $ [\tens{R}_{\ve{n}}(\phi)]_{ij} = \delta_{ij} - i \phi [\tens{X}_{\ve{n}}]_{ij} + o(\phi^2) $, dunque si trova il generatore:
\begin{equation}
	[\tens{X}_{\ve{n}}]_{ij} = -i \sum_{k = 1}^{3} \epsilon_{ijk} n_k
	\label{eq:9.13}
\end{equation}
Data l'Eq. \ref{eq:9.10}, definendo il vettore di matrici $ \ve{X} \equiv (\tens{X}_1, \tens{X}_2, \tens{X}_3) $ si ha:
\begin{equation}
	\tens{X}_{\ve{n}} = \ve{X} \cdot \ve{n}
	\label{eq:9.14}
\end{equation}
Questo generico operatore è hermitiano e traceless, così che la matrice di rotazione sia ortogonale e con determinante unitario. Per rotazioni di angolo arbitrario si ha quindi:
\begin{equation}
	\tens{R}_{\ve{n}}(\phi) = \exp \left( -i \ve{n} \cdot \ve{X} \phi \right)
	\label{eq:9.15}
\end{equation}

\subsection{Relazioni di commutazione}

L'algebra di Lie di $ \SOn{3} $ è  determinata dalle relazioni di commutazione tra i suoi generatori.

\begin{proposition}\label{so3-comm}
	$ [\tens{X}_i, \tens{X}_j] = i \sum_{k = 1}^{3} \epsilon_{ijk} \tens{X}_k $.
\end{proposition}
\begin{proof}
	Calcoli.
\end{proof}

Le relazioni di commutazione trovate per via algebrica possono essere ricavate anche fisicamente, studiando come si compongono le rotazioni infinitesime. Ruotando l'asse di rotazione secondo la generica rotazione $ \ve{n}' = \tens{S}(\theta) \ve{n} $ si trova:
\begin{equation}
	\tens{R}_{\ve{n}'}(\phi) = \tens{S}(\theta) \tens{R}_{\ve{n}}(\phi) \tens{S}(\theta)^{-1}
	\label{eq:9.16}
\end{equation}
Prendendo ad esempio $ \ve{n} = (1,0,0) $ (asse $ x $) e ruotandolo attorno all'asse $ z $ di un angolo $ \theta $ si ha $ \ve{n}' = (\cos \theta, \sin \theta, 0) $, quindi, essendo il generatore di $ \tens{S}(\theta) $ in questo caso $ \tens{X}_3 $:
\begin{equation*}
	\tens{R}_{\ve{n}'} = e^{-i \tens{X}_3 \theta} e^{-i \tens{X}_1 \phi} e^{i \tens{X}_3 \theta} = e^{-i \ve{n}' \cdot \ve{X} \phi} = e^{-i (\tens{X}_1 \cos \theta + \tens{X}_2 \sin \theta) \phi}
\end{equation*}
Espandendo al prim'ordine in $ \phi $:
\begin{equation*}
	e^{-i \tens{X}_3 \theta} (\tens{I} - i \tens{X}_1 \phi) e^{i \tens{X}_3 \theta} = \tens{I} - i (\tens{X}_1 \cos \theta + \tens{X}_2 \sin \theta) \phi
	\quad \Rightarrow \quad
	e^{-i \tens{X}_3 \theta} \tens{X}_1 e^{i \tens{X}_3 \theta} = \tens{X}_1 \cos \theta + \tens{X}_2 \sin \theta
\end{equation*}
Derivando entrambi i lati in $ \theta = 0 $ s trova:
\begin{equation*}
	-i [\tens{X}_3, \tens{X}_1] = \tens{X}_2
\end{equation*}
che è proprio quanto espresso dalla Prop. \ref{so3-comm}.

\subsection{Rappresentazioni}

Dall'Eq. \ref{eq:9.16} si vede che le classi di coniugio di $ \SOn{3} $ contengono rotazioni dello stesso angolo attorno ad assi diversi, dunque possono essere etichettate da esso.\\
Nella rappresentazione tridimensionale $ \mathtt{V} $, si trova (basti guardare $ \tens{R}_1, \tens{R}_2, \tens{R}_3 $):
\begin{equation}
	\chi_{\mathtt{V}}(\phi) = 1 + 2 \cos \phi
	\label{eq:9.17}
\end{equation}
Data l'Eq. \ref{eq:9.15}, il problema di trovare rappresentazioni irriducibili del gruppo si riduce al cercare rappresentazioni irriducibili della sua algebra, ovvero dei suoi generatori.\\
Come già notato, i generatori di $ \SOn{3} $ possono essere identificati con gli operatori del momento angolare in meccanica quantistica (a meno di un fattore $ \hbar $), i quali soddisfano la stessa relazione di commutazione; infatti, dall'Eq. \ref{eq:9.11}:
\begin{equation*}
	\psi'(\ve{x}) = \psi(\tens{R}_{\ve{n}}^{-1} \ve{x}) = \psi(\ve{x} - \delta\ve{x}) = \psi(\ve{x} - \phi \ve{n} \times \ve{x}) = \left( 1 - \phi (\ve{n} \times \ve{x}) \cdot \nabla \right) \psi(\ve{x}) = \left( 1 - \frac{i}{\hbar} \phi \ve{n} \cdot \hat{\ve{J}} \right) \psi(\ve{x})
\end{equation*}
Per trovare le rappresentazioni irriducibili di $ \SOn{3} $ è quindi possibile diagonalizzare tali operatori: il grado della rappresentazione sarà dato dal numero di autovalori così trovati.
Si nota che gli $ \hat{J}_k $ non sono compatibili (non commutano), dunque si definiscono i seguenti operatori:
\begin{equation}
	\hat{J}_{\pm} \defeq \hat{J}_x \pm i \hat{J}_y
	\label{eq:9.18}
\end{equation}

\begin{proposition}
	$ [\hat{J}_{\pm}, \hat{J}^2] = 0 $.
\end{proposition}
\begin{proposition}
	$ [\hat{J}_z, \hat{J}_{\pm}] = \pm \hbar \hat{J}_{\pm} $.
\end{proposition}

È quindi possibile definire una base $ \ket{\beta,m} $ tale per cui:
\begin{equation}
	\hat{J}^2 \ket{\beta,m} = \beta^2 \hbar^2 \ket{\beta,m}
	\label{eq:9.19}
\end{equation}
\begin{equation}
	\hat{J}_z \ket{\beta,m} = m \hbar \ket{\beta,m}
	\label{eq:9.20}
\end{equation}
Gli operatori $ \hat{J}_{\pm} $ possono essere interpretati come operatori di scala:
\begin{equation*}
	\hat{J}_z \hat{J}_{\pm} \ket{\beta,m} = \left( \hat{J}_{\pm} \hat{J}_z + [\hat{J}_z, \hat{J}_{\pm}] \right) \ket{\beta,m} = \left( \hat{J}_{\pm} m\hbar \pm \hbar \hat{J}_{\pm} \right) \ket{\beta,m} = (m \pm 1) \hbar \hat{J}_{\pm} \ket{\beta,m}
\end{equation*}
Si deve dunque avere $ \hat{J}_{\pm} \ket{\beta,m} \propto \ket{\beta,m \pm 1} $. Si dimostra anche che la ladder così creata è limitata:
\begin{equation*}
	\beta^2 \hbar^2 = \braket{\beta,m | \hat{J}^2 | \beta,m} = \braket{\beta,m | ( \hat{J}_x^2 + \hat{J}_y^2 + m^2 \hbar^2 ) | \beta,m} \ge m^2 \hbar^2
\end{equation*}
dove l'ultimo passaggio è giustificato dal fatto che $ \braket{\beta,m | (\hat{J}_x^2 + \hat{J}_y^2) | \beta,m} \ge 0 $, essendo hermitiani; si ha quindi $ m^2 \le \beta^2 $. Si deve avere $ \exists j : \hat{J}_+ \ket{\beta,j} = 0 $, dunque:
\begin{equation*}
	\begin{split}
		\beta^2 \hbar^2 \ket{\beta,j}
		&= \hat{J}^2 \ket{\beta,j} = \left( (\hat{J}_x - i \hat{J}_y) (\hat{J}_x + i \hat{J}_y) - i [\hat{J}_x,\hat{J}_y] + \hat{J}_z^2 \right) \ket{\beta,j} \\
		&= \left( \hat{J}_- \hat{J}_+ + \hat{J}_z (\hat{J}_z + \hbar) \right) \ket{\beta,j} = j\hbar (j\hbar + \hbar) \ket{\beta,j}
	\end{split}
\end{equation*}
Si trova quindi $ \beta^2 = j (j + 1) $. Per trovare lo stato bottom, si scrive alternativamente:
\begin{equation*}
	\beta^2 \hbar^2 \ket{\beta,m} = \hat{J}^2 \ket{\beta,m} = \left( \hat{J}_+ \hat{J}_- + \hat{J}_z (\hat{J}_z - \hbar) \right) \ket{\beta,m} = m\hbar (m\hbar - \hbar) \ket{\beta,m}
\end{equation*}
L'equazione $ j(j + 1) = m(m - 1) $ ha come soluzioni $ m = -j, j+1 $, ma la seconda non ha senso poiché lo stato top è $ m = j $, dunque si trova il vincolo completo $ -j \le m \le j $.\\
È risolto dunque il problema del grado delle rappresentazioni irriducibili di $ \SOn{3} $: distinguendole in base al valore di $ j $, esse hanno grado $ 2j + 1 $. Si deve avere necessariamente $ 2j + 1 \in \N $, dunque $ j $ deve essere intero o semi-intero.
Per trovare le costanti di scala:
\begin{equation*}
	\norm{ \hat{J}_{\pm} \ket{j,m} }^2 = \braket{j,m | \hat{J}_{\mp} \hat{J}_{\pm} | j,m} = \braket{j,m | \left( \hat{J}^2 - \hat{J}_z (\hat{J}_z \pm \hbar) \right) | j,m} = \left( (j (j + 1) - m (m \pm 1) \right) \hbar^2
\end{equation*}
Prendendo convenzionalmente la soluzione positiva, si trova:
\begin{equation}
	\hat{J}_{\pm} \ket{j,m} = \hbar \sqrt{(j \mp m) (j \pm m + 1)} \ket{j, m \pm 1}
	\label{eq:9.21}
\end{equation}
Risulta dunque evidente il vincolo $ -j \le m \le j $. Gli elementi di matrice degli operatori $ \hat{J}_{\pm}, \hat{J}_z $ (matrici quadrate di lato $ 2j + 1 $):
\begin{equation}
	\braket{j,m' | \hat{J}_z | j,m} = \hbar m \delta_{m',m}
	\label{eq:9.22}
\end{equation}
\begin{equation}
	\braket{j,m' | \hat{J}_{\pm} | j,m} = \hbar \sqrt{(j \mp m) (j \pm m + 1)} \delta_{m',m \pm 1}
	\label{eq:9.23}
\end{equation}
Queste matrici diagonali e quasi-diagonali forniscono le rappresentazioni irriducibili (indicizzate da $ j $) di $ \SOn{3} \cong \SUn{2} $.

\begin{example}
	Per la rappresentazione $ j = 1 $ si ha:
	\begin{equation*}
		J_z = \hbar
		\begin{bmatrix}
			-1 & 0 & 0 \\
			0 & 0 & 0 \\
			0 & 0 & 1
		\end{bmatrix}
		\quad
		J_+ = \sqrt{2} \hbar
		\begin{bmatrix}
			0 & 0 & 0 \\
			1 & 0 & 0 \\
			0 & 1 & 0
		\end{bmatrix}
		\quad
		J_- = \sqrt{2} \hbar
		\begin{bmatrix}
			0 & 1 & 0 \\
			0 & 0 & 1 \\
			0 & 0 & 0
		\end{bmatrix}
	\end{equation*}
	Queste matrici sono analoghe a quelle trovate per la rappresentazione $ \mathtt{V} $ di $ \SOn{3} $, con la differenza che non sono espresse nella base cartesiana $ (x,y,z) $ ma in quella sferica $ (-\frac{1}{\sqrt{2}}(x + iy), z, \frac{1}{\sqrt{2}} (x - iy)) $.
\end{example}

\subsubsection{Caratteri}

È possibile calcolare i caratteri per la rappresentazione $ \rho_j $ di $ \SOn{3} $ ricordando che le classi di coniugio distinguono solo il valore dell'angolo e non l'asse di rotazione, dunque è possibile limitare il calcolo alle sole rotazioni attorno all'asse $ z $:
\begin{equation*}
	\tens{R}_3(\phi) = e^{-i \tens{X}_3 \phi} = e^{- \frac{i}{\hbar} J_z \phi} = \diag \left( e^{-ij\phi}, e^{-i (j - 1) \phi}, \dots, e^{i (j - 1) \phi}, e^{i j \phi} \right)
\end{equation*}
Si ha quindi:
\begin{equation*}
	\chi_j(\phi) = \sum_{m = -j}^{j} e^{-im\phi} = e^{-ij\phi} \frac{1 - e^{i (2j + 1) \phi}}{1 - e^{i \phi}} = \frac{e^{-i (j + 1/2) \phi} - e^{i (j + 1/2) \phi}}{e^{-i \phi / 2} - e^{i \phi / 2}}
\end{equation*}
ovvero:
\begin{equation}
	\chi_j(\phi) = \frac{\sin \left( (j + \frac{1}{2}) \phi \right)}{\sin \left( \frac{1}{2} \phi \right)}
	\label{eq:9.24}
\end{equation}
Nel caso in cui $ j \in \N $, dalla serie esponenziale si ottiene (le parti immaginarie si semplificano):
\begin{equation}
	\chi_j(\phi) = 1 + 2 \sum_{k = 1}^{j} \cos k \phi
	\label{eq:9.25}
\end{equation}
Nel caso in cui $ j = n + \frac{1}{2} $ con $ n \in \N $, invece:
\begin{equation}
	\chi_j(\phi) = 2 \sum_{k = 0}^{n} \cos \left( (k + \tfrac{1}{2}) \phi \right)
	\label{eq:9.26}
\end{equation}
Essendo $ \SOn{3} $ compatto, è possibile definire un'appropriata misura $ d\mu(g) $ per poter implementare i risultati della Sec. \ref{sec-char}; in particolare, si deve avere $ \int d\mu(g) = 1 $ per normalizzazione e $ \int d\mu(g) = \int d\mu(g') \,\forall g,g' \in G $ per invarianza di gruppo. Per ottenere la corretta ortogonalità dei caratteri, si definisce la misura:
\begin{equation*}
	\frac{1}{\abs{G}} \sum_{g \in G} \qquad \longrightarrow \qquad \int_0^{2\pi} \frac{d\phi}{2\pi} \left( 1 - \cos \phi \right)
\end{equation*}
Infatti:
\begin{equation*}
	\begin{split}
		\braket{\chi_j, \chi_{\ell}}
		&= \frac{1}{2\pi} \int_0^{2\pi} d\phi\, 2\sin^2 \left( \tfrac{1}{2} \phi \right) \frac{\sin \left( (j + \frac{1}{2}) \phi \right)}{\sin \frac{1}{2} \phi} \frac{\sin \left( (\ell + \frac{1}{2}) \phi \right)}{\sin \frac{1}{2} \phi} \\
		&= \frac{1}{2\pi} \int_0^{2\pi} d\phi \left[ \cos \left( (j - \ell) \phi \right) - \cos \left( (j + \ell + 1) \phi \right) \right] = \delta_{j\ell}
	\end{split}
\end{equation*}

\subsubsection{Addizione del momento angolare}

Se si considera un sistema quantistico di due particelle, nel caso non-interagente gli autostati dell'Hamiltoniana possono essere scritti come prodotto diretto $ \ket{j_1,m_1} \otimes \ket{j_2,m_2} $ e la rappresentazione del gruppo di rotazione risulta dunque irriducibile.\\
Se viene inserito un termine d'interazione nell'Hamiltoniana (ad esempio accoppiamento spin-orbita $ \mathcal{H}_{\text{i}} \sim \hat{\ve{L}} \cdot \hat{\ve{S}} $), l'addizione dei momenti angolari non è più così banale: infatti, l'Hamiltoniana non sarà più indipendente sotto rotazioni indipendenti, ma soltanto sotto rotazioni simultanee, dunque è necessario riscrivere il prodotto diretto $ \ket{j_1,m_1} \otimes \ket{j_2,m_2} $ come combinazione lineare di autostati di $ \hat{J}^2 $ e $ \hat{J}_z $, con $ \hat{\ve{J}} \equiv \hat{\ve{J}}_1 + \hat{\ve{J}}_2 $ (propriamente $ \hat{\ve{J}} = \hat{\ve{J}}_1 \otimes \tens{I}_2 + \tens{I}_1 \otimes \hat{\ve{J}}_2 $), così da rispettare l'invarianza ridotta. Si hanno quindi due basi distinte: la base disaccoppiata $ \ket{j_1,m_1 ; j_2,m_2} $ e quella accoppiata $ \ket{j_1,j_2 ; j,m} $.\\
Nel caso della base disaccoppiata si ha lo stato top $ \ket{j_1, j_1 ; j_2, j_2} $, per il quale si ha dunque:
\begin{equation*}
	\hat{J}_z \ket{j_1,j_1 ; j_2,j_2} = (j_1 + j_2) \hbar \ket{j_1,j_1 ; j_2,j_2}
	\qquad
	\hat{J}_{1,+} \ket{j_1,j_1 ; j_2,j_2} = 0
	\qquad
	\hat{J}_{2,+} \ket{j_1,j_1 ; j_2,j_2} = 0
\end{equation*}
Dato che $ \hat{J}_+ = \hat{J}_{1,+} \otimes \tens{I}_2 + \tens{I}_1 \otimes \hat{J}_{2,+} $, si ha anche $ \hat{J}_+ \ket{j_1,j_1 ; j_2,j_2} = 0 $, dunque si identifica $ \ket{j_1,j_1 ; j_2,j_2} = \ket{j_1,j_2 ; j_1 + j_2, j_1 + j_2} $: da questo stato top è possibile costruire gli stati della ladder applicando l'operatore di scala $ \hat{J}_- $; si noti però che lo stato $ \ket{j_1,j_2 ; j_1 + j_2 , j_1 + j_2 - 1} $ è combinazione lineare di $ \ket{j_1,j_1 - 1 ; j_2,j_2} $ e $ \ket{j_1,j_1 ; j_2, j_2 - 1} $, e così via via applicando $ \hat{J}_- $ ripetutamente. Iterativamente, si ottengono tutti gli stati nell'intervallo $ \abs{j_1 - j_2} \le j \le j_1 + j_2 $: ciascuno ha molteplicità $ 2j + 1 $, dunque il numero totale di stati è:
\begin{equation*}
	\sum_{j = \abs{j_1 - j_2}}^{j_1 + j_2} (2j + 1) = \sum_{j = \abs{j_1 - j_2}}^{j_1 + j_2} ((j+1)^2 - j^2) = (j_1 + j_2 + 1)^2 - (j_1 - j_2)^2 = (2j_1 + 1) (2j_2 + 1)
\end{equation*}
Questo è proprio il numero di stati che si ottengono nella rappresentazione disaccoppiata: queste rappresentazioni sono infatti equivalenti, con la differenza che nella rappresentazione accoppiata non sono noti $ m_1 $ ed $ m_2 $, dato che $ \hat{J}_{1,z},\hat{J}_{2,z} $ non commutano con $ \hat{J}^2 $, mentre in quella disaccoppiata non è noto $ j $ per lo stesso motivo.

\subsection{Serie di Clebsch-Gordan}

Dato che l'addizione di due momenti angolari con autovalori $ j_1,j_2 $ risulta in un momento angolare con autovalore $ \abs{j_1 - j_2} \le j \le j_1 + j_2 $, per le rappresentazioni irriducibili ad essi associate si ha:
\begin{equation*}
	\rho_{j_1} \otimes \rho_{j_2} = \bigoplus_{j = \abs{j_1 - j_2}}^{j_1 + j_2} \alpha_j \rho_j
\end{equation*}
In particolare, ogni rappresentazione irriducibile $ \rho_j $ è su uno spazio vettoriale di dimensione $ 2j + 1 $ sul quale sono definite due basi: quella disaccoppiata e quella accoppiata. Di conseguenza, $ \rho_{j_1} \otimes \rho_{j_2}(g) $ sarà una matrice diagonale a blocchi, dove ogni blocco $ \rho_j(g) $ sulla diagonale ha dimensione crescente al crescere di $ j $. Si noti che tale rappresentazione agisce su autovettori dati da armoniche sferiche, producendo loro combinazioni lineari corrispondenti alle rotazioni delle coordinate spaziali.

\begin{proposition}
	Si ha che $ \alpha_j = 1 \,\forall j \in [\abs{j_1 - j_2}, j_1 + j_2] $, ovvero:
	\begin{equation}
		\rho_{j_1} \otimes \rho_{j_2} = \bigoplus_{j = \abs{j_1 - j_2}}^{j_1 + j_2} \rho_j
		\label{eq:9.27}
	\end{equation}
\end{proposition}
\begin{proof}
	Dall'Eq. \ref{eq:9.24}, assumendo WLOG $ j_1 \ge j_2 $:
	\begin{equation*}
		\begin{split}
			\chi_{j_1}(\phi) \chi_{j_2}(\phi)
			&= \frac{e^{i (j_1 + 1/2) \phi} - e^{-i (j_1 + 1/2) \phi}}{2i \sin \frac{1}{2} \phi} \sum_{m = -j_2}^{j_2} e^{i m \phi} = \frac{1}{2i \sin \frac{1}{2} \phi} \sum_{m = -j_2}^{j_2} \left( e^{i (j_1 + m + 1/2) \phi} - e^{-i (j_1 - m + 1/2) \phi} \right) \\
			&= \frac{1}{2i \sin \frac{1}{2} \phi} \sum_{j = j_1 - j_2}^{j_1 + j_2} \left( e^{i (j + 1/2) \phi} - e^{-i (j + 1/2) \phi} \right) = \sum_{j = j_1 - j_2}^{j_1 + j_2} \frac{\sin \left( (j + \frac{1}{2}) \phi \right)}{\sin \frac{1}{2} \phi} = \sum_{j = j_1 - j_2}^{j_1 + j_2} \chi_j(\phi)
		\end{split}
	\end{equation*}
\end{proof}

\subsubsection{Coefficienti di Clebsch-Gordan}

Mentre $ \rho_{j_1} \otimes \rho_{j_2}(g) $ agisce sulla base disaccoppiata, ciascun blocco $ \rho_j(g) $ agisce sulla base accoppiata. Il pasaggio da una base all'altra è dato dai \textit{coefficienti di Clebsch-Gordan}:
\begin{equation}
	\ket{j_1,m_1 ; j_2,m_2} = \sum_{j = \abs{j_1 - j_2}}^{j_1 + j_2} \sum_{m = -j}^{j} \mathcal{C}(j_1,j_2,j ; m_1,m_2,m) \ket{j_1,j_2 ; j,m}
	\label{eq:9.28}
\end{equation}
Si noti che tutti questi autovettori sono ortogonali tra loro, essendo relativi ad autovalori distinti di operatori hermitiani:
\begin{equation}
	\braket{j_1,m'_1 ; j_2,m'_2 | j_1,m_1 ; j_2,m_2} = \delta_{m_1,m'_1} \delta_{m_2,m'_2}
	\label{eq:9.29}
\end{equation}
\begin{equation}
	\braket{j_1,j_2 ; j',m' | j_1,j_2 ; j,m} = \delta_{j,j'} \delta_{m,m'}
	\label{eq:9.30}
\end{equation}
Di conseguenza, si possono determinare i coefficienti:
\begin{equation}
	\mathcal{C}(j_1,j_2,j ; m_1,m_2,m) = \braket{j_1,j_2 ; j,m | j_1,m_1 ; j_2,m_2}
	\label{eq:9.31}
\end{equation}
Questi possono essere visti come gli elementi di una matrice unitaria in cui il primo indice è $ (j,m) $ (nuova base) ed il secondo $ (m_1,m_2) $ (vecchia base). Inoltre, questi coefficienti possono essere resi realti (e dunque la matrice ortogonale) con una fase arbitraria. Dall'unitarietà:
\begin{equation}
	\sum_{j = \abs{j_1 - j_2}}^{j_1 + j_2} \sum_{m = -j}^{j} \mathcal{C}(j_1,j_2,j ; m_1,m_2,m) \overline{\mathcal{C}(j_1,j_2,j ; m'_1,m'_2,m)} = \delta_{m_1,m'_1} \delta_{m_2,m'_2}
	\label{eq:9.32}
\end{equation}
\begin{equation}
	\sum_{m_1 = -j_1}^{j_1} \sum_{m_2 = -j_2}^{j_2} \mathcal{C}(j_1,j_2,j ; m_1,m_2,m) \overline{\mathcal{C}(j_1,j_2,j' ; m_1,m_2,m')} = \delta_{j,j'} \delta_{m,m'}	
	\label{eq:9.33}
\end{equation}
La prima esprime l'ortogonalità della base disaccoppiata e la completezza di quella accoppiata, mentre la seconda il viceversa.\\
Una relazione più usata dell'Eq. \ref{eq:9.28} è la seguente:
\begin{equation}
	\ket{j_1,j_2 ; j,m} = \sum_{m_1 = -j_1}^{j_1} \sum_{m_2 = -j_2}^{j_2} \overline{\mathcal{C}(j_1,j_2,j ; m_1,m_2,m)} \ket{j_1,m_1 ; j_2,m_2}
	\label{eq:9.34}
\end{equation}
In linea generale, per determinare gli autovettori nella base accoppiata, prima si determina $ \ket{j_1,j_2 ; j,j} $ e poi si ottengono gli stati con $ -j \le m \le j $ applicando ripetutamente $ \hat{J}_- $.\\
Tramite i coefficienti di Clebsch-Gordan è anche possibile esplicitare la rappresentazione delle matrici di rotazione del sistema a partire da quelle per i singoli momenti angolari. Ricordando l'Eq. \ref{eq:9.7}, si ha:
\begin{equation*}
	\begin{split}
		[\rho_j(\tens{R})]_{m',m}
		&= \braket{j_1,j_2 ; j,m' | \hat{U}_{\tens{R}} | j_1,j_2 ; j,m} \\
		&= \sum_{m_1,m'_1 = -j_1}^{j_1} \sum_{m_1,m'_2 = -j_2}^{j_2} \mathcal{C}(j_1,j_2,j ; m'_1,m'_2,m') \overline{\mathcal{C}(j_1,j_2,j ; m_1,m_2,m)} \times \\
		& \qquad \qquad \qquad \qquad \qquad \qquad \qquad \qquad \times\braket{j_1,m'_1 ; j_2,m'_2 | \hat{U}_{\tens{R}} | j_1,m_1 ; j_2,m_2}
	\end{split}
\end{equation*}
Nella base disaccoppiata $ \hat{U}_{\tens{R}} $ agisce separatamente su ciascun momento angolare:
\begin{equation}
	\begin{split}
		[\rho_j(\tens{R})]_{m',m}
		&= \sum_{m_1,m'_1 = -j_1}^{j_1} \sum_{m_2,m'_2 = -j_2}^{j_2} \mathcal{C}(j_1,j_2,j ; m'_1,m'_2,m') \overline{\mathcal{C}(j_1,j_2,j ; m_1,m_2,m)} \times \\
		& \qquad \qquad \qquad \qquad \qquad \qquad \qquad \qquad \qquad \qquad \times[\rho_{j_1}(\tens{R})]_{m'_1,m_1} [\rho_{j_2}(\tens{R})]_{m'_2,m_2}
	\end{split}
	\label{eq:9.35}
\end{equation}

\paragraph{Angoli di Eulero}

Dalla meccanica classica è noto che una generica rotazione in $ \R^3 $ può essere parametrizzata da tre angoli $ \phi,\theta,\psi $, detti \textit{angoli di Eulero}, nel modo seguente: la prima rotazione di $ \phi $ è attorno all'asse $ z $, la successiva di $ \theta $ è attorno al nuovo asse $ y $ (ruotato) e l'ultima di $ \psi $ è attorno al nuovo asse $ z $ (ruotato due volte). In questo modo:
\begin{equation}
	\tens{R}(\phi,\theta,\psi) = \tens{R}_3(\psi) \tens{R}_2(\theta) \tens{R}_3(\phi)
	\label{eq:9.36}
\end{equation}
In ambito quantistico, come si vede in Eq. \ref{eq:9.22}, le rotazioni attorno all'asse $ z $ hanno rappresentazione banale, poiché $ \tens{R}_3(\alpha) = \exp \left( -i \tens{X}_3 \alpha \right) $ con $ \tens{X}_3 $ diagonale; l'unica rotazione con rappresentazione non-banale in Eq. \ref{eq:9.36} è $ \tens{R}_2(\theta) $, per cui si definisce la \textit{matrice di Wigner} $ \tens{d}(\theta) \equiv \exp \left( -i \tens{X}_2 \theta \right) $, con $ \tens{X}_2 $ non-diagonale, di modo che:
\begin{equation}
	[\rho_j(\phi,\theta,\psi)]_{m',m} = e^{-i m' \psi} [\tens{d}_j(\theta)]_{m',m} e^{-i m \phi}
	\label{eq:9.37}
\end{equation}

\subsubsection{Accoppiamento tra fermioni}

I fermioni sono particelle con spin $ \frac{1}{2} $; di conseguenza, i possibili valori di $ j $ sono soltanto 0 e 1:
\begin{equation*}
	\rho_{1/2} \otimes \rho_{1/2} = \rho_1 \oplus \rho_0
\end{equation*}
Per semplicità si sopprimono gli indici $ j_1 = j_2 = \frac{1}{2} $, dunque la base disaccoppiata è $ \ket{m_1,m_2} $ e quella accoppiata $ \ket{j,m} $. Dato che l'unico modo di avere $ m = 1 $ è $ m_1 = m_2 = \frac{1}{2} $, si identifica immediatamente lo stato top (scegliendo la fase arbitraria $ \varphi = +1 $) come $ \ket{1,1} = \ket{\tfrac{1}{2},\tfrac{1}{2}} $.
Ricordando l'Eq. \ref{eq:9.21}, si ha:
\begin{equation*}
	\begin{split}
		\hat{J}_- \ket{1,1}
		&= \sqrt{2} \ket{1,0} \\
		&= ( \hat{J}_{1,-} \otimes \tens{I}_2 + \tens{I}_1 \otimes \hat{J}_{2,-} ) \ket{\tfrac{1}{2},\tfrac{1}{2}} = \ket{-\tfrac{1}{2},\tfrac{1}{2}} + \ket{\tfrac{1}{2},-\tfrac{1}{2}}
	\end{split}
\end{equation*}
Reiterando si trova $ \ket{1,-1} = \ket{-\frac{1}{2},-\frac{1}{2}} $.\\
D'altro canto, $ \ket{0,0} $ deve essere combinazione lineare di $ \ket{-\frac{1}{2},\frac{1}{2}} $ e $ \ket{\frac{1}{2},-\frac{1}{2}} $; si deve inoltre avere:
\begin{equation*}
	0 = \hat{J}_+ \ket{0,0} = \hat{J}_+ ( \alpha \ket{-\tfrac{1}{2},\tfrac{1}{2}} + \beta \ket{\tfrac{1}{2},-\tfrac{1}{2}} ) = \left( \alpha + \beta \right) \ket{\tfrac{1}{2},\tfrac{1}{2}}
\end{equation*}
Risolvendo e normalizzando, si trova $ \alpha = -\beta = \frac{1}{\sqrt{2}} $. Si trovano quindi i coefficienti di Clebsch-Gordan per due spin $ \frac{1}{2} $:
\begin{equation*}
	\ket{1,1} = \ket{\tfrac{1}{2},\tfrac{1}{2}}
	\qquad
	\ket{1,0} = \frac{1}{\sqrt{2}} \left( \ket{\tfrac{1}{2},-\tfrac{1}{2}} + \ket{-\tfrac{1}{2},\tfrac{1}{2}} \right)
	\qquad
	\ket{1,-1} = \ket{-\tfrac{1}{2},-\tfrac{1}{2}}
\end{equation*}
\begin{equation*}
	\ket{0,0} = \frac{1}{\sqrt{2}} \left( \ket{\tfrac{1}{2},-\tfrac{1}{2}} - \ket{-\tfrac{1}{2},\tfrac{1}{2}} \right)
\end{equation*}
Rimane soltanto da calcolare la matrice di Wigner. Innanzitutto, si ricordi che la rappresentazione spinoriale $ \rho_{1/2} $ è una rappresentazione di grado 2 che agisce sulla base $ (u_{1/2},u_{-1/2}) $; in questa base, si ha $ \tens{X}_2 = \sigma_2 $ (matrice di Pauli), dunque:
\begin{equation*}
	\tens{d}_{1/2}(\theta) = e^{-i \sigma_2 \theta} = \tens{I} \cos \frac{1}{2} \theta - i \sigma_2 \sin \frac{1}{2} \theta =
	\begin{bmatrix}
		\cos \frac{1}{2} \theta & -\sin \frac{1}{2} \theta \\
		\sin \frac{1}{2} \theta & \cos \frac{1}{2} \theta
	\end{bmatrix}
\end{equation*}
Diagonalizzando, si ha $ \tens{d}_{1/2}(\theta) = \diag ( e^{- i \frac{\theta}{2}}, e^{i \frac{\theta}{2}} ) $: questa è la matrice che determina la rotazione di un fermione preso singolarmente. La matrice di Wigner per il sistema accoppiato è determinabile dall'Eq. \ref{eq:9.35}, ricordando che gli unici coefficienti non nulli sono:
\begin{equation*}
	\mathcal{C}(\tfrac{1}{2},\tfrac{1}{2},1 ; \tfrac{1}{2},\tfrac{1}{2},1) = \mathcal{C}(\tfrac{1}{2},\tfrac{1}{2},1 ; -\tfrac{1}{2},-\tfrac{1}{2},-1) = 1
\end{equation*}
\begin{equation*}
	\mathcal{C}(\tfrac{1}{2},\tfrac{1}{2},1 ; \tfrac{1}{2},-\tfrac{1}{2},0) = \mathcal{C}(\tfrac{1}{2},\tfrac{1}{2},1 ; -\tfrac{1}{2},\tfrac{1}{2},0) = \frac{1}{\sqrt{2}}
\end{equation*}
\begin{equation*}
	\mathcal{C}(\tfrac{1}{2},\tfrac{1}{2},0 ; \tfrac{1}{2},-\tfrac{1}{2},0) = - \mathcal{C}(\tfrac{1}{2},\tfrac{1}{2},0 ; -\tfrac{1}{2},\tfrac{1}{2},0) = \frac{1}{\sqrt{2}}
\end{equation*}
Ad esempio:
\begin{align*}
	m' = 1, m = 1: [\tens{d}_1(\theta)]_{1,1} &= [\tens{d}_{1/2}(\theta)]_{1/2,1/2} [\tens{d}_{1/2}(\theta)]_{1/2,1/2} = \cos^2 \frac{1}{2} \theta = \frac{1}{2} \left( 1 + \cos \theta \right) \\
	m' = 1, m = 0: [\tens{d}_1(\theta)]_{1,0} &= \frac{1}{\sqrt{2}} [\tens{d}_{1/2}(\theta)]_{1/2,1/2} [\tens{d}_1(\theta)]_{1/2,-1/2} + \frac{1}{\sqrt{2}} [\tens{d}_{1/2}(\theta)]_{1/2,-1/2} [\tens{d}_1(\theta)]_{1/2,1/2} \\
						  &= - \sqrt{2} \cos \frac{1}{2} \theta \sin \frac{1}{2} \theta = - \frac{1}{\sqrt{2}} \sin \theta \\
	m' = 1, m = -1: [\tens{d}_1(\theta)]_{1,-1} &= [\tens{d}_{1/2}(\theta)]_{1/2,-1/2} [\tens{d}_1(\theta)]_{1/2,-1/2} = \sin^2 \frac{1}{2} \theta = \frac{1}{2} \left( 1 - \cos \theta \right)
\end{align*}
In questo modo, si trova:
\begin{equation*}
	\tens{d}_1(\theta) =
	\begin{bmatrix}
		\frac{1}{2} \left( 1 + \cos \theta \right) & - \frac{1}{\sqrt{2}} \sin \theta & \frac{1}{2} \left( 1 - \cos \theta \right) \\
		\frac{1}{\sqrt{2}} \sin \theta & \cos \theta & - \frac{1}{\sqrt{2}} \sin \theta \\
		\frac{1}{2} \left( 1 - \cos \theta \right) & \frac{1}{\sqrt{2}} \sin \theta & \frac{1}{2} \left( 1 + \cos \theta \right)
	\end{bmatrix}
\end{equation*}
Si ricordi che una generica rotazione attorno all'asse $ y $ nella base $ (x,y,z) $ è data dalla seguente matrice:
\begin{equation*}
	\tens{R}_2(\theta) =
	\begin{bmatrix}
		\cos \theta & 0 & \sin \theta \\
		0 & 1 & 0 \\
		-\sin \theta & 0 & \cos \theta
	\end{bmatrix}
\end{equation*}
Applicando tale rotazione alla base $ (-\frac{1}{\sqrt{2}} (x + iy), z, \frac{1}{\sqrt{2}} (x - iy)) $ si trova:
\begin{equation*}
	\begin{pmatrix}
		-\frac{1}{\sqrt{2}} (x' + iy') \\
		z' \\
		\frac{1}{\sqrt{2}} (x' - iy')
	\end{pmatrix}
	=
	\begin{pmatrix}
		- \frac{1}{\sqrt{2}} (x \cos \theta + iy + z \sin \theta) \\
		- x \sin \theta + z \cos \theta \\
		\frac{1}{\sqrt{2}} (x \cos \theta - iy + z \sin \theta)
	\end{pmatrix}
	=
	\tens{d}_1(\theta)
	\begin{pmatrix}
		-\frac{1}{\sqrt{2}} (x + iy) \\
		z \\
		\frac{1}{\sqrt{2}} (x - iy)
	\end{pmatrix}
\end{equation*}
Si vede dunque che $ \tens{d}_1(\theta) $ esprimono una rotazione di $ \theta $ attorno all'asse $ y $ rispetto alla base sferica $ (-\frac{1}{\sqrt{2}} (x + iy), z, \frac{1}{\sqrt{2}} (x - iy)) $.










