\selectlanguage{english}

\section{Metric manifolds}

\begin{definition}
  A \textit{metric} $ g $ is a (0,2) tensor field on a manifold $ \mathcal{M} $ that is:
  \begin{enumerate}
    \item symmetric: $ g(X,Y) = g(Y,X) $;
    \item non-degenerate: $ \exists p \in \mathcal{M} : g(X,Y)\vert_p = 0 \,\forall Y \in T_p \mathcal{M} \,\Rightarrow\, X_p = 0 $.
  \end{enumerate}
\end{definition}

\begin{definition}
  A \textit{metric manifold} $ (\mathcal{M},g) $ is a manifold equipped with a metric.
\end{definition}

With a choice of coordinates, the metric can be written as:
\begin{equation}
  g = g_{\mu \nu}(x) dx^{\mu} \otimes dx^{\nu}
  \label{eq:3.1}
\end{equation}
where:
\begin{equation}
  g_{\mu \nu} = g\left( \frac{\pa}{\pa x^{\mu}}, \frac{\pa}{\pa x^{\nu}} \right)
  \label{eq:3.2}
\end{equation}
It is often written also as $ ds^2 = g_{\mu \nu}(x) dx^{\mu} dx^{\nu} $. The matrix $ g_{\mu \nu}(x) \in \R^{n \times n} $ is symmetric, and there's always a choice of basis on each tangent space such that this matrix is diagonal: the non-degeneracy condition implies that none of the diagonal elements vanish.

\begin{proposition}
  The \textit{signature} of a metric, i.e. the number of negative entries when diagonalized, is independent on the choice of basis.
\end{proposition}
\begin{proof}
  From Sylvester's theorem of inertia.
\end{proof}

\paragraph{Riemannian manifolds}

\begin{definition}
  A \textit{Riemannian manifold} $ (\mathcal{M},g) $ is a manifold equipped with a metric with totally-positive signature.
\end{definition}

\begin{example}
  The Euclidean space $ \R^n $, equipped with the metric $ g_{\mu \nu} = \delta_{\mu \nu} $ (in Cartesian coordinates), is a Riemannian manifold.
\end{example}

\begin{definition}
  Given a Riemannian manifold $ (\mathcal{M},g) $ and $ X \in \xm $, the \textit{length} of $ X $ at $ p \in \mathcal{M} $ is:
  \begin{equation}
    \abs{X_p} \defeq \sqrt{g(X,X)\vert_p}
    \label{eq:3.3}
  \end{equation}
  Given $ Y \in \xm $, the \textit{angle} between $ X $ and $ Y $ at $ p \in \mathcal{M} $ is:
  \begin{equation}
    \cos \theta \defeq \frac{g(X,Y) \vert_p}{\abs{X_p} \abs{Y_p}}
    \label{eq:3.4}
  \end{equation}
\end{definition}

This can be generalized to distances between points on a curve $ \sigma : \mathcal{\R} \rightarrow \mathcal{M} $:
\begin{equation}
  d(p,q) = \int_a^b dt \sqrt{g(X,X)\vert_{\sigma(t)}}
  \label{eq:3.5}
\end{equation}
where $ \sigma(a) = p $, $ \sigma(b) = q $ and $ X $ is the tangent vector field of the curve. With parametrization $ x^{\mu}(t) $, the tangent vector has components $ X^{\mu} = \frac{dx^{\mu}}{dt} $, thus:
\begin{equation}
  d(p,q) = \int_a^b dt \sqrt{g_{\mu \nu}(x) \frac{dx^{\mu}}{dt} \frac{dx^{\nu}}{dt}}
  \label{eq:3.6}
\end{equation}
It is important to note that this distance is independent of the parametrization.

\paragraph{Lorentzian manifolds}

\begin{definition}
  A \textit{Lorentzian manifold} $ (\mathcal{M},g) $ is a manifold equipped with a metric which has a signature with a single negative sign.
\end{definition}

\begin{example}
  The simplest Lorentzian manifold is $ \R^n $ with the \textit{Minkowski metric}:
  \begin{equation}
    \eta = - dx^0 \otimes dx^0 + dx^1 \otimes dx^1 + \dots + dx^{n-1} \otimes dx^{n-1}
    \label{eq:3.7}
  \end{equation}
  Its components are $ \eta_{\mu \nu} = \diag (-1,+1,\dots,+1) $, thus this is a Lorentzian manifold.
\end{example}

On a general Lorentzian manifold, at any point $ p \in \mathcal{M} $ it is always possible to choose an orthonormal basis $ \{e_{\mu}\}_{\mu = 0,\dots,n-1} $ of $ T_p (\mathcal{M}) $ such that $ g_{\mu \nu} \vert_p = \eta_{\mu \nu} $: this fact is closely related to the equivalence principle. Consider a different basis $ \tilde{e}_{\mu} = \tensor{\Lambda}{^\nu _\mu} e_{\nu} $: the condition for it to leave the Minkowski metric unchanged is:
\begin{equation}
  \eta_{\mu \nu} = \tensor{\Lambda}{^\rho _\mu} \tensor{\Lambda}{^\sigma _\nu} \eta_{\rho \sigma}
  \label{eq:3.8}
\end{equation}
This is the defining equation of a Lorentz transformation: on a Lorentzian manifold, the basic features of special relativity are locally recovered. Thus, other ideas from special relativity can be imported.

\begin{definition}
  Given a Lorentzian manifold $ (\mathcal{M},g) $ and $ X \in \xm $, at $ p \in \mathcal{M} $ the vector field is said to be:
  \begin{itemize}
    \item \textit{timelike} if $ g(X_p,X_p) < 0 $;
    \item \textit{null} if $ g(X_p,X_p) = 0 $;
    \item \textit{spacelike} if $ g(X_p,X_p) > 0 $.
  \end{itemize}
\end{definition}

At each point $ p \in \mathcal{M} $ it is possible to draw \textit{lightcones}, i.e. the null tangent vectors at that point, which are past-directed or future-directed: these lightcones vary smoothly as the point is varied smoothly on the manifold, elucidating the causal structure of spacetime.\\
The distance between two points on a curve depends on the nature of the tangent vector field of the curve: a \textit{timelike curve} is a curve whose tangent vector field is everywhere timelike, and analogously for the other cases. The distance on a spacelike curve is defined as in Eq. \ref{eq:3.5}, while that on a timelike curve gets a negative sign in the square root. With parametrization $ x^{\mu}(t) $, it is possible to define the \textit{proper time} on a timelike curve as:
\begin{equation}
  \tau = \int_a^b dt \sqrt{- g_{\mu \nu}(x) \frac{dx^{\mu}}{dt} \frac{dx^{\nu}}{dt}}
  \label{eq:3.9}
\end{equation}
This is precisely the action of a free particle moving in spacetime.

\subsection{Metric properties}

The metric defines a natural isomorphism between vectors and covectors.

\begin{proposition}\label{metric-isom}
  Given a metric manifold $ (\mathcal{M},g) $, the metric defines for each $ p \in \mathcal{M} $ a \textit{natural isomorphism} $ g : X_p \in T_p (\mathcal{M}) \rightarrow \omega_p \in T^*_p (\mathcal{M}) : \omega_p(Y_p) = g(X_p,Y_p) \,\forall Y_p \in T_p (\mathcal{M}) $.
\end{proposition}

In a chosen coordinate basis, the vector $ X = X^{\mu} \pa_{\mu} $ is mapped to the one-form $ X = X_{\mu} dx^{\mu} $, thus the following identity holds:
\begin{equation}
  X_{\mu} = g_{\mu \nu} X^{\nu}
  \label{eq:3.10}
\end{equation}
Being $ g $ non-degenerate, the matrix $ g_{\mu \nu} $ is invertible, with inverse $ g^{\mu \nu} $ such that:
\begin{equation}
  g^{\mu \nu} g_{\nu \rho} = \delta^{\mu}_{\rho}
  \label{eq:3.11}
\end{equation}
Its elements are the components of a (2,0) symmetric tensor $ \hat{g} \defeq g^{\mu \nu} \pa_{\mu} \otimes \pa_{\nu} $, which defines the inverse of the natural isomorphism in Prop. \ref{metric-isom}:
\begin{equation}
  X^{\mu} = g^{\mu \nu} X_{\nu}
  \label{eq:3.12}
\end{equation}
The metric also defines a natural volume form on the manifold.

\begin{definition}
  Given an $ n $-dimensional metric manifold $ (\mathcal{M},g) $, the \textit{volume form} is the top-form:
  \begin{equation}
    v \defeq \sqg\, dx^1 \wedge \dots \wedge dx^n
    \label{eq:3.13}
  \end{equation}
  where $ \tens{g} \defeq \abs{\det g_{\mu \nu}} $.
\end{definition}

\begin{proposition}
  The volume form is basis-independent.
\end{proposition}
\begin{proof}
  Consider a new set of coordinates $ y^{\mu} $ such that $ dx^{\mu} = \tensor{A}{^\mu _\nu} dy^{\nu} $, where $ \tensor{A}{^\mu _\nu} = \frac{\pa x^{\mu}}{\pa y^{\nu}} $. In general:
  \begin{equation*}
    dx^1 \wedge \dots \wedge dx^n = \tensor{A}{^1 _{\mu_1}} \dots \tensor{A}{^n _{\mu_n}} dy^{\mu_1} \wedge \dots \wedge dy^{\mu_n}
  \end{equation*}
  Recalling the anti-symmetry of the wedge product and the definition of determinant, this can be rewritten as:
  \begin{equation*}
    dx^1 \wedge \dots \wedge dx^n = \sum_{\pi \in S^n} \sgn{\pi} \tensor{A}{^1 _{\pi(1)}} \dots \tensor{A}{^n _{\pi(n)}} dy^1 \wedge \dots \wedge dy^n = \det A \, dy^1 \wedge \dots \wedge dy^n
  \end{equation*}
  Note the Jacobian factor which arises when changing the measure. On the other hand:
  \begin{equation*}
    g_{\mu \nu} = \frac{\pa y^{\rho}}{\pa x^{\mu}} \frac{\pa y^{\sigma}}{\pa x^{\nu}} \tilde{g}_{\rho \sigma} = \tensor{(A^{-1})}{^\rho _\mu} \tensor{(A^{-1})}{^\sigma _\nu} \tilde{g}_{\rho \sigma}
    \quad \Rightarrow \quad
    \det g_{\mu \nu} = \frac{\det \tilde{g}_{\mu \nu}}{\left( \det A \right)^2}
  \end{equation*}
  The factors $ \det A $ and $ \left( \det A \right)^{-1} $ cancel, thus yielding the thesis.
\end{proof}

The volume form can be rewritten as:
\begin{equation}
  v = \frac{1}{n!} v_{\mu_1 \dots \mu_n} dx^{\mu_1} \wedge \dots \wedge dx^{\mu_n} \equiv \frac{1}{n!} \sqg\, \epsilon_{\mu_1 \dots \mu_n} dx^{\mu_1} \wedge \dots \wedge dx^{\mu_n}
  \label{eq:3.14}
\end{equation}
where $ \epsilon_{\mu_1 \dots \mu_n} $ is the totally-antisymmetric $ n $-dimensional symbol (generalization of the Levi-Civita symbol). $ \epsilon_{\mu_1 \dots \mu_n} $ cannot be considered a proper tensor, as its components are always $ +1,-1,0 $ indipendently if the indices are covariant or contravariant: it is, in fact, a \textit{tensor density}, i.e. a tensor divided by $ \sqg $. It can be shown that:
\begin{equation}
  v^{\mu_1 \dots \mu_n} = g^{\mu_1 \nu_1} \dots g^{\mu_n \nu_n} v_{\mu_1 \dots \mu_n} = \sigma \frac{1}{\sqg} \epsilon^{\mu_1 \dots \mu_n}
  \label{eq:3.15}
\end{equation}
where $ \sigma $ is the sign of the signature (ex.: $ \sigma = +1 $ for Riemannian manifolds and $ \sigma = -1 $ for Lorentzian manifolds).
As notation, the integral of a generic function $ f $ on $ \mathcal{M} $ is denoted as:
\begin{equation}
  \int_{\mathcal{M}} f v \equiv \int_{\mathcal{M}} d^n x \,\sqg f
  \label{eq:3.16}
\end{equation}

\subsubsection{Hodge theory}

\begin{definition}
  Given an $ n $-dimensional oriented metric manifold $ (\mathcal{M},g) $, the \textit{Hodge dual} is defined as the map $ \star : \lm{p} \rightarrow \lm{n-p} : \omega \mapsto \star\, \omega $ such that:
  \begin{equation}
    \star\, \omega_{\mu_1 \dots \mu_{n-p}} \defeq \frac{1}{(n-p)!} \sqg\, \epsilon_{\mu_1 \dots \mu_{n-p} \nu_1 \dots \nu_p} \omega^{\nu_1 \dots \nu_p}
    \label{eq:3.17}
  \end{equation}
\end{definition}

In this section, the orientedness and $ n $-dimensionality of the manifold are implied.

\begin{proposition}
  The Hodge dual is basis-independent.
\end{proposition}

It is useful to state a lemma for future calculations.

\begin{lemma}
  $ v^{\mu_1 \dots \mu_p \rho_1 \dots \rho_{n-p}} v_{\nu_1 \dots \nu_p \rho_1 \dots \rho_{n-p}} = \sigma p! (n-p)! \delta^{\mu_1}_{[\nu_1} \dots \delta^{\mu_p}_{\nu_p]} $.
\end{lemma}

\begin{proposition}\label{hodge-hodge}
  $ \star \left( \star\, \omega \right) = \sigma \left( -1 \right)^{p (n-p)} \omega $.
\end{proposition}

The Hodge dual defines an inner product on each $ \lm{p} $:
\begin{equation}
  \braket{\omega, \eta} \defeq \int_{\mathcal{M}} \omega \wedge \star\, \eta
  \label{eq:3.18}
\end{equation}
This allows to define operators and their adjoints on the form spaces.

\begin{proposition}
  Given a metric manifold $ (\mathcal{M},g) $ and two forms $ \omega \in \lm{p}, \alpha \in \lm{p-1} $, then:
  \begin{equation}
    \braket{d\alpha,\omega} = \braket{\alpha,d^{\dagger}\omega}
    \label{eq:3.19}
  \end{equation}
  where the adjoint of the exterior derivative $ d^{\dagger} : \lm{p} \rightarrow \lm{p-1} $ is defined as:
  \begin{equation}
    d^{\dagger} \defeq \sigma \left( -1 \right)^{np + n -1} \star d \, \star
    \label{eq:3.20}
  \end{equation}
\end{proposition}
\begin{proof}
  To simplify the proof, consider a closed manifold; then, from Stokes' theorem and Eq. \ref{eq:2.38}:
  \begin{equation*}
    0 = \int_{\mathcal{M}} d ( \alpha \wedge \star\, \omega ) = \braket{d\alpha, \omega} + \int_{\mathcal{M}} \left( -1 \right)^{p-1} \alpha \wedge d \star \omega
  \end{equation*}
  The second term is proportional to $ \braket{\alpha, \star\, d \star \omega} $: to determine the relative sign, note that $ d \star \omega \in \lm{n-p+1} $, thus, from Prop. \ref{hodge-hodge}, $ \star \star d \star \omega = \sigma \left( -1 \right)^{(n-p+1)(p-1)} d \star \omega $. In conclusion:
  \begin{equation*}
    \braket{\alpha, \star\, d \star \omega} = \sigma \left( -1 \right)^{(n-p)(p-1)} \int_{\mathcal{M}} \left( -1 \right)^{p-1} \alpha \wedge d \star \omega
    \quad \Rightarrow \quad
    \braket{d\alpha, \omega} = \sigma \left( -1 \right)^{(n-p)(p-1) + 1} \braket{\alpha, \star\, d \star \omega}
  \end{equation*}
  Noting that $ \left( -1 \right)^{(n-p)(p-1) + 1} = \left( -1 \right)^{np + n - 1} $, as in general $ \left( -1 \right)^{-n} = \left( -1 \right)^n $ and $ \left( -1 \right)^{-p^2 + p + 1} = \left( -1 \right)^{-1} $ due to $ p (p-1) $ being always even, concludes the proof.
\end{proof}

\begin{definition}
  Given a metric manifold $ (\mathcal{M},g) $, the \textit{Laplacian} $ \lap : \lm{p} \rightarrow \lm{p} $ is defined as the operator:
  \begin{equation}
    \lap \defeq ( d + d^{\dagger} )^2
    \label{eq:3.21}
  \end{equation}
\end{definition}

\begin{proposition}
  $ \lap = dd^{\dagger} + d^{\dagger}d = \{d,d^{\dagger}\} $.
\end{proposition}
\begin{proof}
  Trivial, given $ d^2 = d^{\dagger 2} = 0 $.
\end{proof}

It is possible to calculate an explicit expression for the Laplacian of functions.

\begin{lemma}\label{lap-calc}
  Given $ f \in \cm $, then $ d^{\dagger} f = 0 $.
\end{lemma}
\begin{proof}
  Trivial noting that $ \star\, f $ is a top-form.
\end{proof}

\begin{proposition}
  Given $ f \in \cm $, then:
  \begin{equation}
    \lap f = - \frac{\sigma}{\sqg} \pa_{\nu} \left( \sqg\, g^{\mu \nu} \pa_{\mu} f \right)
    \label{eq:3.22}
  \end{equation}
\end{proposition}
\begin{proof}
  Via direct calculation, using Lemma \ref{lap-calc}:
  \begin{equation*}
    \begin{split}
      \lap f
      &= \sigma \left( -1 \right)^{n^2 + n - 1} \star d \star \left( \pa_{\mu} f dx^{\mu} \right) = - \sigma \star d \left( \pa_{\mu} f \star dx^{\mu} \right) \\
      &= - \frac{\sigma}{(n-1)!} \star d \left( \pa_{\mu} f g^{\mu \nu} \sqg\, \epsilon_{\nu \rho_1 \dots \rho_{n-1}} dx^{\rho_1} \wedge \dots \wedge dx^{\rho_{n-1}} \right) \\
      &= - \frac{\sigma}{(n-1)!} \star \pa_{\alpha} \left( \sqg\, g^{\mu \nu} \pa_{\mu} f \right) \epsilon_{\nu \rho_1 \dots \rho_{n-1}} dx^{\alpha} \wedge dx^{\rho_1} \wedge \dots \wedge dx^{\rho_{n-1}} \\
      &= - \sigma \star \pa_{\nu} \left( \sqg\, g^{\mu \nu} \pa_{\mu} f \right) dx^1 \wedge \dots dx^n = - \frac{\sigma}{\sqg} \pa_{\nu} \left( \sqg\, g^{\mu \nu} \pa_{\mu} f \right)
    \end{split}
  \end{equation*}
\end{proof}

The Laplacian operator is linked to the de Rham cohomology.

\begin{definition}
  Given $ \omega \in \lm{p} $, it is said to be \textit{harmonic} if $ \lap \omega = 0 $.
\end{definition}

\begin{definition}
  The space of harmonic $ p $-forms on $ (\mathcal{M},g) $ is denoted as $ \hrm{p} $.
\end{definition}

\begin{proposition}\label{harm-closed}
  A harmonic form is both \textit{closed} and \textit{co-closed}.
\end{proposition}
\begin{proof}
  $ 0 = \braket{\omega, \lap \omega} = \braket{d\omega, d\omega} + \braket{d^{\dagger} \omega, d^{\dagger}{\omega}} $, thus $ d\omega = 0 $ and $ d^{\dagger} \omega = 0 $, for the inner product is positive-defined.
\end{proof}

\begin{theorem}\label{form-decomp}
  Given a compact Riemannian manifold $ (\mathcal{M},g) $, any $ \omega \in \lm{p} $ can be uniquely decomposed as $ \omega = d\alpha + d^{\dagger}\beta + \gamma $, with $ \alpha \in \lm{p-1} $, $ \beta \in \lm{p+1} $ and $ \gamma \in \hrm{p} $.
\end{theorem}

\begin{theorem}[Hodge]
  Given a compact Riemannian manifold $ (\mathcal{M},g) $, there is an isomorphism:
  \begin{equation}
    \hrm{p} \cong H^p(\mathcal{M})
    \label{eq:3.23}
  \end{equation}
\end{theorem}
\begin{proof}
  From Prop. \ref{harm-closed} $ \hrm{p} \subset Z^p(\mathcal{M}) $, but the uniqueness of decomposition in Th. \ref{form-decomp} implies $ \forall \gamma \in \hrm{p} \,\exists \eta_{\gamma} \in \lm{p-1} : \gamma \neq d\eta_{\gamma} $, thus $ \hrm{p} \subset H^p(\mathcal{M}) $. \\
  WTS that any equivalence class $ [\omega] \in H^p(\mathcal{M}) $ can be represented by a harmonic form. By Th, \ref{form-decomp} $ \omega = d\alpha + d^{\dagger}\beta + \gamma $, but $ \omega \in H^p(\mathcal{M}) $ implies $ d\omega = 0 $ by definition, so:
  \begin{equation*}
    0 = \braket{d\omega, \beta} = \braket{\omega, d^{\dagger} \beta} = \braket{d\alpha + d^{\dagger}\beta + \gamma, d^{\dagger}\beta} = \braket{d^{\dagger} \beta, d^{\dagger} \beta}
  \end{equation*}
  The inner product is positive-definite, thus $ d^{\dagger} \beta = 0 $, hence $ \omega = \gamma + d\alpha $. By definition $ H^p(\mathcal{M}) \defeq Z^p(\mathcal{M}) / B^p(\mathcal{M}) $, so $ [\omega] = \gamma $.
\end{proof}










