\selectlanguage{english}

Nelle teorie classiche di campo vengono considerati due oggetti distinti: le particelle e i campi. I campi determinano il moto delle particelle, mentre le particelle determinano le oscillazioni dei campi.

\section{Particelle non-relativistiche}

Per descrivere il moto di una particella tra due punti fissati $ \ve{x}(t_1) \equiv \ve{x}_1 $ e $ \ve{x}(t_2) \equiv \ve{x}_2 $ si studia l'azione $ S $ associata alla traiettoria $ \ve{x}(t) $, definita come:
\begin{equation}
  S\left[ \ve{x}(t) \right] \defeq \int_{t_1}^{t_2} dt\, L(\ve{x}(t),\dot{\ve{x}}(t))
  \label{eq:1.1}
\end{equation}
dove $ L $ è la lagragiana che descrive la particella.\\
La traiettoria percorsa dalla particella, per il principio di minima azione, è quella che estremizza $ S $, ovvero tale per cui $ \delta S = 0 \,\,\forall \delta\ve{x}(t) : \delta\ve{x}(t_1) = \delta\ve{x}(t_2) = 0 $; esplicitando:
\begin{equation}
  \begin{split}
    \delta S = \int_{t_1}^{t_2} dt\, \delta L(\ve{x}(t),\dot{\ve{x}}(t))
    &= \int_{t_1}^{t_2} dt\, \left( \frac{\pa L}{\pa x^i} \delta x^i + \frac{\pa L}{\pa \dot{x}^i} \delta\dot{x}^i \right)\\
    &= \int_{t_1}^{t_2} dt\, \left( \frac{\pa L}{\pa x^i} - \frac{d}{dt}\left(\frac{\pa L}{\pa \dot{x}^i}\right) \right) \delta{x}^i + \left[ \frac{\pa L}{\pa\dot{x}^i} \delta x^i \right]_{t_1}^{t_2}
  \end{split}
  \label{eq:1.2}
\end{equation}
dove si è usata la convenzione di somma di Einstein.\\
Si vede subito che il termine di bordo è nullo, dunque estremizzare l'azione equivale alle equazioni di Eulero-Lagrange:
\begin{equation}
  \frac{\pa L}{\pa x^i} - \frac{d}{dt} \frac{\pa L}{\pa \dot{x}^i} = 0
  \label{eq:1.3}
\end{equation}

\subsection{Equazione geodetica}

In generale, il moto di una particella libera su una generica varietà differenziale è descritto dalla lagrangiana $ L = \frac{1}{2}m \dot{\ve{x}}\cdot\dot{\ve{x}} $; bisogna dunque tener conto della metrica della varietà considerata:
\begin{equation}
  L = \frac{m}{2} g_{ij}(x) \dot{x}^i \dot{x}^j
  \label{eq:1.4}
\end{equation}
dove $ x $ rappresenta collettivamente tutte le coordinate $ x^i $ sulla varietà. Si ricordi che, per una varietà reale $ n $-dimensionale, $ g_{ij}\in\R^{n\times n} $ è una matrice reale simmetrica.\\
Le equazioni di Eulero-Lagrange diventano dunque:
\begin{equation}
  \frac{m}{2} \frac{\pa g_{ij}}{\pa x^k} \dot{x}^i \dot{x}^j - \frac{d}{dt} \left( m g_{ik} \dot{x}^i \right) = 0
  \label{eq:1.5}
\end{equation}
Espandendo il secondo termine:
\begin{equation}
  \frac{1}{2} \frac{\pa g_{ij}}{\pa x^k} \dot{x}^i \dot{x}^j - \frac{\pa g_{ik}}{\pa x^j} \dot{x}^i \dot{x}^j - g_{ik} \ddot{x}^i = 0
  \label{eq:1.6}
\end{equation}
Del termine $ g_{ik,j} - \frac{1}{2} g_{ij,k} $, essendo contratto con un fattore simmetrico $ \dot{x}^i \dot{x}^j $, sopravvive solo la parte simmetrica rispetto agli indici $ i $ e $ j $, ovvero:
\begin{equation}
  g_{ik} \ddot{x}^i + \frac{1}{2}\left( \frac{\pa g_{ik}}{\pa x^j} + \frac{\pa g_{jk}}{\pa x^i} - \frac{\pa g_{ij}}{\pa x^k} \right) \dot{x}^i \dot{x}^j = 0
  \label{eq:1.7}
\end{equation}
A questo punto, si contrae per la metrica inversa $ g^{lk} $, che per definizione soddisfa $ g^{lk} g_{ik} = \delta^l_i $, così da ottenere (rinominando gli indici):
\begin{equation}
  \ddot{x}^i + \Gamma^i_{jk} \dot{x}^j \dot{x}^k = 0
  \label{eq:1.8}
\end{equation}
dove è stato definito il simbolo di Christoffel:
\begin{equation}
  \Gamma^i_{jk} \defeq \frac{1}{2} g^{il} \left( \frac{\pa g_{lj}}{\pa x^k} + \frac{\pa g_{lk}}{\pa x^j} - \frac{\pa g_{jk}}{\pa x^l} \right)
  \label{eq:1.9}
\end{equation}
Questa equazione del moto è nota come equazione geodetica e le sue soluzioni sono dette geodetiche.

\section{Particelle relativistiche}

È possibile estendere la meccanica lagrangiana allo spaziotempo di Minkowski $ \R^{1,3} $, descritto dalla metrica:
\begin{equation}
  \eta_{\mu \nu} = \diag(-1,+1,+1,+1)
  \label{eq:1.10}
\end{equation}
Dato che questa metrica non è definita positiva, è possibile classificare due punti separati da una distanza infinitesima $ ds^2 = \eta_{\mu \nu} dx^{\mu} dx^{\nu} $ in base al segno di $ ds^2 $: se $ ds^2 < 0 $ si dicono timelike-separated, se $ ds^2 > 0 $ spacelike-separated e se $ ds^2 = 0 $ lightlike-separated (o null).\\
A differenza del caso classico, in cui l'orbita è parametrizzata dal tempo (che è assoluto), nel caso relativistico essa deve essere parametrizzata da un generico $ \sigma \in \R $ monotono crescente lungo la traiettoria.\\
In ambito relativistico, il principio di minima azione ha un'interpretazione geometrica: la traiettoria deve estremizzare la distanza tra due punti dello spaziotempo. Di conseguenza, dato che una particella di massa $ m $ deve seguire una traiettoria timelike, si definisce l'azione come:
\begin{equation}
  S = -mc \int_{x_1}^{x_2} \sqrt{-ds^2} = -mc \int_{\sigma_1}^{\sigma_2} \sqrt{-\eta_{\mu \nu} \frac{dx^{\mu}}{d\sigma} \frac{dx^{\nu}}{d\sigma}}
  \label{eq:1.11}
\end{equation}
Il coefficiente è necessario per rendere l'azione dimensionalmente omogenea con $ \hbar $.\\
L'azione così definita presenta due simemtrie:
\begin{enumerate}
  \item invarianza di Lorentz: l'azione è invariante per $ x^{\mu} \mapsto \tensor{\Lambda}{^\mu_\nu} x^{\nu} $, con $ \Lambda : \tensor{\Lambda}{^\mu_\rho} \eta_{\mu \nu} \tensor{\Lambda}{^\nu_\sigma} = \eta_{\rho \sigma} $;
  \item invarianza per riparametrizzazioni: essendo $ \sigma $ un parametro arbitrario, è normale che l'azione non dipenda dalla sua scelta, infatti se si riparametrizza con una funzione monotona $ \tilde{\sigma}(\sigma) $ si ha:
    \begin{equation}
      \tilde{S} = -mc \int_{\tilde{\sigma}_1}^{\tilde{\sigma}_2} d\tilde{\sigma} \sqrt{- \eta_{\mu \nu} \frac{dx^{\mu}}{d\tilde{\sigma}} \frac{dx^{\nu}}{d\tilde{\sigma}}} = -mc \int_{\sigma_1}^{\sigma_2} d\sigma \frac{d\tilde{\sigma}}{d\sigma} \sqrt{- \eta_{\mu \nu} \frac{dx^{\mu}}{d\sigma} \frac{dx^{\nu}}{d\sigma} \left( \frac{d\sigma}{d\tilde{\sigma}} \right)^2} = S
      \label{eq:1.12}
    \end{equation}
\end{enumerate}

Grazie all'invarianza per riparametrizzazioni, il valore dell'azione tra due punti dello spaziotempo assume un significato ben preciso, il tempo proprio, ovvero il tempo misurato dalla particella in moto stessa:
\begin{equation}
  \tau(\sigma) = \frac{1}{c} \int_0^{\sigma} d\sigma' \sqrt{- \eta_{\mu \nu} \frac{dx^{\mu}}{d\sigma'} \frac{dx^{\nu}}{d\sigma'}}
  \label{eq:1.13}
\end{equation}
Una conseguenza dell'identificazione tra azione e tempo proprio è che il principio di minima azione richiede che la traiettoria estremizzi il tempo proprio. È anche possibile riparametrizzare l'azione col tempo proprio, essendo questo una funzione monotona crescente lungo la traiettoria.

\subsection{Equazione geodetica}

Nel caso relativistico su una varietà differenziabile generica, la lagrangiana di una particella libera è:
\begin{equation}
  L = \sqrt{ - g_{\mu \nu} (x) \dot{x}^{\mu} \dot{x}^{\nu}}
  \label{eq:1.14}
\end{equation}
Dunque, le equazioni di Eulero-Lagrange diventano:
\begin{equation}
  - \frac{1}{2L} \frac{\pa g_{\mu \nu}}{\pa x^{\rho}} \dot{x}^{\mu} \dot{x}^{\nu} - \frac{d}{d\sigma} \left( - \frac{1}{L} g_{\rho \nu} \dot{x}^{\nu} \right) = 0
  \label{eq:1.15}
\end{equation}
L'unica differenza con Eq. \ref{eq:1.5} è che $ L = L(\sigma) $, dunque si trova un'equazione analoga all'Eq. \ref{eq:1.7} ma con un termine aggiuntivo:
\begin{equation}
  g_{\mu \rho} \ddot{x}^{\rho} + \frac{1}{2} \left( \frac{\pa g_{\mu \rho}}{\pa x^{\nu}} + \frac{\pa g_{\mu \nu}}{\pa x^{\rho}} - \frac{\pa g_{\nu \rho}}{\pa x^{\mu}} \right) \dot{x}^{\nu} \dot{x}^{\rho} = \frac{1}{L} \frac{dL}{d\sigma} g_{\mu \rho} \dot{x}^{\rho}
  \label{eq:1.16}
\end{equation}
È possibile annullare il termine $ \frac{dL}{d\sigma} $ con un'opportuna scelta di parametrizzazione. Dall'Eq. \ref{eq:1.13} si vede che:
\begin{equation}
  c \frac{d\tau}{d\sigma} = L(\sigma)
  \label{eq:1.17}
\end{equation}
Dunque, riparametrizzando con $ \tau(\sigma) $:
\begin{equation}
  L(\tau) = \sqrt{ -g_{\mu \nu}(x) \frac{dx^{\mu}}{d\tau} \frac{dx^{\nu}}{d\tau}} = \frac{d\sigma}{d\tau} L(\sigma) = c
  \label{eq:1.18}
\end{equation}
In generale, qualsiasi riparametrizzazione con $ \tilde{\tau} = a \tau + b $ (parametri affini della worldline) porta ad avere una lagrangiana costante.\\
Ricordando la definizione di connessione affine in Eq. \ref{eq:1.9}, si trova l'\textit{equazione geodetica}:
\begin{equation}
  \frac{d^2 x^{\mu}}{d\tau^2} + \Gamma^{\mu}_{\nu \rho} \frac{dx^{\nu}}{d\tau} \frac{dx^{\rho}}{d\tau}
  \label{eq:1.19}
\end{equation}

\subsection{Momento coniugato}

La differenza sostanziale tra lo spazio euclideo e lo spaziotempo di Minkowski è che, mentre nello spazio euclideo un corpo può rimanere fermo, nello spaziotempo nessun corpo può fermarsi nella direzione temporale. Questo fatto deve essere rispecchiato dal momento della particella:
\begin{equation}
  p_{\mu} = \frac{dL}{dx^{\mu}} = \frac{d}{dx^{\mu}} \left( -mc \sqrt{- \eta_{\mu \nu}} \dot{x}^{\mu} \dot{x}^{\nu} \right) = mc \frac{\eta_{\mu \nu} \dot{x}^{\nu}}{\sqrt{- \eta_{\rho \sigma} \dot{x}^{\rho} \dot{x}^{\sigma} }} = - \frac{m^2 c^2}{L} \eta_{\mu \nu} \dot{x}^{\nu}
  \label{eq:1.20}
\end{equation}
Non tutte le componenti del 4-momento sono indipendenti:
\begin{equation}
  p^2 = p^{\mu} p_{\mu} = \frac{m^4 c^4}{L^2} \eta_{\mu \nu} \dot{x}^{\mu} \dot{x}^{\nu} = -m^2 c^2
  \label{eq:1.21}
\end{equation}
\begin{equation}
  \left( p^0 \right)^2 = \ve{p}^2 + m^2 c^2
  \label{eq:1.22}
\end{equation}
Di conseguenza, si ha sempre $ \left( p^0 \right)^2 > 0 $.\\
Si noti che riparametrizzando la worldline col tempo proprio, dato che $ \frac{d\tau}{d\sigma} = - \frac{L}{mc^2} $:
\begin{equation}
  p^{\mu} = m \frac{d\sigma}{d\tau} \frac{dx^{\mu}}{d\sigma} = m \frac{dx^{\mu}}{d\tau}
  \label{eq:1.23}
\end{equation}
La non-indipendenza di una delle componenti del 4-momento è naturale: da una descrizione classica del sistema risultano tre gradi di libertà $ x^i (t) $, dunque passando ad una descrizione relativistica non può risultare un grado di libertà in più. Ciò è legato all'invarianza per riparametrizzazione: risolvendo le equazioni del moto si trovano le componenti della traiettoria $ x^{\mu} = x^{\mu} (\sigma) $, ma il parametro $ \sigma $ non può rappresentare dell'informazione sul sistema, dunque una delle quattro equazioni del moto va utilizzata per eliminare la dipendenza da $ \sigma $, riducendo di nuovo a tre i gradi di libertà.

\subsection{Interazioni}

Dall'invarianza per riparametrizzazione, è possibile scegliere come parametro $ \sigma = t $ il tempo misurato in un qualunque RF inerziale; considerando una particella libera nello spaziotempo di Minkowski, l'azione in Eq. \ref{eq:1.11} diventa:
\begin{equation}
  S = -mc^2 \int_{t_0}^{t_1} dt\,\sqrt{1 - \frac{\dot{\ve{x}}^2}{c^2}}
  \label{eq:1.24}
\end{equation}
In questa forma, è chiara la presenza di soli tre gradi di libertà dovuti a $ \ve{x}(t) $.

\subsubsection{Elettromagnetismo}

Analogamente al caso classico, per rappresentare l'interazione elettromagnetica è necessario aggiungere un termine potenziale all'azione. Il problema è che la semplice aggiunta di $ \int d\sigma V(\ve{x}) $ non soddisfa l'invarianza per riparametrizzazione: per soddisfarre questo requisito, è necessario individuare un potenziale che cancelli il fattore Jacobiano derivante dalla trasformazione della misura $ d\sigma $. Un'opzione è considerare un termine lineare in $ \dot{x}^{\mu} $, dunque per l'invarianza di Lorentz è necessario che l'indice $ \mu $ sia contratto:
\begin{equation}
  S = \int_{\sigma_1}^{\sigma_2} d\sigma \left[ -mc^2 \sqrt{-\eta_{\mu \nu} \frac{dx^{\mu}}{d\sigma} \frac{x^{\nu}}{d\sigma}} - q A_{\mu}(x) \dot{x}^{\mu} \right]
  \label{eq:1.25}
\end{equation}
dove $ q $ è la carica associata all'interazione e $ A_{\mu}(x) $ è il suo potenziale quadrivettoriale.\\
Scrivendo $ A_{\mu}(x) = (\phi(x)/c, \ve{A}(x)) $, l'azione in Eq. \ref{eq:1.25} descrive l'interazione elettromagnetica; ciò diventa evidente riparametrizzando con $ \sigma = t $:
\begin{equation}
  S = \int_{t_0}^{t_1} dt \left[ -mc^2 \sqrt{1 - \frac{\dot{\ve{x}}^2}{c^2}} - q\phi(x) - q\ve{A}(x)\cdot\ve{x} \right]
  \label{eq:1.26}
\end{equation}

\subsubsection{Gravitazione}

Per descrivere l'interazione gravitazione è necessario considerare un'azione generalizzata del tipo:
\begin{equation}
  S = \int_{t_0}^{t_1} dt \left[ -mc^2 \sqrt{1 + \frac{2\Phi(\ve{x})}{c^2} - \frac{\dot{\ve{x}}^2}{c^2}} \right]
  \label{eq:1.27}
\end{equation}
Nel limite non-relativistico $ \dot{\ve{x}}^2 \ll c^2 $ e $ 2\Phi(\ve{x}) \ll c^2 $, dunque approssimando al prim'ordine:
\begin{equation}
  S = \int_{t_0}^{t_1} dt \left[ -mc^2 + \frac{m}{2}\dot{\ve{x}}^2 - m\Phi(\ve{x}) \right]
  \label{eq:1.28}
\end{equation}
Il primo termine (l'energia a riposo della particella) non ha effetti sull'azione poiché è costante, mentre gli altri termini descrivono il moto non-relativistico di una particella in un campo gravitazionale $ \Phi(\ve{x}) $.\\
Il termine $ 1 + 2\Phi(\ve{x})/c^2 $ in Eq. \ref{eq:1.27} deriva dalla componente $ \eta_{00} $ della metrica, dunque si osserva che la metrica deve dipendere da $ x $, ovvero la descrizione dell'interazione gravitazionale introduce uno spaziotempo curvo. La condizione che deve soddisfarre la metrica in un weak gravitational field è:
\begin{equation}
  g_{00}(x) \approx - \left( 1 + \frac{2\Phi(x)}{c^2} \right)
  \label{eq:1.29}
\end{equation}
con $ \Phi(x) $ il campo gravitazionale Newtoniano.

\section{Principio di Equivalenza}

Come si evince dall'Eq. \ref{eq:1.28}, la $ \virgolette{carica} $ dell'interazione gravitazionale è porprio la massa della particella: questo fatto viene definito \textit{weak equivalence principle} (WEP) ed è solitamente espresso tramite l'uguaglianza tra la massa inerziale e la massa gravitazionale:
\begin{equation}
  m_{\text{i}} = m_{\text{g}}
  \label{eq:1.30}
\end{equation}
Questo è un fatto sperimentale misurato con una precisione dell'ordine di $ 10^{-13} $.

\subsection{Metrica di Kottler-Möller}

Una conseguenza del WEP è l'indistinguibilità tra un'accellerazione costante ed un campo gravitazionale costante: ciò può essere visto in maniera analitica.\\
Considerando una particella di massa $ m $ con accellerazione costante $ \ve{a} = a \hat{\ve{e}}_x $ in un RF inerziale $ \mathcal{O} $, dalla relatività speciale si vede subito che la traiettoria non può essere $ x(t) = \frac{1}{2} at^2 $, poiché la velocità eccederebbe $ c $; bisogna invece ricordare la composizione relativistica delle velocità:
\begin{equation}
  v = \frac{v_1 + v_2}{1 + v_1 v_2 / c^2}
  \label{eq:1.31}
\end{equation}
È possibile definire la rapidità $ \varphi : v = c \tanh\varphi $, così da poter riscrivere la composizione delle velocità come $ \varphi = \varphi_1 + \varphi_2 $.\\
Un'accelerazione costante significa che la rapidità della particella aumenta linearmente rispetto al tempo proprio, ovvero $ \varphi(\tau) = a\tau/c $, quindi:
\begin{equation}
  v(\tau) = \frac{dx}{d\tau} = c \sech \left( \frac{a\tau}{c} \right)
  \label{eq:1.32}
\end{equation}
La relazione tra il tempo misurato nel RF inerziale ed il tempo proprio è:
\begin{equation}
  \frac{dt}{d\tau} = \gamma(\tau) = \sqrt{\frac{1}{1 - v(\tau)^2 / c^2}} = \cosh \left( \frac{a\tau}{c} \right) \quad\Longrightarrow\quad t(\tau) = \frac{c}{a} \sinh \left( \frac{a\tau}{c} \right)
  \label{eq:1.33}
\end{equation}
con costante d'integrazione tale per cui $ \tau = 0 $ corrisponda a $ t = 0 $. Per quanto riguarda la traiettoria:
\begin{equation}
  x(\tau) = \frac{c^2}{a} \cosh \left( \frac{a\tau}{c} \right) - \frac{c^2}{a}
  \label{eq:1.34}
\end{equation}
con costante d'integrazione tale per cui $ x(0) = 0 $. Si trova dunque un'iperbole nello spaziotempo:
\begin{equation}
  \left( x + \frac{c^2}{a} \right)^2 - c^2 t^2 = \frac{c^4}{a^2}
  \label{eq:1.35}
\end{equation}
con asintoti $ ct = \pm \left( x + c^2 / a \right) $ per $ \tau \rightarrow \pm\infty $. Si può mostrare che la trasformazione tra le coordinate $ (t,x) $ nel RF inerziale e quelle $ (\tau,\rho) $ solidali alla particella è data da:
\begin{equation}
  \begin{split}
    ct &= \left( \rho + \frac{c^2}{a} \right) \sinh \left( \frac{a\tau}{c} \right) \\
    x &= \left( \rho + \frac{c^2}{a} \right) \cosh \left( \frac{a\tau}{c} \right) - \frac{c^2}{a}
  \end{split}
  \label{eq:1.36}
\end{equation}
Infatti, l'orbita della particella è giustamente descritta da $ \rho = 0 $. Inoltre, si vede che le coordinate $ (\tau,\rho) $ non ricoprono tutto lo spaziotempo di Minkowski $ (t,x) $: questo dimostra che ci sono delle regioni dello spaziotempo causalmente disconnesse dalla particella.\\
Ricordando che $ ds^2 = -c^2 dt^2 + d\ve{r}^2 $, sostituendo l'Eq. \ref{eq:1.36} si ottiene la \textit{metrica di Kottler-Möller}:
\begin{equation}
  ds^2 = - \left( 1 + \frac{a\rho}{c^2} \right)^2 c^2 d\tau^2 + d\rho^2 + dy^2 + dz^2
  \label{eq:1.37}
\end{equation}
Mentre la parte spaziale rimane piatta, si vede che $ g_{00} = g_{00} (x) $; inoltre, nel caso sub-relativistico:
\begin{equation}
  g_{00} \approx - \left( 1 + \frac{2a\rho}{c^2} \right)
  \label{eq:1.38}
\end{equation}
Definendo $ \Phi(\rho) = a\rho $, si trova proprio la condizione \ref{eq:1.29}: questo è proprio l'assero del WEP, poiché un'accellerazione costante dà una metrica indistinguibile da quella di un campo gravitazionale costante (sub-relativistico).

\subsubsection{Principio di equivalenza di Einstein}

Dal WEP deriva il fatto che un campo gravitazionale costante può essere annullato dalla scelta di un particolare RF, il free-fall RF.\\
Il \textit{principio di equivalenza di Einstein} è una generalizzazione del WEP: esso afferma che esiste sempre un local RF in cui gli effetti di un qualsiasi campo gravitazionale sono localmente annullati. Formalmente, ciò equivale a dire che la metrica $ g_{\mu \nu} $ è sempre localmente approssimabile con la metrica di Minkowski $ \eta_{\mu \nu} $.\\
Gli effetti di un campo gravitazionale non uniforme diventano evidenti quando è possibile svolgere misurazioni su una regione estesa di spazio. Si consideri ad esempio un osservatore confinato in un cubo chiuso in free-fall verso la Terra: non esiste alcun esperimento locale in grado di distinguere se l'osservatore stia fluttuando nello spazio oppure sia in free-fall, bensì è necessario un esperimento non-locale; un esempio di questo tipo di esperimenti consiste nel lasciare libere due masse test (quindi non influenzate vicendevolmente per interazione gravitazionale) separate da una certa distanza: se il cubo sta fluttuando nello spazio, le masse rimaranno nella loro posizione iniziale per inerzia, mentre se esso è in free-fall ciascuna massa sarà attratta verso il centro della Terra, dunque il loro spostamento avrà non solo una componente verticale (rispetto alla caduta), ma anche una orizzontale, per quanto piccola: le masse si sposteranno dunque l'una verso l'altra per effetto di una \textit{tidal force}, uno dei principali fattori discriminanti dei campi gravitazionali non uniformi.

\subsection{Gravitational time dilation}

In condizioni di campo gravitazionale debole si ha $ g_{00}(x) = 1 + \frac{2\Phi(x)}{c^2} $. Considerando il campo gravitazionale di un corpo sferico di massa $ M $ uniforme, $ \Phi(r) = - \frac{GM}{r} $, si ha che un osservatore ad una distanza fissa $ r $ misurerà degli intervalli di tempo dati da:
\begin{equation}
  d\tau^2 = g_{00} dt^2 = \left( 1 - \frac{2GM}{rc^2} \right) dt^2
  \label{eq:1.39}
\end{equation}
Dunque, definendo $ t $ il tempo misurato da un osservatore a $ r \rightarrow \infty $, l'osservatore a $ r $ misurerà:
\begin{equation}
  T(r) = t \sqrt{1 - \frac{2GM}{rc^2}}
  \label{eq:1.40}
\end{equation}
ovvero il tempo scorre più lentamente vicino ad un corpo massivo.\\
È anche possibile mettere in relazione i tempi misurati a due distanze finite $ r_1 $ ed $ r_2 = r_1 + \Delta r $:
\begin{equation}
  \begin{split}
    T_2 = t \sqrt{1 - \frac{2GM}{(r_1 + \Delta r) c^2}} &\approx t \sqrt{1 - \frac{2GM}{r_1 c^2} + \frac{2GM \Delta r}{r_1^2 c^2}}\\
                                                        &\approx t \sqrt{1 - \frac{2GM}{r_1 c^2}} \left( 1 + \frac{GM \Delta r}{r_1^2 c^2} \right) = T_1 \left( 1 + \frac{GM \Delta r}{r_1^2 c^2} \right)
  \end{split}
  \label{eq:1.41}
\end{equation}
dove si è assunto che $ \Delta r \ll r_1 $ e $ 2GM \ll r_1 c^2 $. Ad esempio, con un per un dislivello di $ 10^3 \m $ sul livello del mare ($ \sim 6\cdot10^6 \m $) si ha una differenza di $ 10^{-12}\,\text{s} $, confermato sperimentalmente.

\subsubsection{Gravitational redshift}

Un'importante conseguenza della gravitational time dilation è il gravitational redshift.\\
Sempre in condizioni di campo debole, si consideri un segnale a distanza $ r_1 $ che si ripete ad intervalli $ \Delta T_1 $: un osservatore a $ r_2 $ misurerà:
\begin{equation}
  \Delta T_2 = \sqrt{\frac{1 + 2\Phi(r_2)/c^2}{1 + 2\Phi(r_1)/c^2}} \Delta T_1 \approx \left( 1 + \frac{\Phi(r_2) - \Phi(r_1)}{c^2} \right) \Delta T_1
  \label{eq:1.42}
\end{equation}
Dato che $ \omega \sim T^{-1} $, si ha:
\begin{equation}
  \omega_2 \approx \left( 1 + \frac{\Phi(r_2) - \Phi(r_1)}{c^2} \right)^{-1} \omega_1
  \label{eq:1.43}
\end{equation}
Dato che $ \Phi(r) \sim -r^{-1} $ (nel caso considerato), se $ r_2 > r_1 $ si ha $ \omega_2 < \omega_1 $ (redshift), mentre se $ r_2 < r_1 $ si ha $ \omega_2 > \omega_1 $ (blueshift).










