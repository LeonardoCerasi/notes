\selectlanguage{english}

\section{Topological spaces}

\begin{definition}
  The \textit{topology} $ \mathcal{T} $ of a set $ X $ is a family of subsets of $ X $, i.e. $ \mathcal{T} \subseteq \mathcal{P}(X) $, defined as \textit{open sets}, with the following properties:
  \begin{enumerate}
    \item $ \emptyset,X \in \mathcal{T} $;
    \item $ O_{\alpha},O_{\beta} \in \mathcal{T} \, \Rightarrow\, O_{\alpha}\cap O_{\beta} \in \mathcal{T} $;
    \item $ \{O_{\alpha}\}_{\alpha \in I} \subset \mathcal{T} $ ($ I $ arbitrary index set) $ \Rightarrow \bigcup_{\alpha \in I} O_{\alpha} \in \mathcal{T} $.
  \end{enumerate}
\end{definition}

\begin{definition}
  A \textit{topological space} $ M $ is a set of points, endowed with a topology $ \mathcal{T} $.
\end{definition}

\begin{definition}
  Given a topological space $ (M,\mathcal{T}) $, $ O \in \mathcal{T} $ is a \textit{neighbourhood} of a point $ p \in M $ if $ p \in O $.
\end{definition}

\begin{definition}
  A topological space $ (M,\mathcal{T}) $ is \textit{Hausdorff} if $ \forall p,q \in M \, \exists O_1, O_2 \in \mathcal{T} $ neighbourhoods of $ p $ and $ q $ respectively such that $ O_1 \cap O_2 = \emptyset $.
\end{definition}

\begin{definition}
  A \textit{homeomorphism} between two topological spaces $ (M_1, \mathcal{T}_1) $ and $ (M_2, \mathcal{T}_2) $ is a bijective map $ f : M_1 \rightarrow M_2 $ which is bicontinuous, i.e. both $ f $ and $ f^{-1} $ are continuous: $ f $ is continuous if $ O \in \mathcal{T}_2 \,\Rightarrow\, f^{-1}(O)\in \mathcal{T}_1 $.
\end{definition}

\section{Differentiable Manifolds}

\begin{definition}
  An $ n $-dimensionale \textit{differentiable manifold} $ \mathcal{M} $ is a Hausdorff topological space such that:
  \begin{enumerate}
    \item $ \mathcal{M} $ is locally homeomorphic to $ \R^n $, i.e. $ \forall p\in\mathcal{M} \, \exists O \in \mathcal{T}(\mathcal{M}) : p \in O \land \exists \varphi : O \rightarrow U \in \mathcal{T}(\R^n) $ homeomorphism;
    \item given $ O_{\alpha},O_{\beta} \in \mathcal{T}(\mathcal{M}) : O_{\alpha} \cap O_{\beta} \neq \emptyset $, the corresponding maps $ \varphi_{\alpha} : O_{\alpha} \rightarrow U_{\alpha}, \varphi_{\beta} : O_{\beta} \rightarrow U_{\beta} $ must be \textit{compatible}, i.e. $ \varphi_{\beta} \circ \varphi_{\alpha}^{-1} : \varphi_{\alpha}(O_{\alpha} \cap O_{\beta}) \rightarrow \varphi_{\beta}(O_{\alpha} \cap O_{\beta}) $ and its inverse must be smooth (of $ \mathcal{C}^{\infty} $ class).
  \end{enumerate}
\end{definition}

The maps $ \varphi_{\alpha} $ are called \textit{charts} and a collection of compatible charts is called an \textit{atlas}: a \textit{maximal atlas} $ \mathcal{A} $ is an atlas such that $ \bigcup_{\alpha \in I} O_{\alpha} = \mathcal{M} $. Two atlases are compatible if each chart of one atlas is compatible with every chart of the other: they define the same \textit{differentiable structure} on the manifold.\\
Each chart $ \varphi_{\alpha} $ provides a coordinate system on the region $ O_{\alpha} $: $ \varphi_{\alpha}(p) = \left( x^1(p), \dots, x^{\mu}(p), \dots, x^n(p) \right) $. The \textit{transition functions} $ \varphi_{\beta} \circ \varphi_{\alpha}^{-1} $ are therefore coordinate transformations on overlapping regions.

\begin{example}
  The $ n $-sphere $ \mathbb{S}^n $ is a differentiable manifold.
\end{example}
\begin{example}
  To define a differentiable structure on $ \mathcal{S}^1 $ an atlas of two charts is needed: the standard parametrization $ \theta \in [0, 2\pi) $ is not a well-defined chart because $ [0,2\pi) $ is not an open set in the Euclidean topology of $ \R $, therefore the elimination of a point is necessary; usually, the two charts of the atlas are defined by $ \theta_1 \in (0,2\pi) $, excluding $ (1,0) $ (in the embedding space $ \R^2 $), and $ \theta_2 \in (-\pi,\pi) $, excluding $ (-1,0) $: they are evidently compatible, thus they form a maximal atlas.
\end{example}

\subsection{Maps between manifolds}

Locally mapping $ \mathcal{M} $ to $ \R^n $ allows to import concepts of Analysis from $ \R^n $ to $ \mathcal{M} $.

\begin{definition}
  A function $ f : \mathcal{M} \rightarrow \R $ on a differentiable manifold $ (\mathcal{M},\mathcal{A}) $ is \textit{smooth} if $ f \circ \varphi_{\alpha}^{-1} : U_{\alpha} \rightarrow \R $ is smooth for all charts $ (U_{\alpha},\varphi_{\alpha}) \in \mathcal{A} $.
\end{definition}

\begin{definition}
  A map $ f : \mathcal{M} \rightarrow \mathcal{N} $ between two differentiable manifolds $ (\mathcal{M},\mathcal{A}_1), (\mathcal{N},\mathcal{A}_2) $ is \textit{smooth} if $ \psi_{\alpha_2} \circ f \circ \varphi_{\alpha_1}^{-1} : U_{\alpha_1} \rightarrow V_{\alpha_2} $ is smooth for all charts $ (U_{\alpha_1},\varphi_{\alpha_1}) \in \mathcal{A}_1, (V_{\alpha_2},\varphi_{\alpha_2}) \in \mathcal{A}_2 $.
\end{definition}

\begin{definition}
  A \textit{diffeomorphism} between two differentiable manifolds $ \mathcal{M},\mathcal{N} $ is a smooth homeomorphism $ f : \mathcal{M} \rightarrow \mathcal{N} $.
\end{definition}

\begin{proposition}
  If $ \mathcal{M} $ and $ \mathcal{N} $ are diffeomorphic, then $ \dim_{\R}\mathcal{M} = \dim_{\R}\mathcal{N} $.
\end{proposition}

\begin{example}
  $ \mathbb{S}^7 $ can be covered by multiple incompatible atlases: the resulting manifolds are homeomorphic but not diffeomorphic.
\end{example}

\begin{example}
  $ \R^n $ has a unique differentiable structure for all $ n \in \N $, except for $ n = 4 $: $ \R^4 $ can be covered by infinitely-many incompatible atlases.
\end{example}

\section{Tangent spaces}

The notions of calculus can be defined on a differential manifold $ (\mathcal{M},\mathcal{A}) $ via tangent spaces.

\begin{definition}
  The derivative of a function $ f : \mathcal{M} \rightarrow \R $ at a point $ p \in \mathcal{M} $, covered by the chart $ (\varphi,U) $, is defined as:
  \begin{equation}
    \frac{\pa f}{\pa x^{\mu}}\bigg\vert_p \defeq \frac{\pa (f \circ \varphi^{-1})}{\pa x^{\mu}}\bigg\vert_{\varphi(p)}
    \label{eq:2.1}
  \end{equation}
\end{definition}

Evidently, this definition depends on the choise of coordinates $ x^{\mu} $, thus it depends on the chart.

\subsection{Tangent vectors}

\begin{definition}
  The set of all smooth functions on $ \mathcal{M} $ is denoted by $ \cm $.
\end{definition}

\begin{definition}\label{tang-vec}
  A \textit{tangent vector} to $ \mathcal{M} $ in $ p \in \mathcal{M} $ is an operator $ X_p : \cm \rightarrow \R $ such that:
  \begin{enumerate}
    \item $ X_p(f + g) = X_p(f) + X_p(g) \,\forall f,g \in\cm $;
    \item $ X_p(f) = 0 $ for all constant functions;
    \item $ X_p(fg) = X_p(f)g(p) + f(p)X_p(g) \,\forall f,g \in\cm $.
  \end{enumerate}
\end{definition}

\begin{proposition}
  $ X_p(\alpha f) = \alpha X_p(f) \,\forall \alpha \in \R $.
\end{proposition}
\begin{proof}
  Trivial from conditions 2. and 3. of Def. \ref{tang-vec}.
\end{proof}

It is simple to check that $ \pa_{\mu}\vert_p $ satisfies the conditions of Def. \ref{tang-vec}.

\begin{theorem}
  The set $ T_p\mathcal{M} $ of all tangent vectors at a point $ p\in\mathcal{M} $ forms an $ n $-dimensional space, called \textit{tangent space}, and $ \{\pa_{\mu}\vert_p\}_{\mu = 1,\dots,n} $ is a base of such space.
\end{theorem}
\begin{proof}
  Defining $ f \circ \varphi^{-1} \equiv F : U \subset \mathcal{M} \rightarrow \R $, with $ f : \mathcal{M} \rightarrow \mathcal{M} $ and $ (\varphi,U) \in \mathcal{A} $, it can be proved that, in some neighbourhood of $ p $ (not necessarily $ U $), $ F $ cal always be written as:
  \begin{equation*}
    F(x) = F(x^{\mu}(p)) + \left( x^{\mu} - x^{\mu}(p) \right) F_{\mu}(x)
  \end{equation*}
  for some $ n $ functions $ F_{\mu} $ (ex.: a Taylor series, or more generally $ F(x) = F(0) + x \int_0^1 dt\,F(xt) $). Applying $ \pa_{\mu}\vert_{x(p)} $:
  \begin{equation*}
    \frac{\pa F}{\pa x^{\mu}}\bigg\vert_{x(p)} = F_{\mu}(x(p))
  \end{equation*}
  Defining $ f_{\mu} \equiv F_{\mu} \circ \varphi $, for any $ q \in \mathcal{M} $ in an appropriate neighbourhood of $ p $:
  \begin{equation*}
    f(q) = f(p) + \left( x^{\mu}(q) - x^{\mu}(p) \right) f_{\mu}(q)
  \end{equation*}
  Moreover, remembering Eq. \ref{eq:2.1}:
  \begin{equation*}
    f_{\mu}(p) = F_{\mu} \circ \varphi(p) = F_{\mu}(x(p)) = \frac{\pa F}{\pa x^{\mu}}\bigg\vert_{x(p)} = \frac{\pa f}{\pa x^{\mu}}\bigg\vert_p
  \end{equation*}
  Using these facts, the action of a tangent vector can be written explicitly:
  \begin{equation*}
    \begin{split}
      X_p(f)
      &= X_p\left( f(p) + \left( x^{\mu} - x^{\mu}(p) \right) f_{\mu} \right)\\
      &= X_p\left( f(p) \right) + X_p\left( \left( x^{\mu} - x^{\mu}(p) \right) \right) f_{\mu}(p) + \left( x^{\mu} - x^{\mu}(p) \right)(p) X_p\left( f_{\mu} \right)\\
      &= X_p\left( x^{\mu} \right) f_{\mu}(p)
    \end{split}
  \end{equation*}
  because $ f(p) $ is a constant and $ \left( x^{\mu} - x^{\mu}(p) \right)(p) = x^{\mu}(p) - x^{\mu}(p) = 0 $. Therefore, remembering the expression for $ f_{\mu}(p) $:
  \begin{equation*}
    X_p = X_p(x^{\mu}) \frac{\pa}{\pa x^{\mu}}\bigg\vert_p \equiv X^{\mu} \frac{\pa}{\pa x^{\mu}}\bigg\vert_p
  \end{equation*}
  Thus, $ T_p\mathcal{M} = \lspan\{\pa_{\mu}\vert_p\} $. To check for linear independence, suppose $ \alpha = \alpha^{\mu} \pa_{\mu}\vert_p \equiv 0 $: acting on $ f = x^{\nu} $, it gives $ \alpha(f) = \alpha_{\mu} \pa_{\mu}(x^{\nu})\vert_p = \alpha_{\nu} = 0 $. This concludes the proof.
\end{proof}

\subsubsection{Changing coordinates}

Although $ \pa_{\mu}\vert_p $ depends on the choice of coordinates (it is a \textit{coordinate basis}), the existence of $ X_p $ is independent of that choice.\\
If two different charts $ (\varphi,U),(\tilde{\varphi},V) $ intersect in a neighbourhood of $ p \in U \cap V $, the transition from $ x^{\mu} $ to $ y^{\mu} $ can be expressed as:
\begin{equation}
  X_p(f) = X^{\mu} \frac{\pa f}{\pa x^{\mu}}\bigg\vert_p = X^{\mu} \frac{\pa y^{\nu}}{\pa x^{\mu}}\bigg\vert_{\varphi(p)} \frac{\pa f}{\pa y^{\nu}}\bigg\vert_p
\end{equation}
This equation can have two interpretations: the alibi interpretation:
\begin{equation}
  \frac{\pa}{\pa x^{\mu}}\bigg\vert_p = \frac{\pa y^{\nu}}{\pa x^{\mu}}\bigg\vert_{\varphi(p)} \frac{\pa}{\pa y^{\nu}}\bigg\vert_p
  \label{eq:2.3}
\end{equation}
and the alias interpretation:
\begin{equation}
  \tilde{X}^{\nu} = X^{\mu} \frac{\pa y^{\nu}}{\pa x^{\mu}}\bigg\vert_{\varphi(p)}
  \label{eq:2.4}
\end{equation}
Components of vectors which transform this way are called \textit{contravariant}.

\subsubsection{Curves}

Consider a smooth curve on $ \mathcal{M} $, i.e. a smooth map $ \sigma : I \in \mathcal{T}(\R) \rightarrow \mathcal{M} $, parametrized as $ \sigma(t) : \sigma(0) = p \in \mathcal{M} $; with a given chart $ (\varphi,U) $, this curve becomes $ \varphi \circ \sigma : I \rightarrow \R^n $, parametrized by $ x^{\mu}(t) $.
The \textit{tangent vector} to the curve in $ p $ is:
\begin{equation}
  X_p = \frac{dx^{\mu}(t)}{dt}\bigg\vert_{t=0} \frac{\pa}{\pa x^{\mu}}\bigg\vert_p
  \label{eq:2.5}
\end{equation}
This operator, applied to a function $ f \in\cm $, calculates the directional derivative of $ f $ along the curve. It can be showed that every tangent vector can be written as in Eq. \ref{eq:2.5}, therefore the tangent space is literally the space of all possible tangents to curves passing through $ p $.\\
It must be noted that tangent spaces at different points are entirely different spaces: there's no way to directly compare vectors between them.

\subsection{Vector fields}

\begin{definition}
  A \textit{vector field} $ X $ is a smooth map $ X : p \in \mathcal{M} \mapsto X_p \in T_p\mathcal{M} $. It can also be viewed as a smooth map $ X : \cm \rightarrow \cm $, as $ (X(f))(p) = X_p(f) \in \R $.
\end{definition}

\begin{definition}
  The space of all vector fields on $ \mathcal{M} $ is denoted by $ \xm $.
\end{definition}

Given a chart $ (\varphi,U) $, a vector field $ X $ can be expressed as:
\begin{equation}
  X = X^{\mu} \frac{\pa}{\pa x^{\mu}}
  \label{eq:2.6}
\end{equation}
with $ X^{\mu} \in \cm $. This expression is only defined on $ U $.

\subsubsection{Lie brakets}

Given two vector fields $ X,Y \in\xm $, their product is clearly not a vector field, as it does not satisfy Leibniz rule:
\begin{equation*}
  XY(fg) = XY(f) g + Y(f) X(g) + X(f) Y(g) + f XY(g) \neq XY(f) g + f XY(g)
\end{equation*}
where $ XY(f) \equiv X(Y(f)) $.

\begin{definition}
  Given two vector fields $ X,Y \in\xm $, their \textit{commutator} (or \textit{Lie bracket}) is defined as:
  \begin{equation}
    \left[ X,Y \right](f) = XY(f) - YX(f)
    \label{eq:2.7}
  \end{equation}
\end{definition}

With a given chart:
\begin{equation*}
  \begin{split}
    \left[ X,Y \right](f)
    &= X^{\mu} \frac{\pa}{\pa x^{\mu}} \left( Y^{\nu} \frac{\pa f}{\pa x^{\nu}} \right) - Y^{\mu} \frac{\pa}{\pa x^{\mu}} \left( X^{\nu} \frac{\pa f}{\pa x^{\nu}} \right)\\
    &= \left( X^{\mu} \frac{\pa Y^{\nu}}{\pa x^{\mu}} - Y^{\mu} \frac{\pa X^{\nu}}{\pa x^{\mu}} \right) \frac{\pa f}{\pa x^{\nu}}
  \end{split}
\end{equation*}
therefore:
\begin{equation}
  \left[ X,Y \right] = \left( X^{\mu} \frac{\pa Y^{\nu}}{\pa x^{\mu}} - Y^{\mu} \frac{\pa X^{\nu}}{\pa x^{\mu}} \right) \frac{\pa}{\pa x^{\nu}}
  \label{eq:2.8}
\end{equation}

\begin{theorem}[Jacobi]
  Given $ X,Y,Z \in\xm $, the \textit{Jacobi identity} holds:
  \begin{equation}
    \left[ X, \left[ Y,Z \right] \right] + \left[ Y, \left[ Z,X \right] \right] + \left[ Z, \left[ X,Y \right] \right] = 0
    \label{eq:2.9}
  \end{equation}
\end{theorem}

\begin{proposition}
  $ \xm $ is a \textit{Lie algebra}.
\end{proposition}

\subsubsection{Integral curves}

\begin{definition}
  A \textit{flow} on $ \mathcal{M} $ is a one-parameter family of diffeomorphisms $ \sigma_t : \mathcal{M} \rightarrow \mathcal{M} $, labelled by $ t\in\R $, with group structure: $ \sigma_0 = \id_{\mathcal{M}} $ and $ \sigma_s \circ \sigma_t = \sigma_{s+t} $, thus $ \sigma_{-t} = \sigma_t^{-1} $.
\end{definition}

Such flows give rise to streamlines on the manifold: these streamlines are required to be smooth.
Defining $ x^{\mu}(\sigma_t) \equiv x^{\mu}(t) $, a vector field can be defined by the tangent to the streamlines at each point on the manifold:
\begin{equation}
  X^{\mu}(x^{\mu}(t)) = \frac{dx^{\mu}(t)}{dt}
  \label{eq:2.10}
\end{equation}
The inverse reasoning is also possible.

\begin{definition}
  Given a vector field $ X \in\xm $, streamlines described by Eq. \ref{eq:2.10} are called \textit{integral curves} generated by $ X $.
\end{definition}

\begin{proposition}
  The \textit{infinitesimal flow} generated by $ X \in\xm $ is:
  \begin{equation}
    x^{\mu}(t) = x^{\mu}(0) + tX^{\mu}(x(t)) + o(t)
    \label{eq:2.11}
  \end{equation}
\end{proposition}

\begin{definition}
  A vector field which generates a flow defined for all $ t \in \R $ is called \textit{complete}.
\end{definition}

\begin{theorem}
  If $ \mathcal{M} $ is compact, then all $ X \in \xm $ are complete.
\end{theorem}

\begin{example}
  On $ \mathbb{S}^2 $, the flow generated by $ X = \pa_{\phi} $ is described by $ \dot{\phi} = 1, \dot{\theta} = 0 $, thus $ \theta(t) = \theta_0 $ and $ \phi(t) = \phi_0 + t $: the flow lines are lines of constant latitude.
\end{example}

\subsection{Lie derivative}

Defining calculus for vector fields requires a way to compare vectors of different tangent spaces.

\begin{definition}
  Given a diffeomorphism between two manifolds $ \varphi : \mathcal{M} \rightarrow \mathcal{N} $ and a function $ f : \mathcal{N} \rightarrow \R $, the \textit{pull-back} of $ f $ is the function $ \varphi^*f : \mathcal{M} \rightarrow \R $ such that $ \varphi^*f(p) = f(\varphi(p)) $.
\end{definition}

\begin{definition}
  Given a diffeomorphism between two manifolds $ \varphi : \mathcal{M} \rightarrow \mathcal{N} $ and a vector field $ X \in \xm $, the \textit{push-forward} of $ X $ is the vector field $ \varphi_*X \in \mathfrak{X}(\mathcal{N}) $ such that $ \varphi_*X(f) = X(\varphi^*f) $.
\end{definition}

This last equality must be evaluated at the appropriate points: $ [ \varphi_*X(f) ](\varphi(p)) = [ X(\varphi^*f) ](p) $.\\
With the appropriate charts on $ \mathcal{M} $ and $ \mathcal{N} $, the definitions above can be rewritten with coordinates:
\begin{equation}
  \varphi^*f(x) = f(y(x))
  \label{eq:2.12}
\end{equation}
\begin{equation}
  \varphi_*X(f) = X^{\mu} \frac{\pa f(y(x))}{\pa x^{\mu}} = X^{\mu} \frac{\pa y^{\alpha}}{\pa x^{\mu}} \frac{\pa f(y)}{\pa y^{\alpha}}
  \label{eq:2.13}
\end{equation}

The notions of pull-back and push-forward allow to compare tangent vectors at neighbouring points and, in particular, to define the derivative along a vector field.

\begin{definition}
  Given a function $ f : \mathcal{M} \rightarrow \R $ and a vector field $ X \in\xm $, the derivative of $ f $ along $ X $ (called \textit{Lie derivative}) is defined as:
  \begin{equation}
    \ld_X f(x) \defeq \lim_{t \rightarrow 0} \frac{f(\sigma_t(x)) - f(x)}{t} = \frac{df(\sigma_t(x))}{dt}\bigg\vert_{t=0}
    \label{eq:2.14}
  \end{equation}
  where $ \sigma_t $ is the flow generated by $ X $.
\end{definition}

\begin{proposition}
  $ \ld_X f = X(f) $.
\end{proposition}
\begin{proof}
  $ \ld_X f = \frac{df(\sigma_t)}{dt} = \frac{\pa f}{\pa x^{\mu}} \frac{dx^{\mu}(t)}{dt} = X^{\mu} \frac{\pa f}{\pa x^{\mu}} = X(f) $.
\end{proof}

\begin{definition}
  Given two vector fields $ X,Y \in\xm $, the \textit{Lie derivative} of $ Y $ along $ X $ is defined as:
  \begin{equation}
    \ld_X Y_p \defeq \lim_{t \rightarrow 0} \frac{((\sigma_{-t})_*Y)_p - Y_p}{t}
    \label{eq:2.15}
  \end{equation}
  where $ \sigma_t $ is the flow generated by $ X $.
\end{definition}

The use of the inverse flow $ \sigma_{-t} $ is necessary because to evaluate the vector field $ \ld_X Y $ at the point $ p \in \mathcal{M} $, the tangent vector $ Y_{\sigma_t(p)} \in T_{\sigma_t(p)}\mathcal{M} $ must be $ \virgolette{pushed-back} $ to $ T_p\mathcal{M} = T_{\sigma_0(p)}\mathcal{M} $.\\
With $ t \rightarrow 0 $, the infinitesimal flow $ \sigma_{-t} $ is, according to Eq. \ref{eq:2.11}, $ x^{\mu}(t) = x^{\mu}(0) - tX^{\mu} + o(t) $, therefore the Lie derivative of base tangent vectors can be expressed as:
\begin{equation}
  (\sigma_{-t})_* \pa_{\mu} = \frac{\pa x^{\nu}(t)}{\pa x^{\mu}} \frac{\pa}{\pa x^{\nu}(t)} = \left( \delta^{\nu}_{\mu} - t \frac{\pa X^{\nu}}{\pa x^{\mu}} + o(t) \right) \pa_{\nu}(t)
  \quad\Longrightarrow\quad
  \ld_X \pa_{\mu} = - \frac{\pa X^{\nu}}{\pa x^{\mu}} \pa_{\nu}
  \label{eq:2.16}
\end{equation}

\begin{proposition}
  $ \ld_X Y = \left[ X,Y \right] $.
\end{proposition}
\begin{proof}
  $ \ld_X Y = \ld_X (Y^{\mu} \pa_{\mu}) = \left( \ld_X Y^{\mu} \right) \pa_{\mu} + Y^{\mu} \left( \ld_X \pa_{\mu} \right) = X^{\nu} \frac{\pa Y^{\mu}}{\pa x^{\nu}} \pa_{\mu} - Y^{\mu} \frac{\pa X^{\nu}}{\pa x^{\mu}} \pa_{\nu} = \left[ X,Y \right] $.
\end{proof}
\begin{proposition}
  $ \ld_X \ld_Y Z - \ld_Y \ld_X Z = \ld_{\left[ X,Y \right]} Z $.
\end{proposition}
\begin{proof}
  Trivial with Jacobi identity.
\end{proof}

\section{Tensors}

\subsection{Dual Spaces}

\begin{definition}
  Given a vector space $ V $, its \textit{dual} $ V^* $ is the space of all linear maps $ f : V \rightarrow \R $.
\end{definition}

Given a basis $ \{\ve{e}_{\mu}\}_{\mu = 1,\dots,n} $ of $ V $, its \textit{dual basis} $ \{\ve{f}^{\mu}\}_{\mu=1,\dots,n} $ of $ V^* $ can be defined by:
\begin{equation}
  \ve{f}^{\nu}(\ve{e}_{\mu}) = \delta^{\nu}_{\mu}
  \label{eq:2.17}
\end{equation}
A general vector in $ V $ can be written as $ X = X^{\mu} \ve{e}_{\mu} $, thus according to Eq. \ref{eq:2.17} $ X^{\mu} = \ve{f}^{\mu}(X) $.

\begin{proposition}
  The map $ f : \ve{e}_{\mu} \mapsto \ve{f}^{\mu} $ is an isomorphism between $ V $ and $ V^* $.
\end{proposition}

This isomorphism, however, is basis-dependent.

\begin{proposition}
  $ \dim_{\R}V = \dim_{\R}V^* $.
\end{proposition}

\begin{proposition}
  $ (V^*)^* = V $.
\end{proposition}
\begin{proof}
  The natural isomorphism between $ (V^*)^* $ and $ V $ is basis-independent: suppose $ X \in V $ and $ \omega \in V^* $, so that $ \omega(X) \in \R $; $ X $ can be viewed as $ X \in (V^*)^* $ by setting $ V(\omega) \equiv \omega(V) $.
\end{proof}

\subsection{Cotangent vectors}

\begin{definition}
  Given a differentiable manifold $ (\mathcal{M},\mathcal{A}) $ and a point $ p \in \mathcal{M} $, the \textit{cotangent space} to $ \mathcal{M} $ at $ p $ is defined as $ T^*_p\mathcal{M} \defeq (T_p\mathcal{M})^* $.
\end{definition}

Elements of $ T^*_p\mathcal{M} $ are called \textit{cotangent vectors} (or \textit{covectors}).

\begin{definition}
  A \textit{covector field} (or \textit{1-form}) is a smooth map $ \omega : p \in \mathcal{M} \mapsto \omega_p \in T^*_p\mathcal{M} $. It can also be viewed as a smooth map $ \omega : \xm \rightarrow \cm $, as $ (\omega(X))(p) = \omega_p(X_p) \in \R $.
\end{definition}

\begin{definition}
  The space of all 1-forms on $ \mathcal{M} $ is denoted by $ \lm{1} $.
\end{definition}

\begin{proposition}
  $ \{dx^{\mu}\}_{\mu = 1,\dots,n} $ is a basis of $ \lm{1} $ dual to the basis $ \{\pa_{\mu}\}_{\mu = 1,\dots,n} $ of $ \xm $.
\end{proposition}
\begin{proof}
  Consider $ f \in \cm $ and define $ df \in \lm{1} $ by $ df(X) = X(f) $: taking $ f = x^{\mu} $ and $ X = \pa_{\mu} $, $ df(X) = \pa_{\nu}(x^{\mu}) = \delta^{\mu}_{\nu} $, therefore $ \{dx^{\mu}\}_{\mu = 1,\dots,n} $ is the dual basis of $ \lm{1} $.
\end{proof}

This is also confirmed by $ df = \frac{\pa f}{\pa x^{\mu}} dx^{\mu} $. These are coordinate basis: in fact, given two different charts $ (\varphi,U), (\tilde{\varphi},V) $:
\begin{equation}
  dy^{\mu} = \frac{dy^{\mu}}{dx^{\nu}}dx^{\nu}
  \label{eq:}
\end{equation}
which is the inverse of Eq. \ref{eq:2.3} (not evaluated at a specific point). This ensures that:
\begin{equation*}
  dy^{\mu}\left( \frac{\pa}{\pa y^{\nu}} \right) = \frac{\pa y^{\mu}}{\pa x^{\alpha}} \frac{\pa x^{\beta}}{\pa y^{\nu}} dx^{\alpha}\left( \frac{\pa}{\pa x^{\beta}} \right) = \frac{\pa y^{\mu}}{\pa x^{\alpha}} \frac{\pa x^{\alpha}}{\pa y^{\nu}} = \delta^{\mu}_{\nu}
\end{equation*}
A 1-form $ \omega \in \lm{1} $ can thus be expressed both as $ \omega = \omega_{\mu} dx^{\mu} = \tilde{\omega}_{\mu} dx^{\mu} $, with:
\begin{equation}
  \tilde{\omega}_{\omega} = \frac{\pa x^{\nu}}{\pa y^{\mu}} \omega_{\nu}
  \label{eq:2.19}
\end{equation}
Components of 1-forms which transform this way are called \textit{covariant}.

\begin{definition}
  Given a diffeomorphism between two manifolds $ \varphi : \mathcal{M} \rightarrow \mathcal{N} $ and a 1-form $ \omega \in \Lambda^1(\mathcal{N}) $, the \textit{pull-back} of $ \omega $ is the 1-form $ \varphi^*\omega \in \lm{1} $ such that $ \varphi^*\omega(X) = \omega(\varphi_*X) $.
\end{definition}

With the appropriate charts on $ \mathcal{M} $ and $ \mathcal{N} $, the definition above can be rewritten with coordiantes:
\begin{equation}
  \varphi^*\omega = \omega_{\alpha} \frac{\pa y^{\alpha}}{\pa x^{\mu}} dx^{\mu}
  \label{eq:2.20}
\end{equation}

\begin{definition}
  Given a vector field $ X \in \xm $ and a 1-form $ \omega \in \lm{1} $, the \textit{Lie derivative} of $ \omega $ along $ X $ is defined as:
  \begin{equation}
    \ld_X \omega \defeq_{t \rightarrow 0} \frac{(\sigma_t^* \omega)_p - \omega_p}{t}
    \label{eq:2.21}
  \end{equation}
  where $ \sigma_t $ is the flow generated by $ X $.
\end{definition}

In contranst with the Lie derivative of a vector field, which pushes forward with $ \sigma_{-t} $ (i.e. pushes back), the Lie derivative of a 1-form pulls back with $ \sigma_t $: this results in the difference of a minus sign with respect to Eq. \ref{eq:2.16}, giving:
\begin{equation}
  \ld_X dx^{\mu} = \frac{\pa X^{\mu}}{\pa x^{\nu}} dx^{\nu}
  \label{eq:2.22}
\end{equation}
Therefore, on a general 1-form $ \omega = \omega_{\mu} dx^{\mu} $:
\begin{equation}
  \ld_X \omega = \left( X^{\nu} \pa_{\nu} \omega_{\mu} + \omega_{\nu} \pa_{\mu} X^{\nu} \right) dx^{\mu}
  \label{eq:2.23}
\end{equation}

\subsection{Tensor fields}










