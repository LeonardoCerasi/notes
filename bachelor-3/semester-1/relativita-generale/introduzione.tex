\selectlanguage{italian}

Nello studio delle interazioni a distanza si introducono le cosiddette teorie di campo: un campo è un'entità fisica che esiste in ogni punto dello spaziotempo (es: campo elettrico, magnetico, etc...) e che viene modificata dalla presenza di portatori della carica associata al campo.\\
Nel caso del di una teoria di campo per descrivere la gravità, è necessario un campo gravitazionale che sia influenzato dalla massa. Nel caso Newtoniano il campo gravitazionale $ \Phi(\ve{r},t) $ è legato alla densità di massa $ \rho(\ve{r},t) $ da un'equazione di Poisson:
\begin{equation}
  \lap\Phi = 4\pi G \rho
  \label{eq:0.1}
\end{equation}
dove $ G \approx 6.67\cdot10^{-11}\m^3\text{kg}^{-1}\text{s}^{-2} $ è la costante universale di Newton.
