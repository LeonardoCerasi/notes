\selectlanguage{italian}

Nello studio delle interazioni a distanza si introducono le cosiddette teorie di campo: un campo è un'entità fisica che esiste in ogni punto dello spaziotempo (es: campo elettrico, magnetico, etc...) e che viene modificata dalla presenza di portatori della carica associata al campo.\\
Nel caso del di una teoria di campo per descrivere la gravità, è necessario un campo gravitazionale che sia influenzato dalla massa. Nel caso Newtoniano il campo gravitazionale $ \Phi(\ve{r},t) $ è legato alla densità di massa $ \rho(\ve{r},t) $ da un'equazione di Poisson:
\begin{equation}
  \lap\Phi = 4\pi G \rho
  \label{eq:0.1}
\end{equation}
dove $ G \approx 6.67\cdot10^{-11}\m^3\text{kg}^{-1}\text{s}^{-2} $ è la costante universale di Newton.\\
È banale ricavare il campo gravitazionale di una massa puntiforme $ M $: in questo caso $ \rho(\ve{r}) = M \delta^3(\ve{r}) $, dunque $ \Phi(\ve{r}) = - \frac{GM}{r} $. Il caso in cui invece $ \rho $ dipende dal tempo è non banale e per essere trattato necessita di un'equazione più generale di Eq. \ref{eq:0.1}: l'equazione di campo di Einstein.

\paragraph{Analogie e differenze con l'Elettromagnetismo}

Superficialmente, il problema della generalizzazione relativistica della gravitazione potrebbe sembrare analogo a quello dell'elettromagnetismo: entrambe le forze, nel caso stazionario, sono governate da una legge proporzionale a $ r^{-2} $ ed entrambi i campi sono determinati da equazioni di Poisson in cui cambia solo la costante dimensionale.\\
La differenza tra le due teorie di campo, però, sta proprio nella descrizione matematica delle sorgenti che subentrano nelle equazioni di Poisson: nel caso dell'Elettromagnetismo, in regime stazionario il campo elettromagnetico è determinato dalla densità di carica $ \rho_e $ e dalla densità di corrente $ \ve{J} $, e per avere una descrizione relativistica bisogna combinarle in una densità di corrente quadrivettoriale $ j^{\mu} = \left( c\rho_e,\ve{J} \right)$ (si può vedere che $ \rho_e $ trasforma come una componente temporale poiché $ \rho_e \sim \text{Vol}_3^{-1} \sim \left( \text{Vol}_4 / ct \right)^{-1} \sim ct $, dato che il quadrivolume è un invariante di Lorentz): ciò risulta naturalmente in un potenziale quadrivettoriale $ A^{\mu} = \left( \phi/c, \ve{A} \right) $.\\
D'altra parte, per quanto riguarda la gravitazione, bisogna ricordare l'uguaglianza relativistica tra massa energia; inoltre, a differenza della carica elettrica, l'energia non è un invariante relativistico, ma è la componente temporale del quadrivettore impulso: in particolare, a generare il campo gravitazionale sono la densità di energia $ \rho $ e la densità di momento $ \ve{\rho}_p $, alle quali sono associate una densità di corrente di energia $ \ve{j} $ ed una densità di corrente di momento $ \ve{T}^i $ per ciascuna componente $ \rho_p^i $. Risulta evidente che l'equazione relativistica che descrive il campo gravitazionale sia notevolmente più complicata di quella del campo elettromagnetico, poiché le sorgenti non sono descritte da un quadrivettore, bensì da un tensore, il tensore energia-impulso:
\begin{equation}
  T^{\mu \nu} \sim
  \begin{bmatrix}
    \rho c & \ve{\rho}_p c \\
    \ve{j} & \tens{T}
  \end{bmatrix}
  \label{eq:0.2}
\end{equation}
Naturalmente, anche il potenziale gravitazionale sarà un tensore $ h_{\mu \nu} $, ed il potenziale Newtoniano sarà $ h_{00} \sim \Phi $.

\paragraph{Scala della Relatività Generale}

Tramite le costanti fondamentali $ G $ e $ c $ è possibile associare ad una massa $ M $ una sua lunghezza caratteristica, detta raggio di Schwarzschild:
\begin{equation}
  R_s \defeq \frac{2GM}{c^2}
  \label{eq:0.3}
\end{equation}
Le correzioni relativistiche alla teoria della gravitazione sono determinate dal parametro $ R_s / r $ e, nella maggior parte delle situazioni, sono trascurabili: basti calcolare che per la Terra $ R_s \approx 10^{-2}\m $, mentre il suo raggio è $ R_T = 6\cdot10^6\m $, dunque sulla superficie terrestre le correzioni relativistiche alla gravità Newtoniana sono dell'ordine di $ 10^{-8} $.\\
Gli effetti relativistici diventano importanti quando si considerano oggetti compatti come stelle di neutroni e buchi neri.










