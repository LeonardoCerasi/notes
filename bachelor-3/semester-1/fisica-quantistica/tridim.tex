\selectlanguage{italian}

\section{Equazione di Schrödinger radiale}

Si consideri una generica Hamiltoniana invariante per rotazioni, ad esempio:
\begin{equation}
	\mathcal{H} = \frac{
	p^2}{2m} + V(r) = \frac{p_r^2}{2m} + \frac{L^2}{2mr^2} + V(r)
	\label{eq:3.1}
\end{equation}
È evidente che questa Hamiltoniana non si possa separare in parte radiale e parte angolare a causa del termine $ \frac{L^2}{r^2} $. È però possibile diagonalizzare simultaneamente $ \hat{H} $, $ \hat{L}^2 $ e $ \hat{L}_z $, dunque si proietta sugli autostati del momento angolare:
\begin{equation}
	\psi(\ve{x}) = \sum_{\ell = 0}^{\infty} \sum_{m = -\ell}^{\ell} \braket{\ve{x} | \ell,m} \braket{\ell,m | \psi} = \sum_{\ell = 0}^{\infty} \sum_{m = -\ell}^{\ell} Y_{\ell,m}(\vartheta,\varphi) \phi_{\ell,m}(r)
	\label{eq:3.2}
\end{equation}
L'equazione di Schrödinger si riduce quindi in una PDE con una sola incognita:
\begin{equation}
	\left[ \frac{p_r^2}{2m} + \frac{\hbar^2 \ell (\ell + 1)}{2mr^2} + V(r) \right] \phi_{\ell,m}(r) = E \phi_{\ell,m}(r)
	\label{eq:3.3}
\end{equation}
Questa non dipende da $ m $, dunque fissati $ \ell $ ed $ E $ c'è una degerazione di $ 2\ell + 1 $; si pone $ \phi_{\ell,m}(r) \equiv \phi_{\ell}(r) $.
È inoltre utile porre:
\begin{equation}
	\phi_{\ell}(r) \equiv \frac{u_{\ell}(r)}{r}
	\label{eq:3.4}
\end{equation}

\begin{proposition}
	$ \hat{p}_r^n \phi_{\ell}(r) = \left( -i\hbar \right)^n \frac{1}{r} \frac{\pa^n}{\pa r^n} u_{\ell}(r) $.
\end{proposition}
\begin{proof}
	$ \hat{p}_r \phi_{\ell}(r) = -i\hbar \left( \frac{\pa}{\pa r} + \frac{1}{r} \right) \frac{u_{\ell}(r)}{r} = -i\hbar \frac{1}{r} \frac{\pa}{\pa r} u_{\ell}(r) $.
\end{proof}
Una ragione $ \virgolette{fisica} $ per definire $ u_{\ell}(r) $ è che assorbe la misura d'integrazione nel prodotto scalare:
\begin{equation*}
	\braket{\psi' | \psi} = \int_{0}^{\infty} dr \,r^2 \phi'^*_{\ell'}(r) \phi_{\ell}(r) \int_{\mathbb{S}^2} d\cos\vartheta \,d\varphi \, Y^*_{\ell',m'}(\vartheta,\varphi) Y_{\ell,m}(\vartheta,\varphi) = \delta_{\ell,\ell'} \delta_{m,m'} \int_0^{\infty} dr \, u'^*_{\ell'}(r) u_{\ell}(r)
\end{equation*}
Ciò rende $ \hat{p}_r $ un operatore hermitiano sulle $ u_{\ell} $, ed infatti l'equazione di Schrödinger diventa:
\begin{equation}
	\left[ - \frac{\hbar^2}{2m} + \frac{\hbar^2 \ell (\ell + 1)}{2m r^2} + V(r) \right] u_{\ell}(r) = E u_{\ell}(r)
	\label{eq:3.5}
\end{equation}

\subsection{Condizioni al contorno}

È necessario che la funzione d'onda radiale $ \phi_{\ell}(r) $ abbia densità di probabilità integrabile su $ [0,+\infty) $: in particolare, si richiede che il seguente integrale non diverga:
\begin{equation}
	\braket{\phi_{\ell} | \phi_{\ell}} = \int_0^{\infty} dr \, r^2 \abs{\phi_{\ell}(r)}^2 = \int_{0}^{\infty} dr  \abs{u_{\ell}(r)}^2
	\label{eq:3.6}
\end{equation}
Nell'origine $ \abs{u_{\ell}(r)}^2 $ deve avere al più una singolarità integrabile, dunque:
\begin{equation}
	u_{\ell}(r) \overset{r \rightarrow 0}{\sim} \frac{1}{r^{\delta}} : \delta < \frac{1}{2}
	\label{eq:3.7}
\end{equation}
Dato che $ \lap \frac{1}{r} = - 4\pi \delta^3 (\ve{x}) $, se $ \phi_{\ell}(r) $ diverge nell'origine almeno come $ \frac{1}{r} $ è possibile soddisfare l'equazione di Schrödinger solo se il potenziale nell'origine diverge almeno come una delta di Dirac. Per potenziali non-distribuzionali (ovvero funzioni, quindi non singolari) $ \phi_{\ell}(r) $ deve divergere nell'origine meno di $ \frac{1}{r} $, dunque:
\begin{equation}
	\lim_{r \rightarrow 0} r \phi_{\ell}(r) = 0 \quad\Longrightarrow\quad \lim_{r \rightarrow 0} u_{\ell}(r) = 0
	\label{eq:3.8}
\end{equation}

\paragraph{Andamento nell'origine}

I potenziali d'interesse fisico sono quelli che per $ r \rightarrow 0 $ divergono meno di $ \frac{1}{r^2} $: il caso in cui nell'origine $ V(r) $ vada come $ \frac{1}{r^k} $ con $ k \ge 2 $ è patologicamente attrattivo e si dimostra non avere un'energia minima, ovvero non presenta stati stabili.\\
Se quindi nell'origine $ V(r) \sim \frac{1}{r^k} $ con $ k < 2 $, per $ r \rightarrow 0 $ a dominare è il termine centrifugo:
\begin{equation}
	- \frac{\hbar^2}{2m} \frac{d^2 u_{\ell}(r)}{d r^2} + \frac{\hbar^2 \ell (\ell + 1)}{2mr^2} u_{\ell}(r) = 0 \quad \Longrightarrow \quad \frac{d^2 u_{\ell}(r)}{d r^2} = \frac{\ell (\ell + 1)}{r^2} u_{\ell}(r)
	\label{eq:3.9}
\end{equation}
La soluzione generale di questa equazione è $ u_{\ell}(r) = A r^{\ell + 1} + B r^{-\ell} $, ma il secondo termine non soddisfa la condizione in Eq. \ref{eq:3.8}, dunque:
\begin{equation}
	u_{\ell}(r) \overset{r \rightarrow 0}{\sim} A r^{\ell + 1}
	\label{eq:3.10}
\end{equation}

\paragraph{Andamento all'infinito}

Se all'infinito il potenziale si annulla, per $ r \rightarrow \infty $ l'andamento della funzione d'onda è quello della particella libera: se esistono stati legati, ovvero autostati di $ \hat{\mathcal{H}} $ con $ E < 0 $, l'andamento della soluzione è:
\begin{equation}
	u_{\ell}(r) \overset{r \rightarrow \infty}{\sim} C e^{- \beta r}, \,\, \beta \equiv \frac{\sqrt{2m\abs{E}}}{\hbar}
	\label{eq:3.11}
\end{equation}
Se invece $ \lim_{r \rightarrow \infty} V(r) \neq 0 $, l'andamento va studiato caso per caso.

\paragraph{Stati legati}

Per un potenziale unidimensionale esiste sempre almeno uno stato legato; inoltre, lo stato fondamentale ha una funzione d'onda pari e gli stati eccitati hanno parità alternata.\\
Nel caso tridimensionale, è possibile interpretare il problema come un problema unidimensionale con dominio $ [0,\infty) $: la condizione in Eq. \ref{eq:3.8}, però, impone che solo le soluzioni dispari sono accettabili, dunque in generale non è detto che esista lo stato fondamentale. Per un potenziale tridimensionale, quindi, non è detto a priori che esistano stati legati.

\section{Particella libera}

L'equazione di Schrödinger per la particella libera, quindi per $ V(\ve{x}) = 0 $, è:
\begin{equation}
	-\frac{\hbar^2}{2m} \lap \psi(\ve{x}) = E \psi(\ve{x})
	\label{eq:3.12}
\end{equation}
Questa è una PDE separabile e, in coordinate cartesiane, le soluzioni sono delle onde piane:
\begin{equation}
	\psi_{\ve{k}}(\ve{x}) = \frac{1}{(2\pi)^{3/2}} e^{i \ve{k}\cdot\ve{x}}, \,\, E = \frac{\hbar^2 \ve{k}^2}{2m}
	\label{eq:3.13}
\end{equation}
normalizzate in senso improprio:
\begin{equation}
	\braket{\psi_{\ve{k}'} | \psi_{\ve{k}}} = \int_{\R^3} d^3\ve{x}\, \psi^*_{\ve{k}'}(\ve{x}) \psi_{\ve{k}}(\ve{x}) = \delta^3(\ve{k} - \ve{k}')
	\label{eq:3.14}
\end{equation}
Un differente approccio risolutivo è quello che sfrutta la simmetria rotazionale del problema:
\begin{equation}
	\psi_{\ell,m}(\ve{x}) = Y_{\ell,m}(\vartheta,\varphi) \frac{u_{\ell}(r)}{r}
	\label{eq:3.15}
\end{equation}
Dall'Eq. \ref{eq:3.5}:
\begin{equation}
	\frac{\hbar^2}{2mE} \left[ -\frac{d^2}{dr^2}u_{\ell}(r) + \frac{\ell(\ell + 1)}{r^2}u_{\ell}(r) \right] - u_{\ell}(r) = 0
	\label{eq:3.16}
\end{equation}
Ponendo $ k = \sqrt{\frac{2mE}{\hbar^2}} $ e $ r' = kr $, si ottiene:
\begin{equation}
	\frac{d^2}{dr'^2} u_{\ell}(r') - \frac{\ell(\ell + 1)}{r'^2} u_{\ell}(r') + u_{\ell}(r') = 0
	\label{eq:3.17}
\end{equation}
Questa è una ODE di Bessel e la sua soluzione si ottiene introducendo la funzione di Bessel $ j_{\ell}(r') $ (è noto che $ j_{\ell}(x) \sim x^{\ell} $ per $ x \rightarrow 0 $):
\begin{equation}
	u_{\ell}(r') = r j_{\ell}(r')
	\label{eq:3.18}
\end{equation}
Le autofunzioni della particella libera tridimensionale sono dunque:
\begin{equation}
	\psi_{\ell,m}(\ve{x}) = Y_{\ell,m}(\vartheta,\varphi) j_{\ell}(kr), \,\, k = \frac{\sqrt{2mE}}{\hbar}
	\label{eq:3.19}
\end{equation}

\section{Oscillatore armonico isotropo}

L'oscillatore armonico isotropo è descritto dal seguente potenziale:
\begin{equation}
	V(r) = \frac{1}{2} m\omega^2 r^2
	\label{eq:3.20}
\end{equation}
Essendo un potenziale centrale, il problema si riduce al problema radiale.
L'Hamiltoniana radiale corrispondente è:
\begin{equation}
	\mathcal{H}_{\ell} = \frac{p_r^2}{2m} + \frac{\hbar^2 \ell(\ell + 1)}{2mr^2} + \frac{1}{2} m\omega^2 r^2
	\label{eq:3.21}
\end{equation}
La sua equazione agli autovalori può essere scritta come:
\begin{equation}
	\hat{\mathcal{H}}_{\ell} \ket{k,\ell,m} = E_{k,\ell} \ket{k,\ell,m}
	\label{eq:3.22}
\end{equation}

\subsection{Stati con \texorpdfstring{$ \ell = 0 $}{TEXT}}

Per $ \ell = 0 $, il problema si riduce all'oscillatore armonico unidimensionale:
\begin{equation}
	\mathcal{H}_0 = \frac{p_r^2}{2m} + \frac{1}{2} m \omega^2 r^2
	\label{eq:3.23}
\end{equation}
Per risolvere con metodo algebrico, si defisce l'operatore di distruzione come:
\begin{equation}
	\hat{d}_0 \defeq \sqrt{\frac{m\omega}{2\hbar}} \left( \hat{r} + i \frac{\hat{p}_r}{m\omega} \right)
	\label{eq:3.24}
\end{equation}
ed il relativo operatore di creazione $ d_0^{\dagger} $. In questo modo, si può scrivere:
\begin{equation}
	\hat{\mathcal{H}}_0 = \hbar \omega \left( \hat{d}_0^{\dagger}\hat{d}_0 + \frac{1}{2} \right)
	\label{eq:3.25}
\end{equation}
Dato che $ \left[ \hat{r},\hat{p}_r \right] = i\hbar $, si ha $ \left[ \hat{d}_0, \hat{d}_0^{\dagger} \right] = 1 $, dunque lo spettro di $ \hat{\mathcal{H}}_0 $ è lo stesso dell'oscillatore armonico unidimensionale:
\begin{equation}
	u_{k,0}(r) = \mathcal{N}_k e^{-c x^2} H_k(r)
	\label{eq:3.26}
\end{equation}
Dato che $ u_{k,0}(-x) = (-1)^k u_{k,0}(x) $, per la condizione al contorno in Eq. \ref{eq:3.8} sono ammissibili solo le soluzioni dispari. Ridefinendo $ k $ così che $ u_{k,0}(r) $ sia la $ (2k+1) $-esima soluzione, si trova lo spettro dell'Hamiltoniana:
\begin{equation}
	E_{k,0} = \hbar \omega \left( 2k + \frac{3}{2} \right)
	\label{eq:3.27}
\end{equation}
Dato che il caso $ \ell = 0 $ è quello maggiormente attrattivo, per l'assenza del termine centrifugo, si trova che lo stato fondamentale è $ \ket{0,0,0} $, con energia $ E_{0,0} = \frac{3}{2} \hbar \omega $.

\subsection{Stati con \texorpdfstring{$ \ell $}{TEXT} generico}

Per scrivere l'Hamiltoniana $ \hat{\mathcal{H}}_{\ell} $ in una forma simile a Eq. \ref{eq:3.25} è necessario definire degli operatori di distruzione/creazione generalizzati:
\begin{equation}
	\hat{d}_{\ell} \defeq \sqrt{\frac{m\omega}{2\hbar}} \left( \left( \hat{r} + \frac{\hbar\ell}{m\omega\hat{r}} \right) + i \frac{\hat{p}_r}{m\omega} \right)
	\label{eq:3.28}
\end{equation}
È utile inoltre ricordare che, essendo $ \left[ \hat{p}_r, f(\hat{r}) \right] = -i\hbar f(\hat{r}) $, si ha $ \left[ \hat{p}_r, \frac{1}{\hat{r}} \right] = \frac{i\hbar}{\hat{r}^2} $.\\
Si può generalizzare l'operatore numero come:
\begin{equation}
	\hat{D}_{\ell} \defeq \hat{d}_{\ell}^{\dagger} \hat{d}_{\ell}
	\label{eq:3.29}
\end{equation}
La sua espressione esplicita è:
\begin{equation*}
	\begin{split}
		D_{\ell}
		&= d_{\ell}^{\dagger} d_{\ell} = \frac{m\omega}{2\hbar} \left( \left( r + \frac{\hbar\ell}{m\omega r} \right) - i \frac{p_r}{m\omega} \right) \left( \left( r + \frac{\hbar\ell}{m\omega r} \right) + i \frac{p_r}{m\omega} \right)\\
		&= \frac{m\omega}{2\hbar} \left( \left( r + \frac{\hbar\ell}{m\omega r} \right)^2 + \frac{p_r^2}{m^2 \omega^2} + \left[ \left( r + \frac{\hbar\ell}{m\omega r} \right), i \frac{p_r}{m\omega} \right] \right)\\
		&= \frac{m\omega}{2\hbar} \left( r^2 + \frac{\hbar^2\ell^2}{m^2 \omega^2 r^2} + \frac{p_r^2}{m^2 \omega^2} + \frac{2\hbar\ell}{m\omega} - \frac{i}{m\omega} \left( -i\hbar + \frac{\hbar\ell}{m\omega} \frac{i\hbar}{r^2} \right) \right)\\
		&= \frac{1}{\hbar\omega} \left( \frac{p_r^2}{2m} + \frac{1}{2}m\omega^2 r^2 + \frac{\hbar^2 \ell (\ell + 1)}{2mr^2} \right) + \ell - \frac{1}{2}\\
	\end{split}
\end{equation*}
Risulta quindi che:
\begin{equation}
	\hat{D}_{\ell} = \frac{1}{\hbar\omega} \hat{\mathcal{H}}_{\ell} + \ell - \frac{1}{2}
	\label{eq:3.30}
\end{equation}
Di conseguenza, $ \hat{D}_{\ell} $ e $ \hat{\mathcal{H}}_{\ell} $ hanno gli stessi autostati $ \ket{k,\ell} $:
\begin{equation}
	\hat{D}_{\ell} \ket{k,\ell} = \mathcal{E}_{k,\ell} \ket{k,\ell}, \,\, \mathcal{E}_{k,\ell} = \frac{1}{\hbar \omega} E_{k,\ell} + \ell - \frac{1}{2}
	\label{eq:3.31}
\end{equation}
Ovviamente $ \hat{D}_0 $ è il consueto operatore numero: infatti, il suo spettro è $ \mathcal{E}_{k,0} = 2k + 1 $.\\
È necessario definire un ulteriore operatore:
\begin{equation}
	\hat{\overline{D}}_{\ell} \defeq \hat{d}_{\ell} \hat{d}_{\ell}^{\dagger}
	\label{eq:3.32}
\end{equation}
Con un calcolo analogo al precedente, si trova che:
\begin{equation}
	\hat{\overline{D}}_{\ell} = \frac{1}{\hbar \omega} \hat{\mathcal{H}}_{\ell - 1} + \ell + \frac{1}{2}
	\label{eq:3.33}
\end{equation}
Risulta evidente quindi che:
\begin{equation}
	\hat{\overline{D}}_{\ell + 1} = \hat{D}_{\ell} + 2
	\label{eq:3.34}
\end{equation}
Questi operatori legano gli spettri di Hamiltoniane con $ \ell $ diversi: si supponga di avere $ \ket{k,\ell} $ autostato di $ \hat{\mathcal{H}}_{\ell} $ e $ \hat{D}_{\ell} $: $ \hat{d}_{\ell + 1}^{\dagger} \ket{k,\ell} $ è un autostato di $ \hat{\mathcal{H}}_{\ell + 1} $ e $ \hat{D}_{\ell + 1} $, dato che:
\begin{equation*}
	\hat{D}_{\ell + 1} \hat{d}_{\ell + 1}^{\dagger} \ket{k,\ell} = \hat{d}_{\ell + 1}^{\dagger} \hat{d}_{\ell + 1} \hat{d}_{\ell + 1}^{\dagger} \ket{k,\ell} = \hat{d}_{\ell + 1}^{\dagger} (\hat{D}_{\ell} + 2) \ket{k,\ell} = (\mathcal{E}_{k,\ell} + 2) \hat{d}_{\ell + 1}^{\dagger} \ket{k,\ell}
\end{equation*}
Questi operatori sono simili a degli operatori di scala:
\begin{equation}
	\hat{D}_{\ell + 1} \hat{d}_{\ell + 1} \ket{k,\ell} = (\mathcal{E}_{k,\ell} + 2) \hat{d}_{\ell + 1}^{\dagger} \ket{k,\ell}
	\label{eq:3.35}
\end{equation}
\begin{equation}
	\hat{D}_{\ell - 1} \hat{d}_{\ell} \ket{k,\ell} = (\mathcal{E}_{k,\ell} - 2) \hat{d}_{\ell} \ket{k,\ell}
	\label{eq:3.36}
\end{equation}
In questo modo, è possibile costruire l'intero spettro, ricordando che $ \mathcal{E}_{k,0} = 2k + 1 $:
\begin{equation*}
	\begin{split}
		&\hat{D}_1 \hat{d}_1^{\dagger} \ket{k,0} = (\mathcal{E}_{k,0} + 2) \hat{d}_1^{\dagger} \ket{k,0} = (2k + 3) \hat{d}_1^{\dagger} \ket{k,0}\\
		&\hat{D}_2 \hat{d}_2^{\dagger} \hat{d}_1^{\dagger} \ket{k,0} = (\mathcal{E}_{k,0} + 4) \hat{d}_2^{\dagger} \hat{d}_1^{\dagger} \ket{k,0} = (2k + 5) \hat{d}_2^{\dagger} \hat{d}_1^{\dagger} \ket{k,0}\\
		&\qquad\quad\vdots\\
		&\hat{D}_{\ell + 1} \hat{d}_{\ell + 1}^{\dagger} \dots \hat{d}_1^{\dagger} \ket{k,0} = (2k + 2\ell + 1) \hat{d}_{\ell + 1}^{\dagger} \dots \hat{d}_1^{\dagger} \ket{k,0}
	\end{split}
\end{equation*}
Si evince che $ \mathcal{E}_{k,\ell} = 2k + 2\ell + 1 $, quindi dall'Eq. \ref{eq:3.31} si ricava lo spettro dell'$ \ell $-esima Hamiltoniana:
\begin{equation}
	E_{k,\ell} = \hbar \omega \left( 2k + \ell + \frac{3}{2} \right)
	\label{eq:3.37}
\end{equation}
Questi sono tutti e soli i possibili autovalori di $ \hat{\mathcal{H}}_{\ell} $: se per assurdo esistesse un suo autostato con un autovalore non presente in questa sequenza, ci si potrebbe comunque sempre ricondurre ad uno stato con $ \ell = 0 $ applicando ripetutamente $ \hat{d}_{\ell} $, ottenendo così un nuovo autovalore di $ \hat{\mathcal{H}}_0 $, il che è assurdo poiché il suo spettro completo (Eq. \ref{eq:3.27}) è dato dall'Eq. \ref{eq:3.37}.\\
Ovviamente, in maniera analoga, si può partire da uno stato $ \ket{k,\ell} $ ed ottenere gli stati minori fino a $ \ket{k,0} $ applicando $ \hat{D}_{\ell} $.

\subsection{Degenerazione dello spettro}

In coordinate cartesiane, lo spettro dell'oscillatore armonico isotropo è dato da $ E_n = \hbar \omega \left( n + \frac{3}{2} \right) $, con degenerazione $ d_n = \frac{1}{2}(n + 1)(n + 2) $.\\
Questo stesso spettro si ottiene in coordinate sferiche ponendo in Eq. \ref{eq:3.37} $ n = 2k + \ell $, e si dimostra che la degenerazione è la stessa.

\begin{proposition}
	Ponendo $ n = 2k + \ell $, si ha $ d_n = \frac{1}{2}(n + 1)(n + 2) $.
\end{proposition}
\begin{proof}
	Fissato $ n $ (dunque fissato il sottospazio di energia considerato), si ha che $ \ell $ è vincolato da $ 0 \le \ell \le n $, ma si nota anche che se $ n $ è pari/dispari anche $ \ell $ deve essere pari/dispari. Dunque, se $ n = 2n' $ si ha $ \ell = 2\ell' $, con $ 0 \le \ell' \le n' $, mentre se $ n = 2n' + 1 $ si ha $ \ell = 2\ell' + 1 $, con $ 0 \le \ell' \le n' $. Allora, ricordando che ogni stato $ \ell $ ha una degernazione $ 2\ell + 1 $ a causa di $ m $:
	\begin{equation*}
		d_n^{\text{(pari)}} = \sum_{\ell' = 0}^{n'} \left( 2(2\ell') + 1 \right) = 2 n' (n' + 1) + n' + 1 = n \left( \frac{n}{2} + 1 \right) + \frac{n}{2} + 1 = \frac{1}{2} (n + 1) (n + 2)
	\end{equation*}
	\begin{equation*}
		d_n^{\text{(dispari)}} = \sum_{\ell' = 0}^{n'} \left( 2 (2\ell' + 1) + 1) \right) = 2n' (2n' + 1) + 3 (n' + 1) = \frac{1}{2} (n + 1) (n + 2)
	\end{equation*}
\end{proof}

Dunque, sia che gli stati vengano scritti in coordinate cartesiane come $ \ket{n_1, n_2, n_3} $, sia che vengano scritti in coordinate sferiche come $ \ket{n,\ell,m} $, si ottiene giustamente lo stesso spettro. Naturalmente, è sempre possibile passare da una rappresentazione all'altra tramite una trasformazione unitaria in ciascun sottospazio di energia fissata.

\subsubsection{Teorema di degenerazione}

Questo risultato deriva dal teorema di degenerazione.

\begin{theorem}
	Data una Hamiltoniana $ \hat{\mathcal{H}} $ e due operatori $ \hat{A}, \hat{B} : [\hat{A},\hat{\mathcal{H}}] = [\hat{B},\hat{\mathcal{H}}] = 0 $, allora $ [\hat{A},\hat{B}] \neq 0 \,\Rightarrow\, \hat{\mathcal{H}} $ ha spettro degenere.
\end{theorem}
\begin{proof}
	Per assurdo sia lo spettro di $ \hat{\mathcal{H}} $ non degenere, ovvero per ogni autovalore $ E_n $ ci sia un solo autostato $ \ket{n} : \hat{\mathcal{H}}\ket{n} = E_n\ket{n} $. Dato che $ [\hat{A},\hat{\mathcal{H}}] = 0 $, essi sono diagonalizzabili simultaneamente, e la stessa cosa vale per $ \hat{B} $: essi hanno la comune base di autostati $ \ket{n} $, dunque commutano, il che contraddice l'ipotesi.
\end{proof}

Si vede dunque che in ogni sottospazio degenere di autostati diverse combinazioni di autostati associati allo stesso autovalore diagonalizzano l'uno o l'altro operatore, ma non entrambi.\\
Questo teorema permette di legare la degenerazione dello spettro alle simmetrie dell'Hamiltoniana: ad esempio, nel caso del momento angolare, l'Hamiltoniana invariante per rotazione commuta con ciascuno dei $ \hat{L}_i $, ma questi non commutano tra loro; di conseguenza, l'Hamiltoniana può avere un termine proporzionale a $ \hat{L}^2 $, ma non hai singoli $ \hat{L}_i $, quindi si può scegliere di diagonalizzare $ \hat{L}^2 $ e uno dei $ \hat{L}_i $, ottenendo un grado di degenerazione come visto in precedenza.\\
Nel caso più generale, si determinano tutti gli operatori che commutano con l'Hamiltoniana e poi tutti gli stati ottenibili l'uno dall'altro tramite trasformazioni generate dai suddetti operatori (esponenziandoli): la degenerazione è il numero di stati contenuti in ciascun insieme di stati di questo tipo, e può essere vista come la dimensione della rappresentazione irriducibile del gruppo di trasformazioni generate dagli operatori considerati.

\subsection{Simmetria dell'oscillatore armonico isotropo}

Per l'oscillatore armonico isotropo si vede che il grado di degenerazione è maggiore di quello dovuto all'invarianza per rotazioni: infatti, per ogni valore di $ n $ ci sono più valori di $ \ell $ corrispondenti allo stesso valore di energia. Ciò significa che ci devono essere altri operatori commutanti con l'Hamiltoniana (ma non fra loro).\\
In coordinate cartesiane, si definiscono i 3 operatori di distruzione (e i 3 di creazione):
\begin{equation}
	\hat{a}_i \defeq \sqrt{\frac{m \omega}{2\hbar}} \left( \hat{x}_i + i \frac{\hat{p}_i}{m\omega} \right)
	\label{eq:3.38}
\end{equation}
In questo modo, è possibile separare l'Hamiltoniana come:
\begin{equation}
	\hat{\mathcal{H}} = \sum_{i = 1}^{3} \hbar \omega \left( \hat{a}_i^{\dagger} \hat{a}_i + \frac{1}{2} \right)
	\label{eq:3.39}
\end{equation}
Si definiscono dunque i 9 operatori:
\begin{equation}
	\hat{\mathcal{O}}_{ij} \defeq \hat{a}_i^{\dagger} \hat{a}_j
	\label{eq:3.40}
\end{equation}
\begin{proposition}
	$ [\hat{\mathcal{O}}_{ij}, \hat{\mathcal{H}}] = 0 \,\land\, [\hat{\mathcal{O}}_{ij}, \hat{\mathcal{O}}_{ab}] \neq 0 $.
\end{proposition}
\begin{proof}
	Ricordando che $ [\hat{a}_i^{\dagger},\hat{a}_j] = -\delta_{ij} $:
	\begin{equation*}
		[ \hat{\mathcal{O}}_{ij}, \hat{\mathcal{H}} ] = \hbar \omega \sum_{k = 1}^{3} \,[\hat{a}_i^{\dagger}\hat{a}_j,\hat{a}_k^{\dagger}\hat{a}_k] = \hbar \omega \sum_{k = 1}^{3} \left( \hat{a}_i^{\dagger} [\hat{a}_j,\hat{a}_k^{\dagger}\hat{a}_k] + [\hat{a}_i^{\dagger},\hat{a}_k^{\dagger}\hat{a}_k]\hat{a}_j \right) = \hbar \omega \left( \hat{a}_i^{\dagger} \hat{a}_j - \hat{a}_i^{\dagger} \hat{a}_j \right) = 0
	\end{equation*}
	\begin{equation*}
		[\hat{\mathcal{O}}_{ij},\hat{\mathcal{O}}_{ab}] = [\hat{a}_i^{\dagger}\hat{a}_j,\hat{a}_a^{\dagger}\hat{a}_b] = \hat{a}_i^{\dagger} [\hat{a}_{j},\hat{a}_a^{\dagger}\hat{a}_b] + [\hat{a}_i^{\dagger},\hat{a}_a^{\dagger},\hat{a}_b]\hat{a}_j = \hat{a}_i^{\dagger} \delta_{ja} \hat{a}_b - \delta_{ib} \hat{a}_a^{\dagger} \hat{a}_j \neq 0
	\end{equation*}
\end{proof}
L'Hamiltoniana può essere scritta come una combinazione lineare di 3 di quesi operatori:
\begin{equation}
	\hat{\mathcal{H}} = \sum_{i = 1}^{3} \hbar \omega \left( \hat{\mathcal{O}}_{ii} + \frac{1}{2} \right)
	\label{eq:3.41}
\end{equation}
Rimangono dunque 8 operatori che commutano con l'Hamiltoniana ma non tra di loro e che, per il teorema di degenerazione, determinano il grado di degenerazione dello spettro. In particolare, 3 di questi sono proprio gli operatori del momento angolare:
\begin{equation}
	\hat{L}_i = - i\hbar \sum_{j,k = 1}^{3} \epsilon_{ijk} \hat{\mathcal{O}}_{jk}
	\label{eq:3.42}
\end{equation}
Ad esempio, con calcolo esplicito:
\begin{equation*}
	\hat{\mathcal{O}}_{12} - \hat{\mathcal{O}}_{21} = \frac{m \omega}{2\hbar} \left[ \left( \hat{x} - i \frac{\hat{p}_x}{m\omega} \right) \left( \hat{y} + i \frac{\hat{p}_y}{n\omega} \right) - \left( \hat{y} - i \frac{\hat{p}_y}{m\omega} \right) \left( \hat{x} + i \frac{\hat{p}_x}{m\omega} \right) \right] = \frac{i}{\hbar} \left( \hat{x} \hat{p}_y - \hat{y} \hat{p}_x \right) = \frac{i}{\hbar} \hat{L}_z
\end{equation*}
Gli operatori $ \hat{\mathcal{O}}_{ij} $ sono i generatori del gruppo $ \SUn{3} $ (si vede dalle relazioni di commutazione): in generale, l'oscillatore armonico $ n $-dimensionale ha simmetria $ \SUn{n} $. Gli operatori $ \hat{L}_i $ generano invece $ \SUn{2} $, che è un sottogruppo di $ \SUn{3} $ e corrisponde all'invarianza per rotazioni (se si restringe allo spin intero si ha solo $ \SOn{3} $).

\section{Potenziale coulombiano}

Un problema di fondamentale importanza è quello della descrizione quantitativa degli atomi idrogenoidi, ovvero quelli in cui un nucleo atomico con $ Z $ protoni è circondato da un solo elettrone. Per tale sistema, l'equazione di Schrödinger radiale Eq. \ref{eq:3.5} diventa:
\begin{equation}
	\left[ -\frac{\hbar^2}{2m} \frac{\pa^2}{\pa r^2} + \frac{\hbar^2 \ell (\ell + 1)}{2mr^2} - \frac{Ze^2}{r} \right] u(r) = E u(r)
	\label{eq:3.43}
\end{equation}
Il problema può essere formulato in termini adimensionali definendo alcune quantità caratteristiche; innanzitutto, si definisce il \textit{raggio di Bohr}:
\begin{equation}
	a_0 \defeq \frac{\hbar^2}{mZe^2}
	\label{eq:3.44}
\end{equation}
Dall'analisi dimensionale dell'Eq. \ref{eq:3.43}, si trova correttamente che $ a_0 $ ha le dimensioni di una lunghezza, così da poter esprimere le lunghezze rispetto ad esso: $ r \mapsto r a_0 $. In questo modo:
\begin{equation*}
	\begin{split}
		E u(r)
		&= \left[ -\frac{\hbar^2}{2m a_0^2} \frac{\pa}{\pa r^2} + \frac{\hbar^2 \ell(\ell + 1)}{2ma_0^2 r^2} - \frac{Ze^2}{a_0 r} \right] u(r) \\
		&= \left[ \frac{m (Ze^2)^2}{\hbar^2} \frac{1}{2} \left( - \frac{\pa^2}{\pa r^2} + \frac{\ell (\ell + 1)}{r^2} \right) - \frac{m (Ze^2)^2}{\hbar^2} \frac{1}{r} \right] u(r) \\
		&= \frac{m(Ze^2)^2}{\hbar^2} \left[ - \frac{1}{2} \frac{\pa^2}{\pa r^2} + \frac{1}{2} \frac{\ell (\ell + 1)}{r^2} - \frac{1}{r} \right] u(r)
	\end{split}
\end{equation*}
Si può dunque definire una costante d'energia:
\begin{equation}
	W_0 \defeq \frac{m (Ze^2)^2}{\hbar^2}
	\label{eq:3.45}
\end{equation}
Studiando gli stati legati, quindi con energie negative (si vede graficando il potenziale), dato che essi sono discreti si può scrivere $ E_n = c_n W_0 $, dunque la determinazione dello spettro si riduce al determinare i coefficienti $ c_n $. L'equazione radiale adimensionale diventa quindi:
\begin{equation}
	\left[ -\frac{1}{2} \frac{\pa^2}{\pa r^2} + \frac{1}{2} \frac{\ell (\ell + 1)}{r^2} - \frac{1}{r} \right] u(r) = c_n u(r)
	\label{eq:3.46}
\end{equation}

\paragraph{Teorema del viriale}

In questi termini adimensionali, è possibile vedere perché i potenziali che nell'origine hanno andamento $ V(r) \sim r^{-\alpha} $ con $ \alpha \ge 2 $ sono patologici. Dato che $ W_0 \sim a_0 $, si vede che $ \braket{T} \sim a_0^{-2} $ e $ \braket{V} \sim a_0^{-\alpha} $, dunque per $ \alpha \ge 2 $ il termine potenziale può essere reso a piacere più grande di quello cinetico considerando sistemi con $ a_0 $ via via più piccolo: data una qualunque autofunzione, è possibile fabbricarne una ad energia più bassa prendendola più localizzata, ovvero non esiste uno stato di energia minima.

\paragraph{Risoluzione analitica}

Il primo metodo utilizzato per la risoluzione dell'Eq. \ref{eq:3.46} fu quello analitico (da Schrödinger). Innanzitutto, da stime asintotiche per $ r \rightarrow \infty $ e $ r \rightarrow 0 $ si ottiene che:
\begin{equation*}
	u(r) \overset{r \rightarrow \infty}{\sim} e^{-kr} \qquad u(r) \overset{r \rightarrow 0}{\sim} r^{\ell + 1}
\end{equation*}
con $ k^2 = 2c_n $.
L'Ansatz risolutivo è quindi:
\begin{equation*}
	u(r) = e^{-kr} r^{\ell + 1} f(r) \,, \quad f(r) = \sum_{n = 0}^{\infty} \alpha_n r^n
\end{equation*}
Affinché la serie si normalizzabile, è necessario che essa si arresti ad un certo $ n_{\text{max}} $: ciò avviene solo per particolari valori di $ k $, ovvero $ k_n = (n + \ell)^{-1} $ con $ n \in \N $. Si trova dunque lo spettro dell'energia, ridefinendo $ n \equiv n + \ell $ come il numero quantico principale:
\begin{equation}
	E_n = -\frac{1}{2} \frac{m(Ze^2)^2}{\hbar^2} \frac{1}{n^2}
	\label{eq:3.47}
\end{equation}
La degenerazione dello spettro deriva dal fatto che diversi valori di $ \ell $ possono essere associati allo stesso $ n $, oltre che al fatto che ogni $ \ell $ ha $ 2\ell + 1 $ stati associati.

\subsection{Modello di Bohr}
\label{bohr-hydrogen}

È propedeutico analizzare il modello quantizzato ad hoc proposto da Bohr per spiegare i dati sperimentali sullo spettro dell'atomo d'idrogeno. In particolare, Bohr suppose che l'elettrone nell'atomo di $ \ch{^1H} $ si muovesse lungo orbite circolari e che il momento angolare fosse quantizzato in unità intere di $ \hbar $. Sebbene dal punto di vista della meccanica quantistica moderna non abbia senso parlare di orbite ben definite, oltre al fatto che gli autovalori di $ \hat{L}^2 $ sono $ \hbar^2 \ell (\ell + 1) $, dunque il momento angolare non è un multiplo intero di $ \hbar $, partendo da queste ipotesi Bohr riuscì a spiegare esattamente i dati sperimentali.\\
Seguendo il ragionamento di Bohr, si supponga che il sistema sia descritto dalla seguente Lagrangiana:
\begin{equation}
	\mathcal{L} = \frac{1}{2} m (\dot{r}^2 + r^2\dot{\theta}^2) + \frac{e^2}{r}
	\label{eq:3.48}
\end{equation}
Se l'orbita è circolare, $ \dot{r} = 0 $, quindi l'equazione di Lagrange per $ r $ diventa:
\begin{equation*}
	\frac{\pa \mathcal{L}}{\pa r} = 0 \quad \Rightarrow \quad mr \dot{\theta}^2 = \frac{Ze^2}{r^2} \quad \Rightarrow \quad 2T = -V
\end{equation*}
Questa stessa relazione si poteva trovare col teorema del viriale, che vale anche per valori medi quantistici per il teorema di Ehrenfest, e per questo il ragionamento di Bohr porta al risultato corretto anche partendo da ipotesi non corrette. Si trova che l'energia vale quindi:
\begin{equation}
	E = -T
	\label{eq:3.49}
\end{equation}
L'equazione di Lagrange per $ \theta $ è:
\begin{equation*}
	\frac{d}{dt} \frac{\pa \mathcal{L}}{\pa \dot{\theta}} = \frac{\pa \mathcal{L}}{\pa \theta} \quad \Rightarrow \quad \frac{\pa \mathcal{L}}{\pa \dot{\theta}} = mr^2 \dot{\theta} \equiv \ell \,\text{ const.}
\end{equation*}
La seconda ipotesi di Bohr diventa quindi $ \ell = n \hbar $, con $ n \in \N $; dall'equazione per $ r $ si trova:
\begin{equation*}
	mr^2 \dot{\theta}^2 = \frac{Ze^2}{r^2} \quad \Rightarrow \quad \frac{\ell^2}{mr^3} = \frac{Ze^2}{r^2} \quad \Rightarrow \quad r = \frac{\ell^2}{mZe^2} = n^2 a_0
\end{equation*}
Rielaborando l'Eq. \ref{eq:3.49}:
\begin{equation*}
	E = -T = -\frac{1}{2} m r^2 \dot{\theta}^2 = -\frac{1}{2} \frac{\ell^2}{mr^2} = -\frac{1}{2} \frac{m(Ze^2)^2}{\ell^2} = -\frac{1}{2} \frac{m (Ze^2)^2}{\hbar^2} \frac{1}{n^2}
\end{equation*}
che è proprio lo spettro trovato in Eq. \ref{eq:3.47}.

\subsection{Simmetrie del problema di Keplero}

\subsubsection{Caso classico}

In termini adimensionali, il problema di Keplero classico può essere formulato tramite la seguente Hamiltoniana:
\begin{equation}
	\mathcal{H} = \frac{\ve{p}^2}{2} - \frac{1}{r}
	\label{eq:3.50}
\end{equation}
Le equazioni del moto sono quindi:
\begin{equation}
	\begin{cases}
		\dot{\ve{x}} = \ve{p} \\
		\dot{\ve{p}} = \nabla \frac{1}{r} = - \frac{\ve{x}}{r^3}
	\end{cases}
	\label{eq:3.51}
\end{equation}
È necessario indagare gli invarianti del sistema, derivanti dalle sue simmetrie per il teorema di Noether. Una prima simmetria ovvia è quella per rotazioni.

\begin{proposition}
	$ \ve{L} \defeq \ve{x} \times \ve{p} $ è una costante del moto.
\end{proposition}
\begin{proof}
	$ \frac{d}{dt} (\ve{x} \times \ve{p}) = \dot{\ve{x}} \times \ve{p} + \ve{x} \times \dot{\ve{p}} = \ve{p} \times \ve{p} - \frac{1}{r^3} \ve{x} \times \ve{x} = \ve{0} $.
\end{proof}

Un'altra simmetria banale è quella per traslazione temporale, che risulta nella conservazione dell'energia $ E \equiv \mathcal{H} $. Un'invariante non-banale è invece il \textit{vettore di Laplace-Runge-Lenz}.

\begin{proposition}
	$ \ve{M} \defeq \ve{p} \times \ve{L} - \frac{1}{r} \ve{x} $ è una costante del moto.
\end{proposition}
\begin{proof}
	Basta mostrare che $ \frac{d}{dt} (\ve{p} \times \ve{L}) = \frac{d}{dt} \frac{\ve{x}}{r} $:
	\begin{equation*}
		\begin{split}
			\frac{d}{dt} \left( \ve{p} \times \ve{L} \right)_i
			&= \left( \dot{\ve{p}} \times \ve{L} \right)_i = \left( -\frac{\ve{x}}{r^3} \times \ve{L} \right)_i = - \frac{1}{r^3} \sum_{j,k = 1}^{3} \epsilon_{ijk} x_j L_k = - \frac{1}{r^3} \sum_{j,k,a,b = 1}^{3} \epsilon_{ijk} \epsilon_{kab} x_j x_a p_b \\
			&= - \frac{1}{r^3} \sum_{j,k,a,b = 1}^{3} \left( \delta_{ia} \delta_{jb} - \delta_{ib} \delta_{ja} \right) x_j x_a p_b = - \frac{1}{r^3} \left( x_i \ve{x} \cdot \ve{p} - r^2 p_i \right) = \frac{\dot{x}_i}{r} - \frac{x_i}{r^3} \sum_{j = 1}^{3} x_j \dot{x}_j = \frac{d}{dt} \frac{x_i}{r}
		\end{split}
	\end{equation*}
\end{proof}

A priori si trovano quindi 7 invarianti. Tuttavia, essendo lo spazio delle fasi di dimensione 6 per il problema di Keplero, vi possono essere al massimo 6 costanti del moto indipendenti; inoltre, se ci si restringe a considerare orbite chiuse, il numero d'invarianti indipendenti scende a 5 poiché l'origine temporale diventa irrilevante.\\
Il corretto numero di costanti del moto indipendenti si ritrova osservando che solo una delle tre componenti del vettore LRL è indipendente.

\begin{proposition}
	$ \ve{M} \cdot \ve{L} = 0 $.
\end{proposition}
\begin{proof}
	$ \ve{M} \cdot \ve{L} = (\ve{p} \times \ve{L}) \cdot \ve{L} - \frac{1}{r} \ve{x} \cdot \ve{L} = - \frac{1}{r} \ve{x} \cdot (\ve{x} \times \ve{p}) = 0 $.
\end{proof}

\begin{proposition}
	$ \ve{M}^2 = 1 + 2 \ve{L}^2 \mathcal{H} $.
\end{proposition}
\begin{proof}
	Ricordando la Prop. \ref{cross-class}:
	\begin{equation*}
		\begin{split}
			\ve{M}^2
			&= \norm{\ve{p} \times \ve{L}}^2 + \norm{\frac{\ve{x}}{r}}^2 - 2 \frac{\ve{x}}{r} \cdot \left( \ve{p} \times \ve{L} \right) = \ve{p}^2 \ve{L}^2 - \left( \ve{p} \cdot \ve{L} \right)^2 + 1 - \frac{2}{r} \sum_{i,j,k = 1}^{3} x_i \epsilon_{ijk} p_j L_k \\
			&= \ve{p}^2 \ve{L}^2 + 1 - \frac{2}{r} \sum_{k = 1}^{3} L_k^2 = 1 + \ve{L}^2 \left( \ve{p}^2 - \frac{2}{r} \right) = 1 + 2 \ve{L}^2 \mathcal{H}
		\end{split}
	\end{equation*}
\end{proof}

Inoltre, per un orbita a sezione conica di eccentricità $ \varepsilon $, si dimostra che $ \norm{\ve{M}} = \varepsilon $: si vede quindi che per orbite chiuse, dato che $ \varepsilon \le 1 $, l'energia deve essere negativa. Inoltre, per orbite ellittiche si dimostra che $ \ve{M} $ è diretto lungo l'asse maggiore dell'ellisse.\\
Si vede dunque che l'unica informazione utile contenuta nel vettore LRL è quella relativa al suo verso, poiché il piano d'appartenenza e il modulo sono fissati dagli altri invarianti del sistema.

\subsubsection{Caso quantistico}

Si consideri la seguente Hamiltoniana idrogenoide adimensionale:
\begin{equation}
	\hat{\mathcal{H}} = \frac{\hat{\ve{p}}^2}{2} - \frac{1}{r}
	\label{eq:3.52}
\end{equation}
È evidente che, analogamente al caso classico, si conservano momento angolare (generatore delle rotazioni) ed energia.
\begin{definition}
	Si definisce l'\textit{operatore di Laplace-Runge-Lenz} come:
	\begin{equation}
		\hat{\ve{M}} \defeq \frac{1}{2} \left( \hat{\ve{p}} \times \hat{\ve{L}} - \hat{\ve{L}} \times \hat{\ve{p}} \right) - \frac{\hat{\ve{x}}}{r}
		\label{eq:3.53}
	\end{equation}
\end{definition}
In componenti:
\begin{equation}
	\hat{M}_i = \frac{1}{2} \sum_{j,k = 1}^{3} \epsilon_{ijk} \{\hat{p}_j,\hat{L}_k\} - \frac{\hat{x}_i}{r}
	\label{eq:3.54}
\end{equation}
La definizione tramite l'anticommutatore è necessaria per il fatto che $ \hat{\ve{p}} $ e $ \hat{\ve{L}} $ non commutano: l'anticommutatore di due operatori hermitiani non commutanti è infatti hermitiano.\\
Anche nel caso quantistico si trova che il vettore LRL è un invariante del sistema, ovvero commuta con l'Hamiltoniana.

\begin{proposition}
	$ [\hat{M}_i, \hat{\mathcal{H}}] = 0 $.
\end{proposition}
\begin{proof}
	Ricordando che $ [\hat{L}_i,\hat{x}_j] = i\hbar \sum_{k = 1}^{3} \epsilon_{ijk} \hat{x}_k $ e $ [\hat{L}_i,\hat{p}_j] = i\hbar \sum_{k = 1}^{3} \epsilon_{ijk} \hat{p}_k $, oltre al fatto che $ [\hat{L}_i, \hat{r}] = [\hat{L}_i, \hat{p}] = 0 $ ($ \hat{p} \equiv \norm{\hat{\ve{p}}} $), dato che in generale $ [\{\hat{A},\hat{B}\},\hat{C}] = \{\hat{A},[\hat{B},\hat{C}]\} + \{\hat{B},[\hat{A},\hat{C}]\} $:
	\begin{equation*}
		\begin{split}
			[\hat{M}_i,\hat{\mathcal{H}}]
			&= \frac{1}{2} \sum_{j,k = 1}^{3} \epsilon_{ijk} [\{\hat{p}_j,\hat{L}_k\}, \hat{\mathcal{H}}] - \sum_{j = 1}^{3} \left[ \frac{\hat{x}_i}{r}, \hat{p}_j \hat{p}_j \right] = \frac{1}{2} \sum_{j,k = 1}^{3} \epsilon_{ijk} \{\hat{L}_k,[\hat{p}_j,\hat{\mathcal{H}}]\} - \sum_{j = 1}^{3} \left[ \frac{\hat{x}_i}{r}, \hat{p}_j\hat{p}_j \right] \\
			&= \frac{1}{2} \sum_{j,k = 1}^{3} \epsilon_{ijk} \bigg\{ \hat{L}_k, \left[\hat{p}_j, - \frac{1}{r}\right] \bigg\} - \frac{1}{2} \sum_{j = 1}^{3} \left( \hat{p}_j \left[ \frac{\hat{x}_i}{r}, \hat{p}_j \right] + \left[ \frac{\hat{x}_i}{r}, \hat{p}_j \right] \hat{p}_j \right)
		\end{split}
	\end{equation*}
	Dato $ \hat{L}_i \sum_{j,k = 1}^{3} \epsilon_{ijk} \hat{x}_j \hat{p}_k = \sum_{j,k = 1}^{3} \epsilon_{ijk} \left( \hat{p}_k \hat{x}_j + [\hat{x}_j,\hat{p}_k] \right) = \sum_{j,k = 1}^{3} \epsilon_{ijk} \left( \hat{p}_k \hat{x}_j + i\hbar \delta_{jk} \right) = \sum_{j,k = 1}^{3} \epsilon_{ijk} \hat{p}_k \hat{x}_j $ ed espandendo gli ultimi due commutatori sulla base delle coordinate:
	\begin{equation*}
		\begin{split}
			[\hat{M}_i, \hat{\mathcal{H}}]
			&= -\frac{i\hbar}{2} \sum_{j,k = 1}^{3} \epsilon_{ijk} \bigg\{ \hat{L}_k, \frac{\hat{x}_j}{r^3} \bigg\} - \frac{i\hbar}{2} \sum_{j = 1}^{3} \left( \hat{p}_j \left( \pa_j \frac{x_i}{r} \right) + \left( \pa_j \frac{x_i}{r} \right) \hat{p}_j \right) \\
			&= -\frac{i\hbar}{2} \sum_{j,k,a,b = 1}^{3} \epsilon_{ijk} \epsilon_{kab} \left( \frac{\hat{x}_j}{r^3} \hat{x}_a \hat{p}_b + \hat{p}_b \hat{x}_a \frac{\hat{x}_j}{r^3} \right) - \frac{i\hbar}{2} \sum_{j = 1}^{3} \left( \hat{p}_j \left( \frac{\delta_{ij}}{r} - \frac{\hat{x}_i \hat{x}_j}{r^3} \right) + \left( \frac{\delta_{ij}}{r} - \frac{\hat{x}_i \hat{x}_j}{r^3} \right) \hat{p}_j \right) \\
			&= - \frac{i\hbar}{2} \sum_{j,a,b = 1}^{3} \left( \delta_{ia} \delta_{jb} - \delta_{ib} \delta_{ja} \right) \left( \frac{\hat{x}_j}{r^3} \hat{x}_a \hat{p}_b + \hat{p}_b \hat{x}_a \frac{\hat{x}_j}{r^3} \right) - \frac{i\hbar}{2} \left( \hat{p}_i \frac{1}{r} - \frac{\hat{x}_i}{r^3} \hat{\ve{x}}\cdot\hat{\ve{p}} + \frac{1}{r} \hat{p}_i - \hat{\ve{p}}\cdot\hat{\ve{x}} \frac{\hat{x}_i}{r^3} \right) \\
			&= - \frac{i\hbar}{2} \left( \frac{\hat{x}_i}{r^3} \hat{\ve{x}}\cdot\hat{\ve{p}} - \frac{1}{r} \hat{p}_i + \hat{\ve{p}}\cdot\hat{\ve{x}} \frac{\hat{x}_i}{r^3} - \hat{p}_i \frac{1}{r} \right) - \frac{i\hbar}{2} \left( \hat{p}_j \frac{1}{r} - \frac{\hat{x}_i}{r^3} \hat{\ve{x}}\cdot\hat{\ve{p}} + \frac{1}{r} \hat{p}_j - \hat{\ve{p}}\cdot\hat{\ve{x}} \frac{\hat{x}_i}{r^3} \right) = 0
		\end{split}
	\end{equation*}
\end{proof}

Dato che la stessa discussione sulle costanti del moto classiche si applica al caso quantistico, ci sono due invarianti non-indipendenti, che anche in questo caso sono componenti del vettore LRL.

\begin{proposition}
	$ \hat{\ve{M}}\cdot\hat{\ve{L}} = \hat{\ve{L}}\cdot\hat{\ve{M}} = 0 $.
\end{proposition}
\begin{proof}
	Date le identità $ \hat{L}_k \hat{p}_j = \hat{p}_j \hat{L}_k + i\hbar \sum_{a = 1}^{3} \epsilon_{kja} \hat{p}_a $ e $ \sum_{j,k = 1}^{3} \epsilon_{ijk} \epsilon_{kja} = - \sum_{j,k = 1}^{3} \epsilon_{ijk} \epsilon_{ajk} = -2\delta_{ia} $, si può riscrivere il vettore LRL come:
	\begin{equation*}
		\hat{M}_i = \sum_{j,k = 1}^{3} \epsilon_{ijk} \frac{\hat{p}_j \hat{L}_k + \hat{L}_k \hat{p}_j}{2} - \frac{\hat{x}_j}{r} = \sum_{j,k = 1}^{3} \epsilon_{ijk} \hat{p}_j \hat{L}_k + \frac{i\hbar}{2} \sum_{j,k,a = 1}^{3} \epsilon_{ijk} \epsilon_{kja} \hat{p}_a - \frac{\hat{x}_i}{r} = \sum_{j,k = 1}^{3} \epsilon_{ijk} \hat{p}_j \hat{L}_k - i\hbar \hat{p}_i - \frac{\hat{x}_i}{r}
	\end{equation*}
	Di conseguenza, ricordando che il simbolo di Levi-Civita estrae la parte antisimmetrica:
	\begin{equation*}
		\begin{split}
			\hat{\ve{M}}\cdot\hat{\ve{L}}
			&= \sum_{i = 1}^{3} \hat{M}_i \hat{L}_i = \sum_{i,j,k = 1}^{3} \epsilon_{ijk} \hat{p_j} \hat{L}_k \hat{L}_i - i\hbar \hat{\ve{p}} \cdot \hat{\ve{L}} - \frac{1}{r} \hat{\ve{x}} \cdot \hat{\ve{L}} = \sum_{i,j,k = 1}^{3} \epsilon_{ijk} \hat{p}_j \hat{L}_k \hat{L}_i \\
			&= \frac{1}{2} \sum_{i,j,k = 1}^{3} \epsilon_{ijk} \hat{p}_j \left( [\hat{L}_k, \hat{L}_i] + \{\hat{L}_k, \hat{L}_i\} \right) = \frac{i\hbar}{2} \sum_{i,j,k,a = 1}^{3} \epsilon_{ijk} \epsilon_{kia} \hat{p}_j \hat{L}_a = \frac{i\hbar}{2} \sum_{j = 1}^{3} \hat{p}_j \hat{L}_j = \frac{i\hbar}{2} \hat{\ve{p}} \cdot \hat{\ve{L}} = 0
		\end{split}
	\end{equation*}
\end{proof}

\begin{proposition}\label{lrl-h-l}
	$ \hat{M}^2 = \tens{I} + 2\hat{\mathcal{H}} (\hat{L}^2 + \hbar^2) $.
\end{proposition}

Si ritrovano quindi i consueti 5 integrali del moto.

\begin{proposition}\label{comm-l-m}
	$ [\hat{L}_i, \hat{M}_j] = i\hbar \sum_{k = 1}^{3} \epsilon_{ijk} \hat{M}_k $.
\end{proposition}
\begin{proof}
	Ricordando che $ [\hat{L}_i, \hat{A}_j] = i\hbar \sum_{k = 1}^{3} \epsilon_{ijk} \hat{A}_k $, con $ \hat{A}_j = \hat{x}_j, \hat{p}_j, \hat{L}_j $
\end{proof}

\begin{proposition}\label{comm-m-m}
	$ [\hat{M}_i, \hat{M}_j] = -2i\hbar \hat{\mathcal{H}} \sum_{k = 1}^{3} \epsilon_{ijk} \hat{L}_k $.
\end{proposition}
\begin{proof}
	Ricordando che $ [\hat{L}_i, \hat{A}_j] = i\hbar \sum_{k = 1}^{3} \epsilon_{ijk} \hat{A}_k $, con $ \hat{A}_j = \hat{x}_j, \hat{p}_j, \hat{L}_j $
\end{proof}

\subsection{Spettro}

La Prop. \ref{lrl-h-l} lega lo spettro di $ \hat{\mathcal{H}} $ a quello di $ \hat{M}^2 $ e $ \hat{L}^2 $, ma non è ovvio stabilire a priori se questi siano diagonalizzabili simultaneamente per le Prop. \ref{comm-l-m}-\ref{comm-m-m}.\\
Si supponga l'esistenza di autostati dell'energia tali che:
\begin{equation}
	\hat{\mathcal{H}} \ket{E} = E \ket{E}
	\label{eq:3.55}
\end{equation}
Per questi si ha che $ [\hat{M}_i, \hat{M}_j] \ket{E} = -2E i\hbar \sum_{k = 1}^{3} \epsilon_{ijk} \hat{L}_k $; dato che si cercano stati legati, dunque con $ E < 0 $, ciò suggerisce di definire dei nuovi operatori $ \virgolette{normalizzati} $:
\begin{equation}
	\hat{N}_i \defeq \frac{1}{\sqrt{-2E}} \hat{M}_i
	\label{eq:3.56}
\end{equation}

\begin{proposition}
	$ [\hat{L}_i, \hat{N}_j] = i\hbar \sum_{k = 1}^{3} \epsilon_{ijk} \hat{N}_k $.
\end{proposition}
\begin{proof}
	Dalla Prop. \ref{comm-l-m}.
\end{proof}

\begin{proposition}
	$ [\hat{N}_i, \hat{N}_j] = i\hbar \sum_{k = 1}^{3} \epsilon_{ijk} \hat{L}_k $.
\end{proposition}
\begin{proof}
	Dalla Prop. \ref{comm-m-m}.
\end{proof}

Le relazioni di commutazione degli operatori $ \hat{L}_i $ e $ \hat{N}_i $ ricordano quelle dei generatori del gruppo di Lorentz \footnote{$ \SOn{3,1} $, sottogruppo del gruppo di Poincaré composto da tutte le isometrie dello spaziotempo di Minkowski che lasciano fissa l'origine.}, con i primi identificati con i generatori delle rotazioni e i secondi con quelli dei boost.

\begin{proposition}\label{mod-n-l}
	$ -\tens{I} = 2E (\hat{N}^2 + \hat{L}^2 + \hbar^2) $.
\end{proposition}
\begin{proof}
	Dalla Prop. \ref{lrl-h-l}.
\end{proof}

È possibile riscrivere in forma nota le relazioni di commutazione definendo i 6 operatori:
\begin{equation}
	\hat{F}_{\pm}^i \defeq \frac{1}{2} (\hat{L}_i \pm \hat{N}_i)
	\label{eq:3.57}
\end{equation}

\begin{proposition}
	$ [\hat{F}_{\pm}^i, \hat{F}_{\pm}^j] = i\hbar \sum_{k = 1}^{3} \epsilon_{ijk} \hat{F}_{\pm}^k $.
\end{proposition}
\begin{proof}
	$ [\hat{F}_{\pm}^i, \hat{F}_{\pm}^j] = \frac{1}{4} [\hat{L}_i \pm \hat{N}_i, \hat{L}_j \pm \hat{N}_j] = \frac{i\hbar}{4} \sum_{k = 1}^{3} \epsilon_{ijk} (\hat{L}_k + \hat{L}_k \pm \hat{N}_k \pm \hat{N}_k) $.
\end{proof}

\begin{proposition}
	$ [\hat{F}_{\pm}^i, \hat{F}_{\mp}^i] = 0 $.
\end{proposition}
\begin{proof}
	$ [\hat{F}_{\pm}^i, \hat{F}_{\mp}^j] = \frac{1}{4} [\hat{L}_i \pm \hat{N}_i, \hat{L}_j \mp \hat{N}_j] = \frac{i\hbar}{4} \sum_{k = 1}^{3} \epsilon_{ijk} (\hat{L}_k - \hat{L}_k \pm \hat{N}_k \mp \hat{N}_k) = 0 $.
\end{proof}

Si riconoscono le relazioni di commutazione tra due operatori di momento angolare indipendenti: i 6 operatori $ \hat{F}_{\pm}^i $ possono essere presi come invarianti del sistema, dunque le simmetrie del problema portano ad avere due operatori compatibili $ \hat{\ve{F}}_{\pm} $ le cui componenti commutano con l'Hamiltoniana. Un insieme completo di operatori diagonalizzabili simultaneamente all'Hamiltoniana è quindi $ \hat{F}_+^2 $, $ \hat{F}_-^2 $, $ \hat{F}_+^z $ ed $ \hat{F}_-^z $.\\
Dato che $ \hat{\ve{F}}_{\pm} $ si comportano come operatori di momento angolare, si può affermare che la simmetria dell'Hamiltoniana \ref{eq:3.52} è data dal gruppo $ \SOn{3} \times \SOn{3} \cong \SOn{4} $, e da qui si capisce la somiglianza coi generatori del gruppo di Lorentz: $ \SOn{4} $ è il gruppo di tutte le isometrie di $ \R^4 $, ovvero quelle trasformazioni che lasciano invariata la forma quadratica $ x_1^2 + x_2^2 + x_3^2 + x_4^2 $, mentre $ \SOn{3,1} $ è il gruppo di tutte le isometrie di $ \R^{3,1} $, che lasciano invariata $ x_1^2 + x_2^2 + x_3^2 - x_0^2 $.

\begin{proposition}\label{f-p-f-m}
	$ \hat{F}_+^2 = \hat{F}_-^2 = \frac{1}{4} (\hat{L}^2 + \hat{N}^2) $.
\end{proposition}
\begin{proof}
	Banale ricordando che $ \hat{\ve{M}} \cdot \hat{\ve{L}} = 0 $, dunque $ \hat{\ve{N}} \cdot \hat{\ve{L}} = 0 $.
\end{proof}

Questo conferma che soltanto 5 delle 6 costanti del moto trovate sono indipendenti.

\begin{proposition}\label{f-spec}
	$ -\tens{I} = 2E (4\hat{F}_{\pm}^2 + \hbar^2) $.
\end{proposition}
\begin{proof}
	Dalle Prop. \ref{mod-n-l}-\ref{f-p-f-m}.
\end{proof}

La determinazione dello spettro d'energia è così ridotto alla determinazione dello spettro (identico) di due operatori momento angolare. Indicati con $ \ket{f,f_+^z,f_-^z} $ i loro comuni autostati:
\begin{equation}
	\hat{F}_{\pm}^2 \ket{f,f_+^z,f_-^z} = \hbar^2 f (f + 1) \ket{f,f_+^z,f_-^z}
	\label{eq:3.58}
\end{equation}
\begin{equation}
	\hat{F}_{\pm}^z \ket{f,f_+^z,f_-^z} = \hbar f_{\pm}^z \ket{f,f_+^z,f_-^z}
	\label{eq:3.59}
\end{equation}
con la condizione $ -f \le f_{\pm}^z \le f $. Questi sono anche autostati dell'Hamiltoniana, data la simultanea diagonalizzabilità, e l'autovalore corrispondente a $ \ket{f,f_+^z,f_-^z} $ è dato dalla Prop. \ref{f-spec}:
\begin{equation}
	E_f = -\frac{1}{2\hbar^2} \frac{1}{(1 + 2f)^2}
	\label{eq:3.60}
\end{equation}
È possibile definire il \textit{numero quantico principale} $ n \equiv 2f + 1 \in \N $, così da poter completamente determinare gli autostati con gli autovalori di $ \hat{\mathcal{H}} $, $ \hat{F}_+^z $ e $ \hat{F}_-^z $:
\begin{equation}
	\hat{\mathcal{H}} \ket{n,f_+^z,f_-^z} = E_n \ket{n,f_+^z,f_-^z}
	\label{eq:3.61}
\end{equation}
Dall'Eq. \ref{eq:3.45} si può ripristinare la giusta dimensionalità dell'autovalore dell'energia:
\begin{equation}
	E_n = - \frac{m (Ze^2)^2}{2\hbar^2} \frac{1}{n^2}
	\label{eq:3.62}
\end{equation}

\subsubsection{Degenerazione}

Dato che tutti gli autostati $ \ket{n,f_+,f_-} $ a fisso $ n $ sono associati allo stesso autovalore di energia, la degenerazione è determinata dai valori possibili di $ f_+ $ ed $ f_- $ (si ricordi $ -f \le f_{\pm} \le f $):
\begin{equation}
	d(n) = (2f + 1) (2f + 1) = n^2
	\label{eq:3.63}
\end{equation}
È preferibile esprimere gli autostati su una base che diagonalizzi operatori con un'interpretazione fisica immediata. Dato che $ \hat{\ve{F}}_+ + \hat{\ve{F}}_- = \hat{\ve{L}} $, è possibile scrivere gli autostati di $ \hat{F}_{\pm} $, $ \hat{F}_+^z $ ed $ \hat{F}_-^z $ sulla base di $ \hat{F}_{\pm} $, $ \hat{L}^2 $ ed $ \hat{L}_z $ grazie ai coefficienti di Clebsch-Gordan:
\begin{equation}
	\ket{n,f_+^z,f_-^z} = \sum_{\ell \ge 0} \sum_{m = -\ell}^{\ell} \ket{n,\ell,m} \braket{n,\ell,m | n,f_+^z,f_-^z}
	\label{eq:3.64}
\end{equation}
Dato che la condizione sulla somma di momenti angolari $ \hat{\ve{J}} = \hat{\ve{L}}_1 + \hat{\ve{L}}_2 $ è $ \abs{\ell_1 - \ell_2} \le j \le \ell_1 + \ell_2 $, poiché in questo caso $ f_+^z $ ed $ f_-^z $ variano entrambi tra $ -f $ ed $ f $ si ha che, fissato $ n $, $ \ell $ può variare tra $ 0 $ e $ 2f $:
\begin{equation}
	0 \le \ell \le n - 1
	\label{eq:3.65}
\end{equation}
Si trova dunque la corretta degenerazione, data ora dai valori di $ m $:
\begin{equation*}
	d(n) = \sum_{\ell = 0}^{n - 1} (2\ell + 1) = (n - 1) n + n = n^2
\end{equation*}

\subsubsection{Autofunzioni sulla base delle coordinate}

Le autofunzioni espresse sulla base delle coordinate possono essere scritte come:
\begin{equation}
	\braket{r,\vartheta,\varphi | n,\ell,m} = Y_{\ell,m}(\vartheta,\varphi) \phi_{n,\ell}(r)
	\label{eq:3.66}
\end{equation}
Queste possono essere costruite anche come autofunzioni di $ \hat{\ve{F}}_{\pm} $. Lo stato fondamentale $ \ket{1,0,0} $ è caratterizzato da $ f = 0 $, dunque ha autovalore nullo per qualsiasi $ \hat{F}_{\pm}^i $; in particolare:
\begin{equation}
	(\hat{F}_+^i \pm \hat{F}_-^i) \ket{1,0,0} = 0
	\label{eq:3.67}
\end{equation}
Per facilitare l'espressione sulla base delle coordinate:
\begin{equation}
	\hat{L}_i \ket{1,0,0} = 0
	\label{eq:3.68}
\end{equation}
\begin{equation}
	\hat{N}_i \ket{1,0,0} = 0
	\label{eq:3.69}
\end{equation}
La prima è banale notando che $ \ell = 0 $, mentre la seconda può essere riscritta ricordando che $ \hat{N}_i $ è proporzionale a $ \hat{M}_i $:
\begin{equation*}
	\left[ \sum_{j,k = 1}^{3} \epsilon_{ijk} \hat{p}_j \hat{L}_k - i\hbar \hat{p}_i - \frac{\hat{x}_i}{r} \right] \ket{1,0,0} = 0
\end{equation*}
Dall'Eq. \ref{eq:3.68} si annulla il primo termine, dunque moltiplicando per $ \frac{1}{r} \hat{x}_i $ e sommando su $ i = 1,2,3 $ si ottiene:
\begin{equation*}
	\left[ i\hbar \frac{\hat{\ve{x}}}{r} \cdot \hat{\ve{p}} + 1 \right] \ket{1,0,0} = 0
\end{equation*}
Dalle Eq. \ref{eq:29}-\ref{eq:30} si ottiene quindi una ODE di primo grado:
\begin{equation*}
	\left[ \hbar^2 \frac{d}{dr} + 1 \right] \phi_{1,0}(r) = 0
\end{equation*}
La risoluzione è banale:
\begin{equation*}
	\phi_{1,0}(r) = \mathcal{N} e^{- r / \hbar^2}
\end{equation*}
Ricordando la misura d'integrazione $ r^2 dr $ per $ r \in [0,\infty) $, si ottiene la normalizzazione usando il fatto che $ \int_0^{\infty} dx\, x^n e^{-\sigma x} = n! \sigma^{-(n + 1)} $:
\begin{equation*}
	\phi_{1,0}(r) = \frac{2}{\hbar^3} e^{- r / \hbar^2}
\end{equation*}
L'armonica sferica è costante e determinata solo dalla normalizzazione su $ \mathbb{S}^2 $, ovvero $ Y_{0,0}(\vartheta,\varphi) = \frac{1}{\sqrt{4\pi}} $, dunque l'autofunzione dello stato fondamentale risulta essere:
\begin{equation}
	\psi_{1,0,0}(r,\vartheta,\vartheta) = \frac{1}{\sqrt{\pi}\hbar^3} e^{- r / \hbar^2}
	\label{eq:3.70}
\end{equation}
Al netto del fattore $ \hbar^{-2} $, la dimensionalità dell'espressione si ritrova ricordando che il raggio è espresso in unità del raggio di Bohr, dunque bisogna porre $ r \mapsto (mZe^2) r $, così da ottenere:
\begin{equation}
	\psi_{1,0,0}(r,\vartheta,\varphi) = \frac{1}{\sqrt{\pi}} \frac{\left( m Z e^2 \right)^{2/3}}{\hbar^3} e^{ -\frac{mZe^2}{\hbar^2} r }
	\label{eq:3.71}
\end{equation}
Per costruire gli stati eccitati sono possibili due metodi. Il primo consiste nel costruirli come autostati di $ \hat{F}_{\pm}^2 $ e $ \hat{F}_{\pm}^z $, passando poi alla base di $ \hat{L}^2 $ e $ \hat{L}_z $ tramite i coefficienti di Clebsch-Gordan. Il secondo invece richiede di definire degli opportuni operatori di creazione e distruzione generalizzati; in questo caso, l'operatore di distruzione risulta essere:
\begin{equation}
	\hat{d}_{\ell} \defeq i \hat{p}_r + \frac{\hbar \ell}{r} - \frac{1}{\hbar \ell}
	\label{eq:3.72}
\end{equation}
e $ \hat{d}_{\ell}^{\dagger} $ l'operatore di creazione. Si dimostra infatti che:
\begin{equation}
	\hat{d}_{\ell + 1}^{\dagger} \ket{n,\ell,m} = \mathcal{N} \ket{n,\ell+1,m}
	\label{eq:3.73}
\end{equation}
\begin{equation}
	\hat{d}_{\ell} \ket{n,\ell,m} = \mathcal{N} \ket{n,\ell-1,m}
	\label{eq:3.74}
\end{equation}
Per ogni $ n $, si ha $ \hat{d}_n^{\dagger}\ket{n,n-1,m} = 0 $ (mentre $ \hat{d}_0 $ non è proprio definito), dunque:
\begin{equation*}
	\left[ -\hbar \left( \frac{d}{dr} + \frac{1}{r} \right) + \frac{\hbar n}{r} - \frac{1}{\hbar n} \right] \phi_{n,n-1}(r) = 0
	\quad \Rightarrow \quad
	\frac{d\phi_{n,n-1}}{\phi_{n,n-1}} = (n-1) \frac{dr}{r} - \frac{dr}{n\hbar^2}
\end{equation*}
La soluzione è, a meno della normalizzazione:
\begin{equation*}
	\phi_{n,n-1}(r) = \mathcal{N} r^{n-1} e^{- \frac{r}{n\hbar^2}}
\end{equation*}
Le altre autofunzioni si trovano applicando $ \hat{d}_n $, $ \hat{d}_{n-1} $ e così via, ed in generale si ha:
\begin{equation}
	\phi_{n,\ell}(r) = \mathcal{N} L_{n,\ell}(r) e^{-kr}
	\label{eq:3.75}
\end{equation}
dove $ L_{n,\ell}(r) $ è un polinomio di grado $ n - 1 $ (poiché $ \hat{d}_{\ell} $ non cambia il grado del polinomio), legato alla famiglia dei polinomi di Laguerre.










