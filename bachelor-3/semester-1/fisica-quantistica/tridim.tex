\selectlanguage{italian}

\section{Equazione di Schrödinger radiale}

Si consideri una generica Hamiltoniana invariante per rotazioni, ad esempio:
\begin{equation}
	\mathcal{H} = \frac{
	p^2}{2m} + V(r) = \frac{p_r^2}{2m} + \frac{L^2}{2mr^2} + V(r)
	\label{eq:3.1}
\end{equation}
È evidente che questa Hamiltoniana non si possa separare in parte radiale e parte angolare a causa del termine $ \frac{L^2}{r^2} $. È però possibile diagonalizzare simultaneamente $ \hat{H} $, $ \hat{L}^2 $ e $ \hat{L}_z $, dunque si proietta sugli autostati del momento angolare:
\begin{equation}
	\psi(\ve{x}) = \sum_{\ell = 0}^{\infty} \sum_{m = -\ell}^{\ell} \braket{\ve{x} | \ell,m} \braket{\ell,m | \psi} = \sum_{\ell = 0}^{\infty} \sum_{m = -\ell}^{\ell} Y_{\ell,m}(\vartheta,\varphi) \phi_{\ell,m}(r)
	\label{eq:3.2}
\end{equation}
L'equazione di Schrödinger si riduce quindi in una PDE con una sola incognita:
\begin{equation}
	\left[ \frac{p_r^2}{2m} + \frac{\hbar^2 \ell (\ell + 1)}{2mr^2} + V(r) \right] \phi_{\ell,m}(r) = E \phi_{\ell,m}(r)
	\label{eq:3.3}
\end{equation}
Questa non dipende da $ m $, dunque fissati $ \ell $ ed $ E $ c'è una degerazione di $ 2\ell + 1 $; si pone $ \phi_{\ell,m}(r) \equiv \phi_{\ell}(r) $.
È inoltre utile porre:
\begin{equation}
	\phi_{\ell}(r) \equiv \frac{u_{\ell}(r)}{r}
	\label{eq:3.4}
\end{equation}

\begin{proposition}
	$ \hat{p}_r^n \phi_{\ell}(r) = \left( -i\hbar \right)^n \frac{1}{r} \frac{\pa^n}{\pa r^n} u_{\ell}(r) $.
\end{proposition}
\begin{proof}
	$ \hat{p}_r \phi_{\ell}(r) = -i\hbar \left( \frac{\pa}{\pa r} + \frac{1}{r} \right) \frac{u_{\ell}(r)}{r} = -i\hbar \frac{1}{r} \frac{\pa}{\pa r} u_{\ell}(r) $.
\end{proof}
Una ragione $ \virgolette{fisica} $ per definire $ u_{\ell}(r) $ è che assorbe la misura d'integrazione nel prodotto scalare:
\begin{equation*}
	\braket{\psi' | \psi} = \int_{0}^{\infty} dr \,r^2 \phi'^*_{\ell'}(r) \phi_{\ell}(r) \int_{\mathbb{S}^2} d\cos\vartheta \,d\varphi \, Y^*_{\ell',m'}(\vartheta,\varphi) Y_{\ell,m}(\vartheta,\varphi) = \delta_{\ell,\ell'} \delta_{m,m'} \int_0^{\infty} dr \, u'^*_{\ell'}(r) u_{\ell}(r)
\end{equation*}
Ciò rende $ \hat{p}_r $ un operatore hermitiano sulle $ u_{\ell} $, ed infatti l'equazione di Schrödinger diventa:
\begin{equation}
	\left[ - \frac{\hbar^2}{2m} + \frac{\hbar^2 \ell (\ell + 1)}{2m r^2} + V(r) \right] u_{\ell}(r) = E u_{\ell}(r)
	\label{eq:3.5}
\end{equation}

\subsection{Condizioni al contorno}

È necessario che la funzione d'onda radiale $ \phi_{\ell}(r) $ abbia densità di probabilità integrabile su $ [0,+\infty) $: in particolare, si richiede che il seguente integrale non diverga:
\begin{equation}
	\braket{\phi_{\ell} | \phi_{\ell}} = \int_0^{\infty} dr \, r^2 \abs{\phi_{\ell}(r)}^2 = \int_{0}^{\infty} dr  \abs{u_{\ell}(r)}^2
	\label{eq:3.6}
\end{equation}
Nell'origine $ \abs{u_{\ell}(r)}^2 $ deve avere al più una singolarità integrabile, dunque:
\begin{equation}
	u_{\ell}(r) \overset{r \rightarrow 0}{\sim} \frac{1}{r^{\delta}} : \delta < \frac{1}{2}
	\label{eq:3.7}
\end{equation}
Dato che $ \lap \frac{1}{r} = - 4\pi \delta^3 (\ve{x}) $, se $ \phi_{\ell}(r) $ diverge nell'origine almeno come $ \frac{1}{r} $ è possibile soddisfare l'equazione di Schrödinger solo se il potenziale nell'origine diverge almeno come una delta di Dirac. Per potenziali non-distribuzionali (ovvero funzioni, quindi non singolari) $ \phi_{\ell}(r) $ deve divergere nell'origine meno di $ \frac{1}{r} $, dunque:
\begin{equation}
	\lim_{r \rightarrow 0} r \phi_{\ell}(r) = 0 \quad\Longrightarrow\quad \lim_{r \rightarrow 0} u_{\ell}(r) = 0
	\label{eq:3.8}
\end{equation}

\paragraph{Andamento nell'origine}

I potenziali d'interesse fisico sono quelli che per $ r \rightarrow 0 $ divergono meno di $ \frac{1}{r^2} $: il caso in cui nell'origine $ V(r) $ vada come $ \frac{1}{r^k} $ con $ k \ge 2 $ è patologicamente attrattivo e si dimostra non avere un'energia minima, ovvero non presenta stati stabili.\\
Se quindi nell'origine $ V(r) \sim \frac{1}{r^k} $ con $ k < 2 $, per $ r \rightarrow 0 $ a dominare è il termine centrifugo:
\begin{equation}
	- \frac{\hbar^2}{2m} \frac{d^2 u_{\ell}(r)}{d r^2} + \frac{\hbar^2 \ell (\ell + 1)}{2mr^2} u_{\ell}(r) = 0 \quad \Longrightarrow \quad \frac{d^2 u_{\ell}(r)}{d r^2} = \frac{\ell (\ell + 1)}{r^2} u_{\ell}(r)
	\label{eq:3.9}
\end{equation}
La soluzione generale di questa equazione è $ u_{\ell}(r) = A r^{\ell + 1} + B r^{-\ell} $, ma il secondo termine non soddisfa la condizione in Eq. \ref{eq:3.8}, dunque:
\begin{equation}
	u_{\ell}(r) \overset{r \rightarrow 0}{\sim} A r^{\ell + 1}
	\label{eq:3.10}
\end{equation}

\paragraph{Andamento all'infinito}

Se all'infinito il potenziale si annulla, per $ r \rightarrow \infty $ l'andamento della funzione d'onda è quello della particella libera: se esistono stati legati, ovvero autostati di $ \hat{\mathcal{H}} $ con $ E < 0 $, l'andamento della soluzione è:
\begin{equation}
	u_{\ell}(r) \overset{r \rightarrow \infty}{\sim} C e^{- \beta r}, \,\, \beta \equiv \frac{\sqrt{2m\abs{E}}}{\hbar}
	\label{eq:3.11}
\end{equation}
Se invece $ \lim_{r \rightarrow \infty} V(r) \neq 0 $, l'andamento va studiato caso per caso.

\paragraph{Stati legati}

Per un potenziale unidimensionale esiste sempre almeno uno stato legato; inoltre, lo stato fondamentale ha una funzione d'onda pari e gli stati eccitati hanno parità alternata.\\
Nel caso tridimensionale, è possibile interpretare il problema come un problema unidimensionale con dominio $ [0,\infty) $: la condizione in Eq. \ref{eq:3.8}, però, impone che solo le soluzioni dispari sono accettabili, dunque in generale non è detto che esista lo stato fondamentale. Per un potenziale tridimensionale, quindi, non è detto a priori che esistano stati legati.

\section{Particella libera}

L'equazione di Schrödinger per la particella libera, quindi per $ V(\ve{x}) = 0 $, è:
\begin{equation}
	-\frac{\hbar^2}{2m} \lap \psi(\ve{x}) = E \psi(\ve{x})
	\label{eq:3.12}
\end{equation}
Questa è una PDE separabile e, in coordinate cartesiane, le soluzioni sono delle onde piane:
\begin{equation}
	\psi_{\ve{k}}(\ve{x}) = \frac{1}{(2\pi)^{3/2}} e^{i \ve{k}\cdot\ve{x}}, \,\, E = \frac{\hbar^2 \ve{k}^2}{2m}
	\label{eq:3.13}
\end{equation}
normalizzate in senso improprio:
\begin{equation}
	\braket{\psi_{\ve{k}'} | \psi_{\ve{k}}} = \int_{\R^3} d^3\ve{x}\, \psi^*_{\ve{k}'}(\ve{x}) \psi_{\ve{k}}(\ve{x}) = \delta^3(\ve{k} - \ve{k}')
	\label{eq:3.14}
\end{equation}
Un differente approccio risolutivo è quello che sfrutta la simmetria rotazionale del problema:
\begin{equation}
	\psi_{\ell,m}(\ve{x}) = Y_{\ell,m}(\vartheta,\varphi) \frac{u_{\ell}(r)}{r}
	\label{eq:3.15}
\end{equation}
Dall'Eq. \ref{eq:3.5}:
\begin{equation}
	\frac{\hbar^2}{2mE} \left[ -\frac{d^2}{dr^2}u_{\ell}(r) + \frac{\ell(\ell + 1)}{r^2}u_{\ell}(r) \right] - u_{\ell}(r) = 0
	\label{eq:3.16}
\end{equation}
Ponendo $ k = \sqrt{\frac{2mE}{\hbar^2}} $ e $ r' = kr $, si ottiene:
\begin{equation}
	\frac{d^2}{dr'^2} u_{\ell}(r') - \frac{\ell(\ell + 1)}{r'^2} u_{\ell}(r') + u_{\ell}(r') = 0
	\label{eq:3.17}
\end{equation}
Questa è una ODE di Bessel e la sua soluzione si ottiene introducendo la funzione di Bessel $ j_{\ell}(r') $ (è noto che $ j_{\ell}(x) \sim x^{\ell} $ per $ x \rightarrow 0 $):
\begin{equation}
	u_{\ell}(r') = r j_{\ell}(r')
	\label{eq:3.18}
\end{equation}
Le autofunzioni della particella libera tridimensionale sono dunque:
\begin{equation}
	\psi_{\ell,m}(\ve{x}) = Y_{\ell,m}(\vartheta,\varphi) j_{\ell}(kr), \,\, k = \frac{\sqrt{2mE}}{\hbar}
	\label{eq:3.19}
\end{equation}

\section{Oscillatore armonico isotropo}

L'oscillatore armonico isotropo è descritto dal seguente potenziale:
\begin{equation}
	V(r) = \frac{1}{2} m\omega^2 r^2
	\label{eq:3.20}
\end{equation}
Essendo un potenziale centrale, il problema si riduce al problema radiale.
L'Hamiltoniana radiale corrispondente è:
\begin{equation}
	\mathcal{H}_{\ell} = \frac{p_r^2}{2m} + \frac{\hbar^2 \ell(\ell + 1)}{2mr^2} + \frac{1}{2} m\omega^2 r^2
	\label{eq:3.21}
\end{equation}
La sua equazione agli autovalori può essere scritta come:
\begin{equation}
	\hat{\mathcal{H}}_{\ell} \ket{n,\ell,m} = E_{n,\ell} \ket{n,\ell,m}
	\label{eq:3.22}
\end{equation}

\subsection{Stati con \texorpdfstring{$ \ell = 0 $}{TEXT}}

Per $ \ell = 0 $, il problema si riduce all'oscillatore armonico unidimensionale:
\begin{equation}
	\mathcal{H}_0 = \frac{p_r^2}{2m} + \frac{1}{2} m \omega^2 r^2
	\label{eq:3.23}
\end{equation}
Per risolvere con metodo algebrico, si defisce l'operatore di distruzione come:
\begin{equation}
	\hat{d}_0 \defeq \sqrt{\frac{m\omega}{2\hbar}} \left( \hat{r} + i \frac{\hat{p}_r}{m\omega} \right)
	\label{eq:3.24}
\end{equation}
ed il relativo operatore di creazione $ d_0^{\dagger} $. In questo modo, si può scrivere:
\begin{equation}
	\hat{\mathcal{H}}_0 = \hbar \omega \left( \hat{d}_0^{\dagger}\hat{d}_0 + \frac{1}{2} \right)
	\label{eq:3.25}
\end{equation}
Dato che $ \left[ \hat{r},\hat{p}_r \right] = i\hbar $, si ha $ \left[ \hat{d}_0, \hat{d}_0^{\dagger} \right] = 1 $, dunque lo spettro di $ \hat{\mathcal{H}}_0 $ è lo stesso dell'oscillatore armonico unidimensionale:
\begin{equation}
	u_{n,0}(r) = \mathcal{N}_n e^{-c x^2} H_n(r)
	\label{eq:3.26}
\end{equation}
Dato che $ u_{n,0}(-x) = (-1)^n u_{n,0}(x) $, per la condizione al contorno in Eq. \ref{eq:3.8} sono ammissibili solo le soluzioni dispari. Ridefinendo $ n $ così che $ u_{n,0}(r) $ sia la $ (2n+1) $-esima soluzione, si trova lo spettro dell'Hamiltoniana:
\begin{equation}
	E_{n,0} = \hbar \omega \left( 2n + \frac{3}{2} \right)
	\label{eq:3.27}
\end{equation}
Dato che il caso $ \ell = 0 $ è quello maggiormente attrattivo, per l'assenza del termine centrifugo, si trova che lo stato fondamentale è $ \ket{0,0,0} $, con energia $ E_{0,0} = \frac{3}{2} \hbar \omega $.

\subsection{Stati con \texorpdfstring{$ \ell $}{TEXT} generico}

Per scrivere l'Hamiltoniana $ \hat{\mathcal{H}}_{\ell} $ in una forma simile a Eq. \ref{eq:3.25} è necessario definire degli operatori di distruzione/creazione generalizzati:
\begin{equation}
	\hat{d}_{\ell} \defeq \sqrt{\frac{m\omega}{2\hbar}} \left( \left( \hat{r} + \frac{\hbar\ell}{m\omega\hat{r}} \right) + i \frac{\hat{p}_r}{m\omega} \right)
	\label{eq:3.28}
\end{equation}
È utile inoltre ricordare che, essendo $ \left[ \hat{p}_r, f(\hat{r}) \right] = -i\hbar f(\hat{r}) $, si ha $ \left[ \hat{p}_r, \frac{1}{\hat{r}} \right] = \frac{i\hbar}{\hat{r}^2} $.\\
Si può generalizzare l'operatore numero come:
\begin{equation}
	\hat{D}_{\ell} \defeq \hat{d}_{\ell}^{\dagger} \hat{d}_{\ell}
	\label{eq:3.29}
\end{equation}
La sua espressione eplicita è:
\begin{equation*}
	\begin{split}
		D_{\ell}
		&= d_{\ell}^{\dagger} d_{\ell} = \frac{m\omega}{2\hbar} \left( \left( r + \frac{\hbar\ell}{m\omega r} \right) - i \frac{p_r}{m\omega} \right) \left( \left( r + \frac{\hbar\ell}{m\omega r} \right) + i \frac{p_r}{m\omega} \right)\\
		&= \frac{m\omega}{2\hbar} \left( \left( r + \frac{\hbar\ell}{m\omega r} \right)^2 + \frac{p_r^2}{m^2 \omega^2} + \left[ \left( r + \frac{\hbar\ell}{m\omega r} \right), i \frac{p_r}{m\omega} \right] \right)\\
		&= \frac{m\omega}{2\hbar} \left( r^2 + \frac{\hbar^2\ell^2}{m^2 \omega^2 r^2} + \frac{p_r^2}{m^2 \omega^2} + \frac{2\hbar\ell}{m\omega} - \frac{i}{m\omega} \left( -i\hbar + \frac{\hbar\ell}{m\omega} \frac{i\hbar}{r^2} \right) \right)\\
		&= \frac{1}{\hbar\omega} \left( \frac{p_r^2}{2m} + \frac{1}{2}m\omega^2 r^2 + \frac{\hbar^2 \ell (\ell + 1)}{2mr^2} \right) + \ell - \frac{1}{2}\\
	\end{split}
\end{equation*}
Risulta quindi che:
\begin{equation}
	\hat{D}_{\ell} = \frac{1}{\hbar\omega} \hat{\mathcal{H}}_{\ell} + \ell - \frac{1}{2}
	\label{eq:3.30}
\end{equation}
Di conseguenza, $ \hat{D}_{\ell} $ e $ \hat{\mathcal{H}}_{\ell} $ hanno gli stessi autostati $ \ket{n,\ell} $:
\begin{equation}
	\hat{D}_{\ell} \ket{n,\ell} = \mathcal{E}_{n,\ell} \ket{n,\ell}, \,\, \mathcal{E}_{n,\ell} = \frac{1}{\hbar \omega} E_{n,\ell} + \ell - \frac{1}{2}
	\label{eq:3.31}
\end{equation}
Ovviamente $ \hat{D}_0 $ è il consueto operatore numero: infatti, il suo spettro è $ \mathcal{E}_{n,0} = 2n + 1 $.\\
È necessario definire un ulteriore operatore:
\begin{equation}
	\hat{\overline{D}}_{\ell} \defeq \hat{d}_{\ell} \hat{d}_{\ell}^{\dagger}
	\label{eq:3.32}
\end{equation}
Con un calcolo analogo al precedente, si trova che:
\begin{equation}
	\hat{\overline{D}}_{\ell} = \frac{1}{\hbar \omega} \hat{\mathcal{H}}_{\ell - 1} + \ell + \frac{1}{2}
	\label{eq:3.33}
\end{equation}
Risulta evidente quindi che:
\begin{equation}
	\hat{\overline{D}}_{\ell + 1} = \hat{D}_{\ell} + 2
	\label{eq:3.34}
\end{equation}
Questi operatori legano gli spettri di Hamiltoniane con $ \ell $ diversi: si supponga di avere $ \ket{n,\ell} $ autostato di $ \hat{\mathcal{H}}_{\ell} $ e $ \hat{D}_{\ell} $: $ \hat{d}_{\ell + 1}^{\dagger} \ket{m,\ell} $ è un autostato di $ \hat{\mathcal{H}}_{\ell + 1} $ e $ \hat{D}_{\ell + 1} $, dato che:
\begin{equation*}
	\hat{D}_{\ell + 1} \hat{d}_{\ell + 1}^{\dagger} \ket{n,\ell} = \hat{d}_{\ell + 1}^{\dagger} \hat{d}_{\ell + 1} \hat{d}_{\ell + 1}^{\dagger} \ket{n,\ell} = \hat{d}_{\ell + 1}^{\dagger} (\hat{D}_{\ell} + 2) \ket{n,\ell} = (\mathcal{E}_{n,\ell} + 2) \hat{d}_{\ell + 1}^{\dagger} \ket{n,\ell}
\end{equation*}
Questi operatori sono simili a degli operatori di scala:
\begin{equation}
	\hat{D}_{\ell + 1} \hat{d}_{\ell + 1} \ket{n,\ell} = (\mathcal{E}_{n,\ell} + 2) \hat{d}_{\ell + 1}^{\dagger} \ket{n,\ell}
	\label{eq:3.35}
\end{equation}
\begin{equation}
	\hat{D}_{\ell - 1} \hat{d}_{\ell} \ket{n,\ell} = (\mathcal{E}_{n,\ell} - 2) \hat{d}_{\ell} \ket{n,\ell}
	\label{eq:3.36}
\end{equation}
In questo modo, è possibile costruire l'intero spettro, ricordando che $ \mathcal{E}_{n,0} = 2n + 1 $:
\begin{equation*}
	\begin{split}
		&\hat{D}_1 \hat{d}_1^{\dagger} \ket{n,0} = (\mathcal{E}_{n,0} + 2) \hat{d}_1^{\dagger} \ket{n,0} = (2n + 3) \hat{d}_1^{\dagger} \ket{n,0}\\
		&\hat{D}_2 \hat{d}_2^{\dagger} \hat{d}_1^{\dagger} \ket{n,0} = (\mathcal{E}_{n,0} + 4) \hat{d}_2^{\dagger} \hat{d}_1^{\dagger} \ket{n,0} = (2n + 5) \hat{d}_2^{\dagger} \hat{d}_1^{\dagger} \ket{n,0}\\
		&\qquad\quad\vdots\\
		&\hat{D}_{\ell + 1} \hat{d}_{\ell + 1}^{\dagger} \dots \hat{d}_1^{\dagger} \ket{n,0} = (2n + 2\ell + 1) \hat{d}_{\ell + 1}^{\dagger} \dots \hat{d}_1^{\dagger} \ket{n,0}
	\end{split}
\end{equation*}
Si evince che $ \mathcal{E}_{n,\ell} = 2n + 2\ell + 1 $, quindi dall'Eq. \ref{eq:3.31} si ricava lo spettro dell'$ \ell $-esima Hamiltoniana:
\begin{equation}
	E_{n,\ell} = \hbar \omega \left( 2n + \ell + \frac{3}{2} \right)
	\label{eq:3.37}
\end{equation}
Questi sono tutti e soli i possibili autovalori di $ \hat{\mathcal{H}}_{\ell} $: se per assurdo esistesse un suo autostato con un autovalore non presente in questa sequenza, ci si potrebbe comunque sempre ricondurre ad uno stato con $ \ell = 0 $ applicando ripetutamente $ \hat{d}_{\ell} $, ottenendo così un nuovo autovalore di $ \hat{\mathcal{H}}_0 $, il che è assurdo poiché il suo spettro completo (Eq. \ref{eq:3.27}) è dato dall'Eq. \ref{eq:3.37}.










