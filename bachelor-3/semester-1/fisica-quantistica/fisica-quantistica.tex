\documentclass[a4paper, 12pt, openany]{book}
\usepackage[utf8]{inputenc}
\usepackage[italian]{babel}

\usepackage[]{csvsimple}
\usepackage{float}

\usepackage{ragged2e}
\usepackage[left=25mm, right=25mm, top=15mm]{geometry}
\geometry{a4paper}
\usepackage{graphicx}
\usepackage{booktabs}
\usepackage{paralist}
\usepackage{subfig} 
\usepackage{fancyhdr}
\usepackage{amsmath}
\usepackage{amssymb}
\usepackage{amsfonts}
\usepackage{amsthm}
\usepackage{mathtools}
\usepackage{enumitem}
\usepackage{titlesec}
\usepackage{braket}
\usepackage{gensymb}
\usepackage{url}
\usepackage{hyperref}
\usepackage{csquotes}
\usepackage{multicol}
\usepackage{graphicx}
\usepackage{wrapfig}
\usepackage{caption}

\usepackage{esint}

\captionsetup{font=small}
\pagestyle{fancy}
\renewcommand{\headrulewidth}{0pt}
\lhead{}\chead{}\rhead{}
\lfoot{}\cfoot{\thepage}\rfoot{}
\usepackage{sectsty}
\usepackage[nottoc,notlof,notlot]{tocbibind}
\usepackage[titles,subfigure]{tocloft}
\renewcommand{\cftsecfont}{\rmfamily\mdseries\upshape}
\renewcommand{\cftsecpagefont}{\rmfamily\mdseries\upshape}

\let\oldsection\section% Store \section
\renewcommand{\section}{% Update \section
	\renewcommand{\theequation}{\thesection.\arabic{equation}}% Update equation number
	\oldsection}% Regular \section
\let\oldsubsection\subsection% Store \subsection
\renewcommand{\subsection}{% Update \subsection
	\renewcommand{\theequation}{\thesubsection.\arabic{equation}}% Update equation number
	\oldsubsection}% Regular \subsection

\newcommand{\abs}[1]{\left\lvert#1\right\rvert}
\newcommand{\norm}[1]{\left\lVert#1\right\rVert}

\newcommand{\g}{\text{g}}
\newcommand{\m}{\text{m}}
\newcommand{\cm}{\text{cm}}
\newcommand{\mm}{\text{mm}}
\newcommand{\s}{\text{s}}
\newcommand{\N}{\text{N}}
\newcommand{\Hz}{\text{Hz}}

\newcommand{\virgolette}[1]{``\text{#1}"}
\newcommand{\tildetext}{\raise.17ex\hbox{$\scriptstyle\mathtt{\sim}$}}

\renewcommand{\arraystretch}{1.2}

\addto\captionsenglish{\renewcommand{\figurename}{Fig.}}
\addto\captionsenglish{\renewcommand{\tablename}{Tab.}}

\DeclareCaptionLabelFormat{andtable}{#1~#2  \&  \tablename~\thetable}

\setlength{\parindent}{0pt}


\author{Leonardo Cerasi%
	\thanks{\scriptsize\href{mailto:leonardo.cerasi@studenti.unimi.it}{leo.cerasi@pm.me}}%
	, Lucrezia Bioni\\
	\small GitHub repository: \href{https://github.com/LeonardoCerasi/notes}{LeonardoCerasi/notes}}

\title{\Huge\textbf{Fisica Quantistica 2} \\ \large Prof. S. Forte, a.a. 2024-25}

\begin{document}

\frontmatter

\maketitle

\tableofcontents
\pagestyle{indice}

\mainmatter

\chapter*{Introduzione}
\pagestyle{introd}
\addcontentsline{toc}{chapter}{Introduzione}
\markboth{Introduzione}{}
\selectlanguage{italian}

\paragraph{Scale di grandezza}

Nello studio della fisica dei nuclei e delle particelle subatomiche, le scale di grandezza tipiche sono estremamente piccole: la scala tipica delle dimensioni di un atomo è $ 1\ang = 10^{-10}\m  $, mentre quella del nucleo è di $ 4 $ ordini di grandezza minore ($ 10^{-14}\m = 10\fm $); per un singolo nucleone, invece, le dimensioni sono dell'ordine di $ 1\fm = 10^{-15}\m $, e il range tipico delle interazioni deboli è $ 10^{-18}\m $.\\
Per quanto riguarda la scala di energie, i processi atomici hanno energie solitamente attorno a $ 1\ev = 1.602\cdot10^{-19}\,\text{J} $, mentre quelli nucleari arrivano anche a $ 10\mev $; per le interazioni ad alte energie studiate dalla fisica particellare, i moderni accelleratori arrivano a scale di $ 1\tev $.\\
Per studiare la struttura dei costituenti della materia a vari livelli, è necessario utilizzare fasci di particelle (fotoni, elettroni, etc.) con determinate lunghezze d'onda (relazioni di de Broglie $ \lambda = \frac{h}{p} $), corrispondenti a determinate energie: per sondare i nuclei atomici sono necessari $ \lambda \sim 10\fm $ ed $ E \sim 1\mev $; per evidenziare la struttura a molti corpi del nucleo atomico servono $ \lambda \sim 1\fm $ ed $ E \sim 10\mev $; se si vogliono studiare gli stati eccitati dei singoli nucleoni occorrono $ \lambda \sim 10^{-3}\fm $ ed $ E \sim 1\gev $; infine, se si vuole mettere in luce la struttura composta da quark dei nucleoni, bisogna raggiungere $ \lambda < 10^{-4}\fm $ ed $ E > 200\gev $.

\paragraph{Interazioni fondamentali}

I vari costituenti della materia interagiscono tramite $ 4 $ interazioni fondamentali:
\begin{enumerate}
  \item interazione elettromagnetica: mediata dal fotone ($ m_{\gamma} = 0 $), con coupling constant $ \alpha \approx 1/137 $ e raggio d'azione infinito (essendo il fotone massless);
  \item interazione debole: mediata dai bosoni $ \w^{\pm} $ e $ \z^0 $ ($ m_W = 80.4\gev $, $ m_Z = 90.1\gev $), con coupling constant $ G_F \approx 1\cdot10^{-5} $) e raggio d'azione $ < 10^{-3}\fm $, dovuto al fatto che i bosoni bosoni $ \w^{\pm} $ e $ \z^0 $ sono molto pesanti e dunque, per il principio d'indeterminazione ($ \Delta E \Delta t \ge \frac{\hbar}{2} $), possono essere prodotti solo come particelle virtuali in processi di scattering per periodi di tempo estremamente brevi;
  \item interazione forte: mediata dai gluoni ($ m_g = 0 $), con coupling constant $ \alpha_s \approx 1 $ e raggio d'azione $ \approx 1\fm $, dovuto al fatto che i gluoni, sebbene massless, possono interagire tra loro;
  \item interazione gravitazionale: mediata dall'ipotetico gravitone ($ m_G = 0 $), con coupling constant $ G_N \approx 6\cdot10^{-39} $ e raggio d'azione infinito.
\end{enumerate}
Come si può notare, la gravità ha un'intensità di decine di ordini di grandezza inferiore alle altre interazioni fondamentali, per questo in ambito atomico, nucleare e particellare può essere trascurata.

\paragraph{Esperimenti}

A differenza della fisica atomica, che è descritta completamente dalla QED (Quantum Electrodynamics), la fisica nucleare non ha un'unica teoria coerente: la teoria fondamentale dell'interazione forte, la QCD (Quantum Chromodynamics), descrive le interazioni tra quark (mediate da gluoni), non quelle tra nucleoni (mediate da mesoni virtuali); inoltre, in ambito atomico le energie che entrano in gioco nei decadimenti ($ \sim 10\mev $) sono meno dello $ 0.1\% $ della massa del nucleo (espressa in unità naturali), dunque gli effetti relativistici possono essere ignorati, mentre per quanto riguarda i processi tra nucleoni le energie possono essere anche $ 100 $ volte la massa equivalente del protone, rendendo necessario l'utilizzo della meccanica quantistica relativistica; infine, bisogna considerare che sia il nucleo atomico che i nucleoni sono sistemi complessi a molti corpi, dunque, anche avendo una teoria dell'interazione tra singole coppie di costituenti, è estremamente difficile sviluppare modelli teorici per descrivere questi sistemi, e la trattazione è principalmente di natura fenomenologica, con tante teorie dei singoli processi che vengono sviluppate a partire dai dati sperimentali.\\
Gli esperimenti in fisica nucleare (utilizzati anche per studiare gli adroni in generale) sono principalmente di due tipi:
\begin{enumerate}
  \item scattering: un fascio di particelle, di cui si conoscono energia e momento lineare, è diretto verso l'oggetto bersagio da studiare e, attraverso le variazioni di quantità cinematiche misurabili, è possibile studiare le proprietà dell'interazioni e la struttura del bersaglio (risoluzione data dalla relazione di de Broglie);
  \item spettroscopia: nucleoni (o anche mesoni e barioni) vengono eccitati e si studiano i prodotti del decadimento di questi stati eccitati, inferendo le proprietà degli stati eccitati e le interazioni tra i prodotti di decadimento.
\end{enumerate}
Sia esperimenti di scattering che esperimenti spettroscopici necessitano di energie di ordini di grandezza simili.\\
Nel caso dello scattering è importante studiare la sezione d'urto d'interazione (cross section), ovvero la probabilità che avvenga una determinata reazione: in base all'angolo solido $ \Delta\Omega $ del rilevatore, alla cross section $ \frac{d\sigma}{s\Omega} $, all'intensità $ I_0 $ del fascio incidente e alla densità numerica di particelle $ n_0 $ che attraversano lo spessore $ dz $ del rilevatore, si può calcolare il numero di particelle rilevate in funzione dell'angolo d'emissione:
\begin{equation}
  \frac{dn(\theta)}{dt} = I_0 n_0 dz \frac{d\sigma}{d\Omega} \Delta\Omega
  \label{eq:1}
\end{equation}
La cross section è un'area geometrica (l'area effettiva di collisione) ed è solitamente misurata in barn: $ 1\barn = 100\fm^2 $; questa sezione d'urto è in realtà molto grande e misure più tipiche sono espresse in microbarn.

\thispagestyle{introd}


\part{Meccanica Quantistica in più Dimensioni}
\pagestyle{body}

\chapter{Sistemi Quantistici Multidimensionali}
\selectlanguage{italian}

\section{Spazio prodotto diretto}

Per definire formalmente i sistemi quantistici in più dimensioni, è necessario definire prima il prodotto diretto tra spazi di Hilbert.

\begin{definition}
	Dati due spazi di Hilbert $ \hilb $ e $ \mathscr{K} $ con basi $ \{\ket{e_i}\} $ e $ \{\ket{\tilde{e}_j}\} $, si definisce il loro prodotto diretto come $ \hilb \otimes \mathscr{K} \defeq \{ \ket{\psi} = \sum_{i,j} c_{ij} \ket{e_i} \otimes \ket{\tilde{e}_j} \} $. In questo spazio si definisce il prodotto scalare tra due vettori $ \ket{\psi_1} = \ket{e_{i_1}}\otimes\ket{\tilde{e}_{j_1}} $ e $ \ket{\psi_2} = \ket{e_{i_2}}\otimes\ket{\tilde{e}_{j_2}} $ come $ \braket{\psi_1 | \psi_2} = \braket{e_{i_1} | e_{i_2}}\braket{\tilde{e}_{j_1} | \tilde{e}_{j_2}} $.
\end{definition}

Per semplificare la scrittura, si adotta la notazione $ \ket{e_i}\otimes\ket{e_j} \equiv \ket{e_i e_j} $ (o si sottintende $ \otimes $).\\
Si noti che osservabili relative a spazi diversi sono sempre compatibili.\\
In generale, il generico $ \ket{\psi}\in\hilb\otimes\mathscr{K} $ non è scrivibile come prodotto diretto $ \ket{\psi} = \ket{\phi}\otimes\ket{\tilde{\phi}} $, con $ \ket{\phi}\in\hilb $ e $ \ket{\tilde{\phi}}\in\mathscr{K} $, poiché in generale non è detto che $ c_{ij} $ sia fattorizzabile in $ \alpha_i $ e $ \tilde{\alpha}_j $: in questo caso si dice che lo stato è entangled.

\begin{definition}
	Uno stato $ \ket{\psi}\in\hilb\otimes\mathscr{K} $ si dice entangled se non è fattorizzabile.
\end{definition}

\begin{example}
	Dati due qubit, uno stato entangled è $ \ket{\psi} = \frac{1}{\sqrt{2}} \left(\ket{01} + \ket{10}\right) $, dato che il generico stato fattorizzabile è $ \left(a\ket{0} + b\ket{1}\right)\otimes\left(c\ket{0} + d\ket{1}\right) = ac\ket{00} + ad\ket{01} + bc\ket{10} + bd\ket{11} $.
\end{example}

La probabilità $ P_{ij} = \abs{c_{ij}}^2 $ è detta probabilità congiunta: in generale essa non è il prodotto delle probabilità dei singoli eventi per i fenomeni di interferenza quantistica, i quali rendono tale probabilità dipendente dallo stato dell'intero sistema.

\section{Sistemi multidimensionali}

Per generalizzare la meccanica quantistica in $ d $ dimensioni, si introduce l'operatore posizione $ \hat{\ve{x}} $:
\begin{equation}
	\hat{\ve{x}} \defeq
	\begin{pmatrix}
		\hat{x}_1 \\
		\vdots \\
		\hat{x}_d
	\end{pmatrix}
	\label{eq:1}
\end{equation}
Ciascuna componente di questo vettore è un operatore hermitiano che agisce su uno spazio di Hilbert, mentre il vettore $ \hat{\ve{x}} $ agisce sul loro prodotto diretto $ \hilb \defeq \hilb_1 \otimes \dots \otimes \hilb_d $. Su ciascuno spazio $ \hilb_j $ viene definita la base delle posizioni da $ \hat{x}_j \ket{x_j} = x_j \ket{x_j} $, dunque la base delle posizioni in $ \hilb $ sarà $ \ket{\ve{x}} \defeq \ket{x_1} \otimes \dots \otimes \ket{x_d} $: data $ \ket{\psi} \in \hilb $, la sua rappresentazione sulla base delle posizioni è $ \braket{\ve{x} | \psi} \equiv \psi(\ve{x}) $, con $ \psi : \R^d \rightarrow \C $, il cui modulo quadro dà una densità di di probabilità $ d $-dimensionale $ dP_{\ve{x}} = \abs{\psi(\ve{x})}^2 d^d\ve{x} $.\\
In questo caso, l'entanglement consiste nel fatto che, in generale, $ \psi(\ve{x}) \neq \psi_1(x_1) \dots \psi_d(x_d) $.\\
Tale formalismo è generalizzabile al caso di $ n $ corpi in $ d $ dimensioni, nel qual caso si ha uno spazio prodotto diretto di $ nd $ spazi di Hilbert.

\begin{example}
	Nel caso di $ 2 $ corpi in $ 3 $ dimensioni, si ha:
	\begin{equation*}
		\hat{\ve{x}} =
		\begin{pmatrix}
			\hat{x}_{1,1} \\
			\hat{x}_{1,2} \\
			\hat{x}_{1,3} \\
			\hat{x}_{2,1} \\
			\hat{x}_{2,2} \\
			\hat{x}_{2,3} \\
		\end{pmatrix}
	\end{equation*}
	In questo sistema, la funzione d'onda è $ \braket{\ve{x}_1 \ve{x}_2 | \psi} = \psi(\ve{x}_1, \ve{x}_2) $.
\end{example}

\begin{equation}
	\braket{\ve{x}' | \ve{x}} = \delta^{(d)}(\ve{x} - \ve{x}')
	\label{eq:2}
\end{equation}

La $ \delta^{(d)} $ è il prodotto di $ d $ delte di Dirac ed è definita da $ \int_{\R^d} d^d\ve{x} \, \delta^{(d)}(\ve{x} - \ve{x}') f(\ve{x}) = f(\ve{x}') $ come distribuzione.

\subsection{Coordinate cartesiane}

Analogamente al caso monodimensionale, per definite l'operatore impulso si considera una traslazione spaziale; le componenti del vettore operatore impulso $ \hat{\ve{p}} $ sulla base delle posizioni sono definite da:

\begin{equation}
  \braket{\ve{x} | \hat{p}_j | \psi} = - i\hbar \frac{\pa}{\pa x_j} \psi(\ve{x})
	\label{eq:3}
\end{equation}
In forma vettoriale, è possibile scrivere:
\begin{equation}
	\hat{\ve{p}} = -i\hbar\nabla
	\label{eq:4}
\end{equation}

A questo punto, è facile definite le autofunzioni dell'impulso tali per cui $ \hat{\ve{p}}\ket{\ve{k}} = \hbar \ve{k}\ket{\ve{k}} $:

\begin{equation}
	\braket{\ve{x} | \ve{k}} = \frac{1}{(2\pi)^{d/2}} e^{i \ve{k}\cdot\ve{x}} \equiv \psi_{\ve{k}}(\ve{x})
	\label{eq:5}
\end{equation}

Il fatto che operatori su spazi diversi commutino tra loro implica che:

\begin{equation}
	\left[ \hat{p}_j, \hat{p}_k \right] = 0 \quad \forall j,k = 1, \dots, d
	\label{eq:6}
\end{equation}

Dal punto di vista matematico, questo è ovvio per il lemma di Schwarz (assumendo una well-behaved $ \psi $), mentre da quello fisico ciò esprime il fatto che traslazioni lungo assi diversi commutano tra loro: ciò non è scontato, infatti ad esempio le rotazioni rispetto ad assi diversi non commutano (dunque le componenti del momento angolare non commuteranno).\\
%
È facile vedere che $ \hat{\ve{p}}^2 = -\hbar^2 \lap $, dunque è possibile definire l'Hamiltoniana del sistema (e con essa la sua evoluzione temporale):

\begin{equation}
	\mathcal{H} = - \frac{\hbar^2}{2m} \lap + V(\ve{x})
	\label{eq:7}
\end{equation}

Ricordando che $ \hat{\mathcal{H}}\ket{\psi} = E\ket{\psi} $, si ottiene l'equazione di Schrödinger sulla base delle coordinate:
\begin{equation}
	- \frac{\hbar^2}{2m} \lap \psi(\ve{x}) + V(\ve{x}) \psi(\ve{x}) = E \psi(\ve{x})
	\label{eq:8}
\end{equation}

\section{Separabilità}

Nel caso di sistemi non-entangled, è possibile separare il problema multidimensionale in $ d $ problemi monodimensionali e scrivere la soluzione come prodotto delle soluzioni dei problemi ridotti.

\subsection{Problemi separabili in coordinate cartesiane}

\begin{proposition}
	In coordinate cartesiane, condizione sufficiente affinché il problema sia separabile è che:
	\begin{equation}
		V(\ve{x}) = V_1(x_1) + \dots + V_d(x_d)
		\label{eq:9}
	\end{equation}
\end{proposition}

In tal caso, l'Hamiltoniana del sistema è somma di $ d $ sotto-Hamiltoniane (e di conseguenza lo è anche l'evoluzione temporale):

\begin{equation}
	\hat{\mathcal{H}}_j = \frac{\hat{p}_j^2}{2m} + \hat{V}(x_j)
	\label{eq:10}
\end{equation}












\chapter{Momento Angolare}
\selectlanguage{italian}

\section{Momento angolare e rotazioni}

\paragraph{Caso classico}

Per il Th. di Noether, associate alle invarianze per rotazioni attorno ai tre assi coordianti si hanno tre cariche di Noether conservate.\\
Si considerino $ \ve{x} = \left( r\cos \phi, r\sin \phi \right) \equiv \left( x_1, x_2 \right) $ nel piano $ z = 0 $ ed una rotazione attorno all'asse $ z $ di un angolo infinitesimo $ \varepsilon $: questa causa uno spostamento $ \delta\ve{x} $ dato da:
\begin{equation*}
	\begin{split}
		\delta\ve{x}
		&= \left( r\cos (\phi + \varepsilon), r\sin (\phi + \varepsilon) \right) - \left( r\cos \phi, r\sin \phi \right)\\
		&= \left( -r \varepsilon \sin \phi, r \varepsilon \cos \phi \right) + o(\varepsilon) = \varepsilon \left( -x_2, x_1 \right) + o(\varepsilon)
	\end{split}
\end{equation*}
Quindi, per una generica rotazione attorno ad un asse dato dal versore $ \ve{n} $ si ha:
\begin{equation}
	\delta x_i = \varepsilon \sum_{j,k = 1}^{3} \epsilon_{ijk} n_j x_k \quad\Longleftrightarrow\quad \delta\ve{x} = \varepsilon \ve{n}\times\ve{x}
	\label{eq:2.1}
\end{equation}
Nel caso di una rotazione attorno al $ j $-esimo asse coordinato $ \delta x_i^{(j)} = \varepsilon \sum_{k = 1}^{3} \epsilon_{ijk} x_k $, quindi la carica di Noether associata è:
\begin{equation}
	q_j \defeq \sum_{i = 1}^{3} \frac{\pa L}{\pa \dot{x}_i} \delta x_i^{(j)} = \varepsilon \sum_{i,k = 1}^{3} \epsilon_{jki} x_k p_i = \varepsilon L_j
	\label{eq:2.2}
\end{equation}
Dunque l'invarianza per rotazioni attorno ad un asse ha come quantità conservata associata la componente del momento angolare lungo tale asse.

\paragraph{Caso quantistico}

Bisogna innanzitutto verificare che $ \hat{\ve{L}} $ definito in Eq. \ref{eq:1.31} sia effettivamente il momento angolare, ovvero il generatore delle rotazioni (a meno di un fattore $ \hbar $): questo equivale a verificare che l'operatore $ \hat{R}_{\varepsilon} $, definito come:
\begin{equation}
	\hat{R}_{\varepsilon} = e^{i\frac{\varepsilon}{\hbar} \ve{n}\cdot\hat{\ve{L}}} = \tens{I}_3 + i\frac{\varepsilon}{\hbar} \ve{n}\cdot\hat{	\ve{L}} + o(\varepsilon)
	\label{eq:2.3}
\end{equation}
realizzi una rotazione di angolo infinitesimo $ \varepsilon $ attorno all'asse $ \ve{n} $, ovvero:
\begin{equation}
	\braket{\ve{x} | \hat{R}_{\varepsilon} | \psi} = \psi(\ve{x} + \delta_{\ve{n}}\ve{x}) = \psi(\ve{x}) + \delta_{\ve{n}}\ve{x}\cdot\nabla\psi(\ve{x}) + o(\varepsilon)
	\label{eq:2.4}
\end{equation}
dove $ \delta_{\ve{n}}\ve{x} = \varepsilon \ve{n}\times\ve{x} $. Calcolando gli elementi di matrice di $ \hat{R}_{\varepsilon} $ sulla base delle posizioni:
\begin{equation}
	\braket{\ve{x} | \hat{R}_{\varepsilon} | \psi} = \psi(\ve{x}) + i \frac{\varepsilon}{\hbar} \cdot (-i\hbar) \sum_{i,j,k = 1}^{3} n_i \epsilon_{ijk} x_j \pa_k \psi(\ve{x}) + o(\varepsilon)
	\label{eq:2.5}
\end{equation}
Confontando le Eq. \ref{eq:2.4} - \ref{eq:2.5}, si vede che sono uguali, dunque $ \hat{\ve{L}} $ è il generatore delle rotazioni.

\section{Proprietà}

\subsection{Espressione esplicita}

Innanzitutto si noti che dalla definizione in Eq. \ref{eq:1.31} discende subito che $ \hat{\ve{L}} $ è hermitiano:
\begin{equation}
	\hat{L}_i^{\dagger} = \sum_{j,k = 1}^{3} \epsilon_{ijk} \hat{p}_k \hat{x}_j = \sum_{j,k = 1}^{3} \epsilon_{ijk} \left( [\hat{p}_k,\hat{x}_j] + \hat{x}_j \hat{p}_k \right) = L_i + i\hbar \sum_{j,k = 1}^{3} \epsilon_{ijk} \delta_{jk} = L_i
	\label{eq:2.6}
\end{equation}
È anche possibile calcolare esplicitamente l'espressione di $ \hat{\ve{L}} $ in coordinate sferiche:
\begin{equation}
	\hat{L}_x = i\hbar \left( \sin \phi \frac{\pa}{\pa \theta} + \frac{\cos \theta}{\sin \theta} \cos \phi \frac{\pa}{\pa \phi} \right)
	\label{eq:2.7}
\end{equation}
\begin{equation}
	\hat{L}_y = i\hbar \left( -\cos \phi \frac{\pa}{\pa \theta} + \frac{\cos \theta}{\sin \theta} \sin \phi \frac{\pa}{\pa \phi} \right)
	\label{eq:2.8}
\end{equation}
\begin{equation}
	L_z = -i\hbar \frac{\pa}{\pa \phi}
	\label{eq:2.9}
\end{equation}
Si ha inoltre:
\begin{equation}
	\hat{\ve{L}}^2 \equiv \hat{L}^2 = -\hbar^2 \left( \frac{\pa^2}{\pa \theta^2} + \frac{\sin \theta}{\cos \theta} \frac{\pa}{\pa \theta} + \frac{1}{\sin^2 \theta} \frac{\pa^2}{\pa \phi^2} \right)
	\label{eq:2.10}
\end{equation}

\subsection{Commutatori}

Sebbene in un sistema invariante per rotazioni il momento angolare commuti con l'Hamiltoniana, le componenti di $ \hat{\ve{L}} $ non commutano tra loro

\begin{lemma}\label{lem-l-comm}
	$ \hat{x}_i \hat{p}_j - \hat{x}_j \hat{p}_i = \sum_{k = 1}^{3} \epsilon_{ijk}	\hat{L}_k $.
\end{lemma}
\begin{proof}
	$ \sum_{k = 1}^{3} \epsilon_{ijk} \hat{L}_k = \sum_{k,a,b = 1}^{3} \epsilon_{ijk}\epsilon_{kab} \hat{x}_a \hat{p}_b = \sum_{k,a,b = 1}^{3} \left( \delta_{ia}\delta_{jb} - \delta_{ib}\delta_{ja} \right) \hat{x}_a \hat{p}_b = \hat{x}_i \hat{p}_j - \hat{x}_j \hat{p}_i $.
\end{proof}

\begin{proposition}\label{l-comm}
	$ [\hat{L}_i,\hat{L}_j] = i\hbar \sum_{k = 1}^{3} \epsilon_{ijk} \hat{L}_k $.
\end{proposition}
\begin{proof}
	Usando nell'ultima uguaglianza il Lemma \ref{lem-l-comm}:
	\begin{equation*}
		\begin{split}
			[\hat{L}_i,\hat{L}_j]
			&= \sum_{a,b,l,m = 1}^{3} \epsilon_{iab}\epsilon_{jlm} [\hat{x}_a \hat{p}_b,\hat{x}_l \hat{p}_m] = \sum_{a,b,l,m = 1}^{3}  \epsilon_{iab} \epsilon_{jlm} \left( \hat{x}_l [\hat{x}_a,\hat{p}_m] \hat{p}_b + \hat{x}_a [\hat{p}_b,\hat{x}_l] \hat{p}_m \right)\\
			&= i\hbar \sum_{a,b,l = 1}^{3} \epsilon_{bia} \epsilon_{jla} \hat{x}_l \hat{p}_b - i\hbar \sum_{a,b,m = 1}^{3} \epsilon_{iab} \epsilon_{mjb} \hat{x}_a \hat{p}_m\\
			&= i\hbar \sum_{a,b,l = 1}^{3} \left( \delta_{bj}\delta_{il} - \delta_{bl}\delta_{ji} \right) \hat{x}_l \hat{p}_b - i\hbar \sum_{a,b,m = 1}^{3} \left( \delta_{im} \delta_{aj} - \delta_{ij}\delta_{am} \right) \hat{x}_a \hat{p}_m\\
			&= i\hbar \left( \hat{x}_i \hat{p}_j - \delta_{ij} \hat{\ve{x}}\cdot\hat{\ve{p}} - \hat{x}_j \hat{p}_i + \hat{\ve{x}}\cdot\hat{\ve{p}} \delta_{ij} \right) = i\hbar \left( \hat{x}_i \hat{p}_j - \hat{x}_j \hat{p}_i \right) = i\hbar \sum_{k = 1}^{3} \epsilon_{ijk} \hat{L}_k
		\end{split}
	\end{equation*}
\end{proof}

Si ricordi che il commutatore tra un operatore hermitiano $ \hat{G} $, generatore della trasformazione (anch'essa hermitiana) $ \hat{T} = e^{i \varepsilon \hat{G}} $, ed un generico operatore $ \hat{A} $ può essere calcolato da:
\begin{equation}
	\hat{A}' = \hat{T}^{-1} \hat{A} \hat{T} = \left( \tens{I} - i\varepsilon\hat{G} \right) \hat{A} \left( \tens{I} + i\varepsilon\hat{G} \right) = \hat{A} + i\varepsilon [\hat{A},\hat{G}] \quad\Longrightarrow\quad [\hat{A},\hat{G}] = \frac{1}{i\varepsilon} \delta\hat{A}
	\label{eq:2.11}
\end{equation}
Dunque dalla Prop. \ref{l-comm} è possibile vedere come trasforma $ \hat{L}_i $ sotto la rotazione data da $ \hat{L}_j $, e confrontandola con l'Eq. \ref{eq:2.1} si vede che $ \hat{\ve{L}} $ trasforma proprio come un vettore sotto rotazioni (cosa non scontata).\\
Ciò suggerisce naturalmente che $ \hat{L}^2 $, essendo invariante per rotazioni, commuti con ciascuna $ \hat{L}_i $:
\begin{equation}
	[\hat{L}^2,\hat{L}_i] = \sum_{k = 1}^{3} [\hat{L}_k\hat{L}_k,\hat{L}_i] = i\hbar \sum_{j,k = 1}^{3} \epsilon_{kij} \left( \hat{L}_k\hat{L}_j + \hat{L}_j\hat{L}_k \right) = 0
	\label{eq:2.12}
\end{equation}
nullo poiché prodotto di simbolo completamente antisimmetrico con operatore simmetrico.












\chapter{Sistemi Tridimensionali}
\selectlanguage{italian}

\section{Equazione di Schrödinger radiale}

Si consideri una generica Hamiltoniana invariante per rotazioni, ad esempio:
\begin{equation}
	\mathcal{H} = \frac{
	p^2}{2m} + V(r) = \frac{p_r^2}{2m} + \frac{L^2}{2mr^2} + V(r)
	\label{eq:3.1}
\end{equation}
È evidente che questa Hamiltoniana non si possa separare in parte radiale e parte angolare a causa del termine $ \frac{L^2}{r^2} $. È però possibile diagonalizzare simultaneamente $ \hat{H} $, $ \hat{L}^2 $ e $ \hat{L}_z $, dunque si proietta sugli autostati del momento angolare:
\begin{equation}
	\psi(\ve{x}) = \sum_{\ell = 0}^{\infty} \sum_{m = -\ell}^{\ell} \braket{\ve{x} | \ell,m} \braket{\ell,m | \psi} = \sum_{\ell = 0}^{\infty} \sum_{m = -\ell}^{\ell} Y_{\ell,m}(\vartheta,\varphi) \phi_{\ell,m}(r)
	\label{eq:3.2}
\end{equation}
L'equazione di Schrödinger si riduce quindi in una PDE con una sola incognita:
\begin{equation}
	\left[ \frac{p_r^2}{2m} + \frac{\hbar^2 \ell (\ell + 1)}{2mr^2} + V(r) \right] \phi_{\ell,m}(r) = E \phi_{\ell,m}(r)
	\label{eq:3.3}
\end{equation}
Questa non dipende da $ m $, dunque fissati $ \ell $ ed $ E $ c'è una degerazione di $ 2\ell + 1 $; si pone $ \phi_{\ell,m}(r) \equiv \phi_{\ell}(r) $.
È inoltre utile porre:
\begin{equation}
	\phi_{\ell}(r) \equiv \frac{u_{\ell}(r)}{r}
	\label{eq:3.4}
\end{equation}

\begin{proposition}
	$ \hat{p}_r^n \phi_{\ell}(r) = \left( -i\hbar \right)^n \frac{1}{r} \frac{\pa^n}{\pa r^n} u_{\ell}(r) $.
\end{proposition}
\begin{proof}
	$ \hat{p}_r \phi_{\ell}(r) = -i\hbar \left( \frac{\pa}{\pa r} + \frac{1}{r} \right) \frac{u_{\ell}(r)}{r} = -i\hbar \frac{1}{r} \frac{\pa}{\pa r} u_{\ell}(r) $.
\end{proof}
Una ragione $ \virgolette{fisica} $ per definire $ u_{\ell}(r) $ è che assorbe la misura d'integrazione nel prodotto scalare:
\begin{equation*}
	\braket{\psi' | \psi} = \int_{0}^{\infty} dr \,r^2 \phi'^*_{\ell'}(r) \phi_{\ell}(r) \int_{\mathbb{S}^2} d\cos\vartheta \,d\varphi \, Y^*_{\ell',m'}(\vartheta,\varphi) Y_{\ell,m}(\vartheta,\varphi) = \delta_{\ell,\ell'} \delta_{m,m'} \int_0^{\infty} dr \, u'^*_{\ell'}(r) u_{\ell}(r)
\end{equation*}
Ciò rende $ \hat{p}_r $ un operatore hermitiano sulle $ u_{\ell} $, ed infatti l'equazione di Schrödinger diventa:
\begin{equation}
	\left[ - \frac{\hbar^2}{2m} + \frac{\hbar^2 \ell (\ell + 1)}{2m r^2} + V(r) \right] u_{\ell}(r) = E u_{\ell}(r)
	\label{eq:3.5}
\end{equation}

\subsection{Condizioni al contorno}

È necessario che la funzione d'onda radiale $ \phi_{\ell}(r) $ abbia densità di probabilità integrabile su $ [0,+\infty) $: in particolare, si richiede che il seguente integrale non diverga:
\begin{equation}
	\braket{\phi_{\ell} | \phi_{\ell}} = \int_0^{\infty} dr \, r^2 \abs{\phi_{\ell}(r)}^2 = \int_{0}^{\infty} dr  \abs{u_{\ell}(r)}^2
	\label{eq:3.6}
\end{equation}
Nell'origine $ \abs{u_{\ell}(r)}^2 $ deve avere al più una singolarità integrabile, dunque:
\begin{equation}
	u_{\ell}(r) \overset{r \rightarrow 0}{\sim} \frac{1}{r^{\delta}} : \delta < \frac{1}{2}
	\label{eq:3.7}
\end{equation}
Dato che $ \lap \frac{1}{r} = - 4\pi \delta^3 (\ve{x}) $, se $ \phi_{\ell}(r) $ diverge nell'origine almeno come $ \frac{1}{r} $ è possibile soddisfare l'equazione di Schrödinger solo se il potenziale nell'origine diverge almeno come una delta di Dirac. Per potenziali non-distribuzionali (ovvero funzioni, quindi non singolari) $ \phi_{\ell}(r) $ deve divergere nell'origine meno di $ \frac{1}{r} $, dunque:
\begin{equation}
	\lim_{r \rightarrow 0} r \phi_{\ell}(r) = 0 \quad\Longrightarrow\quad \lim_{r \rightarrow 0} u_{\ell}(r) = 0
	\label{eq:3.8}
\end{equation}

\paragraph{Andamento nell'origine}

I potenziali d'interesse fisico sono quelli che per $ r \rightarrow 0 $ divergono meno di $ \frac{1}{r^2} $: il caso in cui nell'origine $ V(r) $ vada come $ \frac{1}{r^k} $ con $ k \ge 2 $ è patologicamente attrattivo e si dimostra non avere un'energia minima, ovvero non presenta stati stabili.\\
Se quindi nell'origine $ V(r) \sim \frac{1}{r^k} $ con $ k < 2 $, per $ r \rightarrow 0 $ a dominare è il termine centrifugo:
\begin{equation}
	- \frac{\hbar^2}{2m} \frac{d^2 u_{\ell}(r)}{d r^2} + \frac{\hbar^2 \ell (\ell + 1)}{2mr^2} u_{\ell}(r) = 0 \quad \Longrightarrow \quad \frac{d^2 u_{\ell}(r)}{d r^2} = \frac{\ell (\ell + 1)}{r^2} u_{\ell}(r)
	\label{eq:3.9}
\end{equation}
La soluzione generale di questa equazione è $ u_{\ell}(r) = A r^{\ell + 1} + B r^{-\ell} $, ma il secondo termine non soddisfa la condizione in Eq. \ref{eq:3.8}, dunque:
\begin{equation}
	u_{\ell}(r) \overset{r \rightarrow 0}{\sim} A r^{\ell + 1}
	\label{eq:3.10}
\end{equation}

\paragraph{Andamento all'infinito}

Se all'infinito il potenziale si annulla, per $ r \rightarrow \infty $ l'andamento della funzione d'onda è quello della particella libera: se esistono stati legati, ovvero autostati di $ \hat{\mathcal{H}} $ con $ E < 0 $, l'andamento della soluzione è:
\begin{equation}
	u_{\ell}(r) \overset{r \rightarrow \infty}{\sim} C e^{- \beta r}, \,\, \beta \equiv \frac{\sqrt{2m\abs{E}}}{\hbar}
	\label{eq:3.11}
\end{equation}
Se invece $ \lim_{r \rightarrow \infty} V(r) \neq 0 $, l'andamento va studiato caso per caso.

\paragraph{Stati legati}

Per un potenziale unidimensionale esiste sempre almeno uno stato legato; inoltre, lo stato fondamentale ha una funzione d'onda pari e gli stati eccitati hanno parità alternata.\\
Nel caso tridimensionale, è possibile interpretare il problema come un problema unidimensionale con dominio $ [0,\infty) $: la condizione in Eq. \ref{eq:3.8}, però, impone che solo le soluzioni dispari sono accettabili, dunque in generale non è detto che esista lo stato fondamentale. Per un potenziale tridimensionale, quindi, non è detto a priori che esistano stati legati.

\section{Particella libera}

L'equazione di Schrödinger per la particella libera, quindi per $ V(\ve{x}) = 0 $, è:
\begin{equation}
	-\frac{\hbar^2}{2m} \lap \psi(\ve{x}) = E \psi(\ve{x})
	\label{eq:3.12}
\end{equation}
Questa è una PDE separabile e, in coordinate cartesiane, le soluzioni sono delle onde piane:
\begin{equation}
	\psi_{\ve{k}}(\ve{x}) = \frac{1}{(2\pi)^{3/2}} e^{i \ve{k}\cdot\ve{x}}, \,\, E = \frac{\hbar^2 \ve{k}^2}{2m}
	\label{eq:3.13}
\end{equation}
normalizzate in senso improprio:
\begin{equation}
	\braket{\psi_{\ve{k}'} | \psi_{\ve{k}}} = \int_{\R^3} d^3\ve{x}\, \psi^*_{\ve{k}'}(\ve{x}) \psi_{\ve{k}}(\ve{x}) = \delta^3(\ve{k} - \ve{k}')
	\label{eq:3.14}
\end{equation}
Un differente approccio risolutivo è quello che sfrutta la simmetria rotazionale del problema:
\begin{equation}
	\psi_{\ell,m}(\ve{x}) = Y_{\ell,m}(\vartheta,\varphi) \frac{u_{\ell}(r)}{r}
	\label{eq:3.15}
\end{equation}
Dall'Eq. \ref{eq:3.5}:
\begin{equation}
	\frac{\hbar^2}{2mE} \left[ -\frac{d^2}{dr^2}u_{\ell}(r) + \frac{\ell(\ell + 1)}{r^2}u_{\ell}(r) \right] - u_{\ell}(r) = 0
	\label{eq:3.16}
\end{equation}
Ponendo $ k = \sqrt{\frac{2mE}{\hbar^2}} $ e $ r' = kr $, si ottiene:
\begin{equation}
	\frac{d^2}{dr'^2} u_{\ell}(r') - \frac{\ell(\ell + 1)}{r'^2} u_{\ell}(r') + u_{\ell}(r') = 0
	\label{eq:3.17}
\end{equation}
Questa è una ODE di Bessel e la sua soluzione si ottiene introducendo la funzione di Bessel $ j_{\ell}(r') $ (è noto che $ j_{\ell}(x) \sim x^{\ell} $ per $ x \rightarrow 0 $):
\begin{equation}
	u_{\ell}(r') = r j_{\ell}(r')
	\label{eq:3.18}
\end{equation}
Le autofunzioni della particella libera tridimensionale sono dunque:
\begin{equation}
	\psi_{\ell,m}(\ve{x}) = Y_{\ell,m}(\vartheta,\varphi) j_{\ell}(kr), \,\, k = \frac{\sqrt{2mE}}{\hbar}
	\label{eq:3.19}
\end{equation}

\section{Oscillatore armonico isotropo}

L'oscillatore armonico isotropo è descritto dal seguente potenziale:
\begin{equation}
	V(r) = \frac{1}{2} m\omega^2 r^2
	\label{eq:3.20}
\end{equation}
Essendo un potenziale centrale, il problema si riduce al problema radiale.
L'Hamiltoniana radiale corrispondente è:
\begin{equation}
	\mathcal{H}_{\ell} = \frac{p_r^2}{2m} + \frac{\hbar^2 \ell(\ell + 1)}{2mr^2} + \frac{1}{2} m\omega^2 r^2
	\label{eq:3.21}
\end{equation}
La sua equazione agli autovalori può essere scritta come:
\begin{equation}
	\hat{\mathcal{H}}_{\ell} \ket{n,\ell,m} = E_{n,\ell} \ket{n,\ell,m}
	\label{eq:3.22}
\end{equation}

\subsection{Stati con \texorpdfstring{$ \ell = 0 $}{TEXT}}

Per $ \ell = 0 $, il problema si riduce all'oscillatore armonico unidimensionale:
\begin{equation}
	\mathcal{H}_0 = \frac{p_r^2}{2m} + \frac{1}{2} m \omega^2 r^2
	\label{eq:3.23}
\end{equation}
Per risolvere con metodo algebrico, si defisce l'operatore di distruzione come:
\begin{equation}
	\hat{d}_0 \defeq \sqrt{\frac{m\omega}{2\hbar}} \left( \hat{r} + i \frac{\hat{p}_r}{m\omega} \right)
	\label{eq:3.24}
\end{equation}
ed il relativo operatore di creazione $ d_0^{\dagger} $. In questo modo, si può scrivere:
\begin{equation}
	\hat{\mathcal{H}}_0 = \hbar \omega \left( \hat{d}_0^{\dagger}\hat{d}_0 + \frac{1}{2} \right)
	\label{eq:3.25}
\end{equation}
Dato che $ \left[ \hat{r},\hat{p}_r \right] = i\hbar $, si ha $ \left[ \hat{d}_0, \hat{d}_0^{\dagger} \right] = 1 $, dunque lo spettro di $ \hat{\mathcal{H}}_0 $ è lo stesso dell'oscillatore armonico unidimensionale:
\begin{equation}
	u_{n,0}(r) = \mathcal{N}_n e^{-c x^2} H_n(r)
	\label{eq:3.26}
\end{equation}
Dato che $ u_{n,0}(-x) = (-1)^n u_{n,0}(x) $, per la condizione al contorno in Eq. \ref{eq:3.8} sono ammissibili solo le soluzioni dispari. Ridefinendo $ n $ così che $ u_{n,0}(r) $ sia la $ (2n+1) $-esima soluzione, si trova lo spettro dell'Hamiltoniana:
\begin{equation}
	E_{n,0} = \hbar \omega \left( 2n + \frac{3}{2} \right)
	\label{eq:3.27}
\end{equation}
Dato che il caso $ \ell = 0 $ è quello maggiormente attrattivo, per l'assenza del termine centrifugo, si trova che lo stato fondamentale è $ \ket{0,0,0} $, con energia $ E_{0,0} = \frac{3}{2} \hbar \omega $.

\subsection{Stati con \texorpdfstring{$ \ell $}{TEXT} generico}

Per scrivere l'Hamiltoniana $ \hat{\mathcal{H}}_{\ell} $ in una forma simile a Eq. \ref{eq:3.25} è necessario definire degli operatori di distruzione/creazione generalizzati:
\begin{equation}
	\hat{d}_{\ell} \defeq \sqrt{\frac{m\omega}{2\hbar}} \left( \left( \hat{r} + \frac{\hbar\ell}{m\omega\hat{r}} \right) + i \frac{\hat{p}_r}{m\omega} \right)
	\label{eq:3.28}
\end{equation}
È utile inoltre ricordare che, essendo $ \left[ \hat{p}_r, f(\hat{r}) \right] = -i\hbar f(\hat{r}) $, si ha $ \left[ \hat{p}_r, \frac{1}{\hat{r}} \right] = \frac{i\hbar}{\hat{r}^2} $.\\
Si può generalizzare l'operatore numero come:
\begin{equation}
	\hat{D}_{\ell} \defeq \hat{d}_{\ell}^{\dagger} \hat{d}_{\ell}
	\label{eq:3.29}
\end{equation}
La sua espressione eplicita è:
\begin{equation*}
	\begin{split}
		D_{\ell}
		&= d_{\ell}^{\dagger} d_{\ell} = \frac{m\omega}{2\hbar} \left( \left( r + \frac{\hbar\ell}{m\omega r} \right) - i \frac{p_r}{m\omega} \right) \left( \left( r + \frac{\hbar\ell}{m\omega r} \right) + i \frac{p_r}{m\omega} \right)\\
		&= \frac{m\omega}{2\hbar} \left( \left( r + \frac{\hbar\ell}{m\omega r} \right)^2 + \frac{p_r^2}{m^2 \omega^2} + \left[ \left( r + \frac{\hbar\ell}{m\omega r} \right), i \frac{p_r}{m\omega} \right] \right)\\
		&= \frac{m\omega}{2\hbar} \left( r^2 + \frac{\hbar^2\ell^2}{m^2 \omega^2 r^2} + \frac{p_r^2}{m^2 \omega^2} + \frac{2\hbar\ell}{m\omega} - \frac{i}{m\omega} \left( -i\hbar + \frac{\hbar\ell}{m\omega} \frac{i\hbar}{r^2} \right) \right)\\
		&= \frac{1}{\hbar\omega} \left( \frac{p_r^2}{2m} + \frac{1}{2}m\omega^2 r^2 + \frac{\hbar^2 \ell (\ell + 1)}{2mr^2} \right) + \ell - \frac{1}{2}\\
	\end{split}
\end{equation*}
Risulta quindi che:
\begin{equation}
	\hat{D}_{\ell} = \frac{1}{\hbar\omega} \hat{\mathcal{H}}_{\ell} + \ell - \frac{1}{2}
	\label{eq:3.30}
\end{equation}
Di conseguenza, $ \hat{D}_{\ell} $ e $ \hat{\mathcal{H}}_{\ell} $ hanno gli stessi autostati $ \ket{n,\ell} $:
\begin{equation}
	\hat{D}_{\ell} \ket{n,\ell} = \mathcal{E}_{n,\ell} \ket{n,\ell}, \,\, \mathcal{E}_{n,\ell} = \frac{1}{\hbar \omega} E_{n,\ell} + \ell - \frac{1}{2}
	\label{eq:3.31}
\end{equation}
Ovviamente $ \hat{D}_0 $ è il consueto operatore numero: infatti, il suo spettro è $ \mathcal{E}_{n,0} = 2n + 1 $.\\
È necessario definire un ulteriore operatore:
\begin{equation}
	\hat{\overline{D}}_{\ell} \defeq \hat{d}_{\ell} \hat{d}_{\ell}^{\dagger}
	\label{eq:3.32}
\end{equation}
Con un calcolo analogo al precedente, si trova che:
\begin{equation}
	\hat{\overline{D}}_{\ell} = \frac{1}{\hbar \omega} \hat{\mathcal{H}}_{\ell - 1} + \ell + \frac{1}{2}
	\label{eq:3.33}
\end{equation}
Risulta evidente quindi che:
\begin{equation}
	\hat{\overline{D}}_{\ell + 1} = \hat{D}_{\ell} + 2
	\label{eq:3.34}
\end{equation}
Questi operatori legano gli spettri di Hamiltoniane con $ \ell $ diversi: si supponga di avere $ \ket{n,\ell} $ autostato di $ \hat{\mathcal{H}}_{\ell} $ e $ \hat{D}_{\ell} $: $ \hat{d}_{\ell + 1}^{\dagger} \ket{m,\ell} $ è un autostato di $ \hat{\mathcal{H}}_{\ell + 1} $ e $ \hat{D}_{\ell + 1} $, dato che:
\begin{equation*}
	\hat{D}_{\ell + 1} \hat{d}_{\ell + 1}^{\dagger} \ket{n,\ell} = \hat{d}_{\ell + 1}^{\dagger} \hat{d}_{\ell + 1} \hat{d}_{\ell + 1}^{\dagger} \ket{n,\ell} = \hat{d}_{\ell + 1}^{\dagger} (\hat{D}_{\ell} + 2) \ket{n,\ell} = (\mathcal{E}_{n,\ell} + 2) \hat{d}_{\ell + 1}^{\dagger} \ket{n,\ell}
\end{equation*}
Questi operatori sono simili a degli operatori di scala:
\begin{equation}
	\hat{D}_{\ell + 1} \hat{d}_{\ell + 1} \ket{n,\ell} = (\mathcal{E}_{n,\ell} + 2) \hat{d}_{\ell + 1}^{\dagger} \ket{n,\ell}
	\label{eq:3.35}
\end{equation}
\begin{equation}
	\hat{D}_{\ell - 1} \hat{d}_{\ell} \ket{n,\ell} = (\mathcal{E}_{n,\ell} - 2) \hat{d}_{\ell} \ket{n,\ell}
	\label{eq:3.36}
\end{equation}
In questo modo, è possibile costruire l'intero spettro, ricordando che $ \mathcal{E}_{n,0} = 2n + 1 $:
\begin{equation*}
	\begin{split}
		&\hat{D}_1 \hat{d}_1^{\dagger} \ket{n,0} = (\mathcal{E}_{n,0} + 2) \hat{d}_1^{\dagger} \ket{n,0} = (2n + 3) \hat{d}_1^{\dagger} \ket{n,0}\\
		&\hat{D}_2 \hat{d}_2^{\dagger} \hat{d}_1^{\dagger} \ket{n,0} = (\mathcal{E}_{n,0} + 4) \hat{d}_2^{\dagger} \hat{d}_1^{\dagger} \ket{n,0} = (2n + 5) \hat{d}_2^{\dagger} \hat{d}_1^{\dagger} \ket{n,0}\\
		&\qquad\quad\vdots\\
		&\hat{D}_{\ell + 1} \hat{d}_{\ell + 1}^{\dagger} \dots \hat{d}_1^{\dagger} \ket{n,0} = (2n + 2\ell + 1) \hat{d}_{\ell + 1}^{\dagger} \dots \hat{d}_1^{\dagger} \ket{n,0}
	\end{split}
\end{equation*}
Si evince che $ \mathcal{E}_{n,\ell} = 2n + 2\ell + 1 $, quindi dall'Eq. \ref{eq:3.31} si ricava lo spettro dell'$ \ell $-esima Hamiltoniana:
\begin{equation}
	E_{n,\ell} = \hbar \omega \left( 2n + \ell + \frac{3}{2} \right)
	\label{eq:3.37}
\end{equation}
Questi sono tutti e soli i possibili autovalori di $ \hat{\mathcal{H}}_{\ell} $: se per assurdo esistesse un suo autostato con un autovalore non presente in questa sequenza, ci si potrebbe comunque sempre ricondurre ad uno stato con $ \ell = 0 $ applicando ripetutamente $ \hat{d}_{\ell} $, ottenendo così un nuovo autovalore di $ \hat{\mathcal{H}}_0 $, il che è assurdo poiché il suo spettro completo (Eq. \ref{eq:3.27}) è dato dall'Eq. \ref{eq:3.37}.












\end{document}
