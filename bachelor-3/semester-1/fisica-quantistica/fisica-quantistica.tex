\documentclass[a4paper, 12pt, openany]{book}
\usepackage[utf8]{inputenc}
\usepackage[italian]{babel}

\usepackage[]{csvsimple}
\usepackage{float}

\usepackage{ragged2e}
\usepackage[left=25mm, right=25mm, top=15mm]{geometry}
\geometry{a4paper}
\usepackage{graphicx}
\usepackage{booktabs}
\usepackage{paralist}
\usepackage{subfig} 
\usepackage{fancyhdr}
\usepackage{amsmath}
\usepackage{amssymb}
\usepackage{amsfonts}
\usepackage{amsthm}
\usepackage{mathtools}
\usepackage{enumitem}
\usepackage{titlesec}
\usepackage{braket}
\usepackage{gensymb}
\usepackage{url}
\usepackage{hyperref}
\usepackage{csquotes}
\usepackage{multicol}
\usepackage{graphicx}
\usepackage{wrapfig}
\usepackage{caption}

\usepackage{esint}

\captionsetup{font=small}
\pagestyle{fancy}
\renewcommand{\headrulewidth}{0pt}
\lhead{}\chead{}\rhead{}
\lfoot{}\cfoot{\thepage}\rfoot{}
\usepackage{sectsty}
\usepackage[nottoc,notlof,notlot]{tocbibind}
\usepackage[titles,subfigure]{tocloft}
\renewcommand{\cftsecfont}{\rmfamily\mdseries\upshape}
\renewcommand{\cftsecpagefont}{\rmfamily\mdseries\upshape}

\let\oldsection\section% Store \section
\renewcommand{\section}{% Update \section
	\renewcommand{\theequation}{\thesection.\arabic{equation}}% Update equation number
	\oldsection}% Regular \section
\let\oldsubsection\subsection% Store \subsection
\renewcommand{\subsection}{% Update \subsection
	\renewcommand{\theequation}{\thesubsection.\arabic{equation}}% Update equation number
	\oldsubsection}% Regular \subsection

\newcommand{\abs}[1]{\left\lvert#1\right\rvert}
\newcommand{\norm}[1]{\left\lVert#1\right\rVert}
\newcommand{\vprod}[2]{\vec{#1}\times\vec{#2}}
\newcommand{\sprod}[2]{\vec{#1}\cdot\vec{#2}}

\newcommand{\g}{\text{g}}
\newcommand{\m}{\text{m}}
\newcommand{\cm}{\text{cm}}
\newcommand{\mm}{\text{mm}}
\newcommand{\s}{\text{s}}
\newcommand{\N}{\text{N}}
\newcommand{\Hz}{\text{Hz}}

\newcommand{\virgolette}[1]{``\text{#1}"}
\newcommand{\tildetext}{\raise.17ex\hbox{$\scriptstyle\mathtt{\sim}$}}

\renewcommand{\arraystretch}{1.2}

\addto\captionsenglish{\renewcommand{\figurename}{Fig.}}
\addto\captionsenglish{\renewcommand{\tablename}{Tab.}}

\DeclareCaptionLabelFormat{andtable}{#1~#2  \&  \tablename~\thetable}

\setlength{\parindent}{0pt}

\newcommand{\dive}{\nabla\cdot}
\newcommand{\rot}{\nabla\times}


\author{Leonardo Cerasi%
	\thanks{\scriptsize\href{mailto:leonardo.cerasi@studenti.unimi.it}{leo.cerasi@pm.me}}%
	, Lucrezia Bioni\\
	\small GitHub repository: \href{https://github.com/LeonardoCerasi/notes}{LeonardoCerasi/notes}}

\title{\Huge\textbf{Fisica Quantistica 2} \\ \large Prof. S. Forte, a.a. 2024-25}

\begin{document}

\frontmatter

\maketitle

\tableofcontents
\pagestyle{indice}

\mainmatter

\chapter*{Introduzione}
\pagestyle{introd}
\addcontentsline{toc}{chapter}{Introduzione}
\markboth{Introduzione}{}
\selectlanguage{italian}

Il problema generale che si va ad analizzare è il sistema di $ N_n $ elettroni ed $ N_n $ nuclei atomici, il cui moto non-relativistico è affetto solo dall'interazione elettromagnetica ed è dunque descritto dall'Hamiltoniana:
\begin{equation*}
	\mathcal{H} = T_n + T_e + V_{ne} + V_{nn} + V_{ee}
\end{equation*}
L'energia cinetica totale dei nuclei è:
\begin{equation*}
	T_n = \frac{1}{2} \sum_\alpha \frac{\ve{P}_{\ve{R}_\alpha}^2}{M_\alpha}
\end{equation*}
dove $ \ve{P}_{\ve{R}_\alpha} $ è il momento coniugato alla posizione $ \ve{R}_\alpha $ dell'$ \alpha $-esimo nucleo, mentre l'energia cinetica degli elettroni è:
\begin{equation*}
	T_e = \frac{1}{2m_e} \sum_i \ve{P}_{\ve{r}_i}^2
\end{equation*}
dove $ \ve{P}_{\ve{r}_i} $ è il momento coniugato alla posizione $ \ve{r}_i $ dell'$ i $-esimo elettrone. I potenziali d'interazione elettromagnetica sono invece:
\begin{equation*}
	V_{ne} = - \frac{q_e^2}{4\pi \epsilon_0} \sum_\alpha \sum_i \frac{Z_\alpha}{\abs{\ve{R}_\alpha - \ve{r}_i}}
\end{equation*}
\begin{equation*}
	V_{nn} = \frac{q_e^2}{4\pi \epsilon_0} \frac{1}{2} \sum_\alpha \sum_{\beta \neq \alpha} \frac{Z_\alpha Z_\beta}{\abs{\ve{r}_\alpha - \ve{r}_\beta}}
\end{equation*}
\begin{equation*}
	V_{ee} = \frac{q_e^2}{4\pi \epsilon_0} \frac{1}{2} \sum_i \sum_{j \neq i} \frac{1}{\abs{\ve{r}_i - \ve{r}_j}}
\end{equation*}
La risoluzione analitica di questo problema è possibile solo in un numero limitato di casi, mentre la sua integrazione numerica scala in complessità esponenzialmente con $ N = N_e + N_n $.\\
Nella formulazione di tale Hamiltoniana, si sono applicate alcune approssimazioni:
\begin{enumerate}
	\item corpi puntiformi: mentre per l'elettrone, in quanto particella fondamentale, questa assunzione è sempre lecita, per il nucleo atomico essa è possibile visto che il rapporto tra raggio nucleare e raggio atomico è dell'ordine di $ 10^{-3} $, oltre al fatto che le energie in gioco nei processi atomici ($ \sim 1\ev $) non sono sufficienti ad eccitare i gradi di libertà interni del nucleo ($ \sim 1\mev $);
	\item moto non-relativistico: alcune correzioni relativistiche (es.: interazione spin-orbita) possono essere trattate perturbativamente;
	\item sistema isolato: si assume il sistema non-interagente con l'ambiente esterno ed in assenza di campi esterni.
\end{enumerate}
Si noti che mentre i sistemi a singola particella godono di determinate simmetrie, quelli a molti corpi possono presentare delle rotture spontanee di simmetria: ciò è particolarmente evidente nei sistemi molecolari, mentre in quelli atomici è presente ma in misura minore, ed avviene poiché nei casi in cui una rottura di simmetria permetta di abbassare l'energia totale del sistema.

\paragraph{Ordini di grandezza}

Innanzitutto, conviene definire la coupling constant dell'interazione elettromagnetica:
\begin{equation*}
	e^2 \equiv \frac{q_e^2}{4\pi \epsilon_0} = 2.3071 \cdot 10^{-28} \,\text{J} \,\text{m} = 14.3387 \ev \ang
\end{equation*}
La scala del moto elettronico nell'atomo è data dal \textit{raggio di Bohr}:
\begin{equation*}
	a_0 \equiv \frac{\hbar^2}{m_e e^2} = 0.529177 \cdot 10^{-10} \,\text{m} = 0.529177 \ang
\end{equation*}
Evidenze sperimentali mostrano che gli atomi nella materia sono distanziati nell'ordine di $ 2a_0 - 10a_0 $. La scala delle interazioni elettromagnetiche in ambito atomico/molecolare è dunque data dall'\textit{energia di Hartree}:
\begin{equation*}
	E_\text{Ha} \equiv \frac{e^2}{a_0} = 4.35974 \cdot 10^{-18} \,\text{J} = 27.2114 \ev
\end{equation*}
La tipica timescale del moto elettronico si ottiene dal principio d'indeterminazione:
\begin{equation*}
	t_0 = \frac{\hbar}{E_\text{Ha}} = 2.4189 \cdot 10^{-17} \,\text{s}
\end{equation*}
Ciò permette di calcolare la scala delle velocità elettroniche, confermando l'approssimazione non-relativistica:
\begin{equation*}
	v_0 = \frac{a_0}{t_0} = 2.1877 \cdot 10^6 \,\text{m} \,\text{s}^{-1} \simeq 0.01 c
\end{equation*}
La scala dei fenomeni relativistici nella dinamica elettronica è data dalla \textit{costante di struttura fine}:
\begin{equation*}
	\alpha = \frac{v_0}{c} = \frac{e^2}{\hbar c} = 7.29734 \cdot 10^{-3} \simeq \frac{1}{137.036}
\end{equation*}
Comparando con le onde elettromagnetiche, la scala delle distanze interatomiche ($ \sim 1\ang $) corrisponde alla regione dei raggi X, mentre quella delle frequenze elettroniche (e dunque di $ E_\text{Ha} $) corrisponde alla fascia UV ($ \lambda \sim 10^3 a_0 $); le frequenze tipiche (e dunque le energie) del moto nucleare sono invece associate alla regione IR ($ \nu \sim 5 \,\text{THz} $).

\paragraph{Spettroscopia}

Gli esperimenti spettroscopici sono quelli in cui una proprietà caratterizzante l'interazione tra radiazione e materia è misurata in funzione della frequenza della radiazione incidente sul campione che si vuole studiare. I principali tipi di spettroscopia sono due:
\begin{enumerate}
	\item assorbimento: un fascio collimato di luce monocromatica incide sul bersaglio; se la frequenza della radiazione coincide con quella di una transizione specifica del campione, ci sarà un importante assorbimento di fotoni: questo sarà visibile plottando l'intensità della radiazione emergente dal campione $ I(\omega) $, ottenendo il cosiddetto spettro d'assorbimento;
	\item emissione: il campione viene portato in uno stato eccitato (es.: bombardandolo di elettroni o fotoni alto-energetici), dunque emetterà della radiazione ad ogni transizione di diseccitamento, la quale va a formare il cosiddetto spettro d'emissione.
\end{enumerate}
Gli spettri atomici e molecolari sono dunque caratterizzati da picchi monocromatici, detti linee, associate a transizioni risonanti tra stati $ \ket{i} , \ket{f} $ che determinano linee a $ \omega_{if} = \frac{1}{\hbar} \abs{E_i - E_f} $. Sebbene a livello teorico queste linee sarebbero delle $ \delta $ di Dirac, sperimentalmente si misurano sempre delle righe più o meno strette; le cause dell'allargamento delle linee spettrali sono sia intrinseche che estrinseche, e principalmente sono:
\begin{enumerate}
	\item risoluzione sperimentale: tipicamente determinata da vari effetti aleatori, dunque determina una forma gaussiana; può essere migliorata con accorgimenti tecnici;
	\item allargamento naturale: dovuto al fatto che gli stati eccitati, sebbene stazionari in prima approssimazione, vengono resi instabili dall'interazione col le fluttuazioni di punto-zero del campo elettromagnetico (quantistico); si determina dunque un decadimento spontaneo di tutti gli autostati d'energia (eccetto il ground state) che, sebbene randomico per un singolo atomo, segue una legge statistica per un sistema a molti atomi:
	\begin{equation*}
		N(t) = N_0 e^{- \gamma t} = N_0 e^{- t / \tau}
	\end{equation*}
	dove $ \gamma $ è la costante di decadimento e $ \tau $ la vita media dello stato eccitato, la quale setta la durata tipica della spettroscopia. Dal principio d'indeterminazione, si trova che l'energia di uno stato eccitato non è misurabile con precisione migliore di:
	\begin{equation*}
		\Delta E = \frac{\hbar}{\tau} = \hbar \gamma
	\end{equation*}
	Ciò causa dunque l'allargamento naturale delle linee spettrali secondo la Lorentziana:
	\begin{equation*}
		I(\omega) = I_0 \frac{\gamma^2}{(\omega - \omega_{if})^2 + \gamma^2}
	\end{equation*}
	Gli stati atomici eccitati hanno $ \tau \sim 1\,\text{ns} $, dunque l'allargamento è di $ \Delta E \sim 1\,\mu\text{eV} $.
	\item allargamento Doppler: nel caso di un campione in fase gassosa, il moto termico randomico degli atomi/molecole determina un red/blue-shift delle frequenze di transizione, a seconda della velocità casuale dell'atomo/molecola che decade; si ha un allargamento gaussiano delle righe spettrali, che determina:
	\begin{equation*}
		\Delta \omega_\text{Doppler} = \omega_{if} \sqrt{8 \ln(2) \frac{k_B T}{M c^2}}
	\end{equation*}
	A temperatura fissata, gli atomi/molecole più leggeri si muoveranno più velocemente, determinando un allargamento maggiore (sempre nell'ordine dei $ \mu\text{eV} $).
\end{enumerate}
Si ha dunque un allargamento totale pari alla somma in quadratura di questi allargamenti singoli.












\part{Meccanica Quantistica in più Dimensioni}
\pagestyle{body}

\chapter{Sistemi Quantistici Multidimensionali}
\selectlanguage{italian}

\section{Spazio prodotto diretto}

Per definire formalmente i sistemi quantistici in più dimensioni, è necessario definire prima il prodotto diretto tra spazi di Hilbert.

\begin{definition}
	Dati due spazi di Hilbert $ \hilb $ e $ \mathscr{K} $ con basi $ \{\ket{e_i}\} $ e $ \{\ket{\tilde{e}_j}\} $, si definisce il loro prodotto diretto come $ \hilb \otimes \mathscr{K} \defeq \{ \ket{\psi} = \sum_{i,j} c_{ij} \ket{e_i} \otimes \ket{\tilde{e}_j} \} $. In questo spazio si definisce il prodotto scalare tra due vettori $ \ket{\psi_1} = \ket{e_{i_1}}\otimes\ket{\tilde{e}_{j_1}} $ e $ \ket{\psi_2} = \ket{e_{i_2}}\otimes\ket{\tilde{e}_{j_2}} $ come $ \braket{\psi_1 | \psi_2} = \braket{e_{i_1} | e_{i_2}}\braket{\tilde{e}_{j_1} | \tilde{e}_{j_2}} $.
\end{definition}

Per semplificare la scrittura, si adotta la notazione $ \ket{e_i}\otimes\ket{e_j} \equiv \ket{e_i e_j} $ (o si sottintende $ \otimes $).\\
Si noti che osservabili relative a spazi diversi sono sempre compatibili.\\
In generale, il generico $ \ket{\psi}\in\hilb\otimes\mathscr{K} $ non è scrivibile come prodotto diretto $ \ket{\psi} = \ket{\phi}\otimes\ket{\tilde{\phi}} $, con $ \ket{\phi}\in\hilb $ e $ \ket{\tilde{\phi}}\in\mathscr{K} $, poiché in generale non è detto che $ c_{ij} $ sia fattorizzabile in $ \alpha_i $ e $ \tilde{\alpha}_j $: in questo caso si dice che lo stato è entangled.

\begin{definition}
	Uno stato $ \ket{\psi}\in\hilb\otimes\mathscr{K} $ si dice entangled se non è fattorizzabile.
\end{definition}

\begin{example}
	Dati due qubit, uno stato entangled è $ \ket{\psi} = \frac{1}{\sqrt{2}} \left(\ket{01} + \ket{10}\right) $, dato che il generico stato fattorizzabile è $ \left(a\ket{0} + b\ket{1}\right)\otimes\left(c\ket{0} + d\ket{1}\right) = ac\ket{00} + ad\ket{01} + bc\ket{10} + bd\ket{11} $.
\end{example}

La probabilità $ P_{ij} = \abs{c_{ij}}^2 $ è detta probabilità congiunta: in generale essa non è il prodotto delle probabilità dei singoli eventi per i fenomeni di interferenza quantistica, i quali rendono tale probabilità dipendente dallo stato dell'intero sistema.

\section{Sistemi multidimensionali}

Per generalizzare la meccanica quantistica in $ d $ dimensioni, si introduce l'operatore posizione $ \hat{\ve{x}} $:
\begin{equation}
	\hat{\ve{x}} \defeq
	\begin{pmatrix}
		\hat{x}_1 \\
		\vdots \\
		\hat{x}_d
	\end{pmatrix}
	\label{eq:1}
\end{equation}
Ciascuna componente di questo vettore è un operatore hermitiano che agisce su uno spazio di Hilbert, mentre il vettore $ \hat{\ve{x}} $ agisce sul loro prodotto diretto $ \hilb \defeq \hilb_1 \otimes \dots \otimes \hilb_d $. Su ciascuno spazio $ \hilb_j $ viene definita la base delle posizioni da $ \hat{x}_j \ket{x_j} = x_j \ket{x_j} $, dunque la base delle posizioni in $ \hilb $ sarà $ \ket{\ve{x}} \defeq \ket{x_1} \otimes \dots \otimes \ket{x_d} $: data $ \ket{\psi} \in \hilb $, la sua rappresentazione sulla base delle posizioni è $ \braket{\ve{x} | \psi} \equiv \psi(\ve{x}) $, con $ \psi : \R^d \rightarrow \C $, il cui modulo quadro dà una densità di di probabilità $ d $-dimensionale $ dP_{\ve{x}} = \abs{\psi(\ve{x})}^2 d^d\ve{x} $.\\
In questo caso, l'entanglement consiste nel fatto che, in generale, $ \psi(\ve{x}) \neq \psi_1(x_1) \dots \psi_d(x_d) $.\\
Tale formalismo è generalizzabile al caso di $ n $ corpi in $ d $ dimensioni, nel qual caso si ha uno spazio prodotto diretto di $ nd $ spazi di Hilbert.

\begin{example}
	Nel caso di $ 2 $ corpi in $ 3 $ dimensioni, si ha:
	\begin{equation*}
		\hat{\ve{x}} =
		\begin{pmatrix}
			\hat{x}_{1,1} \\
			\hat{x}_{1,2} \\
			\hat{x}_{1,3} \\
			\hat{x}_{2,1} \\
			\hat{x}_{2,2} \\
			\hat{x}_{2,3} \\
		\end{pmatrix}
	\end{equation*}
	In questo sistema, la funzione d'onda è $ \braket{\ve{x}_1 \ve{x}_2 | \psi} = \psi(\ve{x}_1, \ve{x}_2) $.
\end{example}

\begin{equation}
	\braket{\ve{x}' | \ve{x}} = \delta^{(d)}(\ve{x} - \ve{x}')
	\label{eq:2}
\end{equation}

La $ \delta^{(d)} $ è il prodotto di $ d $ delte di Dirac ed è definita da $ \int_{\R^d} d^d\ve{x} \, \delta^{(d)}(\ve{x} - \ve{x}') f(\ve{x}) = f(\ve{x}') $ come distribuzione.

\subsection{Coordinate cartesiane}

Analogamente al caso monodimensionale, per definite l'operatore impulso si considera una traslazione spaziale; le componenti del vettore operatore impulso $ \hat{\ve{p}} $ sulla base delle posizioni sono definite da:

\begin{equation}
  \braket{\ve{x} | \hat{p}_j | \psi} = - i\hbar \frac{\pa}{\pa x_j} \psi(\ve{x})
	\label{eq:3}
\end{equation}
In forma vettoriale, è possibile scrivere:
\begin{equation}
	\hat{\ve{p}} = -i\hbar\nabla
	\label{eq:4}
\end{equation}

A questo punto, è facile definite le autofunzioni dell'impulso tali per cui $ \hat{\ve{p}}\ket{\ve{k}} = \hbar \ve{k}\ket{\ve{k}} $:

\begin{equation}
	\braket{\ve{x} | \ve{k}} = \frac{1}{(2\pi)^{d/2}} e^{i \ve{k}\cdot\ve{x}} \equiv \psi_{\ve{k}}(\ve{x})
	\label{eq:5}
\end{equation}

Il fatto che operatori su spazi diversi commutino tra loro implica che:

\begin{equation}
	\left[ \hat{p}_j, \hat{p}_k \right] = 0 \quad \forall j,k = 1, \dots, d
	\label{eq:6}
\end{equation}

Dal punto di vista matematico, questo è ovvio per il lemma di Schwarz (assumendo una well-behaved $ \psi $), mentre da quello fisico ciò esprime il fatto che traslazioni lungo assi diversi commutano tra loro: ciò non è scontato, infatti ad esempio le rotazioni rispetto ad assi diversi non commutano (dunque le componenti del momento angolare non commuteranno).\\
%
È facile vedere che $ \hat{\ve{p}}^2 = -\hbar^2 \lap $, dunque è possibile definire l'Hamiltoniana del sistema (e con essa la sua evoluzione temporale):

\begin{equation}
	\mathcal{H} = - \frac{\hbar^2}{2m} \lap + V(\ve{x})
	\label{eq:7}
\end{equation}

Ricordando che $ \hat{\mathcal{H}}\ket{\psi} = E\ket{\psi} $, si ottiene l'equazione di Schrödinger sulla base delle coordinate:
\begin{equation}
	- \frac{\hbar^2}{2m} \lap \psi(\ve{x}) + V(\ve{x}) \psi(\ve{x}) = E \psi(\ve{x})
	\label{eq:8}
\end{equation}

\section{Separabilità}

Nel caso di sistemi non-entangled, è possibile separare il problema multidimensionale in $ d $ problemi monodimensionali e scrivere la soluzione come prodotto delle soluzioni dei problemi ridotti.

\subsection{Problemi separabili in coordinate cartesiane}

\begin{proposition}
	In coordinate cartesiane, condizione sufficiente affinché il problema sia separabile è che:
	\begin{equation}
		V(\ve{x}) = V_1(x_1) + \dots + V_d(x_d)
		\label{eq:9}
	\end{equation}
\end{proposition}

In tal caso, l'Hamiltoniana del sistema è somma di $ d $ sotto-Hamiltoniane (e di conseguenza lo è anche l'evoluzione temporale):

\begin{equation}
	\mathcal{H}_j = \frac{\hat{p}_j^2}{2m} + \hat{V}(\hat{x}_j)
	\label{eq:10}
\end{equation}

dunque la determinazione dello spettro dell'Hamiltoniana si riduce a $ d $ problemi unidimensionali.

\begin{proposition}\label{ham-sep-cart}
	Data un'Hamiltoniana separabile $ \mathcal{H} $, detti $ \braket{x_j | \psi_{k_j}} = \psi_{k_j}(x_j) $ gli autostati della $ j $-esima sotto-Hamiltoniana $ \mathcal{H}_j \ket{\psi_{k_j}} = E_{k_j} \ket{\psi_{k_j}} $, sono autostati di $ \mathcal{H} $ gli stati prodotto:
	\begin{equation}
		\braket{\ve{x} | \psi_{k_1 \dots k_d}} = \psi_{k_1 \dots k_d} (\ve{x}) \equiv \psi_{k_1}(x_1) \dots \psi_{k_d}(x_d)
		\label{eq:11}
	\end{equation}
\end{proposition}
\begin{proof}
	Si vede facilmente che:
	\begin{equation*}
		\begin{split}
			\braket{\ve{x} | \mathcal{H} | \psi_{k_1 \dots k_d}}
			&= - \frac{\hbar^2}{2m} \frac{\pa^2 \psi_{k_1}(x_1)}{\pa x_1^2} \psi_{k_2}(x_2) \dots \psi_{k_d} (x_d) + V_1(x_1) \psi_{k_1}(x_1) \psi_{k_2}(x_2) \dots \psi_{k_d}(x_d) +\\
			& \vdots\\
			& - \frac{\hbar^2}{2m} \frac{\pa^2 \psi_{k_d}(x_d)}{\pa x_d^2} \psi_{k_1}(x_1) \dots \psi_{k_{d-1}} (x_{d-1}) + V_d(x_d) \psi_{k_d}(x_d) \psi_{k_1}(x_1) \dots \psi_{k_{d-1}}(x_{d-1})\\
			&= E_{k_1} \psi_{k_1}(x_1) \dots \psi_{k_d}(x_d) + \dots + E_{k_d} \psi_{k_1}(x_1) \dots \psi_{k_d}(x_d)\\
			&= E_{k_1 \dots k_d} \psi_{k_1 \dots k_d}(\ve{x})
		\end{split}
	\end{equation*}
	dove è stata definita $ E_{k_1 \dots k_d} \equiv E_{k_1} + \dots + E_{k_d} $.
\end{proof}

\subsection{Hamiltoniane separabili}

Si può vedere che, per un'Hamiltoniana separabile, le autofunzioni \ref{eq:11} sono le più generali.\\
Innanzitutto, il commutatore canonico in $ d $ dimensioni si generalizza come:

\begin{equation}
	\left[\hat{x}_j, \hat{p}_k\right] = i\hbar \delta_{jk} \qquad \left[\hat{x}_j, \hat{x}_k\right] = 0 \qquad \left[\hat{p}_j, \hat{p}_k\right] = 0
	\label{eq:12}
\end{equation}

Da ciò segue che le Hamiltoniane \ref{eq:10} commutano tra loro, dunque sono diagonalizzabili simultaneamente e gli autovalori della loro somma sono la somma dei loro autovalori: di conseguenza, gli autostati di dell'Hamiltoniana del sistema sono tutti e soli quelli trovati nella Prop. \ref{ham-sep-cart}.\\
%
Questo argomento è facilmente generalizzabile: si consideri un'Hamiltoniana generica $ \mathcal{H} $ che è possibile separare come somma di Hamiltoniane commutanti tra loro:

\begin{equation}
	\mathcal{H} = \mathcal{H}_1 + \dots + \mathcal{H}_d \qquad \left[\mathcal{H}_j, \mathcal{H}_k\right] = 0
	\label{eq:13}
\end{equation}

Le $ \mathcal{H}_j $ sono allora diagonalizzabili simultaneamente:

\begin{equation}
	\mathcal{H}_j \ket{k_j} = E_{k_j} \ket{k_j}
	\label{eq:14}
\end{equation}

e tali autostati formano una base per gli autostati di $ \mathcal{H} $:

\begin{equation}
	\ket{k_1 \dots k_d} = \ket{k_1} \otimes \dots \otimes \ket{k_d}
	\label{eq:15}
\end{equation}

mentre i suoi autostati sono:

\begin{equation}
	E_{k_1 \dots k_d} = E_{k_1} + \dots + E_{k_d}
	\label{eq:16}
\end{equation}

\begin{example}
	Un esempio tipico di problema tridimensionale separabile è la buca parallelepipedale di potenziale:
	\begin{equation*}
		V_j(x_j) = 
		\begin{cases}
			0 & \quad \abs{x_j} < a_j\\
			\infty & \quad \abs{x_j} \ge a_j
		\end{cases}
	\end{equation*}
	Ricordando la forma esplicita delle autofunzioni:
	\begin{equation*}
		\braket{x_j | \psi_{n_j}} =
		\begin{cases}
			A_{n_j} \cos\left(k_{n_j} x_j\right) & \quad n_j = 2n + 1\\
			B_{n_j} \sin\left(k_{n_j} x_j\right) & \quad n_j = 2n\\
		\end{cases}
		\qquad\qquad k_{n_j} = \frac{n_j \pi}{2 a_j}
	\end{equation*}
	è facile ricavare lo spettro dell'Hamiltoniana:
	\begin{equation*}
		E_{n_1 n_2 n_3} = \frac{\hbar^2}{2m} \left(k_{n_1}^2 + k_{n_2}^2 + k_{n_3}^2\right) = \frac{\hbar^2 \pi^2}{8m} \left(\frac{n_1^2}{a_1^2} + \frac{n_2^2}{a_2^2} + \frac{n_3^2}{a_3^2}\right)
	\end{equation*}
	Se i valori degli $ a_j $ sono commensurabili, è possibile che lo spettro presenti delle degenerazioni: ad esempio, se si considerano $ a_1 = a_2 = a_3 \equiv a $, lo stato fondamentale $ E_{111} $ non presenta degenerazioni, ma già il primo stato eccitato è triplamente degenere: $ E_{211} = E_{121} = E_{112} $.
\end{example}

\begin{example}
	Un esempio di particolare importanza è l'oscillatore armonico tridimensionale: con lo stesso ragionamento di prima, si trova lo spettro:
	\begin{equation*}
		E_{n_1 n_1 n_3} = \hbar \left(n_1 \omega_1 + n_2 \omega_2 + n_3 \omega_3 + \frac{1}{2} \left(\omega_1 + \omega_2 + \omega_3\right)\right)
	\end{equation*}
	Nel caso in cui $ \omega_1 = \omega_2 = \omega_3 \equiv \omega $, si ha un potenziale a simmetria sferica $ \hat{V}(\hat{\ve{x}}) = \frac{1}{2} m \omega^2 \hat{\ve{x}}^2 $ e lo spettro diventa:
	\begin{equation*}
		E_{n_1 n_2 n_3} = \hbar \omega \left(n_1 + n_2 + n_3 + \frac{3}{2}\right) \equiv \hbar \omega \left(N + \frac{3}{2}\right)
	\end{equation*}
	È possibile calcolare la degenerazione dell'$ N $-esimo stato eccitato: $ n_1 $ può essere scelto in $ N + 1 $ modi, quindi $ n_2 $ può essere scelto in $ N + 1 - n_1 $ e, una volta scelti $ n_1 $ ed $ n_2 $, $ n_3 $ è fissato, dunque la degenerazione $ d(N) $ è:
	\begin{equation*}
		d(N) = \displaystyle\sum_{n_1 = 0}^{N} (N + 1 - n_1) = (N + 1)^2 - \frac{1}{2} N (N + 1) = \frac{1}{2} (N + 1) (N + 2)
	\end{equation*}
\end{example}

\section{Problema dei due corpi quantistico}

Il problema dei due corpi è un sistema in cui due corpi interagiscono tramite un potenziale che dipende solo dalla loro separazione:

\begin{equation}
	\mathcal{H} = \frac{\hat{\ve{p}}_1^2}{2m_1} + \frac{\hat{\ve{p}}_2^2}{2m_2} + \hat{V}(\hat{\ve{x}}_1 - \hat{\ve{x}}_2)
	\label{eq:17}
\end{equation}

Le variabili canoniche soddisfano la relazione di commutazione:

\begin{equation}
	\left[\hat{x}_{j,a}, \hat{p}_{k,b}\right] = i\hbar \delta_{jk} \delta_{ab} \qquad \left[\hat{x}_{j,a}, \hat{x}_{k,b}\right] = 0 \qquad \left[\hat{p}_{j,a}, \hat{p}_{k,b}\right] = 0
	\label{eq:18}
\end{equation}

dove $ a,b = 1,2 $ e $ j,k = 1,2,3 $.\\
%
Il problema è separabile definendo le coordinate relative e quelle del baricentro:

\begin{equation}
	\begin{split}
		\hat{\ve{r}} &\defeq \hat{\ve{x}}_1 - \hat{\ve{x}}_2\\
		\hat{\ve{R}} &\defeq \frac{m_1 \hat{\ve{x}}_1 + m_2 \hat{\ve{x}}_2}{m_1 + m_2}
	\end{split}
	\label{eq:19}
\end{equation}

A queste vanno associate i rispettivi impulsi congiunti:

\begin{equation}
	\begin{split}
		\hat{\ve{p}} &\defeq \frac{m_2 \hat{\ve{p}}_1 - m_1 \hat{\ve{p}}_2}{m_1 + m_2}\\
		\hat{\ve{P}} &\defeq \hat{\ve{p}}_1 + \hat{\ve{p}}_2
	\end{split}
	\label{eq:20}
\end{equation}

È pura algebra verificare che le variabili così definite soddisfino le relazioni di commutazione canoniche.\\
È altrettanto facile verificare che l'Hamiltoniana si può scrivere come:

\begin{equation}
	\mathcal{H} = \frac{\hat{\ve{P}}^2}{2M} + \frac{\hat{\ve{p}}^2}{2\mu} + \hat{V}(\hat{\ve{r}})
	\label{eq:21}
\end{equation}

dove sono state definite la massa totale $ M \equiv m_1 + m_2 $ e quella ridotta $ \mu^{-1} = m_1^{-1} + m_2^{-1} $.\\
Questa Hamiltoniana è manifestamente separabile come $ \mathcal{H} = \mathcal{H}_B(\hat{\ve{R}}, \hat{\ve{P}}) + \mathcal{H}_r(\hat{\ve{r}}, \hat{\ve{p}}) $:

\begin{equation}
	\begin{split}
		\mathcal{H}_B(\hat{\ve{R}}, \hat{\ve{P}}) &= \frac{\hat{\ve{P}}^2}{2M}\\
		\mathcal{H}_r(\hat{\ve{r}}, \hat{\ve{p}}) &= \frac{\hat{\ve{p}}^2}{2\mu} + \hat{V}(\hat{\ve{r}})
	\end{split}
	\label{eq:22}
	\qquad\qquad \left[\mathcal{H}_B, \mathcal{H}_r\right] = 0
\end{equation}

Lo spettro è facilmente determinabile poiché sono due problemi unidimensionali.\\
È importante capire che la scelta di variabili canoniche trasformate non è casuale, ma dettata dalla separabilità del termine potenziale, che fissa $ \hat{\ve{r}} $, dalle relazioni di commutazione, che per ogni scelta di $ \hat{\ve{R}} $ fissano gli impulsi coniugati, e dalla separabilità del termine cinetico che va a fissare di conseguenza $ \hat{\ve{R}} $ poiché rende univoca la scelta degli impulsi.

\subsection{Trasformazioni lineari di coordinate}

È possibile definire una generica trasformazione lineare di coordinate tramite una matrice di trasformazione $ \tens{M}\in\R^{d\times d} $:

\begin{equation}
	\hat{\ve{x}}' = \tens{M} \hat{\ve{x}}
	\label{eq:23}
\end{equation}

ovvero in componenti $ \hat{x}'_j = \sum_{k = 1}^{d} M_{jk} \hat{x}_k $.

\begin{proposition}
	Data una trasformazione lineare di coordiante $ \tens{M} $, gli impulsi coniugati trasformano secondo:
	\begin{equation}
		\hat{\ve{p}}'^{\,\intercal} = \hat{\ve{p}}^{\intercal} \tens{M}^{-1}
		\label{eq:24}
	\end{equation}
\end{proposition}
\begin{proof}
	Considerando $ \hat{\ve{p}}'^{\,\intercal} = \hat{\ve{p}}^{\intercal} \tens{N} $, in componenti $ \hat{p}'_j = \sum_{k = 1}^{d} \hat{p}_k N_{kj} $, dalle relazioni di commutazione canoniche si ha:
	\begin{equation*}
		\left[\hat{x}'_j, \hat{p}'_k\right] = \sum_{m = 1}^{d} \sum_{n = 1}^{d} M_{jm} N_{nk} \underbrace{\left[\hat{x}_m, \hat{p}_n\right]}_{i \hbar \delta_{mn}} = i\hbar \sum_{n = 1}^{d} M_{jn} N_{nk} \,\dot{=}\, i\hbar \delta_{jk} \quad \Longleftrightarrow \quad \tens{M}\tens{N} = \tens{I}_d
	\end{equation*}
\end{proof}

È possibile ricavare la trasformazione \ref{eq:24} anche partendo dai principi, costruendo gli impulsi coniugati come generatori di traslazioni spaziali. Nella rappresentazione delle coordinate:

\begin{equation}
	\braket{\hat{\ve{x}} | \hat{\ve{p}} | \hat{\ve{x}}'} = - i\hbar \nabla_{\ve{x}}\delta(\ve{x} - \ve{x}')
	\label{eq:25}
\end{equation}

Con abuso di notazione si può scrivere $ \hat{p}_j = -i\hbar\pa_j $, dunque la relazione di trasformazione è data dalla derivata composta:

\begin{equation}
	\hat{p}'_j = -i\hbar \frac{\pa }{\pa x'_j} = -i\hbar \displaystyle\sum_{k = 1}^{d} \frac{\pa x_k}{\pa x'_j} \frac{\pa}{\pa x_k}
	\label{eq:26}
\end{equation}

Dall'Eq. \ref{eq:23} si ha $ \frac{\pa x'_j}{\pa x_k} = M_{jk} $, dunque $ \frac{\pa x_k}{\pa x'_j} = M^{-1}_{kj} $, ovvero l'Eq. \ref{eq:24}.

\section{Problemi centrali}

Un generico problema centrale è quello determinato da un'Hamiltoniana del tipo:

\begin{equation}
	\mathcal{H} = \frac{\hat{\ve{p}}^2}{2m} + \hat{V}(\norm{\hat{\ve{x}}})
	\label{eq:27}
\end{equation}

Ovvero il potenziale dipende solo dal modulo dell'operatore posizione.\\
Analogamente al caso classico, l'obbiettivo è quello di separare il moto angolare da quello radiale; per fare ciò, è preferibile lavorare in coordinate sferiche:

\begin{equation}
	\begin{cases}
		x_1 = r \sin \vartheta \cos \varphi \\
		x_2 = r \sin \vartheta \sin \varphi \\
		x_3 = r \cos \vartheta
	\end{cases}
	\label{eq:28}
\end{equation}

In queste coordinare, si ha $ V = V(r) $.\\
In meccanica classica, dall'identità $ (\ve{a}\cdot\ve{b})^2 = \norm{\ve{a}}^2 \norm{\ve{b}}^2 - \norm{\ve{a}\times\ve{b}}^2 $ si può scomporre il termine cinetico in parte radiale e parte angolare, ottenendo $ \ve{p}^2 = p_r^2 + \frac{1}{r^2}\ve{L}^2 $. Quantisticamente, ciò non è così immediato poiché $ \hat{\ve{x}} $ e $ \hat{\ve{p}} $ non commutano.\\
Per capire come procedere, conviene prima dimostrare l'identità vettoriale utilizzata.

\begin{proposition}\label{cross-class}
	Dati $ \ve{a}, \ve{b} \in\R^3 $, si ha $ (\ve{a}\cdot\ve{b})^2 = \norm{\ve{a}}^2 \norm{\ve{b}}^2 - \norm{\ve{a}\times\ve{b}}^2 $.
\end{proposition}
\begin{proof}
	Ricordando che $ (\ve{a}\times\ve{b})_i = \sum_{j,k = 1}{3} \epsilon_{ijk} a_j b_k $, si ha:
	\begin{equation*}
		\norm{\ve{a}\times\ve{b}}^2 = \sum_{i,j,k,l,m = 1}^{3} \epsilon_{ijk} a_j b_k \epsilon_{ilm} a_l b_m = \sum_{i,j,k,l,m = 0}^{3} (\delta_{jl}\delta_{km} - \delta_{jm}\delta_{kl}) a_j b_k a_l b_m = \norm{\ve{a}}^2 \norm{\ve{b}}^2 - \norm{\ve{a}\times\ve{b}}^2
	\end{equation*}
\end{proof}

È necessario, inoltre, definire $ p_r $ ed $ \ve{L} $ in ambito quantistico:

\begin{equation}
	\tilde{p}_r \defeq \frac{1}{r} \hat{\ve{x}}\cdot\hat{\ve{p}}
	\label{eq:29}
\end{equation}

dove il tilde sta ad indicare il fatto che $ \tilde{p}_r $ non è un operatore hermitiano, dunque non è associato ad un'osservabile fisica.

\begin{proposition}
	Nella rappresentazione delle coordinate, si ha:
	\begin{equation}
		\tilde{p}_r = - i \hbar \frac{\pa}{\pa r}
		\label{eq:30}
	\end{equation}
\end{proposition}
\begin{proof}
	Nella rappresentazione delle coordinate:
	\begin{equation*}
		\begin{split}
			\tilde{p}_r &= -i\hbar \sum_{j = 1}^{3} \frac{x_j}{r} \pa_j = -i\hbar \sum_{j = 0}^{3} \frac{x_j}{r} \left(\pa_j r \frac{\pa}{\pa r} + \pa_j \vartheta \frac{\pa}{\pa \vartheta} + \pa_j \varphi \frac{\pa}{\pa \varphi}\right)\\
				    &= -i\hbar \sum_{j = 1}^{3} \frac{x_j}{r} \frac{x_j}{r} \frac{\pa}{\pa r} = -i\hbar \frac{\pa}{\pa r}
		\end{split}
	\end{equation*}
	dove si è usato il dato che $ \sum_{j = 1}^{3} x_j \pa_j \vartheta = \sum_{j = 1}^{3} x_j \pa_j \varphi $ ($ \nabla\vartheta, \nabla\varphi \perp \ve{x} = r\ve{e}_r $) e $ \pa_j r = \frac{x_j}{r} $.
\end{proof}

\begin{proposition}
	$ \left[\hat{r}, \tilde{p}_r\right] = i\hbar $.
\end{proposition}
\begin{proof}
	$ \left[\hat{r}, \tilde{p}_r\right]\psi = -i\hbar \left(r \frac{\pa}{\pa r} - \frac{\pa}{\pa r}r\right)\psi = i\hbar\psi $.
\end{proof}

Si evince quindi che $ \tilde{p}_r $ è canonicamente coniugato a $ \hat{r} $, ovvero genera le traslazioni lungo la coordinata radiale.\\
A questo punto, è possibile definire l'analogo quantistico di $ \ve{L} $:
\begin{equation}
	\hat{\ve{L}} \defeq \hat{\ve{x}}\times\hat{\ve{p}}
	\label{eq:1.31}
\end{equation}
A priori, non si può dire che questo sia l'operatore quantistico associato al momento angolare, ma si dimostrerà essere tale. Sulla base delle coordinate:
\begin{equation}
	L_j = -i\hbar \sum_{j,k = 1}^{3} \epsilon_{ijk} x_j \frac{\pa}{\pa x_k}
	\label{eq:1.32}
\end{equation}

\begin{proposition}\label{r-l-comm}
	$ [ \hat{\ve{r}}, \hat{L}_j ] = 0 $.
\end{proposition}
\begin{proof}
	Basta dimostrare che $ \hat{\ve{L}} $ non ha componenti radiali:
	\begin{equation*}
		\hat{\ve{x}}\cdot\hat{\ve{L}} = \sum_{i = 1}^{3} \hat{x}_i \hat{L}_i = \sum_{i,j,k = 1}^{3} \epsilon_{ijk} \hat{x}_i \hat{x}_j \pa_k = \frac{1}{2} \sum_{i,j,k = 1}^{3} \epsilon_{ijk} [\hat{x}_i,\hat{x}_j] \pa_k = \frac{1}{2} \sum_{i,j,k = 1}^{3} \epsilon_{ijk} \delta_{ij} \pa_k = 0
	\end{equation*}
\end{proof}
Utilizzando lo stesso procedimento usato per dimostrare la Prop. \ref{cross-class}:
\begin{equation*}
	\begin{split}
		\hat{\ve{L}}^2
		&= \sum_{i,j,k,a,b = 1}^{3} \epsilon_{ijk} \epsilon_{iab} \hat{x}_j \hat{p}_k \hat{x}_a \hat{p}_b = \sum_{i,j,k,a,b = 1}^{3} \left( \delta_{ja}\delta_{kb} - \delta_{jb}\delta_{ka} \right) \hat{x}_j \hat{p}_k \hat{x}_a \hat{p}_b = \sum_{j,k = 1}^{3} \left( \hat{x}_j \hat{p}_k \hat{x}_j \hat{p}_k - \hat{x}_j \hat{p}_k \hat{x}_k \hat{p}_j \right)\\
		&= \sum_{j,k = 1}^{3} \left( \hat{x}_j \hat{x}_j \hat{p}_k \hat{p}_k + \hat{x}_j [\hat{p}_k,\hat{x}_j] \hat{p}_k - \hat{x}_j \hat{x}_k \hat{p}_k \hat{p}_j - \hat{x}_j [\hat{p}_k,\hat{x}_k]\hat{p}_j \right)\\
		&= \hat{\ve{x}}^2 \hat{\ve{p}}^2 - i\hbar \hat{\ve{x}}\cdot\hat{\ve{p}} + \sum_{j,k = 1}^{3} \left( - \hat{x}_k \hat{x}_j \hat{p}_k \hat{p}_j + i\hbar \delta_{kk} \hat{x}_j \hat{p}_j \right)\\
		&= \hat{\ve{x}}^2 \hat{\ve{p}}^2 - i\hbar \hat{\ve{x}}\cdot\hat{\ve{p}} - \sum_{j,k = 1}^{3} \left( \hat{x}_k \hat{p}_k \hat{x}_j \hat{p}_j + \hat{x}_k [\hat{x}_j,\hat{p}_k] \hat{p}_j \right) + 3i\hbar \hat{\ve{x}}\cdot\hat{\ve{p}}\\
		&= \hat{\ve{x}}^2 \hat{\ve{p}}^2 + 2i\hbar\hat{\ve{x}}\cdot\hat{\ve{p}} - \left( \hat{\ve{x}}\cdot\hat{\ve{p}} \right)^2 - i\hbar\hat{\ve{x}}\cdot\hat{\ve{p}} = \hat{\ve{x}}^2 \hat{\ve{p}}^2 - \left( \hat{\ve{x}}\cdot\hat{\ve{p}} \right)^2 + i\hbar \hat{\ve{x}}\cdot\hat{\ve{p}}
	\end{split}
\end{equation*}
Rispetto al caso classico è presente un termine in più. Ricordando che $ \hat{\ve{x}}^2 \equiv \hat{r}^2 $:
\begin{equation*}
	\begin{split}
		\hat{\ve{p}}^2
		&= \frac{1}{r^2} \hat{\ve{L}}^2 + \frac{1}{r^2} \left( \hat{\ve{x}}\cdot\hat{\ve{p}} \right)^2 - \frac{i\hbar}{r^2} \hat{\ve{x}}\cdot\hat{\ve{p}} = \frac{1}{r^2} \hat{\ve{L}}^2 - \frac{\hbar^2}{r^2} r \frac{\pa}{\pa r} r \frac{\pa}{\pa r} - \frac{\hbar^2}{r^2} r \frac{\pa}{\pa r}\\
		&= \frac{1}{r^2} \hat{\ve{L}}^2 - \frac{\hbar^2}{r^2} r \frac{\pa}{\pa r} \left( r \frac{\pa}{\pa r} + 1 \right) - \frac{\hbar^2}{r^2} r \frac{\pa}{\pa r} = \frac{1}{r^2} \hat{\ve{L}}^2 - \frac{\hbar^2}{r^2} r^2 \frac{\pa^2}{\pa r^2} - \frac{\hbar^2}{r^2} r \frac{\pa}{\pa r} - \frac{\hbar^2}{r^2} r \frac{\pa}{\pa r}
	\end{split}
\end{equation*}
Data la Prop. \ref{r-l-comm}, è indifferente l'ordine in cui si applicano $ \frac{1}{r^2} $ e $ \hat{\ve{L}}^2 $, dunque:
\begin{equation}
	\hat{\ve{p}}^2 = -\hbar^2 \frac{\pa^2}{\pa r^2} - 2\hbar^2 \frac{1}{r} \frac{\pa}{\pa r} + \frac{\hat{\ve{L}}^2}{r^2}
	\label{eq:1.33}
\end{equation}
È possibile ricondurre l'Hamiltoniana in Eq. \ref{eq:27} alla sua forma separata classica hermitianizzando l'operatore $ \tilde{p}_r $:
\begin{equation*}
	\tilde{p}^{\dagger}_r = \hat{\ve{p}}^{\dagger} \cdot \frac{\hat{\ve{x}}^{\dagger}}{\hat{r}^{\dagger}} = \hat{\ve{p}} \cdot \frac{\hat{\ve{x}}}{\hat{r}} = -i\hbar \sum_{i = 1}^{3} \pa_i \frac{x_i}{r} = - i\hbar \sum_{i = 1}^{3} \left( \frac{x_i}{r} \pa_i + \pa_i \left( \frac{x_i}{r} \right) \right) = \tilde{p}_r - \frac{2i\hbar}{\hat{r}}
\end{equation*}
Ricordando che l'hermitianizzazione avviene tramite $ \hat{a} = \frac{1}{2} (\tilde{a} + \tilde{a}^{\dagger}) $, si definisce l'impulso radiale autoaggiunto come:
\begin{equation}
	\hat{p}_r \defeq \hat{p}_r - \frac{i\hbar}{\hat{r}}
	\label{eq:1.34}
\end{equation}
ovvero, sulla base delle coordinate:
\begin{equation}
	\hat{p}_r = -i\hbar \left( \frac{\pa}{\pa r} + \frac{1}{r} \right)
	\label{eq:1.35}
\end{equation}
Per esprimere $ \hat{\ve{p}}^2 $ in funzione di $ \hat{p}_r $, si calcola:
\begin{equation*}
	\begin{split}
		\hat{p}^2_r
		&= -\hbar^2 \left( \frac{\pa}{\pa r} + \frac{1}{r} \right) \left( \frac{\pa}{\pa r} + \frac{1}{r} \right) = -\hbar^2 \left( \frac{\pa^2}{\pa r^2} + \frac{1}{r} \frac{\pa}{\pa r} + \frac{\pa}{\pa r} \frac{1}{r} + \frac{1}{r^2} \right)\\
		&= -\hbar^2 \left( \frac{\pa^2}{\pa r^2} + \frac{2}{r} \frac{\pa}{\pa r} - \frac{1}{r^2} + \frac{1}{r^2} \right) = -\hbar^2 \left( \frac{\pa^2}{\pa r^2} + \frac{2}{r} \frac{\pa}{\pa r} \right)
	\end{split}
\end{equation*}
Si trova dunque un'espressione che coincide con quella classica:
\begin{equation}
	\hat{\ve{p}}^2 = \hat{p}^2_r + \frac{\hat{\ve{L}}^2}{\hat{r}^2}
	\label{eq:1.36}
\end{equation}
L'Hamiltoniana si separa come:
\begin{equation}
	\mathcal{H} = \frac{\hat{p}_r^2}{2m} + \hat{V}(\hat{r}) + \frac{\hat{\ve{L}}^2}{2m\hat{r}^2}
	\label{eq:1.37}
\end{equation}
Questa Hamiltoniana non è separata in senso proprio, poiché i due termini non agiscono su spazi separati; tuttavia, si vede che $ [\hat{\ve{L}}^2,\mathcal{H}] = 0 $, dunque sono diagonalizzabili simultaneamente: una volta determinato lo spettro di $ \hat{\ve{L}}^2 $, il problema diventa unidimensionale (radiale).\\
In questo caso, quindi, le autofunzioni non sono esprimibili come prodotto di autofunzioni su spazi separati, ma la semplificazione del problema deriva da una simmetria: la simmetria per rotazioni.












\chapter{Momento Angolare}
\selectlanguage{italian}

\section{Momento angolare e rotazioni}

\paragraph{Caso classico}

Per il Th. di Noether, associate alle invarianze per rotazioni attorno ai tre assi coordianti si hanno tre cariche di Noether conservate.\\
Si considerino $ \ve{x} = \left( r\cos \phi, r\sin \phi \right) \equiv \left( x_1, x_2 \right) $ nel piano $ z = 0 $ ed una rotazione attorno all'asse $ z $ di un angolo infinitesimo $ \varepsilon $: questa causa uno spostamento $ \delta\ve{x} $ dato da:
\begin{equation*}
	\begin{split}
		\delta\ve{x}
		&= \left( r\cos (\phi + \varepsilon), r\sin (\phi + \varepsilon) \right) - \left( r\cos \phi, r\sin \phi \right)\\
		&= \left( -r \varepsilon \sin \phi, r \varepsilon \cos \phi \right) + o(\varepsilon) = \varepsilon \left( -x_2, x_1 \right) + o(\varepsilon)
	\end{split}
\end{equation*}
Quindi, per una generica rotazione attorno ad un asse dato dal versore $ \ve{n} $ si ha:
\begin{equation}
	\delta x_i = \varepsilon \sum_{j,k = 1}^{3} \epsilon_{ijk} n_j x_k \quad\Longleftrightarrow\quad \delta\ve{x} = \varepsilon \ve{n}\times\ve{x}
	\label{eq:2.1}
\end{equation}
Nel caso di una rotazione attorno al $ j $-esimo asse coordinato $ \delta x_i^{(j)} = \varepsilon \sum_{k = 1}^{3} \epsilon_{ijk} x_k $, quindi la carica di Noether associata è:
\begin{equation}
	q_j \defeq \sum_{i = 1}^{3} \frac{\pa L}{\pa \dot{x}_i} \delta x_i^{(j)} = \varepsilon \sum_{i,k = 1}^{3} \epsilon_{jki} x_k p_i = \varepsilon L_j
	\label{eq:2.2}
\end{equation}
Dunque l'invarianza per rotazioni attorno ad un asse ha come quantità conservata associata la componente del momento angolare lungo tale asse.

\paragraph{Caso quantistico}

Bisogna innanzitutto verificare che $ \hat{\ve{L}} $ definito in Eq. \ref{eq:1.31} sia effettivamente il momento angolare, ovvero il generatore delle rotazioni (a meno di un fattore $ \hbar $): questo equivale a verificare che l'operatore $ \hat{R}_{\varepsilon} $, definito come:
\begin{equation}
	\hat{R}_{\varepsilon} = e^{i\frac{\varepsilon}{\hbar} \ve{n}\cdot\hat{\ve{L}}} = \tens{I}_3 + i\frac{\varepsilon}{\hbar} \ve{n}\cdot\hat{	\ve{L}} + o(\varepsilon)
	\label{eq:2.3}
\end{equation}
realizzi una rotazione di angolo infinitesimo $ \varepsilon $ attorno all'asse $ \ve{n} $, ovvero:
\begin{equation}
	\braket{\ve{x} | \hat{R}_{\varepsilon} | \psi} = \psi(\ve{x} + \delta_{\ve{n}}\ve{x}) = \psi(\ve{x}) + \delta_{\ve{n}}\ve{x}\cdot\nabla\psi(\ve{x}) + o(\varepsilon)
	\label{eq:2.4}
\end{equation}
dove $ \delta_{\ve{n}}\ve{x} = \varepsilon \ve{n}\times\ve{x} $. Calcolando gli elementi di matrice di $ \hat{R}_{\varepsilon} $ sulla base delle posizioni:
\begin{equation}
	\braket{\ve{x} | \hat{R}_{\varepsilon} | \psi} = \psi(\ve{x}) + i \frac{\varepsilon}{\hbar} \cdot (-i\hbar) \sum_{i,j,k = 1}^{3} n_i \epsilon_{ijk} x_j \pa_k \psi(\ve{x}) + o(\varepsilon)
	\label{eq:2.5}
\end{equation}
Confontando le Eq. \ref{eq:2.4} - \ref{eq:2.5}, si vede che sono uguali, dunque $ \hat{\ve{L}} $ è il generatore delle rotazioni.

\section{Proprietà}

\subsection{Espressione esplicita}

Innanzitutto si noti che dalla definizione in Eq. \ref{eq:1.31} discende subito che $ \hat{\ve{L}} $ è hermitiano:
\begin{equation}
	\hat{L}_i^{\dagger} = \sum_{j,k = 1}^{3} \epsilon_{ijk} \hat{p}_k \hat{x}_j = \sum_{j,k = 1}^{3} \epsilon_{ijk} \left( [\hat{p}_k,\hat{x}_j] + \hat{x}_j \hat{p}_k \right) = L_i + i\hbar \sum_{j,k = 1}^{3} \epsilon_{ijk} \delta_{jk} = L_i
	\label{eq:2.6}
\end{equation}
È anche possibile calcolare esplicitamente l'espressione di $ \hat{\ve{L}} $ in coordinate sferiche:
\begin{equation}
	\hat{L}_x = i\hbar \left( \sin \phi \frac{\pa}{\pa \theta} + \frac{\cos \theta}{\sin \theta} \cos \phi \frac{\pa}{\pa \phi} \right)
	\label{eq:2.7}
\end{equation}
\begin{equation}
	\hat{L}_y = i\hbar \left( -\cos \phi \frac{\pa}{\pa \theta} + \frac{\cos \theta}{\sin \theta} \sin \phi \frac{\pa}{\pa \phi} \right)
	\label{eq:2.8}
\end{equation}
\begin{equation}
	L_z = -i\hbar \frac{\pa}{\pa \phi}
	\label{eq:2.9}
\end{equation}
Si ha inoltre:
\begin{equation}
	\hat{\ve{L}}^2 \equiv \hat{L}^2 = -\hbar^2 \left( \frac{\pa^2}{\pa \theta^2} + \frac{\sin \theta}{\cos \theta} \frac{\pa}{\pa \theta} + \frac{1}{\sin^2 \theta} \frac{\pa^2}{\pa \phi^2} \right)
	\label{eq:2.10}
\end{equation}

\subsection{Commutatori}

Sebbene in un sistema invariante per rotazioni il momento angolare commuti con l'Hamiltoniana, le componenti di $ \hat{\ve{L}} $ non commutano tra loro

\begin{lemma}\label{lem-l-comm}
	$ \hat{x}_i \hat{p}_j - \hat{x}_j \hat{p}_i = \sum_{k = 1}^{3} \epsilon_{ijk}	\hat{L}_k $.
\end{lemma}
\begin{proof}
	$ \sum_{k = 1}^{3} \epsilon_{ijk} \hat{L}_k = \sum_{k,a,b = 1}^{3} \epsilon_{ijk}\epsilon_{kab} \hat{x}_a \hat{p}_b = \sum_{k,a,b = 1}^{3} \left( \delta_{ia}\delta_{jb} - \delta_{ib}\delta_{ja} \right) \hat{x}_a \hat{p}_b = \hat{x}_i \hat{p}_j - \hat{x}_j \hat{p}_i $.
\end{proof}

\begin{proposition}\label{l-comm}
	$ [\hat{L}_i,\hat{L}_j] = i\hbar \sum_{k = 1}^{3} \epsilon_{ijk} \hat{L}_k $.
\end{proposition}
\begin{proof}
	Usando nell'ultima uguaglianza il Lemma \ref{lem-l-comm}:
	\begin{equation*}
		\begin{split}
			[\hat{L}_i,\hat{L}_j]
			&= \sum_{a,b,l,m = 1}^{3} \epsilon_{iab}\epsilon_{jlm} [\hat{x}_a \hat{p}_b,\hat{x}_l \hat{p}_m] = \sum_{a,b,l,m = 1}^{3}  \epsilon_{iab} \epsilon_{jlm} \left( \hat{x}_l [\hat{x}_a,\hat{p}_m] \hat{p}_b + \hat{x}_a [\hat{p}_b,\hat{x}_l] \hat{p}_m \right)\\
			&= i\hbar \sum_{a,b,l = 1}^{3} \epsilon_{bia} \epsilon_{jla} \hat{x}_l \hat{p}_b - i\hbar \sum_{a,b,m = 1}^{3} \epsilon_{iab} \epsilon_{mjb} \hat{x}_a \hat{p}_m\\
			&= i\hbar \sum_{a,b,l = 1}^{3} \left( \delta_{bj}\delta_{il} - \delta_{bl}\delta_{ji} \right) \hat{x}_l \hat{p}_b - i\hbar \sum_{a,b,m = 1}^{3} \left( \delta_{im} \delta_{aj} - \delta_{ij}\delta_{am} \right) \hat{x}_a \hat{p}_m\\
			&= i\hbar \left( \hat{x}_i \hat{p}_j - \delta_{ij} \hat{\ve{x}}\cdot\hat{\ve{p}} - \hat{x}_j \hat{p}_i + \hat{\ve{x}}\cdot\hat{\ve{p}} \delta_{ij} \right) = i\hbar \left( \hat{x}_i \hat{p}_j - \hat{x}_j \hat{p}_i \right) = i\hbar \sum_{k = 1}^{3} \epsilon_{ijk} \hat{L}_k
		\end{split}
	\end{equation*}
\end{proof}

Si ricordi che il commutatore tra un operatore hermitiano $ \hat{G} $, generatore della trasformazione (anch'essa hermitiana) $ \hat{T} = e^{i \varepsilon \hat{G}} $, ed un generico operatore $ \hat{A} $ può essere calcolato da:
\begin{equation}
	\hat{A}' = \hat{T}^{-1} \hat{A} \hat{T} = \left( \tens{I} - i\varepsilon\hat{G} \right) \hat{A} \left( \tens{I} + i\varepsilon\hat{G} \right) = \hat{A} + i\varepsilon [\hat{A},\hat{G}] \quad\Longrightarrow\quad [\hat{A},\hat{G}] = \frac{1}{i\varepsilon} \delta\hat{A}
	\label{eq:2.11}
\end{equation}
Dunque dalla Prop. \ref{l-comm} è possibile vedere come trasforma $ \hat{L}_i $ sotto la rotazione data da $ \hat{L}_j $, e confrontandola con l'Eq. \ref{eq:2.1} si vede che $ \hat{\ve{L}} $ trasforma proprio come un vettore sotto rotazioni (cosa non scontata).\\
Ciò suggerisce naturalmente che $ \hat{L}^2 $, essendo invariante per rotazioni, commuti con ciascuna $ \hat{L}_i $:
\begin{equation}
	[\hat{L}^2,\hat{L}_i] = \sum_{k = 1}^{3} [\hat{L}_k\hat{L}_k,\hat{L}_i] = i\hbar \sum_{j,k = 1}^{3} \epsilon_{kij} \left( \hat{L}_k\hat{L}_j + \hat{L}_j\hat{L}_k \right) = 0
	\label{eq:2.12}
\end{equation}
nullo poiché prodotto di simbolo completamente antisimmetrico con operatore simmetrico.












\chapter{Sistemi Tridimensionali}
\selectlanguage{italian}

\section{Equazione di Schrödinger radiale}

Si consideri una generica Hamiltoniana invariante per rotazioni, ad esempio:
\begin{equation}
	\mathcal{H} = \frac{
	p^2}{2m} + V(r) = \frac{p_r^2}{2m} + \frac{L^2}{2mr^2} + V(r)
	\label{eq:3.1}
\end{equation}
È evidente che questa Hamiltoniana non si possa separare in parte radiale e parte angolare a causa del termine $ \frac{L^2}{r^2} $. È però possibile diagonalizzare simultaneamente $ \hat{H} $, $ \hat{L}^2 $ e $ \hat{L}_z $, dunque si proietta sugli autostati del momento angolare:
\begin{equation}
	\psi(\ve{x}) = \sum_{\ell = 0}^{\infty} \sum_{m = -\ell}^{\ell} \braket{\ve{x} | \ell,m} \braket{\ell,m | \psi} = \sum_{\ell = 0}^{\infty} \sum_{m = -\ell}^{\ell} Y_{\ell,m}(\vartheta,\varphi) \phi_{\ell,m}(r)
	\label{eq:3.2}
\end{equation}
L'equazione di Schrödinger si riduce quindi in una PDE con una sola incognita:
\begin{equation}
	\left[ \frac{p_r^2}{2m} + \frac{\hbar^2 \ell (\ell + 1)}{2mr^2} + V(r) \right] \phi_{\ell,m}(r) = E \phi_{\ell,m}(r)
	\label{eq:3.3}
\end{equation}
Questa non dipende da $ m $, dunque fissati $ \ell $ ed $ E $ c'è una degerazione di $ 2\ell + 1 $; si pone $ \phi_{\ell,m}(r) \equiv \phi_{\ell}(r) $.
È inoltre utile porre:
\begin{equation}
	\phi_{\ell}(r) \equiv \frac{u_{\ell}(r)}{r}
	\label{eq:3.4}
\end{equation}

\begin{proposition}
	$ \hat{p}_r^n \phi_{\ell}(r) = \left( -i\hbar \right)^n \frac{1}{r} \frac{\pa^n}{\pa r^n} u_{\ell}(r) $.
\end{proposition}
\begin{proof}
	$ \hat{p}_r \phi_{\ell}(r) = -i\hbar \left( \frac{\pa}{\pa r} + \frac{1}{r} \right) \frac{u_{\ell}(r)}{r} = -i\hbar \frac{1}{r} \frac{\pa}{\pa r} u_{\ell}(r) $.
\end{proof}
Una ragione $ \virgolette{fisica} $ per definire $ u_{\ell}(r) $ è che assorbe la misura d'integrazione nel prodotto scalare:
\begin{equation*}
	\braket{\psi' | \psi} = \int_{0}^{\infty} dr \,r^2 \phi'^*_{\ell'}(r) \phi_{\ell}(r) \int_{\mathbb{S}^2} d\cos\vartheta \,d\varphi \, Y^*_{\ell',m'}(\vartheta,\varphi) Y_{\ell,m}(\vartheta,\varphi) = \delta_{\ell,\ell'} \delta_{m,m'} \int_0^{\infty} dr \, u'^*_{\ell'}(r) u_{\ell}(r)
\end{equation*}
Ciò rende $ \hat{p}_r $ un operatore hermitiano sulle $ u_{\ell} $, ed infatti l'equazione di Schrödinger diventa:
\begin{equation}
	\left[ - \frac{\hbar^2}{2m} + \frac{\hbar^2 \ell (\ell + 1)}{2m r^2} + V(r) \right] u_{\ell}(r) = E u_{\ell}(r)
	\label{eq:3.5}
\end{equation}

\subsection{Condizioni al contorno}

È necessario che la funzione d'onda radiale $ \phi_{\ell}(r) $ abbia densità di probabilità integrabile su $ [0,+\infty) $: in particolare, si richiede che il seguente integrale non diverga:
\begin{equation}
	\braket{\phi_{\ell} | \phi_{\ell}} = \int_0^{\infty} dr \, r^2 \abs{\phi_{\ell}(r)}^2 = \int_{0}^{\infty} dr  \abs{u_{\ell}(r)}^2
	\label{eq:3.6}
\end{equation}
Nell'origine $ \abs{u_{\ell}(r)}^2 $ deve avere al più una singolarità integrabile, dunque:
\begin{equation}
	u_{\ell}(r) \overset{r \rightarrow 0}{\sim} \frac{1}{r^{\delta}} : \delta < \frac{1}{2}
	\label{eq:3.7}
\end{equation}
Dato che $ \lap \frac{1}{r} = - 4\pi \delta^3 (\ve{x}) $, se $ \phi_{\ell}(r) $ diverge nell'origine almeno come $ \frac{1}{r} $ è possibile soddisfare l'equazione di Schrödinger solo se il potenziale nell'origine diverge almeno come una delta di Dirac. Per potenziali non-distribuzionali (ovvero funzioni, quindi non singolari) $ \phi_{\ell}(r) $ deve divergere nell'origine meno di $ \frac{1}{r} $, dunque:
\begin{equation}
	\lim_{r \rightarrow 0} r \phi_{\ell}(r) = 0 \quad\Longrightarrow\quad \lim_{r \rightarrow 0} u_{\ell}(r) = 0
	\label{eq:3.8}
\end{equation}

\paragraph{Andamento nell'origine}

I potenziali d'interesse fisico sono quelli che per $ r \rightarrow 0 $ divergono meno di $ \frac{1}{r^2} $: il caso in cui nell'origine $ V(r) $ vada come $ \frac{1}{r^k} $ con $ k \ge 2 $ è patologicamente attrattivo e si dimostra non avere un'energia minima, ovvero non presenta stati stabili.\\
Se quindi nell'origine $ V(r) \sim \frac{1}{r^k} $ con $ k < 2 $, per $ r \rightarrow 0 $ a dominare è il termine centrifugo:
\begin{equation}
	- \frac{\hbar^2}{2m} \frac{d^2 u_{\ell}(r)}{d r^2} + \frac{\hbar^2 \ell (\ell + 1)}{2mr^2} u_{\ell}(r) = 0 \quad \Longrightarrow \quad \frac{d^2 u_{\ell}(r)}{d r^2} = \frac{\ell (\ell + 1)}{r^2} u_{\ell}(r)
	\label{eq:3.9}
\end{equation}
La soluzione generale di questa equazione è $ u_{\ell}(r) = A r^{\ell + 1} + B r^{-\ell} $, ma il secondo termine non soddisfa la condizione in Eq. \ref{eq:3.8}, dunque:
\begin{equation}
	u_{\ell}(r) \overset{r \rightarrow 0}{\sim} A r^{\ell + 1}
	\label{eq:3.10}
\end{equation}

\paragraph{Andamento all'infinito}

Se all'infinito il potenziale si annulla, per $ r \rightarrow \infty $ l'andamento della funzione d'onda è quello della particella libera: se esistono stati legati, ovvero autostati di $ \hat{\mathcal{H}} $ con $ E < 0 $, l'andamento della soluzione è:
\begin{equation}
	u_{\ell}(r) \overset{r \rightarrow \infty}{\sim} C e^{- \beta r}, \,\, \beta \equiv \frac{\sqrt{2m\abs{E}}}{\hbar}
	\label{eq:3.11}
\end{equation}
Se invece $ \lim_{r \rightarrow \infty} V(r) \neq 0 $, l'andamento va studiato caso per caso.

\paragraph{Stati legati}

Per un potenziale unidimensionale esiste sempre almeno uno stato legato; inoltre, lo stato fondamentale ha una funzione d'onda pari e gli stati eccitati hanno parità alternata.\\
Nel caso tridimensionale, è possibile interpretare il problema come un problema unidimensionale con dominio $ [0,\infty) $: la condizione in Eq. \ref{eq:3.8}, però, impone che solo le soluzioni dispari sono accettabili, dunque in generale non è detto che esista lo stato fondamentale. Per un potenziale tridimensionale, quindi, non è detto a priori che esistano stati legati.

\section{Particella libera}

L'equazione di Schrödinger per la particella libera, quindi per $ V(\ve{x}) = 0 $, è:
\begin{equation}
	-\frac{\hbar^2}{2m} \lap \psi(\ve{x}) = E \psi(\ve{x})
	\label{eq:3.12}
\end{equation}
Questa è una PDE separabile e, in coordinate cartesiane, le soluzioni sono delle onde piane:
\begin{equation}
	\psi_{\ve{k}}(\ve{x}) = \frac{1}{(2\pi)^{3/2}} e^{i \ve{k}\cdot\ve{x}}, \,\, E = \frac{\hbar^2 \ve{k}^2}{2m}
	\label{eq:3.13}
\end{equation}
normalizzate in senso improprio:
\begin{equation}
	\braket{\psi_{\ve{k}'} | \psi_{\ve{k}}} = \int_{\R^3} d^3\ve{x}\, \psi^*_{\ve{k}'}(\ve{x}) \psi_{\ve{k}}(\ve{x}) = \delta^3(\ve{k} - \ve{k}')
	\label{eq:3.14}
\end{equation}
Un differente approccio risolutivo è quello che sfrutta la simmetria rotazionale del problema:
\begin{equation}
	\psi_{\ell,m}(\ve{x}) = Y_{\ell,m}(\vartheta,\varphi) \frac{u_{\ell}(r)}{r}
	\label{eq:3.15}
\end{equation}
Dall'Eq. \ref{eq:3.5}:
\begin{equation}
	\frac{\hbar^2}{2mE} \left[ -\frac{d^2}{dr^2}u_{\ell}(r) + \frac{\ell(\ell + 1)}{r^2}u_{\ell}(r) \right] - u_{\ell}(r) = 0
	\label{eq:3.16}
\end{equation}
Ponendo $ k = \sqrt{\frac{2mE}{\hbar^2}} $ e $ r' = kr $, si ottiene:
\begin{equation}
	\frac{d^2}{dr'^2} u_{\ell}(r') - \frac{\ell(\ell + 1)}{r'^2} u_{\ell}(r') + u_{\ell}(r') = 0
	\label{eq:3.17}
\end{equation}
Questa è una ODE di Bessel e la sua soluzione si ottiene introducendo la funzione di Bessel $ j_{\ell}(r') $ (è noto che $ j_{\ell}(x) \sim x^{\ell} $ per $ x \rightarrow 0 $):
\begin{equation}
	u_{\ell}(r') = r j_{\ell}(r')
	\label{eq:3.18}
\end{equation}
Le autofunzioni della particella libera tridimensionale sono dunque:
\begin{equation}
	\psi_{\ell,m}(\ve{x}) = Y_{\ell,m}(\vartheta,\varphi) j_{\ell}(kr), \,\, k = \frac{\sqrt{2mE}}{\hbar}
	\label{eq:3.19}
\end{equation}

\section{Oscillatore armonico isotropo}

L'oscillatore armonico isotropo è descritto dal seguente potenziale:
\begin{equation}
	V(r) = \frac{1}{2} m\omega^2 r^2
	\label{eq:3.20}
\end{equation}
Essendo un potenziale centrale, il problema si riduce al problema radiale.
L'Hamiltoniana radiale corrispondente è:
\begin{equation}
	\mathcal{H}_{\ell} = \frac{p_r^2}{2m} + \frac{\hbar^2 \ell(\ell + 1)}{2mr^2} + \frac{1}{2} m\omega^2 r^2
	\label{eq:3.21}
\end{equation}
La sua equazione agli autovalori può essere scritta come:
\begin{equation}
	\hat{\mathcal{H}}_{\ell} \ket{k,\ell,m} = E_{k,\ell} \ket{k,\ell,m}
	\label{eq:3.22}
\end{equation}

\subsection{Stati con \texorpdfstring{$ \ell = 0 $}{TEXT}}

Per $ \ell = 0 $, il problema si riduce all'oscillatore armonico unidimensionale:
\begin{equation}
	\mathcal{H}_0 = \frac{p_r^2}{2m} + \frac{1}{2} m \omega^2 r^2
	\label{eq:3.23}
\end{equation}
Per risolvere con metodo algebrico, si defisce l'operatore di distruzione come:
\begin{equation}
	\hat{d}_0 \defeq \sqrt{\frac{m\omega}{2\hbar}} \left( \hat{r} + i \frac{\hat{p}_r}{m\omega} \right)
	\label{eq:3.24}
\end{equation}
ed il relativo operatore di creazione $ d_0^{\dagger} $. In questo modo, si può scrivere:
\begin{equation}
	\hat{\mathcal{H}}_0 = \hbar \omega \left( \hat{d}_0^{\dagger}\hat{d}_0 + \frac{1}{2} \right)
	\label{eq:3.25}
\end{equation}
Dato che $ \left[ \hat{r},\hat{p}_r \right] = i\hbar $, si ha $ \left[ \hat{d}_0, \hat{d}_0^{\dagger} \right] = 1 $, dunque lo spettro di $ \hat{\mathcal{H}}_0 $ è lo stesso dell'oscillatore armonico unidimensionale:
\begin{equation}
	u_{k,0}(r) = \mathcal{N}_k e^{-c x^2} H_k(r)
	\label{eq:3.26}
\end{equation}
Dato che $ u_{k,0}(-x) = (-1)^k u_{k,0}(x) $, per la condizione al contorno in Eq. \ref{eq:3.8} sono ammissibili solo le soluzioni dispari. Ridefinendo $ k $ così che $ u_{k,0}(r) $ sia la $ (2k+1) $-esima soluzione, si trova lo spettro dell'Hamiltoniana:
\begin{equation}
	E_{k,0} = \hbar \omega \left( 2k + \frac{3}{2} \right)
	\label{eq:3.27}
\end{equation}
Dato che il caso $ \ell = 0 $ è quello maggiormente attrattivo, per l'assenza del termine centrifugo, si trova che lo stato fondamentale è $ \ket{0,0,0} $, con energia $ E_{0,0} = \frac{3}{2} \hbar \omega $.

\subsection{Stati con \texorpdfstring{$ \ell $}{TEXT} generico}

Per scrivere l'Hamiltoniana $ \hat{\mathcal{H}}_{\ell} $ in una forma simile a Eq. \ref{eq:3.25} è necessario definire degli operatori di distruzione/creazione generalizzati:
\begin{equation}
	\hat{d}_{\ell} \defeq \sqrt{\frac{m\omega}{2\hbar}} \left( \left( \hat{r} + \frac{\hbar\ell}{m\omega\hat{r}} \right) + i \frac{\hat{p}_r}{m\omega} \right)
	\label{eq:3.28}
\end{equation}
È utile inoltre ricordare che, essendo $ \left[ \hat{p}_r, f(\hat{r}) \right] = -i\hbar f(\hat{r}) $, si ha $ \left[ \hat{p}_r, \frac{1}{\hat{r}} \right] = \frac{i\hbar}{\hat{r}^2} $.\\
Si può generalizzare l'operatore numero come:
\begin{equation}
	\hat{D}_{\ell} \defeq \hat{d}_{\ell}^{\dagger} \hat{d}_{\ell}
	\label{eq:3.29}
\end{equation}
La sua espressione esplicita è:
\begin{equation*}
	\begin{split}
		D_{\ell}
		&= d_{\ell}^{\dagger} d_{\ell} = \frac{m\omega}{2\hbar} \left( \left( r + \frac{\hbar\ell}{m\omega r} \right) - i \frac{p_r}{m\omega} \right) \left( \left( r + \frac{\hbar\ell}{m\omega r} \right) + i \frac{p_r}{m\omega} \right)\\
		&= \frac{m\omega}{2\hbar} \left( \left( r + \frac{\hbar\ell}{m\omega r} \right)^2 + \frac{p_r^2}{m^2 \omega^2} + \left[ \left( r + \frac{\hbar\ell}{m\omega r} \right), i \frac{p_r}{m\omega} \right] \right)\\
		&= \frac{m\omega}{2\hbar} \left( r^2 + \frac{\hbar^2\ell^2}{m^2 \omega^2 r^2} + \frac{p_r^2}{m^2 \omega^2} + \frac{2\hbar\ell}{m\omega} - \frac{i}{m\omega} \left( -i\hbar + \frac{\hbar\ell}{m\omega} \frac{i\hbar}{r^2} \right) \right)\\
		&= \frac{1}{\hbar\omega} \left( \frac{p_r^2}{2m} + \frac{1}{2}m\omega^2 r^2 + \frac{\hbar^2 \ell (\ell + 1)}{2mr^2} \right) + \ell - \frac{1}{2}\\
	\end{split}
\end{equation*}
Risulta quindi che:
\begin{equation}
	\hat{D}_{\ell} = \frac{1}{\hbar\omega} \hat{\mathcal{H}}_{\ell} + \ell - \frac{1}{2}
	\label{eq:3.30}
\end{equation}
Di conseguenza, $ \hat{D}_{\ell} $ e $ \hat{\mathcal{H}}_{\ell} $ hanno gli stessi autostati $ \ket{k,\ell} $:
\begin{equation}
	\hat{D}_{\ell} \ket{k,\ell} = \mathcal{E}_{k,\ell} \ket{k,\ell}, \,\, \mathcal{E}_{k,\ell} = \frac{1}{\hbar \omega} E_{k,\ell} + \ell - \frac{1}{2}
	\label{eq:3.31}
\end{equation}
Ovviamente $ \hat{D}_0 $ è il consueto operatore numero: infatti, il suo spettro è $ \mathcal{E}_{k,0} = 2k + 1 $.\\
È necessario definire un ulteriore operatore:
\begin{equation}
	\hat{\overline{D}}_{\ell} \defeq \hat{d}_{\ell} \hat{d}_{\ell}^{\dagger}
	\label{eq:3.32}
\end{equation}
Con un calcolo analogo al precedente, si trova che:
\begin{equation}
	\hat{\overline{D}}_{\ell} = \frac{1}{\hbar \omega} \hat{\mathcal{H}}_{\ell - 1} + \ell + \frac{1}{2}
	\label{eq:3.33}
\end{equation}
Risulta evidente quindi che:
\begin{equation}
	\hat{\overline{D}}_{\ell + 1} = \hat{D}_{\ell} + 2
	\label{eq:3.34}
\end{equation}
Questi operatori legano gli spettri di Hamiltoniane con $ \ell $ diversi: si supponga di avere $ \ket{k,\ell} $ autostato di $ \hat{\mathcal{H}}_{\ell} $ e $ \hat{D}_{\ell} $: $ \hat{d}_{\ell + 1}^{\dagger} \ket{k,\ell} $ è un autostato di $ \hat{\mathcal{H}}_{\ell + 1} $ e $ \hat{D}_{\ell + 1} $, dato che:
\begin{equation*}
	\hat{D}_{\ell + 1} \hat{d}_{\ell + 1}^{\dagger} \ket{k,\ell} = \hat{d}_{\ell + 1}^{\dagger} \hat{d}_{\ell + 1} \hat{d}_{\ell + 1}^{\dagger} \ket{k,\ell} = \hat{d}_{\ell + 1}^{\dagger} (\hat{D}_{\ell} + 2) \ket{k,\ell} = (\mathcal{E}_{k,\ell} + 2) \hat{d}_{\ell + 1}^{\dagger} \ket{k,\ell}
\end{equation*}
Questi operatori sono simili a degli operatori di scala:
\begin{equation}
	\hat{D}_{\ell + 1} \hat{d}_{\ell + 1} \ket{k,\ell} = (\mathcal{E}_{k,\ell} + 2) \hat{d}_{\ell + 1}^{\dagger} \ket{k,\ell}
	\label{eq:3.35}
\end{equation}
\begin{equation}
	\hat{D}_{\ell - 1} \hat{d}_{\ell} \ket{k,\ell} = (\mathcal{E}_{k,\ell} - 2) \hat{d}_{\ell} \ket{k,\ell}
	\label{eq:3.36}
\end{equation}
In questo modo, è possibile costruire l'intero spettro, ricordando che $ \mathcal{E}_{k,0} = 2k + 1 $:
\begin{equation*}
	\begin{split}
		&\hat{D}_1 \hat{d}_1^{\dagger} \ket{k,0} = (\mathcal{E}_{k,0} + 2) \hat{d}_1^{\dagger} \ket{k,0} = (2k + 3) \hat{d}_1^{\dagger} \ket{k,0}\\
		&\hat{D}_2 \hat{d}_2^{\dagger} \hat{d}_1^{\dagger} \ket{n,0} = (\mathcal{E}_{k,0} + 4) \hat{d}_2^{\dagger} \hat{d}_1^{\dagger} \ket{k,0} = (2k + 5) \hat{d}_2^{\dagger} \hat{d}_1^{\dagger} \ket{k,0}\\
		&\qquad\quad\vdots\\
		&\hat{D}_{\ell + 1} \hat{d}_{\ell + 1}^{\dagger} \dots \hat{d}_1^{\dagger} \ket{k,0} = (2k + 2\ell + 1) \hat{d}_{\ell + 1}^{\dagger} \dots \hat{d}_1^{\dagger} \ket{k,0}
	\end{split}
\end{equation*}
Si evince che $ \mathcal{E}_{k,\ell} = 2k + 2\ell + 1 $, quindi dall'Eq. \ref{eq:3.31} si ricava lo spettro dell'$ \ell $-esima Hamiltoniana:
\begin{equation}
	E_{k,\ell} = \hbar \omega \left( 2k + \ell + \frac{3}{2} \right)
	\label{eq:3.37}
\end{equation}
Questi sono tutti e soli i possibili autovalori di $ \hat{\mathcal{H}}_{\ell} $: se per assurdo esistesse un suo autostato con un autovalore non presente in questa sequenza, ci si potrebbe comunque sempre ricondurre ad uno stato con $ \ell = 0 $ applicando ripetutamente $ \hat{d}_{\ell} $, ottenendo così un nuovo autovalore di $ \hat{\mathcal{H}}_0 $, il che è assurdo poiché il suo spettro completo (Eq. \ref{eq:3.27}) è dato dall'Eq. \ref{eq:3.37}.\\
Ovviamente, in maniera analoga, si può partire da uno stato $ \ket{k,\ell} $ ed ottenere gli stati minori fino a $ \ket{k,0} $ applicando $ \hat{D}_{\ell} $.

\subsection{Degenerazione dello spettro}

In coordinate cartesiane, lo spettro dell'oscillatore armonico isotropo è dato da $ E_n = \hbar \omega \left( n + \frac{3}{2} \right) $, con degenerazione $ d_n = \frac{1}{2}(n + 1)(n + 2) $.\\
Questo stesso spettro si ottiene in coordinate sferiche ponendo in Eq. \ref{eq:3.37} $ n = 2k + \ell $, e si dimostra che la degenerazione è la stessa.

\begin{proposition}
	Ponendo $ n = 2k + \ell $, si ha $ d_n = \frac{1}{2}(n + 1)(n + 2) $.
\end{proposition}
\begin{proof}
	Fissato $ n $ (dunque fissato il sottospazio di energia considerato), si ha che $ \ell $ è vincolato da $ 0 \le \ell \le n $, ma si nota anche che se $ n $ è pari/dispari anche $ \ell $ deve essere pari/dispari. Dunque, se $ n = 2n' $ si ha $ \ell = 2\ell' $, con $ 0 \le \ell' \le n' $, mentre se $ n = 2n' + 1 $ si ha $ \ell = 2\ell' + 1 $, con $ 0 \le \ell' \le n' $. Allora, ricordando che ogni stato $ \ell $ ha una degernazione $ 2\ell + 1 $ a causa di $ m $:
	\begin{equation*}
		d_n^{\text{(pari)}} = \sum_{\ell' = 0}^{n'} \left( 2(2\ell') + 1 \right) = 2 n' (n' + 1) + n' + 1 = n \left( \frac{n}{2} + 1 \right) + \frac{n}{2} + 1 = \frac{1}{2} (n + 1) (n + 2)
	\end{equation*}
	\begin{equation*}
		d_n^{\text{(dispari)}} = \sum_{\ell' = 0}^{n'} \left( 2 (2\ell' + 1) + 1) \right) = 2n' (2n' + 1) + 3 (n' + 1) = \frac{1}{2} (n + 1) (n + 2)
	\end{equation*}
\end{proof}

Dunque, sia che gli stati vengano scritti in coordinate cartesiane come $ \ket{n_1, n_2, n_3} $, sia che vengano scritti in coordinate sferiche come $ \ket{n,\ell,m} $, si ottiene giustamente lo stesso spettro. Naturalmente, è sempre possibile passare da una rappresentazione all'altra tramite una trasformazione unitaria in ciascun sottospazio di energia fissata.

\subsubsection{Teorema di degenerazione}

Questo risultato deriva dal teorema di degenerazione.

\begin{theorem}
	Data una Hamiltoniana $ \hat{\mathcal{H}} $ e due operatori $ \hat{A}, \hat{B} : [\hat{A},\hat{\mathcal{H}}] = [\hat{B},\hat{\mathcal{H}}] = 0 $, allora $ [\hat{A},\hat{B}] = 0 \,\Rightarrow\, \hat{\mathcal{H}} $ ha spettro degenere.
\end{theorem}
\begin{proof}
	Per assurdo sia lo spettro di $ \hat{\mathcal{H}} $ non degenere, ovvero per ogni autovalore $ E_n $ ci sia un solo autostato $ \ket{n} : \hat{\mathcal{H}}\ket{n} = E_n\ket{n} $. Dato che $ [\hat{A},\hat{\mathcal{H}}] = 0 $, essi sono diagonalizzabili simultaneamente, e la stessa cosa vale per $ \hat{B} $: essi hanno la comune base di autostati $ \ket{n} $, dunque commutano, il che contraddice l'ipotesi.
\end{proof}

Si vede dunque che in ogni sottospazio degenere di autostati diverse combinazioni di autostati associati allo stesso autovalore diagonalizzano l'uno o l'altro operatore, ma non entrambi.\\
Questo teorema permette di legare la degenerazione dello spettro alle simmetrie dell'Hamiltoniana: ad esempio, nel caso del momento angolare, l'Hamiltoniana invariante per rotazione commuta con ciascuno dei $ \hat{L}_i $, ma questi non commutano tra loro; di conseguenza, l'Hamiltoniana può avere un termine proporzionale a $ \hat{L}^2 $, ma non hai singoli $ \hat{L}_i $, quindi si può scegliere di diagonalizzare $ \hat{L}^2 $ e uno dei $ \hat{L}_i $, ottenendo un grado di degenerazione come visto in precedenza.\\
Nel caso più generale, si determinano tutti gli operatori che commutano con l'Hamiltoniana e poi tutti gli stati ottenibili l'uno dall'altro tramite trasformazioni generate dai suddetti operatori (esponenziandoli): la degenerazione è il numero di stati contenuti in ciascun insieme di stati di questo tipo, e può essere vista come la dimensione della rappresentazione irriducibile del gruppo di trasformazioni generate dagli operatori considerati.

\subsection{Simmetria dell'oscillatore armonico isotropo}

Per l'oscillatore armonico isotropo si vede che il grado di degenerazione è maggiore di quello dovuto all'invarianza per rotazioni: infatti, per ogni valore di $ n $ ci sono più valori di $ \ell $ corrispondenti allo stesso valore di energia. Ciò significa che ci devono essere altri operatori commutanti con l'Hamiltoniana (ma non fra loro).\\
In coordinate cartesiane, si definiscono i 3 operatori di distruzione (e i 3 di creazione):
\begin{equation}
	\hat{a}_i \defeq \sqrt{\frac{m \omega}{2\hbar}} \left( \hat{x}_i + i \frac{\hat{p}_i}{m\omega} \right)
	\label{eq:3.38}
\end{equation}
In questo modo, è possibile separare l'Hamiltoniana come:
\begin{equation}
	\hat{\mathcal{H}} = \sum_{i = 1}^{3} \hbar \omega \left( \hat{a}_i^{\dagger} \hat{a}_i + \frac{1}{2} \right)
	\label{eq:3.39}
\end{equation}
Si definiscono dunque i 9 operatori:
\begin{equation}
	\hat{\mathcal{O}}_{ij} \defeq \hat{a}_i^{\dagger} \hat{a}_j
	\label{eq:3.40}
\end{equation}
\begin{proposition}
	$ [\hat{\mathcal{O}}_{ij}, \hat{\mathcal{H}}] = 0 \,\land\, [\hat{\mathcal{O}}_{ij}, \hat{\mathcal{O}}_{ab}] \neq 0 $.
\end{proposition}
\begin{proof}
	Ricordando che $ [\hat{a}_i^{\dagger},\hat{a}_j] = -\delta_{ij} $:
	\begin{equation*}
		[ \hat{\mathcal{O}}_{ij}, \hat{\mathcal{H}} ] = \hbar \omega \sum_{k = 1}^{3} \,[\hat{a}_i^{\dagger}\hat{a}_j,\hat{a}_k^{\dagger}\hat{a}_k] = \hbar \omega \sum_{k = 1}^{3} \left( \hat{a}_i^{\dagger} [\hat{a}_j,\hat{a}_k^{\dagger}\hat{a}_k] + [\hat{a}_i^{\dagger},\hat{a}_k^{\dagger}\hat{a}_k]\hat{a}_j \right) = \hbar \omega \left( \hat{a}_i^{\dagger} \hat{a}_j - \hat{a}_i^{\dagger} \hat{a}_j \right) = 0
	\end{equation*}
	\begin{equation*}
		[\hat{\mathcal{O}}_{ij},\hat{\mathcal{O}}_{ab}] = [\hat{a}_i^{\dagger}\hat{a}_j,\hat{a}_a^{\dagger}\hat{a}_b] = \hat{a}_i^{\dagger} [\hat{a}_{j},\hat{a}_a^{\dagger}\hat{a}_b] + [\hat{a}_i^{\dagger},\hat{a}_a^{\dagger},\hat{a}_b]\hat{a}_j = \hat{a}_i^{\dagger} \delta_{ja} \hat{a}_b - \delta_{ib} \hat{a}_a^{\dagger} \hat{a}_j \neq 0
	\end{equation*}
\end{proof}
L'Hamiltoniana può essere scritta come una combinazione lineare di 3 di quesi operatori:
\begin{equation}
	\hat{\mathcal{H}} = \sum_{i = 1}^{3} \hbar \omega \left( \hat{\mathcal{O}}_{ii} + \frac{1}{2} \right)
	\label{eq:3.41}
\end{equation}
Rimangono dunque 8 operatori che commutano con l'Hamiltoniana ma non tra di loro e che, per il teorema di degenerazione, determinano il grado di degenerazione dello spettro. In particolare, 3 di questi sono proprio gli operatori del momento angolare:
\begin{equation}
	\hat{L}_i = - i\hbar \sum_{j,k = 1}^{3} \epsilon_{ijk} \hat{\mathcal{O}}_{jk}
	\label{eq:3.42}
\end{equation}
Ad esempio, con calcolo esplicito:
\begin{equation*}
	\hat{\mathcal{O}}_{12} - \hat{\mathcal{O}}_{21} = \frac{m \omega}{2\hbar} \left[ \left( \hat{x} - i \frac{\hat{p}_x}{m\omega} \right) \left( \hat{y} + i \frac{\hat{p}_y}{n\omega} \right) - \left( \hat{y} - i \frac{\hat{p}_y}{m\omega} \right) \left( \hat{x} + i \frac{\hat{p}_x}{m\omega} \right) \right] = \frac{i}{\hbar} \left( \hat{x} \hat{p}_y - \hat{y} \hat{p}_x \right) = \frac{i}{\hbar} \hat{L}_z
\end{equation*}
Gli operatori $ \hat{\mathcal{O}}_{ij} $ sono i generatori del gruppo $ \SUn{3} $ (si vede dalle relazioni di commutazione): in generale, l'oscillatore armonico $ n $-dimensionale ha simmetria $ \SUn{n} $. Gli operatori $ \hat{L}_i $ generano invece $ \SUn{2} $, che è un sottogruppo di $ \SUn{3} $ e corrisponde all'invarianza per rotazioni (se si restringe allo spin intero si ha solo $ \SOn{3} $).












\part{Metodi di Approssimazione}
\pagestyle{body}

\chapter{Limite Classico}
\selectlanguage{italian}

È possibile mostrare rigorosamente come le leggi della fisica classica emergano da quelle quantistiche. La principale difficoltà è determinata dal fatto che in fisica classica la meccanica consiste nel calcolare la traiettoria di un sistema date le sue condizioni iniziali: in particolare, nella formulazione hamiltoniana la traiettoria è determinata a partire da $ q_0 $ e $ p_0 $, mentre nella formulazione lagrangiana da $ q_0 $ e $ \dot{q}_0 $. Ciò in ambito quantistico non è possibile, a causa del principio d'indeterminazione.\\
È però possibile formulare la fisica classica in maniera compatibile col principio d'indeterminazione tramite il principio d'azione, il quale determina la traiettoria a partire da $ q_0 $ e $ q_1 $ (posizioni iniziale e finale). È inoltre possibile trovare un analogo classico della funzione d'onda nella teoria di Hamilton-Jacobi.

\section{Principio d'azione classico}

\begin{definition}
	Dato un sistema classico descritto da una lagrangiana $ \mathscr{L}(q,\dot{q},t) $ che si muove lungo una traiettoria $ q(t) $, detti $ q_0 \equiv q(t_0) $ e $ q_1 \equiv q(t_1) $, si definisce la sua \textit{azione} lungo tale traiettoria come:
	\begin{equation}
		\mathcal{S}(q_0,t_0 ; q_1,t_1) \defeq \int_{t_0}^{t_1} dt\, \mathscr{L}(q(t), \dot{q}(t), t)
		\label{eq:4.1}
	\end{equation}
\end{definition}

Il \textit{principio di minima azione} (o principio di Hamilton) afferma che, fissati $ (q_0,t_0) $ e $ (q_1,t_1) $, la traiettoria percorsa dal sistema è quella che estremizza l'azione, vista come un funzionale di $ q(t) $:
\begin{equation*}
	\begin{split}
		\delta \mathcal{S}
		&= 0 \\
		&= \int_{t_0}^{t_1} dt\, \delta \mathscr{L} (q(t), \dot{q}(t), t) = \int_{t_0}^{t_1} dt \left( \frac{\pa \mathscr{L}}{\pa q} \delta q + \frac{\pa \mathscr{L}}{\pa \dot{q}} \delta \dot{q} \right) \\
		&= \int_{t_0}^{t_1} dt \left( \frac{\pa \mathscr{L}}{\pa q} \delta q + \frac{\pa \mathscr{L}}{\pa \dot{q}} \frac{d}{dt} \delta q \right) = \int_{t_0}^{t_1} dt \left( \frac{\pa \mathscr{L}}{\pa q} - \frac{d}{dt} \frac{\pa \mathscr{L}}{\pa \dot{q}} \right) \delta q + \left[ \frac{\pa \mathscr{L}}{\pa \dot{q}} \delta q \right]_{t_0}^{t_1}
	\end{split}
\end{equation*}
Essendo gli estremi della traiettoria fissati, si ha $ \delta q(t_0) = \delta q(t_1) = 0 $, dunque dall'arbitrarietà di $ \delta q $ si ottengono le \textit{equazioni di Eulero-Lagrange}:
\begin{equation}
	\frac{\pa \mathscr{L}}{\pa q} - \frac{d}{dt} \frac{\pa \mathscr{L}}{\pa \dot{q}} = 0
	\label{eq:4.2}
\end{equation}
Una volta trovata la traiettoria e fissati $ (q_0,t_0) $, si vede immediatamente che una variazione $ \delta q(t) $ della traiettoria (tale che $ \delta q(t_0) = 0 $) determina una variazione dell'azione data da:
\begin{equation}
	\delta \mathcal{S}(t) = \frac{\pa \mathscr{L}}{\pa \dot{q}} \delta q(t)
	\label{eq:4.3}
\end{equation}
Ricordando che $ p \defeq \frac{\pa \mathscr{L}}{\pa \dot{q}} $, si trova:
\begin{equation}
	p(t) = \frac{\pa \mathcal{S}(q,t)}{\pa q}
	\label{eq:4.4}
\end{equation}

\begin{definition}
	Dato un sistema classico che si muove lungo una traiettoria $ q(t) $, si definisce la \textit{funzione principale di Hamilton} come l'azione valutata lungo la traiettoria, ovvero:
	\begin{equation}
		\mathcal{S}(q,t) = \mathcal{S}(q_0,t_0,q(t),t)
		\label{eq:4.5}
	\end{equation}
\end{definition}

Dato che l'impulso del sistema è il gradiente della funzione principale lungo la traiettoria, è naturale l'associazione di $ S(q,t) $ con la funzione d'onda quantistica.

\subsection{Teoria di Hamilton-Jacobi}

Il collegamento tra la fisica classica e quella quantistica è fornito dalla teoria di Hamilton-Jacobi.

\begin{theorem}[Hamilton-Jacobi]
	Dato un sistema classico unidimensionale descritto da un'hamiltoniana $ \mathcal{H}(q,p,t) $ che si muove lungo una traiettoria $ q(t) $, la funzione principale di Hamilton soddisfa l'\textit{equazione di Hamilton-Jacobi}:
	\begin{equation}
		\frac{\pa \mathcal{S}(q,t)}{\pa t} + \mathcal{H} \left( q, \frac{\pa \mathcal{S}}{\pa q}, t \right) = 0
		\label{eq:4.6}
	\end{equation}
\end{theorem}
\begin{proof}
	Si consideri la derivata totale nel tempo della funzione principale lungo la traiettoria:
	\begin{equation*}
		\begin{split}
			\frac{d \mathcal{S}(q,t)}{dt}
			&= \mathscr{L}(q,\dot{q},t) = p \dot{q} - \mathcal{H}(q,p,t) \\
			&= \frac{\pa \mathcal{S}(q,t)}{\pa t} + \frac{\pa \mathcal{S}(q,t)}{\pa q} \dot{q} = \frac{\pa \mathcal{S}(q,t)}{\pa t} + p \dot{q}
		\end{split}
		\quad \Rightarrow \quad
		\frac{\pa \mathcal{S}(q,t)}{\pa t} = - \mathcal{H} \left( q, \frac{\pa \mathcal{S}}{\pa q}, t \right)
	\end{equation*}
\end{proof}

\begin{example}
	Si consideri un oscillatore armonico unidimensionale, descritto dall'hamiltoniana $ \mathcal{H}(q,p) = \frac{p^2}{2m} + \frac{1}{2} m \omega^2 q^2 $; l'equazione di Hamilton-Jacobi in questo caso è:
	\begin{equation*}
		\frac{\pa \mathcal{S}}{\pa t} + \frac{1}{2m} \left( \frac{\pa \mathcal{S}}{\pa q} \right)^2 + \frac{1}{2} m \omega^2 q^2 = 0
	\end{equation*}
	Essendo il potenziale indipendente dal tempo, il sistema è invariante per traslazioni temporali, dunque lungo la traiettoria l'energia si conserva: fissata la traiettoria $ q(t) $, si ha $ \mathcal{H}(q,p) = E $. Di conseguenza:
	\begin{equation*}
		\frac{\pa \mathcal{S}}{\pa t} = -E
		\quad \Rightarrow \quad
		\mathcal{S}(q,t) = -E t + W(q)
	\end{equation*}
	dove $ W(q) $ è detta \textit{funzione caratteristica di Hamilton}. Per determinare quest'ultima, si risolve $ \mathcal{H}(q,p) = E $, che ora è una ODE:
	\begin{equation*}
		\frac{1}{2m} \left( \frac{d W(q)}{d q} \right)^2 + \frac{1}{2} m \omega^2 q^2 = E
		\quad \Rightarrow \quad
		W(q) = \pm \sqrt{2mE} \int_{q_0}^q d\xi \sqrt{1 - \frac{m \omega^2}{2E} \xi^2}
	\end{equation*}
\end{example}

Questo procedimento ha carattere generale.

\begin{proposition}
	Dato un sistema classico unidimensionale descritto da un'hamiltoniana indipendente dal tempo $ \mathcal{H}(q,p) = \frac{p^2}{2m} + V(q) $ che si muove lungo una traiettoria $ q(t) $ di energia $ E $, la funzione principale di Hamilton è data da $ \mathcal{S}(q,t) = -E t + W(q) $, dove la funzione caratteristica di Hamilton è determinata come:
	\begin{equation}
		W(q) = \pm \sqrt{2m} \int_{q_0}^q d\xi \sqrt{E - V(\xi)}
		\label{eq:4.7}
	\end{equation}
\end{proposition}

Una volta determinata la funzione principale sono noti anche i momenti canonici del sistema, dunque la traiettoria.\\
Il caso multidimensionale ($ q \equiv (q_1, \dots, q_f) $) è più complicato da trattare; in generale, l'equazione da risolvere per la funzione caratteristica avrà la forma:
\begin{equation}
	(\nabla W)^2 = 2m \left( E - V(q) \right)
	\label{eq:4.8}
\end{equation}
Dato che $ p \equiv (p_1, \dots, p_f) = \nabla W(q) $ e che la traiettoria è determinata da $ v_i = \frac{1}{m} p_i $, si può vedere la traiettoria percorsa dal sistema come quella determinata da un'onda che sospinge un oggetto, poiché essa segue il cammino di minima pendenza e la sua velocità è data dalla pendenza stessa.\\
Interpretando la velocità del sistema come la velocità di gruppo, si può vedere che essa è diversa dalla velocità di fase, ovvero quella a cui si muovono i fronti d'onda, ovvero le superfici con $ \mathcal{S}(q,t) $ costante. Assumendo che il sistema (unidimensionale) sia invariante per traslazioni temporali, dato che $ \mathcal{S}(q,t) = -E t + W(q) $, il fronte d'onda $ q_0(t) $ è determinato da:
\begin{equation*}
	\frac{d \mathcal{S}(q_0(t),t)}{dt} = 0
	\quad \Rightarrow \quad
	\frac{dW(q)}{dq} \bigg\vert_{q_0} \dot{q}_0 - E = 0
\end{equation*}
Essendo la velocità di fase $ v_f \equiv \dot{q}_0 $, si trova:
\begin{equation}
	v_f(t) = \pm \frac{E}{\sqrt{2m \left[ E - V(q_0(t)) \right]}}
	\label{eq:4.9}
\end{equation}
La velocità di gruppo è invece data dall'Eq. \ref{eq:4.4} come $ v_g = \frac{1}{m} \frac{\pa W}{\pa q} $:
\begin{equation}
	v_g(t) = \pm \sqrt{\frac{2}{m} \left[ E - V(q_0(t)) \right]}
	\label{eq:4.10}
\end{equation}

\section{Principio d'azione quantistico}

In ambito quantistico, parlando di traiettoria si intende solo fissare (ovvero misurare) $ q_0 \equiv q(t_0) $ e $ q_1 \equiv q(t_1) $. Si ricordi l'operatore di evoluzione temporale:
\begin{equation}
	\ket{\psi(t)} = \hat{S}(t,t_0) \ket{\psi(t_0)}
	\label{eq:4.11}
\end{equation}
La funzione d'onda in un generico punto $ q(t) $ è dunque:
\begin{equation}
	\psi(q,t) = \braket{q | \hat{S}(t,t_0) | \psi(t_0)}
	\label{eq:4.12}
\end{equation}

\begin{definition}
	Dato un sistema quantistico con operatore di evoluzione temporale $ \hat{S}(t,t_0) $, si definisce il suo \textit{propagatore} come l'autofunzione della posizione evoluta nel tempo:
	\begin{equation}
		K(q,t ; q_0,t_0) \defeq \braket{q | \hat{S}(t,t_0) | q_0}
		\label{eq:4.13}
	\end{equation}
\end{definition}

Il propagatore non è altro che l'elemento di matrice dell'operatore di evoluzione temporale tra autostati della posizione. Si noti che non necessariamente $ t > t_0 $, è ammesso anche $ t < t_0 $: l'evoluzione temporale quantistica è deterministica ed unitaria, dunque reversibile.

\begin{proposition}
	Dato un sistema quantistico con propagatore $ K(q,t ; q_0,t_0) $, si ha:
	\begin{equation}
		\psi(q,t) = \int dq'\, K(q,t ; q',t_0) \psi(q',t_0)
		\label{eq:4.14}
	\end{equation}
\end{proposition}
\begin{proof}
	Essendo $ \int dq \ket{q}\bra{q} = \tens{I} $, si vede banalmente dall'Eq. \ref{eq:4.12}.
\end{proof}

Dato che le autofunzioni della posizione sono $ \braket{q | q_0} = \delta(q - q_0) $, si ha la conferma che $ K(q,t ; q_0,t_0) $ è proprio la funzione d'onda dell'evoluto temporale dello stato iniziale $ \ket{q_0} $: da qui l'analogia con la funzione principale di Hamilton. Il propagatore contiene tutta la dinamica del sistema.

\begin{proposition}\label{prop-ass-conv}
	Il propagatore è associativo sotto convoluzione:
	\begin{equation}
		K(q,t ; q_0,t_0) = \int dq_1\, K(q,t ; q_1,t_1) K(q_1,t_1 ; q_0,t_0)
		\label{eq:4.15}
	\end{equation}
\end{proposition}
\begin{proof}
	Ricordando l'associatività dell'evoluzione temporale $ \hat{S}(t,t_0) = \hat{S}(t,t_1) \hat{S}(t_1,t_0) $:
	\begin{equation*}
		K(q,t ; q_0,t_0) = \braket{q | \hat{S}(t,t_0) | q_0} = \int dq_1 \braket{q | \hat{S}(t,t_1) | q_1} \braket{q_1 | \hat{S}(t_1,t_0) | q_0}
	\end{equation*}
\end{proof}

\begin{proposition}
	Dato un sistema quantistico descritto da un'hamiltoniana indipendente dal tempo $ \hat{\mathcal{H}} = \frac{\hat{p}^2}{2m} + \hat{V}(\hat{q}) $, considerando una traslazione temporale infinitesima $ dt \equiv \varepsilon $ si ha il propagatore:
	\begin{equation}
		K(q',t + \varepsilon ; q,t) = \sqrt{\frac{m}{2\pi i \varepsilon \hbar}} e^{i \frac{d\mathcal{S}(t)}{\hbar}}
		\label{eq:4.16}
	\end{equation}
	dove $ d\mathcal{S}(t) $ è l'elemento infinitesimo d'azione lungo l'evoluzione temporale del sistema.
\end{proposition}
\begin{proof}
	Per un'Hamiltoniana indipendente dal tempo l'operatore di evoluzione temporale è:
	\begin{equation*}
		\hat{S}(t,t_0) = e^{\frac{1}{i\hbar} (t - t_0) \hat{\mathcal{H}}}
	\end{equation*}
	Per calcolare esplicitamente il propagatore si sfrutta la risoluzione dell'identità sugli impulsi, il che permette di sostituire gli operatori con i rispettivi autovalori (si può mostrare formalmente):
	\begin{equation*}
		\begin{split}
			K(q',t+\varepsilon ; q,t)
			&= \braket{q' | \hat{S}(t+\varepsilon,t) | q} = \braket{q' | e^{\frac{\varepsilon}{i\hbar} \hat{\mathcal{H}}} | q} = \braket{q' | \left[ 1 + \frac{\varepsilon}{i\hbar} \left( \frac{\hat{p}^2}{2m} + \hat{V}(\hat{q}) \right) + o(\varepsilon) \right] | q} \\
			&= \int dp \braket{q' | p} \braket{p | \left[ 1 + \frac{\varepsilon}{i\hbar} \left( \frac{\hat{p}^2}{2m} + \hat{V}(\hat{q}) \right) + o(\varepsilon) \right] q} \\
			&= \int dp \braket{q' | p} \left[ 1 + \frac{\varepsilon}{i\hbar} \left( \frac{p^2}{2m} + V(q) \right) + o(\varepsilon) \right] \braket{p | q} \\
			&= \int dp\, \frac{1}{\sqrt{2\pi\hbar}} e^{\frac{i}{\hbar} p q'} e^{- \frac{i}{\hbar} ( \frac{p^2}{2m} + V(q) ) \varepsilon} \frac{1}{\sqrt{2\pi\hbar}} e^{ - \frac{i}{\hbar} pq} = \int \frac{dp}{2\pi\hbar}\, e^{\frac{i}{\hbar} p (q' - q)} e^{- \frac{i}{\hbar} ( \frac{p^2}{2m} + V(q) ) \varepsilon}
		\end{split}
	\end{equation*}
	Questo integrale non contiene operatori ed è riconducibile ad un integrale gaussiano notando che $ t \mapsto t + \varepsilon \,\Rightarrow\, q \mapsto q + \varepsilon \dot{q} \equiv q' $:
	\begin{equation*}
		\begin{split}
			K(q',t+\varepsilon ; q,t)
			&= \int \frac{dp}{2\pi\hbar}\, e^{\frac{i}{\hbar} ( p\dot{q} - \frac{p^2}{2m} - V(q) ) \varepsilon} = e^{- \frac{i}{\hbar} V(q) \varepsilon} \int \frac{dp}{2\pi\hbar}\, e^{- \frac{i}{\hbar} \frac{1}{2m} (p^2 - 2m\hbar p\dot{q}) \varepsilon} \\
			&= \frac{1}{2\pi\hbar} e^{-\frac{i}{\hbar} V(q) \varepsilon} \int dp\, e^{-\frac{i \varepsilon}{2m\hbar} (p - m\dot{q})^2} e^{\frac{i m \varepsilon}{2\hbar} \dot{q}^2} \\
			&= \frac{1}{2\pi\hbar} e^{-\frac{i}{\hbar} (\frac{1}{2}m \dot{q}^2 - V(q)) \varepsilon} \sqrt{\frac{2m\hbar}{i\varepsilon}} \int d\lambda\, e^{-\lambda^2} = \sqrt{\frac{m}{2\pi i \varepsilon \hbar}} e^{\frac{i\varepsilon}{\hbar} (\frac{1}{2} m \dot{q}^2 - V(q))}
		\end{split}
	\end{equation*}
	La dimostrazione è completa notando che, fissata l'evoluzione temporale (la $ \virgolette{traiettoria} $):
	\begin{equation*}
		\varepsilon \left( \frac{1}{2} m \dot{q}^2 - V(q) \right) = dt\,\mathscr{L}(q,\dot{q}) = d\mathcal{S}(q,t) \equiv d\mathcal{S}(t)
	\end{equation*}
\end{proof}

Il propagatore è quindi dato da una fase pari alla variazione d'azione in unità di $ \hbar $. Il fattore di normalizzazione è fissato dal fatto che:
\begin{equation}
	\lim_{\varepsilon \rightarrow 0} K(q',t+\varepsilon ; q,t) = \braket{q',t | q,t} = \delta(q' - q)
	\label{eq:4.17}
\end{equation}
Infatti:
\begin{equation*}
	\lim_{\varepsilon \rightarrow 0} \sqrt{\frac{m}{2\pi i \varepsilon \hbar}} e^{\frac{i}{\hbar} \varepsilon ( \frac{1}{2} m \dot{q}^2 - V(q))} = \lim_{\varepsilon \rightarrow 0} \sqrt{\frac{m}{2\pi i \varepsilon \hbar}} e^{-\frac{m}{2i\hbar} \frac{(q' - q)^2}{\varepsilon}} = \delta(q' - q)
\end{equation*}
che è proprio la rappresentazione della delta di Dirac come limite di gaussiane.

\subsection{Path integral}

Dalla Prop. \ref{prop-ass-conv} si sviluppa l'idea di Feynman che porta ad una riformulazione della fisica quantistica: si può pensare l'evoluzione temporale di un sistema come una successione di evoluzioni temporali infinitesime. Dato un sistema quantistico con propagatore $ K(q,t ; q_0,t_0) $ e definito $ t_k \defeq t_0 + k \varepsilon $, con $ t = t_0 + \Delta t = t_0 + n \varepsilon $, si ha:
\begin{equation*}
	\begin{split}
		K(q,t ; q_0,t_0)
		&= \int dq_1 \dots dq_{n-1}\, K(q,t ; q_{n-1},t_{n-1}) \dots K(q_1,t_1 ; q_0,t_0) \\
		&= \int dq_1 \dots dq_{n-1} \left( \frac{m}{2\pi i \varepsilon \hbar} \right)^n e^{\frac{i}{\hbar} ( d\mathcal{S}(t_{n-1}) + \dots + d\mathcal{S}(t_0) )}
	\end{split}
\end{equation*}
Questo è un modo compatto di esprimere il principio quantistico di sovrapposizione, che detta come si compongono le ampiezze di transizione (il cui modulo quadro dà la probabilità di transizione): l'ampiezza di transizione da un certo stato iniziale ad un certo stato finale si calcola considerando tutti i possibili stati intermedi e sommando su tutte le traiettorie (i $ \virgolette{cammini} $ di Feynman). A differenza di un sistema classico, che percorre solo un insieme discreto di traiettorie (quelle che estremizzano l'azione), è come se il sistema quantistico percorresse tutte le traiettorie possibili.\\
Prendendo il limite per $ n \rightarrow \infty $ si ottiene un integrale funzionale, o integrale di Kac, il quale associa un numero ad un funzionale (in questo caso l'azione): dato che quest'ultimo è una mappa da uno spazio di funzioni a $ \R $ (o anche $ \C $), la misura d'integrazione di un integrale di Kac è il differenziale di una funzione.

\begin{proposition}
	Dato un sistema quantistico con azione $ \mathcal{S}[q(t)] $ e definito l'insieme di possibili traiettorie da $ q_0 $ a $ q $ come $ \mathscr{P} \defeq \{q(t) : q(t_0) = q_0 \land q(t) = q\} $ , il propagatore $ K(q,t ; q_0,t_0) $ si può esprimere come un integrale di Kac, detto \textit{path integral}:
	\begin{equation}
		K(q,t ; q_0,t_0) = \int_{\mathscr{P}} \mathcal{D}q(t)\, e^{\frac{i}{\hbar} \mathcal{S}[q(t)]}
		\label{eq:4.18}
	\end{equation}
	dove la misura $ \mathcal{D}q(t) $ è definita in modo da rispettare la condizione di normalizzazione Eq. \ref{eq:4.17}.
\end{proposition}

Quella del path integral è una formulazione alternativa della fisica quantistica che, con l'aggiunta della regola di Born (probabilità è modulo quadro della funzione d'onda), è completamente analoga a quella di Schrödinger: infatti, sia il principio di sovrapposizione che quello di ortonormalità degli stati fisici sono naturalmente soddisfatti dal path integral. Per quanto riguarda l'evoluzione temporale, si può mostrare che il path integral soddisfa l'equazione di Schrödinger.

\begin{proposition}
	Una funzione d'onda del tipo in Eq. \ref{eq:4.14}, con propagatore dato dall'Eq. \ref{eq:4.18}, soddisfa l'equazione di Schrödinger.
\end{proposition}
\begin{proof}
	L'evoluzione temporale della funzione d'onda è data dal path integral:
	\begin{equation*}
		\psi(q,t) = \int dq_0 \int_{\mathscr{P}} \mathcal{D}q(t)\, e^{\frac{i}{\hbar}\mathcal{S}[q(t)]} \psi(q_0,t_0)
	\end{equation*}
	Si consideri un'evoluzione temporale infinitesima, con $ t = t_0 + \varepsilon $ e $ q = q_0 + \delta $ (Eq. \ref{eq:4.16}):
	\begin{equation*}
		\begin{split}
			\psi(q,t)
			&= \int d\delta\, \sqrt{\frac{m}{2\pi i \varepsilon \hbar}} e^{\frac{i}{\hbar} (\frac{1}{2} m \frac{\delta^2}{\varepsilon^2} - V(q)) \varepsilon} \psi(q_0,t_0) = e^{-\frac{i}{\hbar} V(q) \varepsilon} \sqrt{\frac{m}{2\pi i \varepsilon \hbar}} \int d\delta\, e^{\frac{i}{\hbar} \frac{1}{2} m \frac{\delta^2}{\varepsilon}} \psi(q_0,t_0) \\
			&= e^{-\frac{i}{\hbar} V(q)\varepsilon} \sqrt{\frac{m}{2\pi i \varepsilon \hbar}} \int d\delta\, e^{\frac{i}{\hbar} \frac{1}{2} m \frac{\delta^2}{\varepsilon}} \left[ \psi(q,t_0) - \delta \frac{\pa}{\pa q} \psi(q,t_0) + \frac{1}{2} \delta^2 \frac{\pa^2}{\pa q^2} \psi(q,t_0) + o(\delta^2) \right]
		\end{split}
	\end{equation*}
	Si noti ora che l'integrale ha misura $ d\delta $, dunque il primo termine risulta in un integrale gaussiano, il secondo si annulla per parità dell'integranda (dispari su intervallo simmetrico) ed il termo è calcolato usando $ \int_{\R} dx\, x^2 \exp (-\alpha x^2) = \frac{1}{2\alpha} \sqrt{\frac{\pi}{\alpha}} $:
	\begin{equation*}
		\begin{split}
			\psi(q,t)
			&= e^{-\frac{i}{\hbar} V(q) \varepsilon} \sqrt{\frac{m}{2\pi i \varepsilon \hbar}} \left[ \sqrt{-\frac{2\pi \hbar \varepsilon}{i m}} \psi(q,t_0) + \frac{1}{2} \left( - \frac{1}{2} \frac{2\hbar \varepsilon}{im} \right) \sqrt{- \frac{2\pi\hbar \varepsilon}{im}} \frac{\pa^2}{\pa q^2} \psi(q,t_0) \right] \\
			&= e^{-\frac{i}{\hbar} V(q) \varepsilon} \left[ \psi(q,t_0) + \frac{i \varepsilon \hbar}{2m} \frac{\pa^2}{\pa q^2} \psi(q,t_0) \right] = \left[ 1 - \frac{i}{\hbar} V(q) \varepsilon + o(\varepsilon) \right] \left[ \psi(q,t_0) + \frac{i \varepsilon \hbar}{2m} \frac{\pa^2}{\pa q^2} \psi(q,t_0) \right] \\
			&= \left[ 1 - \frac{i}{\hbar} V(q) \varepsilon + o(\varepsilon) \right] \left[ \psi(q,t) - \varepsilon \frac{\pa}{\pa t} \psi(q,t) + \frac{i \varepsilon \hbar}{2m} \frac{\pa^2}{\pa q^2} \psi(q,t) + o(\varepsilon) \right] \\
			&= \psi(q,t) + \varepsilon \left[ - \frac{i}{\hbar} V(q) \psi(q,t) - \frac{\pa}{\pa t} \psi(q,t) + \frac{i\hbar}{2m} \frac{\pa^2}{\pa q^2} \psi(q,t) \right] + o(\varepsilon)
		\end{split}
	\end{equation*}
	Perché valga l'equazione, si deve avere proprio l'equazione di Schrödinger:
	\begin{equation*}
		i\hbar \frac{\pa}{\pa t} \psi(q,t) = -\frac{\hbar^2}{2m} \frac{\pa^2}{\pa q^2} \psi(q,t) + V(q) \psi(q,t)
	\end{equation*}
\end{proof}

È particolarmente agevole trovare il limite semiclassico per $ \hbar \rightarrow 0 $. Si noti che per un tempo complesso $ t \mapsto it $ si ha $ \mathcal{S} \mapsto i\mathcal{S} $, quindi in tal caso il peso di ciasun cammino è $ \exp\left( - \frac{1}{\hbar} \mathcal{S}[q(t)] \right) $: la probabilità è esponenzialmente smorzata dall'azione, dunque l'unico cammino ad avere effettiva rilevanza fisica è quello di minima azione, ovvero la traiettoria classica. Per continuazione analitica si estende il risultato ai tempi reali: si ritrova il principio di corrispondenza.\\
Anche senza invocare la continuazione analitica, si vede che la dipendenza del propagatore dall'azione è data da una fase: per $ \mathcal{S} $ grande rispetto ad $ \hbar $, si integra su una fase che oscilla in maniera estremamente veloce, risultando quindi in media nulla (questo è il fenomeno della decoerenza); d'altro canto, se $ \mathcal{S} $ è comparabile ad $ \hbar $, molti cammini contribuiscono al path integral e si hanno effetti di interferenza quantistica. Si vede dunque che, in generale, la fisica quantistica descrive sistemi con un basso numero di gradi di libertà: infatti, gli effetti quantistici si manifestano quando il valore numerico dell'azione, associato al volume occupato dal sistema nello spazio delle fasi e quindi al numero dei suoi gradi di libertà, è piccolo in unità di $ \hbar $.

\section{WKB approximation}

Dalla formulazione del path integral si ha che nel limite semiclassico la funzione d'onda è determinata da un solo cammino, ed inoltre essa è esprimibile come una pura fase $ \psi \sim \exp (-\frac{i}{\hbar} \mathcal{S}) $, dove l'azione è valutata lungo tale cammino. Ciò suggerisce di scrivere la funzione d'onda nel limite semiclassico come:
\begin{equation}
	\psi(\ve{x},t) = e^{\frac{i}{\hbar} \Theta(\ve{x},t)}
	\label{eq:4.19}
\end{equation}
dove $ \Theta(\ve{x},t) $ è un'opportuna funzione esprimibile come serie di potenze di $ \hbar $. Questa è nota come \textit{approssimazione WKB} (Wentzel, Kramers, Brillouin).

\begin{proposition}
	Per un sistema quantistico descritto dall'hamiltoniana $ \hat{\mathcal{H}} = \frac{\hat{p}^2}{2m} + \hat{V}(\hat{\ve{x}},t) $, la funzione $ \Theta(\hat{x},t) $ deve soddisfarre:
	\begin{equation}
		-\frac{\pa}{\pa t} \Theta(\ve{x},t) = \frac{1}{2m} \left( \nabla \Theta(\ve{x},t) \right)^2 + V(\ve{x},t) - \frac{i\hbar}{2m} \lap \Theta(\ve{x},t)
		\label{eq:4.20}
	\end{equation}
\end{proposition}
\begin{proof}
	Inserendo l'Ansatz Eq. \ref{eq:4.19} nell'equazione di Schrödinger:
	\begin{equation*}
		i\hbar e^{\frac{i}{\hbar}\Theta(\ve{x},t)} \frac{i}{\hbar} \frac{\pa}{\pa t} \Theta(\ve{x},t) = - \frac{\hbar^2}{2m} \nabla \left[ e^{\frac{i}{\hbar}\Theta(\ve{x},t)} \frac{i}{\hbar} \nabla \Theta(\ve{x},t) \right] + V(\ve{x},t) e^{\frac{i}{\hbar} \Theta(\ve{x},t)}
	\end{equation*}
	Svolgendo il gradiente e semplificando $ e^{\frac{i}{\hbar}\Theta(\ve{x},t)} $ si ottiene la tesi.
\end{proof}

L'approssimazione semiclassica di $ \Theta(\ve{x},t) $ si ottiene espandendo in serie di potenze di $ \hbar $:
\begin{equation}
	\Theta(\ve{x},t) = \mathcal{S}(\ve{x}.t) + \sum_{n = 1}^{\infty} \left( \frac{\hbar}{i} \right)^n S_n(\ve{x},t)
	\label{eq:4.21}
\end{equation}
Il termine di ordine zero è fissato dal principio di corrispondenza. Infatti, per ordine zero l'Eq. \ref{eq:4.20} diventa:
\begin{equation}
	- \frac{\pa}{\pa t} \mathcal{S}(\ve{x},t) = \frac{1}{2m} (\nabla \mathcal{S}(\ve{x},t))^2 + V(\ve{x},t)
	\label{eq:4.22}
\end{equation}
il che conferma l'identificazione di $ \mathcal{S}(\ve{x},t) $ con la funzione principale di Hamilton.

\subsection{Sistemi invarianti per traslazioni temporali}

Per potenziali invarianti per traslazioni temporali è noto dalla teoria classica di Hamilton-Jacobi che per un sistema di energia $ E $:
\begin{equation}
	\mathcal{S}(\ve{x},t) = -E t + \sigma_0(\ve{x})
	\label{eq:4.23}
\end{equation}
dove $ \sigma_0(\ve{x}) $ è la funzione caratteristica di Hamilton. Questa soddisfa l'equazione:
\begin{equation}
	E = \frac{1}{2m} \left( \nabla \sigma_0(\ve{x}) \right) + V(\ve{x})
	\quad \Rightarrow \quad
	\nabla \sigma_0(\ve{x}) = \pm \sqrt{2m \left[ E - V(\ve{x}) \right]}
	\label{eq:4.24}
\end{equation}
Nel caso unidimensionale la soluzione è data dall'Eq. \ref{eq:4.7}. Per quanto riguarda gli ordini superiori, si ricordi che, per un'hamiltoniana indipendente dal tempo, gli autostati sono stati stazionari:
\begin{equation}
	\psi(\ve{x},t) = e^{-\frac{i}{\hbar} E t} \psi(\ve{x})
	\label{eq:4.25}
\end{equation}
Ne segue immediatamente che la funzione $ \Theta(\ve{x},t) $ può essere espansa come:
\begin{equation}
	\Theta(\ve{x},t) = -E t + \sigma_0(\ve{x}) + \sum_{n = 1}^{\infty} \left( \frac{\hbar}{i} \right)^n \sigma_n(\ve{x})
	\label{eq:4.26}
\end{equation}
Tutti i contributi di ordine superiore sono indipendenti dal tempo. Al prim'ordine, l'Eq. \ref{eq:4.20} diventa:
\begin{equation}
	2 \nabla \sigma_0(\ve{x}) \cdot \nabla \sigma_1(\ve{x}) + \lap \sigma_0(\ve{x}) = \ve{0}
	\label{eq:4.27}
\end{equation}
Al second'ordine, invece:
\begin{equation}
	2 \nabla \sigma_0(\ve{x}) \cdot \sigma_2(\ve{x}) + \left( \nabla \sigma_1(\ve{x}) \right)^2 + \lap \sigma_1(\ve{x}) = \ve{0}
	\label{eq:4.28}
\end{equation}
In generale, $ \sigma_k $ può essere determinato a partire da ogni $ \sigma_j $ con $ j < k $, ai quali è legato da una equazione differenziale del prim'ordine.

\subsubsection{Sistema unidimensionale}

Nel caso unidimensionale indipendente dal tempo è possibile determinare esplicitamente la soluzione. Definendo l'\textit{impulso semiclassico} $ p(x) \defeq \frac{d}{dx} \sigma_0(x) $ (vedere Eq. \ref{eq:4.4}), l'Eq. \ref{eq:4.27} ha come soluzione:
\begin{equation*}
	\sigma_1(x) = -\frac{1}{2} \ln \abs{p(x)} + c_1
\end{equation*}
La soluzione semiclassica al prim'ordine può quindi essere scritta come:
\begin{equation*}
	\begin{split}
		\psi(x,t)
		&= \exp \left[ \frac{i}{\hbar} \left( -Et \pm \int_{x_0}^x d\xi\, p(\xi) - \frac{1}{2} \frac{\hbar}{i} \ln \abs{p(x)} + c_1 \right) \right] \\
		&= \frac{\mathcal{N}}{\sqrt{\abs{p(x)}}} e^{\frac{1}{i\hbar} Et} \exp \left[ \pm \int_{x_0}^x d\xi\, p(\xi) \right]
	\end{split}
\end{equation*}
dove $ \mathcal{N} $ è un fattore di normalizzazione. Si vede che ci sono due soluzioni linearmente indipendenti, corrispondenti a due stati di impulso semiclassico definito $ \pm p(x) $. La soluzione generale è quindi una sovrapposizione:
\begin{equation}
	\psi(x,t) = \frac{1}{\sqrt{\abs{p(x)}}} e^{\frac{1}{i\hbar} Et} \left[ \mathcal{A} \exp \left( \frac{i}{\hbar} \int_{x_0}^x d\xi\, p(\xi) \right) + \mathcal{B} \exp \left( - \frac{i}{\hbar} \int_{x_0}^x d\xi\, p(x) \right) \right]
	\label{eq:4.29}
\end{equation}
Questa viene di solito chiamata \textit{soluzione approssimata WKB}.
Tale soluzione ha un andamento diverso a seconda del valore dell'energia rispetto al potenziale. Infatti, dall'Eq. \ref{eq:4.24} si ha che:
\begin{equation*}
	p(x) = \pm \sqrt{2m \left[ E - V(x) \right]}
\end{equation*}
Di conseguenza, se $ E > V(x) $ l'impulso semiclassico è reale, dunque la soluzione è di tipo sinusoidale:
\begin{equation*}
	\psi(x,t) = \frac{\mathcal{N}}{\sqrt{\abs{p(x)}}} e^{\frac{i}{\hbar} Et} \sin \left( \frac{1}{\hbar} \int_{x_0}^x d\xi\, \sqrt{2m \left[ E - V(\xi) \right]} + \delta \right)
\end{equation*}
D'altro canto, se $ E < V(x) $ l'impulso semiclassico è immaginario, dunque la soluzione è data da due esponenziali.

\paragraph{Validità dell'approssimazione}

Dall'Eq. \ref{eq:4.20} è chiaro che le correzioni quantistiche all'equazione del moto classica sono date dal termine proporzionale ad $ \hbar $. L'approssimazione semiclassica richiede dunque che quest'ultimo sia piccolo rispetto al termine cinetico:
\begin{equation}
	\abs{\hbar \lap \Theta(\ve{x},t)} \ll \abs{(\nabla \Theta(\ve{x},t))^2}
	\label{eq:4.30}
\end{equation}
Nel caso unidimensionale, ciò si riduce, ricordando la lunghezza d'onda di de Broglie ($ \lambda = \frac{2\pi\hbar}{p} $):
\begin{equation*}
	\abs{\hbar \frac{dp}{dx}} \ll \abs{p^2}
	\quad \Rightarrow \quad
	\abs{\frac{d(\lambda / 2\pi)}{dx}} \ll 1
\end{equation*}
L'approssimazione è dunque valida quando la scala di variazione della lunghezza d'onda delle oscillazioni di $ \psi(x,t) $ è piccola rispetto alla lunghezza d'onda stessa in unità di $ 2\pi $, ovvero quando la lunghezza d'onda varia poco nel corso di un periodo d'oscillazione.\\
Dalla forma esplicita di $ p(x) $ si ricava una condizione sull'energia:
\begin{equation*}
	\frac{\hbar}{\sqrt{2m \left[ E - V(x) \right]}} \ll \frac{2 \abs{E - V(x)}}{V'(x)}
\end{equation*}
L'approssimazione è buona quando l'energia è molto maggiore o molto minore del potenziale nel punto, ovvero per stati molto legati o molto eccitati, mentre fallisce in prossimità dei punti d'inversione, quando $ E \approx V(x) $.

\begin{example}
	Si consideri una generica buca di potenziale unidimensionale (es: buca infinita di potenziale, oscillatore armonico, etc.): per un dato valore dell'energia $ E > V_{\text{min}} $ sono presenti due punti d'inversione, detti $ a $ e $ b $, soluzioni dell'equazione $ E - V(x) = 0 $. L'approssimazione semiclassica è valida nelle tre regioni che escludono gli intorni di $ a $ e $ b $: in questi ultimi, la soluzione può essere trovata considerando il potenziale lineare, dato che:
	\begin{equation*}
		V(x) = V(a) + V'(a) (x - a) + o(x - a)
	\end{equation*}
	ed idem con $ b $. La soluzione del problema del potenziale lineare è nota ed esattamente esprimibile in termini di funzioni di Airy: infatti, sulla base degli impulsi l'equazione di Schrödinger diventa:
	\begin{equation*}
		\left[ \frac{p^2}{2m} + i \lambda \frac{d}{dp} \right] \psi(p) = E \psi(p)
	\end{equation*}
	La risoluzione è banale e dalla conseguente trasformata di Fourier (per passare alla base delle coordinate) si ottengono appunto le funzioni di Airy.\\
	Con l'approssimazione semiclassica, invece, nelle regioni esterne della buca ($ x \ll b $ e $ x \gg a $, assumendo WLOG $ a > b $) si hanno soluzioni di tipo esponenziale, mentre in quella interna ($ b \ll x \ll a $) la soluzione è di tipo sinusoidale. Le relazioni tra le normalizzazioni e la fase della sinusoide sono date dalle condizioni di raccordo (continuità e differenziabilità); per un potenziale prima decrescente e poi crescente si trovano:
	\begin{equation*}
		\psi_b(x) =
		\begin{cases}
			\frac{\mathcal{N}}{\sqrt{\beta(x)}} \exp \left( - \int_x^b d\xi\, \frac{\beta(\xi)}{\hbar} \right) & x \ll b \\
			\frac{2\mathcal{N}}{\sqrt{p(x)}} \sin \left( \int_b^x d\xi\, \frac{p(\xi)}{\hbar} + \frac{\pi}{4} \right) & x \gg b
		\end{cases}
	\end{equation*}
	\begin{equation*}
		\psi_a(x) =
		\begin{cases}
			\frac{2\mathcal{N}'}{\sqrt{p(x)}} \sin \left( \int_x^a d\xi\, \frac{p(\xi)}{\hbar} + \frac{\pi}{4} \right) & x \ll a \\
			\frac{\mathcal{N}'}{\sqrt{\beta(x)}} \exp \left( \int_x^a d\xi\, \frac{\beta(\xi)}{\hbar} \right) & x \gg a
		\end{cases}
	\end{equation*}
	dove $ p(x) = i \beta(x) $ nelle regioni in cui $ E < V(x) $. Le soluzioni per $ x \gg b $ ed $ x \ll a $ devono essere uguali, poiché sono nella stessa regione, dunque si ha un'equazione del tipo $ \sin \alpha = \sin \beta $: questa ha come soluzioni $ \alpha = \beta $ o $ \alpha = \pi - \beta $, ma la prima non è possibile da soddisfare per ogni $ x $ nell'intervallo, dunque si deve avere che la somma degli argomenti delle sinusoidi sia $ \pi $. Inoltre, si nota che se la somma fosse $ 2\pi $ si avrebbe un segno di differenza tra le sinusoidi, il quale è assorbibile dalla normalizzazione. In generale, si ha:
	\begin{equation*}
		\mathcal{N}' = (-1)^{n + 1} \mathcal{N}
		\qquad \qquad
		\left( \int_b^x d\xi\, \frac{p(x)}{\hbar} + \frac{\pi}{4} \right) + \left( \int_x^a d\xi\, \frac{p(x)}{\hbar} + \frac{\pi}{4} \right) = n\pi
	\end{equation*}
	con $ n \in \N $. La condizione sugli integrali è una condizione di quantizzazione:
	\begin{equation*}
		\int_b^a d\xi\, p(\xi) = \left( n + \frac{1}{2} \right) \hbar \pi
	\end{equation*}
	Questa condizione di quantizzazione era stata già postulata da Bohr e Sommerfeld con considerazioni analoghe a quelle del modello di Bohr per l'atomo d'idrogeno (vedere Sec. \ref{bohr-hydrogen}), supponendo che ogni sistema hamiltoniano debba soddisfarre la \textit{condizione di Bohr-Sommerfeld}:
	\begin{equation}
		\oint p\,dq = 2\hbar \pi n
		\label{eq:4.31}
	\end{equation}
	Per $ n \gg 1 $ questa coincide con la condizione trovara nell'approssimazione semiclassica (si ha un fattore 2 poiché l'integrale è lungo un loop chiuso, dunque percorre l'intervallo due volte).\\
	Si vede inoltre il comportamento qualitativo dello spettro di una buca di potenziale generica: nella regione interna alla buca si ha $ \psi(x) \sim \sin (f(x) + \delta) $, con $ f(b) + \delta = \frac{\pi}{4} $ e $ f(a) + \delta = \frac{\pi}{4} + n\pi $. Ne segue che nello stato fondamentale il seno compie un semiperiodo, nel primo stato eccitato un periodo, nel secondo un periodo e mezzo, e così via: il numero di nodi della funzione d'onda cresce al crescere dell'energia e ad ogni stato eccitato si aggiunge un semiperiodo d'oscillazione aggiuntivo.
\end{example}












\chapter{Teoria delle Perturbazioni}
\selectlanguage{italian}

\section{Perturbazioni indipendenti dal tempo}

I metodi perturbativi indipendenti dal tempo vengono usati in tutti i casi in cui si cerca di determinare lo spettro di un operatore, ed in particolare un'hamiltoniana, che si può scrivere come la somma di un operatore, il cui spettro è noto, ed un termine di correzione indipendente dal tempo.\\
Si consideri un'hamiltoniana del tipo:
\begin{equation}
	\hat{\mathcal{H}} = \hat{\mathcal{H}}_0 + \varepsilon \hat{\mathcal{H}}'
	\label{eq:5.1}
\end{equation}
Si supponga che lo spettro si $ \hat{\mathcal{H}}_0 $ sia ortonormale e noto:
\begin{equation}
	\hat{\mathcal{H}} \ket{n_0} = E_n^{(0)} \ket{n_0}
	\qquad \qquad
	\braket{m_0 | n_0} = \delta_{mn}
	\label{eq:5.2}
\end{equation}
Lo spettro dell'hamiltoniana completa è invece:
\begin{equation}
	\hat{\mathcal{H}} \ket{n} = E_n \ket{n}
	\label{eq:5.3}
\end{equation}
L'idea del metodo perturbativo è determinare autofunzioni ed autovalori dell'hamiltoniana completa come serie di potenze di $ \varepsilon $:
\begin{equation}
	E_n = E_n^{(0)} + \sum_{k \ge 1} \varepsilon^k E_n^{(k)}
	\label{eq:5.4}
\end{equation}
\begin{equation}
	\ket{n} = \ket{n_0} + \sum_{k \ge 1} \varepsilon^k \ket{n_k}
	\label{eq:5.5}
\end{equation}
È importante osservare che, in generale, $ \varepsilon $ non è necessariamente piccolo e le due serie perturbative non è garantito che convergano.

\subsection{Spettro non-degenere}

Si consideri lo spettro dell'hamiltoniana non-perturbata in Eq. \ref{eq:5.2} non-degenere, dunque $ E_n^{(0)} \neq E_k^{(0)} $ per $ n \neq k $. Sostituendo gli sviluppi perturbativi nell'Eq. \ref{eq:5.3}:
\begin{equation*}
	\left( \hat{\mathcal{H}}_0 + \varepsilon \hat{\mathcal{H}}' \right) \left( \ket{n_0} + \varepsilon \ket{n_1} + \dots \right) = \left( E_n^{(0)} + \varepsilon E_n^{(1)} + \dots \right) \left( \ket{n_0} + \varepsilon \ket{n_1} + \dots \right)
\end{equation*}
Identificando i termini dello stesso ordine in $ \varepsilon $ si trova una sequenza di equazioni:
\begin{align*}
	\varepsilon^0 : \,\left( \hat{\mathcal{H}}_0 - E_n^{(0)} \right) \ket{n_0} &= 0 \\
	\varepsilon^1 : \,\left( \hat{\mathcal{H}}_0 - E_n^{(0)} \right) \ket{n_1} &= \left( E_n^{(1)} - \hat{\mathcal{H}}' \right) \ket{n_0} \\
	\varepsilon^2 : \,\left( \hat{\mathcal{H}}_0 - E_n^{(0)} \right) \ket{n_2} &= \left( E_n^{(1)} - \hat{\mathcal{H}} \right) \ket{n_1} + E_n^{(2)} \ket{n_0} \\
								 & \,\,\, \vdots
\end{align*}
In generale:
\begin{equation}
	\left( \hat{\mathcal{H}}_0 - E_n^{(0)} \right) \ket{n_k} = \left( E_n^{(1)} - \hat{\mathcal{H}}' \right) \ket{n_{k-1}} + \sum_{j = 2}^{k} E_n^{(j)} \ket{n_{k-j}}
	\label{eq:5.6}
\end{equation}

\begin{proposition}\label{pert-non-deg-spec}
	È possibile scegliere tutte le correzioni dell'autostato ortogonali all'autostato non-perturbato: $ \braket{n_0 | n_k} = 0 \,\forall k \in \N $.
\end{proposition}
\begin{proof}
	Dato $ \ket{n_k} $, si consideri $ \ket{\tilde{n}_k} \equiv \ket{n_k} + \lambda \ket{n_0} $:
	\begin{equation*}
		\left( \hat{\mathcal{H}}_0 - E_n^{(0)} \right) \ket{\tilde{n}_k}  = \left( \hat{\mathcal{H}}_0 - E_n^{(0)} \right) \ket{n_k} + \lambda \left( \hat{\mathcal{H}}_0 - E_n^{(0)} \right) \ket{n_0} = \left( \hat{\mathcal{H}}_0 - E_n^{(0)} \right) \ket{n_k}
	\end{equation*}
	Dunque l'Eq. \ref{eq:5.6} rimane inalterata per $ \ket{n_k} \mapsto \ket{n_k} + \lambda \ket{n_0} $. Pertanto, data la soluzione $ \ket{n_k} : \braket{n_0 | n_k} \neq 0 $, è sempre possibile trovarne una $ \ket{\tilde{n}_k} : \braket{n_0 | \tilde{n}_k} = 0 $  scegliendo $ \lambda_k = - \braket{n_0 | n_k} $.
\end{proof}

Data la Prop. \ref{pert-non-deg-spec}, si può estrarre la correzione al $ k $-esimo ordine dell'autovalore $ E_n^{(k)} $ dall'Eq. \ref{eq:5.6} semplicemente proiettando su $ \ket{n_0} $: così facendo, tutti i termini nel lato destro si annullano, eccetto quello con $ \ket{n_0} $, lasciando solo il lato sinistro. Si trova quindi:
\begin{equation}
	E_n^{(k)} = \braket{n_0 | \hat{\mathcal{H}}' | n_{k-1}}
	\label{eq:5.7}
\end{equation}
Si vede che la correzione all'ordine $ k $ è determinata da quella all'ordine $ k - 1 $. Per trovare la correzione al $ k $-esimo ordine dell'autostato, invece, è necessario applicare all'Eq. \ref{eq:5.6} l'inverso dell'operatore $ \hat{\mathcal{H}}_0 - E_n^{(0)} $; questo operatore, però, non è invertibile nello spazio degli autostati fisici, poiché $ \ket{n_0} $ ha autovalore nullo, essendo $ (\hat{\mathcal{H}}_0 - E_n^{(0)}) \ket{n_0} = 0 $. Il problema è risolto dalla Prop. \ref{pert-non-deg-spec}: tutte le correzioni $ \ket{n_k} $ appartengono al sottospazio ortogonale ad $ \ket{n_0} $, dunque l'operatore è invertibile in tale sottospazio. In generale si ha:
\begin{equation}
	\ket{n_j} = \sum_{k \neq n} \ket{k_0} \braket{k_0 | n_j}
	\label{eq:5.8}
\end{equation}
Nel sottospazio ortogonale ad $ \ket{n_0} $, dunque, l'operatore $ \hat{\mathcal{H}}_0 - E_n^{(0)} $ si scrive in serie come:
\begin{equation}
	\frac{1}{\hat{\mathcal{H}}_0 - E_n^{(0)}} = \sum_{k \neq n} \frac{1}{E_k^{(0)} - E_n^{(0)}} \ket{k_0} \bra{k_0}
	\label{eq:5.9}
\end{equation}
È possibile applicare questo operatore all'Eq. \ref{eq:5.6} per trovare la correzione dell'autostato a qualsiasi ordine. Per il prim'ordine:
\begin{equation*}
	\ket{n_1} = \sum_{k \neq n} \frac{1}{E_k^{(0)} - E_n^{(0)}} \ket{k_0} \braket{k_0 | \left( E_n^{(1)} - \hat{\mathcal{H}}' \right) | n_0} = \sum_{k \neq n} \frac{1}{E_k^{(0)} - E_n^{(0)}} \ket{k_0} \left[ \delta_{kn} E_n^{(1)} - \braket{k_0 | \hat{\mathcal{H}}' | n_0} \right]
\end{equation*}
La correzione al prim'ordine dell'autostato è dunque:
\begin{equation}
	\ket{n_1} = \sum_{k \neq n} \frac{\braket{k_0 | \hat{\mathcal{H}}' | n_0}}{E_n^{(0)} - E_k^{(0)}} \ket{k_0}
	\label{eq:5.10}
\end{equation}
La correzione al prim'ordine dell'autovalore, invece, si trova banalmente dall'Eq. \ref{eq:5.7}:
\begin{equation}
	E_n^{(1)} = \braket{n_0 | \hat{\mathcal{H}}' | n_0}
	\label{eq:5.11}
\end{equation}
Per trovare la correzione al second'ordine dell'autovalore dall'Eq. \ref{eq:5.7}:
\begin{equation*}
	E_n^{(2)} = \braket{n_0 | \hat{\mathcal{H}}' \sum_{k \neq n} \frac{\braket{k_0 | \hat{\mathcal{H}}' | n_0}}{E_n^{(0)} - E_k^{(0)}} | k_0} = \sum_{k \neq n} \frac{\braket{n_0 | \hat{\mathcal{H}}' | k_0} \braket{k_0 | \hat{\mathcal{H}}' | n_0}}{E_n^{(0)} - E_k^{(0)}}
\end{equation*}
da cui:
\begin{equation}
	E_n^{(2)} = \sum_{k \neq n} \frac{\abs{\braket{k_0 | \hat{\mathcal{H}}' | n_0}}^2}{E_n^{(0)} - E_k^{(0)}}
	\label{eq:5.12}
\end{equation}
Si nota che la correzione al second'ordine per lo stato fondamentale, ossia quello di minima energia, è sempre negativa, proprio come ci si aspetterebbe.
La correzione $ \ket{n_2} $ si trova invece come:
\begin{equation*}
	\begin{split}
		\ket{n_2}
		&= \sum_{k \neq n} \frac{1}{E_k^{(0)} - E_n^{(0)}} \ket{k_0} \bra{k_0} \left[ \left( E_n^{(1)} - \hat{\mathcal{H}}' \right) \ket{n_1} + E_n^{(2)} \ket{n_0} \right] \\
		&= \sum_{k \neq n} \frac{1}{E_k^{(0)} - E_n^{(0)}} \ket{k_0} \left[ E_n^{(1)} \braket{k_0 | n_1} - \braket{k_0 | \hat{\mathcal{H}}' | n_1} + \delta_{kn} E_n^{(2)} \right] \\
		&= \sum_{k \neq n} \frac{1}{E_n^{(0)} - E_k^{(0)}} \ket{k_0} \left[ \sum_{m \neq n} \frac{\braket{m_0 | \hat{\mathcal{H}}' | n_0}}{E_n^{(0)} - E_m^{(0)}} \braket{k_0 | \hat{\mathcal{H}}' | m_0} - E_n^{(1)} \sum_{m \neq n} \frac{\braket{m_0 | \hat{\mathcal{H}}' | n_0}}{E_n^{(0)} - E_m^{(0)}} \underbrace{\braket{k_0 | m_0}}_{\delta_{km}} \right]
	\end{split}
\end{equation*}
La correzione al second'ordine dell'autostato è quindi:
\begin{equation}
	\ket{n_2} = \sum_{k \neq n} \frac{1}{E_n^{(0)} - E_k^{(0)}} \left[ \sum_{m \neq n} \frac{\braket{m_0 | \hat{\mathcal{H}}' | n_0} \braket{k_0 | \hat{\mathcal{H}}' | m_0}}{E_n^{(0)} - E_m^{(0)}} - \frac{\braket{k_0 | \hat{\mathcal{H}}' | n_0} \braket{n_0 | \hat{\mathcal{H}}' | n_0}}{E_n^{(0)} - E_k^{(0)}} \right] \ket{k_0}
	\label{eq:5.13}
\end{equation}

\subsubsection{Spettro degenere}

Nel caso in cui lo spettro dell'hamiltoniana non-perturbata sia degenere, la trattazione diventa più complessa: in generale, la base di autostati di partenza e quella di arrivo saranno diverse.\\
Ad esempio, si prenda l'oscillatore armonico isotropo: due possibili basi sono $ \ket{n_1,n_2,n_3} $ e $ \ket{n,\ell,m} $, ma ce ne sono infinite altre. La perturbazione potrebbe selezionare una base particolare: ad esempio, una perturbazione lungo l'asse $ x $ seleziona la base $ \ket{n_1,n_2,n_3} $, mentre una perturbazione dipendente dal momento angolare seleziona la base $ \ket{n,\ell,m} $.\\
Si supponga che allo stesso autostato di energia $ E_n^{(0)} $ di $ \hat{\mathcal{H}} $ siano associati $ d $ autostati $ \ket{n^{(0)}_k} $, con $ k = 1, \dots, d $. Per effetto della perturbazione, le energie di questi stati possono in generale diventare diverse tra loro, riducendo o eliminando la degenerazione (o anche lasciandola invariata): gli stati perturbati $ \ket{n_k} = \ket{n^{(0)}_k} + \sum_{m \ge 1} \varepsilon^m \ket{n^{(m)}_k} $ avranno in generale energie $ E_{n,k} $ diverse tra loro.\\
Di conseguenza, qualsiasi combinazione lineare di $ \ket{n^{(0)}_k} $ è ancora autostato di $ \hat{\mathcal{H}}_0 $ con autostato $ E^{(0)}_n $, ma soltanto una base degenere precisa è quella a cui si riduce lo sviluppo perturbativo per $ \varepsilon \rightarrow 0 $. Per determinarla, si moltiplichi l'Eq. \ref{eq:5.6} al prim'ordine per $ \bra{n_j^{(0)}} $:
\begin{equation*}
	\braket{n^{(0)}_j | \left( \hat{\mathcal{H}}_0 - E^{(0)}_n \right) | n^{(1)}_k} = \braket{n^{(0)}_j | \left( E^{(1)}_{n,k} - \hat{\mathcal{H}}' \right) | n^{(0)}_k}
\end{equation*}
Ricordando che $ \bra{n^{(0)}_j} \hat{\mathcal{H}}_0 = \bra{n^{(0)}_j} E^{(0)}_n $ e $ \braket{n^{(0)}_j | n^{(0)}_k} = \delta_{jk} $ (base degenere ortonormale), si trova la condizione sulla base di autostati degeneri di $ \hat{\mathcal{H}}_0 $:
\begin{equation}
	\braket{n^{(0)}_j | \hat{\mathcal{H}}' | n^{(0)}_k} = E^{(1)}_{n,k} \delta_{jk}
	\label{eq:5.14}
\end{equation}
Bisogna dunque scegliere una base che diagonalizzi l'hamiltoniana perturbante (e, di conseguenza, quella perturbata completa): le correzioni al prim'ordine dell'energia sono gli autovalori così trovati, mentre gli autoket di ordine zero sono i rispettivi autovettori, ovvero la base diagonalizzante.\\
Se a seguito di questa procedura tutti gli autovalori $ E^{(1)}_{n,k} $ sono diversi tra loro, ovvero se la degenerazione è completamente eliminata, le correzioni ad ordini successivi sono calcolate nel caso non-degenere. Se invece la degenerazione è ancora presente, si ripete questa procedura nel sottospazio rimasto degenere dopo l'introduzione della perturbazione all'ordine precedente.

\section{Perturbazioni dipendenti dal tempo}

La teoria delle perturbazioni dipendenti dal tempo si usa per affrontare quei problemi (es.: fenomeni di diffusione) in cui è necessario valutare la probabilità che il sistema subisca una transizione da un certo stato iniziale ad un certo stato finale per effetto di un potenziale.

\subsection{Rappresentazione d'interazione}

È conveniente introdurre una nuova rappresentazione, intermedia tra quella di Heisenberg e quella di Schrödinger.
Si consideri un sistema descritto da una hamiltoniana del tipo:
\begin{equation}
	\hat{\mathcal{H}} = \hat{\mathcal{H}}_0 + \hat{V}(t)
	\label{eq:5.15}
\end{equation}
composta da un termine imperturbato ed una perturbazione dipendente dal tempo. La \textit{rappresentazione d'interazione} consiste nel trattare $ \hat{\mathcal{H}}_0 $ in rappresentazione di Heisenberg e $ \hat{V}(t) $ in rappresentazione di Schrödinger. Denotando con $ \ket{\psi(t)}_{\text{S}} $ gli stati in rappresentazione di Schrödinger, si definisce la rappresentazione d'interazione come:
\begin{equation}
	\ket{\psi(t)}_{\text{I}} \defeq e^{-\frac{1}{i\hbar} (t - t_0) \hat{\mathcal{H}}_0} \ket{\psi(t)}_{\text{S}} \equiv \hat{S}_0^{-1}(t,t_0) \ket{\psi(t)}_{\text{S}}
	\label{eq:5.16}
\end{equation}

\begin{proposition}
	Gli operatori in rappresentazione d'interazione sono legati a quelli in rappresentazione di Schrödinger da:
	\begin{equation}
		\hat{O}_{\text{I}} = \hat{S}_0^{-1}(t,t_0) \hat{O}_{\text{S}} \hat{S}_0(t,t_0)
		\label{eq:5.17}
	\end{equation}
\end{proposition}

\begin{proposition}
	La dipendenza temporale degli stati in rappresentazione d'interazione è data solo dal termine perturbativo:
	\begin{equation}
		\ket{\psi(t)}_{\text{I}} = \hat{S}_{\text{I}}(t,t_0) \ket{\psi(t_0)}_{\text{I}} \equiv \mathcal{T} \exp \left[ \frac{1}{i\hbar} \int_{t_0}^t dt'\, \hat{V}_{\text{I}}(t') \right] \ket{\psi(t_0)}_{\text{I}}
		\label{eq:5.18}
	\end{equation}
\end{proposition}
\begin{proof}
	Per calcolo esplicito:
	\begin{equation*}
		\begin{split}
			i\hbar \frac{\pa}{\pa t} \ket{\psi(t)}_{\text{I}}
			&= - e^{-\frac{1}{i\hbar} (t - t_0) \hat{\mathcal{H}}_0} \hat{\mathcal{H}}_0 \ket{\psi(t)}_{\text{S}} + i\hbar e^{-\frac{1}{i\hbar} (t - t_0) \hat{\mathcal{H}}_0} \frac{\pa}{\pa t} \ket{\psi(t)}_{\text{S}} \\
			&= - e^{-\frac{1}{i\hbar} (t - t_0) \hat{\mathcal{H}}_0} \hat{\mathcal{H}}_0 \ket{\psi(t)}_{\text{S}} + e^{-\frac{1}{i\hbar} (t - t_0) \hat{\mathcal{H}}_0} ( \hat{\mathcal{H}}_0 + \hat{V}(t) ) \ket{\psi(t)}_{\text{S}} \\
			&= e^{-\frac{1}{i\hbar} (t - t_0) \hat{\mathcal{H}}_0} \hat{V}(t) \ket{\psi(t)}_{\text{S}} = \hat{S}_0^{-1}(t,t_0) \hat{V}(t) \hat{S}_0(t,t_0) \ket{\psi(t)}_{\text{I}} = \hat{V}_{\text{I}}(t) \ket{\psi(t)}_{\text{I}}
		\end{split}
	\end{equation*}
	In generale $ \hat{V}(t) $ non commuta a tempi diversi, da cui la necessità del prodotto cronologico $ \mathcal{T} $.
\end{proof}

Si può verificare che l'evoluzione temporale determinata dalla rappresentazione di Schrödinger coincide con quella in rappresentazione d'interazione: in particolare, si ottengono le stesse probabilità per i risultati delle misure.

\begin{proposition}
	La rappresentazione d'interazione è equivalente a quella di Schrödinger.
\end{proposition}
\begin{proof}
	In rappresentazione di Schrödinger, l'ampiezza di probabilità che il sistema al tempo $ t $ si trovi in un autostato $ \ket{n} $ dell'operatore hermitiano $ \hat{O} $ è data da:
	\begin{equation*}
		\braket{n | \psi(t)}_{\text{S}} = \braket{n | \hat{S}(t,t_0) | \psi(t_0)}_{\text{S}}
	\end{equation*}
	In rappresentazione d'interazione, gli autostati di $ \hat{O} $ sono determinati come:
	\begin{equation*}
		\hat{O}_{\text{I}}(t) \ket{n(t)}_{\text{I}} = \lambda_n \ket{n(t)}_{\text{I}}
	\end{equation*}
	Dalle Eqq. \ref{eq:5.16}-\ref{eq:5.17}:
	\begin{equation*}
		\ket{n(t)}_{\text{I}} = \hat{S}^{-1}_0 (t,t_0) \ket{n}_{\text{S}} \equiv \ket{\tilde{n}(t)}
	\end{equation*}
	Si trova dunque:
	\begin{equation*}
		\braket{\tilde{n}(t) | \psi(t)}_{\text{I}} = \braket{n | \hat{S}_0 (t,t_0) \hat{S}^{-1}_0 (t,t_0) | \psi(t)}_{\text{S}} = \braket{n | \hat{S}(t,t_0) | \psi(t_0)}_{\text{S}}
	\end{equation*}
\end{proof}

\subsection{Sviluppo perturbativo dipendente dal tempo}

Si consideri lo spettro energetico dell'hamiltoniana non-perturbata, in generale degenere:
\begin{equation*}
	\hat{\mathcal{H}}_0 \ket{n} = E_n \ket{n}
\end{equation*}
Si supponga che, a seguito dell'interazione con la perturbazione $ \hat{V}(t) $, il sistema compia una transizione in $ \ket{m} $. L'ampiezza di probabilità della transizione è definita come:
\begin{equation}
	\mathscr{A}^{(\text{S})}_{nm}(t) \defeq \braket{m | \hat{S}(t,t_0) | n}
	\label{eq:5.19}
\end{equation}
Si ricordi che la probabilità è definita come $ \mathscr{P} \defeq \abs{\mathscr{A}}^2 $.

\begin{proposition}
	$ \mathscr{P}^{(\text{I})}_{nm} = \mathscr{P}^{(\text{S})}_{nm} $.
\end{proposition}
\begin{proof}
	Per calcolo esplicito, ricordando le Eqq. \ref{eq:5.16}-\ref{eq:5.17}:
	\begin{equation*}
		\begin{split}
			\mathscr{A}^{(\text{S})}_{nm}(t)
			&= \braket{\tilde{m} | \hat{S}_{\text{I}}(t,t_0) | \tilde{n}} = \braket{m | \hat{S}_0(t,t_o) \hat{S}_{\text{I}}(t,t_0) \hat{S}^{-1}_0 (t,t_o) | n} \\
			&= e^{\frac{1}{i\hbar} \left( E_m t - E_n t_0 \right)} \braket{m | \hat{S}_{\text{I}}(t,t_0) | n} = e^{\frac{1}{i\hbar} \left( E_m t - E_n t_0 \right)} \mathscr{A}^{(\text{I})}_{nm}(t)
		\end{split}
	\end{equation*}
\end{proof}

Si noti che in generale il tempo $ t_o $ da cui si definisce la rappresentazione d'interazione è diverso dal tempo $ t_0 $ in cui si trova lo stato iniziale.\\
Dato che la probabilità è indipendente dalla rappresentazione usata, si calcola $ \mathscr{A}^{(\text{I})}_{nm}(t) $:
\begin{equation*}
	\begin{split}
		\mathscr{A}^{(\text{I})}_{nm}(t)
		&= \braket{m | \id + \frac{1}{i\hbar} \int_{t_0}^t dt'\, \hat{V}_{\text{I}}(t') + \frac{1}{2} \frac{1}{(i\hbar)^2} \mathcal{T} \int_{t_0}^t dt' dt''\, \hat{V}_{\text{I}}(t') \hat{V}_{\text{I}}(t'') + \dots | n} \\
		&= \delta_{nm} + \frac{1}{i\hbar} \int_{t_0}^t dt' \braket{m | \hat{S}^{-1}_0 (t',t_o) \hat{V}(t') \hat{S}_0 (t',t_o) | n} + \dots
	\end{split}
\end{equation*}
Si può interpretare questa come una serie perturbativa, detta \textit{serie di Dyson}:
\begin{equation}
	\mathscr{A}^{(\text{I})}_{nm}(t) = \sum_{i = 0}^{\infty} \mathscr{A}^{(i)}_{nm}(t)
	\label{eq:5.20}
\end{equation}
Ricordando le proprietà del prodotto cronologico, si trovano i primi termini dello sviluppo:
\begin{equation}
	\mathscr{A}^{(0)}_{nm}(t) = \delta_{nm}
	\label{eq:5.21}
\end{equation}
\begin{equation}
	\mathscr{A}^{(1)}_{nm}(t) = \frac{1}{i\hbar} \int_{t_0}^t dt' \braket{m | \hat{V}(t') | n} e^{\frac{1}{i\hbar} \left( E_n - E_m \right) t'}
	\label{eq:5.22}
\end{equation}
\begin{equation}
	\mathscr{A}^{(2)}_{nm}(t) = \frac{1}{(i\hbar)^2} \int_{t_0}^t dt' \int_{t_0}^{t'} dt'' \sum_{k} \braket{m | \hat{V}(t') | k} e^{\frac{1}{i\hbar} \left( E_k - E_m \right) t'} \braket{k | \hat{V}(t'') | n} e^{\frac{1}{i\hbar} \left( E_n - E_k \right) t''}
	\label{eq:5.23}
\end{equation}
dove la sommatoria è sullo spettro di $ \hat{\mathcal{H}}_0 $.

\subsection{Regola aurea di Fermi}

Si consideri il caso in cui la perturbazione sia indipendente dal tempo, attiva solo per $ t > 0 $:
\begin{equation}
	V(t) = V \theta(t)
	\label{eq:5.24}
\end{equation}
dove $ \theta $ è la distribuzione di Heaviside. Ciò permette di trattare il caso in cui il sistema viene preparato in uno stato iniziale in una regione in cui il potenziale è trascurabile, tipico degli esperimenti di scattering. In questo caso, ponendo $ V_{mn} \equiv \braket{m | \hat{V} | n} $, la serie di Dyson al prim'ordine per $ n \neq m $ coincide con:
\begin{equation*}
	\mathscr{A}^{(1)}_{mn}(t) = \frac{1}{i\hbar} \int_0^t dt'\, e^{\frac{1}{i\hbar} \left( E_n - E_m \right) t'} V_{mn} = - \frac{e^{\frac{1}{i\hbar} \left( E_n - E_m \right) t} - 1}{E_m - E_n} V_{mn}
\end{equation*}
Ponendo $ E_n = \hbar \omega_n $, si trova:
\begin{equation*}
	\begin{split}
		\frac{1}{t} \mathscr{P}_{nm}
		&= \frac{1}{t} \frac{\abs{V_{nm}}^2}{(\omega_m - \omega_n)^2 \hbar^2} \abs{e^{it \left( \omega_m - \omega_n \right)} - 1}^2 \\
		&= \frac{1}{t} \frac{\abs{V_{nm}}^2}{(\omega_m - \omega_n)^2 \hbar^2} \abs{e^{\frac{it}{2} \left( \omega_m - \omega_n \right)} \left( e^{\frac{it}{2} \left( \omega_m - \omega_n \right)} - e^{-\frac{it}{2} \left( \omega_m - \omega_n \right)} \right)}^2 \\
		&= \frac{\abs{V_{nm}}^2}{(\omega_m - \omega_n)^2 \hbar^2} \frac{4}{t} \sin^2 \left[ \frac{t}{2} \left( \omega_m - \omega_n \right) \right]
	\end{split}
\end{equation*}
Di particolare interesse è il limite $ t \rightarrow \infty $: la particella incidente arriva da una regione lontana, dunque la perturbazione agisce per un tempo molto lungo rispetto alla scala di tempi tipici del potenziale stesso. La funzione $ x^{-2} \sin^2 (tx^2) $ diventa sempre più piccata nell'origine al crescere di $ t $, inoltre il suo integrale è costante:
\begin{equation*}
	\int_{-\infty}^{\infty} dx\, \frac{\sin^2 (tx^2)}{x^2} = t\pi
	\qquad \Rightarrow \qquad
	\lim_{t \rightarrow \infty} \frac{1}{\pi} \frac{\sin^2 (tx^2)}{tx^2} = \delta(x)
\end{equation*}
Nel limite per $ t \rightarrow \infty $ si ha dunque:
\begin{equation*}
	\frac{1}{t} \mathscr{P}_{nm} = \abs{V_{mn}}^2 \frac{\pi}{\hbar^2} \delta \left( \frac{\omega_m - \omega_n}{2} \right)
\end{equation*}
Ricordando che $ \delta(\alpha x) = \abs{\alpha}^{-1} \delta(x) $, si trova:
\begin{equation}
	\frac{1}{t} \mathscr{P}_{nm} = \frac{2\pi}{\hbar} \abs{V_{mn}}^2 \delta(E_m - E_n)
	\label{eq:5.25}
\end{equation}
Questa è nota come \textit{regola aurea di Fermi}: essa fornisce la probabilità di transizione per unità di tempo ed esprime la conservazione dell'energia nei fenomeni d'urto.

\subsection{Teoria d'urto}

L'osservabile fondamentale in teoria d'urto è la \textit{sezione d'urto} $ \sigma $: dato un flusso costante di particelle incidenti su un bersaglio, essa è il numero di particelle diffuse per unità di flusso entrante ed unità di tempo. La sezione d'urto può essere vista come una generalizzazione quantistica della sezione geometrica di un bersaglio.

\begin{example}
	Nel caso classico, dato un flusso costante di particelle $ j = \frac{dn}{dt ds} $ incidenti in un intervallo $ \Delta t $ su un bersaglio di area $ S $, il numero di particelle diffuse sarà $ N = j S \Delta t $, dunque la sezione geometrica è $ S = \frac{N}{j \Delta t} $.
\end{example}

Nel caso quantistico, dove l'interazione ha natura probabilistica (e non deterministica), è utile definire la \textit{sezione d'urto differenziale}, ovvero la sezione d'urto per unità di spazio delle fasi finale, indicata con $ \frac{d \sigma}{d \Omega} $, dove $ \Omega $ è un osservabile (o più osservabili) che caratterizzano la cinematica dello stato finale.
Dato un flusso $ j_a $ di particelle incidenti nello stato $ \ket{a} $, la sezione d'urto differenziale è definita come:
\begin{equation}
	d \sigma \defeq \lim_{t \rightarrow \infty} \frac{1}{j_a \Delta t} \abs{\braket{b | \hat{S}(t,-t) | a}}^2 db
	\label{eq:5.26}
\end{equation}
dove $ \Delta t = 2t $ e $ \ket{b} $ è lo stato in cui il sistema viene misurato al tempo $ t $. L'elemento di matrice nel limite dei grandi tempi viene detto \textit{elemento di matrice} $ S $:
\begin{equation}
	S_{ab} \equiv \lim_{t \rightarrow \infty} \abs{\braket{b | \hat{S}(t,-t) | a}}^2
	\label{eq:5.27}
\end{equation}
La matrice $ S $ è unitaria.

\subsubsection{Spazio delle fasi}

Si consideri la situazione tipica in cui stato iniziale e finale sono stati d'impulso definito:
\begin{equation*}
	\braket{\ve{x} | \ve{k}} = \psi_\ve{k}(\ve{x}) = \frac{1}{(2\pi)^{3/2}} e^{i \ve{k} \cdot \ve{x}} \,\, : \,\, \braket{\ve{k} | \ve{k}'} = \delta^{(3)}(\ve{k} - \ve{k}')
\end{equation*}
È conveniente passare in coordinate sferiche, dunque parametrizzando l'impulso col modulo $ k $ e gli angoli $ \vartheta_k $ e $ \varphi_k $; inoltre, essendo l'energia l'osservabile più comune negli esperimenti, conviene determinare il modulo $ k $ a partire dall'energia: si vuole dunque passare dagli stati $ \ket{\ve{k}} $ agli stati $ \ket{E, \Omega_k} $.\\
Assumento che per $ t \rightarrow \pm \infty $ l'Hamiltoniana sia quella della particella libera, dunque che il potenziale di scattering sia acceso e spento solo in un intervallo di tempo finito $ [-T, T] $, si ha che:
\begin{equation*}
	E = \frac{\hbar^2 k^2}{2m}
	\quad \Rightarrow \quad
	\delta (E - E') = \frac{2m}{\hbar^2} \delta(k^2 - k'^2) = \frac{m}{\hbar k} \delta(k - k')
\end{equation*}
Inoltre, essendo l'elemento infinitesimo dello spazio delle fasi $ d^3\ve{k} = k^2 dk d \Omega_k $ ($ d\Omega_k \equiv d \cos \vartheta_k d \varphi_k $):
\begin{equation*}
	\delta^{(3)}(\ve{k} - \ve{k}') = \frac{1}{k^2} \delta(k - k') \delta^{(2)}(\Omega_k - \Omega_{k'})
\end{equation*}
Si ha dunque:
\begin{equation*}
	\braket{E, \Omega_k | E', \Omega_{k'}} = \delta(E - E') \delta^{(2)}(\Omega_k - \Omega_{k'}) = \frac{m}k{\hbar} \delta(\ve{k} - \ve{k}')
\end{equation*}
Gli stati differiscono dunque solo per una normalizzazione:
\begin{equation}
	\ket{E, \Omega_k} = \sqrt{\frac{mk}{\hbar}} \ket{\ve{k}}
	\label{eq:5.28}
\end{equation}
Per quanto riguarda il flusso di particelle incidenti, a livello quantistico si parla di flusso di probabilità incidente, il quale è definito come:
\begin{equation}
	\ve{j} = - \frac{i\hbar}{2m} \left[ \psi^* \nabla \psi - \nabla \psi^* \psi \right]
	\label{eq:5.29}
\end{equation}
Usando $ \psi = \braket{\ve{x} | E, \Omega_k} $ e prendendo il modulo, si trova:
\begin{equation}
	j_k = \frac{k^2}{\hbar (2\pi)^3}
	\label{eq:5.30}
\end{equation}
La sezione d'urto differenziale diventa dunque:
\begin{equation*}
	d \sigma = \lim_{t \rightarrow \infty} \frac{(2\pi)^3 \hbar}{k^2} \frac{1}{\Delta t} \abs{\braket{E', \Omega_{k'} | \hat{S}(t,-t) | E, \Omega_k}}^2 dE' d \Omega_{k'}
\end{equation*}
Il rapporto $ \lim_{t \rightarrow \infty} S / \Delta t $ è la probabilità di transizione per unità di tempo, dunque dalla regola aurea di Fermi Eq. \ref{eq:5.25} si ottiene:
\begin{equation*}
	d \sigma = \frac{(2\pi)^4}{k^2} \abs{\braket{E', \Omega_{k'} | V | E, \Omega_k}}^2 \delta(E' - E) dE' d \Omega_{k'}
\end{equation*}
Integrando su tutti i possibili valori di $ E' $ si trova la sezione d'urto differenziale:
\begin{equation*}
	\frac{d \sigma}{d \Omega_{k'}} = \frac{(2\pi)^4}{k^2} \abs{\braket{E, \Omega_{k'} | V | E, \Omega_k}}^2
\end{equation*}
Questa è la sezione d'urto al primo ordine perturbativo, nota come \textit{approssimazione di Born}. Essendo $ E' = E $, si ha $ k' = k $ ed è possibile definire l'impulso trasferito $ \ve{q} \equiv \ve{k}' - \ve{k} $, così che:
\begin{equation}
	\frac{d \sigma}{d \Omega} = \frac{(2\pi)^4 m^2}{\hbar^4} \abs{f(\ve{q})}^2
	\label{eq:5.31}
\end{equation}
dove si è definito il \textit{fattore di forma} del bersaglio:
\begin{equation}
	f(\ve{q}) \defeq \frac{1}{(2\pi)^3} \int d^3 \ve{x}\, V(\ve{x}) e^{i \ve{q} \cdot \ve{x}}
	\label{eq:5.32}
\end{equation}
Questa non è altro che la trasformata di Fourier del potenziale che determina lo scattering.

\begin{example}
	Se si considera un potenziale centrale $ V = V(r) $, scrivendo $ q = 2k \sin \frac{\theta}{2} $, con $ \theta $ scattering angle, si trova:
	\begin{equation*}
		f(\ve{q}) = \frac{1}{(2\pi)^3} \int_0^\infty dr \int_{-1}^1 d \cos \vartheta \int_0^{2\pi} d \varphi\, r^2 V(r) e^{i qr \cos \theta} = \frac{1}{(2\pi)^2} \int_0^\infty dr\, r V(r) \frac{2 \sin (qr)}{q}
	\end{equation*}
	ovvero:
	\begin{equation*}
		\frac{d \sigma}{d \Omega} = \frac{4m^2}{q^2 \hbar^4} \abs{ \int_0^\infty dr\, r \sin(qr) V(r) }^2
	\end{equation*}
\end{example}
Ad esempio, per il potenziale di Yukawa $ V(r) \sim \frac{1}{r} e^{-r / \lambda} $ si trova $ \frac{d \sigma}{d \Omega} \sim q^{-2} (q^2 + \lambda^{-2})^{-1} $, che nel limite Coulombiano $ \lambda \rightarrow \infty $ si riduce alla sezione d'urto di Rutherford $ \frac{d \sigma}{d \Omega} \sim \left( \sin \frac{\theta}{2} \right)^{-4} $.












\end{document}
