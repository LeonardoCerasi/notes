\selectlanguage{italian}

La fisica quantistica è una teoria stocastica, non probabilistica, poiché permette di prevedere la probabilità che il sistema si trovi in un determinato stato e non le probabilità dei singoli eventi: questi avvengono con la misura, la quale fa cambiare l'informazione sul sistema in modo discontinuo. L'evoluzione temporale dello stato del sistema è data da trasformazioni unitarie che permettono di prevedere lo stato futuro del sistema.\\
La generalizzazione della meccanica quantistica unidimensionale a sistemi in più dimensioni e con più corpi introduce una notevole complessità nella trattazione che porta a sviluppi formali legati ai principi della fisica quantistica.\\
La teoria quantistica si sviluppa in direzioni diverse in base a due tipi di sistemi:
\begin{itemize}
	\item sistemi riducibili, i quali vengono ricondotti a problemi più semplici a bassa dimensionalità (analogamente alla separazione del problema dei due corpi nel problema del baricentro e in quello del moto relativo), introducendo di conseguenza nuove osservabili associate alle trasformazioni possibili del sistema (studio dei gruppi di simmestria del sistema);
	\item sistemi irriducibili, che invece non possono essere semplificati per via di fenomeni come l'entanglement (\textit{Verschränkung}) che emergono nei sistemi a più corpi.
\end{itemize}
La trattazione di sistemi complessi può essere semplificata in vari modi:
\begin{itemize}
	\item limite classico: formulazione completamente diversa della meccanica quantistica introdotta da Feynman e basata sul concetto di integrale di cammino (path integral), permette di capire la relazione tra fisica classica e quantistica;
	\item metodi perturbativi: permettono di trovare soluzioni approssimate e non esatte; in particolare, si usano due classi di metodi perturbativi in base al sistema considerato:
	\begin{itemize}
		\item indipendenti dal tempo, importanti per lo studio degli stati legati (es. atomo di elio);
		\item dipendenti dal tempo, utilizzati per studiare gli stati del continuo (es. teoria d'urto).
	\end{itemize}
\end{itemize}
