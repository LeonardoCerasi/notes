\selectlanguage{italian}

\section{Spazio prodotto diretto}

Per definire formalmente i sistemi quantistici in più dimensioni, è necessario definire prima il prodotto diretto tra spazi di Hilbert.

\begin{definition}
	Dati due spazi di Hilbert $ \hilb $ e $ \mathscr{K} $ con basi $ \{\ket{e_i}\} $ e $ \{\ket{\tilde{e}_j}\} $, si definisce il loro prodotto diretto come $ \hilb \otimes \mathscr{K} \defeq \{ \ket{\psi} = \sum_{i,j} c_{ij} \ket{e_i} \otimes \ket{\tilde{e}_j} \} $. In questo spazio si definisce il prodotto scalare tra due vettori $ \ket{\psi_1} = \ket{e_{i_1}}\otimes\ket{\tilde{e}_{j_1}} $ e $ \ket{\psi_2} = \ket{e_{i_2}}\otimes\ket{\tilde{e}_{j_2}} $ come $ \braket{\psi_1 | \psi_2} = \braket{e_{i_1} | e_{i_2}}\braket{\tilde{e}_{j_1} | \tilde{e}_{j_2}} $.
\end{definition}

Per semplificare la scrittura, si adotta la notazione $ \ket{e_i}\otimes\ket{e_j} \equiv \ket{e_i e_j} $ (o si sottintende $ \otimes $).\\
Si noti che osservabili relative a spazi diversi sono sempre compatibili.\\
In generale, il generico $ \ket{\psi}\in\hilb\otimes\mathscr{K} $ non è scrivibile come prodotto diretto $ \ket{\psi} = \ket{\phi}\otimes\ket{\tilde{\phi}} $, con $ \ket{\phi}\in\hilb $ e $ \ket{\tilde{\phi}}\in\mathscr{K} $, poiché in generale non è detto che $ c_{ij} $ sia fattorizzabile in $ \alpha_i $ e $ \tilde{\alpha}_j $: in questo caso si dice che lo stato è entangled.

\begin{definition}
	Uno stato $ \ket{\psi}\in\hilb\otimes\mathscr{K} $ si dice entangled se non è fattorizzabile.
\end{definition}

\begin{example}
	Dati due qubit, uno stato entangled è $ \ket{\psi} = \frac{1}{\sqrt{2}} \left(\ket{01} + \ket{10}\right) $, dato che il generico stato fattorizzabile è $ \left(a\ket{0} + b\ket{1}\right)\otimes\left(c\ket{0} + d\ket{1}\right) = ac\ket{00} + ad\ket{01} + bc\ket{10} + bd\ket{11} $.
\end{example}

La probabilità $ P_{ij} = \abs{c_{ij}}^2 $ è detta probabilità congiunta: in generale essa non è il prodotto delle probabilità dei singoli eventi per i fenomeni di interferenza quantistica, i quali rendono tale probabilità dipendente dallo stato dell'intero sistema.

\section{Sistemi multidimensionali}

Per generalizzare la meccanica quantistica in $ d $ dimensioni, si introduce l'operatore posizione $ \hat{\ve{x}} $:
\begin{equation}
	\hat{\ve{x}} \defeq
	\begin{pmatrix}
		\hat{x}_1 \\
		\vdots \\
		\hat{x}_d
	\end{pmatrix}
	\label{eq:1}
\end{equation}
Ciascuna componente di questo vettore è un operatore hermitiano che agisce su uno spazio di Hilbert, mentre il vettore $ \hat{\ve{x}} $ agisce sul loro prodotto diretto $ \hilb \defeq \hilb_1 \otimes \dots \otimes \hilb_d $. Su ciascuno spazio $ \hilb_j $ viene definita la base delle posizioni da $ \hat{x}_j \ket{x_j} = x_j \ket{x_j} $, dunque la base delle posizioni in $ \hilb $ sarà $ \ket{\ve{x}} \defeq \ket{x_1} \otimes \dots \otimes \ket{x_d} $: data $ \ket{\psi} \in \hilb $, la sua rappresentazione sulla base delle posizioni è $ \braket{\ve{x} | \psi} \equiv \psi(\ve{x}) $, con $ \psi : \R^d \rightarrow \C $, il cui modulo quadro dà una densità di di probabilità $ d $-dimensionale $ dP_{\ve{x}} = \abs{\psi(\ve{x})}^2 d^d\ve{x} $.\\
In questo caso, l'entanglement consiste nel fatto che, in generale, $ \psi(\ve{x}) \neq \psi_1(x_1) \dots \psi_d(x_d) $.\\
Tale formalismo è generalizzabile al caso di $ n $ corpi in $ d $ dimensioni, nel qual caso si ha uno spazio prodotto diretto di $ nd $ spazi di Hilbert.

\begin{example}
	Nel caso di $ 2 $ corpi in $ 3 $ dimensioni, si ha:
	\begin{equation*}
		\hat{\ve{x}} =
		\begin{pmatrix}
			\hat{x}_{1,1} \\
			\hat{x}_{1,2} \\
			\hat{x}_{1,3} \\
			\hat{x}_{2,1} \\
			\hat{x}_{2,2} \\
			\hat{x}_{2,3} \\
		\end{pmatrix}
	\end{equation*}
	In questo sistema, la funzione d'onda è $ \braket{\ve{x}_1 \ve{x}_2 | \psi} = \psi(\ve{x}_1, \ve{x}_2) $.
\end{example}

\begin{equation}
	\braket{\ve{x}' | \ve{x}} = \delta^{(d)}(\ve{x} - \ve{x}')
	\label{eq:2}
\end{equation}

La $ \delta^{(d)} $ è il prodotto di $ d $ delte di Dirac ed è definita da $ \int_{\R^d} d^d\ve{x} \, \delta^{(d)}(\ve{x} - \ve{x}') f(\ve{x}) = f(\ve{x}') $ come distribuzione.

\subsection{Coordinate cartesiane}

Analogamente al caso monodimensionale, per definite l'operatore impulso si considera una traslazione spaziale; le componenti del vettore operatore impulso $ \hat{\ve{p}} $ sulla base delle posizioni sono definite da:

\begin{equation}
  \braket{\ve{x} | \hat{p}_j | \psi} = - i\hbar \frac{\pa}{\pa x_j} \psi(\ve{x})
	\label{eq:3}
\end{equation}
In forma vettoriale, è possibile scrivere:
\begin{equation}
	\hat{\ve{p}} = -i\hbar\nabla
	\label{eq:4}
\end{equation}

A questo punto, è facile definite le autofunzioni dell'impulso tali per cui $ \hat{\ve{p}}\ket{\ve{k}} = \hbar \ve{k}\ket{\ve{k}} $:

\begin{equation}
	\braket{\ve{x} | \ve{k}} = \frac{1}{(2\pi)^{d/2}} e^{i \ve{k}\cdot\ve{x}} \equiv \psi_{\ve{k}}(\ve{x})
	\label{eq:5}
\end{equation}

Il fatto che operatori su spazi diversi commutino tra loro implica che:

\begin{equation}
	\left[ \hat{p}_j, \hat{p}_k \right] = 0 \quad \forall j,k = 1, \dots, d
	\label{eq:6}
\end{equation}

Dal punto di vista matematico, questo è ovvio per il lemma di Schwarz (assumendo una well-behaved $ \psi $), mentre da quello fisico ciò esprime il fatto che traslazioni lungo assi diversi commutano tra loro: ciò non è scontato, infatti ad esempio le rotazioni rispetto ad assi diversi non commutano (dunque le componenti del momento angolare non commuteranno).\\
%
È facile vedere che $ \hat{\ve{p}}^2 = -\hbar^2 \lap $, dunque è possibile definire l'Hamiltoniana del sistema (e con essa la sua evoluzione temporale):

\begin{equation}
	\mathcal{H} = - \frac{\hbar^2}{2m} \lap + V(\ve{x})
	\label{eq:7}
\end{equation}

Ricordando che $ \hat{\mathcal{H}}\ket{\psi} = E\ket{\psi} $, si ottiene l'equazione di Schrödinger sulla base delle coordinate:
\begin{equation}
	- \frac{\hbar^2}{2m} \lap \psi(\ve{x}) + V(\ve{x}) \psi(\ve{x}) = E \psi(\ve{x})
	\label{eq:8}
\end{equation}

\section{Separabilità}

Nel caso di sistemi non-entangled, è possibile separare il problema multidimensionale in $ d $ problemi monodimensionali e scrivere la soluzione come prodotto delle soluzioni dei problemi ridotti.

\subsection{Problemi separabili in coordinate cartesiane}

\begin{proposition}
	In coordinate cartesiane, condizione sufficiente affinché il problema sia separabile è che:
	\begin{equation}
		V(\ve{x}) = V_1(x_1) + \dots + V_d(x_d)
		\label{eq:9}
	\end{equation}
\end{proposition}

In tal caso, l'Hamiltoniana del sistema è somma di $ d $ sotto-Hamiltoniane (e di conseguenza lo è anche l'evoluzione temporale):

\begin{equation}
	\mathcal{H}_j = \frac{\hat{p}_j^2}{2m} + \hat{V}(\hat{x}_j)
	\label{eq:10}
\end{equation}

dunque la determinazione dello spettro dell'Hamiltoniana si riduce a $ d $ problemi unidimensionali.

\begin{proposition}\label{ham-sep-cart}
	Data un'Hamiltoniana separabile $ \mathcal{H} $, detti $ \braket{x_j | \psi_{k_j}} = \psi_{k_j}(x_j) $ gli autostati della $ j $-esima sotto-Hamiltoniana $ \mathcal{H}_j \ket{\psi_{k_j}} = E_{k_j} \ket{\psi_{k_j}} $, sono autostati di $ \mathcal{H} $ gli stati prodotto:
	\begin{equation}
		\braket{\ve{x} | \psi_{k_1 \dots k_d}} = \psi_{k_1 \dots k_d} (\ve{x}) \equiv \psi_{k_1}(x_1) \dots \psi_{k_d}(x_d)
		\label{eq:11}
	\end{equation}
\end{proposition}
\begin{proof}
	Si vede facilmente che:
	\begin{equation*}
		\begin{split}
			\braket{\ve{x} | \mathcal{H} | \psi_{k_1 \dots k_d}}
			&= - \frac{\hbar^2}{2m} \frac{\pa^2 \psi_{k_1}(x_1)}{\pa x_1^2} \psi_{k_2}(x_2) \dots \psi_{k_d} (x_d) + V_1(x_1) \psi_{k_1}(x_1) \psi_{k_2}(x_2) \dots \psi_{k_d}(x_d) +\\
			& \vdots\\
			& - \frac{\hbar^2}{2m} \frac{\pa^2 \psi_{k_d}(x_d)}{\pa x_d^2} \psi_{k_1}(x_1) \dots \psi_{k_{d-1}} (x_{d-1}) + V_d(x_d) \psi_{k_d}(x_d) \psi_{k_1}(x_1) \dots \psi_{k_{d-1}}(x_{d-1})\\
			&= E_{k_1} \psi_{k_1}(x_1) \dots \psi_{k_d}(x_d) + \dots + E_{k_d} \psi_{k_1}(x_1) \dots \psi_{k_d}(x_d)\\
			&= E_{k_1 \dots k_d} \psi_{k_1 \dots k_d}(\ve{x})
		\end{split}
	\end{equation*}
	dove è stata definita $ E_{k_1 \dots k_d} \equiv E_{k_1} + \dots + E_{k_d} $.
\end{proof}

\subsection{Hamiltoniane separabili}

Si può vedere che, per un'Hamiltoniana separabile, le autofunzioni \ref{eq:11} sono le più generali.\\
Innanzitutto, il commutatore canonico in $ d $ dimensioni si generalizza come:

\begin{equation}
	\left[\hat{x}_j, \hat{p}_k\right] = i\hbar \delta_{jk} \qquad \left[\hat{x}_j, \hat{x}_k\right] = 0 \qquad \left[\hat{p}_j, \hat{p}_k\right] = 0
	\label{eq:12}
\end{equation}

Da ciò segue che le Hamiltoniane \ref{eq:10} commutano tra loro, dunque sono diagonalizzabili simultaneamente e gli autovalori della loro somma sono la somma dei loro autovalori: di conseguenza, gli autostati di dell'Hamiltoniana del sistema sono tutti e soli quelli trovati nella Prop. \ref{ham-sep-cart}.\\
%
Questo argomento è facilmente generalizzabile: si consideri un'Hamiltoniana generica $ \mathcal{H} $ che è possibile separare come somma di Hamiltoniane commutanti tra loro:

\begin{equation}
	\mathcal{H} = \mathcal{H}_1 + \dots + \mathcal{H}_d \qquad \left[\mathcal{H}_j, \mathcal{H}_k\right] = 0
	\label{eq:13}
\end{equation}

Le $ \mathcal{H}_j $ sono allora diagonalizzabili simultaneamente:

\begin{equation}
	\mathcal{H}_j \ket{k_j} = E_{k_j} \ket{k_j}
	\label{eq:14}
\end{equation}

e tali autostati formano una base per gli autostati di $ \mathcal{H} $:

\begin{equation}
	\ket{k_1 \dots k_d} = \ket{k_1} \otimes \dots \otimes \ket{k_d}
	\label{eq:15}
\end{equation}

mentre i suoi autostati sono:

\begin{equation}
	E_{k_1 \dots k_d} = E_{k_1} + \dots + E_{k_d}
	\label{eq:16}
\end{equation}

\begin{example}
	Un esempio tipico di problema tridimensionale separabile è la buca parallelepipedale di potenziale:
	\begin{equation*}
		V_j(x_j) = 
		\begin{cases}
			0 & \quad \abs{x_j} < a_j\\
			\infty & \quad \abs{x_j} \ge a_j
		\end{cases}
	\end{equation*}
	Ricordando la forma esplicita delle autofunzioni:
	\begin{equation*}
		\braket{x_j | \psi_{n_j}} =
		\begin{cases}
			A_{n_j} \cos\left(k_{n_j} x_j\right) & \quad n_j = 2n + 1\\
			B_{n_j} \sin\left(k_{n_j} x_j\right) & \quad n_j = 2n\\
		\end{cases}
		\qquad\qquad k_{n_j} = \frac{n_j \pi}{2 a_j}
	\end{equation*}
	è facile ricavare lo spettro dell'Hamiltoniana:
	\begin{equation*}
		E_{n_1 n_2 n_3} = \frac{\hbar^2}{2m} \left(k_{n_1}^2 + k_{n_2}^2 + k_{n_3}^2\right) = \frac{\hbar^2 \pi^2}{8m} \left(\frac{n_1^2}{a_1^2} + \frac{n_2^2}{a_2^2} + \frac{n_3^2}{a_3^2}\right)
	\end{equation*}
	Se i valori degli $ a_j $ sono commensurabili, è possibile che lo spettro presenti delle degenerazioni: ad esempio, se si considerano $ a_1 = a_2 = a_3 \equiv a $, lo stato fondamentale $ E_{111} $ non presenta degenerazioni, ma già il primo stato eccitato è triplamente degenere: $ E_{211} = E_{121} = E_{112} $.
\end{example}

\begin{example}
	Un esempio di particolare importanza è l'oscillatore armonico tridimensionale: con lo stesso ragionamento di prima, si trova lo spettro:
	\begin{equation*}
		E_{n_1 n_1 n_3} = \hbar \left(n_1 \omega_1 + n_2 \omega_2 + n_3 \omega_3 + \frac{1}{2} \left(\omega_1 + \omega_2 + \omega_3\right)\right)
	\end{equation*}
	Nel caso in cui $ \omega_1 = \omega_2 = \omega_3 \equiv \omega $, si ha un potenziale a simmetria sferica $ \hat{V}(\hat{\ve{x}}) = \frac{1}{2} m \omega^2 \hat{\ve{x}}^2 $ e lo spettro diventa:
	\begin{equation*}
		E_{n_1 n_2 n_3} = \hbar \omega \left(n_1 + n_2 + n_3 + \frac{3}{2}\right) \equiv \hbar \omega \left(N + \frac{3}{2}\right)
	\end{equation*}
	È possibile calcolare la degenerazione dell'$ N $-esimo stato eccitato: $ n_1 $ può essere scelto in $ N + 1 $ modi, quindi $ n_2 $ può essere scelto in $ N + 1 - n_1 $ e, una volta scelti $ n_1 $ ed $ n_2 $, $ n_3 $ è fissato, dunque la degenerazione $ d(N) $ è:
	\begin{equation*}
		d(N) = \displaystyle\sum_{n_1 = 0}^{N} (N + 1 - n_1) = (N + 1)^2 - \frac{1}{2} N (N + 1) = \frac{1}{2} (N + 1) (N + 2)
	\end{equation*}
\end{example}

\section{Problema dei due corpi quantistico}

Il problema dei due corpi è un sistema in cui due corpi interagiscono tramite un potenziale che dipende solo dalla loro separazione:

\begin{equation}
	\mathcal{H} = \frac{\hat{\ve{p}}_1^2}{2m_1} + \frac{\hat{\ve{p}}_2^2}{2m_2} + \hat{V}(\hat{\ve{x}}_1 - \hat{\ve{x}}_2)
	\label{eq:17}
\end{equation}

Le variabili canoniche soddisfano la relazione di commutazione:

\begin{equation}
	\left[\hat{x}_{j,a}, \hat{p}_{k,b}\right] = i\hbar \delta_{jk} \delta_{ab} \qquad \left[\hat{x}_{j,a}, \hat{x}_{k,b}\right] = 0 \qquad \left[\hat{p}_{j,a}, \hat{p}_{k,b}\right] = 0
	\label{eq:18}
\end{equation}

dove $ a,b = 1,2 $ e $ j,k = 1,2,3 $.\\
%
Il problema è separabile definendo le coordinate relative e quelle del baricentro:

\begin{equation}
	\begin{split}
		\hat{\ve{r}} &\defeq \hat{\ve{x}}_1 - \hat{\ve{x}}_2\\
		\hat{\ve{R}} &\defeq \frac{m_1 \hat{\ve{x}}_1 + m_2 \hat{\ve{x}}_2}{m_1 + m_2}
	\end{split}
	\label{eq:19}
\end{equation}

A queste vanno associate i rispettivi impulsi congiunti:

\begin{equation}
	\begin{split}
		\hat{\ve{p}} &\defeq \frac{m_2 \hat{\ve{p}}_1 - m_1 \hat{\ve{p}}_2}{m_1 + m_2}\\
		\hat{\ve{P}} &\defeq \hat{\ve{p}}_1 + \hat{\ve{p}}_2
	\end{split}
	\label{eq:20}
\end{equation}

È pura algebra verificare che le variabili così definite soddisfino le relazioni di commutazione canoniche.\\
È altrettanto facile verificare che l'Hamiltoniana si può scrivere come:

\begin{equation}
	\mathcal{H} = \frac{\hat{\ve{P}}^2}{2M} + \frac{\hat{\ve{p}}^2}{2\mu} + \hat{V}(\hat{\ve{r}})
	\label{eq:21}
\end{equation}

dove sono state definite la massa totale $ M \equiv m_1 + m_2 $ e quella ridotta $ \mu^{-1} = m_1^{-1} + m_2^{-1} $.\\
Questa Hamiltoniana è manifestamente separabile come $ \mathcal{H} = \mathcal{H}_B(\hat{\ve{R}}, \hat{\ve{P}}) + \mathcal{H}_r(\hat{\ve{r}}, \hat{\ve{p}}) $:

\begin{equation}
	\begin{split}
		\mathcal{H}_B(\hat{\ve{R}}, \hat{\ve{P}}) &= \frac{\hat{\ve{P}}^2}{2M}\\
		\mathcal{H}_r(\hat{\ve{r}}, \hat{\ve{p}}) &= \frac{\hat{\ve{p}}^2}{2\mu} + \hat{V}(\hat{\ve{r}})
	\end{split}
	\label{eq:22}
	\qquad\qquad \left[\mathcal{H}_B, \mathcal{H}_r\right] = 0
\end{equation}

Lo spettro è facilmente determinabile poiché sono due problemi unidimensionali.\\
È importante capire che la scelta di variabili canoniche trasformate non è casuale, ma dettata dalla separabilità del termine potenziale, che fissa $ \hat{\ve{r}} $, dalle relazioni di commutazione, che per ogni scelta di $ \hat{\ve{R}} $ fissano gli impulsi coniugati, e dalla separabilità del termine cinetico che va a fissare di conseguenza $ \hat{\ve{R}} $ poiché rende univoca la scelta degli impulsi.

\subsection{Trasformazioni lineari di coordinate}

È possibile definire una generica trasformazione lineare di coordinate tramite una matrice di trasformazione $ \tens{M}\in\R^{d\times d} $:

\begin{equation}
	\hat{\ve{x}}' = \tens{M} \hat{\ve{x}}
	\label{eq:23}
\end{equation}

ovvero in componenti $ \hat{x}'_j = \sum_{k = 1}^{d} M_{jk} \hat{x}_k $.

\begin{proposition}
	Data una trasformazione lineare di coordiante $ \tens{M} $, gli impulsi coniugati trasformano secondo:
	\begin{equation}
		\hat{\ve{p}}'^{\,\intercal} = \hat{\ve{p}}^{\intercal} \tens{M}^{-1}
		\label{eq:24}
	\end{equation}
\end{proposition}
\begin{proof}
	Considerando $ \hat{\ve{p}}'^{\,\intercal} = \hat{\ve{p}}^{\intercal} \tens{N} $, in componenti $ \hat{p}'_j = \sum_{k = 1}^{d} \hat{p}_k N_{kj} $, dalle relazioni di commutazione canoniche si ha:
	\begin{equation*}
		\left[\hat{x}'_j, \hat{p}'_k\right] = \sum_{m = 1}^{d} \sum_{n = 1}^{d} M_{jm} N_{nk} \underbrace{\left[\hat{x}_m, \hat{p}_n\right]}_{= i \hbar \delta_{mn}} = i\hbar \sum_{n = 1}^{d} M_{jn} N_{nk} \,\dot{=}\, i\hbar \delta_{jk} \quad \Longleftrightarrow \quad \tens{M}\tens{N} = \tens{I}_d
	\end{equation*}
\end{proof}

È possibile ricavare la trasformazione \ref{eq:24} anche partendo dai principi, costruendo gli impulsi coniugati come generatori di traslazioni spaziali. Nella rappresentazione delle coordinate:

\begin{equation}
	\braket{\hat{\ve{x}} | \hat{\ve{p}} | \hat{\ve{x}}'} = - i\hbar \nabla_{\ve{x}}\delta(\ve{x} - \ve{x}')
	\label{eq:25}
\end{equation}

Con abuso di notazione si può scrivere $ \hat{p}_j = -i\hbar\pa_j $, dunque la relazione di trasformazione è data dalla derivata composta:

\begin{equation}
	\hat{p}'_j = -i\hbar \frac{\pa }{\pa x'_j} = -i\hbar \displaystyle\sum_{k = 1}^{d} \frac{\pa x_k}{\pa x'_j} \frac{\pa}{\pa x_k}
	\label{eq:26}
\end{equation}

Dall'Eq. \ref{eq:23} si ha $ \frac{\pa x'_j}{\pa x_k} = M_{jk} $, dunque $ \frac{\pa x_k}{\pa x'_j} = M^{-1}_{kj} $, ovvero l'Eq. \ref{eq:24}.










