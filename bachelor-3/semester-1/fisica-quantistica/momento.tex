\selectlanguage{italian}

\section{Momento angolare e rotazioni}

\paragraph{Caso classico}

Per il Th. di Noether, associate alle invarianze per rotazioni attorno ai tre assi coordianti si hanno tre cariche di Noether conservate.\\
Si considerino $ \ve{x} = \left( r\cos \varphi, r\sin \varphi \right) \equiv \left( x_1, x_2 \right) $ nel piano $ z = 0 $ ed una rotazione attorno all'asse $ z $ di un angolo infinitesimo $ \varepsilon $: questa causa uno spostamento $ \delta\ve{x} $ dato da:
\begin{equation*}
	\begin{split}
		\delta\ve{x}
		&= \left( r\cos (\varphi + \varepsilon), r\sin (\varphi + \varepsilon) \right) - \left( r\cos \varphi, r\sin \varphi \right)\\
		&= \left( -r \varepsilon \sin \varphi, r \varepsilon \cos \varphi \right) + o(\varepsilon) = \varepsilon \left( -x_2, x_1 \right) + o(\varepsilon)
	\end{split}
\end{equation*}
Quindi, per una generica rotazione attorno ad un asse dato dal versore $ \ve{n} $ si ha:
\begin{equation}
	\delta x_i = \varepsilon \sum_{j,k = 1}^{3} \epsilon_{ijk} n_j x_k \quad\Longleftrightarrow\quad \delta\ve{x} = \varepsilon \ve{n}\times\ve{x}
	\label{eq:2.1}
\end{equation}
Nel caso di una rotazione attorno al $ j $-esimo asse coordinato $ \delta x_i^{(j)} = \varepsilon \sum_{k = 1}^{3} \epsilon_{ijk} x_k $, quindi la carica di Noether associata è:
\begin{equation}
	q_j \defeq \sum_{i = 1}^{3} \frac{\pa L}{\pa \dot{x}_i} \delta x_i^{(j)} = \varepsilon \sum_{i,k = 1}^{3} \epsilon_{jki} x_k p_i = \varepsilon L_j
	\label{eq:2.2}
\end{equation}
Dunque l'invarianza per rotazioni attorno ad un asse ha come quantità conservata associata la componente del momento angolare lungo tale asse.

\paragraph{Caso quantistico}

Bisogna innanzitutto verificare che $ \hat{\ve{L}} $ definito in Eq. \ref{eq:1.31} sia effettivamente il momento angolare, ovvero il generatore delle rotazioni (a meno di un fattore $ \hbar $): questo equivale a verificare che l'operatore $ \hat{R}_{\varepsilon} $, definito come:
\begin{equation}
	\hat{R}_{\varepsilon} = e^{i\frac{\varepsilon}{\hbar} \ve{n}\cdot\hat{\ve{L}}} = \tens{I}_3 + i\frac{\varepsilon}{\hbar} \ve{n}\cdot\hat{	\ve{L}} + o(\varepsilon)
	\label{eq:2.3}
\end{equation}
realizzi una rotazione di angolo infinitesimo $ \varepsilon $ attorno all'asse $ \ve{n} $, ovvero:
\begin{equation}
	\braket{\ve{x} | \hat{R}_{\varepsilon} | \psi} = \psi(\ve{x} + \delta_{\ve{n}}\ve{x}) = \psi(\ve{x}) + \delta_{\ve{n}}\ve{x}\cdot\nabla\psi(\ve{x}) + o(\varepsilon)
	\label{eq:2.4}
\end{equation}
dove $ \delta_{\ve{n}}\ve{x} = \varepsilon \ve{n}\times\ve{x} $. Calcolando gli elementi di matrice di $ \hat{R}_{\varepsilon} $ sulla base delle posizioni:
\begin{equation}
	\braket{\ve{x} | \hat{R}_{\varepsilon} | \psi} = \psi(\ve{x}) + i \frac{\varepsilon}{\hbar} \cdot (-i\hbar) \sum_{i,j,k = 1}^{3} n_i \epsilon_{ijk} x_j \pa_k \psi(\ve{x}) + o(\varepsilon)
	\label{eq:2.5}
\end{equation}
Confontando le Eq. \ref{eq:2.4} - \ref{eq:2.5}, si vede che sono uguali, dunque $ \hat{\ve{L}} $ è il generatore delle rotazioni.

\section{Proprietà}

\subsection{Espressione esplicita}

Innanzitutto si noti che dalla definizione in Eq. \ref{eq:1.31} discende subito che $ \hat{\ve{L}} $ è hermitiano:
\begin{equation}
	\hat{L}_i^{\dagger} = \sum_{j,k = 1}^{3} \epsilon_{ijk} \hat{p}_k \hat{x}_j = \sum_{j,k = 1}^{3} \epsilon_{ijk} \left( [\hat{p}_k,\hat{x}_j] + \hat{x}_j \hat{p}_k \right) = L_i + i\hbar \sum_{j,k = 1}^{3} \epsilon_{ijk} \delta_{jk} = L_i
	\label{eq:2.6}
\end{equation}
È anche possibile calcolare esplicitamente l'espressione di $ \hat{\ve{L}} $ in coordinate sferiche:
\begin{equation}
	\hat{L}_x = i\hbar \left( \sin \varphi \frac{\pa}{\pa \vartheta} + \frac{\cos \vartheta}{\sin \vartheta} \cos \varphi \frac{\pa}{\pa \varphi} \right)
	\label{eq:2.7}
\end{equation}
\begin{equation}
	\hat{L}_y = i\hbar \left( -\cos \varphi \frac{\pa}{\pa \vartheta} + \frac{\cos \vartheta}{\sin \vartheta} \sin \varphi \frac{\pa}{\pa \varphi} \right)
	\label{eq:2.8}
\end{equation}
\begin{equation}
	L_z = -i\hbar \frac{\pa}{\pa \varphi}
	\label{eq:2.9}
\end{equation}
Si ha inoltre:
\begin{equation}
	\hat{\ve{L}}^2 \equiv \hat{L}^2 = -\hbar^2 \left( \frac{\pa^2}{\pa \vartheta^2} + \frac{\sin \vartheta}{\cos \vartheta} \frac{\pa}{\pa \vartheta} + \frac{1}{\sin^2 \vartheta} \frac{\pa^2}{\pa \varphi^2} \right)
	\label{eq:2.10}
\end{equation}

\subsection{Commutatori}

Sebbene in un sistema invariante per rotazioni il momento angolare commuti con l'Hamiltoniana, le componenti di $ \hat{\ve{L}} $ non commutano tra loro

\begin{lemma}\label{lem-l-comm}
	$ \hat{x}_i \hat{p}_j - \hat{x}_j \hat{p}_i = \sum_{k = 1}^{3} \epsilon_{ijk}	\hat{L}_k $.
\end{lemma}
\begin{proof}
	$ \sum_{k = 1}^{3} \epsilon_{ijk} \hat{L}_k = \sum_{k,a,b = 1}^{3} \epsilon_{ijk}\epsilon_{kab} \hat{x}_a \hat{p}_b = \sum_{k,a,b = 1}^{3} \left( \delta_{ia}\delta_{jb} - \delta_{ib}\delta_{ja} \right) \hat{x}_a \hat{p}_b = \hat{x}_i \hat{p}_j - \hat{x}_j \hat{p}_i $.
\end{proof}

\begin{proposition}\label{l-comm}
	$ [\hat{L}_i,\hat{L}_j] = i\hbar \sum_{k = 1}^{3} \epsilon_{ijk} \hat{L}_k $.
\end{proposition}
\begin{proof}
	Usando nell'ultima uguaglianza il Lemma \ref{lem-l-comm}:
	\begin{equation*}
		\begin{split}
			[\hat{L}_i,\hat{L}_j]
			&= \sum_{a,b,l,m = 1}^{3} \epsilon_{iab}\epsilon_{jlm} [\hat{x}_a \hat{p}_b,\hat{x}_l \hat{p}_m] = \sum_{a,b,l,m = 1}^{3}  \epsilon_{iab} \epsilon_{jlm} \left( \hat{x}_l [\hat{x}_a,\hat{p}_m] \hat{p}_b + \hat{x}_a [\hat{p}_b,\hat{x}_l] \hat{p}_m \right)\\
			&= i\hbar \sum_{a,b,l = 1}^{3} \epsilon_{bia} \epsilon_{jla} \hat{x}_l \hat{p}_b - i\hbar \sum_{a,b,m = 1}^{3} \epsilon_{iab} \epsilon_{mjb} \hat{x}_a \hat{p}_m\\
			&= i\hbar \sum_{a,b,l = 1}^{3} \left( \delta_{bj}\delta_{il} - \delta_{bl}\delta_{ji} \right) \hat{x}_l \hat{p}_b - i\hbar \sum_{a,b,m = 1}^{3} \left( \delta_{im} \delta_{aj} - \delta_{ij}\delta_{am} \right) \hat{x}_a \hat{p}_m\\
			&= i\hbar \left( \hat{x}_i \hat{p}_j - \delta_{ij} \hat{\ve{x}}\cdot\hat{\ve{p}} - \hat{x}_j \hat{p}_i + \hat{\ve{x}}\cdot\hat{\ve{p}} \delta_{ij} \right) = i\hbar \left( \hat{x}_i \hat{p}_j - \hat{x}_j \hat{p}_i \right) = i\hbar \sum_{k = 1}^{3} \epsilon_{ijk} \hat{L}_k
		\end{split}
	\end{equation*}
\end{proof}

Si ricordi che il commutatore tra un operatore hermitiano $ \hat{G} $, generatore della trasformazione (anch'essa hermitiana) $ \hat{T} = e^{i \varepsilon \hat{G}} $, ed un generico operatore $ \hat{A} $ può essere calcolato da:
\begin{equation}
	\hat{A}' = \hat{T}^{-1} \hat{A} \hat{T} = \left( \tens{I} - i\varepsilon\hat{G} \right) \hat{A} \left( \tens{I} + i\varepsilon\hat{G} \right) = \hat{A} + i\varepsilon [\hat{A},\hat{G}] \quad\Longrightarrow\quad [\hat{A},\hat{G}] = \frac{1}{i\varepsilon} \delta\hat{A}
	\label{eq:2.11}
\end{equation}
Dunque dalla Prop. \ref{l-comm} è possibile vedere come trasforma $ \hat{L}_i $ sotto la rotazione data da $ \hat{L}_j $, e confrontandola con l'Eq. \ref{eq:2.1} si vede che $ \hat{\ve{L}} $ trasforma proprio come un vettore sotto rotazioni (cosa non scontata).\\
Ciò suggerisce naturalmente che $ \hat{L}^2 $, essendo invariante per rotazioni, commuti con ciascuna $ \hat{L}_i $:
\begin{equation}
	[\hat{L}^2,\hat{L}_i] = \sum_{k = 1}^{3} [\hat{L}_k\hat{L}_k,\hat{L}_i] = i\hbar \sum_{j,k = 1}^{3} \epsilon_{kij} \left( \hat{L}_k\hat{L}_j + \hat{L}_j\hat{L}_k \right) = 0
	\label{eq:2.12}
\end{equation}
nullo poiché prodotto di simbolo completamente antisimmetrico con operatore simmetrico.

\section{Spettro del momento angolare}

Sebbene le componenti del momento angolare non commutano tra loro, e quindi non sono diagonalizzabili simultaneamente, è possibile trovare una terna di operatori compatibili: l'Hamiltoniana $ \mathcal{H} $, il modulo del momento angolare $ \hat{L}^2 $ e la componente $ \hat{L}_z $; in realtà poteva essere scelta qualsiasi componente del momento angolare, ma convenzionalmente si sceglie $ \hat{L}_z $, principalmente per la sua semplice espressione in coordinate sferiche (Eq. \ref{eq:2.9}).\\
Si definisce lo spettro di autofunzioni comuni di $ \hat{L}_z $ ed $ \hat{L}^2 $ come l'insieme di stati $ \ket{\ell,m} $ tali che:
\begin{equation}
	\hat{L}_z \ket{\ell,m} = \hbar m \ket{\ell,m}
	\label{eq:2.13}
\end{equation}
\begin{equation}
	\hat{L}^2 \ket{\ell,m} = \lambda_{\ell} \ket{\ell,m}
	\label{eq:2.14}
\end{equation}
È inoltre lecito supporre che tali stati siano normalizzati in senso proprio, dato che l'operatore momento angolare, visto come operatore differenziale, agisce su un dominio compatto (una superficie omeomorfa a $ \mathbb{S}^2 $), quindi:
\begin{equation}
	\braket{\ell',m' | \ell,m} = \delta_{\ell' \ell} \delta_{m' m}
	\label{eq:2.15}
\end{equation}

\subsection{Costruzione dello spettro}

Per determinare lo spettro, è conveniente definire i seguenti operatori:
\begin{equation}
	\hat{L}_{\pm} \defeq \hat{L}_x \pm i \hat{L}_y
	\label{eq:2.16}
\end{equation}

\begin{proposition}
	$ ( \hat{L}_{\pm} )^{\dagger} = \hat{L}_{\mp} $.
\end{proposition}
\begin{proof}
	Banale ricordando che ogni $ \hat{L}_i $ è hermitiana (Eq. \ref{eq:2.6}).
\end{proof}
\begin{proposition}
	$ [\hat{L}_z,\hat{L}_{\pm}] = \pm\hbar\hat{L}_{\pm} $.
\end{proposition}
\begin{proof}
	$ [\hat{L}_z,\hat{L}_{\pm}] = [\hat{L},\hat{L}_x] \pm i [\hat{L}_z,\hat{L}_{\pm}] = i\hbar \hat{L}_y \pm i (-i\hbar \hat{L}_x) = \pm \hbar (\hat{L}_x \pm i \hat{L}_y) = \pm\hbar\hat{L}_{\pm} $.
\end{proof}

Questi sono operatori di scala.

\begin{proposition}\label{l-ladder}
	$ \hat{L}_z\hat{L}_{\pm} \ket{\ell,m} = \hbar (m \pm 1) \hat{L}_{\pm} \ket{\ell,m} $.
\end{proposition}
\begin{proof}
	$ \hat{L}_z \hat{L}_{\pm} \ket{\ell,m} = \hat{L}_{\pm} \hat{L}_z \ket{\ell,m} + [\hat{L}_z,\hat{L}_{\pm}]\ket{\ell,m} = \hat{L}_{\pm} \hat{L}_z \ket{\ell,m} \pm \hbar \hat{L}_{\pm} \ket{\ell,m} $.
\end{proof}

Con questi operatori si dimostra che la scala degli stati si arresta in entrambe le direzioni.

\begin{proposition}
	Fissato $ \ell\in\R $, la successione $ \{ \ket{\ell,m} \}_{m\in\R} $ ha cardinalità finita.
\end{proposition}
\begin{proof}
	Si definisca $ \ket{\phi_{\pm}} \equiv \hat{L}_{\pm} \ket{\ell,m} $; naturalmente:
	\begin{equation*}
		\begin{split}
			&\braket{\phi_+ | \phi_+} = \braket{\ell,m | \hat{L}_- \hat{L}_+ | \ell,m} \ge 0\\
			&\braket{\phi_- | \phi_-} = \braket{\ell,m | \hat{L}_+ \hat{L}_- | \ell,m} \ge 0
		\end{split}
	\end{equation*}
	Considerando che:
	\begin{equation*}
		\hat{L}_{\pm} \hat{L}_{\mp} = ( \hat{L}_x \pm i\hat{L}_y ) ( \hat{L}_x \mp i\hat{L}_y ) = \hat{L}_x^2 + \hat{L}_y^2 \mp i [\hat{L}_x,\hat{L}_y] = \hat{L}^2 - \hat{L}_z^2 \pm \hbar\hat{L}_z
	\end{equation*}
	si ha:
	\begin{equation*}
		0 \le \braket{\phi_+ | \phi_+} + \braket{\phi_- | \phi_-} = 2 \braket{\ell,m | \hat{L}^2 - \hat{L}_z^2 | \ell,m} = \lambda_{\ell}^2 - \hbar^2 m^2
	\end{equation*}
	Si ha quindi:
	\begin{equation*}
		- \frac{\abs{\lambda_{\ell}}}{\hbar} \le m \le \frac{\abs{\lambda_{\ell}}}{\hbar}
	\end{equation*}
	che dimostra la tesi.
\end{proof}
Dunque, per $ \ell $ fissato devono esistere degli stati limite $ \ket{\ell,m_{\text{min}}} $, $ \ket{\ell,m_{\text{max}}} $ tali che:
\begin{equation}
	\begin{split}
		&\hat{L}_- \ket{\ell,m_{\text{min}}} = 0 \\
		&\hat{L}_+ \ket{\ell,m_{\text{max}}} = 0
	\end{split}
	\label{eq:2.17}
\end{equation}
Ciò determina univocamente i valori ammessi sia di $ \lambda_{\ell} $ che di $ m $. Infatti, applicando $ \hat{L}_{\pm} $ alle Eq. \ref{eq:2.17}:
\begin{equation*}
	\begin{split}
		&0 = \hat{L}_+ \hat{L}_- \ket{\ell,m_{\text{min}}} = (\hat{L}^2 - \hat{L}_z^2 + \hbar\hat{L}_z)\ket{\ell,m_{\text{min}}} = (\lambda_{\ell} - \hbar^2 m_{\text{min}}^2 + \hbar^2 m_{\text{min}}) \ket{\ell,m_{\text{min}}}\\
		&0 = \hat{L}_- \hat{L}_+ \ket{\ell,m_{\text{max}}} = (\hat{L}^2 - \hat{L}_z^2 - \hbar\hat{L}_z)\ket{\ell,m_{\text{max}}} = (\lambda_{\ell} - \hbar^2 m_{\text{max}}^2 - \hbar^2 m_{\text{max}}) \ket{\ell,m_{\text{max}}}
	\end{split}
\end{equation*}
Sottraendo le due equazioni si ottiene:
\begin{equation*}
	m_{\text{max}} (m_{\text{max}} + 1) - m_{\text{min}} (m_{\text{min}} - 1) = 0
\end{equation*}
le cui soluzioni sono
\begin{equation*}
	m_{\text{max}} = \frac{-1 \pm (2m_{\text{min}} - 1)}{2} \in \{ m_{\text{min}} - 2, -m_{\text{min}} \}
\end{equation*}
L'unica soluzione sensata è $ m_{\text{max}} = -m_{\text{min}} $. Dato che $ \hat{L}_{\pm} $ sono operatori di scala, si deve avere $ m_{\text{max}} = m_{\text{min}} + N, \, N\in\N $, dunque
\begin{equation*}
	m_{\text{max}} = \frac{N}{2}
\end{equation*}
che può essere intero o semi-intero.\\
Si ha inoltre che $ \lambda_{\ell} = \hbar^2 m_{\text{max}} (m_{\text{max}} + 1) $, quindi, definendo $ \ell \equiv \frac{N}{2} $ (fin'ora era arbitrario), si può scrivere:
\begin{equation*}
	\lambda_{\ell} = \hbar^2 \ell (\ell + 1)
\end{equation*}
In definitiva:
\begin{equation}
	\hat{L}^2 \ket{\ell,m} = \hbar^2 \ell (\ell + 1) \ket{\ell,m} \qquad \ell = \frac{N}{2},\,N\in\N
	\label{eq:2.18}
\end{equation}
\begin{equation}
	\hat{L}_z \ket{\ell,m} = \hbar m \ket{\ell,m} \qquad -\ell \le m \le \ell
	\label{eq:2.19}
\end{equation}
La normalizzazione propria degli stati è data da:
\begin{equation}
	\braket{\ell',m' | \ell,m} = \delta_{\ell'\ell} \delta_{m'm}
	\label{eq:2.20}
\end{equation}
Per normalizzare correttamente le autofunzioni, si calcola:
\begin{equation*}
	\begin{split}
		&\hat{L}_+\hat{L}_- \ket{\ell,m} = \hbar^2 (\ell(\ell + 1) - m(m - 1)) \ket{\ell,m}\\
		&\hat{L}_-\hat{L}_+ \ket{\ell,m} = \hbar^2 (\ell(\ell + 1) - m(m + 1)) \ket{\ell,m}
	\end{split}
\end{equation*}
Ricordando la Prop. \ref{l-ladder}, si può definire pienamente la scala delle autofunzioni a $ \ell $ fissato:
\begin{equation}
	\ket{\ell,m \pm 1} = \frac{1}{\hbar \sqrt{\ell(\ell + 1) - m(m \pm 1)}} \hat{L}_{\pm} \ket{\ell,m}
	\label{eq:2.21}
\end{equation}

\paragraph{Quantizzazione}

È necessario fare delle osservazioni sugli autovalori di $ \hat{L}_z $ e $ \hat{L}^2 $ trovati. Innanzitutto, si vede che $ L_z $ è quantizzato in multipli interi o semi-interi di $ \hbar $, il che è alquanto notevole, soprattutto se comparato alla fisica classica: quantisticamente, quindi, sebbene non abbia senso parlare di traiettorie ($ \hat{x} $ e $ \hat{p} $ non commutano), si possono descrivere le orbite quantizzate dei corpi, corrispondendi a valori discreti del momento angolare.\\
Un'altro fatto che viene confermato è che le componenti del momento angolare non sono compatibili tra loro: se è completamente determinata $ L_z $, $ L_x $ ed $ L_y $ sono completamente indeterminate (e quindi non avrebbe senso parlare di un vettore tridimensionale); ciò è confermato dal fatto che, quando $ L_z $ assume il suo valore massimo $ \hbar \ell $, questo è comunque strettamente minore del modulo del momento angolare $ \hbar \sqrt{\ell(\ell + 1)} $ (mentre classicamente si avrebbe $ L_z^{(\text{max})} = L $).

\subsection{Autofunzioni sulla base delle coordinate}

È possibile definire le autofunzioni del momento angolare sulla base delle coordinate come:
\begin{equation}
	\braket{\vartheta,\varphi | \ell,m} \defeq Y_{\ell,m}(\vartheta,\varphi)
	\label{eq:2.22}
\end{equation}
Ricordando l'espressione di $ \hat{L}_z $ sulla base delle coordiante (Eq. \ref{eq:2.9}), l'Eq. \ref{eq:2.19} diventa un'equazione differenziale:
\begin{equation}
	-i\hbar \frac{\pa}{\pa \varphi} Y_{\ell,m}(\vartheta,\varphi) = \hbar m Y_{\ell,m}(\vartheta,\varphi)
	\label{eq:2.23}
\end{equation}
È possibile scrivere la soluzione generale come:
\begin{equation}
	Y_{\ell,m}(\vartheta,\varphi) = \mathcal{N}_{\ell,m} e^{im\varphi} P_{\ell,m}(\cos\vartheta)
	\label{eq:2.24}
\end{equation}










