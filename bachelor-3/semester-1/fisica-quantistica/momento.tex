\selectlanguage{italian}

\section{Momento angolare e rotazioni}

\paragraph{Caso classico}

Per il Th. di Noether, associate alle invarianze per rotazioni attorno ai tre assi coordianti si hanno tre cariche di Noether conservate.\\
Si considerino $ \ve{x} = \left( r\cos \phi, r\sin \phi \right) \equiv \left( x_1, x_2 \right) $ nel piano $ z = 0 $ ed una rotazione attorno all'asse $ z $ di un angolo infinitesimo $ \varepsilon $: questa causa uno spostamento $ \delta\ve{x} $ dato da:
\begin{equation*}
	\begin{split}
		\delta\ve{x}
		&= \left( r\cos (\phi + \varepsilon), r\sin (\phi + \varepsilon) \right) - \left( r\cos \phi, r\sin \phi \right)\\
		&= \left( -r \varepsilon \sin \phi, r \varepsilon \cos \phi \right) + o(\varepsilon) = \varepsilon \left( -x_2, x_1 \right) + o(\varepsilon)
	\end{split}
\end{equation*}
Quindi, per una generica rotazione attorno ad un asse dato dal versore $ \ve{n} $ si ha:
\begin{equation}
	\delta x_i = \varepsilon \sum_{j,k = 1}^{3} \epsilon_{ijk} n_j x_k \quad\Longleftrightarrow\quad \delta\ve{x} = \varepsilon \ve{n}\times\ve{x}
	\label{eq:2.1}
\end{equation}
Nel caso di una rotazione attorno al $ j $-esimo asse coordinato $ \delta x_i^{(j)} = \varepsilon \sum_{k = 1}^{3} \epsilon_{ijk} x_k $, quindi la carica di Noether associata è:
\begin{equation}
	q_j \defeq \sum_{i = 1}^{3} \frac{\pa L}{\pa \dot{x}_i} \delta x_i^{(j)} = \varepsilon \sum_{i,k = 1}^{3} \epsilon_{jki} x_k p_i = \varepsilon L_j
	\label{eq:2.2}
\end{equation}
Dunque l'invarianza per rotazioni attorno ad un asse ha come quantità conservata associata la componente del momento angolare lungo tale asse.

\paragraph{Caso quantistico}

Bisogna innanzitutto verificare che $ \hat{\ve{L}} $ definito in Eq. \ref{eq:1.31} sia effettivamente il momento angolare, ovvero il generatore delle rotazioni (a meno di un fattore $ \hbar $): questo equivale a verificare che l'operatore $ \hat{R}_{\varepsilon} $, definito come:
\begin{equation}
	\hat{R}_{\varepsilon} = e^{i\frac{\varepsilon}{\hbar} \ve{n}\cdot\hat{\ve{L}}} = \tens{I}_3 + i\frac{\varepsilon}{\hbar} \ve{n}\cdot\hat{	\ve{L}} + o(\varepsilon)
	\label{eq:2.3}
\end{equation}
realizzi una rotazione di angolo infinitesimo $ \varepsilon $ attorno all'asse $ \ve{n} $, ovvero:
\begin{equation}
	\braket{\ve{x} | \hat{R}_{\varepsilon} | \psi} = \psi(\ve{x} + \delta_{\ve{n}}\ve{x}) = \psi(\ve{x}) + \delta_{\ve{n}}\ve{x}\cdot\nabla\psi(\ve{x}) + o(\varepsilon)
	\label{eq:2.4}
\end{equation}
dove $ \delta_{\ve{n}}\ve{x} = \varepsilon \ve{n}\times\ve{x} $. Calcolando gli elementi di matrice di $ \hat{R}_{\varepsilon} $ sulla base delle posizioni:
\begin{equation}
	\braket{\ve{x} | \hat{R}_{\varepsilon} | \psi} = \psi(\ve{x}) + i \frac{\varepsilon}{\hbar} \cdot (-i\hbar) \sum_{i,j,k = 1}^{3} n_i \epsilon_{ijk} x_j \pa_k \psi(\ve{x}) + o(\varepsilon)
	\label{eq:2.5}
\end{equation}
Confontando le Eq. \ref{eq:2.4} - \ref{eq:2.5}, si vede che sono uguali, dunque $ \hat{\ve{L}} $ è il generatore delle rotazioni.

\section{Proprietà}

\subsection{Espressione esplicita}

Innanzitutto si noti che dalla definizione in Eq. \ref{eq:1.31} discende subito che $ \hat{\ve{L}} $ è hermitiano:
\begin{equation}
	\hat{L}_i^{\dagger} = \sum_{j,k = 1}^{3} \epsilon_{ijk} \hat{p}_k \hat{x}_j = \sum_{j,k = 1}^{3} \epsilon_{ijk} \left( [\hat{p}_k,\hat{x}_j] + \hat{x}_j \hat{p}_k \right) = L_i + i\hbar \sum_{j,k = 1}^{3} \epsilon_{ijk} \delta_{jk} = L_i
	\label{eq:2.6}
\end{equation}
È anche possibile calcolare esplicitamente l'espressione di $ \hat{\ve{L}} $ in coordinate sferiche:
\begin{equation}
	\hat{L}_x = i\hbar \left( \sin \phi \frac{\pa}{\pa \theta} + \frac{\cos \theta}{\sin \theta} \cos \phi \frac{\pa}{\pa \phi} \right)
	\label{eq:2.7}
\end{equation}
\begin{equation}
	\hat{L}_y = i\hbar \left( -\cos \phi \frac{\pa}{\pa \theta} + \frac{\cos \theta}{\sin \theta} \sin \phi \frac{\pa}{\pa \phi} \right)
	\label{eq:2.8}
\end{equation}
\begin{equation}
	L_z = -i\hbar \frac{\pa}{\pa \phi}
	\label{eq:2.9}
\end{equation}
Si ha inoltre:
\begin{equation}
	\hat{\ve{L}}^2 \equiv \hat{L}^2 = -\hbar^2 \left( \frac{\pa^2}{\pa \theta^2} + \frac{\sin \theta}{\cos \theta} \frac{\pa}{\pa \theta} + \frac{1}{\sin^2 \theta} \frac{\pa^2}{\pa \phi^2} \right)
	\label{eq:2.10}
\end{equation}

\subsection{Commutatori}

Sebbene in un sistema invariante per rotazioni il momento angolare commuti con l'Hamiltoniana, le componenti di $ \hat{\ve{L}} $ non commutano tra loro

\begin{lemma}\label{lem-l-comm}
	$ \hat{x}_i \hat{p}_j - \hat{x}_j \hat{p}_i = \sum_{k = 1}^{3} \epsilon_{ijk}	\hat{L}_k $.
\end{lemma}
\begin{proof}
	$ \sum_{k = 1}^{3} \epsilon_{ijk} \hat{L}_k = \sum_{k,a,b = 1}^{3} \epsilon_{ijk}\epsilon_{kab} \hat{x}_a \hat{p}_b = \sum_{k,a,b = 1}^{3} \left( \delta_{ia}\delta_{jb} - \delta_{ib}\delta_{ja} \right) \hat{x}_a \hat{p}_b = \hat{x}_i \hat{p}_j - \hat{x}_j \hat{p}_i $.
\end{proof}

\begin{proposition}\label{l-comm}
	$ [\hat{L}_i,\hat{L}_j] = i\hbar \sum_{k = 1}^{3} \epsilon_{ijk} \hat{L}_k $.
\end{proposition}
\begin{proof}
	Usando nell'ultima uguaglianza il Lemma \ref{lem-l-comm}:
	\begin{equation*}
		\begin{split}
			[\hat{L}_i,\hat{L}_j]
			&= \sum_{a,b,l,m = 1}^{3} \epsilon_{iab}\epsilon_{jlm} [\hat{x}_a \hat{p}_b,\hat{x}_l \hat{p}_m] = \sum_{a,b,l,m = 1}^{3}  \epsilon_{iab} \epsilon_{jlm} \left( \hat{x}_l [\hat{x}_a,\hat{p}_m] \hat{p}_b + \hat{x}_a [\hat{p}_b,\hat{x}_l] \hat{p}_m \right)\\
			&= i\hbar \sum_{a,b,l = 1}^{3} \epsilon_{bia} \epsilon_{jla} \hat{x}_l \hat{p}_b - i\hbar \sum_{a,b,m = 1}^{3} \epsilon_{iab} \epsilon_{mjb} \hat{x}_a \hat{p}_m\\
			&= i\hbar \sum_{a,b,l = 1}^{3} \left( \delta_{bj}\delta_{il} - \delta_{bl}\delta_{ji} \right) \hat{x}_l \hat{p}_b - i\hbar \sum_{a,b,m = 1}^{3} \left( \delta_{im} \delta_{aj} - \delta_{ij}\delta_{am} \right) \hat{x}_a \hat{p}_m\\
			&= i\hbar \left( \hat{x}_i \hat{p}_j - \delta_{ij} \hat{\ve{x}}\cdot\hat{\ve{p}} - \hat{x}_j \hat{p}_i + \hat{\ve{x}}\cdot\hat{\ve{p}} \delta_{ij} \right) = i\hbar \left( \hat{x}_i \hat{p}_j - \hat{x}_j \hat{p}_i \right) = i\hbar \sum_{k = 1}^{3} \epsilon_{ijk} \hat{L}_k
		\end{split}
	\end{equation*}
\end{proof}

Si ricordi che il commutatore tra un operatore hermitiano $ \hat{G} $, generatore della trasformazione (anch'essa hermitiana) $ \hat{T} = e^{i \varepsilon \hat{G}} $, ed un generico operatore $ \hat{A} $ può essere calcolato da:
\begin{equation}
	\hat{A}' = \hat{T}^{-1} \hat{A} \hat{T} = \left( \tens{I} - i\varepsilon\hat{G} \right) \hat{A} \left( \tens{I} + i\varepsilon\hat{G} \right) = \hat{A} + i\varepsilon [\hat{A},\hat{G}] \quad\Longrightarrow\quad [\hat{A},\hat{G}] = \frac{1}{i\varepsilon} \delta\hat{A}
	\label{eq:2.11}
\end{equation}
Dunque dalla Prop. \ref{l-comm} è possibile vedere come trasforma $ \hat{L}_i $ sotto la rotazione data da $ \hat{L}_j $, e confrontandola con l'Eq. \ref{eq:2.1} si vede che $ \hat{\ve{L}} $ trasforma proprio come un vettore sotto rotazioni (cosa non scontata).\\
Ciò suggerisce naturalmente che $ \hat{L}^2 $, essendo invariante per rotazioni, commuti con ciascuna $ \hat{L}_i $:
\begin{equation}
	[\hat{L}^2,\hat{L}_i] = \sum_{k = 1}^{3} [\hat{L}_k\hat{L}_k,\hat{L}_i] = i\hbar \sum_{j,k = 1}^{3} \epsilon_{kij} \left( \hat{L}_k\hat{L}_j + \hat{L}_j\hat{L}_k \right) = 0
	\label{eq:2.12}
\end{equation}
nullo poiché prodotto di simbolo completamente antisimmetrico con operatore simmetrico.










