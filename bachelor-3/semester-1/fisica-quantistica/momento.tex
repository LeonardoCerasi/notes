\selectlanguage{italian}

\section{Momento angolare e rotazioni}

\paragraph{Caso classico}

Per il Th. di Noether, associate alle invarianze per rotazioni attorno ai tre assi coordianti si hanno tre cariche di Noether conservate.\\
Si considerino $ \ve{x} = \left( r\cos \varphi, r\sin \varphi \right) \equiv \left( x_1, x_2 \right) $ nel piano $ z = 0 $ ed una rotazione attorno all'asse $ z $ di un angolo infinitesimo $ \varepsilon $: questa causa uno spostamento $ \delta\ve{x} $ dato da:
\begin{equation*}
	\begin{split}
		\delta\ve{x}
		&= \left( r\cos (\varphi + \varepsilon), r\sin (\varphi + \varepsilon) \right) - \left( r\cos \varphi, r\sin \varphi \right)\\
		&= \left( -r \varepsilon \sin \varphi, r \varepsilon \cos \varphi \right) + o(\varepsilon) = \varepsilon \left( -x_2, x_1 \right) + o(\varepsilon)
	\end{split}
\end{equation*}
Quindi, per una generica rotazione attorno ad un asse dato dal versore $ \ve{n} $ si ha:
\begin{equation}
	\delta x_i = \varepsilon \sum_{j,k = 1}^{3} \epsilon_{ijk} n_j x_k \quad\Longleftrightarrow\quad \delta\ve{x} = \varepsilon \ve{n}\times\ve{x}
	\label{eq:2.1}
\end{equation}
Nel caso di una rotazione attorno al $ j $-esimo asse coordinato $ \delta x_i^{(j)} = \varepsilon \sum_{k = 1}^{3} \epsilon_{ijk} x_k $, quindi la carica di Noether associata è:
\begin{equation}
	q_j \defeq \sum_{i = 1}^{3} \frac{\pa L}{\pa \dot{x}_i} \delta x_i^{(j)} = \varepsilon \sum_{i,k = 1}^{3} \epsilon_{jki} x_k p_i = \varepsilon L_j
	\label{eq:2.2}
\end{equation}
Dunque l'invarianza per rotazioni attorno ad un asse ha come quantità conservata associata la componente del momento angolare lungo tale asse.

\paragraph{Caso quantistico}

Bisogna innanzitutto verificare che $ \hat{\ve{L}} $ definito in Eq. \ref{eq:1.31} sia effettivamente il momento angolare, ovvero il generatore delle rotazioni (a meno di un fattore $ \hbar $): questo equivale a verificare che l'operatore $ \hat{R}_{\varepsilon} $, definito come:
\begin{equation}
	\hat{R}_{\varepsilon} = e^{i\frac{\varepsilon}{\hbar} \ve{n}\cdot\hat{\ve{L}}} = \tens{I}_3 + i\frac{\varepsilon}{\hbar} \ve{n}\cdot\hat{	\ve{L}} + o(\varepsilon)
	\label{eq:2.3}
\end{equation}
realizzi una rotazione di angolo infinitesimo $ \varepsilon $ attorno all'asse $ \ve{n} $, ovvero:
\begin{equation}
	\braket{\ve{x} | \hat{R}_{\varepsilon} | \psi} = \psi(\ve{x} + \delta_{\ve{n}}\ve{x}) = \psi(\ve{x}) + \delta_{\ve{n}}\ve{x}\cdot\nabla\psi(\ve{x}) + o(\varepsilon)
	\label{eq:2.4}
\end{equation}
dove $ \delta_{\ve{n}}\ve{x} = \varepsilon \ve{n}\times\ve{x} $. Calcolando gli elementi di matrice di $ \hat{R}_{\varepsilon} $ sulla base delle posizioni:
\begin{equation}
	\braket{\ve{x} | \hat{R}_{\varepsilon} | \psi} = \psi(\ve{x}) + i \frac{\varepsilon}{\hbar} \cdot (-i\hbar) \sum_{i,j,k = 1}^{3} n_i \epsilon_{ijk} x_j \pa_k \psi(\ve{x}) + o(\varepsilon)
	\label{eq:2.5}
\end{equation}
Confontando le Eq. \ref{eq:2.4} - \ref{eq:2.5}, si vede che sono uguali, dunque $ \hat{\ve{L}} $ è il generatore delle rotazioni.

\section{Proprietà}

\subsection{Espressione esplicita}

Innanzitutto si noti che dalla definizione in Eq. \ref{eq:1.31} discende subito che $ \hat{\ve{L}} $ è hermitiano:
\begin{equation}
	\hat{L}_i^{\dagger} = \sum_{j,k = 1}^{3} \epsilon_{ijk} \hat{p}_k \hat{x}_j = \sum_{j,k = 1}^{3} \epsilon_{ijk} \left( [\hat{p}_k,\hat{x}_j] + \hat{x}_j \hat{p}_k \right) = L_i + i\hbar \sum_{j,k = 1}^{3} \epsilon_{ijk} \delta_{jk} = L_i
	\label{eq:2.6}
\end{equation}
È anche possibile calcolare esplicitamente l'espressione di $ \hat{\ve{L}} $ in coordinate sferiche:
\begin{equation}
	\hat{L}_x = i\hbar \left( \sin \varphi \frac{\pa}{\pa \vartheta} + \frac{\cos \vartheta}{\sin \vartheta} \cos \varphi \frac{\pa}{\pa \varphi} \right)
	\label{eq:2.7}
\end{equation}
\begin{equation}
	\hat{L}_y = i\hbar \left( -\cos \varphi \frac{\pa}{\pa \vartheta} + \frac{\cos \vartheta}{\sin \vartheta} \sin \varphi \frac{\pa}{\pa \varphi} \right)
	\label{eq:2.8}
\end{equation}
\begin{equation}
	L_z = -i\hbar \frac{\pa}{\pa \varphi}
	\label{eq:2.9}
\end{equation}
Si ha inoltre:
\begin{equation}
	\hat{\ve{L}}^2 \equiv \hat{L}^2 = -\hbar^2 \left( \frac{\pa^2}{\pa \vartheta^2} + \frac{\sin \vartheta}{\cos \vartheta} \frac{\pa}{\pa \vartheta} + \frac{1}{\sin^2 \vartheta} \frac{\pa^2}{\pa \varphi^2} \right)
	\label{eq:2.10}
\end{equation}

\subsection{Commutatori}

Sebbene in un sistema invariante per rotazioni il momento angolare commuti con l'Hamiltoniana, le componenti di $ \hat{\ve{L}} $ non commutano tra loro

\begin{lemma}\label{lem-l-comm}
	$ \hat{x}_i \hat{p}_j - \hat{x}_j \hat{p}_i = \sum_{k = 1}^{3} \epsilon_{ijk}	\hat{L}_k $.
\end{lemma}
\begin{proof}
	$ \sum_{k = 1}^{3} \epsilon_{ijk} \hat{L}_k = \sum_{k,a,b = 1}^{3} \epsilon_{ijk}\epsilon_{kab} \hat{x}_a \hat{p}_b = \sum_{k,a,b = 1}^{3} \left( \delta_{ia}\delta_{jb} - \delta_{ib}\delta_{ja} \right) \hat{x}_a \hat{p}_b = \hat{x}_i \hat{p}_j - \hat{x}_j \hat{p}_i $.
\end{proof}

\begin{proposition}\label{l-comm}
	$ [\hat{L}_i,\hat{L}_j] = i\hbar \sum_{k = 1}^{3} \epsilon_{ijk} \hat{L}_k $.
\end{proposition}
\begin{proof}
	Usando nell'ultima uguaglianza il Lemma \ref{lem-l-comm}:
	\begin{equation*}
		\begin{split}
			[\hat{L}_i,\hat{L}_j]
			&= \sum_{a,b,l,m = 1}^{3} \epsilon_{iab}\epsilon_{jlm} [\hat{x}_a \hat{p}_b,\hat{x}_l \hat{p}_m] = \sum_{a,b,l,m = 1}^{3}  \epsilon_{iab} \epsilon_{jlm} \left( \hat{x}_l [\hat{x}_a,\hat{p}_m] \hat{p}_b + \hat{x}_a [\hat{p}_b,\hat{x}_l] \hat{p}_m \right)\\
			&= i\hbar \sum_{a,b,l = 1}^{3} \epsilon_{bia} \epsilon_{jla} \hat{x}_l \hat{p}_b - i\hbar \sum_{a,b,m = 1}^{3} \epsilon_{iab} \epsilon_{mjb} \hat{x}_a \hat{p}_m\\
			&= i\hbar \sum_{a,b,l = 1}^{3} \left( \delta_{bj}\delta_{il} - \delta_{bl}\delta_{ji} \right) \hat{x}_l \hat{p}_b - i\hbar \sum_{a,b,m = 1}^{3} \left( \delta_{im} \delta_{aj} - \delta_{ij}\delta_{am} \right) \hat{x}_a \hat{p}_m\\
			&= i\hbar \left( \hat{x}_i \hat{p}_j - \delta_{ij} \hat{\ve{x}}\cdot\hat{\ve{p}} - \hat{x}_j \hat{p}_i + \hat{\ve{x}}\cdot\hat{\ve{p}} \delta_{ij} \right) = i\hbar \left( \hat{x}_i \hat{p}_j - \hat{x}_j \hat{p}_i \right) = i\hbar \sum_{k = 1}^{3} \epsilon_{ijk} \hat{L}_k
		\end{split}
	\end{equation*}
\end{proof}

Si ricordi che il commutatore tra un operatore hermitiano $ \hat{G} $, generatore della trasformazione (anch'essa hermitiana) $ \hat{T} = e^{i \varepsilon \hat{G}} $, ed un generico operatore $ \hat{A} $ può essere calcolato da:
\begin{equation}
	\hat{A}' = \hat{T}^{-1} \hat{A} \hat{T} = \left( \tens{I} - i\varepsilon\hat{G} \right) \hat{A} \left( \tens{I} + i\varepsilon\hat{G} \right) = \hat{A} + i\varepsilon [\hat{A},\hat{G}] \quad\Longrightarrow\quad [\hat{A},\hat{G}] = \frac{1}{i\varepsilon} \delta\hat{A}
	\label{eq:2.11}
\end{equation}
Dunque dalla Prop. \ref{l-comm} è possibile vedere come trasforma $ \hat{L}_i $ sotto la rotazione data da $ \hat{L}_j $, e confrontandola con l'Eq. \ref{eq:2.1} si vede che $ \hat{\ve{L}} $ trasforma proprio come un vettore sotto rotazioni (cosa non scontata).\\
Ciò suggerisce naturalmente che $ \hat{L}^2 $, essendo invariante per rotazioni, commuti con ciascuna $ \hat{L}_i $:
\begin{equation}
	[\hat{L}^2,\hat{L}_i] = \sum_{k = 1}^{3} [\hat{L}_k\hat{L}_k,\hat{L}_i] = i\hbar \sum_{j,k = 1}^{3} \epsilon_{kij} \left( \hat{L}_k\hat{L}_j + \hat{L}_j\hat{L}_k \right) = 0
	\label{eq:2.12}
\end{equation}
nullo poiché prodotto di simbolo completamente antisimmetrico con operatore simmetrico.

\section{Spettro del momento angolare}

Sebbene le componenti del momento angolare non commutano tra loro, e quindi non sono diagonalizzabili simultaneamente, è possibile trovare una terna di operatori compatibili: l'Hamiltoniana $ \mathcal{H} $, il modulo del momento angolare $ \hat{L}^2 $ e la componente $ \hat{L}_z $; in realtà poteva essere scelta qualsiasi componente del momento angolare, ma convenzionalmente si sceglie $ \hat{L}_z $, principalmente per la sua semplice espressione in coordinate sferiche (Eq. \ref{eq:2.9}).\\
Si definisce lo spettro di autofunzioni comuni di $ \hat{L}_z $ ed $ \hat{L}^2 $ come l'insieme di stati $ \ket{\ell,m} $ tali che:
\begin{equation}
	\hat{L}_z \ket{\ell,m} = \hbar m \ket{\ell,m}
	\label{eq:2.13}
\end{equation}
\begin{equation}
	\hat{L}^2 \ket{\ell,m} = \lambda_{\ell} \ket{\ell,m}
	\label{eq:2.14}
\end{equation}
È inoltre lecito supporre che tali stati siano normalizzati in senso proprio, dato che l'operatore momento angolare, visto come operatore differenziale, agisce su un dominio compatto (una superficie omeomorfa a $ \mathbb{S}^2 $), quindi:
\begin{equation}
	\braket{\ell',m' | \ell,m} = \delta_{\ell' \ell} \delta_{m' m}
	\label{eq:2.15}
\end{equation}

\subsection{Costruzione dello spettro}

Per determinare lo spettro, è conveniente definire i seguenti operatori:
\begin{equation}
	\hat{L}_{\pm} \defeq \hat{L}_x \pm i \hat{L}_y
	\label{eq:2.16}
\end{equation}

\begin{proposition}
	$ ( \hat{L}_{\pm} )^{\dagger} = \hat{L}_{\mp} $.
\end{proposition}
\begin{proof}
	Banale ricordando che ogni $ \hat{L}_i $ è hermitiana (Eq. \ref{eq:2.6}).
\end{proof}
\begin{proposition}
	$ [\hat{L}_z,\hat{L}_{\pm}] = \pm\hbar\hat{L}_{\pm} $.
\end{proposition}
\begin{proof}
	$ [\hat{L}_z,\hat{L}_{\pm}] = [\hat{L},\hat{L}_x] \pm i [\hat{L}_z,\hat{L}_{\pm}] = i\hbar \hat{L}_y \pm i (-i\hbar \hat{L}_x) = \pm \hbar (\hat{L}_x \pm i \hat{L}_y) = \pm\hbar\hat{L}_{\pm} $.
\end{proof}

Questi sono operatori di scala.

\begin{proposition}\label{l-ladder}
	$ \hat{L}_z\hat{L}_{\pm} \ket{\ell,m} = \hbar (m \pm 1) \hat{L}_{\pm} \ket{\ell,m} $.
\end{proposition}
\begin{proof}
	$ \hat{L}_z \hat{L}_{\pm} \ket{\ell,m} = \hat{L}_{\pm} \hat{L}_z \ket{\ell,m} + [\hat{L}_z,\hat{L}_{\pm}]\ket{\ell,m} = \hat{L}_{\pm} \hat{L}_z \ket{\ell,m} \pm \hbar \hat{L}_{\pm} \ket{\ell,m} $.
\end{proof}

Con questi operatori si dimostra che la scala degli stati si arresta in entrambe le direzioni.

\begin{proposition}
	Fissato $ \ell\in\R $, la successione $ \{ \ket{\ell,m} \}_{m\in\R} $ ha cardinalità finita.
\end{proposition}
\begin{proof}
	Si definisca $ \ket{\phi_{\pm}} \equiv \hat{L}_{\pm} \ket{\ell,m} $; naturalmente:
	\begin{equation*}
		\begin{split}
			&\braket{\phi_+ | \phi_+} = \braket{\ell,m | \hat{L}_- \hat{L}_+ | \ell,m} \ge 0\\
			&\braket{\phi_- | \phi_-} = \braket{\ell,m | \hat{L}_+ \hat{L}_- | \ell,m} \ge 0
		\end{split}
	\end{equation*}
	Considerando che:
	\begin{equation*}
		\hat{L}_{\pm} \hat{L}_{\mp} = ( \hat{L}_x \pm i\hat{L}_y ) ( \hat{L}_x \mp i\hat{L}_y ) = \hat{L}_x^2 + \hat{L}_y^2 \mp i [\hat{L}_x,\hat{L}_y] = \hat{L}^2 - \hat{L}_z^2 \pm \hbar\hat{L}_z
	\end{equation*}
	si ha:
	\begin{equation*}
		0 \le \braket{\phi_+ | \phi_+} + \braket{\phi_- | \phi_-} = 2 \braket{\ell,m | \hat{L}^2 - \hat{L}_z^2 | \ell,m} = \lambda_{\ell}^2 - \hbar^2 m^2
	\end{equation*}
	Si ha quindi:
	\begin{equation*}
		- \frac{\abs{\lambda_{\ell}}}{\hbar} \le m \le \frac{\abs{\lambda_{\ell}}}{\hbar}
	\end{equation*}
	che dimostra la tesi.
\end{proof}
Dunque, per $ \ell $ fissato devono esistere degli stati limite $ \ket{\ell,m_{\text{min}}} $, $ \ket{\ell,m_{\text{max}}} $ tali che:
\begin{equation}
	\begin{split}
		&\hat{L}_- \ket{\ell,m_{\text{min}}} = 0 \\
		&\hat{L}_+ \ket{\ell,m_{\text{max}}} = 0
	\end{split}
	\label{eq:2.17}
\end{equation}
Ciò determina univocamente i valori ammessi sia di $ \lambda_{\ell} $ che di $ m $. Infatti, applicando $ \hat{L}_{\pm} $ alle Eq. \ref{eq:2.17}:
\begin{equation*}
	\begin{split}
		&0 = \hat{L}_+ \hat{L}_- \ket{\ell,m_{\text{min}}} = (\hat{L}^2 - \hat{L}_z^2 + \hbar\hat{L}_z)\ket{\ell,m_{\text{min}}} = (\lambda_{\ell} - \hbar^2 m_{\text{min}}^2 + \hbar^2 m_{\text{min}}) \ket{\ell,m_{\text{min}}}\\
		&0 = \hat{L}_- \hat{L}_+ \ket{\ell,m_{\text{max}}} = (\hat{L}^2 - \hat{L}_z^2 - \hbar\hat{L}_z)\ket{\ell,m_{\text{max}}} = (\lambda_{\ell} - \hbar^2 m_{\text{max}}^2 - \hbar^2 m_{\text{max}}) \ket{\ell,m_{\text{max}}}
	\end{split}
\end{equation*}
Sottraendo le due equazioni si ottiene:
\begin{equation*}
	m_{\text{max}} (m_{\text{max}} + 1) - m_{\text{min}} (m_{\text{min}} - 1) = 0
\end{equation*}
le cui soluzioni sono
\begin{equation*}
	m_{\text{max}} = \frac{-1 \pm (2m_{\text{min}} - 1)}{2} \in \{ m_{\text{min}} - 2, -m_{\text{min}} \}
\end{equation*}
L'unica soluzione sensata è $ m_{\text{max}} = -m_{\text{min}} $. Dato che $ \hat{L}_{\pm} $ sono operatori di scala, si deve avere $ m_{\text{max}} = m_{\text{min}} + N, \, N\in\N $, dunque
\begin{equation*}
	m_{\text{max}} = \frac{N}{2}
\end{equation*}
che può essere intero o semi-intero.\\
Si ha inoltre che $ \lambda_{\ell} = \hbar^2 m_{\text{max}} (m_{\text{max}} + 1) $, quindi, definendo $ \ell \equiv \frac{N}{2} $ (fin'ora era arbitrario), si può scrivere:
\begin{equation*}
	\lambda_{\ell} = \hbar^2 \ell (\ell + 1)
\end{equation*}
In definitiva:
\begin{equation}
	\hat{L}^2 \ket{\ell,m} = \hbar^2 \ell (\ell + 1) \ket{\ell,m} \qquad \ell = \frac{N}{2},\,N\in\N
	\label{eq:2.18}
\end{equation}
\begin{equation}
	\hat{L}_z \ket{\ell,m} = \hbar m \ket{\ell,m} \qquad -\ell \le m \le \ell
	\label{eq:2.19}
\end{equation}
La normalizzazione propria degli stati è data da:
\begin{equation}
	\braket{\ell',m' | \ell,m} = \delta_{\ell'\ell} \delta_{m'm}
	\label{eq:2.20}
\end{equation}
Per normalizzare correttamente le autofunzioni, si calcola:
\begin{equation*}
	\begin{split}
		&\hat{L}_+\hat{L}_- \ket{\ell,m} = \hbar^2 (\ell(\ell + 1) - m(m - 1)) \ket{\ell,m}\\
		&\hat{L}_-\hat{L}_+ \ket{\ell,m} = \hbar^2 (\ell(\ell + 1) - m(m + 1)) \ket{\ell,m}
	\end{split}
\end{equation*}
Ricordando la Prop. \ref{l-ladder}, si può definire pienamente la scala delle autofunzioni a $ \ell $ fissato:
\begin{equation}
	\ket{\ell,m \pm 1} = \frac{1}{\hbar \sqrt{\ell(\ell + 1) - m(m \pm 1)}} \hat{L}_{\pm} \ket{\ell,m}
	\label{eq:2.21}
\end{equation}

\paragraph{Quantizzazione}

È necessario fare delle osservazioni sugli autovalori di $ \hat{L}_z $ e $ \hat{L}^2 $ trovati. Innanzitutto, si vede che $ L_z $ è quantizzato in multipli interi o semi-interi di $ \hbar $, il che è alquanto notevole, soprattutto se comparato alla fisica classica: quantisticamente, quindi, sebbene non abbia senso parlare di traiettorie ($ \hat{x} $ e $ \hat{p} $ non commutano), si possono descrivere le orbite quantizzate dei corpi, corrispondendi a valori discreti del momento angolare.\\
Un'altro fatto che viene confermato è che le componenti del momento angolare non sono compatibili tra loro: se è completamente determinata $ L_z $, $ L_x $ ed $ L_y $ sono completamente indeterminate (e quindi non avrebbe senso parlare di un vettore tridimensionale); ciò è confermato dal fatto che, quando $ L_z $ assume il suo valore massimo $ \hbar \ell $, questo è comunque strettamente minore del modulo del momento angolare $ \hbar \sqrt{\ell(\ell + 1)} $ (mentre classicamente si avrebbe $ L_z^{(\text{max})} = L $).

\subsection{Autofunzioni sulla base delle coordinate}

È possibile definire le autofunzioni del momento angolare sulla base delle coordinate come:
\begin{equation}
	\braket{\vartheta,\varphi | \ell,m} \defeq Y_{\ell,m}(\vartheta,\varphi)
	\label{eq:2.22}
\end{equation}
Ricordando l'espressione di $ \hat{L}_z $ sulla base delle coordiante (Eq. \ref{eq:2.9}), l'Eq. \ref{eq:2.19} diventa un'equazione differenziale:
\begin{equation}
	-i\hbar \frac{\pa}{\pa \varphi} Y_{\ell,m}(\vartheta,\varphi) = \hbar m Y_{\ell,m}(\vartheta,\varphi)
	\label{eq:2.23}
\end{equation}
È possibile scrivere la soluzione generale come:
\begin{equation}
	Y_{\ell,m}(\vartheta,\varphi) = \mathcal{N}_{\ell,m} e^{im\varphi} P_{\ell,m}(\cos\vartheta)
	\label{eq:2.24}
\end{equation}
Queste sono dette armoniche sferiche.

\paragraph{Fase}

L'autovalore $ m $ di $ \hat{L}_z $ determina un fattore di fase nell'autofunzione. Se si impone la condizione che la funzione d'onda sia monodroma, così da avere uno spazio degli stati fisici semplicemente connesso, è necessario che $ Y_{\ell,m}(\vartheta,\varphi+2\pi) = Y_{\ell,m}(\vartheta,\varphi) $, ovvero:
\begin{equation}
	e^{im2\pi} = 1
	\label{eq:2.25}
\end{equation}
Questa condizione è soddisfatta solo se $ m $, e di conseguenza $ \ell $, è intero: nel caso di momento angolare con autofunzioni monodrome si parla di momento angolare orbitale.

È possibile determinare esplicitamente le armoniche sferiche senza risolvere l'equazione agli autovalori per $ \hat{L}^2 $, che è una PDF di second'ordine, utilizzando invece la condizione $ \hat{L}_- \ket{\ell,m_{\text{min}}} = 0 $. Innanzitutto:
\begin{equation}
	\begin{split}
		\hat{L}_-
		&= i\hbar \left( \sin \varphi \frac{\pa}{\pa \vartheta} + \cot \vartheta \cos \varphi \frac{\pa}{\pa \varphi} \right) + \hbar \left( -\cos \varphi \frac{\pa}{\pa \vartheta} + \cot \vartheta \sin \varphi \frac{\pa}{\pa \varphi} \right)\\
		&= \hbar e^{-i\varphi} \left( -\frac{\pa}{\pa \vartheta} + i \cot \vartheta \frac{\pa}{\pa \varphi} \right)
	\end{split}
	\label{eq:2.26}
\end{equation}
Si vede subito che il fattore di fase fa abbassare di un'unità $ m $. Si ha quindi l'equazione:
\begin{equation}
	\hbar e^{-i\varphi} \left( - \frac{\pa}{\pa \vartheta} + i \cot \vartheta \frac{\pa}{\pa \varphi} \right) Y_{\ell,-\ell} (\vartheta,\varphi) = 0
	\label{eq:2.27}
\end{equation}
Dall'Eq. \ref{eq:2.24}:
\begin{equation}
	\left( -\frac{\pa}{\pa \vartheta} + \ell \frac{\cos \vartheta}{\sin \vartheta} \right) e^{-i\ell\varphi} P_{\ell,-\ell}(\cos \vartheta) = 0
	\label{eq:2.28}
\end{equation}
Per la chain rule $ \pa_{\vartheta} = \cos \vartheta \pa_{\sin \vartheta} $, dunque:
\begin{equation}
	\frac{\pa}{\pa \sin \vartheta} P_{\ell,-\ell}(\cos \vartheta) = \frac{\ell}{\sin \vartheta} P_{\ell,-\ell}(\cos \vartheta)
	\label{eq:2.29}
\end{equation}
Questa può essere riscritta come:
\begin{equation}
	\frac{dP_{\ell,-\ell}(\cos \vartheta)}{P_{\ell,-\ell}(\cos \vartheta)} = \ell \frac{d\sin \vartheta}{\sin \vartheta}
	\label{eq:2.30}
\end{equation}
La soluzione è immediata:
\begin{equation}
	P_{\ell,-\ell}(\cos \vartheta) = \left( \sin \vartheta \right)^{\ell}
	\label{eq:2.31}
\end{equation}
Tutte le altre armoniche sferiche ad $ \ell $ fissato possono essere trovate con i ladder operators:
\begin{equation}
	Y_{\ell,-\ell+k}(\vartheta,\varphi) = \mathcal{N}_{\ell,-\ell+k} \hat{L}_+^k Y_{\ell,-\ell}(\vartheta,\varphi)
	\label{eq:2.32}
\end{equation}
Svolgendo i calcoli:
\begin{equation}
	P_{\ell,k} \sim \left( \sin \vartheta \right)^k \left( \cos \vartheta \right)^{\ell-k} \quad \forall k \in \left[ 0,\ell \right]
	\label{eq:2.33}
\end{equation}
Le armoniche sferiche sono una base ortonormale completa dello spazio delle funzioni definite su $ \mathbb{S}^2 $, dunque vale la relazione di ortonormalità:
\begin{equation}
	\int_{\mathbb{S}^2} d\Omega \braket{\ell',m' | \vartheta,\varphi}\braket{\vartheta,\varphi | \ell,m} = \int_{\mathbb{S}^2} d\cos\vartheta \,d\varphi \, Y^*_{\ell',m'}(\vartheta,\varphi) Y_{\ell,m}(\vartheta,\varphi) = \delta_{\ell'\ell} \delta_{m'm}
	\label{eq:2.34}
\end{equation}
Vale inoltre la relazione di completezza sulla sfera:
\begin{equation}
	\sum_{l = 0}^{+\infty} \sum_{m = -\ell}^{\ell} \ket{\ell,m}\bra{\ell,m} = \tens{I}
	\label{eq:2.35}
\end{equation}
ovvero:
\begin{equation}
	\sum_{\ell = 0}^{+\infty} \sum_{m = -\ell}^{\ell} \braket{\vartheta,\varphi | \ell,m}\braket{\ell,m | \vartheta',\varphi'} = \sum_{\ell,m} Y^*_{\ell,m}(\vartheta',\varphi')Y_{\ell,m}(\vartheta,\varphi) = \delta(\cos \vartheta - \cos \vartheta') \delta(\varphi - \varphi')
	\label{eq:2.36}
\end{equation}
Ciò equivale a decomporre il sistema in termini di frequenza proprie sulla sfera: la differenza con la trasformata di Fourier, che decompone in frequenza proprie della retta, è che in quel caso le frequenze variano in $ \left( -\infty,+\infty \right) $, mentre in questo caso in $ \left[ 0,2\pi \right] $.\\
Nel caso in cui $ m = 0 $ non si ha alcuna dipendenza da $ \varphi $ e si trova che i $ P_{\ell,0}(\cos \vartheta) \equiv P_{\ell}(\vartheta) $ sono polinomi di $ \cos \vartheta $, detti polinomi di Legendre, formanti una base ortonormale su $ \mathbb{S}^1 $, e dunque sul segmento $ \cos \vartheta \in \left[ -1,1 \right] $ (il caso di questa trattazione).

\section{Spin}

Il significato fisico dei valri di $ \ell $ semi-intero è legato a rotazioni su un diverso spazio di Hilbert.\\
In meccanica classica, la rotazione di osservabili scalari $ \omega(\ve{x}) $ dipende da come tale rotazione agisce sulle coordinate $ \ve{x} $: quantisticamente, questo è il caso associato all'effetto delle rotazioni sulla funzione d'onda $ \braket{\ve{x} | \psi} = \psi(\ve{x}) $, determinato dal momento angolare orbitale (ovvero associato alla traiettoria del sistema).
Se invece si considera un'osservabile vettoriale $ \ve{v}(\ve{x}) $, la rotazione non agisce solo sul suo modulo (che è uno scalare), ma anche sulla sua direzione.\\
Dal punto di vista quantistico, questi due tipi di rotazioni sono nettamente distinti, poiché agiscono su spazi di Hilbert differenti: la rotazione delle coordinate agisce su uno spazio di Hilbert infinito-dimensionale, mentre la rotazione delle $ \virgolette{direzioni} $ agisce su uno spazio di Hilbert finito-dimensionale.
Nel primo caso si parla di momento angolare orbitale, nel secondo caso di momento angolare di spin.

\subsection{Spin \texorpdfstring{$ 1 $}{TEXT}}

Si consideri un sistema tripartito, la cui base dello spazio degli stati è definita come:
\begin{equation*}
	\ket{1} \equiv
	\begin{pmatrix}
		1 \\ 0 \\ 0
	\end{pmatrix}
	\qquad
	\ket{2} \equiv
	\begin{pmatrix}
		0 \\ 1 \\ 0
	\end{pmatrix}
	\qquad
	\ket{3} \equiv
	\begin{pmatrix}
		0 \\ 0 \\ 1
	\end{pmatrix}
\end{equation*}
Questa notazione è ambigua, poiché esprime la base rispetto ad un'altra base implicita, ma è conveniente per rappresentare le rotazioni. Il generico stato (equivalente, classicamente, alla direzione di un vettore in $ \R^3 $) è:
\begin{equation}
	\ket{v} = c_1 \ket{1} + c_2 \ket{2} + c_3 \ket{3} =
	\begin{pmatrix}
		c_1 \\ c_2 \\ c_3
	\end{pmatrix}
	\label{eq:2.37}
\end{equation}
con $ c_1,c_2,c_3 \in \C : \abs{c_1}^2 + \abs{c_2}^2 + \abs{c_3}^2 = 1 $.\\
Considerando il caso particolare di $ \ket{v} = \cos \varphi \ket{1} + \sin \varphi \ket{2} $ ed una rotazione attorno a $ \ket{3} $, si ha:
\begin{equation}
	\ket{v'} = \hat{R}_{\varepsilon}^{(3)} \ket{v} =
	\begin{pmatrix}
		\cos \left( \varphi + \varepsilon \right) \\ \sin \left( \varphi + \varepsilon \right) \\ 0
	\end{pmatrix}
	=
	\begin{pmatrix}
		\cos \varphi - \varepsilon \sin \varphi \\ \sin \varphi + \varepsilon \cos \varphi \\ 0
	\end{pmatrix}
	+ o(\varepsilon)
	\label{eq:2.38}
\end{equation}
Esprimendo la rotazione in funzione di un generatore $ \hat{R}_{\varepsilon}^{(3)} = e^{-\frac{i}{\hbar} \varepsilon \hat{S}_3} $ si ha:
\begin{equation}
	\ket{v'} = \left( \tens{I} - \frac{i}{\hbar} \varepsilon \hat{S}_3 + o(\varepsilon) \right) \ket{v}
	\label{eq:2.39}
\end{equation}
Si trova dunque:
\begin{equation}
	\hat{S}_3 = -i\hbar
	\begin{bmatrix}
		0 & 1 & 0 \\
		-1 & 0 & 0 \\
		0 & 0 & 0
	\end{bmatrix}
	\label{eq:2.40}
\end{equation}
Si vede inoltre che l'Eq. \ref{eq:2.39} è compatibile con l'Eq. \ref{eq:2.3}: la rotazione di $ \psi $ può essere interpretata sia come $ \braket{\ve{x} | \psi'} = \psi'(\ve{x}) $ (alias) sia come $ \braket{\ve{x}' | \psi} = \psi(\ve{x}') $ (alibi), e dal primo caso si ha che $ \ket{\psi'} = \hat{R}_{\varepsilon} \ket{\psi} $, mentre dal secondo $ \bra{\ve{x}'} = \bra{\ve{x}} \hat{R}_{\varepsilon} $, ovvero $ \ket{\ve{x}'} = \hat{R}_{\varepsilon}^{\dagger} \ket{\ve{x}} $, che equivale all'Eq. \ref{eq:2.39}.\\
Replicando il calcolo per gli altri assi di rotazione si trova:
\begin{equation}
	\hat{S}_1 = -i\hbar
	\begin{bmatrix}
		0 & 0 & 0 \\
		0 & 0 & 1 \\
		0 & -1 & 0
	\end{bmatrix}
	\qquad
	\hat{S}_2 = -i\hbar
	\begin{bmatrix}
		0 & 0 & -1 \\
		0 & 0 & 0 \\
		1 & 0 & 0
	\end{bmatrix}
	\label{eq:2.41}
\end{equation}
ovvero, in generale:
\begin{equation}
	[\hat{S}_k]_{ij} = -i\hbar \epsilon_{kij}
	\label{eq:2.42}
\end{equation}
Questi operatori, detti operatori di spin, soddisfano la relazione di commutazione per operatori del momento angolare in Prop. \ref{l-comm}:
\begin{equation*}
	\begin{split}
		[\hat{S}_i,\hat{S}_j]_{ac}
		&= \sum_{b = 1}^{3} \left( [\hat{S}_i]_{ab} [\hat{S}_j]_{bc} - [\hat{S}_j]_{ab} [\hat{S}_i]_{bc} \right) = -\hbar^2 \sum_{b = 1}^{3} \left( \epsilon_{iab} \epsilon_{jbc} - \epsilon_{jab} \epsilon_{ibc} \right)\\
		&= -\hbar^2 \left( \delta_{ic}\delta_{aj} - \delta_{ij}\delta_{ac} - \delta_{jc} \delta_{ai} + \delta_{ji} \delta_{ac} \right) = -\hbar^2 \left( \delta_{ic} \delta_{aj} - \delta_{jc} \delta_{ai} \right)\\
		&= i\hbar \sum_{k = 1}^{3} \epsilon_{ijk} [\hat{S}_k]_{ac} = \hbar^2 \sum_{k = 1}^{3} \epsilon_{ijk} \epsilon_{kac} = \hbar^2 \left( \delta_{ia}\delta_{jc} - \delta_{ic}\delta_{ja} \right)
	\end{split}
\end{equation*}
Gli operatori di spin forniscono dunque una rappresentazione del momento angolare. Per quanto riguarda il modulo, si vede subito che $ \hat{S}^2 \defeq \hat{S}_1^2 + \hat{S}_2^2 + \hat{S}_3^2 $ è espresso come:
\begin{equation}
	\hat{S}^2 = 2\hbar^2 \tens{I}
	\label{eq:2.43}
\end{equation}
ovvero tutti i vettori dello spazio sono suoi autovettori. L'autovalore associato a $ \hat{L}^2 $ è in generale $ \hbar^2 \ell (\ell + 1) $, quindi si vede che in questo caso si ha $ \ell = 1 $: per questo si parla di sistema a spin $ s = 1 $.\\
La dimensione dello spazio di Hilbert è data dal numero di possibili valori di $ m \equiv s_z $, ovvero in totale $ 2s + 1 $, dato che $ -s \le s_z \le s $: in questo caso la dimensione è giustamente 3 e si possono esplicitare gli autovettori $ \ket{v_{s_z}} $ di $ \hat{S}_z $ ($ \hat{S}_z \ket{v_{s_z}} = s_z\hbar \ket{v_{s_z}} $), ottenendo la cosiddetta base sferica:
\begin{equation}
	\ket{v_{\pm}} \equiv \ket{1,\pm 1} = \frac{1}{\sqrt{2}}
	\begin{pmatrix}
		1 \\ \pm i \\ 0
	\end{pmatrix}
	\qquad
	\ket{v_0} \equiv \ket{1,0} =
	\begin{pmatrix}
		0 \\ 0 \\ 1
	\end{pmatrix}
	\label{eq:2.44}
\end{equation}
L'analogia tra un sistema con $ \ell = 1 $ ed uno con spin 1 deriva dal fatto che il primo è descritto da un sottospazio finito-dimensionale di uno spazio infinito-dimensionale, e tale restrizione è equivalente allo spazio che descrive il secondo sistema. Nel caso tridimensionale considerato, il momento angolare orbitale agisce su uno spazio con base $ \ket{\ve{x}} = \ket{x_1} \otimes \ket{x_2} \otimes \ket{x_3} $, mentre il momento angolare di spin su uno spazio con base $ \{\ket{e_i}\}_{i=1,2,3} $ (qutrit).

\subsection{Spin \texorpdfstring{$ \frac{1}{2} $}{TEXT}}

Per un sistema con $ s = \frac{1}{2} $, i possibili valori di $ s_z $ sono 2, $ s_z = \pm \frac{1}{2} $, dunque il sistema è fondamentalmente un qubit e i suoi stati possono essere indicati come $ \ket{\pm} \equiv \ket{\frac{1}{2},\pm\frac{1}{2}} $.\\
Il più generale stato in questo spazio è $ \ket{\psi} = c_+ \ket{+} + c_- \ket{-} $, con $ c_{\pm}\in\C $, dunque è possibile rappresentarlo come uno spinore (vettore a componenti complesse):
\begin{equation}
	\ket{\psi} =
	\begin{pmatrix}
		c_+ \\ c_-
	\end{pmatrix}
	\label{eq:2.45}
\end{equation}
Dato che $ \hat{S}_z \ket{\pm} = \pm \frac{\hbar}{2} \ket{\pm} $, in tale rappresentazione $ \hat{S}_z $ può essere scritto (con abuso di notazione) come una matrice diagonale:
\begin{equation}
	\hat{S}_z = \frac{\hbar}{2}
	\begin{bmatrix}
		1 & 0 \\
		0 & -1
	\end{bmatrix}
	\label{eq:2.46}
\end{equation}
Per calcolare $ \hat{S}_x $ ed $ \hat{S}_y $ si utilizzano i ladder operators $ \hat{S}_{\pm} = \hat{S}_x \pm i \hat{S}_y $, ricordando le relazioni di scala in Eq. \ref{eq:2.21}:
\begin{equation*}
	\hat{S}_+ \ket{-} = \hbar \sqrt{\frac{1}{2} \left( \frac{1}{2} + 1 \right) - \left( - \frac{1}{2} \right) \left( - \frac{1}{2} + 1 \right)} \ket{+} = \hbar \ket{+}
\end{equation*}
\begin{equation*}
	\hat{S}_- \ket{+} = \hbar \sqrt{\frac{1}{2} \left( \frac{1}{2} + 1 \right) - \frac{1}{2} \left( \frac{1}{2} - 1 \right)} \ket{-} = \hbar \ket{-}
\end{equation*}
ovvero:
\begin{equation}
	\hat{S}_+ = \hbar
	\begin{bmatrix}
		0 & 1 \\
		0 & 0
	\end{bmatrix}
	\qquad
	\hat{S}_- = \hbar
	\begin{bmatrix}
		0 & 0 \\
		1 & 0
	\end{bmatrix}
	\label{eq:2.47}
\end{equation}
Dato che $ \hat{S}_x = \frac{1}{2} ( \hat{S}_+ + \hat{S}_- ) $ e $ \hat{S}_y = \frac{1}{2i} ( \hat{S}_+ - \hat{S}_- ) $, ricordando anche Eq. \ref{eq:2.46}, si trova la relazione tra operatori di spin e matrici di Pauli:
\begin{equation}
	\hat{S}_i = \frac{\hbar}{2} \sigma_i
	\label{eq:2.48}
\end{equation}

\begin{proposition}
	$ \left[ \hat{S}_i,\hat{S}_j \right] = i\hbar \sum_{k = 1}^{3} \epsilon_{ijk} \hat{S}_k $.
\end{proposition}
\begin{proof}
	Ricordando che $ \sigma_i \sigma_j = \delta_{ij} \tens{I} + i \sum_{k = 1}^{3} \epsilon_{ijk} \sigma_k $:
	\begin{equation*}
		\left[ \hat{S}_i,\hat{S}_j \right] = \frac{\hbar^2}{4} \left( \sigma_i \sigma_j - \sigma_j \sigma_i \right) = \frac{\hbar^2}{2} i \sum_{k = 1}^{3} \sigma_k = i\hbar \sum_{k = 1}^{3} \hat{S}_k
	\end{equation*}
\end{proof}

Inoltre, si vede immediatamente che:
\begin{equation}
	\hat{S}^2 = \frac{3}{4} \hbar^2 \tens{I}
	\label{eq:2.49}
\end{equation}
Imponendo $ s (s + 1) = \frac{3}{4} $ si trova, per l'appunto, $ s = \frac{1}{2} $.\\
È possibile vedere il comportamento peculiare dei sistemi a spin $ \frac{1}{2} $ sotto rotazioni ricordando l'espressione degli operatori di rotazione in Eq. \ref{eq:2.39}: considerando una rotazione di $ 2\pi $ attorno l'asse $ z $ e ricordando che $ \ket{\pm} $ sono autostati di $ \sigma_z $:
\begin{equation}
	\hat{R}_{2\pi}^{(z)} \ket{\psi} = e^{-i \pi \sigma_z} \left( c_+ \ket{+} + c_- \ket{-} \right) = c_+ e^{-i\pi} \ket{+} + c_- e^{i\pi} \ket{-} = - \ket{\psi}
	\label{eq:2.50}
\end{equation}
Il calcolo è analogo lungo qualsiasi asse. Si vede dunque che ruotando il sistema di $ 2\pi $ uno stato di spin semi-intero acquista un segno negativo, mentre per tornare in sé stesso è necessaria una rotazione di $ 4\pi $: questo impedisce di rappresentare il vettore di stato come una funzione sullo spazio delle coordinate, ma non viola alcun principio fondamentale, dando anzi luogo ad effetti sperimentalmente osservabili verificati.

\section{Composizione di momenti angolari}

Tutte le particelle che formano la materia sono portatrici di spin (ad eccezione del bosone di Higgs): è dunque necessario studiare come si comportano i sistemi quantistici dotati sia di momento angolare orbitale che di spin.\\
Lo stato di un sistema si spin $ s $ può essere espresso sia su autostati di $ \hat{\ve{L}} $ che di $ \hat{\ve{S}} $, dunque:
\begin{equation}
	\braket{\ve{x} | \psi} = \sum_{\ell = 0}^{\infty} \sum_{m = -\ell}^{\ell} \sum_{s_z = -s}^{s} \braket{\ve{x} | \ell,m,s_z} \braket{\ell,m,s_z | \psi}
	\label{eq:2.51}
\end{equation}
Dato che $ \ket{\ve{x}} = \ket{r,\vartheta,\varphi} \equiv \ket{r} \otimes \ket{\vartheta,\varphi} $ e $ \braket{\vartheta,\varphi | \ell,m} = Y_{\ell,m}(\vartheta,\varphi) $ (base ortonormale di $ \mathbb{S}^2 $):
\begin{equation}
	\psi(\ve{x}) = \sum_{\ell = 0}^{\infty} \sum_{m = -\ell}^{\ell} \sum_{s_z = -s}^{s} c_{\ell,m,s_z}(r) Y_{\ell,m}(\vartheta,\varphi) u_{s_z}
	\label{eq:2.52}
\end{equation}
dov'è stata definita la base dello spin $ u_{s_z} \equiv \braket{\vartheta,\varphi | s_z} $. Ad esempio, nel caso $ s = \frac{1}{2} $ gli $ u_{s_z} $ sono i due spinori $ u_+ = (1, 0)^{\intercal} $ e $ u_- = (0,1)^{\intercal} $, dunque $ \psi(\ve{x}) $ sarà anch'esso uno spinore, in generale non fattorizzabile.\\
La probabilità di rilevare il sistema in $ \ve{x} $ è la somma delle probabilità su tutti i valori di spin:
\begin{equation}
	\rho(\ve{x}) = \sum_{s_z = -s}^{s} \abs{\psi_{s_z}(\ve{x})}^2
	\label{eq:2.53}
\end{equation}
La probabilità che la misura dello spin lungo l'asse $ z $ risulti $ s_z $ invece è:
\begin{equation}
	P_{s_z} = \int_{\R^3} d^3\ve{x} \,\psi_{s_z}(\ve{x})
	\label{eq:2.54}
\end{equation}

\subsection{Coefficienti di Clebsch-Gordan}

È utile definire il momento angolare totale:
\begin{equation}
	\hat{\ve{J}} = \hat{\ve{L}} + \hat{\ve{S}}
	\label{eq:2.55}
\end{equation}
Più formalmente, dato che i due operatori agiscono su spazi diversi (quello delle posizioni e quello delle direzioni), la definizione corretta è $ \hat{\ve{J}} \defeq \hat{\ve{L}} \otimes \hat{\tens{I}}_s + \hat{\tens{I}}_x \otimes \hat{\ve{S}} $.

\begin{proposition}
	$ [J_i,J_j] = i\hbar \sum_{k = 1}^{3} i \hbar \epsilon_{ijk} J_k $.
\end{proposition}
\begin{proof}
	Agendo su spazi diversi, si ha $ [L_i,S_j] = 0 $, dunque:
	\begin{equation*}
		[J_i,J_j] = [L_i,L_j] + [S_i,S_j] = i\hbar \sum_{k = 1}^{3} \epsilon_{ijk} (L_k + S_k) = i\hbar \sum_{k = 1}^{3} \epsilon_{ijk} J_k
	\end{equation*}
\end{proof}

Dunque, $ \hat{\ve{J}} $ è effettivamente un operatore di momento angolare: vale di conseguenza che $ [\hat{J}^2,\hat{J}_i] = 0 $.

\begin{proposition}
	$ \hat{J}^2 $, $ \hat{J}_z $, $ \hat{L}^2 $ ed $ \hat{S}^2 $ commutano tra loro.
\end{proposition}
\begin{proof}
	$ \hat{L}^2 $ ed $ \hat{S}^2 $ commutano poiché agiscono su spazi diversi. $ [\hat{J}^2,\hat{S}^2] $ è analogo a $ [\hat{J}^2,\hat{L}^2] $: $ [\hat{J}^2,\hat{L}^2] = [\hat{L}^2 + \hat{S}^2 + 2 \hat{\ve{L}}\cdot\hat{\ve{S}},\hat{L}^2] = 2 \sum_{k = 1}^{3} [\hat{L}_k \hat{S}_k,\hat{L}^2] = 0 $. I commutatori di $ \hat{J}_z $ sono banali.
\end{proof}

\begin{proposition}
	$ \hat{J}^2 $ non commuta con $ \hat{L}_z $ ed $ \hat{S}_z $.
\end{proposition}
\begin{proof}
	Analoghi: $ [\hat{J}^2,\hat{L}_z] = 2 \sum_{i = 1}^{3} [\hat{L}_i,\hat{L_z}]S_i = 2i\hbar \sum_{i,k = 1}^{3} \epsilon_{i3k} L_k S_i \neq 0 $.
\end{proof}

È possibile, dunque, scegliere due diverse basi per esprimere lo stato del sistema: la \textit{base disaccoppiata} $ \ket{\ell,m,s,s_z} $, diagonalizzando $ \hat{L}^2 $, $ \hat{L}_z $, $ \hat{S}^2 $ ed $ \hat{S}_z $, e la \textit{base accoppiata} $ \ket{j,j_z,\ell,s} $, diagonalizzando $ \hat{J}^2 $, $ \hat{J}_z $, $ \hat{L}^2 $ ed $ \hat{S}^2 $. Il passaggio tra le due basi è dato da:
\begin{equation}
	\ket{\ell,m,s,s_z} = \sum_{j = j_{\text{min}}}^{j_{\text{max}}} \sum_{j_z = -j}^{j} \ket{j,j_z,\ell,m} \braket{j,j_z,\ell,s | \ell,m,s,s_z}
	\label{eq:2.56}
\end{equation}
\begin{equation}
	\ket{j,j_z,\ell,s} = \sum_{m = -\ell}^{\ell} \sum_{s_z = -s}^{s} \ket{\ell,m,s,s_z} \braket{\ell,m,s,s_z | j,j_z,\ell,s}
	\label{eq:2.57}
\end{equation}
I coefficienti $ \braket{j,j_z,\ell,s | \ell,m,s,s_z}, \braket{\ell,m,s,s_z | j,j_z,\ell,s} $ sono detti \textit{coefficienti di Clebsh-Gordan}. È necessario esplicitare il range di $ j $.

\begin{proposition}
	$ \braket{\ell,m,s,s_z | j,j_z,\ell,s} \propto \delta_{j_z,m + s_z} $.
\end{proposition}
\begin{proof}
	Dato che $ \hat{J}_z = \hat{L}_z + \hat{S}_z $, si ha $ \hat{J}_z - \hat{L}_z - \hat{S}_z = 0 $:
	\begin{equation*}
		0 = \braket{\ell,m,s,s_z | \hat{J}_z - \hat{L}_z - \hat{S}_z | j,j_z,\ell,s} = \left( j_z - m - s_z \right) \hbar \braket{\ell,m,s,s_z | j,j_z,\ell,s}
	\end{equation*}
	dunque $ \braket{\ell,m,s,s_z | j,j_z,\ell,s} \neq 0 \,\Rightarrow\, j_z = m + s_z $.
\end{proof}
Si vede allora che $ \ket{j,j_z,\ell,s} $ è combinazione lineare solo degli stati $ \ket{\ell,m,s,s_z} : m + s_z = j_z $: di conseguenza, $ j_z^{\text{max}} = m^{\text{max}} + s_z^{\text{max}} = \ell + s $, ma $ j_z^{\text{max}} = j_{\text{max}} $, quindi $ j_{\text{max}} = \ell + s $: infatti, lo stato $ \ket{j = \ell + s, j_z = j, \ell, s} $ può essere ottenuto solo come $ \ket{\ell,m = \ell,s,s_z = s} $, mentre, ad esempio, lo stato $ \ket{j = \ell + s, j_z = j - 1, \ell, s} $ è combinazione lineare dei due stati $ \ket{\ell, m = \ell - 1, s, s_z = s} $ e $ \ket{\ell, m = \ell, s, s_z = s - 1} $.\\
Affinché la base accoppiata abbia lo stesso numero di elementi della base disaccoppiata ($ (2\ell + 1)(2s + 1) $), è necessario che $ j_{\text{min}} = \abs{\ell - s} $; assumendo WLOG $ \ell > s $:
\begin{equation*}
	\sum_{j = \ell - s}^{\ell + s} (2j + 1) = \sum_{k = 0}^{2s} \left( 2(k + \ell - s) + 1 \right) = 2s (2s + 1) + \left( 2(\ell - s) +1 \right) (2s + 1) = (2\ell + 1)(2s + 1)
\end{equation*}

\subsubsection{Composizione di due spin \texorpdfstring{$ \frac{1}{2} $}{TEXXT}}

Nel caso di un sistema in cui si compongono due spin $ \frac{1}{2} $ (doppio qubit), si può introdurre la notazione semplificata $ \pm \equiv \pm \frac{1}{2} $. Dalla condizione $ \abs{s_1 - s_2} \le s \le s_1 + s_2 $ si trova che i possibili valori di $ s $ sono $ 0 $ e $ 1 $: in particolare, si ha un tripletto di spin 1 ($ \ket{1,1}, \ket{1,0},\ket{1,-1} $) ed un singoletto di spin 0 ($ \ket{0,0} $).\\
Per calcolare i coefficienti di Clebsch-Gordan, si consideri innanzitutto che, per la condizione su $ j_z^{\text{max}} $, si ha:
\begin{equation}
	\ket{1,1} = \ket{+,+}
	\label{eq:2.58}
\end{equation}
Definendo $ \hat{S}^{\pm} \defeq \hat{S}_1^{\pm} + \hat{S}_2^{\pm} $ gli operatori di innalzamento/abbassamento per lo spin totale, si ha, dall'Eq. \ref{eq:2.21}:
\begin{equation*}
	\begin{split}
		\hat{S}^- \ket{1,1}
		&= \hbar \sqrt{1 (1 + 1) - 1 (1 -1)} \ket{1,0} = \hbar \sqrt{2} \ket{1,0}\\
		&= (\hat{S}_1^- + \hat{S}_2^-) \ket{+,+} = \hbar \sqrt{ \frac{1}{2} \left( \frac{1}{2} + 1 \right) - \frac{1}{2} \left( \frac{1}{2} - 1 \right)} \left( \ket{+,-} + \ket{-,+} \right)\\
		&= \hbar \left( \ket{+,-} + \ket{-,+} \right)
	\end{split}
\end{equation*}
Si vede dunque che:
\begin{equation}
	\ket{1,0} = \frac{1}{\sqrt{2}} \left( \ket{+,-} + \ket{-,+} \right)
	\label{eq:2.59}
\end{equation}
Dato che $ j_z = s_1 + s_2 $, l'unico modo per avere lo stato con $ j_z = -1 $ è:
\begin{equation}
	\ket{1,-1} = \ket{-,-}
	\label{eq:2.60}
\end{equation}
Dalla condizione di ortogonalità (in particolare $ \braket{1,0 | 0,0} = 0 $) si ricava infine:
\begin{equation}
	\ket{0,0} = \frac{1}{\sqrt{2}} \left( \ket{+,-} - \ket{-,+} \right)
	\label{eq:2.61}
\end{equation}
Questo metodo è generale: si parte dallo stato più alto e si agisce tramite operatori di innalzamento/abbassamento.\\
I sistemi di doppio spin $ \frac{1}{2} $ sono particolarmente interessanti poiché sono il più semplice sistema quantistico non-banale: i suoi stati non possono essere fattorizzati, ed in particolare gli stati con $ j_z = 0 $ sono massimamente entangled.\\
Inoltre, si nota che questo è un sistema di particelle identiche: non c'è misura in grado di distinguere le due particelle. C'è quindi simmetria finita per scambio delle particelle: gli stati del tripletto sono simmetrici, mentre il singoletto è antisimmetrico.










