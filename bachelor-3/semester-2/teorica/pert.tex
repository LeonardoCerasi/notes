\selectlanguage{english}

In the case of free Hamiltonians/Lagrangians, free-particle states are eigenstates of the Hamiltonian and they cannot interact. To describe interactions between particles, non-linear terms must be included in $ \ham / \lag $, which will couple Fourier modes (and particles that occupy them) to one another. \\
To preserve causality, these interaction terms must include only products of fields at the same spacetime point. Moreover, the discussion is restricted to terms which do not include derivatives of fields, so that $ \ham_\text{int} = - \lag_\text{int} $.

\begin{example}{$ \phi^4 $ theory}{}
  The so-called $ \phi^4 $ theory is a scalar field theory with an interaction term of the kind:
  \begin{equation}
    \lag = \frac{1}{2} \pa_\mu \phi \pa^\mu \phi - \frac{1}{2} m^2 \phi^2 - \frac{\lambda}{4!} \phi^4
  \end{equation}
  for $ \lambda \in \R $ a dimensionless coupling constant. This interaction describes the Higgs self-interaction in the Standard Model.
\end{example}

\section{Time evolution}

Consider a system described by a Hamiltonian $ H $. Two different ways of describing its time evolution are possible.

\subsection{Schrödinger picture}

In the Schrödinger picture, states are considered time-dependent and operators time-independent. Consider an initial state $ \ket{\psi(0)} \equiv \ket{\psi} $: its time-evolution is:
\begin{equation}
  \ket{\psi(t)} = e^{-i H t} \ket{\psi}
\end{equation}
In this picture, the amplitude for a process $ \ket{a(t_0)} \rightarrow \ket{b(t)} $ is:
\begin{equation}
  \mathcal{A} = \braket{b | e^{-i H (t - t_0)} | a} \eqdef \braket{b | S(t,t_0) | a}
  \label{eq:schr-ampl}
\end{equation}
In the $ t - t_0 \rightarrow \infty $ limit, $ S $ is known as \bctxt{$ S $-matrix}, and scattering amplitudes simply are its elements. This can be seen as an operator which realizes time evolution, as:
\begin{equation}
  \ket{\psi(t)} = S(t) \ket{\psi}
\end{equation}
where $ S(t) \equiv S(t,0) $.

\begin{proposition}{Unitarity of time evolution}{}
  $ S(t,t_0) $ is a unitary operator on $ \hilb $.
\end{proposition}

\begin{proofbox}
  \begin{proof}
    Consider a normalized initial state $ \ket{\psi} : \braket{\psi | \psi} = 1 $. Given a complete orthonormal eigenbasis $ \{\ket{n}\}_{n \in \N} $ of $ \hilb $, then it must be:
    \begin{equation}
      \sum_{n \in \N} \abs{\braket{n | S | \psi}}^2 = 1
    \end{equation}
    This can also be written as:
    \begin{equation}
      \sum_{n \in \N} \abs{\braket{n | S | \psi}}^2 = \sum_{n \in \N} \braket{\psi | S\dg | n} \braket{n | S | \psi} = \braket{\psi | S\dg S | \psi}
    \end{equation}
    Being $ \ket{\psi} $ arbitrary, it must be $ S\dg S = S S\dg = \id_\hilb $.
  \end{proof}
\end{proofbox}

The unitarity of the $ S $-matrix expresses the conservation of probabilities. Moreover, it can be decomposed as:
\begin{equation}
  S = \tens{I} + i T
\end{equation}
where the \bctxt{$ T $-matrix} contains the information on the interactions, while the identity $ \tens{I} $ only represents a part of the incident wave-packet unaffected by interactions.

\subsection{Heisenberg picture}

In the Heisenberg picture, states are considered time-independent and operators time-dependent. This picture is better suited for QFT, as fields (which are operators) depend both on $ \ve{x} $ and $ t $, therefore the Heisenberg picture is natural in light of Lorentz covariance. \\
Given a state $ \ket{\psi(t)} $ in the Schrödinger picture, in the Heisenberg picture it is defined as:
\begin{equation}
  \ket{\psi,t} \defeq e^{i H t} \ket{\psi(t)}
\end{equation}
which is manifestly time-independent. An operator $ \mathcal{O} $ in the Schrödinger picture instead becomes:
\begin{equation}
  \mathcal{O}(t) \defeq e^{i H t} \mathcal{O} e^{-i H t}
\end{equation}
The label $ t $ in the definition of state is necessary to distinguish whose operator the state is eigenstate of: in fact, in general $ \mathcal{O}(t) \neq \mathcal{O}(t') $ for $ t \neq t' $, so they will have different eigenstates too. \\
In the Heisenberg picture, the $ S $-matrix becomes:
\begin{equation}
  \braket{b | S(t,t_0) | a} = \braket{b,t | a,t_0}
\end{equation}

\section{Asymptotic theory}

On a macroscopic scale, interaction times are extremely small. Therefore, it is convenient to make an assumption, the \bctxt{adiabatic hypothesis}: while the interaction is described by $ \ham = \ham_0 + \ham_\text{int} $, where $ \ham_0 $ is the free Hamiltonian, in the far past and the far future $ \ham_\text{int} $ is adiabatically turned off, i.e. $ \ham \rightarrow \ham_0 $ as $ t \rightarrow \pm \infty $. Moreover, as the out-going states can represent in-coming states for a successive process, the two Fock spaces\footnotemark must be isomorphic, i.e. $ \fock_\text{in} \cong \fock_\text{out} $: in particular, this implies the uniqueness of the vacuum, as $ \ket{0_\text{in}} = \ket{0_\text{out}} \equiv \ket{0} $. This isomorphism is realized by $ S $-matrix $ S \equiv S(+\infty,-\infty) $ (recall \eref{eq:schr-ampl}):
%
\footnotetext{Formally, given a single-particle Hilbert space $ \hilb $, the Fock space is defined as the following completion:
\begin{equation}
  \fock_\nu(\hilb) \defeq \overline{\bigoplus_{n = 0}^\infty S_\nu \hilb^{\otimes n}}
\end{equation}
where $ S_\nu $ is the operator which symmetrizes ($ \nu = + $, for bosons) or antisymmetrizes ($ \nu = - $, for fermions) the tensors it acts upon. In general:
\begin{equation*}
  \fock_\nu(\hilb) = \C \oplus \hilb \oplus S_\nu(\hilb \otimes \hilb) \oplus S_\nu(\hilb \otimes \hilb \otimes \hilb) \oplus \dots
\end{equation*}
where $ S_\nu \hilb^{\otimes n} $ consists of $ n $-particle states ($ \C $ is the vacuum). A general state then is:
\begin{equation*}
  \ket{\Psi}_\nu = \sum_{n = 0}^{\infty} \ket{\Psi_n}_\nu = a \ket{0} \oplus \sum_{i} a_i \ket{\psi_i} \oplus \sum_{i,j} a_{ij} \ket{\psi_i , \psi_j}_\nu \oplus \dots
\end{equation*}
The need for this infinite sum to converge in $ \fock_\nu(\hilb) $ is solved by the completion, as it restricts the Fock space only to states with a finite inner-product-induced norm:
\begin{equation*}
  \norm{\ket{\Psi}_\nu}^2 = \sum_{n = 0}^{\infty} \braket{\Psi_n | \Psi_n}_\nu < \infty
\end{equation*}}
%
\begin{equation*}
  \braket{\beta_\text{out} | \alpha_\text{in}} = \braket{\beta_\text{in} | S\dg | \alpha_\text{in}}
  \qquad \Rightarrow \qquad
  \ket{\beta_\text{out}} = S \ket{\beta_\text{in}}
  \qquad \Rightarrow \qquad
  \phi_\text{out}(x) = S \phi_\text{in}(x) S\dg
\end{equation*}
where $ \phi_\text{in}(x) , \phi_\text{out}(x) $ are the free fields which generate $ \fock_\text{in} , \fock_\text{out} $. Note that, to preserve covariance, the $ S $-matrix must commute with Poincaré transformations.
The adiabatic hypothesis asserts then:
\begin{equation}
  \phi(x) \xrightarrow[t \rightarrow -\infty]{} \sqrt{Z} \phi_\text{in}(x)
  \qquad \qquad
  \phi(x) \xrightarrow[t \rightarrow +\infty]{} \sqrt{Z} \phi_\text{out}(x)
  \label{eq:adiabatic-hyp}
\end{equation}
where $ Z \in \C $ is a renormalization factor. These limits must be understood in the weak sense, as they are not operator equations, but they are only valid for each matrix element separately\footnotemark.
%
\footnotetext{If that wasn't the case, then canonical quantization would imply $ Z = 1 $, i.e. that $ \phi(x) $ is a free field.}

\subsection{LSZ reduction formula}

\subsubsection{Scalar fields}

Consider a scattering process of a single species of neutral scalar particles. As the ladder operators of a free real scalar field can be expressed via \eref{eq:frsf-ladder} (which is time-independent, as of \pref{prop:kg-scal-prod-ind}), in the adiabatic hypothesis it is possible to define \bctxt{in-} and \bctxt{out-ladder operators}:
\begin{equation}
  \sqrt{2E_\ve{p}} {a_\ve{p}^\text{in,out}}\dg = -i \int_{t \rightarrow \mp \infty} \dd^3x\, e^{-i p_\mu x^\mu} \smlra{\pa}_0 \phi_\text{in,out}(x) = - \frac{i}{\sqrt{Z}} \lim_{t \rightarrow \mp \infty} \int \dd^3x\, e^{-i p_\mu x^\mu} \smlra{\pa}_0 \phi(x)
  \label{eq:in-out-lad-op}
\end{equation}
These operators respectively act on $ \fock_\text{in,out} $. Note that now these integrals are time-dependent, as they contain the interacting field and not the free fields: the relation between in- and out-ladder operators is therefore not trivial.

\begin{lemma}[before upper = {\tcbtitle}]{}{in-out-rel}
  \begin{equation}
    \sqrt{2E_\ve{p}} ({a_\ve{p}^\text{in}}\dg - {a_\ve{p}^\text{out}}\dg) = \frac{i}{\sqrt{Z}} \int \dd^4x\, e^{-i k_\mu x^\mu} (\Box + m^2) \phi(x)
    \label{eq:lsz-scal-op}
  \end{equation}
\end{lemma}

\begin{proofbox}
  \begin{proof}
    For any integrable $ f(t,\ve{x}) $ the following identity holds:
    \begin{equation}
      \left( \lim_{t \rightarrow +  \infty} - \lim_{t \rightarrow -\infty} \right) \int \dd^3x\, f(t,\ve{x}) = \int_{-\infty}^{+\infty} \dd t\, \frac{\pa}{\pa t} \int \dd^3x\, f(t,\ve{x})
      \label{eq:int-func-lemma}
    \end{equation}
    Applying this to $ f(t,\ve{x}) = -i Z^{-1/2} e^{-i k_\mu x^\mu} \smlra{\pa}_0 \phi(x) $ and using \eref{eq:in-out-lad-op}:
    \begin{equation*}
      \begin{split}
        \sqrt{2E_\ve{p}} ({a_\ve{p}^\text{in}}\dg - {a_\ve{p}^\text{out}}\dg)
        & = \frac{i}{\sqrt{Z}} \int \dd^4x\, \pa_0 \left[ e^{-i k_\mu x^\mu} \smlra{\pa}_0 \phi(x) \right] \\
        & = \frac{i}{\sqrt{Z}} \int \dd^4x\, \pa_0 \left[ e^{-i k_\mu x^\mu} \pa_0 \phi(x) - \phi(x) \pa_0 e^{-i k_\mu x^\mu} \right] \\
        & = \frac{i}{\sqrt{Z}} \int \dd^4x \left[ e^{-i k_\mu x^\mu} \pa_0^2 \phi(x) - \phi(x) \pa_0^2 e^{-i k_\mu x^\mu} \right] \\
        & = \frac{i}{\sqrt{Z}} \int \dd^4x \left[ e^{-i k_\mu x^\mu} \pa_0^2 \phi(x) - \phi(x) (\bs{\nabla}^2 - m^2) e^{-i k_\mu x^\mu} \right]
      \end{split}
    \end{equation*}
    where the last line follows from $ \pa_0^2 e^{-i k_\mu x^\mu} = - k_0^2 e^{-i k_\mu x^\mu} = (\ve{k}^2 - k^2) e^{-i k_\mu x^\mu} = (\bs{\nabla}^2 - m^2) e^{-i k_\mu x^\mu} $. In order to perform integration by parts, note that initial and final particle states are understood to be convoluted to form wave-packets, which are localized in space, while $ \phi(x) $ is not localized in time; hence, the second term can be integrated by parts twice, resulting in:
    \begin{equation*}
      \sqrt{2E_\ve{p}} ({a_\ve{p}^\text{in}}\dg - {a_\ve{p}^\text{out}}\dg) = \frac{i}{\sqrt{Z}} \int \dd^4x\, e^{-i k_\mu x^\mu} (\pa_0^2 - \bs{\nabla}^2 + m^2) \phi(x)
    \end{equation*}
    which is the thesis.
  \end{proof}
\end{proofbox}

Note that if $ \phi(x) $ were a free field, this operator would identically vanish: this is because the KG scalar product is time-independent in a free field theory, therefore the ladder operators too are time-independent. \\
It is now possible to write the amplitude for a generic scattering process (i.e. $ S $-matrix element). In particular, consider a process $ \ket{\ve{k}_1, \dots, \ve{k}_n; -\infty} \rightarrow \ket{\ve{p}_1, \dots, \ve{p}_m; + \infty} $ with $ \ve{k}_i \neq \ve{p}_j \,\,\forall i = 1, \dots, n,\, j = 1, \dots, m $: in this case, as there are no particles which remain unchanged by the interaction, the $ S $-matrix reduces to the $ T $-matrix.

\begin{theorem}{LSZ reduction formula for scalar fields}{lsz-scalar}
  The amplitude for $ \ket{\ve{k}_1 , \dots , \ve{k}_n; -\infty} \rightarrow \ket{\ve{p}_1 , \dots , \ve{p}_m; +\infty} : \ve{k}_i \neq \ve{p}_j $ is:
  \begin{multline}
    \braket{\ve{p}_1 , \dots , \ve{p}_m ; +\infty | \ve{k}_1 , \dots , \ve{k}_n ; -\infty} = \\
    = \prod_{j = 1}^n \frac{k_j^2 - m^2}{i \sqrt{Z}} \int \dd^4x_j\, e^{-i {k_j}_\mu {x_j}^\mu} \prod_{k = 1}^m \frac{p_k^2 - m^2}{i \sqrt{Z}} \int \dd^4y_k\, e^{+i {p_k}_\mu {y_k}^\mu} \times \\
    \times \braket{0 | \tempord\{\phi(x_1) \dots \phi(x_n) \phi(y_1) \dots \phi(y_m)\} | 0}
  \end{multline}
  where $ \tempord $ is the chronological product:
  \begin{equation}
    \begin{split}
      \tempord\{f(x) g(y)\} & \defeq
      \begin{cases}
        g(y) f(x) & y^0 > x^0 \\
        f(x) g(y) & y^0 < x^0
      \end{cases} \\
      & = \theta(y^0 - x^0) g(y) f(x) + \theta(x^0 - y^0) f(x) g(y)
    \end{split}
  \end{equation}
  where $ \theta(x) $ is the Heaviside distribution.
\end{theorem}

\begin{proofbox}
  \begin{proof}
    Using \eref{eq:in-out-lad-op} it is possible to extract a particle from the initial state:
    \begin{equation*}
      \begin{split}
        \braket{\ve{p}_1,\dots,\ve{p}_m;+\infty | \ve{k}_1,\dots,\ve{k}_n;-\infty}
        & = \sqrt{2E_{\ve{k}_1}} \braket{\ve{p}_1,\dots,\ve{p}_m;+\infty | {a_{\ve{k}_1}^\text{in}}\dg | \ve{k}_2,\dots,\ve{k}_n;-\infty} \\
        & = \sqrt{2E_{\ve{k}_1}} \braket{\ve{p}_1,\dots,\ve{p}_m;+\infty | {a_{\ve{k}_1}^\text{in}}\dg - {a_{\ve{k}_1}^\text{out}}\dg | \ve{k}_2,\dots,\ve{k}_n;-\infty}
      \end{split}
    \end{equation*}
    as $ {a_{\ve{k}_1}^\text{out}} \ket{\ve{p}_1,\dots,\ve{p}_n;+\infty} = 0 $ since $ \ve{k}_i \neq \ve{p}_j $. Then, by \lref{lemma:in-out-rel}:
    \begin{equation*}
      \begin{split}
        \braket{\ve{p}_1,\dots,\ve{p}_m;+\infty | \ve{k}_1,\dots,\ve{k}_n;-\infty}
        & = \frac{i}{\sqrt{Z}} \int \dd^4x_1\, e^{-i {k_1}_\mu {x_1}^\mu} (\Box_{x_1} + m^2) \times \\
        & \qquad \qquad \times \braket{\ve{p}_1,\dots,\ve{p}_m;+\infty | \phi(x_1) | \ve{k}_2,\dots,\ve{k}_n;-\infty}
      \end{split}
    \end{equation*}
    Using the same argument (noting that $ \tempord\{a_{\ve{p}_1}^\text{in} \phi(x_1)\} = \phi(x_1) a_{\ve{p}_1}^\text{in} $, $ \tempord\{a_{\ve{p}_1}^\text{out} \phi(x_1)\} = a_{\ve{p}_1}^\text{out} \phi(x_1) $):
    \begin{equation*}
      \begin{split}
        & \braket{\ve{p}_1,\dots,\ve{p}_m;+\infty | \phi(x_1) | \ve{k}_2,\dots,\ve{k}_n;-\infty} = \\
        & \qquad = \sqrt{2E_{\ve{p}_1}} \braket{\ve{p}_2,\dots,\ve{p}_m;+\infty | \tempord\{(a_{\ve{p}_1}^\text{out} - a_{\ve{p}_1}^\text{in}) \phi(x_1)\} | \ve{k}_2,\dots,\ve{k}_n;-\infty} \\
        & \qquad = \frac{i}{\sqrt{Z}} \int \dd^4y_1\, e^{+i {p_1}_\mu {y_1}^\mu} (\Box_{y_1} + m^2) \braket{\ve{p}_2,\dots,\ve{p}_m;+\infty | \tempord\{\phi(y_1) \phi(x_1)\} | \ve{k}_2,\dots,\ve{k}_n;-\infty}
      \end{split}
    \end{equation*}
    This procedure can be iterated\footnote{Technically, $ \Box_{y_1} $ cannot be extracted from $ \tempord $, as it does not commute with the Heaviside distribution in its definition. However, the extraction of $ \Box_{y_1} $ can be performed accounting for an additional term proportional to $ \pa_0 \theta({y_1}^0 - {x_1}^0) [\pa_0 \phi(y_1),\phi(x_1)] \sim \delta^{(4)}(y_1 - x_1) $, which does not alter the singular structure of the LSZ formula, thus leaving the residue and the resulting amplitude unchanged.}, obtaining:
    \begin{equation*}
      \begin{split}
        & \braket{\ve{p}_1,\dots\ve{p}_m;+\infty | \ve{k}_1,\dots,\ve{k}_n;-\infty} = \prod_{j = 1}^n \frac{i}{\sqrt{Z}} \int \dd^4x_j\, e^{-i {k_j}_\mu {x_j}^\mu} \prod_{k = 1}^m \frac{i}{\sqrt{Z}} \int \dd^4y_k\, e^{+i {p_k}_\mu {x_k}^\mu} \times \\
        & \qquad \qquad \qquad \qquad \qquad \qquad \times (\Box_{x_j} + m^2) (\Box_{y_k} + m^2) \braket{0 | \tempord\{\phi(x_1) \dots \phi(x_n) \phi(y_1) \dots \phi(y_m)\} | 0}
      \end{split}
    \end{equation*}
    It is possible to define the \bctxt{$ N $-point Green function} as:
    \begin{equation}
      G(x_1, \dots, x_N) \equiv \braket{0 | \tempord\{\phi(x_1) \dots \phi(x_N)\} | 0}
    \end{equation}
    In terms of its (4-dimensional) Fourier transform it reads:
    \begin{equation*}
      G(x_1, \dots, x_N) = \prod_{j = 1}^N \int \frac{\dd^4\xi_j}{(2\pi)^4} e^{i {\xi_j}_\mu {x_j}^\mu} \tilde{G}(\xi_1, \dots, \xi_N)
    \end{equation*}
    Then:
    \begin{equation*}
      (\Box_{x_j} + m^2) G(x_1, \dots, x_N) = - \prod_{j = 1}^N \int \frac{\dd^4\xi_j}{(2\pi)^4} e^{i {\xi_j}_\mu {x_j}^\mu} (\xi_j^2 - m^2) \tilde{G}(\xi_1, \dots, \xi_N)
    \end{equation*}
    Substituting into the above expression:
    \begin{equation*}
      \begin{split}
        & \prod_{j = 1}^N \frac{i}{\sqrt{Z}} \int \dd^4x_j e^{-i {k_j}_\mu {x_j}^\mu} (\Box_{x_j} + m^2) G(x_1, \dots, x_N) = \\
        & \qquad \qquad = - \prod_{j = 1} \frac{i}{\sqrt{Z}} \int \frac{\dd^4x_j \dd^4\xi_j}{(2\pi)^4} e^{i (\xi_j - k_j)_\mu x^\mu} (\xi_j^2 - m^2) \tilde{G}(\xi_1, \dots, \xi_N) \\
        & \qquad \qquad = \prod_{j = 1}^N \frac{1}{i \sqrt{Z}} \int \dd^4\xi_j\, \delta^{(4)}(\xi_j - k_j) (\xi_j^2 - m^2) \tilde{G}(\xi_1, \dots, \xi_N) \\
        & \qquad \qquad = \prod_{j = 1}^N \frac{k_j^2 - m^2}{i \sqrt{Z}} \tilde{G}(k_1, \dots, k_N) = \prod_{j = 1}^N \frac{k_j^2 - m^2}{i\sqrt{Z}} \int \dd^4x_j\, e^{-i {k_j}_\mu {x_j}^\mu} G(x_1, \dots, x_N)
      \end{split}
    \end{equation*}
    Using $ G(x_1, \dots, x_n , y_1, \dots, y_m) =  \braket{0 | \tempord\{\phi(x_1) \dots \phi(x_n) \phi(y_1) \dots \phi(y_m)\} | 0} $ yields the thesis.
  \end{proof}
\end{proofbox}

Going on mass-shell $ p^2 \rightarrow m^2 $, the Green function (or \bctxt{correlation function}) can be shown to have singularities $ \sim (p^2 - m^2)^{-1} $ for each particle involved: these poles are precisely cancelled by the factors in the LSZ reduction formula, thus leaving a finite result. In this sense, the amplitude of the process can be seen as the multi-pole residue of the Green function.

\subsubsection{Spinor fields}

Consider now a scattering process of a single species of spin-$ \frac{1}{2} $ particles (and anti-particles). The adiabatic hypothesis now yields (recall \eeref{eq:dirac-ladd-op-a}{eq:dirac-ladd-op-b}):
\begin{equation}
  \sqrt{2E_\ve{p}} {a_{\ve{p},s}^\text{in,out}}\dg = \int_{t \rightarrow \mp \infty} \dd^3x\, e^{-i p_\mu x^\mu} \bar{\Psi}_\text{in,out}(x) \gamma^0 u^s(p) = \frac{1}{\sqrt{Z}} \lim_{t \rightarrow \mp \infty} \int \dd^3x\, e^{-i p_\mu x^\mu} \bar{\Psi}(x) \gamma^0 u^s(p)
\end{equation}
\begin{equation}
  \sqrt{2E_\ve{p}} {b_{\ve{p},s}^\text{in,out}}\dg = \int_{t \rightarrow \mp \infty} \dd^3x\, e^{i p_\mu x^\mu} \bar{v}^s(p) \gamma^0 \Psi_\text{in,out}(x) = \frac{1}{\sqrt{Z}} \lim_{t \rightarrow \mp \infty} \int \dd^3x\, e^{i p_\mu x^\mu} \bar{v}^s(p) \gamma^0 \Psi(x)
\end{equation}

\begin{lemma}[before upper = {\tcbtitle}]{}{}
  \begin{equation}
    \sqrt{2E_\ve{p}} ({a_{\ve{p},s}^\text{in}}\dg - {a_{\ve{p},s}^\text{out}}\dg) = \frac{i}{\sqrt{Z}} \int \dd^4x\, \bar{\Psi}(x) ( i \smla{\slashed{\pa}} + m ) u^s(p) e^{-i p_\mu x^\mu}
  \end{equation}
  \begin{equation}
    \sqrt{2E_\ve{p}} ({a_{\ve{p},s}^\text{in}} - {a_{\ve{p},s}^\text{out}}) = \frac{i}{\sqrt{Z}} \int \dd^4x\, e^{i p_\mu x^\mu} \bar{u}^s(p) ( - i \smra{\slashed{\pa}} + m ) \Psi(x)
  \end{equation}
  \begin{equation}
    \sqrt{2E_\ve{p}} ({b_{\ve{p},s}^\text{in}}\dg - {b_{\ve{p},s}^\text{out}}\dg) = - \frac{i}{\sqrt{Z}} \int \dd^4x\, e^{i p_\mu x^\mu} \bar{v}^s(p) ( - i \smra{\slashed{\pa}} + m ) \Psi(x)
  \end{equation}
  \begin{equation}
    \sqrt{2E_\ve{p}} ({b_{\ve{p},s}^\text{in}} - {b_{\ve{p},s}^\text{out}}) = - \frac{i}{\sqrt{Z}} \int \dd^4x\, \bar{\Psi}(x) ( i \smla{\slashed{\pa}} + m ) v^s(p) e^{-i p_\mu x^\mu}
  \end{equation}
\end{lemma}

\begin{proofbox}
  \begin{proof}
    Using \eref{eq:int-func-lemma}:
    \begin{equation*}
      \begin{split}
        \sqrt{2E_\ve{p}} ({a_{\ve{p},s}^\text{in}}\dg - {a_{\ve{p},s}^\text{out}}\dg)
        & = - \frac{1}{\sqrt{Z}} \int \dd^4x\, \pa_0 \left[ e^{-i p_\mu x^\mu} \bar{\Psi}(x) \gamma^0 u^s(p) \right] \\
        & = - \frac{1}{\sqrt{Z}} \int \dd^4x \left[ \pa_0 e^{-i p_\mu x^\mu} \bar{\Psi}(x) + e^{-i p_\mu x^\mu} \pa_0 \bar{\Psi}(x) \right] \gamma^0 u^s(p) \\
        & = - \frac{1}{\sqrt{Z}} \int \dd^4x\, \bar{\Psi}(x) \left[ -i \gamma ^0 p_0 + \gamma^0 \smla{\pa}_0 \right] u^s(p) e^{-i p_\mu x^\mu}
      \end{split}
    \end{equation*}
    By \eref{eq:dirac-eq-spinors} $ (\gamma^0 p_0 - \gamma^k p_k - m) u^s(p) = 0 $, so:
    \begin{equation*}
      \begin{split}
        \sqrt{2E_\ve{p}} ({a_{\ve{p},s}^\text{in}}\dg - {a_{\ve{p},s}^\text{out}}\dg)
        & = - \frac{1}{\sqrt{Z}} \int \dd^4x\, \bar{\Psi}(x) \left[ -i \gamma ^k p_k - i m + \gamma^0 \smla{\pa}_0 \right] u^s(p) e^{-i p_\mu x^\mu} \\
        & = - \frac{1}{\sqrt{Z}} \int \dd^4x\, \bar{\Psi}(x) \left[ \gamma^0 \smla{\pa}_0 + \gamma^k \smra{\pa}_k - im \right] u^s(p) e^{-i p_\mu x^\mu}
      \end{split}
    \end{equation*}
    Assuming that initial and final particle states are convoluted to form wave-packets, integartion by parts in spatial dimensions can be carried out, yielding:
    \begin{equation*}
      \begin{split}
        \sqrt{2E_\ve{p}} ({a_{\ve{p},s}^\text{in}}\dg - {a_{\ve{p},s}^\text{out}}\dg)
        & = - \frac{1}{\sqrt{Z}} \int \dd^4x\, \bar{\Psi}(x) \left[ \gamma^0 \smla{\pa}_0 - \gamma^k \smla{\pa}_k - im \right] u^s(p) e^{-i p_\mu x^\mu} \\
        & = \frac{i}{\sqrt{Z}} \int \dd^4x\, \bar{\Psi}(x) \left[ i \smla{\slashed{\pa}} + m \right] u^s(p) e^{-i p_\mu x^\mu}
      \end{split}
    \end{equation*}
    Other operators are found analogously.
  \end{proof}
\end{proofbox}

Note that, like \eref{eq:lsz-scal-op}, these operators vanish identically when dealing with a free field. \\
Consider now a general scattering process involving both fermions and anti-fermions (quantities denoted by a tilde) $ \ket{\ve{k}_1, \dots, \ve{k}_{n_1} , \tilde{\ve{k}}_1, \dots, \tilde{\ve{k}}_{n_2} ; -\infty} \rightarrow \ket{\ve{p}_1, \dots, \ve{p}_{m_1} , \tilde{\ve{p}}_1, \dots, \tilde{\ve{p}}_{m_2} ; +\infty} $ in which no particles remain unchanged by the interaction, i.e. $ \ve{k}_i \neq \ve{p}_j , \tilde{\ve{k}}_k \neq \tilde{\ve{p}}_l \,\,\forall i = 1,\dots,n_1 ,\, j = 1,\dots,m_1 ,\, k = 1,\dots,n_2 ,\, l = 1,\dots,m_2 $, so that the $ S $-matrix reduces to the $ T $-matrix. By the same reasoning of \tref{th:lsz-scalar}, it is clear how to extract particles from this generic amplitude:
\begin{align}
  \braket{\beta_\text{fin} ; + \infty | \ve{k},s ; -\infty} &\mapsto \frac{i}{\sqrt{Z}} \int \dd^4x \sum_{\sigma = 1}^{4} \braket{\beta_\text{fin} | [\bar{\Psi}(x)]_\sigma | 0} [( i\smla{\slashed{\pa}} + m ) u^s(k)]_\sigma e^{-i k_\mu x^\mu} \\
  \braket{\beta_\text{fin} ; + \infty | \tilde{\ve{k}},\tilde{s} ; -\infty} &\mapsto -\frac{i}{\sqrt{Z}} \int \dd^4\tilde{x} \sum_{\sigma = 1}^{4} e^{i \tilde{k}_\mu \tilde{x}^\mu} [\bar{v}^{\tilde{s}}(\tilde{k})( -i\smra{\slashed{\pa}} + m )]_\sigma \braket{\beta_\text{fin} | [\Psi(\tilde{x})]_\sigma | 0} \\
  \braket{\ve{p},s ; +\infty | \alpha_\text{in} ; -\infty} &\mapsto \frac{i}{\sqrt{Z}} \int \dd^4y \sum_{\sigma = 1}^{4} e^{i p_\mu y^\mu} [\bar{u}^s(p) ( -i \smra{\slashed{\pa}} + m )]_\sigma \braket{0 | [\Psi(y)]_\sigma | \alpha_\text{in}} \\
  \braket{\tilde{\ve{p}},\tilde{s} ; +\infty | \alpha_\text{in} ; -\infty} &\mapsto -\frac{i}{\sqrt{Z}} \int \dd^4\tilde{y} \sum_{\sigma = 1}^{4} \braket{0 | [\bar{\Psi}(\tilde{y})]_\sigma | \alpha_\text{in}} [( i \smla{\slashed{\pa}} + m ) v^{\tilde{s}}(\tilde{p})]_\sigma e^{-i \tilde{p}_\mu \tilde{y}^\mu}
\end{align}
In general, the final $ \virgolette{reduced} $ amplitude formula will contain a $ (n_1 + n_2 + m_1 + m_2) $-point Green function like (omitting spinor indices):
\begin{equation*}
  \braket{0 | \tempord\{\bar{\Psi}(x_1) \dots \bar{\Psi}(x_{n_1}) \Psi(\tilde{x}_1) \dots \Psi(\tilde{x}_{n_2}) \Psi(y_1) \dots \Psi(y_{m_1}) \bar{\Psi}(\tilde{y}_1) \dots \bar{\Psi}(\tilde{y}_{m_2})\} | 0}
\end{equation*}

\subsection{Correlation functions}

With the LSZ reduction formulae, the problem of computing scattering amplitudes is reduced to that of computing correlation functions. \\
Consider a generic quantum field $ \phi(x) $ described by a Hamiltonian $ \ham = \ham_0 + \ham_\text{int} $: the exact form of $ \phi(x) $ is in general too difficult to obtain, as it satisfies a complicated non-linear equation of motion, so it cannot be written through an expansion in plane waves. To compute correlation functions, then, it is convenient to define a field related to $ \phi(x) $:
\begin{equation}
  \phi_I(t,\ve{x}) \defeq e^{i H_0 (t-t_0)} \phi(t_0, \ve{x}) e^{-i H_0 (t-t_0)}
\end{equation}
This field evolves with $ H_0 $, i.e. is a free field, and it is called the \bctxt{interaction picture} field. Being this a free field, it can be expanded as:
\begin{equation}
  \phi_I(t,\ve{x}) = \int \frac{\dd^3p}{(2\pi)^3 \sqrt{2E_\ve{p}}} \left[ a_\ve{p} e^{-i p_\mu x^\mu} + a_\ve{p}\dg e^{i p_\mu x^\mu} \right]_{p^0 = E_\ve{p} ,\, x^0 = t - t_0}
\end{equation}

\begin{proposition}{Interaction picture}{}
  The transformation from the interaction picture to the Heisenberg picture is given by a unitary operator:
  \begin{equation}
    \phi(t,\ve{x}) = \tens{U}\dg(t,t_0) \phi_I \tens{U}(t,t_0)
    \label{eq:int-heis-pic}
  \end{equation}
  with:
  \begin{equation}
    \tens{U}(t,t_0) \equiv e^{i H_0 (t-t_0)} e^{-i H (t-t_0)}
  \end{equation}
\end{proposition}

\begin{proofbox}
  \begin{proof}
    By direct calculation:
    \begin{equation*}
      \begin{split}
        \phi(t,\ve{x})
        & = e^{i H (t-t_0)} \phi(t_0,\ve{x}) e^{-i H (t-t_0)} \\
        & = e^{i H (t-t_0)} e^{-i H_0 (t-t_0)} \left[ e^{i H_0 (t-t_0)} \phi(t_0,\ve{x}) e^{-i H_0 (t-t_0)} \right] e^{i H_0 (t-t_0)} e^{-i H (t-t_0)} \\
        & = e^{i H (t-t_0)} e^{-i H_0 (t-t_0)} \phi_I(t,\ve{x}) e^{i H_0 (t-t_0)} e^{-i H (t-t_0)} \equiv \tens{U}\dg(t,t_0) \phi_I(t,\ve{x}) \tens{U}(t,t_0)
      \end{split}
    \end{equation*}
    Unitarity is obvious, as $ H $ is a Hermitian operator.
  \end{proof}
\end{proofbox}

Note that, since $ [H_0,H_\text{int}] \neq 0 $ in general, the two exponentials cannot be combined trivially (the Baker-Campbell-Hausdorff formula should be used).

\begin{proposition}{Time evolution operator}
  The time evolution operator is:
  \begin{equation}
    \tens{U}(t,t_0) = \tempord \exp \left[ -i \int_{t_0}^t \dd\tau\, H_I(\tau) \right]
  \end{equation}
  where:
  \begin{equation}
    H_I(t) \equiv e^{i H_0 (t-t_0)} H_\text{int} e^{-i H_0 (t-t_0)}
  \end{equation}
\end{proposition}

\begin{proofbox}
  \begin{proof}
    First of all, the time evolution operator solves the Schrödinger equation:
    \begin{equation*}
      \begin{split}
        i \frac{\pa}{\pa t} \tens{U}(t,t_0)
        & = e^{i H_0 (t-t_0)} (H - H_0) e^{-i H (t-t_0)} = e^{i H_0 (t-t_0)} H_\text{int} e^{-i H (t-t_0)} \\
        & = e^{i H_0 (t-t_0)} H_\text{int} e^{-i H_0 (t-t_0)} e^{i H_0 (t-t_0)} e^{-i H (t-t_0)} \equiv H_I(t) \tens{U}(t,t_0)
      \end{split}
    \end{equation*}
    The solution to this equation which satisfies the initial condition $ \tens{U}(t_0,t_0) = 1 $ is unique:
    \begin{equation*}
      \begin{split}
        \tens{U}(t,t_0)
        & = \tempord \exp \left[ -i \int_{t_0}^t \dd\tau\, H_I(\tau) \right] = \sum_{n = 0}^{\infty} \frac{(-i)^n}{n!} \int_{[t_0,t]^n} \dd t_1 \dots \dd t_n\, \tempord\{H_I(t_1) \dots H_I(t_n)\} \\
        & = \sum_{n = 0}^{\infty} (-i)^n \int_{t_0}^t \dd t_1 \int_{t_0}^{t_1} \dd t_2 \dots \int_{t_0}^{t_{n-1}} \dd t_n\, H_I(t_1) \dots H_I(t_n)
      \end{split}
    \end{equation*}
    It is in fact clear that:
    \begin{equation*}
      \frac{\pa}{\pa t} \tens{U}(t,t_0) = \sum_{n = 0}^{\infty} (-i)^n \int_{t_0}^{t} \dd t_2 \dots \int_{t_0}^{t_{n-1}} \dd t_n\, H_I(t) H_I(t_2) \dots H_I(t_n) = -i H_I(t) \tens{U}(t,t_0)
    \end{equation*}
    Hence, this is a solution to the above equation.
  \end{proof}
\end{proofbox}

\begin{lemma}[before upper = {\tcbtitle}]{}{time-evol-func}
  \begin{equation}
    \tens{U}(t_1,t_0) \tens{U}(t_0,t_2) = \tens{U}(t_1,t_2)
  \end{equation}
\end{lemma}

\begin{proofbox}
  \begin{proof}
    Observe that:
    \begin{equation*}
      i \frac{\pa}{\pa t} [\tens{U}(t,t_0) \tens{U}(t_0,t_2)] = \left[ i \frac{\pa}{\pa t} \tens{U}(t,t_0) \right] \tens{U}(t_0,t_2) = H_I(t) \tens{U}(t,t_0) \tens{U}(t_0,t_2)
    \end{equation*}
    The boundary condition now reads $ [\tens{U}(t,t_0) \tens{U}(t_0,t_2)]_{t = t_0} = \tens{U}(t_0,t_2) $, and the unique solution for this boundary condition is:
    \begin{equation*}
      \tens{U}(t,t_0) \tens{U}(t_0,t_2) = \tempord \exp \left[ -i \int_{t_0}^t \dd\tau\, H_I(\tau) \right]
    \end{equation*}
    which is precisely $ \tens{U}(t,t_2) $.
  \end{proof}
\end{proofbox}

It is now possible to compute correlation functions.

\begin{theorem}{Correlation functions}{}
  Given a quantum field $ \phi(x) $ described by a Hamiltonian $ \ham = \ham_0 + \ham_\text{int} $, then:
  \begin{equation}
    \braket{0 | \tempord\{\phi(x_1) \dots \phi(x_n)\} | 0} = \frac{\braket{0 | \tempord\{\phi_I(x_1) \dots \phi_I(x_n) \exp \left[ -i \int \dd^4x\, \ham_I \right]\} | 0}}{\braket{0 | \tempord\{\exp \left[ -i \int \dd^4x\, \ham_I \right]\} |0}}
    \label{eq:corr-func-pert}
  \end{equation}
\end{theorem}

\begin{proofbox}
  \begin{proof}
    Consider WLOG $ \{t_k\}_{k = 1,\dots,n} : t_i < t_j $. Using \eref{eq:int-heis-pic} and \lref{lemma:time-evol-func}:
    \begin{equation*}
      \begin{split}
        \braket{0 | \phi(x_1) \dots \phi(x_n) | 0}
        & = \braket{0 | \tens{U}\dg(t_1,t_0) \phi_I(x_1) \tens{U}(t_1,t_0) \tens{U}\dg(t_2,t_0) \dots \tens{U}\dg(t_n,t_0) \phi_I(x_n) \tens{U}(t_n,t_0) | 0} \\
        & = \braket{0 | \tens{U}\dg(t_1,t_0) \phi_I(x_1) \tens{U}(t_1,t_2) \dots \tens{U}(t_{n-1},t_n) \phi_I(x_n) \tens{U}(t_n,t_0) | 0}
      \end{split}
    \end{equation*}
    Consider $ t \in \R : t \gg t_1 > \dots > t_n \gg -t $, so that $ \tens{U}(t_n,t_0) = \tens{U}(t_n,-t) \tens{U}(-t,t_0) $ and $ \tens{U}\dg(t_1,t_0) = \tens{U}\dg(t,t_0) \tens{U}(t,t_1) $. Therefore:
    \begin{equation*}
      \begin{split}
        & \braket{0 | \phi(x_1) \dots \phi(x_n) | 0} \\
        & = \braket{0 | \tens{U}\dg(t,t_0) [\tens{U}(t,t_1) \phi_I(x_1) \tens{U}(t_1,t_2) \phi_I(x_2) \tens{U}(t_2,t_3) \dots \tens{U}(t_{n-1},t_n) \phi_I(x_n) \tens{U}(t_n,-t)] \tens{U}(-t,t_0) | 0}
      \end{split}
    \end{equation*}
    Note that this expression inside the square brackets is automatically time-ordered, so it may be rewritten as:
    \begin{equation*}
      \begin{split}
        [\dots]
        & = \tempord\{\phi_I(x_1) \dots \phi_I(x_n) \tens{U}(t,t_1) \tens{U}(t_1,t_2) \dots \tens{U}(t_n,-t)\} \\
        & = \tempord\bigg\{ \phi_I(x_1) \dots \phi_I(x_n) \exp \left[ -i \int_{-t}^t \dd\tau\, H_I(\tau) \right]\bigg\}
      \end{split}
    \end{equation*}
    These considerations hold for arbitrary $ t_0 $, so it can be set $ t_0 = -t \rightarrow -\infty $. Then:
    \begin{equation*}
      \braket{0 | \tempord\{\phi(x_1) \dots \phi(x_n)\} | 0} = \braket{0 | \tens{U}\dg(+\infty,-\infty) \tempord\bigg\{ \phi_I(x_1) \dots \phi_I(x_n) \exp \left[ -i \int \dd^4x\, \ham_I \right] \bigg\} | 0}
    \end{equation*}
    Note that $ \bra{0} \tens{U}\dg(\infty,-\infty) $ is the Hermitian conjugate of $ \tens{U}(\infty,-\infty) \ket{0} $, i.e. the state obtained evolving in time the vacuum state. As it was already assumed, the initial-state vacuum coincides physically to the final-state vacuum, i.e.:
    \begin{equation*}
      \tens{U}(\infty,-\infty) \ket{0} = e^{i\alpha} \ket{0}
      \qquad \Rightarrow \qquad
      e^{-i \alpha} = \left( \braket{0 | \tempord\bigg\{ \exp \left[ -i \int \dd^4x\, \ham_I \right] \bigg\} | 0} \right)^{-1}
    \end{equation*}
    Finally:
    \begin{equation*}
    \braket{0 | \tempord\{\phi(x_1) \dots \phi(x_n)\} | 0} = \frac{\braket{0 | \tempord\{\phi_I(x_1) \dots \phi_I(x_n) \exp \left[ -i \int \dd^4x\, \ham_I \right]\} | 0}}{\braket{0 | \tempord\{\exp \left[ -i \int \dd^4x\, \ham_I \right]\} |0}}
    \end{equation*}
    which is the thesis.
  \end{proof}
\end{proofbox}

This allows to compute correlation functions from free fields, rather than interaction fields. Moreover, note that the functional dependence of $ \ham_I $ in terms of $ \phi_I(x) $ is the same as that of $ \ham_\text{int} $ in terms of $ \phi(x) $.

\begin{example}{$ \phi^4 $ potential}{}
  Consider the $ \phi^4 $-interaction Hamiltonian $ \ham_\text{int} = \frac{\lambda}{4!} \phi^4 $. Then:
  \begin{equation*}
    \begin{split}
      \ham_I = \frac{\lambda}{4!} e^{i H_0 (t-t_0)} \phi^4 e^{-i H_0 (t-t_0)} = \frac{\lambda}{4!} \left[ e^{i H_0 (t-t_0)} \phi e^{-i H_0 (t-t_0)} \right]^4 = \frac{\lambda}{4!} \phi_I^4
    \end{split}
  \end{equation*}
\end{example}

Note that \eref{eq:corr-func-pert} is naturally suited for perturbative evaluation with respect to coupling constants which appear in $ \ham_\text{int} $, thanks to the exponential function.

\section{Feynman diagrams}

The problem of computing scattering amplitudes has been further reduced to that of computing $ n $-point Green functions of free (interaction picture) fields.

\subsection{Feynman propagator}

\begin{theorem}{Feynman propagator}{}
  Given a real scalar field $ \phi(x) $, the \bcth{Feynman propagator} is computed as:
  \begin{equation}
    D_\text{F}(x-y) \defeq \braket{0 | \tempord\{\phi_I(x) \phi_I(y)\} | 0} = \int \frac{\dd^4p}{(2\pi)^4} \frac{i}{p^2 - m^2 + i \epsilon} e^{-i p_\mu (x - y)^\mu}
    \label{eq:feynm-prop}
  \end{equation}
  with $ \epsilon \rightarrow 0^+ $.
\end{theorem}

\begin{proofbox}
  \begin{proof}
    First, decompose $ \phi_I(x) $ into its positive- and negative-frequency parts (i.e. annihilation and creation parts):
    \begin{equation*}
      \phi_I^+(x) \equiv \int \frac{\dd^3p}{(2\pi)^3 \sqrt{2E_\ve{p}}} a_\ve{p} e^{-i p_\mu x^\mu}
      \qquad \qquad
      \phi_I^-(x) \equiv \int \frac{\dd^3p}{(2\pi)^3 \sqrt{2E_\ve{p}}} a_\ve{p}\dg e^{i p_\mu x^\mu}
    \end{equation*}
    Clearly $ \phi_I(x)^+ \ket{0} = 0 $ and $ \bra{0} | \phi_I^-(x) = 0 $. Consider first the case $ x^0 > y^0 $:
    \begin{equation*}
      \begin{split}
        \tempord\{\phi_I(x) \phi_I(y)\}
        & = \phi_I^+(x) \phi_I^+(y) + \phi_I^+(x) \phi_I^-(y) + \phi_I^-(x) \phi_I^+(y) + \phi_I^-(x) \phi_I^-(y) \\
        & = \phi_I^+(x) \phi_I^+(y) + \phi_I^-(y) \phi_I^+(x) + \phi_I^-(x) \phi_I^+(y) + \phi_I^-(x) \phi_I^-(y) + [\phi_I^+(x) , \phi_I^-(y)] \\
        & = \normord\{\phi_I(x) \phi_I(y)\} + [\phi^+(x) , \phi^-(y)]
      \end{split}
    \end{equation*}
    Similarly, for $ 0 > x^0 $:
    \begin{equation*}
      \tempord\{\phi_I(x) \phi_I(y)\} = \normord\{\phi_I(x) \phi_I(y)\} + [\phi^+(y) , \phi^-(x)]
    \end{equation*}
    Therefore, in general:
    \begin{equation*}
      \begin{split}
        \tempord\{\phi_I(x) \phi_I(y)\}
        & = \normord\{\phi_I(x) \phi_I(y)\} + \theta(x^0 - y^0) [\phi_I^+(x) , \phi_I^-(y)] + \theta(y^0 - x^0) [\phi_I^+(y) , \phi_I^-(x)] \\
        & \equiv \normord\{\phi_I(x) \phi_I(y)\} + D_\text{F}(x-y)
      \end{split}
    \end{equation*}
    Now, observe that $ \braket{0 | \normord\{\phi_I(x) \phi_I(y)\} | 0} = 0 $, as there is always either an annihilation operator acting on $ \ket{0} $ or a creation operator acting on $ \bra{0} $. On the other hand, $ D_\text{F}(x-y) \in \C $, as $ [a_\ve{p} , a\dg_\ve{q}] \in \C $, therefore $ \braket{0 | D_\text{F}(x-y) | 0} = D_\text{F}(x-y) \braket{0 | 0} = D_\text{F}(x-y) $. There only remains to compute the commutators:
    \begin{equation*}
      \begin{split}
        D_\text{F}(x-y)
        & = \int \frac{\dd^3p \dd^3q}{(2\pi)^6 \sqrt{4E_\ve{p}E_\ve{q}}} \left[ \theta(x^0 - y^0) [a_\ve{p} , a\dg_\ve{q}] e^{-i p_\mu x^\mu + i q_\mu y^\mu} + \theta(y^0 - x^0) [a_\ve{p} , a\dg_\ve{q}] e^{-i p_\mu y^\mu + i q_\mu x^\mu} \right] \\
        & = \int \frac{\dd^3p \dd^3q}{(2\pi)^3 \sqrt{4E_\ve{p}E_\ve{q}}} \delta^{(3)}(\ve{p} - \ve{q}) \left[ \theta(x^0 - y^0) e^{-i p_\mu x^\mu + i q_\mu y^\mu} + \theta(y^0 - x^0) e^{-i p_\mu y^\mu + i q_\mu x^\mu} \right] \\
        & = \int \frac{\dd^3p}{(2\pi)^3 2E_\ve{p}} \left[ \theta(x^0 - y^0) e^{-i p_\mu (x-y)^\mu} + \theta(y^0 - x^0) e^{i p_\mu (x-y)^\mu} \right]
      \end{split}
    \end{equation*}
    To show that this is equivalent to the thesis, note that \eref{eq:feynm-prop} can be rewritten as:
    \begin{equation*}
      D_\text{F}(x-y) = \int \frac{\dd^3p}{(2\pi)^3} e^{i \ve{p} \cdot (\ve{x} - \ve{y})} \int_\R \frac{\dd p^0}{2\pi} \frac{i}{(p^0)^2 - E_\ve{p}^2 + i\epsilon} e^{-i p^0 (x^0 - y^0)}
    \end{equation*}
    as $ E_\ve{p} = + \sqrt{\ve{p}^2 + m^2} $. The integral in $ \dd p^0 $ can be computed in the complex $ p^0 $-plane\footnote{As the integrand is $ \sim (p^0)^{-2} $, the contribution of the curved part of a semicircular contour vanishes, only leaving the integral over the real axis.}: the $ i\epsilon $-prescription slightly displaces the poles from the real axis\footnote{The poles are $ p^0 = \pm \sqrt{E_\ve{p}^2 - i\epsilon} \simeq \pm \left( E_\ve{p} - \frac{i\epsilon}{2E_\ve{p}} \right) $.}, and in particular that in $ p^0 = + E_\ve{p} $ is slightly below it while that in $ p^0 = - E_\ve{p} $ is slightly above it. If $ x^0 - y^0 > 0 $, choose a semicircular (clockwise) contour in the lower-half plane, so that only the $ p^0 = E_\ve{p} $ singularity is enclosed:
    \begin{equation*}
      D_\text{F}(x-y)\vert_{x^0 > y^0} = \int \frac{\dd^3p}{(2\pi)^3} e^{i \ve{p} \cdot (\ve{x} - \ve{y})} \frac{i}{2\pi} \left[ (-2\pi i) \frac{e^{-i E_\ve{p} (x-y)}}{2E_\ve{p}} \right] = \int \frac{\dd^3p}{(2\pi)^3 2E_\ve{p}} e^{-i p_\mu (x-y)^\mu}
    \end{equation*}
    If instead $ x^0 - y^0 < 0 $, choose a semicircular (counter-clockwise) contour in the upper-half plane:
    \begin{equation*}
      D_\text{F}(x-y)\vert_{y^0 > x^0} = \int \frac{\dd^3p}{(2\pi)^3} e^{i \ve{p} \cdot (\ve{x} - \ve{y})} \frac{i}{2\pi} \left[ (2\pi i) \frac{e^{i E_\ve{p} (x^0 - y^0)}}{-2E_\ve{p}} \right] = \int \frac{\dd^3p}{(2\pi)^3 2E_\ve{p}} e^{i p_\mu (x-y)^\mu}
    \end{equation*}
    where in the last integral $ \ve{p} \rightarrow -\ve{p} $ was renamed. The proof is complete.
  \end{proof}
\end{proofbox}

The expression for the Feynman propagator in momentum space is trivially found:
\begin{equation}
  \tilde{D}_\text{F}(p) = \frac{i}{p^2 - m^2 + i\epsilon}
\end{equation}
Moreover, the Feynman propagator is just a Green function for the KG operator:
\begin{equation*}
  (\Box_x + m^2) D_\text{F}(x-y) = \int \frac{\dd^4p}{(2\pi)^4} \frac{i}{p^2 - m^2 + i\epsilon} (-p^2 + m^2) e^{-i p_\mu (x-y)^\mu} = -i \delta^{(4)}(x-y)
\end{equation*}
Note that this result is independent of the $ i\epsilon $-prescription.

\subsection{Wick's theorem}










