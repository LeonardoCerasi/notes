\selectlanguage{english}

\section{Continuous limit}

\subsection{One-dimensional harmonic crystal}

Consider a simple one-dimensional model of a crystal where atoms of mass $ m \equiv 1 $ lie at rest on the $ x $-axis, with equilibrium positions labelled by $ n \in \N $ and equally spaced by a distance $ a $.\\
Assuming these atoms are free to vibrate only in the $ x $ direction (longitudinal waves), and denoting the displacement of the atom at position $ n $ as $ \eta_n $, one can write the Lagrangian for a \textit{harmonic crystal} as:
\begin{equation}
  L = \sum_{n} \left[ \frac{1}{2} \dot{\eta}_n^2 - \frac{\lambda}{2} \left( \eta_n - \eta_{n-1} \right)^2 \right]
  \label{eq:1.1}
\end{equation}
where $ \lambda $ is the spring constant. From the Lagrange equations, the classical equations of motions are:
\begin{equation}
  \ddot{\eta}_n = \lambda \left( \eta_{n+1} - 2 \eta_n + \eta_{n-1} \right)
  \label{eq:1.2}
\end{equation}
The solutions can be written as complex travelling waves:
\begin{equation}
  \eta_n (t) = e^{i \left( k n - \omega t \right)}
  \label{eq:1.3}
\end{equation}
where the dispersion relation is:
\begin{equation}
  \omega^2 = 2\lambda \left( 1 - \cos k \right) \approx \lambda k^2
  \label{eq:1.4}
\end{equation}
Therefore, in the long-wavelength limit $ k \ll 1 $ waves propagate with velocity $ c = \sqrt{\lambda} $. To determine the normal modes, there need to be boundary conditions: imposing boundary conditions:
\begin{equation}
  \eta_{n + N} = \eta_n \qquad \Rightarrow \qquad k_m = \frac{2\pi m}{N} \,,\, m = 0, 1, \dots, N - 1
  \label{eq:1.5}
\end{equation}
The normal-mode expansion can then be written as:
\begin{equation}
  \eta (t) = \sum_{m = 0}^{N - 1} \left[ \mathcal{A}_m e^{i \left( k_m n - \omega_m t \right)} + \mathcal{A}^* e^{-i \left( k_m n - \omega_m t \right)} \right]
  \label{eq:1.6}
\end{equation}
where the complex conjugate is added to ensure that the total displacement is real. The momentum canonically-conjucated to the displacement is defined as:
\begin{equation}
  \pi_n \defeq \frac{\pa L}{\pa \dot{\eta}_n} = \dot{\eta}_n
  \label{eq:1.7}
\end{equation}
In quantum mechanics, $ \eta_n $ and $ \Pi_n $ become operators with canonical commutator $ [ \hat{\eta}_j, \hat{\pi}_k ] = i \hbar \delta_{jk} $. Implementing time evolution with the \textit{Heisenberg picture}\footnote{Recall that $ \hat{\mathcal{O}}(t) = e^{\frac{i}{\hbar} \hat{\mathcal{H}} t} \hat{\mathcal{O}}(0) e^{-\frac{i}{\hbar} \hat{\mathcal{H}} t} $ and $ \frac{\dd \hat{\mathcal{O}}}{\dd t} = \frac{i}{\hbar} [ \hat{\mathcal{H}}, \hat{\mathcal{O}} ] $.}:
\begin{equation}
  [ \hat{\eta}_j(t), \hat{\pi}_k(t) ] = i \hbar \delta_{jk}
  \label{eq:1.8}
\end{equation}
The commutator of operators evaluated at different times requires solving the dynamics of the system. It is useful to introduce \textit{annihilation} and \textit{creation operators} $\footnote{For a harmonic oscillator $ \hat{\mathcal{H}} = \frac{1}{2} \hat{p}^2 + \frac{1}{2} \omega^2 \hat{x}^2 $, so $ \frac{\dd \hat{x}}{\dd t} = \hat{p}(t) $ and $ \frac{\dd \hat{p}}{\dd t} = -\omega^2 \hat{x}(t) $ and the solution can be written as: $$ \hat{x}(t) = \sqrt{\frac{\hbar}{2\omega}} \left[ \hat{a}(t) + \hat{a}^\dagger(t) \right] \qquad \qquad \hat{p}(t) = -i \omega \sqrt{\frac{\hbar}{2\omega}} \left[ \hat{a}(t) - \hat{a}^\dagger(t) \right] $$ Inverting these expressions one finds $ [\hat{a}(t), \hat{a}^\dagger(t)] = 1 $ and $ \hat{\mathcal{H}} = \hbar \omega \left( \hat{a}^\dagger(t) \hat{a} + \frac{1}{2} \right) $. The time evolution $ \hat{a}(t) = e^{-i \omega t} \hat{a}(0) $ ensures that $ \hat{\mathcal{H}} $ is times-independent.} \hat{a}(t) $ and $ \hat{a}^\dagger(t) $, so that Eq. \ref{eq:1.6} becomes:
\begin{equation}
  \hat{\eta}_n(t) = \sum_{m = 0}^{N - 1} \sqrt{\frac{\hbar}{2 \omega_m}} \frac{1}{\sqrt{N}} \left[ e^{i \left( k_m n - \omega_m t \right)} \hat{a}_m + e^{-i \left( k_m n - \omega_m t \right)} \hat{a}_m^\dagger \right]
  \label{eq:1.9}
\end{equation}
where $ [ \hat{a}_j, \hat{a}_k^\dagger ] = \delta_{jk} $ and the $ N^{-1/2} $ ensures the normalization of normal modes. The proof of Eq. \ref{eq:1.8} follows from the finite Fourier series identity (sum of a geometric progression):
\begin{equation}
  \sum_{m = 0}^{N - 1} e^{i k_m \left( n - n' \right)} = N \delta_{n n'}
  \label{eq:1.10}
\end{equation}
The Hamiltonian of the system can be written as:
\begin{equation}
  \hat{\mathcal{H}} = \sum_{n} \left[ \frac{1}{2} \hat{\pi}_n^2 + \frac{\lambda}{2} \left( \hat{\eta}_n - \hat{\eta}_{n-1} \right)^2 \right] = \sum_{m = 0}^{N - 1} \hbar \omega_m \left( \hat{a}_m^\dagger \hat{a}_m + \frac{1}{2} \right)
  \label{eq:1.11}
\end{equation}
Generalizing the harmonic oscillator operator algebra (proven unique by Von Neumann), one can construct the Hilbert space for the harmonic crystal as:
\begin{equation}
  \hat{a}_m \ket{0} \quad \forall m = 0, 1, \dots, N - 1
  \label{eq:1.12}
\end{equation}
\begin{equation}
  \ket{n_0, n_1, \dots, n_{N-1}} = \prod_{m = 0}^{N - 1} \frac{( \hat{a}_m^\dagger )^{n_m}}{\sqrt{n_m!}} \ket{0}
  \label{eq:1.13}
\end{equation}
These are normalized eigenstates of Eq. \ref{eq:1.1} with energy eigenvalues:
\begin{equation}
  E_0 = \frac{1}{2} \sum_{m = 0}^{N - 1} \hbar \omega_m
  \label{eq:1.14}
\end{equation}
\begin{equation}
  E_{n_0, n_1, \dots, n_{N-1}} = E_0 + \sum_{m = 0}^{N - 1} n_m \hbar \omega_m
  \label{eq:1.15}
\end{equation}
This Hilbert space is called \textit{Fock space} and the excited states \textit{phonons}: these can be thought as $ \virgolette{particles} $ and $ n_m $ is the number of phonons in the $ m^{\mathrm{th}} $ normal mode.

\subsection{One-dimensional harmonic string}

Taking the continuum limit, the crystal becomes a string: to achieve this, one takes the limits $ a \rightarrow 0 $ and $ N \rightarrow \infty $ while keeping the total length $ R \equiv N a $ fixed. In this context, the displacement becomes a field $ \eta(x,t) $ dependent on the continuous real coordinate $ x \in [0, R] $, therefore:
\begin{equation*}
  \left( \eta_{n + 1} - \eta_n \right)^2 \longrightarrow a^2 \left( \frac{\pa \eta}{\pa x} \right)^2
  \qquad \qquad
  \sum_{n} \longrightarrow \frac{1}{a} \int_0^R \dd x
\end{equation*}

\begin{proposition}{}{continuous-limit}
  In the continuous limit:
  \begin{equation*}
    \frac{\delta_{nn'}}{a} \longrightarrow \delta(x - x') = \int_\R \frac{\dd k}{2\pi} e^{i k (x - x')}
  \end{equation*}
  \tcblower
\begin{proof}
  By direct calculation:
  \begin{equation*}
    a \sum_{n} f(an) \frac{\delta_{nm}}{a} = f(ma) \longrightarrow f(y) = \int_0^R \dd x\, f(x) \delta(x - y)
  \end{equation*}
  Recalling Eq. \ref{eq:1.10}, since $ k_m n = \frac{k_m}{a} na \rightarrow k x $, symmetrizing $ k_m \in [-\pi, \pi] $ (instead of $ [0, 2\pi] $) one finds:
  \begin{equation*}
    \delta(x - x') \longleftarrow \frac{\delta_{nn'}}{a} = \frac{1}{Na} \sum_{m} e^{ik_m (n - n')} \longrightarrow \int_\R \frac{\dd k}{2\pi} e^{ik (x - x')}
  \end{equation*}
  where integration limits are $ \pm \frac{\pi}{a} \rightarrow \pm \infty $.
\end{proof}
\end{proposition}
\begin{proposition}{}{}
  The inverse Fourier transform of the Dirac Delta reads:
  \begin{equation*}
    \int_0^R \dd x\, e^{i (k - k') x} = 2\pi \delta(k - k')
  \end{equation*}
\end{proposition}
By these relations, it can be seen that $ \frac{\dd k}{2\pi} $ has the physical meaning of the number of normal modes per unit spatial volume with wavenumber between $ k $ and $ k + \dd k $, while the interpretation of the divergent $ \delta(0) $ varies: in $ x $ space, it is the reciprocal of the lattice spacing, i.e. the number of normal modes per unit spatial volume, but in $ k $ space $ 2\pi \delta(0) $ is the (hyper-)volume of the system.\\
In the continuous limit, the Lagrangian of the harmonic string becomes:
\begin{equation*}
  L = \int_0^R \dd x \left[ \frac{1}{2} \rho_0 (\pa_t \eta)^2 - \frac{\kappa}{2} (\pa_x \eta)^2 \right]
\end{equation*}
where $ \rho_0 $ is the equilibrium mass density of the string. It is customary to absorb constants in the fields, thus, setting $ \phi(x,t) \equiv \sqrt{\rho_0} \eta(x,t) $ and $ \kappa = c^2 \rho_0 $ and adding a pinning term $ \propto \varphi^2 $, the Lagrangian can be written as:
\begin{equation}
  L = \int_0^R \dd x \left[ \frac{1}{2} (\pa_t \phi)^2 - \frac{c^2}{2} (\pa_x \phi)^2 - \frac{m^2 c^4}{2} \phi^2 \right]
  \label{eq:1.16}
\end{equation}
The classical equation of motion of this field yields:
\begin{equation}
  \pa_t^2 \phi = c^2 \pa_x^2 \phi - m^2 c^4 \phi
  \label{eq:1.17}
\end{equation}
The solutions of this wave equation can be written as:
\begin{equation}
  \phi(x,t) = e^{i \left( kx - \omega_k t \right)}
  \label{eq:1.18}
\end{equation}
with dispersion relation:
\begin{equation}
  \omega_k^2 = c^2 k^2 + m^2 c^4
  \label{eq:1.19}
\end{equation}
To quantize this system, one needs to compute the Hamiltonian. The canonical momentum field is:
\begin{equation}
  \Pi(x,t) \defeq \frac{\pa L}{\pa (\pa_t \phi)} = \pa_t \phi(x,t)
  \label{eq:1.20}
\end{equation}
The classical Hamiltonian can then be found as:
\begin{equation}
  \hat{\mathcal{H}} = \int_0^R \dd x \left[ \frac{1}{2} \Pi^2 + \frac{c^2}{2} (\pa_x \phi)^2 + \frac{m^2 c^4}{2} \phi^2 \right]
  \label{eq:1.21}
\end{equation}
The quantum field is analogous to Eq. \ref{eq:1.9}:
\begin{equation}
  \hat\phi(x,t) = \int_\R \frac{\dd k}{2\pi} \sqrt{\frac{\hbar}{2\omega_k}} \left[ e^{i \left( kx - \omega_k t \right)} \hat{a}_k + e^{-i \left( kx - \omega_k t \right)} \hat{a}_k^\dagger \right]
  \label{eq:1.22}
\end{equation}
with commutation relations:
\begin{equation}
  [\hat{a}_k, \hat{a}_{k'}^\dagger] = 2\pi \delta(k - k')
  \label{eq:1.23}
\end{equation}
\begin{equation}
  [\hat\phi(x,t), \hat\Pi(x',t)] = i\hbar \delta(x - x')
  \label{eq:1.24}
\end{equation}
The quantum Hamiltonian can be written as:
\begin{equation}
  \hat{\mathcal{H}} = \int_\R \frac{\dd k}{2\pi} \frac{1}{2} \hbar \omega_k \left( \hat{a}_k^\dagger \hat{a}_k + \hat{a}_k \hat{a}_k^\dagger \right) = E_0 + \int_\R \frac{\dd k}{2\pi} \hbar \omega_k \hat{a}_k^\dagger \hat{a}_k
  \label{eq:1.25}
\end{equation}
The ground-state energy can be computed from Eq. \ref{eq:1.14}, defining $ \mathrm{Vol} \defeq 2\pi \delta(k = 0) $:
\begin{equation}
  E_0 = \mathrm{Vol} \int_\R \frac{\dd k}{2\pi} \frac{1}{2} \hbar \omega_k
  \label{eq:1.26}
\end{equation}
For a strictly continuous system there is no cut-off in the $ k $ integral, thus the zero-point energy diverges: however, this is not necessarily a problem, as often only changes in $ E_0 $ are relevant (and experimentally accessible), and in this case it is known as \textit{Casimir energy}.

\newpage

\section{Spacetime symmetries}

\subsection{Lie groups}

\begin{definition}{Lie group}{}
  A \textit{Lie group} is a group whose elements depend in a continuous and differentiable way on a set of real parameters $ \{\theta_a\}_{a = 1, \dots, d} \subset \R^d $.
\end{definition}

A Lie group can be seen both as a group and as a $ d $-dimensional differentiable manifold (with coordinates $ \theta_a $). WLOG it is always possible to choose $ g(0,\dots,0) = e $.

\begin{definition}{Representation}{}
  Given a group $ G $ and a vector space $ V(\K) $, a \textit{representation} of $ G $ on $ V $ is a homomorphism $ \rho : G \rightarrow \mathrm{GL}(V) $.
\end{definition}

Given the isomorphism $ \mathrm{GL}(V) \rightarrow \K^{n \times n} $, with $ n \equiv \dim_\K V $, it is usual to \textit{de facto} represent $ G $ as matrices acting on elements of $ V $, i.e. $ \rho : G \rightarrow \K^{n \times n} $.

\begin{theorem}{}{}
  Given a Lie group $ G $ and $ g \in G $ connected with the identity, a representation of degree $ n $ on $ V(\C) $ as:
  \begin{equation}
    \rho(g(\theta)) = e^{i \theta_a T^a}
    \label{eq:1.27}
  \end{equation}
  where $ \{T^a\}_{a = 1, \dots, d} \subset \C^{n \times n} $ are the \textit{generators} of $ G $ on $ V $.
\end{theorem}

\begin{definition}{Lie algebra}{}
  Given a Lie group $ G $ with generators $ \{T^a\}_{a = 1, \dots, d} \subset \C^{n \times n} $ on $ V(\C) $, its \textit{Lie algebra} is:
  \begin{equation}
    [T^a, T^b] = i \tensor{f}{^a^b_c} T^c
    \label{eq:1.28}
  \end{equation}
  where $ \tensor{f}{^a^b_c} $ are called the \textit{structure constants}.
\end{definition}

\begin{proposition}{}{}
  The Lie algebra of a Lie group is independent of the representation.
\end{proposition}

\begin{proposition}{}{}
  Any $ d $-dimensional abelian Lie algebra is isomorphic to the direct sum of $ d $ one-dimensional Lie algebras.
\end{proposition}

As a consequence, all irreducible representations of an abelian Lie group are of degree $ n = 1 $.

\begin{definition}{Casimir operator}{}
  Given a Lie group with generators $ \{T^a\}_{a = 1, \dots, d} \subset \C^{n \times n} $ on $ V(\C) $, a \textit{Casimir operator} is an operator which commutes with each generator.
\end{definition}

Given an irreducible representation, Casimir operators are operators proportional to $ \id_V $, and the proportionality constants can be used to label the representation: they correspond to conserved physical quantities.

\begin{proposition}{}{non-compact-rep}
  A non-compact group cannot have finite unitary representations, except for those with trivial non-compact generators.
\end{proposition}

This means that the non-compact component of a group cannot be represented with unitary operators of finite dimension.

\subsection{Lorentz group}

Consider the group of linear transformations $ x^\mu \mapsto \tensor{\Lambda}{^\mu_\nu} x^\nu $ on $ \R^{1,3} $ which leave invariant the quantity $ \eta_{\mu \nu} x^\mu x^\nu $, i.e. the orthogonal group $ \On{1,3} $ (with signature $ (+,-,-,-) $). The condition that $ \tensor{\Lambda}{^\mu_\nu} $ must satisfy reads:
\begin{equation}
  \eta_{\rho \sigma} = \eta_{\mu \nu} \tensor{\Lambda}{^\mu_\rho} \tensor{\Lambda}{^\nu_\sigma}
  \label{eq:1.29}
\end{equation}
This implies that $ \det \Lambda = \pm 1 $: a transformation with $ \det \Lambda = -1 $ can always be written as the product of a transformation with $ \det \Lambda = 1 $ and a discrete transformation which reverses the sign of an odd number of coordinates. One further defines $ \SOn{1,3} \defeq \{\Lambda \in \On{1,3} : \det \Lambda = 1\} $.\\
Writing explicitly the temporal component $ 1 = (\tensor{\Lambda}{^0_0})^2 - (\tensor{\Lambda}{^1_0})^2 - (\tensor{\Lambda}{^2_0})^2 - (\tensor{\Lambda}{^3_0})^2 $, it is clear that $ (\tensor{\Lambda}{^0_0})^2 \ge 1 $. Therefore, $ \On{1,3} $ has two disconnected components: the orthochronous component with $ \tensor{\Lambda}{^0_0} \ge 1 $ and the non-orthochronous component with $ \tensor{\Lambda}{^0_0} \le 1 $. Any non-orthochronous transformation can be written as the product of an orthochronous transformation and a dircrete transformation which reverses the sign of the temporal component.

\begin{definition}{Lorentz group}{}
  The \textit{Lorentz group} $ \lorg $ is the orthochronous component of $ \SOn{1,3} $.
\end{definition}

The discrete transformations are factored out of the Lorentz group: these are \textit{parity} and \textit{time reversal}, which can be represented as $ \tensor{\mathcal{P}}{^\mu_\nu} = \diag \left( +1,-1,-1,-1 \right) $ and $ \tensor{\mathcal{T}}{^\mu_\nu} = \diag \left( -1,+1,+1,+1 \right) $. Applying these discrete transformations in all combinations ($ \id $, $ \mathcal{P} $, $ \mathcal{T} $ and $ \mathcal{P} \mathcal{T} $) one gets the four disconnected components of $ \SOn{1,3} $, which are not simply connected. This means that Lorentz invariance does not include parity and time reversal invariance.\\
Considering the infinitesimal transformation and applying Eq. \ref{eq:1.29}:
\begin{equation*}
  \tensor{\Lambda}{^\mu_\nu} = \delta^\mu_\nu + \tensor{\omega}{^\mu_\nu}
  \qquad \Rightarrow \qquad
  \omega_{\mu \nu} = - \omega_{\nu \mu}
\end{equation*}
Anti-symmetry means that $ \omega_{\mu \nu} $ has only 6 parameters, which define the Lorentz group: these can be identified by the 3 angles of spherical rotations in the $ (x,y) $, $ (y,z) $ and $ (z,x) $ planes and the 3 rapidities of hyperbolic rotations in the $ (t,x) $, $ (t,y) $ and $ (t,z) $ planes.

\begin{theorem}{}{}
  The Lorentz group is a non-compact Lie group.
\end{theorem}
\begin{proof}
  Spherical and hyperbolic rotations are continuous and differential w.r.t. angles and rapidities, but while angles vary in $ [0, 2\pi) $, rapidities vary in $ \R $, so the differentiable manifold associated to $ \lorg $ is not compact.
\end{proof}

\subsubsection{Lorentz algebra}

The 6 parameters of the Lorentz group correspond to 6 generators of the associated Lorentz algebra. Labelling these generators as $ J^{\mu \nu} : J^{\mu \nu} = - J^{\nu \mu} $, the generic element $ \Lambda \in \lorg $ can be written as:
\begin{equation}
  \Lambda = e^{- \frac{i}{2} \omega_{\mu \nu} J^{\mu \nu}}
  \label{eq:1.30}
\end{equation}
The $ \frac{1}{2} $ factor arises from each generator being counted twice (product of two anti-symmetric objects). Given a $ n $-dimensional representation of $ \lorg $, both $ \tensor{[J^{\mu \nu}]}{^i_j} $ and $ \tensor{[\Lambda]}{^i_j} $ are $ \C^{n \times n} $ matrices ($ \Lambda $ is real): for example, the $ n = 1 $ representation acts on \textit{scalars}, which are invariant under Lorentz transformations, so $ J^{\mu \nu} \equiv 0 \,\forall \mu,\nu = 0, \dots, 3 $.

\paragraph{4-vectors}

The $ n = 4 $ representation acts on \textit{contravariant 4-vectors} $ v^\mu $, which transform according to $ v^\mu \mapsto \tensor{\Lambda}{^\mu_\nu} v^\nu $, and \textit{covariant 4-vectors} $ v_\mu $, which transform according to $ v_\mu \mapsto \tensor{\Lambda}{_\mu^\nu} v_\nu $. In this representation, the generators are represented as $ \C^{4 \times 4} $ matrices:
\begin{equation}
  \tensor{[J^{\mu \nu}]}{^\rho_\sigma} = i \left( \eta^{\mu \rho} \delta^\nu_\sigma - \eta^{\nu \rho} \delta^\mu_\sigma \right)
  \label{eq:1.31}
\end{equation}
This is an irreducible representation, and the associated Lie algebra $ \lora $, called the \textit{Lorentz algebra}, is:
\begin{equation}
  [J^{\mu \nu}, J^{\sigma \rho}] = i \left( \eta^{\nu \rho} J^{\mu \sigma} - \eta^{\mu \rho} J^{\nu \sigma} - \eta^{\nu \sigma} J^{\mu \rho} + \eta^{\mu \rho} J^{\nu \sigma} \right)
  \label{eq:1.32}
\end{equation}
It is convenient to rearrange the 6 components of $ J^{\mu \nu} $ into two spatial vectors:
\begin{equation}
  J^i \defeq \frac{1}{2} \epsilon^{ijk} J^{jk}
  \qquad \qquad
  K^i \defeq J^{i0}
  \label{eq:1.33}
\end{equation}
The $ \lora $ can then be rewritten as:
\begin{equation}
  [J^i, J^j] = i \epsilon^{ijk} J^k
  \qquad \qquad
  [J^i, K^j] = i \epsilon^{ijk} K^k
  \qquad \qquad
  [K^i, K^j] = - i \epsilon^{ijk} J^k
  \label{eq:1.34}
\end{equation}
The first equation defines a $ \mathfrak{su}(2) $ sub-algebra, thus showing that $ J^i $ are the generators of angular momentum. Angles and rapidities are then defined as:
\begin{equation}
  \theta^i \defeq \frac{1}{2} \epsilon^{ijk} \omega^{jk}
  \qquad \qquad
  \eta^i \defeq \omega^{i0}
  \label{eq:1.35}
\end{equation}
so that:
\begin{equation}
  \Lambda = \exp \left[ -i \boldsymbol{\theta} \cdot \ve{J} + i \boldsymbol{\eta} \cdot \ve{K} \right]
  \label{eq:1.36}
\end{equation}
This definition reflect the \textit{alias} interpretation: the angles define counterclockwise rotations of vectors with respect to a fixed reference frame, while rapidities define boosts wich increase velocities with respect to said frame.

\subsubsection{Tensor Representations}

A generic $ (p,q) $-tensor transforms as:
\begin{equation}
  \tensor{T}{^{\mu_1 \dots \mu_p}_{\nu_1 \dots \nu_q}} \mapsto \tensor{\Lambda}{^{\mu_1}_{\alpha_1}} \dots \tensor{\Lambda}{^{\mu_p}_{\alpha_p}} \tensor{\Lambda}{_{\nu_1}^{\beta_1}} \dots \tensor{\Lambda}{_{\nu_q}^{\beta_q}} \tensor{T}{^{\alpha_1 \dots \alpha_p}_{\beta_1 \dots \beta_q}}
  \label{eq:1.37}
\end{equation}
The representation of the Lorentz group which acts on $ (p,q) $-tensors is of degree $ n = 4^{p + q} $, however it is reducible into the direct product of $ p + q $ 4-dimensional representations as of Eq. \ref{eq:1.38}.\\
Moreover, consider the action of the Lorentz group on $ (2,0) $-tensors: being $ T^{\mu \nu} \mapsto \tensor{\Lambda}{^\mu_\alpha} \tensor{\Lambda}{^\nu_\beta} T^{\alpha \beta} $, if $ T^{\mu \nu} $ is (anti-)symmetric it will remain so under a Lorentz transformation. Therefore, the 16-dimensional representation reduces to a 6-dimensional representation on anti-symmetric tensors and a 10-dimensional representation of symmetric tensors. Furthermore, the trace of a symmetric tensor is invariant, as $ T \equiv \eta_{\mu \nu} T^{\mu \nu} \mapsto \eta_{\mu \nu} \tensor{\Lambda}{^\mu _\alpha} \tensor{\Lambda}{^\nu _\beta} T^{\alpha \beta} = T $, so the latter representation further reduces into a 9-dimensional representation on symmetric traceless tensors and a 1-dimensional representation on scalars. This means that:
\begin{equation}
  \mathtt{4} \otimes \mathtt{4} = \mathtt{1} \oplus \mathtt{6} \oplus \mathtt{9}
  \label{eq:1.38}
\end{equation}
These are irreducible representations which, given a generic tensor $ T^{\mu \nu} $, act on $ S $, $ A^{\mu \nu} $ and $ S^{\mu \nu} - \frac{1}{4} \eta^{\mu \nu} S $ respectively, with $ A^{\mu \nu} \equiv \frac{1}{2} \left( T^{\mu \nu} - T^{\nu \mu} \right) $ and $ S^{\mu \nu} \equiv \frac{1}{2} \left( T^{\mu \nu} + T^{\nu \mu} \right) $.

\paragraph{Decomposition under rotations}

Restricting the action to the $ \SOn{3} $ sub-group of $ \lorg $, tensors can be decomposed according to irreducible representations of $ \SOn{3} $, which are labelled by the angular momentum $ j \in \N_0 $ and are of degree $ n = 2j + 1 $. Also recall the Clebsh-Gordan composition of angular momenta:
\begin{equation}
  \ve{j}_1 \otimes \ve{j}_2 = \bigoplus_{j = \abs{j_1 - j_2}}^{j_1 + j_2} \ve{j}
  \label{eq:1.39}
\end{equation}
A Lorentz scalar $ \alpha $ is a scalar under rotations too, so $ \alpha \in \ve{0} $. A 4-vector $ v^\mu $ is irreducible under the action of $ \lorg $, but under $ \SOn{3} $ it is decomposed into $ v^0 $ and $ \ve{v} $, so $ v^\mu \in \ve{0} \oplus \ve{1} $. A $ (2,0) $-tensor then is:
\begin{equation*}
  \begin{split}
    T^{\mu \nu} \in \left( \ve{0} \oplus \ve{1} \right) \otimes \left( \ve{0} \oplus \ve{1} \right)
    &= \left( \ve{0} \otimes \ve{0} \right) \oplus \left( \ve{0} \otimes \ve{1} \right) \oplus \left( \ve{1} \otimes \ve{0} \right) \oplus \left( \ve{1} \otimes \ve{1} \right) \\
    &= \ve{0} \oplus \ve{1} \oplus \ve{1} \oplus \left( \ve{0} \oplus \ve{1} \oplus \ve{2} \right)
  \end{split}
\end{equation*}
This is equivalent to Eq. \ref{eq:1.38}: the trace is a scalar, so $ S \in \ve{0} $, while the anti-symmetric part can be written as two spatial vectors $ A^{0i} $ and $ \frac{1}{2} \epsilon^{ijk} A^{jk} $, so $ A^{\mu \nu} \in \ve{1} \oplus \ve{1} $. The traceless symmetric part then decomposes as $ \bar{S}^{\mu \nu} \in \ve{0} \oplus \ve{1} \oplus \ve{2} $ under spatial rotations.\\
Equivalently, $ T^{\mu \nu} $ can be decomposed into $ T^{00} \in \left( \ve{0} \otimes \ve{0} \right) $, $ T^{0i} \in \left( \ve{0} \otimes \ve{1} \right) $, $ T^{i0} \in \left( \ve{1} \otimes \ve{0} \right) $ and $ T^{ij} \in \left( \ve{1} \otimes \ve{1} \right) $: the formers are a scalar and two spatial vectors associated to $ \ve{0} \oplus \ve{1} \oplus \ve{1} $, while the latter can be decomposed into the trace, which is $ \ve{0} $, the anti-symmetric part, which is $ \ve{1} $ ($ \epsilon^{ijk} A^{jk} $), and the traceless symmetric part, which is $ \ve{2} $.

\begin{example}{}{}
  Gravitational waves in de Donder gauge are described by a traceless symmetric matrix, therefore they have $ j = 2 $ (spin of the graviton).
\end{example}

There are two \textit{invariant tensors} under $ \lorg $: the metric $ \eta_{\mu \nu} $, by Eq. \ref{eq:1.29}, and the Levi-Civita symbol $ \epsilon^{\mu \nu \sigma \rho} $:
\begin{equation*}
  \epsilon^{\mu \nu \sigma \rho} \mapsto \tensor{\Lambda}{^\mu_\alpha} \tensor{\Lambda}{^\nu_\beta} \tensor{\Lambda}{^\sigma_\gamma} \tensor{\Lambda}{^\rho_\delta} \epsilon^{\alpha \beta \gamma \delta} = \left( \det \Lambda \right) \epsilon^{\mu \nu \sigma \rho} = \epsilon^{\mu \nu \sigma \rho}
\end{equation*}

\subsubsection{Spinorial representations}

The Lie algebras $ \mathfrak{su}(2) $ and $ \mathfrak{so}(3) $ are the same, which means that $ \SUn{2} $ and $ \SOn{3} $ are indistinguishable by infinitesimal transformations; however, they are globally different, as $ \SOn{3} $ rotations are periodic by $ 2\pi $, while $ \SUn{2} $ rotations are periodic by $ 4\pi $: in particular, it can be shown that $ \SOn{3} \cong \SUn{2} / \Z_2 $, i.e. $ \SUn{2} $ is the universal cover of $ \SOn{3} $. This means that $ \SUn{2} $ representations can be labelled by $ j \in \frac{1}{2} \N_0 $, where half-integer spin representations are known as \textit{spinorial representations}: they act on spinors, i.e. objects which change sign under rotations of $ 2\pi $ (thus not suitable to represent $ \SOn{3} $).

\begin{example}{}{}
  The $ \ve{\frac{1}{2}} $ representation of $ \SUn{2} $ is a 2-dimensional representation where $ J^i = \frac{\sigma^i}{2} $: Pauli matrices satisfy $ \sigma^i \sigma^j = \delta^{ij} + i \epsilon^{ijk} \sigma^k $, thus the $ \mathfrak{su}(2) $ algebra is satisfied. Denoting the $ m = \pm \frac{1}{2} $ states in the $ \ve{\frac{1}{2}} $ representation as $ \ket{\uparrow} $ and $ \ket{\downarrow} $, the Clebsch-Gordan decomposition $ \ve{\frac{1}{2}} \otimes \ve{\frac{1}{2}} = \ve{0} \oplus \ve{1} $ yields the triplet ($ j = 1 $) $ \ket{\uparrow\uparrow} $, $ \frac{1}{\sqrt{2}}\left( \ket{\uparrow\downarrow} + \ket{\downarrow\uparrow} \right) $, $ \ket{\downarrow\downarrow} $ and the singlet ($ j = 0 $) $ \frac{1}{\sqrt{2}}\left( \ket{\uparrow\downarrow} - \ket{\downarrow\uparrow} \right) $.
\end{example}

\begin{proposition}{}{lorentz-algebra}
  The Lorentz algebra $ \lora $ can be decomposed as $ \mathfrak{su}(2) \times \mathfrak{su}(2) $.

  \tcblower

  \begin{proof}
    Given the $ \lora $ algebra in Eq. \ref{eq:1.33}, it is possible to define:
    \begin{equation*}
      \ve{J}_{\pm} \defeq \frac{1}{2} \left( \ve{J} \pm i \ve{K} \right)
    \end{equation*}
    The Lie algebra then becomes:
    \begin{equation*}
      [\ve{J}_{\pm}^i, \ve{J}_{\pm}^j] = i \epsilon^{ijk} \ve{J}_{\pm}^k
      \qquad \qquad
      [\ve{J}_{\pm}^i, \ve{J}_{\mp}^j] = 0
    \end{equation*}
    These are two commuting $ \mathfrak{so}(2) $ algebras, thus proving the thesis.
  \end{proof}
\end{proposition}

As observed before, this does not imply that $ \lorg $ is globally equivalent to $ \SUn{2} \times \SUn{2} $: in fact, $ \SUn{2} \times \SUn{2} / \Z_2 \cong \SOn{4} $, while the universal cover of $ \lorg $ is $ \SL{2}{\C} $, as it can be shown that $ \lorg \cong \SL{2}{\C} / \Z_2 $.\\
By Prop. \ref{prop:lorentz-algebra}, representations of $ \lorg $ can be labelled by $ (j_-,j_+) \in \frac{1}{2} \N_0 \times \frac{1}{2} \N_0 $, with each index labelling a representation of $ \SUn{2} $: as $ \ve{J} = \ve{J}_+ + \ve{J}_- $, the $ (j_-,j_+) $ representation contains states with all possible spins $ \abs{j_+ - j_-} \le j \le j_+ + j_- $, and it is a representation of degree $ n = \left( 2j_- + 1 \right) \left( 2j_+ + 1 \right) $.\\
$ \ve{\left( 0,0 \right)} $ is the trivial (scalar) representation, as both $ \ve{J}_{\pm} = 0 $ and $ \ve{J} = \ve{K} = 0 $.\\
$ \ve{\left( \frac{1}{2},0 \right)} $ and $ \ve{\left( 0,\frac{1}{2} \right)} $ are 2-dimensional spinorial representations. These representations act on different spinors $ (\psi_\text{L})_\alpha \in \ve{\left( \frac{1}{2},0 \right)} $ and $ (\psi_\text{R})_\alpha \in \ve{\left( 0,\frac{1}{2} \right)} $, with $ \alpha = 1,2 $, which are called \textit{left-} and \textit{right-handed  Weyl spinors}. In $ \ve{\left( \frac{1}{2},0 \right)} $ the generators are $ \ve{J}_- = \frac{\boldsymbol{\sigma}}{2} $ and $ \ve{J}_+ = \ve{0} $, while in $ \ve{\left( 0,\frac{1}{2} \right)} $ they are $ \ve{J}_- = \ve{0} $ and $ \ve{J}_+ = \frac{\boldsymbol{\sigma}}{2} $, thus one finds $ \ve{J}_\text{L} = \ve{J}_\text{R} = \frac{\boldsymbol{\sigma}}{2} $ and $ \ve{K}_\text{L} = - \ve{K}_\text{R} = i \frac{\boldsymbol{\sigma}}{2} $, so that:
\begin{equation}
  \psi_\text{L} \mapsto \Lambda_\text{L} \psi_\text{L} = \exp \left[ \left( -i \boldsymbol{\theta} - \boldsymbol{\eta} \right) \cdot \frac{\boldsymbol{\sigma}}{2} \right] \psi_\text{L}
  \label{eq:1.40}
\end{equation}
\begin{equation}
  \psi_\text{R} \mapsto \Lambda_\text{R} \psi_\text{R} = \exp \left[ \left( -i \boldsymbol{\theta} + \boldsymbol{\eta} \right) \cdot \frac{\boldsymbol{\sigma}}{2} \right] \psi_\text{R}
  \label{eq:1.41}
\end{equation}
Note that the generators $ K^i $ are not hermitian, as expected from Prop. \ref{prop:non-compact-rep}. Furthermore, note that $ \Lambda_{\text{L},\text{L}} \in \C^{2 \times 2} $, therefore $ \psi_{\text{L},\text{R}} \in \C^2 $.

\begin{proposition}{}{weyl-spin-par}
  Given $ \psi_\text{L} \in \ve{\left( \frac{1}{2},0 \right)} $ and $ \psi_\text{R} \in \ve{\left( 0,\frac{1}{2} \right)} $, then $ \sigma^2 \psi_\text{L}^* \in \ve{\left( 0,\frac{1}{2} \right)} $ and $ \sigma^2 \psi_\text{R}^* \in \ve{\left( \frac{1}{2},0 \right)} $.

  \tcblower

  \begin{proof}
    Recall that for Pauli matrices $ \sigma^2 \sigma^i \sigma^2 = -(\sigma^i)^* $, so $ \sigma^2 \Lambda_\text{L}^* \sigma^2 = \Lambda_\text{R} $ and:
    \begin{equation*}
      \sigma^2 \psi_\text{L}^* \mapsto \sigma^2 \left( \Lambda_\text{L} \psi_\text{L} \right)^* = \left( \sigma^2 \Lambda_\text{L}^* \sigma^2 \right) \sigma^2 \psi_\text{L}^* = \Lambda_\text{R} \sigma^2 \psi_\text{L}^*
      \quad \Rightarrow \quad
      \sigma^2 \psi_\text{L}^* \in \ve{\left( 0,\tfrac{1}{2} \right)}
    \end{equation*}
    where $ \sigma^2 \sigma^2 = \tens{I}_2 $ was used. The proof for $ \sigma^2 \psi_\text{R}^* $ is analogous.
  \end{proof}
\end{proposition}

\begin{definition}{Charge conjugation}{}
  On Weyl spinors, the \textit{charge conjugation operator} is defined as:
  \begin{equation}
    \psi_\text{L}^c \defeq i \sigma^2 \psi_\text{L}^*
    \qquad \qquad
    \psi_\text{R}^c \defeq -i \sigma^2 \psi_\text{R}^*
    \label{eq:1.42}
  \end{equation}
\end{definition}

By Prop. \ref{prop:weyl-spin-par}, charge conjugation changes transforms a left-handed Weyl spinor into a right-handed one and vice versa. Moreover, the $ i $ factor ensures that applying this operator twice yields the identity operator.\\
$ \ve{\left( \frac{1}{2},\frac{1}{2} \right)} $ is a 4-dimensional complex representation: as $ j = 0,1 $, this representation acts on complex 4-vectors of the form $ \left( (\psi_\text{L})_\alpha, (\xi_\text{R})_\beta \right) \in \C^4 $, called \textit{Dirac spinors}, and $ \Lambda = \diag \left( \Lambda_\text{L}, \Lambda_\text{R} \right) \in \C^{4 \times 4} $. To explicit this relation, set $ \psi_\text{R} \equiv i \sigma^2 \psi_\text{L}^* $, $ \xi_\text{L} \equiv -i \sigma^2 \xi_\text{R}^* $ and $ \sigma^\mu \equiv \left( 1, \boldsymbol{\sigma} \right) $, $ \bar\sigma^\mu \equiv \left( 1, -\boldsymbol{\sigma} \right) $: it can be shown, then, that $ \xi_\text{R}^\dagger \sigma^\mu \psi_\text{R} $ and $ \xi_\text{L}^\dagger \bar\sigma^\mu \psi_\text{L} $ are contravariant 4-vectors. Although these 4-vectors are complex by construction, being the matrix $ \tensor{\Lambda}{^\mu_\nu} $ which represents the Lorentz transformation of a 4-vector real, a reality condition $ v_\mu^* = v_\mu $ is Lorentz invariant.

\paragraph{Parity}

Note that $ \mathcal{P} \ve{K} = -\ve{K} $, as the velocity of the boost gets reversed, while $ \mathcal{P} \ve{J} = \ve{J} $: this means that $ \mathcal{P} \ve{J}_\pm = \ve{J}_\mp $, i.e. parity exchanges a $ \ve{\left( j_-,j_+ \right)} $ representation into a $ \ve{\left( j_+,j_- \right)} $ representation. Therefore, a $ \ve{\left( j_-,j_+ \right)} $ representation of $ \lorg $ is a basis for the representation of the parity transformation iff $ j_- = j_+ $.

\begin{example}{}{}
  Weyl spinors (separately) are not a basis for a representation of the parity transformation, but Dirac spinors are.
\end{example}

\subsubsection{Field representations}

Given a field $ \phi(x) $, under a Lorentz transformation $ x^\mu \mapsto {x'}^\mu = \tensor{\Lambda}{^\mu_\nu} x^\nu $ it transforms as $ \phi(x) \mapsto \phi'(x') $.

\paragraph{Scalar fields}

A scalar field transforms as:
\begin{equation}
  \phi'(x') = \phi(x)
  \label{eq:1.43}
\end{equation}
Consider an infinitesimal transformation $ x'^\rho = x^\rho + \delta x^\rho $, with $ \delta x^\rho = - \frac{i}{2} \omega_{\mu \nu} \tensor{\left[ J^{\mu \nu} \right]}{^\rho_\sigma} x^\sigma $ as of Eq. \ref{eq:1.30}. Then, by definition, $ \delta \phi \equiv \phi'(x') - \phi(x) = 0 $, which corresponds to the fact that the scalar representation of $ \lorg $ is the trivial one ($ J^{\mu \nu} = 0 $).\\
However, one can consider the variation at fixed coordinate $ \delta_0 \phi \equiv \phi'(x) - \phi(x) $: while $ \delta \phi $ studies only a single degree of freedom, as the point $ \text{P} \in \R^{1,3} $ is kept constant and only $ \phi(\text{P}) $ can vary (i.e. the base space is one-dimensional), $ \delta_0 \phi $ studies $ \phi(\text{P}) $ with $ \text{P} $ varying over $ \R^{1,3} $, thus the base space is now a space of functions, which is infinite-dimensional. Therefore, $ \delta \phi $ provides a finite-dimensional representation of the generators, while $ \delta_0 \phi $ an infinite-dimensional one.\\
To explicit this representation:
\begin{equation*}
  \delta_0 \phi = \phi'(x) - \phi(x) = \phi'(x' - \delta x) - \phi(x) = - \delta x^\rho \pa_\rho \phi = \frac{i}{2} \omega_{\mu \nu} \tensor{\left[ J^{\mu \nu} \right]}{^\rho_\sigma} x^\sigma \pa_\rho \phi \equiv - \frac{i}{2} \omega_{\mu \nu} L^{\mu \nu} \phi
\end{equation*}
Recalling Eq. \ref{eq:1.31}, the generators can be expressed as:
\begin{equation}
  L^{\mu \nu} \defeq i \left( x^\mu \pa^\nu - x^\nu \pa^\mu \right)
  \label{eq:1.44}
\end{equation}
This is an infinite-dimensional representation, as it acts on the space of scalar fields. As $ p^\mu = i \pa^\mu $ (with signature $ (+,-,-,-) $), it is clear that $ L^i \equiv \frac{1}{2} \epsilon^{ijk} L^{jk} $ is the orbital angular momentum.

\paragraph{Weyl fields}

A left-handed Weyl field transforms as:
\begin{equation}
  \psi'_\text{L}(x') = \Lambda_\text{L} \psi_\text{L}(x)
  \label{eq:1.45}
\end{equation}
with $ \Lambda_\text{L} $ defined in Eq. \ref{eq:1.40}, and similarly for right-handed Weyl fields. The infinite-dimensional representation of the Lorentz generators determined by Weyl spinors can be found as:
\begin{equation*}
  \begin{split}
    \delta_0 \psi_\text{L} &\equiv \psi'_\text{L}(x) - \psi_\text{L}(x) = \psi'_\text{L}(x' - \delta x) - \psi_\text{L}(x) \\
                           &= \psi'_\text{L}(x') - \delta x^\rho \pa_\rho \psi_\text{L}(x) - \psi_\text{L}(x) = \left( \Lambda_\text{L} - \tens{I}_2 \right) \psi_\text{L}(x) - \delta x^\rho \pa_\rho \psi_\text{L}(x)
  \end{split}
\end{equation*}
The second term yields $ L^{\mu \nu} $, while the first can be further elaborated by writing:
\begin{equation}
  \Lambda_\text{L} = e^{-\frac{i}{2} \omega_{\mu \nu} S^{\mu \nu}}
  \label{eq:1.46}
\end{equation}
Thus:
\begin{equation*}
  \delta_0 \psi_\text{L} = - \frac{i}{2} \omega_{\mu \nu} J^{\mu \nu} \psi_\text{L}
\end{equation*}
where the angular momentum separates into the orbital and the spin components:
\begin{equation}
  J^{\mu \nu} = L^{\mu \nu} + S^{\mu \nu}
  \label{eq:1.47}
\end{equation}
This separation is general: $ L^{\mu \nu} $ is always expressed as in Eq. \ref{eq:1.44}, while $ S^{\mu \nu} $ depends on the specific representation. In the scalar representation $ S^{\mu \nu} = 0 $, while in the left/right-handed Weyl representation $ S^{i0} = \pm i \frac{\sigma^i}{2} $.

\paragraph{Vector fields}

A (contravariant) vector field transforms as:
\begin{equation}
  V'^\mu(x') = \tensor{\Lambda}{^\mu_\nu} V^\nu(x)
  \label{eq:1.48}
\end{equation}
A general vector field has a spin-0 and a spin-1 component, and it is acted on by the $ \ve{\left( \frac{1}{2},\frac{1}{2} \right)} $ representation.

\newpage

\subsection{Poincaré group}

\begin{definition}{Poincaré group}{}
  The \textit{Poincaré group} is defined as $ \pog \defeq \mathrm{T}^{1,3} \times \lorg $, where $ \mathrm{T}^{1,3} \cong \R^{1,3} $ is the group of translations of $ \R^{1,3} $.
\end{definition}

Given a translation $ x^\mu \mapsto x^\mu + a^\mu $, the associated group element can be written as:
\begin{equation}
  T = e^{-i a_\mu P^\mu}
  \label{eq:1.49}
\end{equation}
where $ P^\mu $ is the 4-momentum operator. Clearly translations commute, and so do their generators; on the other hand, as $ \ve{P} $ is a vector under rotations, while $ P^0 $ (energy) a scalar, one has:
\begin{equation*}
  [J^i,P^j] = i e^{ijk} P^k
  \qquad \qquad
  [J^i,P^0] = 0
\end{equation*}
These equations uniquely determine the \textit{Poincaré algebra} $ \poa $:
\begin{equation}
  \begin{gathered}
    [P^\mu,P^\nu] = 0 \\
    [J^{\mu \nu}, J^{\sigma \rho}] = i \left( \eta^{\nu \rho} J^{\mu \sigma} - \eta^{\mu \rho} J^{\nu \sigma} - \eta^{\nu \sigma} J^{\mu \rho} + \eta^{\mu \rho} J^{\nu \sigma} \right) \\
    [P^\mu,J^{\rho \sigma}] = i \left( \eta^{\mu \rho} P^\sigma - \eta^{\mu \sigma} P^\rho \right)
  \end{gathered}
  \label{eq:1.50}
\end{equation}
It's easy to check that $ [K^i,P^0] = i P^i $, while $ [J^i,P^0] = [P^i,P^0] = 0 $: given that $ P^0 $ generates time translations, linear and angular momentum are conserved quantities, while $ \ve{K} $ is not.

\subsubsection{Field representations}

Fields provide an infinite-dimensional representation of the Lorentz group as $ J^{\mu \nu} = L^{\mu \nu} + S^{\mu \nu} $, where $ S^{\mu \nu} $ does not depend on $ x^\mu $, but only on the spin of the field.\\
To represent $ P^\mu $ on fields, their transformation law must be specified: all fields are required to be scalars under translations, independently of their spin. This means that, given a generic field $ \phi(x) $, under a translation $ x' = x + a $ it transforms as $ \phi'(x') = \phi(x) $, so, under an infinitesimal translation $ x' = x + \varepsilon $:
\begin{equation*}
  \begin{split}
    \delta_0 \phi \equiv \phi'(x) - \phi(x) &= \phi'(x' - \varepsilon) - \phi(x) = - \varepsilon^\mu \pa_\mu \phi(x) \\
                                            &= e^{-i (-\varepsilon_\mu) P^\mu} \phi'(x') - \phi(x) = \left( e^{i \varepsilon_\mu P^\mu} - \tens{I} \right) \phi(x) = i \varepsilon_\mu P^\mu \phi(x)
  \end{split}
\end{equation*}
It is clear then that:
\begin{equation}
  P^\mu = +i \pa^\mu
  \label{eq:1.51}
\end{equation}
Explicitly, $ P^0 = i \pa_t $ and $ \ve{P} = - i \nabla $. It is trivial to check that these generators obey the Poincaré algebra.

\subsubsection{Particle representations}

The Poincaré group can also be represented using the Hilbert space $ \mathscr{H} $ of a free particle as a basis. Denoting a generic state as $ \ket{\ve{p},s} \in \mathscr{H} $, where $ \ve{p} $ is the particle's momentum and $ s $ collectively labels all other quantum numbers, it is clear that $ \mathscr{H} $ is infinite-dimensional, as $ \ve{p} $ is a continuous unbounded variable.

\begin{theorem}{Wigner's theorem}{}
  On the Hilbert space of physical states, any symmetry transformation can be represented by a linear and unitary or anti-linear and anti-unitary operator.
\end{theorem}

By this theorem, Poincaré transformations can be represented by unitary matrices, i.e. $ \ve{J} $, $ \ve{K} $, $ \ve{P} $ and $ P^0 $ can be represented by hermitian operators. These representations can be labeled by Casimir operators, which for $ \pog $ are easily found as $ P_\mu P^\mu $ and $ W_\mu W^\mu $, where $ W^\mu $ is the \textit{Pauli-Lubanski operator}:
\begin{equation}
  W^\mu \defeq -\frac{1}{2} \epsilon^{\mu \nu \sigma \rho} J_{\nu \sigma} P_\rho
  \label{eq:1.52}
\end{equation}
On single-particle states $ P_\mu P^\mu = m^2 $, while $ W_\mu W^\mu $ can be conveniently computed in a particular frame (due to Lorentz invariance). If $ m \neq 0 $, this frame is the rest-frame of the particle:
\begin{equation*}
  W^\mu = - \frac{m}{2} \epsilon^{\mu \nu \sigma 0} J_{\nu \sigma} = \frac{m}{2} \delta^{\mu i} \epsilon^{ijk} J^{jk} = \delta^{\mu i} m J^i
\end{equation*}
Therefore, on single-particle states of mass $ m $ and spin $ j $, the Casimir operator takes the form:
\begin{equation}
  W_\mu W^\mu = - m^2 j \left( j + 1 \right)
  \label{eq:1.53}
\end{equation}
If $ m = 0 $, the rest-frame does not exist, but it is possible to choose a frame where $ P^\mu = (\omega,0,0,\omega) $, where $ W^0 = W^3 = \omega J^3 $, $ W^1 = \omega \left( J^1 - K^2 \right) $ and $ W^2 = \omega \left( J^2 + K^1 \right) $, so that:
\begin{equation}
  W_\mu W^\mu = - \omega^2 \left[ \left( K^2 - J^1 \right)^2 + \left( K^1 + J^2 \right)^2 \right]
  \label{eq:1.54}
\end{equation}
It is clear that the $ m \rightarrow 0 $ limit is not trivial, and massive and massless representation need to be studied separately.

\paragraph{Massive representations}

Restricting to $ m \in \R^+ $ ($ m^2 < 0 $ states, called tachyons, are excluded), the massive representations are labeled by mass $ m $ and spin $ j $: in fact, after a Lorentz transformation such that $ P^\mu = (m,\ve{0}) $, spatial rotations can still be performed, i.e. the subspace of single-particle states with momentum $ P^\mu = (m,\ve{0}) $ is still a basis for the representation of $ \SUn{2} $ (as spinors must be included too). The group of transformations which leaves invariant a certain choice of $ P^\mu $ is called the \textit{little group}, so $ \SUn{2} $ is the little group of massive single-particle states: massive representations are labelled by $ m $ and $ j $, which means that massive particles of spin $ j $ have $ 2j + 1 $ degrees of freedom.

\paragraph{Massless representations}

The little group for $ P^\mu = (\omega,0,0,\omega) $ clearly is $ \SOn{2} $, the group of rotations in the $ (x,y) $ plane generated by $ J^3 $: as for any abelian group, its irreducible representations are one-dimensional, and they are labeled by the eigenvalue $ h $ of $ J^3 $, which represents the angular momentum in the direction of propagation of the particle and is called \textit{helicity}. Helicity can be shown to be quantized as $ h \in \frac{1}{2}\Z_0 $ (by topologic considerations on $ \pog \equiv \R^4 \times \SL{2}{\C} / \Z_2 $).\\
As a consequence, massless particles only have one degree of freedom and are characterized by their helicity $ h $. As $ \SOn{2} \equiv \Un{1} $, on a state of helicity $ h $ the little group is represented as:
\begin{equation}
  U(\theta) = e^{-i h \theta}
  \label{eq:1.55}
\end{equation}
Although massless particles with opposite helicities are logically two different species of particles, it can be written as $ h = \hat{\ve{p}} \cdot \hat{\ve{J}} $ (unit vectors), so $ h $ is a pseudoscalar such that $ h \mapsto -h $ under parity: this means that, if the interaction conserves parity, $ h $ and $ -h $ must be symmetric.

\begin{example}{}{}
  The electromagnetic and gravitational interactions conserve parity, thus photons and gravitons must be a basis for the representation of both $ \pog $ and parity: photons can have $ h = \pm 1 $ (left- and right-handed), while gravitons have $ h = \pm 2 $.
\end{example}

\begin{example}{}{}
  Neutrinos only interact via the weak interaction, which does not conserve parity, and in fact the two states $ h = \pm \frac{1}{2} $ are different particles: neutrinos have $ h = - \frac{1}{2} $, while antineutrinos have $ h = + \frac{1}{2} $.
\end{example}

\section{Classical equations of motion}

Consider a \textit{local field theory} of fields $ \{\phi_i(x)\}_{i \in \mathcal{I}} \equiv \phi(x) $, where $ x \in \R^{1,3} $ is a point in Minkoski spacetime. Its Lagrangian takes the form:
\begin{equation}
  L = \int \dd^3 x\, \mathcal{L}(\phi, \pa_\mu \phi)
  \label{eq:1.56}
\end{equation}
where $ \mathcal{L} $ is the \textit{Lagrangian density} of the theory (often referred to simply as the Lagrangian), which depends only on a finite number of derivatives. The action is then:
\begin{equation}
  \mathcal{S} = \int \dd t\, L = \int \dd^4 x\, \mathcal{L}(\phi, \pa_\mu \phi)
  \label{eq:1.57}
\end{equation}
The integration is carried on the whole space-time, with usual boundary conditions that all fields decrease sufficiently fast at infinity; this also allows to drop all boundary terms.

\begin{theorem}{}{}
  The \textit{stationary action principle} $ \delta \mathcal{S} = 0 $ determines the classical equations of motion:
  \begin{equation}
    \frac{\pa \mathcal{L}}{\pa \phi_i} - \pa_\mu \frac{\pa \mathcal{L}}{\pa (\pa_\mu \phi_i)} = 0
    \label{eq:1.58}
  \end{equation}

  \tcblower

  \begin{proof}
    Varying Eq. \ref{eq:1.57}:
    \begin{equation*}
      \delta \mathcal{S} = \int \dd^4 x\, \sum_{i \in \mathcal{I}} \left[ \frac{\pa \mathcal{L}}{\pa \phi_i} \delta \phi_i + \frac{\pa \mathcal{L}}{\pa (\pa_\mu \phi)} \delta (\pa_\mu \phi_i) \right] = \int \dd^4 x\, \sum_{i \in \mathcal{I}} \left[ \frac{\pa \mathcal{L}}{\pa \phi_i} - \pa_\mu \frac{\pa \mathcal{L}}{\pa (\pa_\mu \phi)} \right] \delta \phi_i = 0
    \end{equation*}
  \end{proof}
\end{theorem}

\begin{corollary}{}{}
  Two Lagrangians which differ by a total divergence $ \mathcal{L}' = \mathcal{L} + \pa_\mu K^\mu $ yield the same equations of motion.

  \tcblower

  \begin{proof}
    This is a consequence of Stokes theorem:
    \begin{equation*}
      \int_\Sigma \dd^4 x\, \pa_\mu K^\mu = \int_{\pa \Sigma} \dd A\, n_\mu K^\mu
    \end{equation*}
  \end{proof}
\end{corollary}

From the Lagrangian, it is possible to define the conjugate momenta and the Hamiltonian density:
\begin{equation}
  \Pi_i(x) \defeq \frac{\pa \mathcal{L}}{\pa (\pa_0 \phi_i)}
  \label{eq:1.59}
\end{equation}
\begin{equation}
  \mathcal{H} = \sum_{i \in \mathcal{I}} \Pi_i(x) \pa_0 \phi(x) - \mathcal{L}
  \label{eq:1.60}
\end{equation}
Unlike the Hamiltonian formalism, the Lagrangian formalism keeps Lorentz covariance explicit.

\subsection{Noether's theorem}

\begin{definition}{Infinitesimal transformation}{}
  Given a field theory with fields $ \{\phi_i\}_{i = 1, \dots, k} $ and action $ \mathcal{S}[\phi] $, an \textit{infinitesimal transformation} parametrized by $ \{\varepsilon^a\}_{a = 1, \dots, N} : \abs{\varepsilon^a} \ll 1 $ is defined by two sets of functions $ \{A^\mu_a(x)\}_{a = 1, \dots, N} $ and $ \{F_{i,a}(\phi,\pa \phi)\}_{i = 1, \dots, k ;\, a = 1, \dots, N} $ such that:
  \begin{equation}
    \begin{gathered}
      x^\mu \mapsto x'^\mu = x^\mu + \varepsilon^a A^\mu_a(x) \\
      \phi_i(x) \mapsto \phi'_i(x) = \phi_i(x) + \varepsilon^a F_{i,a}(\phi, \pa \phi)
    \end{gathered}
    \label{eq:1.61}
  \end{equation}
\end{definition}

\begin{definition}{Symmetry transformation}{}
  An infinitesimal transformation is a \textit{symmetry transformation} if it leaves $ \mathcal{S}[\phi] $ invariant, regardless of $ \phi $ being a solution of the equations of motion. It can further be classified as:
  \begin{itemize}
    \item \textit{global symmetry}, if $ \varepsilon^a \equiv \mathrm{const.} $;
    \item \textit{local symmetry}, if $ \varepsilon^a = \varepsilon^a(x) $.
  \end{itemize}
\end{definition}

Symmetry transformations which leave spacetime unchanged, i.e. with $ A^\mu_a(x) = 0 $, are called \textit{internal symmetries}.

\begin{theorem}{Noether's theorem}{}
  Given a global (but not local) symmetry parametrized by $ N $ generators, then there are $ N $ conserved currents $ \{j^\mu_a(\phi)\}_{a = 1, \dots, N} $ such that:
  \begin{equation}
    \pa_\mu j^\mu_a(\phi^\mathrm{cl}) = 0
    \label{eq:1.62}
  \end{equation}
  where $ \phi^\mathrm{cl} $ is a classical solution of the equations of motion.

  \tcblower

  \begin{proof}
    First, consider an infinitesimal transformation with slowly-varying parameters, i.e. $ l \abs{\pa_\mu \varepsilon^a} \ll \abs{\varepsilon^a} $ ($ l $ characteristic scale of the field theory): being it not a local symmetry, $ \delta \mathcal{S} \neq 0 $ at $ o(\varepsilon) $ and:
    \begin{equation*}
      \mathcal{S}[\phi'] = \mathcal{S}[\phi] + \int \dd^4 x\, \left[ \varepsilon^a(x) K_a(\phi) - \left( \pa_\mu \varepsilon^a(x) \right) j^\mu_a(\phi) + o(\pa \pa \varepsilon) \right] + o(\varepsilon^2)
    \end{equation*}
    If $ \varepsilon^a \equiv \mathrm{const.} $ (global symmetry) then $ \delta \mathcal{S}[\phi] = 0 \,\forall \phi $, therefore $ K_a(\phi) = 0 \,\forall \phi $ (independent of $ \varepsilon $). Assuming $ \varepsilon^a(x) \rightarrow 0 $ sufficiently fast as $ x \rightarrow \infty $, then integration by parts yields:
    \begin{equation*}
      \mathcal{S}[\phi'] = \mathcal{S}[\phi] + \int \dd^4 x\, \varepsilon^a(x) \pa_\mu j^\mu_a(\phi) + o(\pa \pa \varepsilon) + o(\varepsilon^2)
    \end{equation*}
    This expression is independent of the choice of $ \phi $. Moreover, note that Eq. \ref{eq:1.61} can be rewritten as an internal transformation by setting:
    \begin{equation*}
      \phi_i(x) \mapsto \phi'_i(x) = \phi_i(x - \varepsilon^a A_a) + \varepsilon^a F_{i,a} = \phi_i(x) + \varepsilon^a F_{i,a} - \varepsilon^a A^\mu_a \pa_\mu \phi_i \equiv \phi_i(x) + \delta \phi_i(x)
    \end{equation*}
    $ \delta \phi_i(x) $ vanishes at infinity, therefore it is the kind of variation used to derive the equations of motion: choosing $ \phi \equiv \phi^\mathrm{cl} $ classical solution then implies $ \delta \mathcal{S} = 0 $ independently of $ \varepsilon $, i.e. the thesis.
  \end{proof}
\end{theorem}

These are often called \textit{Noether currents}, and the associated \textit{Noether charges} are defined as:
\begin{equation}
  Q_a \defeq \int \dd^3 x\, j^0_a(t,\ve{x})
  \label{eq:1.63}
\end{equation}
These are time-independent, as $ \pa_0 Q_a = \int \dd^3 x\, \pa_0 j^0_a = - \int \dd^3 x\, \pa_i j^i_a $: on all spacetime it vanishes by divergence theorem (fields vanish at infinity), but on a finite volume it yields a boundary term interpreted as the incoming and outgoing flux.

\begin{proposition}{Noether currents}{}
  The explicit expression of Noether currents is:
  \begin{equation}
    j^\mu_a = \frac{\pa \mathcal{L}}{\pa (\pa_\mu \phi_i)} \left[ A^\nu_a(x) \pa_\nu \phi_i - F_{i,a}(\phi, \pa \phi) \right] - A^\mu_a(x) \mathcal{L}
    \label{eq:1.64}
  \end{equation}

  \tcblower

  \begin{proof}
    Varying the action at $ o(\pa \varepsilon) $:
    \begin{equation*}
        \delta_\varepsilon \mathcal{S} = \delta_\varepsilon \int \dd^4 x\, \mathcal{L} = \int \left[ \delta_\varepsilon (\dd^4 x) \mathcal{L} + \dd^4 x \left( \frac{\pa \mathcal{L}}{\pa \phi_i} \delta_\varepsilon \phi_i + \frac{\pa \mathcal{L}}{\pa (\pa_\mu \phi_i)} \delta_\varepsilon (\pa_\mu \phi_i) \right) \right]
    \end{equation*}
    The Jacobian of Eq. \ref{eq:1.61} gives $ \dd^4 x \mapsto \dd^4 x \left( 1 + A^\mu_a \pa_\mu \varepsilon^a \right) + o(\varepsilon) $, while $ \delta_\varepsilon \phi_i $ is not $ o(\pa \varepsilon) $ and:
    \begin{equation*}
      \delta_\varepsilon (\pa_\mu \phi_i) = \frac{\pa \phi'_i}{\pa x'^\mu} - \frac{\pa \phi_i}{\pa x^\mu} = \frac{\pa x^\nu}{\pa x'^\mu} \frac{\pa}{\pa x^\nu} \left( \phi_i + \varepsilon^a F_{i,a} \right) - \frac{\pa \phi_i}{\pa x^\mu} = - (\pa_\mu \varepsilon^a) \left( A^\nu_a \pa_\nu \phi_i - F_{i,a} \right) + o(\varepsilon)
    \end{equation*}
    The thesis follows from $ \delta_\varepsilon \mathcal{S} = - \int \dd^4 x\, (\pa_\mu \varepsilon^a) j^\mu_a + o(\pa \pa \varepsilon) + o(\varepsilon^2) $.
  \end{proof}
\end{proposition}

If the considered infinitesimal transformation is not a global symmetry, then $ \delta_\varepsilon \mathcal{S} $ has a non-vanishing $ o(\varepsilon) $ term which gives rise to a quasi-conserved current:
\begin{equation}
  \pa_\mu j^a_\mu = - (\delta_a \mathcal{L})_\mathrm{global}
  \label{eq:1.65}
\end{equation}

\subsubsection{Energy-momentum tensor}

Consider spacetime translations: as all fields must be scalars under these transformations, they define a Noether current. In particular, translations have $ A^\mu_\nu = \delta^\mu_\nu $ and $ F_{i,\mu} = 0 $ (the parameter index is a Lorentz index), so the conserved current is the \textit{energy-momentum tensor} $ j^\mu_\nu \equiv \tensor{\theta}{^\mu_\nu} $:
\begin{equation}
  \theta^{\mu \nu} = \frac{\pa \mathcal{L}}{\pa (\pa_\mu \phi_i)} \pa_\nu \phi_i - \eta^{\mu \nu} \mathcal{L}
  \label{eq:1.66}
\end{equation}
which is covariantly conserved on classical solutions of the equations of motion. The conserved Noether charge associated to the energy-momentum tensor is the \textit{four-momentum}:
\begin{equation}
  P^\mu \defeq \int \dd^3 x\, \theta^{0 \mu}
  \label{eq:1.67}
\end{equation}
The energy-momentum tensor so defined is not symmetric, however it can be made to via a tensor $ A^{\rho \mu \nu} $ which is anti-symmetric w.r.t. $ (\rho,\mu) $: $ T^{\mu \nu} \equiv \theta^{\mu \nu} + \pa_\rho A^{\rho \mu \nu} $ is physically equivalent from $ \theta^{\mu \nu} $, as the second term is a vanishing spatial divergence in the definition of $ P^\mu $ and is contracted to 0 in the conservation law.

\subsection{Real scalar fields}

Consider a real scalar field $ \phi $: a non-trivial Poincaré-invariant action must contain $ \pa_\mu \phi $ and must saturate each Lorentz index. For example:
\begin{equation}
  \mathcal{S}[\phi] = \frac{1}{2} \int \dd^4 x \left( \pa_\mu \phi \pa^\mu \phi - m^2 \phi^2 \right)
  \label{eq:1.68}
\end{equation}
The resulting equation of motion is the \textit{Klein-Gordon equation}:
\begin{equation}
  \left( \Box + m^2 \right) \phi = 0
  \label{eq:1.69}
\end{equation}
where $ \Box \equiv \pa_\mu \pa^\mu $. A plane wave $ e^{\pm i p_\mu x^\mu} $ is a solution if $ p^2 = m^2 $, so the KG equation imposes the relativistic dispersion relation and $ m $ can be interpreted as the mass. As $ \phi $ must be real, the general solution is a superposition of waves:
\begin{equation}
  \phi(x) = \int \frac{\dd^3 p}{(2\pi)^3 \sqrt{2E_\ve{p}}} \left[ a_\ve{p} e^{-i p_\mu x^\mu} + a^*_\ve{p} e^{i p_\mu x^\mu} \right]_{p^0 = E_\ve{p}}
  \label{eq:1.70}
\end{equation}
The positive energy solution has $ E_\ve{p} = + \sqrt{\ve{p}^2 + m^2} $, but it contains both \textit{positive} and \textit{negative frequency modes} $ e^{\mp i p_\mu x^\mu} $, while the $ (2E_\ve{p})^{-1/2} $ factor is a convenient normalization of the $ a_\ve{p} $ coefficients. The overall normalization of $ \mathcal{S}[\phi] $ does not influence the equations of motion, however it is important for obtaining a positive-definite Hamiltonian. As the momentum conjugate to $ \phi $ is $ \Pi_\phi = \pa_0 \phi $, the Hamiltonian density is found as:
\begin{equation}
  \mathcal{H} = \frac{1}{2} \left[ \Pi_\phi^2 + (\nabla \phi)^2 + m^2 \phi^2 \right]
  \label{eq:1.71}
\end{equation}
The energy-momentum tensor is computed to be:
\begin{equation}
  \theta^{\mu \nu} = \pa^\mu \phi \pa^\nu \phi - \eta^{\mu \nu} \mathcal{L}
  \label{eq:1.72}
\end{equation}
It is trivial to see that $ \theta^{00} = \mathcal{H} $: the Hamiltonian is the conserved charge related to the invariance under time translations.\\
To compute the conserved currents associated to Lorentz invariance, it is convenient to label the transformation parameters $ \omega^{\mu \nu} $ by an anti-symmetric pair of Lorentz indices, so that Eq. \ref{eq:1.61} become:
\begin{equation*}
  x^\mu \mapsto x'^\mu = x^\mu + \tensor{\omega}{^\mu_\nu} x^\nu = x^\mu + \frac{1}{2} \tensor{\omega}{^\rho^\sigma} \left( \delta^\mu_\rho x_\sigma - \delta^\mu_\sigma x_\rho \right) \equiv x^\mu + \frac{1}{2} A^\mu_{(\rho \sigma)} \omega^{\rho \sigma}
\end{equation*}
As $ F_{i,a} = 0 $, from Eq. \ref{eq:1.64} the conserved currents are:
\begin{equation}
  j^{(\rho \sigma) \mu} = x^\rho \theta^{\mu \sigma} - x^\sigma \theta^{\mu \rho}
  \label{eq:1.73}
\end{equation}
For spatial rotations, the conserved charge is:
\begin{equation*}
  M^{ij} = \int \dd^3 x \left( x^i \theta^{0j} - x^j \theta^{0i} \right) = \int \dd^3 x\, \pa_0 \phi \left( x^i \pa^j - x^j \pa^i \right) \phi = \frac{i}{2} \int \dd^3 x \left[ \phi L^{ij} (\pa_0 \phi) - (\pa_0 \phi) L^{ij} \phi \right]
\end{equation*}
where integration by parts was carried and $ L^{ij} $ is defined by Eq. \ref{eq:1.44}. This can be generalized.

\begin{definition}{Scalar product}{}
  Given two real scalar fields $ \phi_1 $ and $ \phi_2 $, their \textit{scalar product} is defined as:
  \begin{equation}
    \braket{\phi_1 | \phi_2} \defeq \frac{i}{2} \int \dd^3 x\, \phi_1 \overleftrightarrow{\pa_0} \phi_2
    \label{eq:1.74}
  \end{equation}
  where $ f \overleftrightarrow{\pa_\mu} g \defeq f \pa_\mu g - (\pa_\mu f) g $.
\end{definition}

\begin{proposition}{}{}
  If $ \phi_1 $ and $ \phi_2 $ are KG solutions, then $ \braket{\phi_1 | \phi_2} $ is time-independent.

  \tcblower

  \begin{proof}
    By the KG equation:
    \begin{equation*}
      \pa_0 \left[ \phi_1 \pa_0 \phi_2 - (\pa_0 \phi_1) \phi_2 \right] = \phi_1 \pa_0^2 \phi_2 - (\pa_0^2 \phi_1) \phi_2 = \phi_1 \nabla^2 \phi_2 - (\nabla^2 \phi_1) \phi_2
    \end{equation*}
    which vanishes after integration by parts.
  \end{proof}
\end{proposition}

Note that this scalar product is not positive-definite.

\begin{theorem}{Conserved charges}{cons-charge-val}
  Given a symmetry represented by a Lie group and a representation $ L^{\mu \nu} $ of its generators as operators acting on fields, the value of the associated conserved charges on a solution $ \phi $ of the equations of motion is:
  \begin{equation}
    M^{\mu \nu} = \braket{\phi | L^{\mu \nu} | \phi}
    \label{eq:1.75}
  \end{equation}
\end{theorem}

\begin{example}{Four-momentum}{}
  Applying Th. \ref{th:cons-charge-val} to four-momentum $ P^\mu = \braket{\phi | i \pa^\mu | \phi} $; for example, the $ \mu = 0 $ component is:
  \begin{equation*}
    \begin{split}
      P^0 &= \braket{\phi | i \pa^0 | \phi} = \braket{\phi | i \pa_0 | \phi} = \frac{i}{2} \int \dd^3 x \left[ \phi (i \pa_0) \pa_0 \phi - (\pa_0 \phi) i \pa_0 \phi \right] \\
          &= \frac{1}{2} \int \dd^3 x \left[ - \phi \pa_0^2 \phi + (\pa_0 \phi)^2 \right] = \frac{1}{2} \int \dd^3 x \left[ - \phi \left( \nabla^2 - m^2 \right) \phi + (\pa_0 \phi)^2 \right] \\
          &= \frac{1}{2} \int \dd^3 x \left[ (\nabla \phi)^2 + m^2 \phi^2 + (\pa_0 \phi)^2 \right] = \frac{1}{2} \int \dd^3 x\, \theta^{00}
    \end{split}
  \end{equation*}
\end{example}

The free KG action can be generalized to a self-interacting real scalar field introducing a general potential:
\begin{equation}
  \mathcal{S}[\phi] = \int \dd^4 x \left[ \frac{1}{2} \pa_\mu \phi \pa^\mu \phi - V(\phi) \right]
  \label{eq:1.76}
\end{equation}
The quadratic term in $ V(\phi) $ is the mass term, while higher-order terms describe the self-interaction.










