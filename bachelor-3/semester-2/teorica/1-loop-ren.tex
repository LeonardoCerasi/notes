\selectlanguage{english}

As already noted in \secref{sssec-div}, the computation of Next-to-Leading Order (NLO) corrections often produces diverging amplitudes. However, this seems to be a consequence of Heisenberg's principle, implying that a quantum field theory cannot be valid at all energy scales at once.

\section{Renormalization of \texorpdfstring{$ \lambda \phi^4 $}{λφ4}-theory}

\subsection{1-Loop diagrams}

Consider $ \lambda \phi^4 $-theory and compute the scattering cross-section for $ \phi \phi \rightarrow \phi \phi $ at LO:
\begin{equation*}
  \begin{tikzpicture}[baseline=(r.base)]
    \begin{feynman}[inline=(r.base)]
      \vertex (a1) {};
      \vertex[right=4cm of a1] (a2) {};
      \vertex[below=4em of a1] (b1) {};
      \vertex[below=4em of a2] (b2) {};
      \vertex[below=2em of a1] (c1) {};
      \vertex[right=2cm of c1, dot] (c2) {};

      \vertex[below=2.2em of a1] (r);

      \diagram* {
        (a1) -- [scalar, momentum = \(p_1\)] (c2),
        (b1) -- [scalar, momentum' = \(p_2\)] (c2),
        (c2) -- [scalar, momentum = \(p_3\)] (a2),
        (c2) -- [scalar, momentum' = \(p_4\)] (b2),
      };
    \end{feynman}
  \end{tikzpicture}
  \qquad \Rightarrow \qquad
  \dd \sigma^\text{LO} = \frac{1}{\Phi} \frac{\dd^3p_3}{(2\pi)^3 2E_{p_3}} \frac{\dd^3p_4}{(2\pi)^4 2E_{p_4}} (2\pi)^4 \delta^{(4)}(p_1 + p_2 - p_3 - p_4) \lambda^2
\end{equation*}
In the CM frame:
\begin{equation*}
  p_1 = (E, 0, 0, p)
  \quad
  p_2 = (E, 0, 0, -p)
  \qquad
  p_3 = (E, p \sin \theta, 0, p \cos \theta)
  \quad
  p_4 = (E, -p \sin \theta, 0, - p \cos \theta)
\end{equation*}
\begin{equation*}
  s = 4E^2
  \qquad
  t = -2p^2 (1 + \cos \theta)
  \qquad
  u = -2p^2 (1 - \cos \theta)
\end{equation*}
with flux factor $ \Phi = 8pE $. Then the cross-section becomes:
\begin{equation*}
  \dd \sigma^\text{LO} = \frac{\lambda^2}{8pE} \frac{\dd^3p_3}{(2\pi)^2 (2E_{p_3})^2} \delta(2E - 2\sqrt{p_3^2 + m^2}) = \frac{\lambda^2}{8pE} \frac{p_3 \dd p_3 \dd \Omega_2}{(2\pi)^2 4E_{p_3}} \delta(E_{p_3} - E) = \frac{\lambda^2}{8pE} \frac{p}{16\pi E} \dd \cos \theta
\end{equation*}
Therefore, there's no angular dependence:
\begin{equation}
  \frac{\dd \sigma^\text{LO}}{\dd \cos \theta} = \frac{\lambda^2}{128 \pi} \frac{1}{E^2}
\end{equation}
Now, consider the NLO corrections to this cross-section, i.e. the 1-loop corrections:
\begin{equation*}
  \begin{tikzpicture}[baseline=(r.base)]
    \begin{feynman}[inline=(r.base)]
      \vertex (a1) {};
      \vertex[right=4cm of a1] (a2) {};
      \vertex[below=4em of a1] (b1) {};
      \vertex[below=4em of a2] (b2) {};
      \vertex[below=2em of a1] (c1) {};
      \vertex[right=2cm of c1, blob] (c2) {};

      \vertex[below=2.2em of a1] (r);

      \diagram* {
        (a1) -- [scalar, momentum = \(p_1\)] (c2),
        (b1) -- [scalar, momentum' = \(p_2\)] (c2),
        (c2) -- [scalar, momentum = \(p_3\)] (a2),
        (c2) -- [scalar, momentum' = \(p_4\)] (b2),
      };
    \end{feynman}
  \end{tikzpicture}
  \quad = \quad
  \begin{tikzpicture}[baseline=(r.base)]
    \begin{feynman}[inline=(r.base)]
      \vertex (a1) {};
      \vertex[right=4cm of a1] (a2) {};
      \vertex[below=4em of a1] (b1) {};
      \vertex[below=4em of a2] (b2) {};
      \vertex[below=2em of a1] (c1) {};
      \vertex[right=2cm of c1, dot] (c2) {};

      \vertex[below=2.2em of a1] (r);

      \diagram* {
        (a1) -- [scalar, momentum = \(p_1\)] (c2),
        (b1) -- [scalar, momentum' = \(p_2\)] (c2),
        (c2) -- [scalar, momentum = \(p_3\)] (a2),
        (c2) -- [scalar, momentum' = \(p_4\)] (b2),
      };
    \end{feynman}
  \end{tikzpicture}
  \quad + \smo(\lambda^2) \equiv -i \lambda F(s,t)
\end{equation*}
where $ F(s,t) $ is called \bctxt{form factor}.

\begin{proposition}{Form factor}{}
  In $ \lambda \phi^4 $-theory, the form factor at NLO is:
  \begin{equation}
    F(s,t) = 1 - i\lambda \left( V(s) + V(t) + V(u) \right) + \smo(\lambda^2)
  \end{equation}
  with:
  \begin{equation}
    V(p^2) \defeq \frac{1}{2} \int \frac{\dd^4k}{(2\pi)^4} \frac{i}{k^2 - m^2} \frac{i}{(p + k)^2 - m^2}
    \label{eq:vps-def}
  \end{equation}
\end{proposition}

\begin{proofbox}
  \begin{proof}
    The NLO correction to the scattering amplitude is given by three loop diagrams:
    \begin{equation*}
      \begin{tikzpicture}[baseline = (r.base)]
        \begin{feynman}[inline = (r.base)]
          \vertex (x1) {};
          \vertex[below=3em of x1] (p1) {};
          \vertex[below=3em of p1] (x2) {};

          \vertex[right=3em of p1, dot] (v1) {};
          \vertex[right=5em of v1, dot] (v2) {};

          \vertex[right=3em of v2] (p2) {};
          \vertex[above=3em of p2] (x3) {};
          \vertex[below=3em of p2] (x4) {};

          \vertex[above=0.5em of v1] (r) {};

          \diagram* {
            (x1) -- [scalar, momentum = \(p_1\)] (v1),
            (x2) -- [scalar, momentum' = \(p_2\)] (v1),
            (v2) -- [scalar, momentum = \(p_3\)] (x3),
            (v2) -- [scalar, momentum' = \(p_4\)] (x4),

            (v1) -- [scalar, half left, momentum' = \(p'\)] (v2),
            (v2) -- [scalar, half left, momentum' = \(p\)] (v1),
          };
        \end{feynman}
      \end{tikzpicture}
      \qquad
      \begin{tikzpicture}[baseline = (r.base)]
        \begin{feynman}[inline = (r.base)]
          \vertex (x1) {};
          \vertex[right=4em of x1] (p1) {};
          \vertex[right=4em of p1] (x3) {};

          \vertex[below=2em of p1, dot] (v1) {};
          \vertex[below=5em of v1, dot] (v2) {};

          \vertex[below=9em of x1] (x2) {};
          \vertex[below=9em of x3] (x4) {};

          \vertex[below=2em of v1] (r) {};

          \diagram* {
            (x1) -- [scalar, momentum = \(p_1\)] (v1) -- [scalar, momentum = \(p_3\)] (x3),
            (x2) -- [scalar, momentum' = \(p_2\)] (v2) -- [scalar, momentum' = \(p_4\)] (x4),

            (v1) -- [scalar, half left, momentum = \(p + p_1 - p_3\)] (v2),
            (v2) -- [scalar, half left, momentum = \(p\)] (v1),
          };
        \end{feynman}
      \end{tikzpicture}
      \qquad
      \begin{tikzpicture}[baseline = (r.base)]
        \begin{feynman}[inline = (r.base)]
          \vertex (x1) {};
          \vertex[right=4em of x1] (p1) {};
          \vertex[right=4em of p1] (x3) {};

          \vertex[below=2em of p1, dot] (v1) {};
          \vertex[below=5em of v1, dot] (v2) {};

         \vertex[below=9em of x1] (x2) {};
          \vertex[below=9em of x3] (x4) {};

          \vertex[below=2em of v1] (r) {};

          \diagram* {
            (x1) -- [scalar, momentum = \(p_1\)] (v1) -- [scalar, momentum = \(p_4\)] (x3),
            (x2) -- [scalar, momentum' = \(p_2\)] (v2) -- [scalar, momentum' = \(p_3\)] (x4),

            (v1) -- [scalar, half left, momentum = \(p + p_1 - p_4\)] (v2),
            (v2) -- [scalar, half left, momentum = \(p\)] (v1),
          };
        \end{feynman}
      \end{tikzpicture}
    \end{equation*}
    with $ p' = p_1 + p_2 + p $. These correspond to loop diagrams in the $ s $, $ t $ and $ u $ channels, therefore, recalling \eref{eq:f4-loop-ampl}, the thesis is trivially found.
  \end{proof}
\end{proofbox}

\begin{lemma}[before upper = {\tcbtitle}]{Feynman parameters}{}
  \begin{equation}
    \frac{1}{AB} = \int_0^1 \dd x \frac{1}{\left( x A + (1-x) B \right)^2}  
  \end{equation}
\end{lemma}

This allows to rewrite $ V(p^2) $:
\begin{equation*}
  V(p^2) = - \frac{1}{2} \int \frac{\dd^4k}{(2\pi)^4} \int_0^1 \dd x \left[ k^2 - m^2 + x \left( 2 k \cdot p + p^2 \right) \right]^{-2} \equiv - \frac{1}{2} \int_0^1 \dd x \int \frac{\dd^4\ell}{(2\pi)^4} \frac{1}{\left( \ell^2 - M^2 \right)^2}
\end{equation*}
having defined:
\begin{equation}
  \ell \equiv k + x p
  \qquad \qquad
  M^2 \equiv m^2 - x (1-x) p^2
\end{equation}

\begin{lemma}[before upper = {\tcbtitle}]{}{}
  \begin{equation}
    I_d(n) \equiv \int \frac{\dd^dk}{(2\pi)^d} \frac{1}{(k^2 - M^2 + i\epsilon)^n} = \frac{i (-1)^n}{(4\pi)^{d/2}} \frac{\Gamma(n - \frac{d}{2})}{\Gamma(n)} \left( \frac{1}{M^2} \right)^{n - d/2}
    \label{eq:int-dn}
  \end{equation}
\end{lemma}

\begin{proofbox}
  \begin{proof}
    The poles of the integrand function are $ k_0 = \pm \left[ \sqrt{M^2 + \ve{k}^2} - i\epsilon \right] $, therefore the integration path $ I = \R $ can be rotated counterclockwise by $ \frac{\pi}{2} $ without overlapping with any pole. The resulting Wick rotation $ k_0 \mapsto i k_0 $, i.e. $ k \mapsto k_\text{E} $ implies $ k_E \in \R $, therefore the $ d $-dimensional integral in Minkowskian metric becomes a $ d $-dimensional integral with (negative) Euclidean metric:
    \begin{equation*}
      I_d(n) = i (-1)^n \int \frac{\dd^dk_\text{E}}{(2\pi)^d} \frac{1}{(k_\text{E}^2 + M^2 - i\epsilon)^n} = i (-1)^n \int_{\mathbb{S}^d} \dd \Omega_d \int_{\R^+} \dd k_\text{E} \frac{k_\text{E}^{d-1}}{(k_\text{E}^2 + M^2)^n}
    \end{equation*}
    It can be shown that:
    \begin{equation}
      \int_{\R^+} \dd k_\text{E} \frac{k_\text{E}^{d-1}}{(k_\text{E}^2 + M^2)^n} = \frac{1}{2} \frac{\Gamma(n - \frac{d}{2}) \Gamma(\frac{d}{2})}{\Gamma(n)} \left( \frac{1}{M^2} \right)^{n - \frac{d}{2}}
    \end{equation}
    Inserting \eref{eq:solid-angle} yields the thesis.
  \end{proof}
\end{proofbox}

\eref{eq:int-dn} diverges for $ (d,n) = (4,2) $, as $ \Gamma(z) = \frac{1}{z} - \gamma_\text{E} + o(z) $, therefore a cutoff needs to be introduced to regularize the integral:
\begin{equation*}
  \begin{split}
    V(p^2)
    & = \lim_{\Lambda \rightarrow \infty} \frac{-i}{2} \int_0^1 \dd x \int_0^\Lambda \frac{\dd k_\text{E}}{8\pi^2} \frac{k_\text{E}^3}{(k_\text{E}^2 + M^2)^2} = \lim_{\Lambda \rightarrow \infty} \frac{-i}{2} \int_0^1 \dd x \int_{M^2}^{\Lambda^2 + M^2} \frac{\dd \rho}{16\pi^2} \frac{\rho - M^2}{\rho^2} \\
    & = \lim_{\Lambda \rightarrow \infty} \frac{-i}{32 \pi^2} \int_0^1 \dd x \left[ \log \left( 1 + \frac{\Lambda^2}{M^2} \right) - \frac{\Lambda^2}{\Lambda^2 + M^2} \right] \\
    & = \lim_{\Lambda \rightarrow \infty} \frac{-i}{32\pi^2} \int_0^1 \dd x \left[ \log \frac{\Lambda^2}{M^2} - 1 + \smo  \left( \frac{M^2}{\Lambda^2} \right) \right]
  \end{split}
\end{equation*}
The form factor can then be rewritten as:
\begin{equation}
  F(s,t) = 1 + \frac{\lambda}{32\pi^2} \left[ 3 + \lim_{\Lambda \rightarrow \infty} \int_0^1 \dd x \left( \log \frac{M^2(s)}{\Lambda^2} + \log \frac{M^2(t)}{\Lambda^2} + \log \frac{M^2(u)}{\Lambda^2} \right) \right] + \smo(\lambda^2)
  \label{eq:form-factor}
\end{equation}
Now, define the threshold values $ s_0 \equiv 4m^2 $, $ t_0 = u_0 \equiv 0 $. The \bctxt{physical coupling constant} $ \lambda_\text{phys} $ is defined as:
\begin{equation}
  \frac{\dd \sigma}{\dd \cos \theta}\bigg\vert_{s_0 , t_0} = \frac{\lambda_\text{phys}^2}{128\pi m^2}
\end{equation}
which is linked to the \bctxt{bare coupling constant} $ \lambda_\text{bare} \equiv \lambda $ by:
\begin{equation}
  \lambda_\text{bare} F(s_0,t_0) = \lambda_\text{phys}
  \label{eq:bare-phys-def}
\end{equation}

\begin{proposition}{Bare and physical coupling}{bare-phys-coupl}
  The NLO form factor can be expressed as:
  \begin{equation}
    \lambda F(s,t) = \lambda_\text{phys} \left[ 1 - \frac{\lambda_\text{phys}}{32\pi^2} \int_0^1 \dd x \left( \log \frac{M^2(s)}{M^2(s_0)} + \log \frac{M^2(t)}{M^2(t_0)} + \log \frac{M^2(u)}{M^2(u_0)} \right) \right] + \smo(\lambda_\text{phys}^3)
    \label{eq:l-f-cutoff}
  \end{equation}
\end{proposition}

\begin{proofbox}
  \begin{proof}
    By \eref{eq:bare-phys-def}:
    \begin{equation*}
      \lambda_\text{phys} = \lambda (1 - c_0 \lambda) + \smo(\lambda^3)
      \qquad \Leftrightarrow \qquad
      \lambda = \lambda_\text{phys} (1 + c_0 \lambda_\text{phys}) + \smo(\lambda_\text{phys}^3)
    \end{equation*}
    Therefore:
    \begin{equation*}
      \lambda (1 - c \lambda) = \lambda_\text{phys} (1 + c_0 \lambda_\text{phys}) (1 - c \lambda_\text{phys}) + \smo(\lambda_\text{phys}^3) = \lambda_\text{phys} (1 - (c - c_0) \lambda_\text{phys}) + \smo(\lambda_\text{phys}^3)
    \end{equation*}
    Inserting \eref{eq:form-factor} yields the thesis.
  \end{proof}
\end{proofbox}

Expressing $ \lambda_\text{bare} $ as a function of $ \lambda_\text{phys} $ eliminates the divergence regularized by the cutoff, which is absorbed into $ \lambda_\text{phys} $, which is the observable coupling constant. Note that the choice of $ s_0 $ and $ t_0 $ to define $ \lambda_\text{phys} $ is purely conventional: a more useful choice is imposing:
\begin{equation}
  M^2(s_0) = M^2(t_0) = M^2(u_0) = \mu^2
\end{equation}
It is important to stress the difference between $ \Lambda $ and $ \mu $: $ \Lambda $ is a \bctxt{regularization scale}, introduced to regularize diverging loop integrals, while $ \mu $ is a \bctxt{renormalization point}, introduced to eliminate the $ \Lambda $-depedence\footnotemark. Both parameters are arbitrary, hence physical observables cannot depend on any of them: this means that the explicit $ \mu $-dependence of $ F(s,t) $ must cancel the implicit $ \mu $-dependence of $ \lambda $, so that $ \lambda_\text{phys} $. \\
The infinities which cancel expressing $ F(s,t) $ in terms of $ \lambda_\text{phys} $ are explicit when expressing $ \lambda_\text{bare} = \lambda_\text{bare}(\lambda_\text{phys}) $. In fact:
\begin{equation*}
  \lambda_\text{phys} = \lambda_\text{bare} \left[ 1 + \frac{3\lambda_\text{bare}}{32\pi^2} \left( 1 + \log \frac{\mu^2}{\Lambda^2} \right) \right] + \smo(\lambda_\text{bare}^3)
\end{equation*}
Thus, choosing $ \mu^2 \propto \Lambda^2 $, it is possible to express $ \lambda_\text{phys} $ as a limit of $ \lambda_\text{bare} $, and eventually $ \lambda_\text{phys} = \lambda_\text{bare} $: the bare coupling constant is the physical coupling at the scale $ \Lambda \rightarrow \infty $ (infinitely-small distances), i.e. including the fluctuations at all scales. Due to Heisenberg's principle, then, $ \lambda_\text{bare} $ is expected to diverge: excitations of the fields in momentum space are defined as normal coordinates of the system in position space, i.e. $ p \sim \pa_x \phi \sim \frac{1}{\Delta x} $, which diverges as $ \Delta x \rightarrow 0 $ (as $ \Lambda \rightarrow \infty $).

\footnotetext{The regularization scale $ \Lambda $ is an unphysical artifact introduce to isolate UV divergences, it depends on the specific regularization scheme adopted (UV cut-off, dimensional regolarization, etc.) and it disappears in renormalized results, while the renormalization point $ \mu $ is the physical reference scale which defines renormalized physical parameters.}

\subsection{Renormalized perturbation theory}

The process of renormalization, which allows to obtain finite results from diverging amplitudes, can be implemented in various ways. That of introducing a UV cut-off $ \Lambda $, expressing the bare observables in terms of the physical observables at some scale $ \mu $ and then setting $ \Lambda \rightarrow \infty $ is rather inefficient at higher orders in perturbation theory: thus, a more efficient renormalization scheme must be adopted. \\
First of all, substitute the UV cut-off with a \bctxt{dimensional regularization scheme} ('t Hooft--Veltman scheme): this scheme consists in turning diverging loop integrals into meromorphic\footnotemark functions of $ d \in \C $, which is an analytic continuation of the number of spacetime dimensions. In particular:
\begin{equation}
  d = 4 - 2 \epsilon
\end{equation}
with $ \epsilon \in \C : \Re{\epsilon} < 0 $. By analytic continuation of the Lagrangian of the theory, the dimensionality of the various objects changes.

\footnotetext{Given an open set $ D \subset \C $, then $ f : D \rightarrow \C $ is \textit{meromorphic} if it is holomorphic on $ D - P $, where $ P \subset D $ is a set of isolated points called \textit{poles}. Recall that a function $ f : D \rightarrow \C $ is \textit{holomorphic} on $ D $ if it is complex differentiable at every point in $ D $.}

\begin{proposition}{Dimensional analysis}{}
  In $ d = 4 $:
  \begin{equation}
    [\phi] = [M]
    \qquad \qquad
    [\lambda] = 1
  \end{equation}
\end{proposition}

\begin{proofbox}
  \begin{proof}
    In natural units $ \hbar = c = 1 $, each physical quantity has the dimension of a power of mass, as:
    \begin{equation*}
      [\hbar] = [E] [T] = [M] [T] \equiv 1
      \qquad \qquad
      [c] = [L] [T]^{-1} \equiv 1
    \end{equation*}
    where $ E = m $ ($ c^2 = 1 $) was used. Therefore, $ [L] = [T] = [M]^{-1} $. In this system, the action must be a dimensionless quantity:
    \begin{equation*}
      \act = \int \dd^4x \,\lag
      \qquad \Rightarrow \qquad
      1 = [L]^4 [\lag] = [M]^{-4} [\lag]
      \qquad \Rightarrow \qquad
      [\lag] = [M]^4
    \end{equation*}
    As $ [\pa_\mu] = [L]^{-1} = [M] $, then, it is clear that $ [\phi] = [M] $ and, being $ \lag_\text{int} = - \frac{\lambda}{4!} \phi^4 $, $ [\lambda] = 1 $.
  \end{proof}
\end{proofbox}

\begin{proposition}{Dimensionality in 't Hooft--Veltman scheme}{}
  In $ d = 4 - 2\epsilon $:
  \begin{equation}
    [\phi] = [M]^{1 - \epsilon}
    \qquad \qquad
    [\lambda] = [M]^{2\epsilon}
  \end{equation}
\end{proposition}

\begin{proofbox}
  \begin{proof}
    In the dimensional regularization scheme the differential becomes $ \dd^4x \mapsto \dd^dx $, therefore the Lagrangian must have:
    \begin{equation*}
      [\lag] = [M]^d = [M]^{4 - 2\epsilon}
    \end{equation*}
    $ [\pa_\mu] = [M] $ remains unchanged, therefore $ [\phi] = [M]^{1 - \epsilon} $ and, as a consequence, $ [\lambda] = [M]^{2\epsilon} $.
  \end{proof}
\end{proofbox}

To explicit the dimensionality of the (bare) coupling constant, a regularization scale $ \mu_\text{reg} $ is introduced, so that:
\begin{equation}
  \lambda = \mu_\text{reg}^{2\epsilon} \lambda_\text{bare}
\end{equation}

\begin{proposition}{}{}
  In $ d = 4 - 2\epsilon $:
  \begin{equation}
    -i \lambda V(p^2) = - \frac{\lambda_\text{bare}}{32\pi^2} \int_0^1 \dd x \left[ \frac{1}{\epsilon} - \gamma_\text{E} + \ln 4\pi - \log \frac{M^2}{\mu_\text{reg}^2} + \smo(\epsilon) \right]
  \end{equation}
  where $ \gamma_\text{E} \simeq 0.5772 $ is the Euler--Mascheroni constant.
\end{proposition}

\begin{proofbox}
  \begin{proof}
    From \eref{eq:vps-def} and using \eref{eq:int-dn}:
    \begin{equation*}
      \begin{split}
        -i \lambda V(p^2)
        & = \frac{i \lambda}{2} \int_0^1 \dd x \int \frac{\dd^d\ell}{(2\pi)^d} \frac{1}{(\ell^2 - M^2)^2} = \frac{i \lambda_\text{bare}}{2} \mu_\text{reg}^{2\epsilon} \int_0^1 \dd x \frac{i}{(4\pi)^{2 - \epsilon}} \frac{\Gamma(\epsilon)}{2} (M^2)^{-\epsilon} \\
        & = - \frac{\lambda_\text{bare}}{32\pi^2} \int_0^1 \dd x \, (4\pi)^\epsilon \Gamma(\epsilon) \left( \frac{M^2}{\mu_\text{reg}^2} \right)^{-\epsilon}
      \end{split}
    \end{equation*}
    Now, the Laurent series in $ \epsilon $ for this expression is:
    \begin{equation*}
      \begin{split}
        -i \lambda V(p^2)
        &= - \frac{\lambda_\text{bare}}{32\pi^2} \int_0^1 \dd x \left( 1 + \epsilon \log 4\pi + \smo(\epsilon) \right) \left( \frac{1}{\epsilon} - \gamma_\text{E} + \smo(\epsilon) \right) \left( 1 - \epsilon \log \frac{M^2}{\mu_\text{reg}^2} + \smo(\epsilon) \right) \\
        & = - \frac{\lambda_\text{bare}}{32\pi^2} \int_0^1 \dd x \left[ \frac{1}{\epsilon} - \gamma_\text{E} + \log 4\pi - \log \frac{M^2}{\mu_\text{reg}^2} + \smo(\epsilon) \right]
      \end{split}
    \end{equation*}
    which is the thesis.
  \end{proof}
\end{proofbox}

The dimensionally-regularized form factor, then, takes the form:
\begin{equation*}
  F(s,t) = 1 - \frac{\lambda_\text{bare}}{32\pi^2} \left[ 3 \left( \frac{1}{\epsilon} - \gamma_\text{E} + \log 4\pi \right) - \int_0^1 \dd x \left( \log \frac{M^2(s)}{\mu_\text{reg}^2} + \log \frac{M^2(t)}{\mu_\text{reg}^2} + \log \frac{M^2(u)}{\mu_\text{reg}^2} \right) \right] + \smo(\epsilon)
\end{equation*}
It would then be possible to proceed with the renormalization of the coupling constant specifying a cinematic $ (s_0,t_0) $ and setting $ \lambda_\text{phys} = \lambda_\text{bare} F(s_0,t_0) $. However, given the arbitrariness of the renormalization point $ \mu_\text{ren} $, it is algebrically convenient to choose an unphysical cinematic which simplifies the expression of $ \lambda_\text{phys} $.

\begin{definition}{Minimal subtraction scheme}{}
  The \bcdef{minimal subtraction scheme} (MS) is a renormalization scheme with renormalization point such that:
  \begin{equation}
    \lambda_\text{MS} = \lambda_\text{bare} \left[ 1 - \frac{3\lambda_\text{bare}}{32\pi^2} \frac{1}{\epsilon} \right] + \smo(\lambda_\text{bare}^3)
  \end{equation}
\end{definition}

When working with dimensional regularization, it is more useful to adopt the \bctxt{modified MS scheme} ($ \msb $), which includes the universal constant $ -\gamma_\text{E} + \log 4\pi $ in the expression for the physical coupling constant:
\begin{equation}
  \lambda_{\msb} = \lambda_\text{bare} \left[ 1 - \frac{3\lambda_\text{bare}}{32\pi^2} \left( \frac{1}{\epsilon} - \gamma_\text{E} + \log 4\pi \right) \right] + \smo(\lambda_\text{bare}^3)
  \label{eq:l-msb}
\end{equation}

\begin{theorem}{Scales in the $ \msb $ scheme}{}
  In the $ \msb $ scheme:
  \begin{equation}
    \mu_\text{reg} = \mu_\text{ren}
  \end{equation}
\end{theorem}

\begin{proofbox}
  \begin{proof}
    As of \eref{eq:l-msb}, using the same reasoning as in the proof of \pref{prop:bare-phys-coupl}:
    \begin{equation}
      \lambda_\text{bare} = \lambda_{\msb} \left[ 1 + \frac{3\lambda_{\msb}}{32\pi^2} \left( \frac{1}{\epsilon} - \gamma_\text{E} + \log 4\pi \right) \right] + \smo(\lambda_{\msb}^3)
    \end{equation}
    Moreover:
    \begin{equation*}
      \begin{split}
        \lambda_\text{bare} F(s,t)
        & = \lambda_\text{bare} \left[ 1 - \frac{3\lambda_\text{bare}}{32\pi^2} \left( \frac{1}{\epsilon} - \gamma_\text{E} + \log 4\pi \right) \right] \\
        & \qquad \qquad \qquad \qquad  + \frac{\lambda_\text{bare}^2}{32\pi^2} \int_0^1 \dd x \left( \log \frac{M^2(s)}{\mu_\text{reg}^2} + \log \frac{M^2(t)}{\mu_\text{reg}^2} + \log \frac{M^2(u)}{\mu_\text{reg}^2} \right) + \smo(\lambda_\text{bare}^3) \\
      \end{split}
    \end{equation*}
    Then, noting that $ \lambda_\text{bare}^2 = \lambda_{\msb}^2 + \smo(\lambda_{\msb}^3) $:
    \begin{equation*}
      \lambda_\text{bare} F(s,t) = \lambda_{\msb} \left[ 1 + \frac{\lambda_{\msb}}{32\pi^2} \int_0^1 \dd x \left( \log \frac{M^2(s)}{\mu_\text{reg}^2} + \log \frac{M^2(t)}{\mu_\text{reg}^2} + \log \frac{M^2(u)}{\mu_\text{reg}^2} \right) \right] + \smo(\lambda_{\msb}^3)
    \end{equation*}
    Comparing this expression to \eref{eq:l-f-cutoff}, it is clear that $ \mu_\text{reg} $ is also the renormalization point of the scheme.
  \end{proof}
\end{proofbox}

Hence, in the $ \msb $ scheme $ \mu_\text{reg} = \mu_\text{ren} \equiv \mu $.

\subsubsection{Lagrangian counterterms}

Define $ Z_\lambda : \lambda_\text{bare} = Z_\lambda \lambda_{\msb} \equiv \lambda_{\msb} + \delta_\lambda $, so that:
\begin{equation*}
  Z_\lambda = 1 + \frac{3\lambda_{\msb}}{32\pi^2} \left( \frac{1}{\epsilon} - \gamma_\text{E} + \log 4\pi \right) + \smo(\lambda_{\msb}^2)
  \qquad
  \delta_\lambda = 1 + \frac{3\lambda_{\msb}}{32\pi^2} \left( \frac{1}{\epsilon} - \gamma_\text{E} + \log 4\pi \right) + \smo(\lambda_{\msb}^3)
\end{equation*}
Inserting into the Lagrangian:
\begin{equation*}
  \begin{split}
    \lag
    & = \frac{1}{2} \left( \pa_\mu \phi \pa^\mu \phi - m^2 \phi^2 \right) - \frac{\lambda}{4!} \phi^4 \\
    & = \frac{1}{2} \left( \pa_\mu \phi \pa^\mu \phi - m^2 \phi^2 \right) - \frac{\lambda_{\msb}}{4!} \phi^4 - \frac{\delta_\lambda}{4!} \phi^4 + \smo(\lambda_{\msb}^3)
  \end{split}
\end{equation*}
It is then clear that renormalizing the coupling constant is equivalent to adding a counterterm to the Lagrangian, which contains the original divergence. This counterterm induces a new interaction vertex:
\begin{equation*}
  \begin{tikzpicture}[baseline=(r.base)]
    \begin{feynman}[inline=(r.base)]
      \vertex (a1) {};
      \vertex[right=4cm of a1] (a2) {};
      \vertex[below=4em of a1] (b1) {};
      \vertex[below=4em of a2] (b2) {};
      \vertex[below=2em of a1] (c1) {};
      \vertex[right=2cm of c1, crossed dot] (c2) {};

      \vertex[below=2.2em of a1] (r);

      \diagram* {
        (a1) -- [scalar] (c2),
        (b1) -- [scalar] (c2),
        (c2) -- [scalar] (a2),
        (c2) -- [scalar] (b2),
      };
    \end{feynman}
  \end{tikzpicture}
  \quad = \quad -i \delta_\lambda
\end{equation*}
This process is general: divergences are cured order by order in perturbation theory by adding each time new counterterms to the Lagrangian: hence the name of renormalized perturbation theory\footnote{The name may be confusing, as the counterterms still contain divergences.}. At each order, each new counterterms are fixed by an appropriate physical condition.

\subsection{Renormalizability}

The problem of renormalization has been reduced to that of classifying all possible counterterms to the Lagrangian, where each counterterm arises from the normalization of a different class of diverging Feynman diagrams.

\begin{definition}{Superficial degree of divergence}{}
  Given a scalar field theory in $ d $ dimensions, the \bcdef{superficial degree of divergence} of a Feynman diagram with $ L $ loops and $ P_\text{s} $ scalar propagators is:
  \begin{equation}
    D \defeq d L - 2 P_\text{s}
    \label{eq:sup-deg-div-def}
  \end{equation}
\end{definition}

The superficial degree of divergence gives an idea of how badly the considered diagram diverges: indeed, each loop corresponds to an integration with measure $ \dd^dk $, which has dimension $ [M]^d $, and each propagator corresponds to a factor of $ \tilde{D}_\text{F}(p) \sim (p^2 - m^2)^{-1} $, which has dimension $ [M]^{-2} $, so $ D $ is the net power of momentum present in the amplitude:
\begin{equation*}
  \mat \sim \int \frac{\dd k}{k^{1-D}}
\end{equation*}
Introducing a UV cut-off $ \Lambda $, the divergence will be different for different values of $ D $:
\begin{itemize}
  \item $ D > 0 $: $ \mat \sim \Lambda^D $, hence the diagram has a UV divergence which must be renormalized;
  \item $ D = 0 $: $ \mat \sim \log \Lambda $, hence the diagram has a logarithmic UV divergence;
  \item $ D < 0 $: $ \mat \sim \Lambda^{-\abs{D}} $, hence the diagram is UV finite.
\end{itemize}
This, however, is a naive analysis: in fact, the diagram may have diverging subdiagrams which make the actual degree of divergence worse, or it may have symmetries which make divergences partially cancel, leaving a better degree of divergence.

\subsubsection{\texorpdfstring{$ \lambda \phi^n $}{λφn}-theory}

\begin{lemma}{Loop number}{}
  Given a diagram with $ P $ propagators and $ V $ vertices, then:
  \begin{equation}
    L = P - V + 1
    \label{eq:loop-num}
  \end{equation}
\end{lemma}

\begin{proofbox}
  \begin{proof}
    Each loop is associated to an unconstrained momentum: each propagator corresponds to a momentum, while each vertex to a $ \delta $ which constrains the momenta of the diagram, but there's also an overall momentum-conserving $ \delta $ which makes one of the vertex-associated $ \delta $ redundant, hence $ L = P - (V - 1) $, which is the thesis.
  \end{proof}
\end{proofbox}

This result is valid in general for any field theory, and $ P $ represents the total number of propagators (even of different kinds). \\
Considering the specific case of $ \lambda \phi^n $-theory, instead, there's another relation which can simplify the computation of the superficial degrees of freedom: as each vertex has $ n $ scalar lines, then:
\begin{equation}
  n V = 2 P_\text{s} + N_\text{s}
  \label{eq:fn-v}
\end{equation}
where $ N_\text{s} $ is the number of external scalar lines. This expression takes into account the fact that each propagator links two vertices.

\begin{theorem}{Superficial degree of divergence}{lfn-sup-deg-div}
  In $ \lambda \phi^n $-theory in $ d $ dimensions, the \bcth{superficial degree of divergence} is:
    \begin{equation}
      D = d + \left( n \frac{d - 2}{2} - d \right) V - \frac{d - 2}{2} N
      \label{eq:lfn-sup-deg-div}
    \end{equation}
\end{theorem}

\begin{proofbox}
  \begin{proof}
    By \eref{eq:fn-v} $ P = \frac{1}{2} (n V - N) $, thus by \eref{eq:loop-num}:
    \begin{equation*}
      L = \left( \frac{n}{2} - 1 \right) V - \frac{N}{2} + 1
    \end{equation*}
    Recalling \eref{eq:sup-deg-div-def}:
    \begin{equation*}
      D = d \left[ \left( \frac{n}{2} - 1 \right) V - \frac{N}{2} + 1 \right] - n V + N = d + \left( n \frac{d - 2}{2} - d \right) V - \frac{d - 2}{2} N
    \end{equation*}
    which is the thesis.
  \end{proof}
\end{proofbox}

\begin{example}{$ \lambda \phi^4 $-theory}{p4-th}
  The case of $ \lambda \phi^4 $-theory in $ 4 $ dimensions is a particularly simple one, as:
  \begin{equation}
    D = 4 - N
  \end{equation}
  Hence, the only classes of diverging diagrams are vacuum-to-vacuum diagrams, which diverge as $ \sim \Lambda^4 $, the $ 1 \rightarrow 1 $ diagrams (i.e. the $ 2 $-point Green function of the theory and all its possible loop corrections), which diverges as $ \sim \Lambda^2 $, and the $ 2 \rightarrow 2 $ scattering, which diverges as $ \sim \log \Lambda $ (agreeing with \eref{eq:form-factor}). In this case there is only a finite number of classes of diverging diagrams.
\end{example}

\tref{th:lfn-sup-deg-div} can be derived also with dimensional analysis alone. In $ d $-dimensional $ \lambda \phi^n $-theory, it is trivial to show that:
\begin{equation*}
  [\phi] = [M]^{\frac{d - 2}{2}}
  \qquad \qquad
  [\lambda] = [M]^{d - n \frac{d - 2}{2}}
\end{equation*}
The dimensionality of $ \lambda $ is the dimensionality of a diagram with a single vertex, i.e. with $ N = n $ amputated legs, and, as diagrams with the same $ N $ have the same dimensionality, of any diagram with $ N = n $: generalizing, the dimensionality of a diagram with $ N $ amputated legs should be $ d - N \frac{d - 2}{2} $. On the other hand, if the diagram has $ V $ vertices, by definition it should have dimension $ [d]^V [M]^D $, thus:
\begin{equation*}
  \left( d - n \frac{d - 2}{2} \right) V + D = d - N \frac{d - 2}{2}
\end{equation*}
which is exactly \eref{eq:lfn-sup-deg-div}. It is then clear that $ D $ decreases as $ V $ increases (i.e. going to higher orders in perturbation theory) if and only if the coupling constant has positive mass dimension, leaving with only three possible classes theories:
\begin{itemize}
  \item super-renormalizable theories: the coupling constant has strictly negative mass dimension, hence there is only a finite number of diverging diagrams.
  \item renormalizable theories: the coupling constant is dimensionless, hence there is only a finite number of classes of diverging diagrams (although their absolute number may be infinite);
  \item non-renormalizable theories: the coupling constant has strictly negative mass dimension, hence there is an infinite number of diverging classes of diagrams.
\end{itemize}

\begin{example}{$ d = 4 $ theories}{}
  Consider $ \lambda \phi^n $-theory in $ d = 4 $ dimensions. The coupling constant in $ \lag_\text{int} = - \frac{1}{n!} \lambda \phi^n $ has dimension $ 4 - n $, hence:
  \begin{itemize}
    \item $ \lambda \phi^3 $-theory is super-renormalizable: $ D = 4 - V - N $, thus the only diverging diagrams are the $ 2 $-point Green function (loop corrections have $ D < 0 $) and the single interaction vertex, while all other diagrams have a finite amplitude;
    \item $ \lambda \phi^4 $-theory is renormalizable: $ D = 4 - N $, so the finitely-many classes of diverging diagrams are those listed in \exref{ex:p4-th};
    \item $ \lambda \phi^n $-theory with $ n \ge 5 $ is non-renormalizable: for example, in $ \lambda \phi^5 $-theory $ D = 4 + V - N $, hence increasing the number of vertices increases the superficial degree of divergence, i.e. $ \forall N \in N_{\ge 2} \, \exists V \in \N : D > 0 $.
  \end{itemize}
\end{example}

\subsubsection{\texorpdfstring{$ \lambda \phi^4 $}{λφ4}-theory}










