\documentclass[a4paper, 12pt, openany]{book}
\usepackage[utf8]{inputenc}
\usepackage[english]{babel}

% page layout
\usepackage[left=15mm, right=15mm, top=25mm]{geometry}
\geometry{a4paper}

% appendices
\usepackage[toc,page]{appendix}

% images
\usepackage{graphicx}

% colored headings
\usepackage{xcolor}

% equation numbering
\usepackage{amsmath}
\makeatletter
\def\tagform@#1{\maketag@@@{\ignorespaces\sffamily(#1)\unskip\@@italiccorr}}
\makeatother

% bookmarks
\usepackage{bookmark}

% sectioning
\usepackage{titlesec}

% titleformat{name}
% {title style}
% {number style}
% {space between number and title}
% {before-title code}

\titleformat{\part}[display]
  {\normalfont \sffamily \bfseries \centering}
  {\color{blue!70!black} \Large \partname \, \thepart}
  {10pt} % vertical spacing
  {\Huge}

\titleformat{\chapter}[display]
  {\normalfont \sffamily \bfseries}
  {\color{blue!70!black} \large \chaptertitlename \, \thechapter}
  {1pt}
  {\titlerule[1pt] \vspace{7pt} \Huge}

\titleformat{\section}
  {\normalfont \sffamily \Large}
  {\color{red!70!black} §\bfseries\thesection}
  {10pt}
  {\bfseries}

\titleformat{\subsection}
  {\normalfont \sffamily \large}
  {\color{red!70!black} §\bfseries\thesubsection}
  {10pt}
  {\bfseries}

\titleformat{\subsubsection}
  {\normalfont \sffamily}
  {\color{red!70!black} §\bfseries\thesubsubsection}
  {10pt}
  {\bfseries}

\renewcommand{\appendixtocname}{\sffamily Appendices}
\renewcommand{\appendixpagename}{\sffamily \Huge \bfseries Appendices}

% custom table of contents
\usepackage{tocloft}

\renewcommand{\cfttoctitlefont}{\sffamily \Huge \bfseries} % title font
% part
\renewcommand{\cftpartfont}{\sffamily \large \bfseries}
\renewcommand{\cftpartpagefont}{\sffamily \large \bfseries}
\renewcommand{\cftpartpresnum}{\sffamily \color{blue!70!black}}
% chapter
\renewcommand{\cftchapfont}{\sffamily \bfseries}
\renewcommand{\cftchappagefont}{\sffamily \bfseries}
\renewcommand{\cftchappresnum}{\sffamily \color{blue!70!black}}
% section
\renewcommand{\cftsecfont}{\sffamily}
\renewcommand{\cftsecpagefont}{\sffamily}
\renewcommand{\cftsecpresnum}{\sffamily \color{red!70!black}}
% subsection
\renewcommand{\cftsubsecfont}{\sffamily}
\renewcommand{\cftsubsecpagefont}{\sffamily}
\renewcommand{\cftsubsecpresnum}{\sffamily \color{red!70!black}}

% headers and foooters
\usepackage{fancyhdr}
\setlength{\headheight}{15pt}

% part name
\let\Oldpart\part
\newcommand{\parttitle}{}
\renewcommand{\part}[1]{\Oldpart{#1}\renewcommand{\parttitle}{#1}}

% no random page numbers
\fancypagestyle{plain}{%
  \fancyhf{} % clear all header and footer fields
  \renewcommand{\headrulewidth}{0pt}
  \renewcommand{\footrulewidth}{0pt}
}

% custom headers and footers
\renewcommand{\chaptermark}[1]{\markboth{#1}{#1}}
\pagestyle{fancy}
\fancyhead{}
\fancyfoot{}

\fancypagestyle{body}{%
  \fancyhead[LE,RO]{\sffamily\thepage}%
  \fancyhead[LO]{\sffamily \color{blue!70!black}\chaptername\ \thechapter\color{black} :\ \leftmark}%
  \fancyhead[RE]{\sffamily \color{blue!70!black}\partname\ \thepart\color{black} :\ \parttitle}%
}

% special header for Introduction
\fancypagestyle{introd}{%
  \fancyhead[LE,RO]{\sffamily \thepage}%
  \fancyhead[RE,LO]{\sffamily \leftmark}%
}

% special header for Contents
\fancypagestyle{contents}{%
  \fancyhead[LE,RO]{\sffamily \thepage}%
  \fancyhead[RE,LO]{\sffamily Contents}%
}

% special header for Appendix
\fancypagestyle{append}{%
  \fancyhead[LE,RO]{\sffamily \thepage}%
  \fancyhead[LO]{\sffamily \color{blue!70!black}\appendixname\ \thechapter \color{black}:\ \leftmark}%
  \fancyhead[RE]{\sffamily Appendices}%
}

% special header for Bibliography
\fancypagestyle{biblio}{%
  \fancyhead[LE,RO]{\sffamily \thepage}%
  \fancyhead[RE,LO]{\sffamily Bibliography}%
}

% no blank pages
\let\cleardoublepage\clearpage
% removes indentation
\setlength{\parindent}{0pt}
% add subsubsection numbering
\setcounter{secnumdepth}{3}

% math
\usepackage{amsmath}
\usepackage{amssymb}
\usepackage{amsfonts}
\usepackage{amsthm}
\usepackage{mathtools}
\usepackage{mathrsfs}
\usepackage{tensor}

% physics
\usepackage{braket}

% chemistry
\usepackage{chemformula}

% footnotes
\usepackage{footnote}

% fancy environments
\usepackage[many]{tcolorbox}
\BeforeBeginEnvironment{tcolorbox}{\savenotes}
\AfterEndEnvironment{tcolorbox}{\spewnotes}

\tcbuselibrary{theorems}

\NewTcbTheorem[number within=section]{definition}{Definition}{%
  enhanced,%
  breakable,%
  colback = green!5,%
  colframe = green!5,%
  coltitle = green!35!black,%
  fonttitle = \sffamily\bfseries,%
  sharp corners,%
  boxrule = 0pt,%
  detach title,%
  before upper = {\tcbtitle\\[4pt]},%
  left = 5pt,%
  right = 5pt,%
  top = 5pt,%
  bottom = 5pt,%
  separator sign none,%
  description delimiters parenthesis,%
  description font = \mdseries%
}{def}

\newtcbtheorem[number within=section]{theorem}{Theorem}{%
  enhanced,%
  breakable,%
  colback = red!5,%
  colframe = red!5,%
  coltitle = red!35!black,%
  fonttitle = \sffamily\bfseries,%
  sharp corners,%
  boxrule = 0pt,%
  detach title,%
  before upper = {\tcbtitle\\[4pt]},%
  left = 5pt,%
  right = 5pt,%
  top = 5pt,%
  bottom = 5pt,%
  separator sign none,%
  description delimiters parenthesis,%
  description font = \mdseries%
}{th}

\newtcbtheorem[number within=tcb@cnt@theorem]{corollary}{Corollary}{%
  enhanced,%
  breakable,%
  colback = red!5,%
  colframe = red!5,%
  coltitle = red!35!black,%
  fonttitle = \sffamily\bfseries,%
  sharp corners,%
  boxrule = 0pt,%
  detach title,%
  before upper = {\tcbtitle\\[4pt]},%
  left = 5pt,%
  right = 5pt,%
  top = 5pt,%
  bottom = 5pt,%
  separator sign none,%
  description delimiters parenthesis,%
  description font = \mdseries%
}{cor}

\newtcbtheorem[number within=section]{lemma}{Lemma}{%
  enhanced,%
  breakable,%
  colback = blue!5,%
  colframe = blue!5,%
  coltitle = blue!35!black,%
  fonttitle = \sffamily\bfseries,%
  sharp corners,%
  boxrule = 0pt,%
  detach title,%
  before upper = {\tcbtitle\\[4pt]},%
  left = 5pt,%
  right = 5pt,%
  top = 5pt,%
  bottom = 5pt,%
  separator sign none,%
  description delimiters parenthesis,%
  description font = \mdseries%
}{lemma}

\newtcbtheorem[number within=lemma]{lemcorollary}{Corollary}{%
  enhanced,%
  breakable,%
  colback = blue!5,%
  colframe = blue!5,%
  coltitle = blue!35!black,%
  fonttitle = \sffamily\bfseries,%
  sharp corners,%
  boxrule = 0pt,%
  detach title,%
  before upper = {\tcbtitle\\[4pt]},%
  left = 5pt,%
  right = 5pt,%
  top = 5pt,%
  bottom = 5pt,%
  separator sign none,%
  description delimiters parenthesis,%
  description font = \mdseries%
}{cor}

\newtcbtheorem[number within=section]{proposition}{Proposition}{%
  enhanced,%
  breakable,%
  colback = cyan!5,%
  colframe = cyan!5,%
  coltitle = cyan!35!black,%
  fonttitle = \sffamily\bfseries,%
  sharp corners,%
  boxrule = 0pt,%
  detach title,%
  before upper = {\tcbtitle\\[4pt]},%
  left = 5pt,%
  right = 5pt,%
  top = 5pt,%
  bottom = 5pt,%
  separator sign none,%
  description delimiters parenthesis,%
  description font = \mdseries%
}{prop}

\newtcbtheorem[number within=proposition]{propcorollary}{Corollary}{%
  enhanced,%
  breakable,%
  colback = cyan!5,%
  colframe = cyan!5,%
  coltitle = cyan!35!black,%
  fonttitle = \sffamily\bfseries,%
  sharp corners,%
  boxrule = 0pt,%
  detach title,%
  before upper = {\tcbtitle\\[4pt]},%
  left = 5pt,%
  right = 5pt,%
  top = 5pt,%
  bottom = 5pt,%
  separator sign none,%
  description delimiters parenthesis,%
  description font = \mdseries%
}{cor}

\newtcbtheorem[number within=section]{example}{Example}{%
  enhanced,%
  breakable,%
  colback = yellow!5,%
  colframe = yellow!5,%
  coltitle = yellow!35!black,%
  fonttitle = \sffamily\bfseries,%
  sharp corners,%
  boxrule = 0pt,%
  detach title,%
  before upper = {\tcbtitle\\[4pt]},%
  left = 5pt,%
  right = 5pt,%
  top = 5pt,%
  bottom = 5pt,%
  separator sign none,%
  description delimiters parenthesis,%
  description font = \mdseries%
}{ex}

\newtcolorbox{proofbox}{%
  enhanced,%
  breakable,%
  colback = black!5,%
  colframe = black!5,%
  sharp corners,%
  boxrule = 0pt,%
  left = 5pt,%
  right = 5pt,%
  top = 5pt,%
  bottom = 5pt,%
  borderline west = {1pt}{0pt}{black!70},%
}

% hyper-references
\usepackage{hyperref}

% custom references
\newcommand{\eref}[1]{\textsf{Eq.\,\ref{#1}}}
\newcommand{\eeref}[2]{\textsf{Eq.\,\ref{#1}-\ref{#2}}}
\newcommand{\dref}[1]{\textsf{Def.\,\ref{#1}}}
\newcommand{\ddref}[2]{\textsf{Def.\,\ref{#1}-\ref{#2}}}
\newcommand{\tref}[1]{\textsf{Th.\,\ref{#1}}}
\newcommand{\pref}[1]{\textsf{Prop.\,\ref{#1}}}
\newcommand{\lref}[1]{\textsf{Lemma\,\ref{#1}}}

\newcommand{\secref}[1]{\textcolor{red!70!black}{\textsf{§\ref{#1}}}}

% extended integral symbols
\usepackage{esint}

% text
\newcommand{\virgolette}[1]{``\text{#1}"}
\newcommand{\tildetext}{\raise.17ex\hbox{$\scriptstyle\mathtt{\sim}$}}

% greek
\newcommand{\Chi}{\text{X}}

% custom math symbols
\newcommand{\abs}[1]{\left\lvert#1\right\rvert}
\newcommand{\norm}[1]{\left\lVert#1\right\rVert}
\newcommand{\sgn}[1]{\mathrm{sgn}\,#1}
\newcommand{\pa}{\partial}
\newcommand{\na}{\nabla}
\newcommand{\tens}[1]{\mathrm{#1}}
\newcommand{\defeq}{\mathrel{\vcenter{\baselineskip0.5ex \lineskiplimit0pt
                     \hbox{\scriptsize.}\hbox{\scriptsize.}}}%
                     =}
\newcommand{\eqdef}{=%
                     \mathrel{\vcenter{\baselineskip0.5ex \lineskiplimit0pt
                     \hbox{\scriptsize.}\hbox{\scriptsize.}}}}
\newcommand{\ve}[1]{\mathbf{#1}}
\newcommand{\hilb}{\mathscr{H}}
\newcommand{\fock}{\mathscr{F}}
\DeclareMathOperator{\diag}{diag}
\newcommand{\sqg}{\sqrt{\tens{g}}}
\newcommand{\sqgm}{\sqrt{-\tens{g}}}
\newcommand{\cm}{\mathcal{C}^{\infty}(\mathcal{M})}
\DeclareMathOperator{\lspan}{span}
\newcommand{\xm}{\mathfrak{X}(\mathcal{M})}
\DeclareMathOperator{\id}{id}
\newcommand{\ld}{\mathcal{L}}
\newcommand{\lm}[1]{\Lambda^{#1}(\mathcal{M})}
\newcommand{\hrm}[1]{\mathrm{Harm}^{#1}(\mathcal{M})}
\DeclareMathOperator{\ran}{ran}
\DeclareMathOperator{\tr}{tr}
\DeclareMathOperator{\End}{End}
\DeclareMathOperator{\im}{Im}
\newcommand{\dd}{\mathrm{d}}
\newcommand{\bs}[1]{\boldsymbol{#1}}
\newcommand{\dg}{^\dagger}
\DeclareMathOperator{\Aut}{Aut}
\DeclareMathOperator{\ad}{ad}
\DeclareMathOperator{\Ad}{Ad}
\newcommand{\lag}{\mathcal{L}}
\newcommand{\ham}{\mathcal{H}}
\newcommand{\act}{\mathcal{S}}
\newcommand{\normord}{\mathfrak{N}}
\newcommand{\tempord}{\mathfrak{T}}
\newcommand{\parity}{\mathcal{P}}
\newcommand{\chargec}{\mathcal{C}}
\newcommand{\timer}{\mathcal{T}}
\newcommand{\covder}{D}

\newcommand{\grad}{\boldsymbol{\nabla}}
\newcommand{\dive}{\boldsymbol{\nabla}\cdot}
\newcommand{\rot}{\boldsymbol{\nabla}\times}
\newcommand{\lap}{\triangle}

% small \overleftrightarrow
\makeatletter
\newcommand{\overleftrightsmallarrow}{\mathpalette{\overarrowsmall@\leftrightarrowfill@}}
\newcommand{\overrightsmallarrow}{\mathpalette{\overarrowsmall@\rightarrowfill@}}
\newcommand{\overleftsmallarrow}{\mathpalette{\overarrowsmall@\leftarrowfill@}}
\newcommand{\overarrowsmall@}[3]{%
  \vbox{%
    \ialign{%
      ##\crcr
      #1{\smaller@style{#2}}\crcr
      \noalign{\nointerlineskip}%
      $\m@th\hfil#2#3\hfil$\crcr
    }%
  }%
}
\def\smaller@style#1{%
  \ifx#1\displaystyle\scriptstyle\else
    \ifx#1\textstyle\scriptstyle\else
      \scriptscriptstyle
    \fi
  \fi
}
\makeatother
\newcommand{\smlra}[1]{\overleftrightsmallarrow{#1}}
\newcommand{\smla}[1]{\overleftsmallarrow{#1}}
\newcommand{\smra}[1]{\overrightsmallarrow{#1}}

% functions
\DeclareMathOperator{\sech}{sech}
\DeclareMathOperator{\Exp}{Exp}

% number sets
\newcommand{\N}{\mathbb{N}}
\newcommand{\Z}{\mathbb{Z}}
\newcommand{\Q}{\mathbb{Q}}
\newcommand{\R}{\mathbb{R}}
\newcommand{\C}{\mathbb{C}}
\newcommand{\K}{\mathbb{K}}

% groups
\newcommand{\Ot}{\mathrm{O}(3)}
\newcommand{\SOt}{\mathrm{SO}(3)}
\newcommand{\On}[1]{\mathrm{O}(#1)}
\newcommand{\SOn}[1]{\mathrm{SO}(#1)}
\newcommand{\Un}[1]{\mathrm{U}(#1)}
\newcommand{\SUn}[1]{\mathrm{SU}(#1)}
\newcommand{\GL}[1]{\mathrm{GL}(#1)}
\newcommand{\SL}[1]{\mathrm{SL}(#1)}

% units of measure
\newcommand{\m}{\,\mathrm{m}}
\newcommand{\ang}{\,\mathrm{\AA}}
\newcommand{\fm}{\,\mathrm{fm}}

\newcommand{\barn}{\,\mathrm{barn}}

\newcommand{\cels}{\,^{\circ}\mathrm{C}}

\newcommand{\ev}{\,\mathrm{eV}}
\newcommand{\kev}{\,\mathrm{keV}}
\newcommand{\mev}{\,\mathrm{MeV}}
\newcommand{\gev}{\,\mathrm{GeV}}
\newcommand{\tev}{\,\mathrm{TeV}}

% particles
\newcommand{\g}{\mathrm{g}}
\newcommand{\w}{\mathrm{W}}
\newcommand{\z}{\mathrm{Z}}

% frontmatters defaults
\author{Leonardo Cerasi%
  \thanks{\scriptsize\href{mailto:leonardo.cerasi@studenti.unimi.it}{leo.cerasi@pm.me}}\\
  \small GitHub repository: \href{https://github.com/LeonardoCerasi/notes}{LeonardoCerasi/notes}}
\date{}


\usepackage{overarrows}
\usepackage{slashed}

\newcommand{\lorg}{\mathrm{SO}^+(1,3)}
\newcommand{\lora}{\mathfrak{so}^+(1,3)}
\newcommand{\pog}{\mathrm{ISO}^+(1,3)}
\newcommand{\poa}{\mathfrak{iso}^+(1,3)}

\newcommand{\dira}{\mathfrak{cl}_{1,3}(\C)}

\DeclareMathOperator{\spin}{Spin}

\newcommand{\sua}[1]{\mathfrak{su}(#1)}
\newcommand{\sla}[1]{\mathfrak{sl}(#1)}

\newcommand{\cla}[1]{\mathfrak{cl}(#1)}
\newcommand{\clar}[1]{\mathfrak{cl}_{#1}(\R)}

\title{\Huge\textbf{Quantum Field Theory 1} \\ \large Prof. S. Forte a.a. 2024-25}
\author{Leonardo Cerasi%
	\thanks{\scriptsize\href{mailto:leonardo.cerasi@studenti.unimi.it}{leo.cerasi@pm.me}}\\
	\small GitHub repository: \href{https://github.com/LeonardoCerasi/notes}{LeonardoCerasi/notes}}

\begin{document}

\frontmatter

\maketitle
\tableofcontents
\pagestyle{indice}

\mainmatter

\part{Field Theory}
\pagestyle{body}

\chapter{Classical Field Theory}
\selectlanguage{english}

\section{Continuous limit}

\subsection{One-dimensional harmonic crystal}

Consider a simple one-dimensional model of a crystal where atoms of mass $ m \equiv 1 $ lie at rest on the $ x $-axis, with equilibrium positions labelled by $ n \in \N $ and equally spaced by a distance $ a $.\\
Assuming these atoms are free to vibrate only in the $ x $ direction (longitudinal waves), and denoting the displacement of the atom at position $ n $ as $ \eta_n $, one can write the Lagrangian for a \textit{harmonic crystal} as:
\begin{equation}
  L = \sum_{n} \left[ \frac{1}{2} \dot{\eta}_n^2 - \frac{\lambda}{2} \left( \eta_n - \eta_{n-1} \right)^2 \right]
  \label{eq:1.1}
\end{equation}
where $ \lambda $ is the spring constant. From the Lagrange equations, the classical equations of motions are:
\begin{equation}
  \ddot{\eta}_n = \lambda \left( \eta_{n+1} - 2 \eta_n + \eta_{n-1} \right)
  \label{eq:1.2}
\end{equation}
The solutions can be written as complex travelling waves:
\begin{equation}
  \eta_n (t) = e^{i \left( k n - \omega t \right)}
  \label{eq:1.3}
\end{equation}
where the dispersion relation is:
\begin{equation}
  \omega^2 = 2\lambda \left( 1 - \cos k \right) \approx \lambda k^2
  \label{eq:1.4}
\end{equation}
Therefore, in the long-wavelength limit $ k \ll 1 $ waves propagate with velocity $ c = \sqrt{\lambda} $. To determine the normal modes, there need to be boundary conditions: imposing boundary conditions:
\begin{equation}
  \eta_{n + N} = \eta_n \qquad \Rightarrow \qquad k_m = \frac{2\pi m}{N} \,,\, m = 0, 1, \dots, N - 1
  \label{eq:1.5}
\end{equation}
The normal-mode expansion can then be written as:
\begin{equation}
  \eta (t) = \sum_{m = 0}^{N - 1} \left[ \mathcal{A}_m e^{i \left( k_m n - \omega_m t \right)} + \mathcal{A}^* e^{-i \left( k_m n - \omega_m t \right)} \right]
  \label{eq:1.6}
\end{equation}
where the complex conjugate is added to ensure that the total displacement is real. The momentum canonically-conjucated to the displacement is defined as:
\begin{equation}
  \pi_n \defeq \frac{\pa L}{\pa \dot{\eta}_n} = \dot{\eta}_n
  \label{eq:1.7}
\end{equation}
In quantum mechanics, $ \eta_n $ and $ \Pi_n $ become operators with canonical commutator $ [ \hat{\eta}_j, \hat{\pi}_k ] = i \hbar \delta_{jk} $. Implementing time evolution with the \textit{Heisenberg picture}\footnote{Recall that $ \hat{\mathcal{O}}(t) = e^{\frac{i}{\hbar} \hat{\mathcal{H}} t} \hat{\mathcal{O}}(0) e^{-\frac{i}{\hbar} \hat{\mathcal{H}} t} $ and $ \frac{\dd \hat{\mathcal{O}}}{\dd t} = \frac{i}{\hbar} [ \hat{\mathcal{H}}, \hat{\mathcal{O}} ] $.}:
\begin{equation}
  [ \hat{\eta}_j(t), \hat{\pi}_k(t) ] = i \hbar \delta_{jk}
  \label{eq:1.8}
\end{equation}
The commutator of operators evaluated at different times requires solving the dynamics of the system. It is useful to introduce \textit{annihilation} and \textit{creation operators} $\footnote{For a harmonic oscillator $ \hat{\mathcal{H}} = \frac{1}{2} \hat{p}^2 + \frac{1}{2} \omega^2 \hat{x}^2 $, so $ \frac{\dd \hat{x}}{\dd t} = \hat{p}(t) $ and $ \frac{\dd \hat{p}}{\dd t} = -\omega^2 \hat{x}(t) $ and the solution can be written as: $$ \hat{x}(t) = \sqrt{\frac{\hbar}{2\omega}} \left[ \hat{a}(t) + \hat{a}^\dagger(t) \right] \qquad \qquad \hat{p}(t) = -i \omega \sqrt{\frac{\hbar}{2\omega}} \left[ \hat{a}(t) - \hat{a}^\dagger(t) \right] $$ Inverting these expressions one finds $ [\hat{a}(t), \hat{a}^\dagger(t)] = 1 $ and $ \hat{\mathcal{H}} = \hbar \omega \left( \hat{a}^\dagger(t) \hat{a} + \frac{1}{2} \right) $. The time evolution $ \hat{a}(t) = e^{-i \omega t} \hat{a}(0) $ ensures that $ \hat{\mathcal{H}} $ is times-independent.} \hat{a}(t) $ and $ \hat{a}^\dagger(t) $, so that Eq. \ref{eq:1.6} becomes:
\begin{equation}
  \hat{\eta}_n(t) = \sum_{m = 0}^{N - 1} \sqrt{\frac{\hbar}{2 \omega_m}} \frac{1}{\sqrt{N}} \left[ e^{i \left( k_m n - \omega_m t \right)} \hat{a}_m + e^{-i \left( k_m n - \omega_m t \right)} \hat{a}_m^\dagger \right]
  \label{eq:1.9}
\end{equation}
where $ [ \hat{a}_j, \hat{a}_k^\dagger ] = \delta_{jk} $ and the $ N^{-1/2} $ ensures the normalization of normal modes. The proof of Eq. \ref{eq:1.8} follows from the finite Fourier series identity (sum of a geometric progression):
\begin{equation}
  \sum_{m = 0}^{N - 1} e^{i k_m \left( n - n' \right)} = N \delta_{n n'}
  \label{eq:1.10}
\end{equation}
The Hamiltonian of the system can be written as:
\begin{equation}
  \hat{\mathcal{H}} = \sum_{n} \left[ \frac{1}{2} \hat{\pi}_n^2 + \frac{\lambda}{2} \left( \hat{\eta}_n - \hat{\eta}_{n-1} \right)^2 \right] = \sum_{m = 0}^{N - 1} \hbar \omega_m \left( \hat{a}_m^\dagger \hat{a}_m + \frac{1}{2} \right)
  \label{eq:1.11}
\end{equation}
Generalizing the harmonic oscillator operator algebra (proven unique by Von Neumann), one can construct the Hilbert space for the harmonic crystal as:
\begin{equation}
  \hat{a}_m \ket{0} \quad \forall m = 0, 1, \dots, N - 1
  \label{eq:1.12}
\end{equation}
\begin{equation}
  \ket{n_0, n_1, \dots, n_{N-1}} = \prod_{m = 0}^{N - 1} \frac{( \hat{a}_m^\dagger )^{n_m}}{\sqrt{n_m!}} \ket{0}
  \label{eq:1.13}
\end{equation}
These are normalized eigenstates of Eq. \ref{eq:1.1} with energy eigenvalues:
\begin{equation}
  E_0 = \frac{1}{2} \sum_{m = 0}^{N - 1} \hbar \omega_m
  \label{eq:1.14}
\end{equation}
\begin{equation}
  E_{n_0, n_1, \dots, n_{N-1}} = E_0 + \sum_{m = 0}^{N - 1} n_m \hbar \omega_m
  \label{eq:1.15}
\end{equation}
This Hilbert space is called \textit{Fock space} and the excited states \textit{phonons}: these can be thought as $ \virgolette{particles} $ and $ n_m $ is the number of phonons in the $ m^{\mathrm{th}} $ normal mode.

\subsection{One-dimensional harmonic string}

Taking the continuum limit, the crystal becomes a string: to achieve this, one takes the limits $ a \rightarrow 0 $ and $ N \rightarrow \infty $ while keeping the total length $ R \equiv N a $ fixed. In this context, the displacement becomes a field $ \eta(x,t) $ dependent on the continuous real coordinate $ x \in [0, R] $, therefore:
\begin{equation*}
  \left( \eta_{n + 1} - \eta_n \right)^2 \longrightarrow a^2 \left( \frac{\pa \eta}{\pa x} \right)^2
  \qquad \qquad
  \sum_{n} \longrightarrow \frac{1}{a} \int_0^R \dd x
\end{equation*}

\begin{proposition}\label{prop-delta-cont}
  In the continuous limit:
  \begin{equation*}
    \frac{\delta_{nn'}}{a} \longrightarrow \delta(x - x') = \int_\R \frac{\dd k}{2\pi} e^{i k (x - x')}
  \end{equation*}
\end{proposition}
\begin{proof}
  By direct calculation:
  \begin{equation*}
    a \sum_{n} f(an) \frac{\delta_{nm}}{a} = f(ma) \longrightarrow f(y) = \int_0^R \dd x\, f(x) \delta(x - y)
  \end{equation*}
  Recalling Eq. \ref{eq:1.10}, since $ k_m n = \frac{k_m}{a} na \rightarrow k x $, symmetrizing $ k_m \in [-\pi, \pi] $ (instead of $ [0, 2\pi] $) one finds:
  \begin{equation*}
    \delta(x - x') \longleftarrow \frac{\delta_{nn'}}{a} = \frac{1}{Na} \sum_{m} e^{ik_m (n - n')} \longrightarrow \int_\R \frac{\dd k}{2\pi} e^{ik (x - x')}
  \end{equation*}
  where integration limits are $ \pm \frac{\pi}{a} \rightarrow \pm \infty $.
\end{proof}
\begin{proof}
  The inverse Fourier transform of the Dirac Delta reads:
  \begin{equation*}
    \int_0^R \dd x\, e^{i (k - k') x} = 2\pi \delta(k - k')
  \end{equation*}
\end{proof}
By these relations, it can be seen that $ \frac{\dd k}{2\pi} $ has the physical meaning of the number of normal modes per unit spatial volume with wavenumber between $ k $ and $ k + \dd k $, while the interpretation of the divergent $ \delta(0) $ varies: in $ x $ space, it is the reciprocal of the lattice spacing, i.e. the number of normal modes per unit spatial volume, but in $ k $ space $ 2\pi \delta(0) $ is the (hyper-)volume of the system.\\
In the continuous limit, the Lagrangian of the harmonic string becomes:
\begin{equation*}
  L = \int_0^R \dd x \left[ \frac{1}{2} \rho_0 (\pa_t \eta)^2 - \frac{\kappa}{2} (\pa_x \eta)^2 \right]
\end{equation*}
where $ \rho_0 $ is the equilibrium mass density of the string. It is customary to absorb constants in the fields, thus, setting $ \phi(x,t) \equiv \sqrt{\rho_0} \eta(x,t) $ and $ \kappa = c^2 \rho_0 $ and adding a pinning term $ \propto \varphi^2 $, the Lagrangian can be written as:
\begin{equation}
  L = \int_0^R \dd x \left[ \frac{1}{2} (\pa_t \phi)^2 - \frac{c^2}{2} (\pa_x \phi)^2 - \frac{m^2 c^4}{2} \phi^2 \right]
  \label{eq:1.16}
\end{equation}
The classical equation of motion of this field yields:
\begin{equation}
  \pa_t^2 \phi = c^2 \pa_x^2 \phi - m^2 c^4 \phi
  \label{eq:1.17}
\end{equation}
The solutions of this wave equation can be written as:
\begin{equation}
  \phi(x,t) = e^{i \left( kx - \omega_k t \right)}
  \label{eq:1.18}
\end{equation}
with dispersion relation:
\begin{equation}
  \omega_k^2 = c^2 k^2 + m^2 c^4
  \label{eq:1.19}
\end{equation}
To quantize this system, one needs to compute the Hamiltonian. The canonical momentum field is:
\begin{equation}
  \Pi(x,t) \defeq \frac{\pa L}{\pa (\pa_t \phi)} = \pa_t \phi(x,t)
  \label{eq:1.20}
\end{equation}
The classical Hamiltonian can then be found as:
\begin{equation}
  \hat{\mathcal{H}} = \int_0^R \dd x \left[ \frac{1}{2} \Pi^2 + \frac{c^2}{2} (\pa_x \phi)^2 + \frac{m^2 c^4}{2} \phi^2 \right]
  \label{eq:1.21}
\end{equation}
The quantum field is analogous to Eq. \ref{eq:1.9}:
\begin{equation}
  \hat\phi(x,t) = \int_\R \frac{\dd k}{2\pi} \sqrt{\frac{\hbar}{2\omega_k}} \left[ e^{i \left( kx - \omega_k t \right)} \hat{a}_k + e^{-i \left( kx - \omega_k t \right)} \hat{a}_k^\dagger \right]
  \label{eq:1.22}
\end{equation}
with commutation relations:
\begin{equation}
  [\hat{a}_k, \hat{a}_{k'}^\dagger] = 2\pi \delta(k - k')
  \label{eq:1.23}
\end{equation}
\begin{equation}
  [\hat\phi(x,t), \hat\Pi(x',t)] = i\hbar \delta(x - x')
  \label{eq:1.24}
\end{equation}
The quantum Hamiltonian can be written as:
\begin{equation}
  \hat{\mathcal{H}} = \int_\R \frac{\dd k}{2\pi} \frac{1}{2} \hbar \omega_k \left( \hat{a}_k^\dagger \hat{a}_k + \hat{a}_k \hat{a}_k^\dagger \right) = E_0 + \int_\R \frac{\dd k}{2\pi} \hbar \omega_k \hat{a}_k^\dagger \hat{a}_k
  \label{eq:1.25}
\end{equation}
The ground-state energy can be computed from Eq. \ref{eq:1.14}, defining $ \mathrm{Vol} \defeq 2\pi \delta(k = 0) $:
\begin{equation}
  E_0 = \mathrm{Vol} \int_\R \frac{\dd k}{2\pi} \frac{1}{2} \hbar \omega_k
  \label{eq:1.26}
\end{equation}
For a strictly continuous system there is no cut-off in the $ k $ integral, thus the zero-point energy diverges: however, this is not necessarily a problem, as often only changes in $ E_0 $ are relevant (and experimentally accessible), and in this case it is known as \textit{Casimir energy}.

\section{Spacetime symmetries}

\subsection{Lie groups}

\begin{definition}
  A \textit{Lie group} is a group whose elements depend in a continuous and differentiable way on a set of real parameters $ \{\theta_a\}_{a = 1, \dots, d} \subset \R^d $.
\end{definition}

A Lie group can be seen both as a group and as a $ d $-dimensional differentiable manifold (with coordinates $ \theta_a $). WLOG it is always possible to choose $ g(0,\dots,0) = e $.

\begin{definition}
  Given a group $ G $ and a vector space $ V(\K) $, a \textit{representation} of $ G $ on $ V $ is a homomorphism $ \rho : G \rightarrow \mathrm{GL}(V) $.
\end{definition}

Given the isomorphism $ \mathrm{GL}(V) \rightarrow \K^{n \times n} $, with $ n \equiv \dim_\K V $, it is usual to \textit{de facto} represent $ G $ as matrices acting on elements of $ V $, i.e. $ \rho : G \rightarrow \K^{n \times n} $.

\begin{proposition}
  Given a Lie group $ G $ and $ g \in G $ connected with the identity, a representation of degree $ n $ on $ V(\C) $ as:
  \begin{equation}
    \rho(g(\theta)) = e^{i \theta_a T^a}
    \label{eq:1.27}
  \end{equation}
  where $ \{T^a\}_{a = 1, \dots, d} \subset \C^{n \times n} $ are the \textit{generators} of $ G $ on $ V $.
\end{proposition}

\begin{definition}
  Given a Lie group $ G $ with generators $ \{T^a\}_{a = 1, \dots, d} \subset \C^{n \times n} $ on $ V(\C) $, its \textit{Lie algebra} is:
  \begin{equation}
    [T^a, T^b] = i \tensor{f}{^a^b_c} T^c
    \label{eq:1.28}
  \end{equation}
  where $ \tensor{f}{^a^b_c} $ are called the \textit{structure constants}.
\end{definition}

\begin{proposition}
  The Lie algebra of a Lie group is independent of the representation.
\end{proposition}

\begin{proposition}
  Any $ d $-dimensional abelian Lie algebra is isomorphic to the direct sum of $ d $ one-dimensional Lie algebras.
\end{proposition}

As a consequence, all irreducible representations of an abelian Lie group are of degree $ n = 1 $.

\begin{definition}
  Given a Lie group with generators $ \{T^a\}_{a = 1, \dots, d} \subset \C^{n \times n} $ on $ V(\C) $, a \textit{Casimir operator} is an operator which commutes with each generator.
\end{definition}

Given an irreducible representation, Casimir operators are operators proportional to $ \id_V $, and the proportionality constants can be used to label the representation: they correspond to conserved physical quantities.

\begin{proposition}\label{prop-non-comp-triv}
  A non-compact group cannot have finite unitary representations, except for those with trivial non-compact generators.
\end{proposition}

This means that the non-compact component of a group cannot be represented with unitary operators of finite dimension.

\subsection{Lorentz group}

Consider the group of linear transformations $ x^\mu \mapsto \tensor{\Lambda}{^\mu_\nu} x^\nu $ on $ \R^{1,3} $ which leave invariant the quantity $ \eta_{\mu \nu} x^\mu x^\nu $, i.e. the orthogonal group $ \On{1,3} $ (with signature $ (+,-,-,-) $). The condition that $ \tensor{\Lambda}{^\mu_\nu} $ must satisfy reads:
\begin{equation}
  \eta_{\rho \sigma} = \eta_{\mu \nu} \tensor{\Lambda}{^\mu_\rho} \tensor{\Lambda}{^\nu_\sigma}
  \label{eq:1.29}
\end{equation}
This implies that $ \det \Lambda = \pm 1 $: a transformation with $ \det \Lambda = -1 $ can always be written as the product of a transformation with $ \det \Lambda = 1 $ and a discrete transformation which reverses the sign of an odd number of coordinates. One further defines $ \SOn{1,3} \defeq \{\Lambda \in \On{1,3} : \det \Lambda = 1\} $.\\
Writing explicitly the temporal component $ 1 = (\tensor{\Lambda}{^0_0})^2 - (\tensor{\Lambda}{^1_0})^2 - (\tensor{\Lambda}{^2_0})^2 - (\tensor{\Lambda}{^3_0})^2 $, it is clear that $ (\tensor{\Lambda}{^0_0})^2 \ge 1 $. Therefore, $ \On{1,3} $ has two disconnected components: the orthochronous component with $ \tensor{\Lambda}{^0_0} \ge 1 $ and the non-orthochronous component with $ \tensor{\Lambda}{^0_0} \le 1 $. Any non-orthochronous transformation can be written as the product of an orthochronous transformation and a dircrete transformation which reverses the sign of the temporal component.

\begin{definition}
  The \textit{Lorentz group} is the orthochronous component of $ \SOn{1,3} $.
\end{definition}

The discrete transformations are factored out of the Lorentz group. Considering the infinitesimal transformation and applying Eq. \ref{eq:1.29}:
\begin{equation*}
  \tensor{\Lambda}{^\mu_\nu} = \delta^\mu_\nu + \tensor{\omega}{^\mu_\nu}
  \qquad \Rightarrow \qquad
  \omega_{\mu \nu} = - \omega_{\nu \mu}
\end{equation*}
Anti-symmetry means that $ \omega_{\mu \nu} $ has only 6 parameters, which define the Lorentz group: these can be identified by the 3 angles of spherical rotations in the $ (x,y) $, $ (y,z) $ and $ (z,x) $ planes and the 3 rapidities of hyperbolic rotations in the $ (t,x) $, $ (t,y) $ and $ (t,z) $ planes.

\begin{proposition}
  The Lorentz group is a non-compact Lie group.
\end{proposition}
\begin{proof}
  Spherical and hyperbolic rotations are continuous and differential w.r.t. angles and rapidities, but while angles vary in $ [0, 2\pi) $, rapidities vary in $ \R $, so the differentiable manifold associated to $ \SOn{1,3} $ is not compact.
\end{proof}

\subsubsection{Lorentz algebra}

The 6 parameters of the Lorentz group correspond to 6 generators of the associated Lorentz algebra. Labelling these generators as $ J^{\mu \nu} : J^{\mu \nu} = - J^{\nu \mu} $, the generic element $ \Lambda \in \SOn{1,3} $ can be written as:
\begin{equation}
  \Lambda = e^{- \frac{i}{2} \omega_{\mu \nu} J^{\mu \nu}}
  \label{eq:1.30}
\end{equation}
The $ \frac{1}{2} $ factor arises from each generator being counted twice (product of two anti-symmetric objects). Given a $ n $-dimensional representation of $ \SOn{1,3} $, both $ \tensor{[J^{\mu \nu}]}{^i_j} $ and $ \tensor{[\Lambda]}{^i_j} $ are $ \C^{n \times n} $ matrices ($ \Lambda $ is real): for example, the $ n = 1 $ representation acts on \textit{scalars}, which are invariant under Lorentz transformations, so $ J^{\mu \nu} \equiv 0 \,\forall \mu,\nu = 0, \dots, 3 $.

\paragraph{4-vectors}

The $ n = 4 $ representation acts on \textit{contravariant 4-vectors} $ v^\mu $, which transform according to $ v^\mu \mapsto \tensor{\Lambda}{^\mu_\nu} v^\nu $, and \textit{covariant 4-vectors} $ v_\mu $, which transform according to $ v_\mu \mapsto \tensor{\Lambda}{_\mu^\nu} v_\nu $. In this representation, the generators are represented as $ \C^{4 \times 4} $ matrices:
\begin{equation}
  \tensor{[J^{\mu \nu}]}{^\rho_\sigma} = i \left( \eta^{\mu \rho} \delta^\nu_\sigma - \eta^{\nu \rho} \delta^\mu_\sigma \right)
  \label{eq:1.31}
\end{equation}
This is an irreducible representation, and the associated Lie algebra $ \mathfrak{so}(1,3) $, called the \textit{Lorentz algebra}, is:
\begin{equation}
  [J^{\mu \nu}, J^{\sigma \rho}] = i \left( \eta^{\nu \rho} J^{\mu \sigma} - \eta^{\mu \rho} J^{\nu \sigma} - \eta^{\nu \sigma} J^{\mu \rho} + \eta^{\mu \rho} J^{\nu \sigma} \right)
  \label{eq:1.32}
\end{equation}
It is convenient to rearrange the 6 components of $ J^{\mu \nu} $ into two spatial vectors:
\begin{equation}
  J^i \defeq \frac{1}{2} \epsilon^{ijk} J^{jk}
  \qquad \qquad
  K^i \defeq J^{i0}
  \label{eq:1.33}
\end{equation}
The $ \mathfrak{so}(1,3) $ can then be rewritten as:
\begin{equation}
  [J^i, J^j] = i \epsilon^{ijk} J^k
  \qquad \qquad
  [J^i, K^j] = i \epsilon^{ijk} K^k
  \qquad \qquad
  [K^i, K^j] = - i \epsilon^{ijk} J^k
  \label{eq:1.34}
\end{equation}
The first equation defines a $ \mathfrak{su}(2) $ sub-algebra, thus showing that $ J^i $ are the generators of angular momentum. Angles and rapidities are then defined as:
\begin{equation}
  \theta^i \defeq \frac{1}{2} \epsilon^{ijk} \omega^{jk}
  \qquad \qquad
  \eta^i \defeq \omega^{i0}
  \label{eq:1.35}
\end{equation}
so that:
\begin{equation}
  \Lambda = \exp \left[ -i \boldsymbol{\theta} \cdot \ve{J} + i \boldsymbol{\eta} \cdot \ve{K} \right]
  \label{eq:1.36}
\end{equation}
This definition reflect the \textit{alias} interpretation: the angles define counterclockwise rotations of vectors with respect to a fixed reference frame, while rapidities define boosts wich increase velocities with respect to said frame.

\subsubsection{Tensor Representations}

A generic $ (p,q) $-tensor transforms as:
\begin{equation}
  \tensor{T}{^{\mu_1 \dots \mu_p}_{\nu_1 \dots \nu_q}} \mapsto \tensor{\Lambda}{^{\mu_1}_{\alpha_1}} \dots \tensor{\Lambda}{^{\mu_p}_{\alpha_p}} \tensor{\Lambda}{_{\nu_1}^{\beta_1}} \dots \tensor{\Lambda}{_{\nu_q}^{\beta_q}} \tensor{T}{^{\alpha_1 \dots \alpha_p}_{\beta_1 \dots \beta_q}}
  \label{eq:1.37}
\end{equation}
The representation of the Lorentz group which acts on $ (p,q) $-tensors is of degree $ n = 4^{p + q} $, however it is reducible into the direct product of $ p + q $ 4-dimensional representations as of Eq. \ref{eq:1.38}.\\
Moreover, consider the action of the Lorentz group on $ (2,0) $-tensors: being $ T^{\mu \nu} \mapsto \tensor{\Lambda}{^\mu_\alpha} \tensor{\Lambda}{^\nu_\beta} T^{\alpha \beta} $, if $ T^{\mu \nu} $ is (anti-)symmetric it will remain so under a Lorentz transformation. Therefore, the 16-dimensional representation reduces to a 6-dimensional representation on anti-symmetric tensors and a 10-dimensional representation of symmetric tensors. Furthermore, the trace of a symmetric tensor is invariant, as $ T \equiv \eta_{\mu \nu} T^{\mu \nu} \mapsto \eta_{\mu \nu} \tensor{\Lambda}{^\mu _\alpha} \tensor{\Lambda}{^\nu _\beta} T^{\alpha \beta} = T $, so the latter representation further reduces into a 9-dimensional representation on symmetric traceless tensors and a 1-dimensional representation on scalars. This means that:
\begin{equation}
  \mathtt{4} \otimes \mathtt{4} = \mathtt{1} \oplus \mathtt{6} \oplus \mathtt{9}
  \label{eq:1.38}
\end{equation}
These are irreducible representations which, given a generic tensor $ T^{\mu \nu} $, act on $ S $, $ A^{\mu \nu} $ and $ S^{\mu \nu} - \frac{1}{4} \eta^{\mu \nu} S $ respectively, with $ A^{\mu \nu} \equiv \frac{1}{2} \left( T^{\mu \nu} - T^{\nu \mu} \right) $ and $ S^{\mu \nu} \equiv \frac{1}{2} \left( T^{\mu \nu} + T^{\nu \mu} \right) $.

\paragraph{Decomposition under rotations}

Restricting the action to the $ \SOn{3} $ sub-group of $ \SOn{1,3} $, tensors can be decomposed according to irreducible representations of $ \SOn{3} $, which are labelled by the angular momentum $ j \in \N_0 $ and are of degree $ n = 2j + 1 $. Also recall the Clebsh-Gordan composition of angular momenta:
\begin{equation}
  \ve{j}_1 \otimes \ve{j}_2 = \bigoplus_{j = \abs{j_1 - j_2}}^{j_1 + j_2} \ve{j}
  \label{eq:1.39}
\end{equation}
A Lorentz scalar $ \alpha $ is a scalar under rotations too, so $ \alpha \in \ve{0} $. A 4-vector $ v^\mu $ is irreducible under the action of $ \SOn{1,3} $, but under $ \SOn{3} $ it is decomposed into $ v^0 $ and $ \ve{v} $, so $ v^\mu \in \ve{0} \oplus \ve{1} $. A $ (2,0) $-tensor then is:
\begin{equation*}
  \begin{split}
    T^{\mu \nu} \in \left( \ve{0} \oplus \ve{1} \right) \otimes \left( \ve{0} \oplus \ve{1} \right)
    &= \left( \ve{0} \otimes \ve{0} \right) \oplus \left( \ve{0} \otimes \ve{1} \right) \oplus \left( \ve{1} \otimes \ve{0} \right) \oplus \left( \ve{1} \otimes \ve{1} \right) \\
    &= \ve{0} \oplus \ve{1} \oplus \ve{1} \oplus \left( \ve{0} \oplus \ve{1} \oplus \ve{2} \right)
  \end{split}
\end{equation*}
This is equivalent to Eq. \ref{eq:1.38}: the trace is a scalar, so $ S \in \ve{0} $, while the anti-symmetric part can be written as two spatial vectors $ A^{0i} $ and $ \frac{1}{2} \epsilon^{ijk} A^{jk} $, so $ A^{\mu \nu} \in \ve{1} \oplus \ve{1} $. The traceless symmetric part then decomposes as $ \bar{S}^{\mu \nu} \in \ve{0} \oplus \ve{1} \oplus \ve{2} $ under spatial rotations.\\
Equivalently, $ T^{\mu \nu} $ can be decomposed into $ T^{00} \in \left( \ve{0} \otimes \ve{0} \right) $, $ T^{0i} \in \left( \ve{0} \otimes \ve{1} \right) $, $ T^{i0} \in \left( \ve{1} \otimes \ve{0} \right) $ and $ T^{ij} \in \left( \ve{1} \otimes \ve{1} \right) $: the formers are a scalar and two spatial vectors associated to $ \ve{0} \oplus \ve{1} \oplus \ve{1} $, while the latter can be decomposed into the trace, which is $ \ve{0} $, the anti-symmetric part, which is $ \ve{1} $ ($ \epsilon^{ijk} A^{jk} $), and the traceless symmetric part, which is $ \ve{2} $.

\begin{example}
  Gravitational waves in de Donder gauge are described by a traceless symmetric matrix, therefore they have $ j = 2 $ (spin of the graviton).
\end{example}

There are two \textit{invariant tensors} under $ \SOn{1,3} $: the metric $ \eta_{\mu \nu} $, by Eq. \ref{eq:1.29}, and the Levi-Civita symbol $ \epsilon^{\mu \nu \sigma \rho} $:
\begin{equation*}
  \epsilon^{\mu \nu \sigma \rho} \mapsto \tensor{\Lambda}{^\mu_\alpha} \tensor{\Lambda}{^\nu_\beta} \tensor{\Lambda}{^\sigma_\gamma} \tensor{\Lambda}{^\rho_\delta} \epsilon^{\alpha \beta \gamma \delta} = \left( \det \Lambda \right) \epsilon^{\mu \nu \sigma \rho} = \epsilon^{\mu \nu \sigma \rho}
\end{equation*}


\subsubsection{Spinor representations}


The Lie algebras $ \mathfrak{su}(2) $ and $ \mathfrak{so}(3) $ are the same, which means that $ \SUn{2} $ and $ \SOn{3} $ are indistinguishable by infinitesimal transformations; however, they are globally different, as $ \SOn{3} $ rotations are periodic by $ 2\pi $, while $ \SUn{2} $ rotations are periodic by $ 4\pi $: in particular, it can be shown that $ \SOn{3} \cong \SUn{2} / \Z_2 $, i.e. $ \SUn{2} $ is the universal cover of $ \SOn{3} $. This means that $ \SUn{2} $ representations can be labelled by $ j \in \frac{1}{2} \N_0 $, where half-integer spin representations are known as \textit{spinorial representations}: they act on spinors, i.e. objects which change sign under rotations of $ 2\pi $ (thus not suitable to represent $ \SOn{3} $).

\begin{example}
  The $ \ve{\frac{1}{2}} $ representation of $ \SUn{2} $ is a 2-dimensional representation where $ J^i = \frac{\sigma^i}{2} $: Pauli matrices satisfy $ \sigma^i \sigma^j = \delta^{ij} + i \epsilon^{ijk} \sigma^k $, thus the $ \mathfrak{su}(2) $ algebra is satisfied. Denoting the $ m = \pm \frac{1}{2} $ states in the $ \ve{\frac{1}{2}} $ representation as $ \ket{\uparrow} $ and $ \ket{\downarrow} $, the Clebsch-Gordan decomposition $ \ve{\frac{1}{2}} \otimes \ve{\frac{1}{2}} = \ve{0} \oplus \ve{1} $ yields the triplet ($ j = 1 $) $ \ket{\uparrow\uparrow} $, $ \frac{1}{\sqrt{2}}\left( \ket{\uparrow\downarrow} + \ket{\downarrow\uparrow} \right) $, $ \ket{\downarrow\downarrow} $ and the singlet ($ j = 0 $) $ \frac{1}{\sqrt{2}}\left( \ket{\uparrow\downarrow} - \ket{\downarrow\uparrow} \right) $.
\end{example}

\begin{proposition}\label{prop-lor-alg}
  The Lorentz algebra $ \mathfrak{so}(1,3) $ can be decomposed as $ \mathfrak{su}(2) \times \mathfrak{su}(2) $.
\end{proposition}
\begin{proof}
  Given the $ \mathfrak{so}(1,3) $ algebra in Eq. \ref{eq:1.33}, it is possible to define:
  \begin{equation*}
    \ve{J}_{\pm} \defeq \frac{1}{2} \left( \ve{J} \pm i \ve{K} \right)
  \end{equation*}
  The Lie algebra then becomes:
  \begin{equation*}
    [\ve{J}_{\pm}^i, \ve{J}_{\pm}^j] = i \epsilon^{ijk} \ve{J}_{\pm}^k
    \qquad \qquad
    [\ve{J}_{\pm}^i, \ve{J}_{\mp}^j] = 0
  \end{equation*}
  These are two commuting $ \mathfrak{so}(2) $ algebras, thus proving the thesis.
\end{proof}

As observed before, this does not imply that $ \SOn{1,3} $ is globally equivalent to $ \SUn{2} \times \SUn{2} $: in fact, $ \SUn{2} \times \SUn{2} / \Z_2 \cong \SOn{4} $, while the universal cover of $ \SOn{1,3} $ is $ \SL{2}{\C} $, as it can be shown that $ \SOn{1,3} \cong \SL{2}{\C} / \Z_2 $.\\
By Prop. \ref{prop-lor-alg}, representations of $ \SOn{1,3} $ can be labelled by $ (j_-,j_+) \in \frac{1}{2} \N_0 \times \frac{1}{2} \N_0 $, with each index labelling a representation of $ \SUn{2} $: as $ \ve{J} = \ve{J}_+ + \ve{J}_- $, the $ (j_-,j_+) $ representation contains states with all possible spins $ \abs{j_+ - j_-} \le j \le j_+ + j_- $, and it is a representation of degree $ n = \left( 2j_- + 1 \right) \left( 2j_+ + 1 \right) $.\\
$ \ve{\left( 0,0 \right)} $ is the trivial (scalar) representation, as both $ \ve{J}_{\pm} = 0 $ and $ \ve{J} = \ve{K} = 0 $.\\
$ \ve{\left( \frac{1}{2},0 \right)} $ and $ \ve{\left( 0,\frac{1}{2} \right)} $ are 2-dimensional spinorial representations. These representations act on different spinors $ (\psi_\text{L})_\alpha \in \ve{\left( \frac{1}{2},0 \right)} $ and $ (\psi_\text{R})_\alpha \in \ve{\left( 0,\frac{1}{2} \right)} $, with $ \alpha = 1,2 $, which are called \textit{left-} and \textit{right-handed  Weyl spinors}. In $ \ve{\left( \frac{1}{2},0 \right)} $ the generators are $ \ve{J}_- = \frac{\boldsymbol{\sigma}}{2} $ and $ \ve{J}_+ = \ve{0} $, while in $ \ve{\left( 0,\frac{1}{2} \right)} $ they are $ \ve{J}_- = \ve{0} $ and $ \ve{J}_+ = \frac{\boldsymbol{\sigma}}{2} $, thus one finds $ \ve{J}_\text{L} = \ve{J}_\text{R} = \frac{\boldsymbol{\sigma}}{2} $ and $ \ve{K}_\text{L} = - \ve{K}_\text{R} = i \frac{\boldsymbol{\sigma}}{2} $, so that:
\begin{equation}
  \psi_\text{L} \mapsto \Lambda_\text{L} \psi_\text{L} = \exp \left[ \left( -i \boldsymbol{\theta} - \boldsymbol{\eta} \right) \cdot \frac{\boldsymbol{\sigma}}{2} \right] \psi_\text{L}
\end{equation}
\begin{equation}
  \psi_\text{R} \mapsto \Lambda_\text{R} \psi_\text{R} = \exp \left[ \left( -i \boldsymbol{\theta} + \boldsymbol{\eta} \right) \cdot \frac{\boldsymbol{\sigma}}{2} \right] \psi_\text{R}
  \label{eq:1.40}
\end{equation}
Note that the generators $ K^i $ are not hermitian, as expected from Prop. \ref{prop-non-comp-triv}. Furthermore, note that $ \Lambda_{\text{L},\text{L}} \in \C^{2 \times 2} $, therefore $ \psi_{\text{L},\text{R}} \in \C^2 $.

\begin{proposition}\label{prop-weyl-spin-par}
  Given $ \psi_\text{L} \in \ve{\left( \frac{1}{2},0 \right)} $ and $ \psi_\text{R} \in \ve{\left( 0,\frac{1}{2} \right)} $, then $ \sigma^2 \psi_\text{L}^* \in \ve{\left( 0,\frac{1}{2} \right)} $ and $ \sigma^2 \psi_\text{R}^* \in \ve{\left( \frac{1}{2},0 \right)} $.
\end{proposition}
\begin{proof}
  Recall that for Pauli matrices $ \sigma^2 \sigma^i \sigma^2 = -(\sigma^i)^* $, so $ \sigma^2 \Lambda_\text{L}^* \sigma^2 = \Lambda_\text{R} $ and:
  \begin{equation*}
    \sigma^2 \psi_\text{L}^* \mapsto \sigma^2 \left( \Lambda_\text{L} \psi_\text{L} \right)^* = \left( \sigma^2 \Lambda_\text{L}^* \sigma^2 \right) \sigma^2 \psi_\text{L}^* = \Lambda_\text{R} \sigma^2 \psi_\text{L}^*
    \quad \Rightarrow \quad
    \sigma^2 \psi_\text{L}^* \in \ve{\left( 0,\tfrac{1}{2} \right)}
  \end{equation*}
  where $ \sigma^2 \sigma^2 = \tens{I}_2 $ was used. The proof for $ \sigma^2 \psi_\text{R}^* $ is analogous.
\end{proof}

\begin{definition}
  On Weyl spinors, the \textit{charge conjugation operator} is defined as:
  \begin{equation}
    \psi_\text{L}^c \defeq i \sigma^2 \psi_\text{L}^*
    \qquad \qquad
    \psi_\text{R}^c \defeq -i \sigma^2 \psi_\text{R}^*
    \label{eq:1.41}
  \end{equation}
\end{definition}

By Prop. \ref{prop-weyl-spin-par}, charge conjugation changes transforms a left-handed Weyl spinor into a right-handed one and vice versa. Moreover, the $ i $ factor ensures that applying this operator twice yields the identity operator.\\
$ \ve{\left( \frac{1}{2},\frac{1}{2} \right)} $ is a 4-dimensional complex representation: as $ j = 0,1 $, this representation acts on complex 4-vectors of the form $ \left( (\psi_\text{L})_\alpha, (\xi_\text{R})_\beta \right) \in \C^4 $, and $ \Lambda = \diag \left( \Lambda_\text{L}, \Lambda_\text{R} \right) \in \C^{4 \times 4} $. To explicit this relation, set $ \psi_\text{R} \equiv i \sigma^2 \psi_\text{L}^* $, $ \xi_\text{L} \equiv -i \sigma^2 \xi_\text{R}^* $ and $ \sigma^\mu \equiv \left( 1, \boldsymbol{\sigma} \right) $, $ \bar\sigma^\mu \equiv \left( 1, -\boldsymbol{\sigma} \right) $: it can be shown, then, that $ \xi_\text{R}^\dagger \sigma^\mu \psi_\text{R} $ and $ \xi_\text{L}^\dagger \bar\sigma^\mu \psi_\text{L} $ are contravariant 4-vectors. Although these 4-vectors are complex by construction, being the matrix $ \tensor{\Lambda}{^\mu_\nu} $ which represents the Lorentz transformation of a 4-vector real, a reality condition $ v_\mu^* = v_\mu $ is Lorentz invariant.

\subsubsection{Field representations}

Given a field $ \phi(x) $, under a Lorentz transformation $ x^\mu \mapsto {x'}^\mu = \tensor{\Lambda}{^\mu_\nu} x^\nu $ it transforms as $ \phi(x) \mapsto \phi'(x') $.

\paragraph{Scalar fields}

A scalar field transforms as:
\begin{equation}
  \phi'(x') = \phi(x)
  \label{eq:1.43}
\end{equation}
Consider an infinitesimal transformation $ x'^\rho = x^\rho + \delta x^\rho $, with $ \delta x^\rho = - \frac{i}{2} \omega_{\mu \nu} \tensor{\left[ J^{\mu \nu} \right]}{^\rho_\sigma} x^\sigma $ as of Eq. \ref{eq:1.30}. Then, by definition, $ \delta \phi \equiv \phi'(x') - \phi(x) = 0 $, which corresponds to the fact that the scalar representation of $ \SOn{1,3} $ is the trivial one ($ J^{\mu \nu} = 0 $).\\
However, one can consider the variation at fixed coordinate $ \delta_0 \phi \equiv \phi'(x) - \phi(x) $: while $ \delta \phi $ studies only a single degree of freedom, as the point $ \text{P} \in \R^{1,3} $ is kept constant and only $ \phi(\text{P}) $ can vary (i.e. the base space is one-dimensional), $ \delta_0 \phi $ studies $ \phi(\text{P}) $ with $ \text{P} $ varying over $ \R^{1,3} $, thus the base space is now a space of functions, which is infinite-dimensional. Therefore, $ \delta \phi $ provides a finite-dimensional representation of the generators, while $ \delta_0 \phi $ an infinite-dimensional one.\\
To explicit this representation:
\begin{equation*}
  \delta_0 \phi = \phi'(x) - \phi(x) = \phi'(x' - \delta x) - \phi(x) = - \delta x^\rho \pa_\rho \phi = \frac{i}{2} \omega_{\mu \nu} \tensor{\left[ J^{\mu \nu} \right]}{^\rho_\sigma} x^\sigma \pa_\rho \phi \equiv - \frac{i}{2} \omega_{\mu \nu} L^{\mu \nu} \phi
\end{equation*}
Recalling Eq. \ref{eq:1.31}, the generators can be expressed as:
\begin{equation}
  L^{\mu \nu} \defeq i \left( x^\mu \pa^\nu - x^\nu \pa^\mu \right)
  \label{eq:1.44}
\end{equation}
This is an infinite-dimensional representation, as it acts on the space of scalar fields. As $ p^\mu = i \pa^\mu $ (with signature $ (+,-,-,-) $), it is clear that $ L^i \equiv \frac{1}{2} \epsilon^{ijk} L^{jk} $ is the orbital angular momentum.

\paragraph{Vector fields}

A (contravariant) vector field transforms as:
\begin{equation}
  V'^\mu(x') = \tensor{\Lambda}{^\mu_\nu} V^\nu(x)
  \label{eq:1.45}
\end{equation}
A general vector field has a spin-0 and a spin-1 component, and it is acted on by the $ \ve{\left( \frac{1}{2},\frac{1}{2} \right)} $ representation.

\subsection{Poincaré group}










\newpage

\section{Classical equations of motion}

Consider a \textit{local field theory} of fields $ \{\phi_i(x)\}_{i \in \mathcal{I}} \equiv \phi(x) $, where $ x \in \R^{1,3} $ is a point in Minkoski spacetime. Its Lagrangian takes the form:
\begin{equation}
  L = \int \dd^3 x\, \mathcal{L}(\phi, \pa_\mu \phi)
  \label{eq:1.27xxxxxxxxxxxxxxx}
\end{equation}
where $ \mathcal{L} $ is the \textit{Lagrangian density} of the theory (often referred to simply as the Lagrangian), which depends only on a finite number of derivatives. The action is then:
\begin{equation}
  \mathcal{S} = \int \dd t\, L = \int \dd^4 x\, \mathcal{L}(\phi, \pa_\mu \phi)
  \label{eq:1.28xxxxxxxxxxxxxxxxx}
\end{equation}
The integration is carried on the whole space-time, with usual boundary conditions that all fields decrease sufficiently fast at infinity; this also allows to drop all boundary terms.

\begin{proposition}
  The \textit{stationary action principle} $ \delta \mathcal{S} = 0 $ determines the classical equations of motion:
  \begin{equation}
    \frac{\pa \mathcal{L}}{\pa \phi_i} - \pa_\mu \frac{\pa \mathcal{L}}{\pa (\pa_\mu \phi_i)} = 0
    \label{eq:1.29xxxxxxxxxxxxxxxx}
  \end{equation}
\end{proposition}
\begin{proof}
  Varying Eq. \ref{eq:1.28xxxxxxxxxxxxxxxxx}:
  \begin{equation*}
    \delta \mathcal{S} = \int \dd^4 x\, \sum_{i \in \mathcal{I}} \left[ \frac{\pa \mathcal{L}}{\pa \phi_i} \delta \phi_i + \frac{\pa \mathcal{L}}{\pa (\pa_\mu \phi)} \delta (\pa_\mu \phi_i) \right] = \int \dd^4 x\, \sum_{i \in \mathcal{I}} \left[ \frac{\pa \mathcal{L}}{\pa \phi_i} - \pa_\mu \frac{\pa \mathcal{L}}{\pa (\pa_\mu \phi)} \right] \delta \phi_i = 0
  \end{equation*}
\end{proof}

\begin{proposition}
  Two Lagrangians which differ by a total divergence $ \mathcal{L}' = \mathcal{L} + \pa_\mu K^\mu $ yield the same equations of motion.
\end{proposition}
\begin{proof}
  This is a consequence of Stokes theorem:
  \begin{equation*}
    \int_\Sigma \dd^4 x\, \pa_\mu K^\mu = \int_{\pa \Sigma} \dd A\, n_\mu K^\mu
  \end{equation*}
\end{proof}

From the Lagrangian, it is possible to define the conjugate momenta and the Hamiltonian density:
\begin{equation}
  \Pi_i(x) \defeq \frac{\pa \mathcal{L}}{\pa (\pa_0 \phi_i)}
  \label{eq:1.30xxxxxxxxxxxxxxxxxxxxxx}
\end{equation}
\begin{equation}
  \mathcal{H} = \sum_{i \in \mathcal{I}} \Pi_i(x) \pa_0 \phi(x) - \mathcal{L}
  \label{eq:1.31xxxxxxxxxxxxxxxxxxxxxxxx}
\end{equation}












\chapter{Quantization of Free Fields}
\selectlanguage{english}

\section{Scalar fields}

As for the quantization of a classical system in Quantum Mechanics, the quantization of a scalar field theory is performed promoting $ \phi(t,\ve{x}) $ and $ \Pi(t,\ve{x}) $ to hermitian operators in the Heisenberg picture and imposing the canonical equal-time commutation relation:
\begin{equation}
  [\phi(t,\ve{x}) , \Pi(t,\ve{y})] = i \delta^{(3)}(\ve{x} - \ve{y})
  \label{eq:2.1}
\end{equation}
while of course $ [\phi(t,\ve{x}) , \phi(t,\ve{y})] = [\Pi(t,\ve{x}) , \Pi(t,\ve{y})] = 0 $.

\subsection{Real scalar fields}

A real scalar field is promoted to a real hermitian operator. In particular, by Eq. \ref{eq:1.70}:
\begin{equation}
  \phi(x) = \int \frac{\dd^3p}{(2\pi)^3 \sqrt{2E_\ve{p}}} \left[ a_\ve{p} e^{-i p_\mu x^\mu} + a_\ve{p}^\dagger e^{i p_\mu x^\mu} \right]_{p^0 = E_\ve{p}}
  \label{eq:2.2}
\end{equation}
In terms of creation and annihilation operators, the commutator Eq. \ref{eq:2.1} reads:
\begin{equation}
  [a_\ve{p} , a_\ve{p}^\dagger] = (2\pi)^3 \delta^{(3)}(\ve{p} - \ve{q})
  \label{eq:2.3}
\end{equation}
while $ [a_\ve{p} , a_\ve{q}] = [a_\ve{p}^\dagger , a_\ve{q}^\dagger] = 0 $. These can be regarded as the creation and annihilation operators of a collection of harmonic oscillators, one for each value of the momentum $ \ve{p} $: the \textit{Fock space} of the real scalar field can thus be constructed analogously to the Hilbert space of the harmonic oscillator.\\
Defining the \textit{vacuum state} $ \ket{0} : a_\ve{p} \ket{0} = 0 \,\,\forall \ve{p} $, suitably normalized as $ \braket{0 | 0} = 1 $, the generic state of the Fock space is:
\begin{equation}
  \ket{\ve{p}_1 , \dots , \ve{p}_n} = \sqrt{2E_{\ve{p}_1}} \dots \sqrt{2E_{\ve{p}_n}}\, a_{\ve{p}_1}^\dagger \dots a_{\ve{p}_n}^\dagger \ket{0}
  \label{eq:2.4}
\end{equation}

\begin{proposition}{Normalization}{}
  For one-particle states:
  \begin{equation}
    \braket{\ve{p}_1 | \ve{p}_2} = 2E_{\ve{p}_1} (2\pi)^3 \delta^{(3)}(\ve{p}_1 - \ve{p}_2)
    \label{eq:2.5}
  \end{equation}

  \begin{proof}
    By Eq. \ref{eq:2.3}:
    \begin{equation*}
      \braket{\ve{p}_1 | \ve{p}_2} = \sqrt{2E_{\ve{p}_1}} \sqrt{2E_{\ve{p}_2}} \braket{0 | a_{\ve{p}_1} a_{\ve{p}_2}^\dagger | 0}
      = \sqrt{2E_{\ve{p}_1}} \sqrt{2E_{\ve{p}_2}} \braket{0 | [a_{\ve{p}_1} , a_{\ve{p}_2}^\dagger] | 0} = 2E_{\ve{p}_1} (2\pi)^3 \delta^{(3)}(\ve{p}_1 - \ve{p}_2)
    \end{equation*}
  \end{proof}
\end{proposition}

\begin{lemma}{}{}
  The combination $ E_\ve{p} \delta^{(3)}(\ve{p} - \ve{q}) $ is Lorentz invariant.

  \tcblower

  \begin{proof}
    Recall that:
    \begin{equation}
      \delta(f(x) - f(x_0)) = \frac{1}{\abs{f'(x_0)}} \delta(x - x_0)
      \label{eq:2.6}
    \end{equation}
    A Lorentz boost along $ \ve{e}_i $ yields $ p'_i = \gamma (p_i + \beta E) $ and $ E' = \gamma (E + \beta p_i) $, so:
    \begin{equation*}
        \delta^{(3)}(\ve{p} - \ve{q})
        = \delta^{(3)}(\ve{p}' - \ve{q}') \frac{\dd p'_i}{\dd p_i}
        = \delta^{(3)}(\ve{p}' - \ve{q}') \gamma \left( 1 + \beta \frac{\dd E}{\dd p_i} \right)
    \end{equation*}
    Note that $ p^2 = E^2 - \ve{p}^2 $ is a Lorentz invariant, thus $ E \dd E = p_i \dd p_i $, so:
    \begin{equation*}
      \delta^{(3)}(\ve{p} - \ve{q})
      = \delta^{(3)}(\ve{p}' - \ve{q}') \frac{\gamma (E + \beta p_i)}{E}
      = \delta^{(3)}(\ve{p}' - \ve{q}') \frac{E'}{E}
    \end{equation*}
  \end{proof}
\end{lemma}

This explains the choice of normalization.

\begin{proposition}{KG Hamiltonian}{}
  The Hamiltonian of a real scalar field can be written as:
  \begin{equation}
    H = \int \frac{\dd^3 p}{(2\pi)^3} E_\ve{p} \left( a_\ve{p}^\dagger a_\ve{p} + \frac{1}{2} [a_\ve{p} , a_\ve{p}^\dagger] \right)
    \label{eq:2.7}
  \end{equation}

  \tcblower

  \begin{proof}
    First of all, from Eq. \ref{eq:2.2}:
    \begin{equation*}
      \begin{split}
        \Pi(t,\ve{x}) = \pa_0 \phi(t,\ve{x})
        &= -i \int \frac{\dd^3 p}{(2\pi)^3} \sqrt{\frac{E_\ve{p}}{2}} \left[ a_\ve{p} e^{-i p_\mu x^\mu} - a_\ve{p}^\dagger e^{i p_\mu x^\mu} \right]_{p^0 = E_\ve{p}} \\
        &= -i \int \frac{\dd^3 p}{(2\pi)^3} \sqrt{\frac{E_\ve{p}}{2}} \left[ a_\ve{p} e^{-i E_\ve{p} t} - a_{-\ve{p}}^\dagger e^{i E_\ve{p} t} \right] e^{i \ve{p} \cdot \ve{x}}
      \end{split}
    \end{equation*}
    \begin{equation*}
      \bs{\nabla} \phi(t,\ve{x}) = i \int \frac{\dd^3 p}{(2\pi)^3} \frac{\ve{p}}{\sqrt{2E_\ve{p}}} \left[ a_\ve{p} e^{-i E_\ve{p} t} + a_{-\ve{p}}^\dagger e^{i E_\ve{p} t} \right] e^{i \ve{p} \cdot \ve{x}}
    \end{equation*}
    Inserting these expressions in Eq. \ref{eq:1.71}:
    \begin{equation*}
      \begin{split}
        \mathcal{H}
        &= \frac{1}{2} \int \frac{\dd^3 p}{(2\pi)^3} \int \frac{\dd^3 q}{(2\pi)^3} \bigg\{ - \frac{\sqrt{E_\ve{p} E_\ve{q}}}{2} \left[ a_\ve{p} e^{-i E_\ve{p} t} - a_{-\ve{p}}^\dagger e^{i E_\ve{p} t} \right] \left[ a_\ve{q} e^{-i E_\ve{q} t} - a_{-\ve{q}}^\dagger e^{i E_{\ve{q}} t} \right] + \\
        & \qquad + \frac{- \ve{p} \cdot \ve{q} + m^2}{2\sqrt{E_\ve{p} E_\ve{q}}} \left[ a_\ve{p} e^{-i E_\ve{p} t} + a_{-\ve{p}}^\dagger e^{i E_\ve{p} t} \right] \left[ a_\ve{q} e^{-i E_\ve{q} t} + a_{-\ve{q}}^\dagger e^{i E_\ve{q} t} \right] \bigg\} e^{i (\ve{p} + \ve{q}) \cdot \ve{x}}
      \end{split}
    \end{equation*}
    Recall the identity:
    \begin{equation}
      \int \frac{\dd^3x}{(2\pi)^3} e^{i (\ve{p} - \ve{q}) \cdot \ve{x}} = \delta^{(3)}(\ve{p} - \ve{q})
      \label{eq:2.8}
    \end{equation}
    Then (using $ E_{-\ve{p}} = E_\ve{p} $):
    \begin{equation*}
      \begin{split}
        H
        &= \int \dd^3x\, \mathcal{H} \\
        &= \frac{1}{2} \int \frac{\dd^3 p}{(2\pi)^3} \int \dd^3 q \bigg\{ - \frac{\sqrt{E_\ve{p} E_\ve{q}}}{2} \left[ a_\ve{p} e^{-i E_\ve{p} t} - a_{-\ve{p}}^\dagger e^{i E_\ve{p} t} \right] \left[ a_\ve{q} e^{-i E_\ve{q} t} - a_{-\ve{q}}^\dagger e^{i E_\ve{q} t} \right] + \\
        & \qquad + \frac{- \ve{p} \cdot \ve{q} + m^2}{2\sqrt{E_\ve{p} E_\ve{q}}} \left[ a_\ve{p} e^{-i E_\ve{p} t} + a_{-\ve{p}}^\dagger e^{i E_\ve{p} t} \right] \left[ a_\ve{q} e^{-i E_\ve{q} t} + a_{-\ve{q}}^\dagger e^{i E_\ve{q} t} \right] \bigg\} \delta^{(3)}(\ve{p} + \ve{q}) \\
        &= \frac{1}{2} \int \frac{\dd^3 p}{(2\pi)^3} \bigg\{ - \frac{E_\ve{p}}{2} \left[ a_\ve{p} e^{-i E_\ve{p} t} - a_{-\ve{p}}^\dagger e^{i E_\ve{p} t} \right] \left[ a_{-\ve{p}} e^{-i E_\ve{p} t} - a_\ve{p}^\dagger e^{i E_\ve{p} t} \right] + \\
        & \qquad + \frac{\ve{p}^2 + m^2}{2E_\ve{p}} \left[ a_\ve{p} e^{-i E_\ve{p} t} + a_{-\ve{p}}^\dagger e^{i E_\ve{p} t} \right] \left[ a_{-\ve{p}} e^{-i E_\ve{p} t} + a_\ve{p}^\dagger e^{i E_\ve{p} t} \right] \bigg\} \\
        &= \frac{1}{2} \int \frac{\dd^3p}{(2\pi)^3} \frac{E_\ve{p}}{2} \bigg\{ - \left[ a_\ve{p} a_{-\ve{p}} e^{-2i E_\ve{p} t} - a_{-\ve{p}}^\dagger a_{-\ve{p}} - a_\ve{p} a_\ve{p}^\dagger + a_{-\ve{p}}^\dagger a_\ve{p}^\dagger e^{2i E_\ve{p} t} \right] \\
        & \qquad + \left[ a_\ve{p} a_{-\ve{p}} e^{-2i E_\ve{p} t} + a_{-\ve{p}}^\dagger a_{-\ve{p}} + a_\ve{p} a_\ve{p}^\dagger + a_{-\ve{p}}^\dagger a_\ve{p}^\dagger e^{2i E_\ve{p} t} \right] \bigg\} \\
        &= \frac{1}{2} \int \frac{\dd^3p}{(2\pi)^3} E_\ve{p} \left[ a_{-\ve{p}}^\dagger a_{-\ve{p}} + a_\ve{p} a_\ve{p}^\dagger \right] = \frac{1}{2} \int \frac{\dd^3p}{(2\pi)^3} E_\ve{p} \left( a_\ve{p}^\dagger a_\ve{p} + a_\ve{p} a_\ve{p}^\dagger \right)
      \end{split}
    \end{equation*}
  \end{proof}
\end{proposition}

The second term in the Hamiltonian Eq. \ref{eq:2.7} is the sum of the zero-point energy of all oscillators and is proportional to $ (2\pi)^3 \delta^{(3)}(0) \rightarrow V $, thus:
\begin{equation*}
  E_\text{vac} = \frac{V}{2} \int \frac{\dd^3p}{(2\pi)^3} E_\ve{p}
\end{equation*}
This energy shows two divergences: the one coming from the infinite-volume limit (i.e. small momentum), regularized introducing an \textit{infrared cutoff} in the form of a finite volume, and the one from the ultra-relativistic limit (i.e. large momentum), regularized introducing an \textit{ultraviolet cutoff} in the form of a maximum momentum $ \Lambda $. These divergences are retained in the expression for $ E_\text{vac} $, as $ E_\text{vac} \sim V $ and $ E_\text{vac} \sim \Lambda^4 $, but can be ignored (when ignoring gravity) since experiments are only sensitive to energy differences.\\
Discarding the zero-point energy, the Hamiltonian becomes:
\begin{equation}
  H = \int \frac{\dd^3p}{(2\pi)^3} E_\ve{p} a_\ve{p}^\dagger a_\ve{p} \equiv \mathcal{N} \frac{1}{2} \int \frac{\dd^3p}{(2\pi)^3} E_\ve{p} \left( a_\ve{p}^\dagger a_\ve{p} + a_\ve{p} a_\ve{p}^\dagger \right)
  \label{eq:2.9}
\end{equation}
where the \textit{normal ordering operator} $ \mathcal{N} $ was introduced, which acts by moving all creation operators to the left and all annihilation operators to the right (ex.: $ \mathcal{N} a_\ve{p} a_\ve{p}^\dagger = a_\ve{p}^\dagger a_\ve{p} $). It is now straightforward to compute the energy of a generic state in the Fock space, as $ a_\ve{p}^\dagger a_\ve{p} $ is just a number operator:
\begin{equation}
  H \ket{\ve{p}_1 , \dots , \ve{p}_n} = \left( E_{\ve{p}_1} + \dots + E_{\ve{p}_n} \right) \ket{\ve{p}_1 , \dots , \ve{p}_n}
  \label{eq:2.10}
\end{equation}
Computing the spatial momentum from Eq. \ref{eq:1.67} as $ P^i = \mathcal{N} \int \dd^3x\, \theta^{0i} = \int \dd^3x\, \mathcal{N} \pa_0 \phi \pa^i \phi $:
\begin{equation}
  P^i = \int \frac{\dd^3p}{(2\pi)^3} p^i a_\ve{p}^\dagger a_\ve{p}
  \label{eq:2.11}
\end{equation}
Therefore, the state $ a_\ve{p}^\dagger \ket{0} $ can correctly be interpreted as a one-particle state with momentum $ \ve{p} $, mass $ m $ and energy $ E_\ve{p} = \sqrt{\ve{p}^2 + m^2} $. The generic state in the Fock space is a multiparticle state, with total energy and momentum the sum of the individual energies and momenta.\\
Finally, note that creation operators commute between themselves, hence multiparticle states are symmetric under exchange of pairs of particles, i.e. they obey the Bose-Einstein statistics: this agrees with the fact that quanta of a scalar field have no intrinsic spin. i.e. are spin-0 particles.

\subsection{Complex scalar fields}

When considering a complex scalar field, Eq. \ref{eq:1.78} becomes:
\begin{equation}
  \phi(x) = \int \frac{\dd^3p}{(2\pi)^3 \sqrt{2E_\ve{p}}} \left[ a_\ve{p} e^{-i p_\mu x^\mu} + b_\ve{p}^\dagger e^{i p_\mu x^\mu} \right]_{p^0 = E_\ve{p}}
  \label{eq:2.12}
\end{equation}
\begin{equation}
  \phi^\dagger(x) = \int \frac{\dd^3p}{(2\pi)^3 \sqrt{2E_\ve{p}}} \left[ a_\ve{p}^\dagger e^{i p_\mu x^\mu} + b_\ve{p} e^{-i p_\mu x^\mu} \right]_{p^0 = E_\ve{p}}
  \label{eq:2.13}
\end{equation}
Now there are two independent sets of creation/annihilation operators, which obey the canonical commutation relation:
\begin{equation}
  [a_\ve{p} , a_\ve{q}^\dagger] = [b_\ve{p} , b_\ve{q}^\dagger] = (2\pi)^3 \delta^{(3)}(\ve{p} - \ve{q})
  \label{eq:2.14}
\end{equation}
with all other commutators vanishing. The Fock space is constructed by defining a vacuum state $ \ket{0} : a_\ve{p} \ket{0} = b_\ve{p} \ket{0} = 0 $ and then acting repeatedly with both creation operators. With normal ordering, one finds:
\begin{equation}
  H = \int \frac{\dd^3p}{(2\pi)^3} E_\ve{p} \left( a_\ve{p}^\dagger a_\ve{p} + b_\ve{p}^\dagger b_\ve{p} \right)
  \label{eq:2.15}
\end{equation}
\begin{equation}
  P^i = \int \frac{\dd^3p}{(2\pi)^3} p^i \left( a_\ve{p}^\dagger a_\ve{p} + b_\ve{p}^\dagger b_\ve{p} \right)
  \label{eq:2.16}
\end{equation}
The quanta of a complex scalar field are given by two different species of particles with the same mass.

\begin{proposition}{$ \Un{1} $ charge}{}
  The $ \Un{1} $ charge of the quantized complex scalar field is:
  \begin{equation}
    Q_{\Un{1}} = \int \frac{\dd^3p}{(2\pi)^3} \left( a_\ve{p}^\dagger a_\ve{p} - b_\ve{p}^\dagger b_\ve{p} \right)
    \label{eq:2.17}
  \end{equation}

  \tcblower

  \begin{proof}
    By Eq. \ref{eq:1.79}:
    \begin{equation*}
      \begin{split}
        Q_{\Un{1}}
        &= i \int \dd^3x\, \phi^\dagger \smlra{\pa_0} \phi = i \int \dd^3x \int \frac{\dd^3q}{(2\pi)^3 \sqrt{2E_\ve{q}}} \int \frac{\dd^3p}{(2\pi)^3 \sqrt{2E_\ve{p}}} \,\times \\
        & \quad \times \bigg\{ \left[ a_\ve{q}^\dagger e^{i q_\mu x^\mu} + b_\ve{q} e^{-i q_\mu x^\mu} \right] \pa_0 \left( a_\ve{p} e^{-i p_\mu x^\mu} + b_\ve{p}^\dagger e^{i p_\mu x^\mu} \right) + \\
        & \quad - \pa_0 \left( a_\ve{q}^\dagger e^{i q_\mu x^\mu} + b_\ve{q} e^{-i q_\mu x^\mu} \right) \left[ a_\ve{p} e^{-i p_\mu x^\mu} + b_\ve{p}^\dagger e^{i p_\mu x^\mu} \right] \bigg\}
      \end{split}
    \end{equation*}
    \begin{equation*}
      \begin{split}
        Q_{\Un{1}}
        &= \int \dd^3x \int \frac{\dd^3q}{(2\pi)^3 \sqrt{2E_\ve{q}}} \int \frac{\dd^3p}{(2\pi)^3 \sqrt{2E_\ve{p}}} \,\times \\
        & \quad \times \bigg\{ E_\ve{p} \left[ a_\ve{q}^\dagger e^{i q_\mu x^\mu} + b_\ve{q} e^{-i q_\mu x^\mu} \right] \left[ a_\ve{p} e^{-i p_\mu x^\mu} - b_\ve{p}^\dagger e^{i p_\mu x^\mu} \right] + \\
        & \quad + E_\ve{q} \left[ a_\ve{q}^\dagger e^{i q_\mu x^\mu} - b_\ve{q} e^{-i q_\mu x^\mu} \right] \left[ a_\ve{p} e^{-i p_\mu x^\mu} + b_\ve{p}^\dagger e^{i p_\mu x^\mu} \right] \bigg\} \\
        &= \int \dd^3x \int \frac{\dd^3q}{(2\pi)^3 \sqrt{2E_\ve{q}}} \int \frac{\dd^3p}{(2\pi)^3 \sqrt{2E_\ve{p}}} \,\times \\
        & \quad \times \bigg\{ E_\ve{p} \left[ a_\ve{q}^\dagger e^{i E_\ve{q} t} + b_{-\ve{q}} e^{-i E_\ve{q} t} \right] \left[ a_\ve{p} e^{-i E_\ve{p} t} - b_{-\ve{p}}^\dagger e^{i E_\ve{p} t} \right] + \\
        & \quad + E_\ve{q} \left[ a_\ve{q}^\dagger e^{i E_\ve{q} t} - b_{-\ve{q}} e^{-i E_\ve{q} t} \right] \left[ a_\ve{p} e^{-i E_\ve{p} t} + b_{-\ve{p}}^\dagger e^{i E_\ve{p} t} \right] \bigg\} e^{i (\ve{p} - \ve{q}) \cdot \ve{x}} \\
        &= \frac{1}{2} \int \frac{\dd^3p}{(2\pi)^3} \bigg\{ \left[ a_\ve{p}^\dagger e^{i E_\ve{p} t} + b_{-\ve{p}} e^{-i E_\ve{p} t} \right] \left[ a_\ve{p} e^{-i E_\ve{p} t} - b_{-\ve{p}}^\dagger e^{i E_\ve{p} t} \right] + \\
        & \quad + \left[ a_\ve{p}^\dagger e^{i E_\ve{p} t} - b_{-\ve{p}} e^{-i E_\ve{p} t} \right] \left[ a_\ve{p} e^{-i E_\ve{p} t} + b_{-\ve{p}}^\dagger e^{i E_\ve{p} t} \right] \bigg\} \\
        &= \int \frac{\dd^3p}{(2\pi)^3} \left[ a_\ve{p}^\dagger a_\ve{p} - b_{-\ve{p}} b_{-\ve{p}}^\dagger \right] = \int \frac{\dd^3p}{(2\pi)^3} \left( a_\ve{p}^\dagger a_\ve{p} - b_\ve{p} b_\ve{p}^\dagger \right)
      \end{split}
    \end{equation*}
    Applying normal ordering yields the thesis.
  \end{proof}
\end{proposition}

While normal ordering was justified when considering the Hamiltonian on the grounds that the vacuum energy is unobservable, a charged vacuum would have observable effects; however when promoting $ \phi $ to a quantum operator, the expression $ \phi^\dagger \smlra{\pa_0} \phi $ presents an ordering ambiguity (ex.: $ \phi^\dagger (\pa_0 \phi) $ or $ (\pa_0 \phi) \phi^\dagger $), which is removed requiring the charge of the vacuum to vanish.\\
Being $ a_\ve{p}^\dagger a_\ve{p} $ and $ b_\ve{p}^\dagger b_\ve{p} $ number operators, the $ \Un{1} $ charge is equal to the number of quanta created by $ a_\ve{p}^\dagger $ minus the number of quanta created by $ b_\ve{p}^\dagger $, integrated over all momenta: in particular, $ a_\ve{p}^\dagger \ket{0} $ and $ b_\ve{p}^\dagger \ket{0} $ are both spin-zero particles of mass $ m $ and momentum $ \ve{p} $, but they respectively have charges $ Q_{\Un{1}} = +1 $ and $ Q_{\Un{1}} = -1 $. This allows to properly interpret the negative-energy solutions of the KG equations: they are positive-energy particles with opposite $ \Un{1} $ charge and are called \textit{antiparticles}.\\
For a real scalar field, the reality condition reads $ a_\ve{p} = b_\ve{p} $, thus it describes a field whose particle is its own antiparticle, and it is symmetric under any $ \Un{1} $ symmetry.

\section{Spinor fields}

A principle of QFT is the \textit{spin-statistic theorem}: integer-spin fields are to be quantized imposing equal-time commutation relations, while half-integer-spin with equal-time anticommutation relations.

\subsection{Dirac fields}

From the Dirac Lagrangian Eq. \ref{eq:1.87}, the conjugate momentum to the Dirac field $ \Psi $ is computed as:
\begin{equation}
  \Pi_\Psi = i \bar{\Psi} \gamma^0 = i \Psi^\dagger
  \label{eq:2.18}
\end{equation}
Imposing the canonical anticommutation relation, according to the spin-statistic theorem:
\begin{equation}
  \{\Psi_a(t,\ve{x}) , \Psi_b^\dagger(t,\ve{y})\} = \delta^{(3)}(\ve{x} - \ve{y}) \delta_{ab}
  \label{eq:2.19}
\end{equation}
where $ a,b = 1,2,3,4 $ are Dirac indices. Expanding the free Dirac field in plane waves:
\begin{equation}
  \Psi(x) = \int \frac{\dd^3p}{(2\pi)^3 \sqrt{2E_\ve{p}}} \sum_{s = 1,2} \left[ a_{\ve{p},s} u^s(p) e^{-i p_\mu x^\mu} + b_{\ve{p},s}^\dagger v^s(p) e^{i p_\mu x^\mu} \right]_{p^0 = E_\ve{p}}
  \label{eq:2.20}
\end{equation}
\begin{equation}
  \bar{\Psi}(x) = \int \frac{\dd^3p}{(2\pi)^3 \sqrt{2E_\ve{p}}} \sum_{s = 1,2} \left[ a_{\ve{p},s}^\dagger \bar{u}^s(p) e^{i p_\mu x^\mu} + b_{\ve{p},s} \bar{v}^s(p) e^{-i p_\mu x^\mu} \right]_{p^0 = E_\ve{p}}
  \label{eq:2.21}
\end{equation}
where the spinor wave functions $ u^s(p) , v^s(p) $ are given by Eqq. \ref{eq:1.90}-\ref{eq:1.91}. Translating Eq. \ref{eq:2.19} in terms of creation/annihilation operators:
\begin{equation}
  \{a_\ve{p},s , a_{\ve{q},r}^\dagger\} = \{b_{\ve{p},s} , b_{\ve{q},r}^\dagger\} = (2\pi)^3 \delta^{(3)}(\ve{p} - \ve{q}) \delta_{sr}
  \label{eq:2.22}
\end{equation}
The Fock space is again constructed defining a vacuum state $ \ket{0} : a_{\ve{p},s} \ket{0} = b_{\ve{p},s} \ket{0} = 0 $ and then acting repeatedly on it with $ a_{\ve{p}.s}^\dagger , b_{\ve{p},s}^\dagger $. As these operators anticommute, states in this Fock space are antisymmetric under the exchange of particles, therefore spint-$ \frac{1}{2} $ obey the Fermi-Dirac statistics (as of the spin-statistic theorem).

\begin{proposition}{Dirac Hamiltonian}{}
  The Hamiltoniana for a Dirac field $ \Psi $ is:
  \begin{equation}
    H = \int \frac{\dd^3p}{(2\pi)^3} \sum_{s = 1,2} E_\ve{p} \left[ a_{\ve{p},s}^\dagger a_{\ve{p},s} + b_{\ve{p},s}^\dagger b_{\ve{p},s} \right]
    \label{eq:2.23}
  \end{equation}

  \tcblower

  \begin{proof}
    By Eqq. \ref{eq:2.18}, the Hamiltonian density is:
    \begin{equation*}
      \mathcal{H} = \Pi_\Psi \pa_0 \Psi - \mathcal{L}_\text{D} = i \Psi^\dagger \pa_0 \Psi - \bar{\Psi} \left( i \gamma^0 \pa_0 + i \gamma^i \pa_i - m \right) \Psi = \bar{\Psi} \left( -i \gamma^i \pa_i + m \right) \Psi
    \end{equation*}
    Therefore, using Eqq. \ref{eq:2.20}-\ref{eq:2.21} and Eq. \ref{eq:2.8}:
    \begin{equation*}
      \begin{split}
        H
        &= \int \dd^3x\, \bar{\Psi} \left( -i \gamma^i \pa_i + m \right) \Psi = \int \dd^3x\, \bar{\Psi} \left( -i \bs{\gamma} \cdot \bs{\nabla} + m \right) \Psi \\
        &= \mathcal{N} \int \dd^3x \int \frac{\dd^3p}{(2\pi)^3 \sqrt{2E_\ve{p}}} \int \frac{\dd^3q}{(2\pi)^3 \sqrt{2E_\ve{q}}} \sum_{s = 1,2} \sum_{r = 1,2} \left[ a_{\ve{p},s}^\dagger \bar{u}^s(p) e^{ip_\mu x^\mu} + b_{\ve{p},s} \bar{v}^s(p) e^{-ip_\mu x^\mu} \right] \times \\
        & \qquad \qquad \qquad \qquad \qquad \qquad \qquad \qquad \times \left( -i \bs{\gamma} \cdot \bs{\nabla} + m \right) \left[ a_{\ve{q},r} u^r(q) e^{-iq_\mu x^\mu} + b_{\ve{q},r}^\dagger v^r(q) e^{iq_\mu x^\mu} \right] \\
        &= \mathcal{N} \int \dd^3x \int \frac{\dd^3p}{(2\pi)^3 \sqrt{2E_\ve{p}}} \int \frac{\dd^3q}{(2\pi)^3 \sqrt{2E_\ve{q}}} \sum_{s = 1,2} \sum_{r = 1,2} \left[ a_{\ve{p},s}^\dagger \bar{u}^s(p) e^{i p_\mu x^\mu} + b_{\ve{p},s} \bar{v}^s(p) e^{-i p_\mu x^\mu} \right] \times \\
        & \qquad \qquad \qquad \qquad \qquad \qquad \times \left[ \left( \bs{\gamma} \cdot \ve{q} + m \right) a_{\ve{q},r} u^r(q) e^{-i q_\mu x^\mu} + \left( - \bs{\gamma} \cdot \ve{q} + m \right) b_{\ve{q},r}^\dagger v^r(q) e^{i q_\mu x^\mu} \right]
      \end{split}
    \end{equation*}
    Using the Dirac equation in the form $ (\slashed{p} - m) u(p) = (\slashed{p} + m) v(p) = 0 $:
    \begin{equation*}
      \left( \bs{\gamma} \cdot \ve{q} + m \right) u(q) = \gamma^0 E_\ve{q} u(q)
      \qquad
      \left( -\bs{\gamma} \cdot \ve{q} + m \right) v(q) = - \gamma^0 E_\ve{q} v(q)
    \end{equation*}
    Therefore, omitting the constraint $ p^0 = E_\ve{p} $ in the spinors' arguments and using Eq. \ref{eq:2.8}:
    \begin{equation*}
      \begin{split}
        H
        &= \mathcal{N} \int \dd^3x \int \frac{\dd^3p}{(2\pi)^3 \sqrt{2E_\ve{p}}} \int \frac{\dd^3q}{(2\pi)^3 \sqrt{2E_\ve{q}}} \sum_{s = 1,2} \sum_{r = 1,2} \left[ a_{\ve{p},s}^\dagger \bar{u}^s(\ve{p}) e^{i E_\ve{p} t} + b_{-\ve{p},s} \bar{v}^s(-\ve{p}) e^{-i E_\ve{p} t} \right] \times \\
        & \qquad \qquad \qquad \qquad \qquad \qquad \qquad \qquad \times \gamma^0 E_\ve{q} \left[ a_{\ve{q},r} u^r(\ve{q}) e^{-i E_\ve{q} t} - b_{-\ve{q},r}^\dagger v^r(-\ve{q}) e^{i E_\ve{q} t} \right] e^{i (\ve{q} - \ve{p}) \cdot \ve{x}} \\
        &= \mathcal{N} \int \frac{\dd^3p}{2(2\pi)^3} \sum_{s = 1,2} \sum_{r = 1,2} \left[ a_{\ve{p},s}^\dagger \bar{u}^s(\ve{p}) e^{i E_\ve{p} t} + b_{-\ve{p},s} \bar{v}^s(-\ve{p}) e^{-i E_\ve{p} t} \right] \times \\
        & \qquad \qquad \qquad \qquad \qquad \qquad \qquad \qquad \times \gamma^0 \left[ a_{\ve{p},r} u^r(\ve{p}) e^{-i E_\ve{p} t} - b_{-\ve{p},s}^\dagger v^r(-\ve{p}) e^{i E_\ve{p} t} \right] \\
        &= \mathcal{N} \int \frac{\dd^3p}{2(2\pi)^3} \sum_{s = 1,2} \sum_{r = 1,2} \Big[ a_{\ve{p},s}^\dagger a_{\ve{p},r} u^{s\dagger}(\ve{p}) u^r(\ve{p}) + b_{-\ve{p},s} a_{\ve{p},r} v^{s\dagger}(-\ve{p}) u^r(\ve{p}) e^{-2i E_\ve{p} t} + \\
        & \qquad \qquad \qquad \qquad \qquad \quad - a_{\ve{p},s}^\dagger b_{-\ve{p},r}^\dagger u^{s\dagger}(\ve{p}) v^r(-\ve{p}) e^{2i E_\ve{p} t} - b_{-\ve{p},s} b_{-\ve{p},r}^\dagger v^{s\dagger}(-\ve{p}) v^r(-\ve{p}) \Big]
      \end{split}
    \end{equation*}

    \begin{lemma}{}{haml-lem}
      \begin{equation*}
        u^{s\dagger}(\ve{p}) v^r(-\ve{p}) = v^{s\dagger}(-\ve{p}) u^r(\ve{p}) = 0
      \end{equation*}
    \end{lemma}

    Using Lemma \ref{lemma:haml-lem}, Eq. \ref{eq:1.93} and the antisymmetry $ \mathcal{N} b_{\ve{p},s} b_{\ve{p},s}^\dagger = - b_{\ve{p},s}^\dagger b_{\ve{p},s} $:
    \begin{equation*}
      H = \int \frac{\dd^3p}{(2\pi)^3} \sum_{s = 1,2} E_\ve{p} \mathcal{N} \left[ a_{\ve{p},s}^\dagger a_{\ve{p},s} - b_{\ve{p},s} b_{\ve{p},s}^\dagger \right] = \int \frac{\dd^3p}{(2\pi)^3} \sum_{s = 1,2} E_\ve{p} \left[ a_{\ve{p},s}^\dagger a_{\ve{p},s} + b_{\ve{p},s}^\dagger b_{\ve{p},s} \right]
    \end{equation*}
  \end{proof}
\end{proposition}

It can be seen that, using anticommutators, the Hamiltonian and its interpretation are analogous to that of the complex scalar field: if commutators were used, instead, one would get a final $ - b_{\ve{p},s}^\dagger b_{\ve{p},s} $ term, which is problematic as it yields an energy unbounded from below.











\chapter{Quantum Electrodynamics}
\selectlanguage{english}

\section{Maxwell theory}

The electromagnetic field is described by a 4-vector $ A_\mu $, the \bctxt{gauge potential}. From this, the field strength tensor is defined as:
\begin{equation}
  F_{\mu \nu} \defeq \pa_\mu A_\nu - \pa_\nu A_\mu
\end{equation}
which is related to the electric and magnetic fields as $ F^{0i} = - E^i $ and $ F^{ij} = - \epsilon^{ijk} B^k $. The Lagrangian of the free electromagnetic field is:
\begin{equation}
  \lag_\text{M} = - \frac{1}{4} F_{\mu \nu} F^{\mu \nu}
  \label{eq:maxw-lag}
\end{equation}
The associated equations of motion are:
\begin{equation}
  \pa_\mu F^{\mu \nu} = 0
  \label{eq:maxw-1}
\end{equation}
Moreover, defining $ \tilde{F}^{\mu \nu} \equiv \frac{1}{2} \epsilon^{\mu \nu \rho \sigma} F_{\rho \sigma} $ (the Hodge dual), it is trivial to check that, by Schwarz lemma:
\begin{equation}
  \pa_\mu \tilde{F}^{\mu \nu} = 0
  \label{eq:maxw-2}
\end{equation}
\eeref{eq:maxw-1}{eq:maxw-2} are exactly Maxwell equations in the absence of sources: when written in terms of $ \ve{E} $ and $ \ve{B} $, \eref{eq:maxw-1} gives the equations for $ \bs{\nabla} \cdot \ve{E} $ and $ \bs{\nabla} \times \ve{B} $, while \eref{eq:maxw-2} those for $ \bs{\nabla} \times \ve{E} $ and $ \bs{\nabla} \cdot \ve{B} $.

\subsection{Gauge invariance}

A crucial local symmetry of the Maxwell Lagrangian is the symmetry under local gauge transformations like:
\begin{equation}
  A_\mu(x) \mapsto A_\mu(x) - \pa_\mu \alpha(x)
  \label{eq:qed-gauge-inv}
\end{equation}
with arbitrary $ \alpha \in \mathcal{C}^\infty(\R^{1,3}) $. Considering the free electromagnetic field, the global version of this transformation (that is, $ \alpha $ independent of $ x $) yields no conserved charge, as the associated Noether current vanishes identically.

\begin{theorem}{Radiation gauge}{}
  In the absence of sources, it is always possible to choose the \bcth{radiation gauge}:
  \begin{equation}
    A_0 = 0
    \qquad \qquad
    \bs{\nabla} \cdot \ve{A} = 0
    \label{eq:radiation-gauge}
  \end{equation}
\end{theorem}

\begin{proofbox}
  \begin{proof}
    Starting from a general gauge potential $ A_\mu $, the condition $ A_0 = 0 $ is achieved through:
    \begin{equation*}
      A_\mu \mapsto A_\mu - \pa_\mu \int_{t_0}^t \dd\tau\, A_0(\tau,\ve{x})
    \end{equation*}
    Then, $ A_0 = 0 $ will remain unchanged if another gauge transformation with $ \alpha(x) = \alpha(\ve{x}) $ is performed. Consider:
    \begin{equation*}
      \alpha(\ve{x}) = - \int_{\R^3} \frac{\dd^3y}{4\pi \abs{\ve{x} - \ve{y}}} \pa_i A^i(t,\ve{y})
    \end{equation*}
    which is independent of $ t $ since $ E^i = -\pa_0 A^i $, as $ A_0 = 0 $, so $ \pa_i E^i = 0 $ implies $ \pa_0 \pa_i A^i = 0 $. Recall the identity:
    \begin{equation}
      \lap_\ve{x} \frac{1}{\abs{\ve{x} - \ve{y}}} = - 4\pi \delta^{(3)}(\ve{x} - \ve{y})
    \end{equation}
    Thus:
    \begin{equation*}
      \bs{\nabla} \cdot \ve{A} \mapsto \bs{\nabla} \cdot \ve{A} - \lap_\ve{x} \alpha = \pa_i A^i(t,\ve{x}) - \pa_i A^i(t,\ve{x}) = 0
    \end{equation*}
  \end{proof}
\end{proofbox}

The radiation gauge clearly implies the \bctxt{Lorentz gauge}:
\begin{equation}
  \pa_\mu A^\mu = 0
\end{equation}
In this gauge, the equations of motions \eref{eq:maxw-1} become:
\begin{equation}
  \Box A^\mu = 0
  \label{eq:maxw-lor}
\end{equation}
which are massless KG equations for each component of the gauge potential. Plane-wave solutions take the form:
\begin{equation}
  A_\mu(x) = \epsilon_\mu(k) e^{-i k_\mu x^\mu} + c.c.
\end{equation}
where $ \epsilon_\mu(x) $ is the \bctxt{polarization vector}. Then, \eref{eq:maxw-lor} gives $ k^2 = 0 $, while the chosen radiation gauge implies $ \epsilon_0 = 0 $ and $ \bs{\epsilon} \cdot \ve{k} = 0 $: therefore, an electromagnetic wave has only two degrees of freedom, represented by a polarization vector $ \bs{\epsilon} $ perpendicular to the direction of propagation.\\
The advantage of the radiation gauge is that it exposes clearly the physical degrees of freedom of the electromagnetic field, while sacrificing explicit Lorentz covariance; on the other hand, the Lorentz gauge retains the explicit Lorentz covariance, at the cost of redundant degrees of freedom.

\subsection{Energy-momentum tensor}

By \eref{eq:en-mom-tensor}, writing \eref{eq:maxw-lag} explicitly in terms of $ A_\mu $, the energy-momentum tensor of the electromagnetic field is:
\begin{equation}
  \theta^{\mu \nu} = - F^{\mu \rho} \pa^\nu A_\rho + \frac{1}{4} \eta^{\mu \nu} F^2
\end{equation}
with $ F^2 \equiv F_{\mu \nu} F^{\mu \nu} $. To show the gauge-invariance of this tensor, recall \eref{eq:maxw-1}:
\begin{equation*}
  \theta^{\mu \nu} \mapsto \theta^{\mu \nu} + F^{\mu \rho} \pa^\nu \pa_\rho \alpha
  \qquad \Rightarrow \qquad
  P^\mu \mapsto P^\mu + \int \dd^3x\, \pa_\rho (F^{0 \rho} \pa^\mu \alpha) = P^\mu + \int \dd^3x\, \pa_i (F^{0i} \pa^\mu \alpha) = P^\mu
\end{equation*}
where the last term is a total spatial derivative, hence vanishing by divergence theorem provided that the field decreases sufficiently fast at infinity. To improve the energy-momentum tensor, add $ \pa_\rho (F^{\mu \rho} A^\nu) $, which is covariantly conserved by itself and whose $ \mu = 0 $ component is a total spatial derivative, so to obtain:
\begin{equation}
  T^{\mu \nu} = F^{\mu \rho} \tensor{F}{_\rho^\nu} + \frac{1}{4} \eta^{\mu \nu} F^2
\end{equation}
which is explicitly gauge-invariant and yields the usual expressions for the energy density $ T^{00} = \frac{1}{2} \left( \ve{E}^2 + \ve{B}^2 \right) $ and the momentum density $ T^{0i} = \left( \ve{E} \times \ve{B} \right)^i $.\\
In a general field theory, the observable quantities are the charges, not the currents: two Lagrangian densities which differ by a total 4-divergence are physically equivalent and give the same equations of motion, but the conserved currents obtained through Noether theorem are different, while the associated Noether charges are the same.

\subsection{Matter coupling}

In the presence of an external current $ j^\mu $, \eref{eq:maxw-2} is not modified, as it is a consequence of the definition of $ F^{\mu \nu} $ (assuming regular gauge fields), while \eref{eq:maxw-1} becomes:
\begin{equation}
  \pa_\mu F^{\mu \nu} = j^\nu
  \label{eq:maxw-3}
\end{equation}
By Schwarz lemma, this equation is consistent only if $ \pa_\mu j^\mu = 0 $. This can be understood in light of gauge invariance, considering the action:
\begin{equation}
  \act_\text{M} = - \int \dd^4x \left[ \frac{1}{4} F^2 + j^\mu A_\mu \right]
\end{equation}
A gauge transformation $ A_\mu \mapsto A_\mu - \pa_\mu \alpha $ implies $ \act_\text{M} \mapsto \act_\text{M} + \int \dd^4x\, j^\mu \pa_\mu \alpha $: integrating by parts, it is clear that $ \act_\text{M} $ is gauge invariant only if $ \pa_\mu j^\mu = 0 $.

\subsubsection{Dirac field}

The coupling of the electromagnetic field to the Dirac field is an example of the general procedure of writing a gauge-invariant action for a gauge theory. In particular, consider a theory with a global $ \Un{1} $ invariance, which is a symmetry of the free Dirac action by \eref{eq:dirac-u1-symm}. Now generalize to a local $ \Un{1} $ symmetry:
\begin{equation}
  \Psi(x) \mapsto e^{iq \alpha(x)} \Psi(x)
  \label{eq:qed-dir-inv}
\end{equation}
with $ q \in \R $. This no longer is a symmetry of the Dirac action, however it can be combined with \eref{eq:qed-gauge-inv} defining the \bctxt{covariant derivative}:
\begin{equation}
  \covder_\mu \Psi \defeq (\pa_\mu + i q A_\mu) \Psi
\end{equation}

\begin{proposition}[before upper = {\tcbtitle}]{}{}
  \begin{equation}
    \covder_\mu \Psi(x) \mapsto e^{i q \alpha(x)} \covder_\mu \Psi(x)
    \label{eq:qed-cov-der}
  \end{equation}
\end{proposition}

\begin{proofbox}
  \begin{proof}
      $ \covder_\mu \Psi \mapsto \left[ \pa_\mu + i q (A_\mu - \pa_\mu \alpha) \right] e^{i q \alpha} \Psi = e^{i q \alpha} \left[ i q \pa_\mu \alpha + \pa_\mu + i q A_\mu - i q \pa_\mu \alpha \right] \Psi = e^{i q \alpha} \covder_\mu \Psi $
  \end{proof}
\end{proofbox}

The Lagrangian with a local $ \Un{1} $ symmetry is found replacing $ \pa_\mu \mapsto \covder_\mu $ (\bctxt{minimal coupling}): the global symmetry is gauged to a local symmetry, resulting in a gauge theory with gauge field $ A_\mu $. \\
Applying this to \eref{eq:dirac-lagrangian}:
\begin{equation}
  \lag_\text{D} = \bar{\Psi} (i \slashed{\pa} - m) \Psi - q A_\mu \bar{\Psi} \gamma^\mu \Psi
\end{equation}
where $ j_\text{V}^\mu \defeq \bar{\Psi} \gamma^\mu \Psi $ is the Noether current associated to the global $ \Un{1} $ symmetry. The associated conserved charge then is:
\begin{equation}
  Q = \int \dd^3x\, \bar{\Psi} \gamma^0 \Psi = \int \dd^3x\, \Psi\dg \Psi
\end{equation}

\subsubsection{Complex scalar field}

A complex scalar field has a global $ \Un{1} $ symmetry $ \phi \mapsto e^{i q \alpha} \phi $, thus the covariant derivative is identical to \eref{eq:qed-cov-der} and the gauged Lagrangian reads (recall \eref{eq:compl-scalar-lag}):
\begin{equation}
  \lag = \pa_\mu \phi \pa^\mu \phi + i q A^\mu (\phi \pa_\mu \phi^* - \phi^* \pa_\mu \phi) + q^2 \abs{\phi}^2 A_\mu A^\mu - m^2 \phi^* \phi
\end{equation}
where $ j_\mu \defeq i \phi^* \overleftrightsmallarrow{\pa_\mu} \phi $ is the Noether current associated to the global $ \Un{1} $ symmetry.

\subsubsection{Higher order interaction terms}

Although a real scalar field cannot be coupled to the electromagnetic field through the minimal coupling (as the real condition imposes $ q = 0 $, i.e. a neutral field), interaction terms are possible via higher order terms, as $ \lag_\text{int} \sim \phi F_{\mu \nu} F^{\mu \nu} $ or $ \lag_\text{int} \sim \phi \epsilon_{\mu \nu \rho \sigma} F^{\mu \nu} F^{\rho \sigma} $. \\
The same is possible for the Dirac field too, for example with $ \lag_\text{int} \sim \bar{\Psi} \sigma^{\mu \nu} \Psi F^{\mu \nu} $: note, however, that these non-minimal couplings have dimensional coupling constants (with dimension of the inverse of a mass), which have a less fundamental significance than dimensionless coupling constant.

\begin{example}{Neutral pions}{}
  The neutral pion $ \pi^0 $ is described by a pseudoscalar field, thus its interaction with the electromagnetic field needs to be a pseudoscalar term, like $ \epsilon_{\mu \nu \rho \sigma} F^{\mu \nu} F^{\rho \sigma} $ (this gives a good phenomenological description), as opposed to parity-invariant terms like $ F_{\mu \nu} F^{\mu \nu} $.
\end{example}

\section{Quantization}

Due to gauge symmetry, the gauge field gives a redundant physical description, therefore the quantization procedure can be carried in two different ways: fixing the gauge, thus working with only physical degrees of freedom but at the cost of loosing explicit Lorentz invariance, or considering the whole $ A_\mu $, hence carrying spurious degrees of freedom.

\subsection{Quantization in the radiation gauge}

As of \eref{eq:radiation-gauge}, Maxwell equations \eref{eq:maxw-lor} read $ \Box \ve{A} = \ve{0} $, with general classical solution:
\begin{equation*}
  \ve{A}(x) = \int \frac{\dd^3p}{(2\pi)^3 \sqrt{2\omega_\ve{p}}} \sum_{\lambda = 1,2} \left[ \bs{\epsilon}(\ve{p},\lambda) a_{\ve{p},\lambda} e^{-i p_\mu x^\mu} + \bs{\epsilon}^*(\ve{p},\lambda)a^*_{\ve{p},\lambda} e^{i p_\mu x^\mu} \right]_{p^0 = \omega_\ve{p}}
\end{equation*}
Inserting this expression into $ \Box \ve{A} = \ve{0} $ results in the mass-shell condition $ p^2 = 0 $, i.e. $ \omega_\ve{p} = \abs{\ve{p}} $. On the other hand, the gauge condition $ \dive \ve{A} = 0 $ requires $ \bs{\epsilon} \cdot \ve{p} = 0 $: for each fixed $ \ve{p} $, this solution has two orthonormal solutions labelled by $ \lambda = 1,2 $, which describe the two physical degrees of freedom of the electromagnetic field.
The classical solution can be promoted to a hermitian operator as:
\begin{equation}
  \ve{A}(x) = \int \frac{\dd^3p}{(2\pi)^3 \sqrt{2\omega_\ve{p}}} \sum_{\lambda = 1,2} \left[ \bs{\epsilon}(\ve{p},\lambda) a_{\ve{p},\lambda} e^{-i p_\mu x^\mu} + \bs{\epsilon}^*(\ve{p},\lambda)a\dg_{\ve{p},\lambda} e^{i p_\mu x^\mu} \right]_{p^0 = \omega_\ve{p}}
\end{equation}
and imposing the canonical commutation relations:
\begin{equation}
  [a_{\ve{p},\lambda} , a_{\ve{q},\lambda'}\dg] = (2\pi)^3 \delta^{(3)}(\ve{p} - \ve{q}) \delta_{\lambda \lambda'}
\end{equation}
In terms of commutators of $ A^i $ and conjugate momenta, consider that $ \Pi_0 = 0 $ (as $ A_0 = 0 $) and\footnotemark:
\begin{equation*}
  \Pi^i = \frac{\delta}{\delta(\pa_0 A_i)} \left( -\frac{1}{4} F_{\mu \nu} F^{\mu \nu} \right) = \frac{\delta}{\delta(\pa_0 A_i)} \left( -\frac{1}{2} F_{0i} F^{0i} \right) = - F^{0i} = E^i
\end{equation*}
%
\footnotetext{Note that $ \Pi^i $ is the momentum conjugate to $ A_i $, while $ \Pi_i = - \Pi^i $ is the one conjugate to $ A^i $.}
%
\begin{lemma}[before upper = {\tcbtitle}]{}{}
  \begin{equation}
    \frac{1}{2} \sum_{\lambda = 1,2} \left[ \epsilon^i(\ve{k},\lambda) \epsilon^{*j}(\ve{k},\lambda) + \epsilon^{*i}(-\ve{k},\lambda) \epsilon^j(-\ve{k},\lambda) \right] \delta^{ij} - \frac{k^i k^j}{\ve{k}^2}
  \end{equation}
\end{lemma}

\begin{proofbox}
  \begin{proof}
    Trivially verified in a frame where $ \ve{k} = (0,0,k) $ choosing $ \bs{\epsilon}(\ve{k},1) = (1,0,0) $ and $ \bs{\epsilon}(\ve{k},2) = (0,1,0) $ and valid in any frame\footnote{In particular, the linear polarizations satisfy:
    \begin{equation}
      \sum_{\lambda = 1,2} \epsilon^i(\ve{k},\lambda) \epsilon^j(\ve{k},\lambda) = \delta^{ij} - \frac{k^i k^j}{\ve{k}^2}
    \end{equation}} is ensured as both sides transform as tensor under rotations.
  \end{proof}
\end{proofbox}

\begin{proposition}[before upper = {\tcbtitle}]{}{}
  \begin{equation}
    [A^i(t,\ve{x}) , E^j(t,\ve{y})] = - i \int \frac{\dd^3k}{(2\pi)^3} e^{i \ve{k} \cdot (\ve{x} - \ve{y})} \left( \delta^{ij} - \frac{k^i k^j}{\ve{k}^2} \right)
  \end{equation}
\end{proposition}

The r.h.s. of this equation is similar to a Dirac delta and it is called \bctxt{transverse Dirac delta}, defined in order to get $ [\dive \ve{A}(t,\ve{x}) , \ve{E}(t,\ve{y})] = \ve{0} $ (as $ \dive \ve{A} = 0 $), being the integrand proportional to:
\begin{equation*}
  k^i \left( \delta^{ij} - \frac{k^i k^j}{\ve{k}^2} \right) = k^j - k^j = 0
\end{equation*}

\paragraph{Fock space}

The standard construction of the Fock space proceeds defining the vacuum state $ a_{\ve{p},\lambda} \ket{0} = 0 $. The Hamiltonian and the linear momentum are then found as:
\begin{equation}
  H = \frac{1}{2} \normord \int \dd^3x\, \left[ \ve{E}^2 + \ve{B}^2 \right] = \int \frac{\dd^3k}{(2\pi)^3} \sum_{\lambda = 1,2} \omega_\ve{k} a_{\ve{k},\lambda}\dg a_{\ve{k},\lambda}
\end{equation}
\begin{equation}
  \ve{P} = \normord \int \dd^3x\, \ve{E} \times \ve{B} = \int \frac{\dd^3k}{(2\pi)^3} \sum_{\lambda = 1,2} \ve{k} a_{\ve{k},\lambda}\dg a_{\ve{k},\lambda}
\end{equation}

Therefore, $ a_{\ve{k},\lambda}\dg \ket{0} $ describes a massless particle with energy $ \omega_\ve{k} $ and momentum $ \ve{k} $. To study spin, it is necessary to switch to circular polarizations:
\begin{equation}
  a_{\ve{k},\pm}\dg \defeq \frac{1}{\sqrt{2}} \left( a_{\ve{k},1}\dg \pm i a_{\ve{k},2}\dg \right)
\end{equation}
where $ a_{\ve{k},1} , a_{\ve{k},2} $ are the linear polarizations $ \bs{\epsilon}(\ve{k},1) = (1,0,0) , \bs{\epsilon}(\ve{k},2) = (0,1,0) $. Linear polarizations are not helicity eigenstates, while circular polarizations are:
\begin{align*}
  S^3 a_{\ve{k},1}\dg \ket{0} = + i a_{\ve{k},2}\dg \ket{0} & \qquad & S^3 a_{\ve{k},+}\dg \ket{0} = + a_{\ve{k},+}\dg \ket{0} \\
  S^3 a_{\ve{k},2}\dg \ket{0} = - i a_{\ve{k},1}\dg \ket{0} & \qquad & S^3 a_{\ve{k},-}\dg \ket{0} = - a_{\ve{k},-}\dg \ket{0}
\end{align*}
In conclusion, $ a_{\ve{k},\pm}\dg \ket{0} $ describe massless particles with energy $ \omega_\ve{k} $, momentum $ \ve{k} $, spin $ 1 $ and helicity $ \pm 1 $: these are photons. \\
Defining angular momentum and boost generators too in terms of ladder operators, it is possible to show that Lorentz invariance is preserved by the quantization procedure, although not explicitly.

\paragraph{Discrete transformations}

It is possible to define parity and charge conjugation on photon states. The electric field is a true vector, while the magnetic field is a pseudovector, thus the gauge potential is a true vector: $ \parity \ve{A}(t,\ve{x}) = - \ve{A}(t,-\ve{x}) $. In terms of photon states:
\begin{equation}
  \parity \ket{\gamma ; \ve{k}, \ve{s}} = - \ket{\gamma ; -\ve{k}, \ve{s}}
\end{equation}
so the intrinsic parity of physical photon states is $ -1 $. \\
As the fermionic current changes sign under charge conjugation (\eref{eq:charge-conj-fermion-curr}), it is a symmetry of the QED Lagrangian if $ \chargec A^\mu \chargec = - A^\mu $, i.e. $ \chargec a_{\ve{k},\lambda}\dg \chargec = - a_{\ve{k},\lambda}\dg $. As $ \chargec \ket{0} = +\ket{0} $ and $ \chargec^2 = \id $ by definition, then $ \chargec a_{\ve{k},\lambda}\dg \ket{0} = \chargec a_{\ve{k},\lambda} \chargec \chargec \ket{0} = - a_{\ve{k},\lambda}\dg \ket{0} $, or:
\begin{equation}
  \chargec \ket{\gamma ; \ve{k} , \ve{s}} = - \ket{\gamma ; \ve{k} , \ve{s}}
\end{equation}

\subsection{Covariant quantization}

The Maxwell Lagrangian \eref{eq:maxw-lag} cannot be straightforwardly quantize, as $ \Pi^0 $ cannot be defined due to the absence of $ \pa_0 A_0 $ terms. The basic idea of the covariant quantization of the electromagnetic field, or Gupta-Bleuler quantization, is to start from a modified Lagrangian:
\begin{equation}
  \lag = - \frac{1}{4} F_{\mu \nu} F^{\mu \nu} - \frac{1}{2} (\pa_\mu A^\mu)^2
\end{equation}
Conjugate momenta are then found to be:
\begin{equation*}
  \Pi^\mu = \frac{\pa \lag}{\pa (\pa_0 A_\mu)}
  \qquad \Rightarrow \qquad
  \Pi^i = - F^{0i} = E^i
  \qquad
  \Pi^0 = - \pa_\mu A^\mu
\end{equation*}
Canonical commutation relations now take the form:
\begin{equation}
  [A^\mu(t,\ve{x}) , \Pi^\nu(t,\ve{y})] = i \eta^{\mu \nu} \delta^{(3)}(\ve{x} - \ve{y})
  \label{eq:qed-canon-comm-rel-1}
\end{equation}
The metric ensures Lorentz covariance. The equations of motion are $ \Box A^\mu = 0 $, thus the gauge field operators are:
\begin{equation}
  A_\mu(x) = \int \frac{\dd^3p}{(2\pi)^3 \sqrt{2\omega_\ve{p}}} \sum_{\lambda = 0}^{3} \left[ \epsilon_\mu(\ve{p},\lambda) a_{\ve{p},\lambda} e^{-i p_\mu x^\mu} + \epsilon_\mu^*(\ve{p},\lambda) a_{\ve{p},\lambda}\dg e^{i p_\mu x^\mu} \right]_{p^0 = \omega_\ve{p}}
\end{equation}
$ \Box A_\mu = 0 $ imposes $ p^2 = 0 $. Note that the modified Lagrangian is not gauge invariant, so there is no constraint on $ \epsilon^\mu $; in the frame $ p^\mu = (p,0,0,p) $ a convenient choice of basis is $ \epsilon^\mu(\ve{p},\lambda) = \delta^\mu_\lambda $, hence only $ \lambda = 1,2 $ satisfy $ \epsilon_\mu p^\mu = 0 $. \eref{eq:qed-canon-comm-rel-1} becomes:
\begin{equation}
  [a_{\ve{p},\lambda} , a_{\ve{p},\lambda'}\dg] = - (2\pi)^3 \delta^{(3)}(\ve{p} - \ve{q}) \eta_{\lambda \lambda'}
\end{equation}
Note that the commutator for $ \lambda = \lambda' = 0 $ has a negative sign, which means that the norm is not positive defined on this Fock space (rendering impossible its interpretation as a probability):
\begin{equation}
  \braket{\ve{p} , \lambda | \ve{p} , \lambda} = (2\omega_\ve{p}) \braket{0 | a_{\ve{p},\lambda} a_{\ve{p},\lambda}\dg | 0} = (2\omega_\ve{p}) \braket{0 | [a_{\ve{p},\lambda} , a_{\ve{p},\lambda}\dg] | 0} = - 2\omega_\ve{p} V \eta_{\lambda \lambda}
\end{equation}
(where $ V \equiv (2\pi)^3 \delta^{(3)}(\ve{0}) $) which is negative for $ \lambda = 0 $. However, the only physical states are those associated with transverse polarization vectors, thus these problematic states can be shown to be unphysical. \\
To recover the correct description of QED from the modified Lagrangian, it is necessary to restrict the Fock space; in particular, any two physical states must satisfy:
\begin{equation}
  \braket{\text{phys}' | \pa_\mu A^\mu | \text{phys}} = 0
  \label{eq:qed-phys-cond-1}
\end{equation}
Note that $ \pa_\mu A^\mu $ can be decomposed into its positive- and negative-frequency parts $ \pa_\mu A^\mu = (\pa_\mu A^\mu)^+ + (\pa_\mu A^\mu)^- $ as:
\begin{equation*}
  (\pa_\mu A^\mu)^+ \equiv -i \int \frac{\dd^3p}{(2\pi)^3 \sqrt{2\omega_\ve{p}}} \sum_{\lambda = 0}^{3} p_\mu \epsilon^\mu(\ve{p},\lambda) a_{\ve{p},\lambda} e^{-i p_\mu x^\mu} = {(\pa_\mu A^\mu)}\dg
\end{equation*}
Therefore, \eref{eq:qed-phys-cond-1} is equivalent to:
\begin{equation}
  (\pa_\mu A^\mu)^+ \ket{\text{phys}} = 0
\end{equation}
This is the definition of the physical subspace of the Fock space, as it preserves the linear structure of the physical Hilbert space. Now consider the most general superposition of polarization states with momentum $ \ve{k} $, i.e. $ \ket{\ve{k}} = \sum_{\lambda = 0}^{3} c_\lambda a_{\ve{k},\lambda}\dg \ket{0} $, and choose the frame $ k^\mu = (k,0,0,k) $: in this frame the physical-state condition reads $ c_0 + c_3 = 0 $. Thus, all transverse states $ \lambda = 1,2 $ are physical, while there is only one non-transverse state that remains:
\begin{equation*}
  \ket{\phi} \equiv (a_{\ve{k},0}\dg - a_{\ve{k},3}\dg) \ket{0}
\end{equation*}
The most general one-particle state of the physical subspace can then be written as $ \ket{\ve{k}_\text{T}} + c \ket{\phi} $. However:
\begin{equation*}
  \braket{\phi | \phi} = \braket{0 | (a_{\ve{k},0} - a_{\ve{k},3}) (a_{\ve{k},0}\dg - a_{\ve{k},3}\dg) | 0} = \braket{0 | [a_{\ve{k},0} , a_{\ve{k},0}\dg] + [a_{\ve{k},3} , a_{\ve{k},3}\dg] | 0} = 0
\end{equation*}
This means that $ \ket{\phi} $ is orthogonal to all physical states. Moreover, only transverse states contribute to the energy and to the momentum, as $ H , \ve{P} \sim - \eta^{\lambda \lambda'} a_{\ve{k},\lambda}\dg a_{\ve{k},\lambda'} $ and (as $ c_0 + c_3 = 0 $):
\begin{equation*}
  (a_{\ve{k},0} - a_{\ve{k},3}) \ket{\psi} = 0
  \quad \Rightarrow \quad
  \braket{\text{phys}' | -a_{\ve{k},0}\dg a_{\ve{k},0} + a_{\ve{k},3}\dg a_{\ve{k},3} | \text{phys}} = \braket{\text{phys}' | (-a_{\ve{k},0}\dg + a_{\ve{k},3}\dg) a_{\ve{k},3} | \text{phys}} = 0
\end{equation*}
These facts mean that $ \ket{\ve{k}_\text{T}} $ and $ \ket{\ve{k}_\text{T}} + c \ket{\phi} $ are physically indistinguishable: photons can then be identified as the equivalence classes with respect to the equivalence relation $ \ket{\psi} \sim \ket{\psi} + c \ket{\phi} $. \\
This procedure has eliminated the spurious degrees of freedom, thus showing the equivalence between covariant quantization and gauge quantization.












\end{document}
