\selectlanguage{english}

\section{Maxwell theory}

The electromagnetic field is described by a 4-vector $ A_\mu $, the \textit{gauge potential}. From this, the field strength tensor is defined as:
\begin{equation}
  F_{\mu \nu} \defeq \pa_\mu A_\nu - \pa_\nu A_\mu
\end{equation}
which is related to the electric and magnetic fields as $ F^{0i} = - E^i $ and $ F^{ij} = - \epsilon^{ijk} B^k $. The Lagrangian of the free electromagnetic field is:
\begin{equation}
  \lag_\text{M} = - \frac{1}{2} F_{\mu \nu} F^{\mu \nu}
  \label{eq:maxw-lag}
\end{equation}
The associated equations of motion are:
\begin{equation}
  \pa_\mu F^{\mu \nu} = 0
  \label{eq:maxw-1}
\end{equation}
Moreover, defining $ \tilde{F}^{\mu \nu} \equiv \frac{1}{2} \epsilon^{\mu \nu \rho \sigma} F_{\rho \sigma} $ (the Hodge dual), it is trivial to check that, by Schwarz lemma:
\begin{equation}
  \pa_\mu \tilde{F}^{\mu \nu} = 0
  \label{eq:maxw-2}
\end{equation}
\eeref{eq:maxw-1}{eq:maxw-2} are exactly Maxwell equations in the absence of sources: when written in terms of $ \ve{E} $ and $ \ve{B} $, \eref{eq:maxw-1} gives the equations for $ \bs{\nabla} \cdot \ve{E} $ and $ \bs{\nabla} \times \ve{B} $, while \eref{eq:maxw-2} those for $ \bs{\nabla} \times \ve{E} $ and $ \bs{\nabla} \cdot \ve{B} $.

\subsection{Gauge invariance}

A crucial local symmetry of the Maxwell Lagrangian is the symmetry under local gauge transformations like:
\begin{equation}
  A_\mu(x) \mapsto A_\mu(x) - \pa_\mu \alpha(x)
\end{equation}
with arbitrary $ \alpha \in \mathcal{C}^\infty(\R^{1,3}) $. Considering the free electromagnetic field, the global version of this transformation (that is, $ \alpha $ independent of $ x $) yields no conserved charge, as the associated Noether current vanishes identically.

\begin{theorem}{Radiation gauge}{}
  In the absence of sources, it is always possible to choose the \textit{radiation gauge}:
  \begin{equation}
    A_0 = 0
    \qquad \qquad
    \bs{\nabla} \cdot \ve{A} = 0
  \end{equation}
\end{theorem}

\begin{proofbox}
  \begin{proof}
    Starting from a general gauge potential $ A_\mu $, the condition $ A_0 = 0 $ is achieved through:
    \begin{equation*}
      A_\mu \mapsto A_\mu - \pa_\mu \int_{t_0}^t \dd\tau\, A_0(\tau,\ve{x})
    \end{equation*}
    Then, $ A_0 = 0 $ will remain unchanged if another gauge transformation with $ \alpha(x) = \alpha(\ve{x}) $ is performed. Consider:
    \begin{equation*}
      \alpha(\ve{x}) = - \int_{\R^3} \frac{\dd^3y}{4\pi \abs{\ve{x} - \ve{y}}} \pa_i A^i(t,\ve{y})
    \end{equation*}
    which is independent of $ t $ since $ E^i = -\pa_0 A^i $, as $ A_0 = 0 $, so $ \pa_i E^i = 0 $ implies $ \pa_0 \pa_i A^i = 0 $. Recall the identity:
    \begin{equation}
      \lap_\ve{x} \frac{1}{\abs{\ve{x} - \ve{y}}} = - 4\pi \delta^{(3)}(\ve{x} - \ve{y})
    \end{equation}
    Thus:
    \begin{equation*}
      \bs{\nabla} \cdot \ve{A} \mapsto \bs{\nabla} \cdot \ve{A} - \lap_\ve{x} \alpha = \pa_i A^i(t,\ve{x}) - \pa_i A^i(t,\ve{x}) = 0
    \end{equation*}
  \end{proof}
\end{proofbox}

The radiation gauge clearly implies the \textit{Lorentz gauge}:
\begin{equation}
  \pa_\mu A^\mu = 0
\end{equation}
In this gauge, the equations of motions \eref{eq:maxw-1} become:
\begin{equation}
  \Box A^\mu = 0
  \label{eq:maxw-lor}
\end{equation}
which are massless KG equations for each component of the gauge potential. Plane-wave solutions take the form:
\begin{equation}
  A_\mu(x) = \epsilon_\mu(k) e^{-i k_\mu x^\mu} + c.c.
\end{equation}
where $ \epsilon_\mu(x) $ is the \textit{polarization vector}. Then, \eref{eq:maxw-lor} gives $ k^2 = 0 $, while the chosen radiation gauge implies $ \epsilon_0 = 0 $ and $ \bs{\epsilon} \cdot \ve{k} = 0 $: therefore, an electromagnetic wave has only two degrees of freedom, represented by a polarization vector $ \bs{\epsilon} $ perpendicular to the direction of propagation.\\
The advantage of the radiation gauge is that it exposes clearly the physical degrees of freedom of the electromagnetic field, while sacrificing explicit Lorentz covariance; on the other hand, the Lorentz gauge retains the explicit Lorentz covariance, at the cost of redundant degrees of freedom.

\subsection{Energy-momentum tensor}

By \eref{eq:en-mom-tensor}, writing \eref{eq:maxw-lag} explicitly in terms of $ A_\mu $, the energy-momentum tensor of the electromagnetic field is:
\begin{equation}
  \theta^{\mu \nu} = - F^{\mu \rho} \pa^\nu A_\rho + \frac{1}{4} \eta^{\mu \nu} F^2
\end{equation}
with $ F^2 \equiv F_{\mu \nu} F^{\mu \nu} $. To show the gauge-invariance of this tensor, recall \eref{eq:maxw-1}:
\begin{equation*}
  \theta^{\mu \nu} \mapsto \theta^{\mu \nu} + F^{\mu \rho} \pa^\nu \pa_\rho \alpha
  \qquad \Rightarrow \qquad
  P^\mu \mapsto P^\mu + \int \dd^3x\, \pa_\rho (F^{0 \rho} \pa^\mu \alpha) = P^\mu + \int \dd^3x\, \pa_i (F^{0i} \pa^\mu \alpha) = P^\mu
\end{equation*}
where the last term is a total spatial derivative, hence vanishing by divergence theorem provided that the field decreases sufficiently fast at infinity. To improve the energy-momentum tensor, add $ \pa_\rho (F^{\mu \rho} A^\nu) $, which is covariantly conserved by itself and whose $ \mu = 0 $ component is a total spatial derivative, so to obtain:
\begin{equation}
  T^{\mu \nu} = F^{\mu \rho} \tensor{F}{_\rho^\nu} + \frac{1}{4} \eta^{\mu \nu} F^2
\end{equation}
which is explicitly gauge-invariant and yields the usual expressions for the energy density $ T^{00} = \frac{1}{2} \left( \ve{E}^2 + \ve{B}^2 \right) $ and the momentum density $ T^{0i} = \left( \ve{E} \times \ve{B} \right)^i $.\\
In a general field theory, the observable quantities are the charges, not the currents: two Lagrangian densities which differ by a total 4-divergence are physically equivalent and give the same equations of motion, but the conserved currents obtained through Noether theorem are different, while the associated Noether charges are the same.

\subsection{Matter coupling}

In the presence of an external current $ j^\mu $, \eref{eq:maxw-2} is not modified, as it is a consequence of the definition of $ F^{\mu \nu} $ (assuming regular gauge fields), while \eref{eq:maxw-1} becomes:
\begin{equation}
  \pa_\mu F^{\mu \nu} = j^\nu
  \label{eq:maxw-3}
\end{equation}
By Schwarz lemma, this equation is consistent only if $ \pa_\mu j^\mu = 0 $. This can be understood in light of gauge invariance, considering the action:
\begin{equation}
  \mathcal{S}_\text{M} = - \int \dd^4x \left[ \frac{1}{4} F^2 + j^\mu A_\mu \right]
\end{equation}
A gauge transformation $ A_\mu \mapsto A_\mu - \pa_\mu \alpha $ implies $ \mathcal{S}_\text{M} \mapsto \mathcal{S}_\text{M} + \int \dd^4x\, j^\mu \pa_\mu \alpha $: integrating by parts, it is clear that $ \mathcal{S}_\text{M} $ is gauge invariant only if $ \pa_\mu j^\mu = 0 $.

\subsubsection{Dirac field}










