\selectlanguage{english}

\section{Maxwell theory}

The electromagnetic field is described by a 4-vector $ A_\mu $, the \textit{gauge potential}. From this, the field strength tensor is defined as:
\begin{equation}
  F_{\mu \nu} \defeq \pa_\mu A_\nu - \pa_\nu A_\mu
\end{equation}
which is related to the electric and magnetic fields as $ F^{0i} = - E^i $ and $ F^{ij} = - \epsilon^{ijk} B^k $. The Lagrangian of the free electromagnetic field is:
\begin{equation}
  \lag_\text{M} = - \frac{1}{4} F_{\mu \nu} F^{\mu \nu}
  \label{eq:maxw-lag}
\end{equation}
The associated equations of motion are:
\begin{equation}
  \pa_\mu F^{\mu \nu} = 0
  \label{eq:maxw-1}
\end{equation}
Moreover, defining $ \tilde{F}^{\mu \nu} \equiv \frac{1}{2} \epsilon^{\mu \nu \rho \sigma} F_{\rho \sigma} $ (the Hodge dual), it is trivial to check that, by Schwarz lemma:
\begin{equation}
  \pa_\mu \tilde{F}^{\mu \nu} = 0
  \label{eq:maxw-2}
\end{equation}
\eeref{eq:maxw-1}{eq:maxw-2} are exactly Maxwell equations in the absence of sources: when written in terms of $ \ve{E} $ and $ \ve{B} $, \eref{eq:maxw-1} gives the equations for $ \bs{\nabla} \cdot \ve{E} $ and $ \bs{\nabla} \times \ve{B} $, while \eref{eq:maxw-2} those for $ \bs{\nabla} \times \ve{E} $ and $ \bs{\nabla} \cdot \ve{B} $.

\subsection{Gauge invariance}

A crucial local symmetry of the Maxwell Lagrangian is the symmetry under local gauge transformations like:
\begin{equation}
  A_\mu(x) \mapsto A_\mu(x) - \pa_\mu \alpha(x)
  \label{eq:qed-gauge-inv}
\end{equation}
with arbitrary $ \alpha \in \mathcal{C}^\infty(\R^{1,3}) $. Considering the free electromagnetic field, the global version of this transformation (that is, $ \alpha $ independent of $ x $) yields no conserved charge, as the associated Noether current vanishes identically.

\begin{theorem}{Radiation gauge}{}
  In the absence of sources, it is always possible to choose the \textit{radiation gauge}:
  \begin{equation}
    A_0 = 0
    \qquad \qquad
    \bs{\nabla} \cdot \ve{A} = 0
    \label{eq:radiation-gauge}
  \end{equation}
\end{theorem}

\begin{proofbox}
  \begin{proof}
    Starting from a general gauge potential $ A_\mu $, the condition $ A_0 = 0 $ is achieved through:
    \begin{equation*}
      A_\mu \mapsto A_\mu - \pa_\mu \int_{t_0}^t \dd\tau\, A_0(\tau,\ve{x})
    \end{equation*}
    Then, $ A_0 = 0 $ will remain unchanged if another gauge transformation with $ \alpha(x) = \alpha(\ve{x}) $ is performed. Consider:
    \begin{equation*}
      \alpha(\ve{x}) = - \int_{\R^3} \frac{\dd^3y}{4\pi \abs{\ve{x} - \ve{y}}} \pa_i A^i(t,\ve{y})
    \end{equation*}
    which is independent of $ t $ since $ E^i = -\pa_0 A^i $, as $ A_0 = 0 $, so $ \pa_i E^i = 0 $ implies $ \pa_0 \pa_i A^i = 0 $. Recall the identity:
    \begin{equation}
      \lap_\ve{x} \frac{1}{\abs{\ve{x} - \ve{y}}} = - 4\pi \delta^{(3)}(\ve{x} - \ve{y})
    \end{equation}
    Thus:
    \begin{equation*}
      \bs{\nabla} \cdot \ve{A} \mapsto \bs{\nabla} \cdot \ve{A} - \lap_\ve{x} \alpha = \pa_i A^i(t,\ve{x}) - \pa_i A^i(t,\ve{x}) = 0
    \end{equation*}
  \end{proof}
\end{proofbox}

The radiation gauge clearly implies the \textit{Lorentz gauge}:
\begin{equation}
  \pa_\mu A^\mu = 0
\end{equation}
In this gauge, the equations of motions \eref{eq:maxw-1} become:
\begin{equation}
  \Box A^\mu = 0
  \label{eq:maxw-lor}
\end{equation}
which are massless KG equations for each component of the gauge potential. Plane-wave solutions take the form:
\begin{equation}
  A_\mu(x) = \epsilon_\mu(k) e^{-i k_\mu x^\mu} + c.c.
\end{equation}
where $ \epsilon_\mu(x) $ is the \textit{polarization vector}. Then, \eref{eq:maxw-lor} gives $ k^2 = 0 $, while the chosen radiation gauge implies $ \epsilon_0 = 0 $ and $ \bs{\epsilon} \cdot \ve{k} = 0 $: therefore, an electromagnetic wave has only two degrees of freedom, represented by a polarization vector $ \bs{\epsilon} $ perpendicular to the direction of propagation.\\
The advantage of the radiation gauge is that it exposes clearly the physical degrees of freedom of the electromagnetic field, while sacrificing explicit Lorentz covariance; on the other hand, the Lorentz gauge retains the explicit Lorentz covariance, at the cost of redundant degrees of freedom.

\subsection{Energy-momentum tensor}

By \eref{eq:en-mom-tensor}, writing \eref{eq:maxw-lag} explicitly in terms of $ A_\mu $, the energy-momentum tensor of the electromagnetic field is:
\begin{equation}
  \theta^{\mu \nu} = - F^{\mu \rho} \pa^\nu A_\rho + \frac{1}{4} \eta^{\mu \nu} F^2
\end{equation}
with $ F^2 \equiv F_{\mu \nu} F^{\mu \nu} $. To show the gauge-invariance of this tensor, recall \eref{eq:maxw-1}:
\begin{equation*}
  \theta^{\mu \nu} \mapsto \theta^{\mu \nu} + F^{\mu \rho} \pa^\nu \pa_\rho \alpha
  \qquad \Rightarrow \qquad
  P^\mu \mapsto P^\mu + \int \dd^3x\, \pa_\rho (F^{0 \rho} \pa^\mu \alpha) = P^\mu + \int \dd^3x\, \pa_i (F^{0i} \pa^\mu \alpha) = P^\mu
\end{equation*}
where the last term is a total spatial derivative, hence vanishing by divergence theorem provided that the field decreases sufficiently fast at infinity. To improve the energy-momentum tensor, add $ \pa_\rho (F^{\mu \rho} A^\nu) $, which is covariantly conserved by itself and whose $ \mu = 0 $ component is a total spatial derivative, so to obtain:
\begin{equation}
  T^{\mu \nu} = F^{\mu \rho} \tensor{F}{_\rho^\nu} + \frac{1}{4} \eta^{\mu \nu} F^2
\end{equation}
which is explicitly gauge-invariant and yields the usual expressions for the energy density $ T^{00} = \frac{1}{2} \left( \ve{E}^2 + \ve{B}^2 \right) $ and the momentum density $ T^{0i} = \left( \ve{E} \times \ve{B} \right)^i $.\\
In a general field theory, the observable quantities are the charges, not the currents: two Lagrangian densities which differ by a total 4-divergence are physically equivalent and give the same equations of motion, but the conserved currents obtained through Noether theorem are different, while the associated Noether charges are the same.

\subsection{Matter coupling}

In the presence of an external current $ j^\mu $, \eref{eq:maxw-2} is not modified, as it is a consequence of the definition of $ F^{\mu \nu} $ (assuming regular gauge fields), while \eref{eq:maxw-1} becomes:
\begin{equation}
  \pa_\mu F^{\mu \nu} = j^\nu
  \label{eq:maxw-3}
\end{equation}
By Schwarz lemma, this equation is consistent only if $ \pa_\mu j^\mu = 0 $. This can be understood in light of gauge invariance, considering the action:
\begin{equation}
  \mathcal{S}_\text{M} = - \int \dd^4x \left[ \frac{1}{4} F^2 + j^\mu A_\mu \right]
\end{equation}
A gauge transformation $ A_\mu \mapsto A_\mu - \pa_\mu \alpha $ implies $ \mathcal{S}_\text{M} \mapsto \mathcal{S}_\text{M} + \int \dd^4x\, j^\mu \pa_\mu \alpha $: integrating by parts, it is clear that $ \mathcal{S}_\text{M} $ is gauge invariant only if $ \pa_\mu j^\mu = 0 $.

\subsubsection{Dirac field}

The coupling of the electromagnetic field to the Dirac field is an example of the general procedure of writing a gauge-invariant action for a gauge theory. In particular, consider a theory with a global $ \Un{1} $ invariance, which is a symmetry of the free Dirac action by \eref{eq:dirac-u1-symm}. Now generalize to a local $ \Un{1} $ symmetry:
\begin{equation}
  \Psi(x) \mapsto e^{iq \alpha(x)} \Psi(x)
  \label{eq:qed-dir-inv}
\end{equation}
with $ q \in \R $. This no longer is a symmetry of the Dirac action, however it can be combined with \eref{eq:qed-gauge-inv} defining the \textit{covariant derivative}:
\begin{equation}
  D_\mu \Psi \defeq (\pa_\mu + i q A_\mu) \Psi
\end{equation}

\begin{proposition}[before upper = {\tcbtitle}]{}{}
  \begin{equation}
    D_\mu \Psi(x) \mapsto e^{i q \alpha(x)} D_\mu \Psi(x)
    \label{eq:qed-cov-der}
  \end{equation}
\end{proposition}

\begin{proofbox}
  \begin{proof}
      $ D_\mu \Psi \mapsto \left[ \pa_\mu + i q (A_\mu - \pa_\mu \alpha) \right] e^{i q \alpha} \Psi = e^{i q \alpha} \left[ i q \pa_\mu \alpha + \pa_\mu + i q A_\mu - i q \pa_\mu \alpha \right] \Psi = e^{i q \alpha} D_\mu \Psi $
  \end{proof}
\end{proofbox}

The Lagrangian with a local $ \Un{1} $ symmetry is found replacing $ \pa_\mu \mapsto D_\mu $ (\textit{minimal coupling}): the global symmetry is gauged to a local symmetry, resulting in a gauge theory with gauge field $ A_\mu $. \\
Applying this to \eref{eq:dirac-lagrangian}:
\begin{equation}
  \lag_\text{D} = \bar{\Psi} (i \slashed{\pa} - m) \Psi - q A_\mu \bar{\Psi} \gamma^\mu \Psi
\end{equation}
where $ j_\text{V}^\mu \defeq \bar{\Psi} \gamma^\mu \Psi $ is the Noether current associated to the global $ \Un{1} $ symmetry. The associated conserved charge then is:
\begin{equation}
  Q = \int \dd^3x\, \bar{\Psi} \gamma^0 \Psi = \int \dd^3x\, \Psi\dg \Psi
\end{equation}

\subsubsection{Complex scalar field}

A complex scalar field has a global $ \Un{1} $ symmetry $ \phi \mapsto e^{i q \alpha} \phi $, thus the covariant derivative is identical to \eref{eq:qed-cov-der} and the gauged Lagrangian reads (recall \eref{eq:compl-scalar-lag}):
\begin{equation}
  \lag = \pa_\mu \phi \pa^\mu \phi + i q A^\mu (\phi \pa_\mu \phi^* - \phi^* \pa_\mu \phi) + q^2 \abs{\phi}^2 A_\mu A^\mu - m^2 \phi^* \phi
\end{equation}
where $ j_\mu \defeq i \phi^* \overleftrightsmallarrow{\pa_\mu} \phi $ is the Noether current associated to the global $ \Un{1} $ symmetry.

\subsubsection{Higher order interaction terms}

Although a real scalar field cannot be coupled to the electromagnetic field through the minimal coupling (as the real condition imposes $ q = 0 $, i.e. a neutral field), interaction terms are possible via higher order terms, as $ \lag_\text{int} \sim \phi F_{\mu \nu} F^{\mu \nu} $ or $ \lag_\text{int} \sim \phi \epsilon_{\mu \nu \rho \sigma} F^{\mu \nu} F^{\rho \sigma} $. \\
The same is possible for the Dirac field too, for example with $ \lag_\text{int} \sim \bar{\Psi} \sigma^{\mu \nu} \Psi F^{\mu \nu} $: note, however, that these non-minimal couplings have dimensional coupling constants (with dimension of the inverse of a mass), which have a less fundamental significance than dimensionless coupling constant.

\begin{example}{Neutral pions}{}
  The neutral pion $ \pi^0 $ is described by a pseudoscalar field, thus its interaction with the electromagnetic field needs to be a pseudoscalar term, like $ \epsilon_{\mu \nu \rho \sigma} F^{\mu \nu} F^{\rho \sigma} $ (this gives a good phenomenological description), as opposed to parity-invariant terms like $ F_{\mu \nu} F^{\mu \nu} $.
\end{example}

\section{Quantization}

Due to gauge symmetry, the gauge field gives a redundant physical description, therefore the quantization procedure can be carried in two different ways: fixing the gauge, thus working with only physical degrees of freedom but at the cost of loosing explicit Lorentz invariance, or considering the whole $ A_\mu $, hence carrying spurious degrees of freedom.

\subsection{Quantization in the radiation gauge}

As of \eref{eq:radiation-gauge}, Maxwell equations \eref{eq:maxw-lor} read $ \Box \ve{A} = \ve{0} $, with general classical solution:
\begin{equation*}
  \ve{A}(x) = \int \frac{\dd^3p}{(2\pi)^3 \sqrt{2\omega_\ve{p}}} \sum_{\lambda = 1,2} \left[ \bs{\epsilon}(\ve{p},\lambda) a_{\ve{p},\lambda} e^{-i p_\mu x^\mu} + \bs{\epsilon}^*(\ve{p},\lambda)a^*_{\ve{p},\lambda} e^{i p_\mu x^\mu} \right]_{p^0 = \omega_\ve{p}}
\end{equation*}
Inserting this expression into $ \Box \ve{A} = \ve{0} $ results in the mass-shell condition $ p^2 = 0 $, i.e. $ \omega_\ve{p} = \abs{\ve{p}} $. On the other hand, the gauge condition $ \dive \ve{A} = 0 $ requires $ \bs{\epsilon} \cdot \ve{p} = 0 $: for each fixed $ \ve{p} $, this solution has two orthonormal solutions labelled by $ \lambda = 1,2 $, which describe the two physical degrees of freedom of the electromagnetic field. \\
The classical solution can be promoted to a hermitian operator as:
\begin{equation}
  \ve{A}(x) = \int \frac{\dd^3p}{(2\pi)^3 \sqrt{2\omega_\ve{p}}} \sum_{\lambda = 1,2} \left[ \bs{\epsilon}(\ve{p},\lambda) a_{\ve{p},\lambda} e^{-i p_\mu x^\mu} + \bs{\epsilon}^*(\ve{p},\lambda)a\dg_{\ve{p},\lambda} e^{i p_\mu x^\mu} \right]_{p^0 = \omega_\ve{p}}
\end{equation}
and imposing the canonical commutation relations:
\begin{equation}
  [a_{\ve{p},\lambda} , a_{\ve{q},\lambda'}\dg] = (2\pi)^3 \delta^{(3)}(\ve{p} - \ve{q}) \delta_{\lambda \lambda'}
\end{equation}
In terms of commutators of $ A^i $ and conjugate momenta, consider that $ \Pi_0 = 0 $ (as $ A_0 = 0 $) and\footnotemark:
\begin{equation*}
  \Pi^i = \frac{\delta}{\delta(\pa_0 A_i)} \left( -\frac{1}{4} F_{\mu \nu} F^{\mu \nu} \right) = \frac{\delta}{\delta(\pa_0 A_i)} \left( -\frac{1}{2} F_{0i} F^{0i} \right) = - F^{0i} = E^i
\end{equation*}

\footnotetext{Note that $ \Pi^i $ is the momentum conjugate to $ A_i $, while $ \Pi_i = - \Pi^i $ is the one conjugate to $ A^i $.}

\begin{lemma}[before upper = {\tcbtitle}]{}{}
  \begin{equation}
    \frac{1}{2} \sum_{\lambda = 1,2} \left[ \epsilon^i(\ve{k},\lambda) \epsilon^{*j}(\ve{k},\lambda) + \epsilon^{*i}(-\ve{k},\lambda) \epsilon^j(-\ve{k},\lambda) \right] \delta^{ij} - \frac{k^i k^j}{\ve{k}^2}
  \end{equation}
\end{lemma}

\begin{proofbox}
  \begin{proof}
    Trivially verified in a frame where $ \ve{k} = (0,0,k) $ choosing $ \bs{\epsilon}(\ve{k},1) = (1,0,0) $ and $ \bs{\epsilon}(\ve{k},2) = (0,1,0) $ and valid in any frame\footnote{In particular, the linear polarizations satisfy:
    \begin{equation}
      \sum_{\lambda = 1,2} \epsilon^i(\ve{k},\lambda) \epsilon^j(\ve{k},\lambda) = \delta^{ij} - \frac{k^i k^j}{\ve{k}^2}
    \end{equation}} is ensured as both sides transform as tensor under rotations.
  \end{proof}
\end{proofbox}

\begin{proposition}[before upper = {\tcbtitle}]{}{}
  \begin{equation}
    [A^i(t,\ve{x}) , E^j(t,\ve{y})] = - i \int \frac{\dd^3k}{(2\pi)^3} e^{i \ve{k} \cdot (\ve{x} - \ve{y})} \left( \delta^{ij} - \frac{k^i k^j}{\ve{k}^2} \right)
  \end{equation}
\end{proposition}

The r.h.s. of this equation is similar to a Dirac delta and it is called \textit{transverse Dirac delta}, defined in order to get $ [\dive \ve{A}(t,\ve{x}) , \ve{E}(t,\ve{y})] = \ve{0} $ (as $ \dive \ve{A} = 0 $), being the integrand proportional to:
\begin{equation*}
  k^i \left( \delta^{ij} - \frac{k^i k^j}{\ve{k}^2} \right) = k^j - k^j = 0
\end{equation*}

\paragraph{Fock space}

The standard construction of the Fock space proceeds defining the vacuum state $ a_{\ve{p},\lambda} \ket{0} = 0 $. The Hamiltonian and the linear momentum are then found as:
\begin{equation}
  H = \frac{1}{2} \normord \int \dd^3x\, \left[ \ve{E}^2 + \ve{B}^2 \right] = \int \frac{\dd^3k}{(2\pi)^3} \sum_{\lambda = 1,2} \omega_\ve{k} a_{\ve{k},\lambda}\dg a_{\ve{k},\lambda}
\end{equation}
\begin{equation}
  \ve{P} = \normord \int \dd^3x\, \ve{E} \times \ve{B} = \int \frac{\dd^3k}{(2\pi)^3} \sum_{\lambda = 1,2} \ve{k} a_{\ve{k},\lambda}\dg a_{\ve{k},\lambda}
\end{equation}

Therefore, $ a_{\ve{k},\lambda}\dg \ket{0} $ describes a massless particle with energy $ \omega_\ve{k} $ and momentum $ \ve{k} $. To study spin, it is necessary to switch to circular polarizations:
\begin{equation}
  a_{\ve{k},\pm}\dg \defeq \frac{1}{\sqrt{2}} \left( a_{\ve{k},1}\dg \pm i a_{\ve{k},2}\dg \right)
\end{equation}
where $ a_{\ve{k},1} , a_{\ve{k},2} $ are the linear polarizations $ \bs{\epsilon}(\ve{k},1) = (1,0,0) , \bs{\epsilon}(\ve{k},2) = (0,1,0) $. Linear polarizations are not helicity eigenstates, while circular polarizations are:
\begin{align*}
  S^3 a_{\ve{k},1}\dg \ket{0} = + i a_{\ve{k},2}\dg \ket{0} & \qquad & S^3 a_{\ve{k},+}\dg \ket{0} = + a_{\ve{k},+}\dg \ket{0} \\
  S^3 a_{\ve{k},2}\dg \ket{0} = - i a_{\ve{k},1}\dg \ket{0} & \qquad & S^3 a_{\ve{k},-}\dg \ket{0} = - a_{\ve{k},-}\dg \ket{0}
\end{align*}
In conclusion, $ a_{\ve{k},\pm}\dg \ket{0} $ describe massless particles with energy $ \omega_\ve{k} $, momentum $ \ve{k} $, spin $ 1 $ and helicity $ \pm 1 $: these are photons. \\
Defining angular momentum and boost generators too in terms of ladder operators, it is possible to show that Lorentz invariance is preserved by the quantization procedure, although not explicitly.

\paragraph{Discrete transformations}

It is possible to define parity and charge conjugation on photon states. The electric field is a true vector, while the magnetic field is a pseudovector, thus the gauge potential is a true vector: $ \mathcal{P} \ve{A}(t,\ve{x}) = - \ve{A}(t,-\ve{x}) $. In terms of photon states:
\begin{equation}
  \mathcal{P} \ket{\gamma ; \ve{k}, \ve{s}} = - \ket{\gamma ; -\ve{k}, \ve{s}}
\end{equation}
so the intrinsic parity of physical photon states is $ -1 $. \\
As the fermionic current changes sign under charge conjugation (\eref{eq:charge-conj-fermion-curr}), it is a symmetry of the QED Lagrangian if $ \mathcal{C} A^\mu \mathcal{C} = - A^\mu $, i.e. $ \mathcal{C} a_{\ve{k},\lambda}\dg \mathcal{C} = - a_{\ve{k},\lambda}\dg $. As $ \mathcal{C} \ket{0} = +\ket{0} $ and $ \mathcal{C}^2 = \id $ by definition, then $ \mathcal{C} a_{\ve{k},\lambda}\dg \ket{0} = \mathcal{C} a_{\ve{k},\lambda} \mathcal{C} \mathcal{C} \ket{0} = - a_{\ve{k},\lambda}\dg \ket{0} $, or:
\begin{equation}
  \mathcal{C} \ket{\gamma ; \ve{k} , \ve{s}} = - \ket{\gamma ; \ve{k} , \ve{s}}
\end{equation}

\subsection{Covariant quantization}

The Maxwell Lagrangian \eref{eq:maxw-lag} cannot be straightforwardly quantize, as $ \Pi^0 $ cannot be defined due to the absence of $ \pa_0 A_0 $ terms. The basic idea of the covariant quantization of the electromagnetic field, or Gupta-Bleuler quantization, is to start from a modified Lagrangian:
\begin{equation}
  \lag = - \frac{1}{4} F_{\mu \nu} F^{\mu \nu} - \frac{1}{2} (\pa_\mu A^\mu)^2
\end{equation}
Conjugate momenta are then found to be:
\begin{equation*}
  \Pi^\mu = \frac{\pa \lag}{\pa (\pa_0 A_\mu)}
  \qquad \Rightarrow \qquad
  \Pi^i = - F^{0i} = E^i
  \qquad
  \Pi^0 = - \pa_\mu A^\mu
\end{equation*}
Canonical commutation relations now take the form:
\begin{equation}
  [A^\mu(t,\ve{x}) , \Pi^\nu(t,\ve{y})] = i \eta^{\mu \nu} \delta^{(3)}(\ve{x} - \ve{y})
  \label{eq:qed-canon-comm-rel-1}
\end{equation}
The metric ensures Lorentz covariance. The equations of motion are $ \Box A^\mu = 0 $, thus the gauge field operators are:
\begin{equation}
  A_\mu(x) = \int \frac{\dd^3p}{(2\pi)^3 \sqrt{2\omega_\ve{p}}} \sum_{\lambda = 0}^{3} \left[ \epsilon_\mu(\ve{p},\lambda) a_{\ve{p},\lambda} e^{-i p_\mu x^\mu} + \epsilon_\mu^*(\ve{p},\lambda) a_{\ve{p},\lambda}\dg e^{i p_\mu x^\mu} \right]_{p^0 = \omega_\ve{p}}
\end{equation}
$ \Box A_\mu = 0 $ imposes $ p^2 = 0 $. Note that the modified Lagrangian is not gauge invariant, so there is no constraint on $ \epsilon^\mu $; in the frame $ p^\mu = (p,0,0,p) $ a convenient choice of basis is $ \epsilon^\mu(\ve{p},\lambda) = \delta^\mu_\lambda $, hence only $ \lambda = 1,2 $ satisfy $ \epsilon_\mu p^\mu = 0 $. \eref{eq:qed-canon-comm-rel-1} becomes:
\begin{equation}
  [a_{\ve{p},\lambda} , a_{\ve{p},\lambda'}\dg] = - (2\pi)^3 \delta^{(3)}(\ve{p} - \ve{q}) \eta_{\lambda \lambda'}
\end{equation}
Note that the commutator for $ \lambda = \lambda' = 0 $ has a negative sign, which means that the norm is not positive defined on this Fock space (rendering impossible its interpretation as a probability):
\begin{equation}
  \braket{\ve{p} , \lambda | \ve{p} , \lambda} = (2\omega_\ve{p}) \braket{0 | a_{\ve{p},\lambda} a_{\ve{p},\lambda}\dg | 0} = (2\omega_\ve{p}) \braket{0 | [a_{\ve{p},\lambda} , a_{\ve{p},\lambda}\dg] | 0} = - 2\omega_\ve{p} V \eta_{\lambda \lambda}
\end{equation}
(where $ V \equiv (2\pi)^3 \delta^{(3)}(\ve{0}) $) which is negative for $ \lambda = 0 $. However, the only physical states are those associated with transverse polarization vectors, thus these problematic states can be shown to be unphysical. \\
To recover the correct description of QED from the modified Lagrangian, it is necessary to restrict the Fock space; in particular, any two physical states must satisfy:
\begin{equation}
  \braket{\text{phys}' | \pa_\mu A^\mu | \text{phys}} = 0
  \label{eq:qed-phys-cond-1}
\end{equation}
Note that $ \pa_\mu A^\mu $ can be decomposed into its positive- and negative-frequency parts $ \pa_\mu A^\mu = (\pa_\mu A^\mu)^+ + (\pa_\mu A^\mu)^- $ as:
\begin{equation*}
  (\pa_\mu A^\mu)^+ \equiv -i \int \frac{\dd^3p}{(2\pi)^3 \sqrt{2\omega_\ve{p}}} \sum_{\lambda = 0}^{3} p_\mu \epsilon^\mu(\ve{p},\lambda) a_{\ve{p},\lambda} e^{-i p_\mu x^\mu} = {(\pa_\mu A^\mu)}\dg
\end{equation*}
Therefore, \eref{eq:qed-phys-cond-1} is equivalent to:
\begin{equation}
  (\pa_\mu A^\mu)^+ \ket{\text{phys}} = 0
\end{equation}
This is the definition of the physical subspace of the Fock space, as it preserves the linear structure of the physical Hilbert space. Now consider the most general superposition of polarization states with momentum $ \ve{k} $, i.e. $ \ket{\ve{k}} = \sum_{\lambda = 0}^{3} c_\lambda a_{\ve{k},\lambda}\dg \ket{0} $, and choose the frame $ k^\mu = (k,0,0,k) $: in this frame the physical-state condition reads $ c_0 + c_3 = 0 $. Thus, all transverse states $ \lambda = 1,2 $ are physical, while there is only one non-transverse state that remains:
\begin{equation*}
  \ket{\phi} \equiv (a_{\ve{k},0}\dg - a_{\ve{k},3}\dg) \ket{0}
\end{equation*}
The most general one-particle state of the physical subspace can then be written as $ \ket{\ve{k}_\text{T}} + c \ket{\phi} $. However:
\begin{equation*}
  \braket{\phi | \phi} = \braket{0 | (a_{\ve{k},0} - a_{\ve{k},3}) (a_{\ve{k},0}\dg - a_{\ve{k},3}\dg) | 0} = \braket{0 | [a_{\ve{k},0} , a_{\ve{k},0}\dg] + [a_{\ve{k},3} , a_{\ve{k},3}\dg] | 0} = 0
\end{equation*}
This means that $ \ket{\phi} $ is orthogonal to all physical states. Moreover, only transverse states contribute to the energy and to the momentum, as $ H , \ve{P} \sim - \eta^{\lambda \lambda'} a_{\ve{k},\lambda}\dg a_{\ve{k},\lambda'} $ and (as $ c_0 + c_3 = 0 $):
\begin{equation*}
  (a_{\ve{k},0} - a_{\ve{k},3}) \ket{\psi} = 0
  \quad \Rightarrow \quad
  \braket{\text{phys}' | -a_{\ve{k},0}\dg a_{\ve{k},0} + a_{\ve{k},3}\dg a_{\ve{k},3} | \text{phys}} = \braket{\text{phys}' | (-a_{\ve{k},0}\dg + a_{\ve{k},3}\dg) a_{\ve{k},3} | \text{phys}} = 0
\end{equation*}
These facts mean that $ \ket{\ve{k}_\text{T}} $ and $ \ket{\ve{k}_\text{T}} + c \ket{\phi} $ are physically indistinguishable: photons can then be identified as the equivalence classes with respect to the equivalence relation $ \ket{\psi} \sim \ket{\psi} + c \ket{\phi} $. \\
This procedure has eliminated the spurious degrees of freedom, thus showing the equivalence between covariant quantization and gauge quantization.










