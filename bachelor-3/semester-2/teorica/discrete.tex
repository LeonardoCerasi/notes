\selectlanguage{english}

\section{One-dimensional harmonic crystal}

Consider a simple one-dimensional model of a crystal where atoms of mass $ m \equiv 1 $ lie at rest on the $ x $-axis, with equilibrium positions labelled by $ n \in \N $ and equally spaced by a distance $ a $.\\
Assuming these atoms are free to vibrate only in the $ x $ direction (longitudinal waves), and denoting the displacement of the atom at position $ n $ as $ \eta_n $, one can write the Lagrangian for a \textit{harmonic crystal} as:
\begin{equation}
  L = \sum_{n} \left[ \frac{1}{2} \dot{\eta}_n^2 - \frac{\lambda}{2} \left( \eta_n - \eta_{n-1} \right)^2 \right]
  \label{eq:1.1}
\end{equation}
where $ \lambda $ is the spring constant. From the Lagrange equations, the classical equations of motions are:
\begin{equation}
  \ddot{\eta}_n = \lambda \left( \eta_{n+1} - 2 \eta_n + \eta_{n-1} \right)
  \label{eq:1.2}
\end{equation}
The solutions can be written as complex travelling waves:
\begin{equation}
  \eta_n (t) = e^{i \left( k n - \omega t \right)}
  \label{eq:1.3}
\end{equation}
where the dispersion relation is:
\begin{equation}
  \omega^2 = 2\lambda \left( 1 - \cos k \right) \approx \lambda k^2
  \label{eq:1.4}
\end{equation}
Therefore, in the long-wavelength limit $ k \ll 1 $ waves propagate with velocity $ c = \sqrt{\lambda} $. To determine the normal modes, there need to be boundary conditions: imposing boundary conditions:
\begin{equation}
  \eta_{n + N} = \eta_n \qquad \Rightarrow \qquad k_m = \frac{2\pi m}{N} \,,\, m = 0, 1, \dots, N - 1
  \label{eq:1.5}
\end{equation}
The normal-mode expansion can then be written as:
\begin{equation}
  \eta (t) = \sum_{m = 0}^{N - 1} \left[ \mathcal{A}_m e^{i \left( k_m n - \omega_m t \right)} + \mathcal{A}^* e^{-i \left( k_m n - \omega_m t \right)} \right]
  \label{eq:1.6}
\end{equation}
where the complex conjugate is added to ensure that the total displacement is real. The momentum canonically-conjucated to the displacement is defined as:
\begin{equation}
  \pi_n \defeq \frac{\pa L}{\pa \dot{\eta}_n} = \dot{\eta}_n
  \label{eq:1.7}
\end{equation}
In quantum mechanics, $ \eta_n $ and $ \Pi_n $ become operators with canonical commutator $ [ \hat{\eta}_j, \hat{\pi}_k ] = i \hbar \delta_{jk} $. Implementing time evolution with the \textit{Heisenberg picture}\footnote{Recall that $ \hat{\mathcal{O}}(t) = e^{\frac{i}{\hbar} \hat{\mathcal{H}} t} \hat{\mathcal{O}}(0) e^{-\frac{i}{\hbar} \hat{\mathcal{H}} t} $ and $ \frac{\dd \hat{\mathcal{O}}}{\dd t} = \frac{i}{\hbar} [ \hat{\mathcal{H}}, \hat{\mathcal{O}} ] $.}:
\begin{equation}
  [ \hat{\eta}_j(t), \hat{\pi}_k(t) ] = i \hbar \delta_{jk}
  \label{eq:1.8}
\end{equation}
The commutator of operators evaluated at different times requires solving the dynamics of the system. It is useful to introduce \textit{annihilation} and \textit{creation operators} $\footnote{For a harmonic oscillator $ \hat{\mathcal{H}} = \frac{1}{2} \hat{p}^2 + \frac{1}{2} \omega^2 \hat{x}^2 $, so $ \frac{\dd \hat{x}}{\dd t} = \hat{p}(t) $ and $ \frac{\dd \hat{p}}{\dd t} = -\omega^2 \hat{x}(t) $ and the solution can be written as: $$ \hat{x}(t) = \sqrt{\frac{\hbar}{2\omega}} \left[ \hat{a}(t) + \hat{a}^\dagger(t) \right] \qquad \qquad \hat{p}(t) = -i \omega \sqrt{\frac{\hbar}{2\omega}} \left[ \hat{a}(t) - \hat{a}^\dagger(t) \right] $$ Inverting these expressions one finds $ [\hat{a}(t), \hat{a}^\dagger(t)] = 1 $ and $ \hat{\mathcal{H}} = \hbar \omega \left( \hat{a}^\dagger(t) \hat{a} + \frac{1}{2} \right) $. The time evolution $ \hat{a}(t) = e^{-i \omega t} \hat{a}(0) $ ensures that $ \hat{\mathcal{H}} $ is times-independent.} \hat{a}(t) $ and $ \hat{a}^\dagger(t) $, so that Eq. \ref{eq:1.6} becomes:
\begin{equation}
  \hat{\eta}_n(t) = \sum_{m = 0}^{N - 1} \sqrt{\frac{\hbar}{2 \omega_m}} \frac{1}{\sqrt{N}} \left[ e^{i \left( k_m n - \omega_m t \right)} \hat{a}_m + e^{-i \left( k_m n - \omega_m t \right)} \hat{a}_m^\dagger \right]
  \label{eq:1.9}
\end{equation}
where $ [ \hat{a}_j, \hat{a}_k^\dagger ] = \delta_{jk} $ and the $ N^{-1/2} $ ensures the normalization of normal modes. The proof of Eq. \ref{eq:1.8} follows from the finite Fourier series identity (sum of a geometric progression):
\begin{equation}
  \sum_{m = 0}^{N - 1} e^{i k_m \left( n - n' \right)} = N \delta_{n n'}
  \label{eq:1.10}
\end{equation}
The Hamiltonian of the system can be written as:
\begin{equation}
  \hat{\mathcal{H}} = \sum_{n} \left[ \frac{1}{2} \hat{\pi}_n^2 + \frac{\lambda}{2} \left( \hat{\eta}_n - \hat{\eta}_{n-1} \right)^2 \right] = \sum_{m = 0}^{N - 1} \hbar \omega_m \left( \hat{a}_m^\dagger \hat{a}_m + \frac{1}{2} \right)
  \label{eq:1.11}
\end{equation}
Generalizing the harmonic oscillator operator algebra (proven unique by Von Neumann), one can construct the Hilbert space for the harmonic crystal as:
\begin{equation}
  \hat{a}_m \ket{0} \quad \forall m = 0, 1, \dots, N - 1
  \label{eq:1.12}
\end{equation}
\begin{equation}
  \ket{n_0, n_1, \dots, n_{N-1}} = \prod_{m = 0}^{N - 1} \frac{( \hat{a}_m^\dagger )^{n_m}}{\sqrt{n_m!}} \ket{0}
  \label{eq:1.13}
\end{equation}
These are normalized eigenstates of Eq. \ref{eq:1.1} with energy eigenvalues:
\begin{equation}
  E_0 = \frac{1}{2} \sum_{m = 0}^{N - 1} \hbar \omega_m
  \label{eq:1.14}
\end{equation}
\begin{equation}
  E_{n_0, n_1, \dots, n_{N-1}} = E_0 + \sum_{m = 0}^{N - 1} n_m \hbar \omega_m
  \label{eq:1.15}
\end{equation}
This Hilbert space is called \textit{Fock space} and the excited states \textit{phonons}: these can be thought as $ \virgolette{particles} $ and $ n_m $ is the number of phonons in the $ m^{\mathrm{th}} $ normal mode.






