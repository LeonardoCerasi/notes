\selectlanguage{english}

\section{Time evolution}

Consider a system described by a Hamiltonian $ H $. Two different ways of describing its time evolution are possible.

\subsection{Schrödinger picture}

In the Schrödinger picture, states are considered time-dependent and operators time-independent. Consider an initial state $ \ket{\psi(0)} \equiv \ket{\psi} $: its time-evolution is:
\begin{equation}
  \ket{\psi(t)} = e^{-i H t} \ket{\psi}
\end{equation}
In this picture, the amplitude for a process $ \ket{a(t_0)} \rightarrow \ket{b(t)} $ is:
\begin{equation}
  \mathcal{A} = \braket{b | e^{-i H (t - t_0)} | a} \eqdef \braket{b | S(t,t_0) | a}
\end{equation}
In the $ t - t_0 \rightarrow \infty $ limit, $ S $ is known as $ S $\textit{-matrix}, and scattering amplitudes simply are its elements. This can be seen as an operator which realizes time evolution, as:
\begin{equation}
  \braket{\psi(t)} = S(t) \ket{\psi}
\end{equation}
where $ S(t) \equiv S(t,0) $.

\begin{proposition}{Unitarity of time evolution}{}
  $ S(t,t_0) $ is a unitary operator on $ \hilb $.
\end{proposition}

\begin{proofbox}
  \begin{proof}
    Consider a normalized initial state $ \ket{\psi} : \braket{\psi | \psi} = 1 $. Given a complete orthonormal eigenbasis $ \{\ket{n}\}_{n \in \N} $ of $ \hilb $, then it must be:
    \begin{equation}
      \sum_{n \in \N} \abs{\braket{n | S | \psi}}^2 = 1
    \end{equation}
    This can also be written as:
    \begin{equation}
      \sum_{n \in \N} \abs{\braket{n | S | \psi}}^2 = \sum_{n \in \N} \braket{\psi | S\dg | n} \braket{n | S | \psi} = \braket{\psi | S\dg S | \psi}
    \end{equation}
    Being $ \ket{\psi} $ arbitrary, it must be $ S\dg S = S S\dg = \id_\hilb $.
  \end{proof}
\end{proofbox}

The unitarity of the $ S $-matrix expresses the conservation of probabilities.

\subsection{Heisenberg picture}

In the Heisenberg picture, states are considered time-independent and operators time-dependent. This picture is better suited for QFT, as fields (which are operators) depend both on $ \ve{x} $ and $ t $, therefore the Heisenberg picture is natural in light of Lorentz covariance. \\
Given a state $ \ket{\psi(t)} $ in the Schrödinger picture, in the Heisenberg picture it is defined as:
\begin{equation}
  \ket{\psi,t} \defeq e^{i H t} \ket{\psi(t)}
\end{equation}
which is manifestly time-independent. An operator $ \mathcal{O} $ in the Schrödinger picture instead becomes:
\begin{equation}
  \mathcal{O}(t) \defeq e^{i H t} \mathcal{O} e^{-i H t}
\end{equation}
The label $ t $ in the definition of state is necessary to distinguish whose operator the state is eigenstate of: in fact, in general $ \mathcal{O}(t) \neq \mathcal{O}(t') $ for $ t \neq t' $, so they will have different eigenstates too. \\
In the Heisenberg picture, the $ S $-matrix becomes:
\begin{equation}
  \braket{b | S(t,t_0) | a} = \braket{b,t | a,t_0}
\end{equation}

\section{Path integrals}

In contrast to canonical quantization, in which fields are promoted to operators, in path-integral quantization they remain ordinary functions. While the former allows for a more direct understanding of the notion of particle (thanks to ladder operators), the latter has the advantage of not being intrinsically perturbative, thus better describing theories with non-perturbative effects.

\subsection{Path integral in Quantum Mechanics}

Consider a general (bosonic) quantum system, with Hermitian operators $ \{\hat{q}_a,\hat{p}_a\}_{a = 1,\dots,n} $ which satisfy the canonical commutation relations:
\begin{equation}
  [\hat{q}_a , \hat{p}_b] = i \delta_{ab}
  \label{eq:qm-path-int-comm}
\end{equation}
Eigenstates of these operators form two improperly-normalized complete orthonormal sets of eigenstates, with scalar product:
\begin{equation}
  \braket{q | p} = \prod_{a = 1}^n \frac{1}{\sqrt{2\pi}} e^{i q_a p_a} \equiv (2\pi)^{-n/2} e^{i q \cdot p}
  \label{qm-plane-wave}
\end{equation}
with $ q \equiv \{q_1 , \dots , q_n\} , p \equiv \{p_1 , \dots , p_n\} $. The switch from the Schrödinger picture to the Heisenberg one is carried by the Hamiltonian $ \hat{H}(\hat{q},\hat{p}) $ of the system, which is assumed to be normal-ordered with all $ q_a $ on the left of all $ p_a $: in this picture, all of the above relations remain valid for equal-time states.

\begin{theorem}{Hamiltonian path integral}{}
  The amplitude of $ \ket{q_\text{i},t_\text{i}} \rightarrow \ket{q_\text{f},t_\text{f}} $ is a functional integral:
  \begin{equation}
    \braket{q_\text{f},t_\text{f} | q_\text{i},t_\text{i}} = \int_{q(t_\text{i}) = q_\text{i}}^{q(t_\text{f}) = q_\text{f}} \prod_{a = 1}^n \dd q_a \frac{\dd p_a}{2\pi} \exp i \int_{t_\text{i}}^{t_\text{f}} \dd t \left[ \sum_{b = 1}^n \dot{q}_b(t) p_b(t) - H(q(t),p(t)) \right]
    \label{eq:qm-ham-path-int}
  \end{equation}
\end{theorem}

\begin{proofbox}
  \begin{proof}
    Consider a partition $ \{t_m\}_{m = 0, \dots, N} : t_m < t_j \,\,\forall m < j $ of the interval $ [t_\text{i} , t_\text{f}] $, so that $ t_0 \equiv t_\text{i} $ and $ t_N \equiv t_\text{f} $: for simplicity, take them equally spaced, i.e. $ t_m = t_0 + m \epsilon $, with $ N \epsilon = t_\text{f} - t_\text{i} $. The amplitude for $ \ket{q_\text{i},t_\text{i}} \rightarrow \ket{q_\text{f},t_\text{f}} $ can thus be computed as:
    \begin{equation*}
      \begin{split}
        \braket{q_\text{f},t_\text{f} | q_\text{i},t_\text{i}} = \int_{\R^{n(N-1)}} \prod_{m = 0}^{N-1} \dd q_m \braket{q_{m+1} , t_{m+1} | q_m , t_m}
      \end{split}
    \end{equation*}
    where the completeness relation was repeatedly used. Switching to the Schrödinger picture:
    \begin{equation*}
      \begin{split}
        \braket{q_{m+1} , t_{m+1} | q_m , t_m}
        & = \braket{q_{m+1} | e^{-i \hat H (t_{m+1} - t_m)} | q_m} \\
        & = \braket{q_{m+1} | e^{-i \hat H \epsilon} | q_m} = \int_{\R^n} \dd p_m \braket{q_{m+1} | p_m} \braket{p_m | e^{-i \hat H \epsilon} | q_m} \\
        & = (2\pi)^{-n/2} \int_{\R^n} \dd p_m\, e^{i q_{m+1} \cdot p_m} \braket{p_m | e^{-i \hat{H}(\hat{q},\hat{p}) \epsilon} | q_m} \\
        & = \frac{1}{(2\pi)^n} \int_{\R^n} \dd p_m\, e^{i (q_{m+1} - q_m) \cdot p_m} e^{-i H(q_m,p_m) \epsilon}
      \end{split}
    \end{equation*}
    The amplitude is then rewritten as:
    \begin{equation*}
      \begin{split}
        \braket{q_\text{f},t_\text{f} | q_\text{i},t_\text{i}}
        & = \frac{1}{(2\pi)^n} \int_{\R^{2n(N+1)}} \prod_{m = 0}^{N-1} \dd q_m \dd p_m\, e^{i \left[ (q_{m+1} - q_m) \cdot p_m - H(q_m,p_m) \epsilon \right]} \\
        & = \frac{1}{(2\pi)^n} \int_{\R^{2n(N+1)}} \prod_{m = 0}^{N-1} \dd q_m \dd p_m \exp \sum_{k = 1}^{N-1} i \epsilon \left[ \frac{q_{k+1} - q_k}{\epsilon} \cdot p_k - H(q_k,p_k) \right] \\
      \end{split}
    \end{equation*}
    Consider now the $ \epsilon \rightarrow 0 $ limit, i.e. $ N \rightarrow \infty $: the integration is then performed on an infinite number of variables, hence becoming a functional integral. The integration is carried on the set of functions $ \{q(t) : \R^n \rightarrow \R^n : q(t_\text{i}) = q_\text{i} \land q(t_\text{f}) = q_\text{f}\} $; moreover, note that, in the $ \epsilon \rightarrow 0 $ limit, the sum becomes a Riemann integral:
    \begin{equation*}
      \braket{q_\text{f},t_\text{f} | q_\text{i},t_\text{i}} = \frac{1}{(2\pi)^n} \int_{q(t_\text{i}) = q_\text{i}}^{q(t_\text{f}) = q_\text{f}} \dd q \dd p \exp i \int_{t_\text{i}}^{t_\text{f}} \dd t \left[ \dot{q}(t) \cdot p(t) - H(q(t),p(t)) \right]
    \end{equation*}
    which is precisely the thesis.
  \end{proof}
\end{proofbox}

\begin{proposition}{Lagrangian path integral}{}
  For a $ p $-quadratic Hamiltonian, the path integral can be written as:
  \begin{equation}
    \braket{q_\text{f},t_\text{f} | q_\text{i},t_\text{i}} = \int_{q(t_\text{i}) = q_\text{i}}^{q(t_\text{f}) = q_\text{f}} [\dd q]\, e^{\frac{i}{\hbar} \act[q,\dot{q}]}
    \label{eq:qm-lag-path-int}
  \end{equation}
  where $ \act[q,\dot{q}] $ is the action functional of the system and $ [\dd q] $ is a suitably-normalized integration measure.
\end{proposition}

\begin{proofbox}
  \begin{proof}
    Consider a system with a $ p $-quadratic Hamiltonian $ H(q,p) = \frac{p^2}{2} + V(q) $ (however, the thesis holds for a general quadratic form of $ p $), so that the previous proof can be rewritten with a Gaussian integral (\eref{eq:gauss-int}):
    \begin{equation*}
      \begin{split}
        \braket{q_{m+1} , t_{m+1} | q_m , t_m}
        & = \frac{1}{(2\pi)^n} e^{-i \epsilon V(q_m)} \int_{\R^n} \dd p_m\, e^{i (q_{m+1} - q_m) \cdot p_m - \frac{i \epsilon}{2} p_m^2} \\
        & = \frac{1}{(2\pi)^n} e^{-i \epsilon V(q_m)} \prod_{a = 1}^n \int_{\R} \dd p_{m,a} e^{i (q_{m+1,a} - q_{m,a}) p_{m,a} - \frac{i \epsilon}{2} p_{m,a}^2} \\
        & = \frac{1}{(2\pi)^n} e^{-i \epsilon V(q_m)} \prod_{a = 1}^n \sqrt{\frac{2\pi }{i\epsilon}} e^{- \frac{1}{2i \epsilon} (q_{m+1,a} - q_{m,a})^2} \\
        & = \left( \frac{1}{2\pi i \epsilon} \right)^{n/2} e^{i \epsilon \left[ \frac{1}{2} \left( \frac{q_{m+1} - q_m}{\epsilon} \right)^2 - V(q_m) \right]}
      \end{split}
    \end{equation*}
    Therefore:
    \begin{equation*}
      \braket{q_\text{f},t_\text{f} | q_\text{i},t_\text{i}} = \left( \frac{1}{2\pi i \epsilon} \right)^{nN/2} \int_{\R^{n(N-1)}} \prod_{m = 0}^{N-1} \dd q_m \exp \sum_{k = 0}^{N - 1} i \epsilon \left[ \frac{1}{2} \left( \frac{q_{k+1} - q_k}{\epsilon} \right)^2 - V(q_k) \right]
    \end{equation*}
    In the $ \epsilon \rightarrow 0 $ limit, with the same considerations as above and absorbing the normalization in the integration measure, this becomes:
    \begin{equation*}
      \braket{q_\text{f},t_\text{f} | q_\text{i},t_\text{i}} = \int_{q(t_\text{i}) = q_\text{i}}^{q(t_\text{f}) = q_\text{f}} [dq] \exp \left[ i \int_{t_\text{i}}^{t_\text{f}} \dd t\, L(q(t),\dot q(t)) \right] \eqdef \int_{q(t_\text{i}) = q_\text{i}}^{q(t_\text{f}) = q_\text{f}} [\dd q]\, e^{i \act[q,\dot q]}
    \end{equation*}
    Reinstating explicitly $ \hbar $ yields the thesis.
  \end{proof}
\end{proofbox}

This formulation elucidates the classical limit $ \hbar \rightarrow 0 $: while quantistically a particle explores all possible trajectories, which are then weighted by a phase $ e^{i\act} $, classically only trajectories which extremize the action are relevant, as those which are far from the extrema get extremely-oscillating phases under small deformations. Thus, the least-action principle has been recovered. \\
The path integral is not only useful for computing amplitudes, but it is also used to express matrix elemets on the coordinate-eigenbasis.

\begin{lemma}{Time-ordered products}{}
  Given $ \{t_k\}_{k = 1, \dots, N} \subset [t_\text{i} , t_\text{f}] $ and operators $ \{\hat{\mathcal{O}}_k(\hat{q}(t))\}_{k = 1, \dots, N} \equiv \{\hat{\mathcal{O}}_k(t)\}_{k = 1, \dots, N} $, $ N \in \N $, then:
  \begin{equation*}
    \int_{q(t_\text{i}) = q_\text{i}}^{q(t_\text{f}) = q_\text{f}} [\dd q]\, \mathcal{O}_1(t_1) \dots \mathcal{O}_N(t_N) e^{i \act} = \braket{q_\text{f},t_\text{f} | \tempord\{\hat{\mathcal{O}}_1(t_1) \dots \hat{\mathcal{O}}_N(t_N)\} | q_\text{i},t_\text{i}}
  \end{equation*}
  where $ \tempord $ is the time-ordered product, which is defined as.
  \begin{equation}
    \tempord \{f(x) g(y)\} \defeq
    \begin{cases}
      f(x) g(y) & x^0 > y^0 \\
      g(y) f(x) & y^0 > x^0
    \end{cases}
  \end{equation}
\end{lemma}

\begin{proofbox}
  \begin{proof}
    First, note that:
    \begin{equation*}
      \int_{q(t_\text{i}) = q_\text{i}}^{q(t_\text{f}) = q_\text{f}} [\dd q] = \int_{\R^n} \dd\bar{q} \int_{q(t_\text{i}) = q_\text{i}}^{q(t) = \bar{q}} [\dd q] \int_{q(t) = \bar{q}}^{q(t_\text{f}) = q_\text{f}} [\dd q]
    \end{equation*}
    so that (thanks to the additivity of the action integral):
    \begin{equation*}
      \begin{split}
        \int_{q(t_\text{i}) = q_\text{i}}^{q(t_\text{f}) = q_\text{f}} [\dd q]\, \mathcal{O}(t) e^{i \act}
        & = \int_{\R^n} \dd\bar{q} \underbrace{\int_{q(t_\text{i}) = q_\text{i}}^{q(t) = \bar{q}} [\dd q] \exp \left[ i \int_{t_\text{i}}^t \dd t\, L \right]}_{\braket{\bar{q},t | q_\text{i},t_\text{i}}} \mathcal{O}_k(\bar{q}) \underbrace{\int_{q(t) = \bar{q}}^{q(t_\text{f}) = q_\text{f}} [\dd q] \left[ i \int_t^{t_\text{f}} \dd t\, L \right]}_{\braket{q_\text{f},t_\text{f} | \bar{q},t}} \\
        & = \int_{\R^n} \dd\bar{q}\, \braket{q_\text{f},t_\text{f} | \bar{q},t} \mathcal{O}_k(\bar{q}) \braket{\bar{q},t | q_\text{i},t_\text{i}} = \int_\R \dd\bar{q}\, \braket{q_\text{f},t_\text{f} | \hat{\mathcal{O}}_k(t) | \bar{q},t} \braket{\bar{q},t | q_\text{i},t_\text{i}} \\
        & = \braket{q_\text{f},t_\text{f} | \hat{\mathcal{O}}_k(t) | q_\text{i},t_\text{i}}
      \end{split}
    \end{equation*}
    This links the matrix element of the operator $ \hat{\mathcal{O}}_k(t) $ to the path integral of the function $ \mathcal{O}_k(t) $. Consider now that, being $ \mathcal{O}_k(t) \in \C $, i.e. commuting:
    \begin{equation*}
      \int_{q(t_\text{i}) = q_\text{i}}^{q(t_\text{f}) = q_\text{f}} [\dd q]\, \mathcal{O}_1(t_1) \dots \mathcal{O}_N(t_N) e^{i \act} = \int_{q(t_\text{i}) = q_\text{i}}^{q(t_\text{f}) = q_\text{f}} [\dd q]\, \mathcal{O}_{\pi(1)}(t_{\pi(1)}) \dots \mathcal{O}_{\pi(N)}(t_{\pi(N)}) e^{i \act}
    \end{equation*}
    where $ \pi \in S^N : t_{\pi(1)} \ge \dots \ge t_{\pi(N)} $ is a time-ordering permutation. Applying the above result:
    \begin{equation*}
      \int_{q(t_\text{i}) = q_\text{i}}^{q(t_\text{f}) = q_\text{f}} [\dd q]\, \mathcal{O}_1(t_1) \dots \mathcal{O}_N(t_N) e^{i \act} = \braket{q_\text{f},t_\text{f} | \hat{\mathcal{O}}_{\pi(1)}(t_{\pi(1)}) \dots \hat{\mathcal{O}}_{\pi(N)}(t_{\pi(N)}) | q_\text{i},t_\text{i}}
    \end{equation*}
    But $ \,\hat{\mathcal{O}}_{\pi(1)}(t_{\pi(1)}) \dots \hat{\mathcal{O}}_{\pi(N)}(t_{\pi(N)}) \eqdef \tempord \{\hat{\mathcal{O}}_1(t_1) \dots \hat{\mathcal{O}}_N(t_N)\} $, thus completing the proof.
  \end{proof}
\end{proofbox}

\subsection{Path integral in Quantum Field Theory}

To generalize the path-integral formalism to QFT, consider the index $ a $ as running over points $ \ve{x} $ in space and over a spin/helicity (i.e. species/helicity) index $ m $:
\begin{equation*}
  \hat{q}_a(t) \longrightarrow \hat{q}_m(t,\ve{x})
  \qquad \qquad
  \ket{q,t} \longrightarrow \ket{q(\ve{x}),t}
\end{equation*}
The generalizations of \eeref{eq:qm-ham-path-int}{eq:qm-lag-path-int} is then straightforward:
\begin{equation}
  \begin{split}
    \braket{q_\text{f}(\ve{x}),t_\text{f} | q_\text{i}(\ve{x}),t_\text{i}}
    & = \int_{q(t_\text{i},\ve{x}) = q_\text{i}(\ve{x})}^{q(t_\text{f},\ve{x}) = q_\text{f}(\ve{x})} \prod_m \dd q(\ve{x}) \frac{\dd p(\ve{x})}{2\pi} \times \\
    & \quad \times \exp i \int_{t_\text{i}}^{t_\text{f}} \dd t \int_{\R^3} \dd^3x \left[ \sum_m \dot{q}_m(t,\ve{x}) p_m(t,\ve{x}) - \ham[q(t,\ve{x}),p(t,\ve{x})] \right]
  \end{split}
  \label{eq:ham-path-int-q}
\end{equation}
\begin{equation}
  \braket{q_\text{f}(\ve{x}),t_\text{f} | q_\text{i}(\ve{x}),t_\text{i}} = \int_{q(t_\text{i},\ve{x}) = q_\text{i}(\ve{x})}^{q(t_\text{f},\ve{x}) = q_\text{f}(\ve{x})} \mathcal{D}q \exp i \int_{t_\text{i}}^{t_\text{f}} \dd t \int_{\R^3} \dd^3 x\, \lag[q(t,\ve{x}),\dot{q}(t,\ve{x})]
  \label{eq:lag-path-int-q}
\end{equation}
It is best to express the path integral not for coordinate eigenstates, but for experimentally-observed states: these are states at $ t_\text{i} \rightarrow -\infty , t_\text{f} \rightarrow +\infty $ with a definite number of particles (of various types).

\begin{theorem}{(Bosonic) Path integral}{}
  The amplitude for the (bosonic) process $ \ket{\alpha_\text{in}} \rightarrow \ket{\beta_\text{out}} $ reads:
  \begin{equation}
    \braket{\beta_\text{out} | \alpha_\text{in}} = \int \mathcal{D}q \exp \left[ i \int_{\R^4} \dd t \dd^3 x \, \lag[q(t,\ve{x}),\dot{q}(t,\ve{x})] \right] \braket{\beta_\text{out} | q(\ve{x}), +\infty} \braket{q(\ve{x}), -\infty | \alpha_\text{in}}
    \label{eq:bos-path-int}
  \end{equation}
  where the functional integral is now unconstrained.
\end{theorem}

\begin{proofbox}
  \begin{proof}
    Using \eref{eq:lag-path-int-q}:
    \begin{equation*}
      \begin{split}
        \braket{\beta_\text{out} | \alpha_\text{in}}
        & = \int \dd q_\text{i}(\ve{x}) \dd q_\text{f}(\ve{x}) \braket{\beta_\text{out} | q_\text{f}(\ve{x}), +\infty} \braket{q_\text{f}(\ve{x}),+\infty | q_\text{i}(\ve{x}),-\infty} \braket{q_\text{i}(\ve{x}), -\infty} \\
        & = \int \dd q_\text{i}(\ve{x}) \dd q_\text{f}(\ve{x}) \int_{q(-\infty,\ve{x}) = q_\text{i}(\ve{x})}^{q(+\infty,\ve{x}) = q_\text{f}(\ve{x})} \mathcal{D}q \exp \left[ i \int_{\R^4} \dd t \dd^3 x\, \lag[q(t,\ve{x}),\dot{q}(t,\ve{x})] \right] \times \\
        & \qquad \qquad \qquad \qquad \qquad \qquad \qquad \qquad \qquad \times \braket{\beta_\text{out} | q_\text{f}(\ve{x}), +\infty} \braket{\alpha_\text{in} | q_\text{i}(\ve{x}), -\infty}
      \end{split}
    \end{equation*}
    This is equivalent to having an unconstrained functional integral, hence the thesis.
  \end{proof}
\end{proofbox}

This result is easily translated into Hamiltonian form. The expression for the wavefunctions which appear in \eref{eq:bos-path-int} depends on the nature of the quantum field $ q_m(t,\ve{x}) $ considered (which, in this case, is constrained to be bosonic as of \eref{eq:qm-path-int-comm}).

\section{Functional quantization of fields}

\subsection{Scalar fields}

\subsection{Electromagnetic field (?)}

\subsection{Spinor fields}










