\selectlanguage{english}

\section{Scalar fields}

As for the quantization of a classical system in Quantum Mechanics, the quantization of a scalar field theory is performed promoting $ \phi(t,\ve{x}) $ and $ \Pi(t,\ve{x}) $ to hermitian operators in the Heisenberg picture and imposing the canonical equal-time commutation relation:
\begin{equation}
  [\phi(t,\ve{x}) , \Pi(t,\ve{y})] = i \delta^{(3)}(\ve{x} - \ve{y})
  \label{eq:2.1}
\end{equation}
while of course $ [\phi(t,\ve{x}) , \phi(t,\ve{y})] = [\Pi(t,\ve{x}) , \Pi(t,\ve{y})] = 0 $.

\subsection{Real scalar fields}

A real scalar field is promoted to a real hermitian operator. In particular, by Eq. \ref{eq:1.70}:
\begin{equation}
  \phi(x) = \int \frac{\dd^3p}{(2\pi)^3 \sqrt{2E_\ve{p}}} \left[ a_\ve{p} e^{-i p_\mu x^\mu} + a_\ve{p}^\dagger e^{i p_\mu x^\mu} \right]_{p^0 = E_\ve{p}}
  \label{eq:2.2}
\end{equation}
In terms of creation and annihilation operators, the commutator Eq. \ref{eq:2.1} reads:
\begin{equation}
  [a_\ve{p} , a_\ve{p}^\dagger] = (2\pi)^3 \delta^{(3)}(\ve{p} - \ve{q})
  \label{eq:2.3}
\end{equation}
while $ [a_\ve{p} , a_\ve{q}] = [a_\ve{p}^\dagger , a_\ve{q}^\dagger] = 0 $. These can be regarded as the creation and annihilation operators of a collection of harmonic oscillators, one for each value of the momentum $ \ve{p} $: the \textit{Fock space} of the real scalar field can thus be constructed analogously to the Hilbert space of the harmonic oscillator.\\
Defining the \textit{vacuum state} $ \ket{0} : a_\ve{p} \ket{0} = 0 \,\,\forall \ve{p} $, suitably normalized as $ \braket{0 | 0} = 1 $, the generic state of the Fock space is:
\begin{equation}
  \ket{\ve{p}_1 , \dots , \ve{p}_n} = \sqrt{2E_{\ve{p}_1}} \dots \sqrt{2E_{\ve{p}_n}}\, a_{\ve{p}_1}^\dagger \dots a_{\ve{p}_n}^\dagger \ket{0}
  \label{eq:2.4}
\end{equation}

\begin{proposition}{Normalization}{}
  For one-particle states:
  \begin{equation}
    \braket{\ve{p}_1 | \ve{p}_2} = 2E_{\ve{p}_1} (2\pi)^3 \delta^{(3)}(\ve{p}_1 - \ve{p}_2)
    \label{eq:2.5}
  \end{equation}

  \begin{proof}
    By Eq. \ref{eq:2.3}:
    \begin{equation*}
      \braket{\ve{p}_1 | \ve{p}_2} = \sqrt{2E_{\ve{p}_1}} \sqrt{2E_{\ve{p}_2}} \braket{0 | a_{\ve{p}_1} a_{\ve{p}_2}^\dagger | 0}
      = \sqrt{2E_{\ve{p}_1}} \sqrt{2E_{\ve{p}_2}} \braket{0 | [a_{\ve{p}_1} , a_{\ve{p}_2}^\dagger] | 0} = 2E_{\ve{p}_1} (2\pi)^3 \delta^{(3)}(\ve{p}_1 - \ve{p}_2)
    \end{equation*}
  \end{proof}
\end{proposition}

\begin{lemma}{}{}
  The combination $ E_\ve{p} \delta^{(3)}(\ve{p} - \ve{q}) $ is Lorentz invariant.

  \tcblower

  \begin{proof}
    Recall that:
    \begin{equation}
      \delta(f(x) - f(x_0)) = \frac{1}{\abs{f'(x_0)}} \delta(x - x_0)
      \label{eq:2.6}
    \end{equation}
    A Lorentz boost along $ \ve{e}_i $ yields $ p'_i = \gamma (p_i + \beta E) $ and $ E' = \gamma (E + \beta p_i) $, so:
    \begin{equation*}
        \delta^{(3)}(\ve{p} - \ve{q})
        = \delta^{(3)}(\ve{p}' - \ve{q}') \frac{\dd p'_i}{\dd p_i}
        = \delta^{(3)}(\ve{p}' - \ve{q}') \gamma \left( 1 + \beta \frac{\dd E}{\dd p_i} \right)
    \end{equation*}
    Note that $ p^2 = E^2 - \ve{p}^2 $ is a Lorentz invariant, thus $ E \dd E = p_i \dd p_i $, so:
    \begin{equation*}
      \delta^{(3)}(\ve{p} - \ve{q})
      = \delta^{(3)}(\ve{p}' - \ve{q}') \frac{\gamma (E + \beta p_i)}{E}
      = \delta^{(3)}(\ve{p}' - \ve{q}') \frac{E'}{E}
    \end{equation*}
  \end{proof}
\end{lemma}

This explains the choice of normalization.

\begin{proposition}{KG Hamiltonian}{}
  The Hamiltonian of a real scalar field can be written as:
  \begin{equation}
    H = \int \frac{\dd^3 p}{(2\pi)^3} E_\ve{p} \left( a_\ve{p}^\dagger a_\ve{p} + \frac{1}{2} [a_\ve{p} , a_\ve{p}^\dagger] \right)
    \label{eq:2.7}
  \end{equation}

  \tcblower

  \begin{proof}
    First of all, from Eq. \ref{eq:2.2}:
    \begin{equation*}
      \begin{split}
        \Pi(t,\ve{x}) = \pa_0 \phi(t,\ve{x})
        &= -i \int \frac{\dd^3 p}{(2\pi)^3} \sqrt{\frac{E_\ve{p}}{2}} \left[ a_\ve{p} e^{-i p_\mu x^\mu} - a_\ve{p}^\dagger e^{i p_\mu x^\mu} \right]_{p^0 = E_\ve{p}} \\
        &= -i \int \frac{\dd^3 p}{(2\pi)^3} \sqrt{\frac{E_\ve{p}}{2}} \left[ a_\ve{p} e^{-i E_\ve{p} t} - a_{-\ve{p}}^\dagger e^{i E_\ve{p} t} \right] e^{i \ve{p} \cdot \ve{x}}
      \end{split}
    \end{equation*}
    \begin{equation*}
      \bs{\nabla} \phi(t,\ve{x}) = i \int \frac{\dd^3 p}{(2\pi)^3} \frac{\ve{p}}{\sqrt{2E_\ve{p}}} \left[ a_\ve{p} e^{-i E_\ve{p} t} + a_{-\ve{p}}^\dagger e^{i E_\ve{p} t} \right] e^{i \ve{p} \cdot \ve{x}}
    \end{equation*}
    Inserting these expressions in Eq. \ref{eq:1.71}:
    \begin{equation*}
      \begin{split}
        \mathcal{H}
        &= \frac{1}{2} \int \frac{\dd^3 p}{(2\pi)^3} \int \frac{\dd^3 q}{(2\pi)^3} \bigg\{ - \frac{\sqrt{E_\ve{p} E_\ve{q}}}{2} \left[ a_\ve{p} e^{-i E_\ve{p} t} - a_{-\ve{p}}^\dagger e^{i E_\ve{p} t} \right] \left[ a_\ve{q} e^{-i E_\ve{q} t} - a_{-\ve{q}}^\dagger e^{i E_{\ve{q}} t} \right] + \\
        & \qquad + \frac{- \ve{p} \cdot \ve{q} + m^2}{2\sqrt{E_\ve{p} E_\ve{q}}} \left[ a_\ve{p} e^{-i E_\ve{p} t} + a_{-\ve{p}}^\dagger e^{i E_\ve{p} t} \right] \left[ a_\ve{q} e^{-i E_\ve{q} t} + a_{-\ve{q}}^\dagger e^{i E_\ve{q} t} \right] \bigg\} e^{i (\ve{p} + \ve{q}) \cdot \ve{x}}
      \end{split}
    \end{equation*}
    Recall the identity:
    \begin{equation}
      \int \frac{\dd^3x}{(2\pi)^3} e^{i (\ve{p} - \ve{q}) \cdot \ve{x}} = \delta^{(3)}(\ve{p} - \ve{q})
      \label{eq:2.8}
    \end{equation}
    Then (using $ E_{-\ve{p}} = E_\ve{p} $):
    \begin{equation*}
      \begin{split}
        H
        &= \int \dd^3x\, \mathcal{H} \\
        &= \frac{1}{2} \int \frac{\dd^3 p}{(2\pi)^3} \int \dd^3 q \bigg\{ - \frac{\sqrt{E_\ve{p} E_\ve{q}}}{2} \left[ a_\ve{p} e^{-i E_\ve{p} t} - a_{-\ve{p}}^\dagger e^{i E_\ve{p} t} \right] \left[ a_\ve{q} e^{-i E_\ve{q} t} - a_{-\ve{q}}^\dagger e^{i E_\ve{q} t} \right] + \\
        & \qquad + \frac{- \ve{p} \cdot \ve{q} + m^2}{2\sqrt{E_\ve{p} E_\ve{q}}} \left[ a_\ve{p} e^{-i E_\ve{p} t} + a_{-\ve{p}}^\dagger e^{i E_\ve{p} t} \right] \left[ a_\ve{q} e^{-i E_\ve{q} t} + a_{-\ve{q}}^\dagger e^{i E_\ve{q} t} \right] \bigg\} \delta^{(3)}(\ve{p} + \ve{q}) \\
        &= \frac{1}{2} \int \frac{\dd^3 p}{(2\pi)^3} \bigg\{ - \frac{E_\ve{p}}{2} \left[ a_\ve{p} e^{-i E_\ve{p} t} - a_{-\ve{p}}^\dagger e^{i E_\ve{p} t} \right] \left[ a_{-\ve{p}} e^{-i E_\ve{p} t} - a_\ve{p}^\dagger e^{i E_\ve{p} t} \right] + \\
        & \qquad + \frac{\ve{p}^2 + m^2}{2E_\ve{p}} \left[ a_\ve{p} e^{-i E_\ve{p} t} + a_{-\ve{p}}^\dagger e^{i E_\ve{p} t} \right] \left[ a_{-\ve{p}} e^{-i E_\ve{p} t} + a_\ve{p}^\dagger e^{i E_\ve{p} t} \right] \bigg\} \\
        &= \frac{1}{2} \int \frac{\dd^3p}{(2\pi)^3} \frac{E_\ve{p}}{2} \bigg\{ - \left[ a_\ve{p} a_{-\ve{p}} e^{-2i E_\ve{p} t} - a_{-\ve{p}}^\dagger a_{-\ve{p}} - a_\ve{p} a_\ve{p}^\dagger + a_{-\ve{p}}^\dagger a_\ve{p}^\dagger e^{2i E_\ve{p} t} \right] \\
        & \qquad + \left[ a_\ve{p} a_{-\ve{p}} e^{-2i E_\ve{p} t} + a_{-\ve{p}}^\dagger a_{-\ve{p}} + a_\ve{p} a_\ve{p}^\dagger + a_{-\ve{p}}^\dagger a_\ve{p}^\dagger e^{2i E_\ve{p} t} \right] \bigg\} \\
        &= \frac{1}{2} \int \frac{\dd^3p}{(2\pi)^3} E_\ve{p} \left[ a_{-\ve{p}}^\dagger a_{-\ve{p}} + a_\ve{p} a_\ve{p}^\dagger \right] = \frac{1}{2} \int \frac{\dd^3p}{(2\pi)^3} E_\ve{p} \left( a_\ve{p}^\dagger a_\ve{p} + a_\ve{p} a_\ve{p}^\dagger \right)
      \end{split}
    \end{equation*}
  \end{proof}
\end{proposition}

The second term in the Hamiltonian Eq. \ref{eq:2.7} is the sum of the zero-point energy of all oscillators and is proportional to $ (2\pi)^3 \delta^{(3)}(0) \rightarrow V $, thus:
\begin{equation*}
  E_\text{vac} = \frac{V}{2} \int \frac{\dd^3p}{(2\pi)^3} E_\ve{p}
\end{equation*}
This energy shows two divergences: the one coming from the infinite-volume limit (i.e. small momentum), regularized introducing an \textit{infrared cutoff} in the form of a finite volume, and the one from the ultra-relativistic limit (i.e. large momentum), regularized introducing an \textit{ultraviolet cutoff} in the form of a maximum momentum $ \Lambda $. These divergences are retained in the expression for $ E_\text{vac} $, as $ E_\text{vac} \sim V $ and $ E_\text{vac} \sim \Lambda^4 $, but can be ignored (when ignoring gravity) since experiments are only sensitive to energy differences.\\
Discarding the zero-point energy, the Hamiltonian becomes:
\begin{equation}
  H = \int \frac{\dd^3p}{(2\pi)^3} E_\ve{p} a_\ve{p}^\dagger a_\ve{p} \equiv \mathcal{N} \frac{1}{2} \int \frac{\dd^3p}{(2\pi)^3} E_\ve{p} \left( a_\ve{p}^\dagger a_\ve{p} + a_\ve{p} a_\ve{p}^\dagger \right)
  \label{eq:2.9}
\end{equation}
where the \textit{normal ordering operator} $ \mathcal{N} $ was introduced, which acts by moving all creation operators to the left and all annihilation operators to the right (ex.: $ \mathcal{N} a_\ve{p} a_\ve{p}^\dagger = a_\ve{p}^\dagger a_\ve{p} $). It is now straightforward to compute the energy of a generic state in the Fock space, as $ a_\ve{p}^\dagger a_\ve{p} $ is just a number operator:
\begin{equation}
  H \ket{\ve{p}_1 , \dots , \ve{p}_n} = \left( E_{\ve{p}_1} + \dots + E_{\ve{p}_n} \right) \ket{\ve{p}_1 , \dots , \ve{p}_n}
  \label{eq:2.10}
\end{equation}
Computing the spatial momentum from Eq. \ref{eq:1.67} as $ P^i = \mathcal{N} \int \dd^3x\, \theta^{0i} = \int \dd^3x\, \mathcal{N} \pa_0 \phi \pa^i \phi $:
\begin{equation}
  P^i = \int \frac{\dd^3p}{(2\pi)^3} p^i a_\ve{p}^\dagger a_\ve{p}
  \label{eq:2.11}
\end{equation}
Therefore, the state $ a_\ve{p}^\dagger \ket{0} $ can correctly be interpreted as a one-particle state with momentum $ \ve{p} $, mass $ m $ and energy $ E_\ve{p} = \sqrt{\ve{p}^2 + m^2} $. The generic state in the Fock space is a multiparticle state, with total energy and momentum the sum of the individual energies and momenta.\\
Finally, note that creation operators commute between themselves, hence multiparticle states are symmetric under exchange of pairs of particles, i.e. they obey the Bose-Einstein statistics: this agrees with the fact that quanta of a scalar field have no intrinsic spin. i.e. are spin-0 particles.

\subsection{Complex scalar fields}

When considering a complex scalar field, Eq. \ref{eq:1.78} becomes:
\begin{equation}
  \phi(x) = \int \frac{\dd^3p}{(2\pi)^3 \sqrt{2E_\ve{p}}} \left[ a_\ve{p} e^{-i p_\mu x^\mu} + b_\ve{p}^\dagger e^{i p_\mu x^\mu} \right]_{p^0 = E_\ve{p}}
  \label{eq:2.12}
\end{equation}
\begin{equation}
  \phi^\dagger(x) = \int \frac{\dd^3p}{(2\pi)^3 \sqrt{2E_\ve{p}}} \left[ a_\ve{p}^\dagger e^{i p_\mu x^\mu} + b_\ve{p} e^{-i p_\mu x^\mu} \right]_{p^0 = E_\ve{p}}
  \label{eq:2.13}
\end{equation}
Now there are two independent sets of creation/annihilation operators, which obey the canonical commutation relation:
\begin{equation}
  [a_\ve{p} , a_\ve{q}^\dagger] = [b_\ve{p} , b_\ve{q}^\dagger] = (2\pi)^3 \delta^{(3)}(\ve{p} - \ve{q})
  \label{eq:2.14}
\end{equation}
with all other commutators vanishing. The Fock space is constructed by defining a vacuum state $ \ket{0} : a_\ve{p} \ket{0} = b_\ve{p} \ket{0} = 0 $ and then acting repeatedly with both creation operators. With normal ordering, one finds:
\begin{equation}
  H = \int \frac{\dd^3p}{(2\pi)^3} E_\ve{p} \left( a_\ve{p}^\dagger a_\ve{p} + b_\ve{p}^\dagger b_\ve{p} \right)
  \label{eq:2.15}
\end{equation}
\begin{equation}
  P^i = \int \frac{\dd^3p}{(2\pi)^3} p^i \left( a_\ve{p}^\dagger a_\ve{p} + b_\ve{p}^\dagger b_\ve{p} \right)
  \label{eq:2.16}
\end{equation}
The quanta of a complex scalar field are given by two different species of particles with the same mass.

\begin{proposition}{$ \Un{1} $ charge}{}
  The $ \Un{1} $ charge of the quantized complex scalar field is:
  \begin{equation}
    Q_{\Un{1}} = \int \frac{\dd^3p}{(2\pi)^3} \left( a_\ve{p}^\dagger a_\ve{p} - b_\ve{p}^\dagger b_\ve{p} \right)
    \label{eq:2.17}
  \end{equation}

  \tcblower

  \begin{proof}
    By Eq. \ref{eq:1.79}:
    \begin{equation*}
      \begin{split}
        Q_{\Un{1}}
        &= i \int \dd^3x\, \phi^\dagger \smlra{\pa_0} \phi = i \int \dd^3x \int \frac{\dd^3q}{(2\pi)^3 \sqrt{2E_\ve{q}}} \int \frac{\dd^3p}{(2\pi)^3 \sqrt{2E_\ve{p}}} \,\times \\
        & \quad \times \bigg\{ \left[ a_\ve{q}^\dagger e^{i q_\mu x^\mu} + b_\ve{q} e^{-i q_\mu x^\mu} \right] \pa_0 \left( a_\ve{p} e^{-i p_\mu x^\mu} + b_\ve{p}^\dagger e^{i p_\mu x^\mu} \right) + \\
        & \quad - \pa_0 \left( a_\ve{q}^\dagger e^{i q_\mu x^\mu} + b_\ve{q} e^{-i q_\mu x^\mu} \right) \left[ a_\ve{p} e^{-i p_\mu x^\mu} + b_\ve{p}^\dagger e^{i p_\mu x^\mu} \right] \bigg\}
      \end{split}
    \end{equation*}
    \begin{equation*}
      \begin{split}
        Q_{\Un{1}}
        &= \int \dd^3x \int \frac{\dd^3q}{(2\pi)^3 \sqrt{2E_\ve{q}}} \int \frac{\dd^3p}{(2\pi)^3 \sqrt{2E_\ve{p}}} \,\times \\
        & \quad \times \bigg\{ E_\ve{p} \left[ a_\ve{q}^\dagger e^{i q_\mu x^\mu} + b_\ve{q} e^{-i q_\mu x^\mu} \right] \left[ a_\ve{p} e^{-i p_\mu x^\mu} - b_\ve{p}^\dagger e^{i p_\mu x^\mu} \right] + \\
        & \quad + E_\ve{q} \left[ a_\ve{q}^\dagger e^{i q_\mu x^\mu} - b_\ve{q} e^{-i q_\mu x^\mu} \right] \left[ a_\ve{p} e^{-i p_\mu x^\mu} + b_\ve{p}^\dagger e^{i p_\mu x^\mu} \right] \bigg\} \\
        &= \int \dd^3x \int \frac{\dd^3q}{(2\pi)^3 \sqrt{2E_\ve{q}}} \int \frac{\dd^3p}{(2\pi)^3 \sqrt{2E_\ve{p}}} \,\times \\
        & \quad \times \bigg\{ E_\ve{p} \left[ a_\ve{q}^\dagger e^{i E_\ve{q} t} + b_{-\ve{q}} e^{-i E_\ve{q} t} \right] \left[ a_\ve{p} e^{-i E_\ve{p} t} - b_{-\ve{p}}^\dagger e^{i E_\ve{p} t} \right] + \\
        & \quad + E_\ve{q} \left[ a_\ve{q}^\dagger e^{i E_\ve{q} t} - b_{-\ve{q}} e^{-i E_\ve{q} t} \right] \left[ a_\ve{p} e^{-i E_\ve{p} t} + b_{-\ve{p}}^\dagger e^{i E_\ve{p} t} \right] \bigg\} e^{i (\ve{p} - \ve{q}) \cdot \ve{x}} \\
        &= \frac{1}{2} \int \frac{\dd^3p}{(2\pi)^3} \bigg\{ \left[ a_\ve{p}^\dagger e^{i E_\ve{p} t} + b_{-\ve{p}} e^{-i E_\ve{p} t} \right] \left[ a_\ve{p} e^{-i E_\ve{p} t} - b_{-\ve{p}}^\dagger e^{i E_\ve{p} t} \right] + \\
        & \quad + \left[ a_\ve{p}^\dagger e^{i E_\ve{p} t} - b_{-\ve{p}} e^{-i E_\ve{p} t} \right] \left[ a_\ve{p} e^{-i E_\ve{p} t} + b_{-\ve{p}}^\dagger e^{i E_\ve{p} t} \right] \bigg\} \\
        &= \int \frac{\dd^3p}{(2\pi)^3} \left[ a_\ve{p}^\dagger a_\ve{p} - b_{-\ve{p}} b_{-\ve{p}}^\dagger \right] = \int \frac{\dd^3p}{(2\pi)^3} \left( a_\ve{p}^\dagger a_\ve{p} - b_\ve{p} b_\ve{p}^\dagger \right)
      \end{split}
    \end{equation*}
    Applying normal ordering yields the thesis.
  \end{proof}
\end{proposition}

While normal ordering was justified when considering the Hamiltonian on the grounds that the vacuum energy is unobservable, a charged vacuum would have observable effects; however when promoting $ \phi $ to a quantum operator, the expression $ \phi^\dagger \smlra{\pa_0} \phi $ presents an ordering ambiguity (ex.: $ \phi^\dagger (\pa_0 \phi) $ or $ (\pa_0 \phi) \phi^\dagger $), which is removed requiring the charge of the vacuum to vanish.\\
Being $ a_\ve{p}^\dagger a_\ve{p} $ and $ b_\ve{p}^\dagger b_\ve{p} $ number operators, the $ \Un{1} $ charge is equal to the number of quanta created by $ a_\ve{p}^\dagger $ minus the number of quanta created by $ b_\ve{p}^\dagger $, integrated over all momenta: in particular, $ a_\ve{p}^\dagger \ket{0} $ and $ b_\ve{p}^\dagger \ket{0} $ are both spin-zero particles of mass $ m $ and momentum $ \ve{p} $, but they respectively have charges $ Q_{\Un{1}} = +1 $ and $ Q_{\Un{1}} = -1 $. This allows to properly interpret the negative-energy solutions of the KG equations: they are positive-energy particles with opposite $ \Un{1} $ charge and are called \textit{antiparticles}.\\
For a real scalar field, the reality condition reads $ a_\ve{p} = b_\ve{p} $, thus it describes a field whose particle is its own antiparticle, and it is symmetric under any $ \Un{1} $ symmetry.

\section{Spinor fields}

A principle of QFT is the \textit{spin-statistic theorem}: integer-spin fields are to be quantized imposing equal-time commutation relations, while half-integer-spin with equal-time anticommutation relations.

\subsection{Dirac fields}

From the Dirac Lagrangian Eq. \ref{eq:1.87}, the conjugate momentum to the Dirac field $ \Psi $ is computed as:
\begin{equation}
  \Pi_\Psi = i \bar{\Psi} \gamma^0 = i \Psi^\dagger
  \label{eq:2.18}
\end{equation}
Imposing the canonical anticommutation relation, according to the spin-statistic theorem:
\begin{equation}
  \{\Psi_a(t,\ve{x}) , \Psi_b^\dagger(t,\ve{y})\} = \delta^{(3)}(\ve{x} - \ve{y}) \delta_{ab}
  \label{eq:2.19}
\end{equation}
where $ a,b = 1,2,3,4 $ are Dirac indices. Expanding the free Dirac field in plane waves:
\begin{equation}
  \Psi(x) = \int \frac{\dd^3p}{(2\pi)^3 \sqrt{2E_\ve{p}}} \sum_{s = 1,2} \left[ a_{\ve{p},s} u^s(p) e^{-i p_\mu x^\mu} + b_{\ve{p},s}^\dagger v^s(p) e^{i p_\mu x^\mu} \right]_{p^0 = E_\ve{p}}
  \label{eq:2.20}
\end{equation}
\begin{equation}
  \bar{\Psi}(x) = \int \frac{\dd^3p}{(2\pi)^3 \sqrt{2E_\ve{p}}} \sum_{s = 1,2} \left[ a_{\ve{p},s}^\dagger \bar{u}^s(p) e^{i p_\mu x^\mu} + b_{\ve{p},s} \bar{v}^s(p) e^{-i p_\mu x^\mu} \right]_{p^0 = E_\ve{p}}
  \label{eq:2.21}
\end{equation}
where the spinor wave functions $ u^s(p) , v^s(p) $ are given by Eqq. \ref{eq:1.90}-\ref{eq:1.91}. Translating Eq. \ref{eq:2.19} in terms of creation/annihilation operators:
\begin{equation}
  \{a_\ve{p},s , a_{\ve{q},r}^\dagger\} = \{b_{\ve{p},s} , b_{\ve{q},r}^\dagger\} = (2\pi)^3 \delta^{(3)}(\ve{p} - \ve{q}) \delta_{sr}
  \label{eq:2.22}
\end{equation}
The Fock space is again constructed defining a vacuum state $ \ket{0} : a_{\ve{p},s} \ket{0} = b_{\ve{p},s} \ket{0} = 0 $ and then acting repeatedly on it with $ a_{\ve{p}.s}^\dagger , b_{\ve{p},s}^\dagger $. As these operators anticommute, states in this Fock space are antisymmetric under the exchange of particles, therefore spint-$ \frac{1}{2} $ obey the Fermi-Dirac statistics (as of the spin-statistic theorem).

\begin{proposition}{Dirac Hamiltonian}{}
  The Hamiltoniana for a Dirac field $ \Psi $ is:
  \begin{equation}
    H = \int \frac{\dd^3p}{(2\pi)^3} \sum_{s = 1,2} E_\ve{p} \left[ a_{\ve{p},s}^\dagger a_{\ve{p},s} + b_{\ve{p},s}^\dagger b_{\ve{p},s} \right]
    \label{eq:2.23}
  \end{equation}

  \tcblower

  \begin{proof}
    By Eqq. \ref{eq:2.18}, the Hamiltonian density is:
    \begin{equation*}
      \mathcal{H} = \Pi_\Psi \pa_0 \Psi - \mathcal{L}_\text{D} = i \Psi^\dagger \pa_0 \Psi - \bar{\Psi} \left( i \gamma^0 \pa_0 + i \gamma^i \pa_i - m \right) \Psi = \bar{\Psi} \left( -i \gamma^i \pa_i + m \right) \Psi
    \end{equation*}
    Therefore, using Eqq. \ref{eq:2.20}-\ref{eq:2.21} and Eq. \ref{eq:2.8}:
    \begin{equation*}
      \begin{split}
        H
        &= \int \dd^3x\, \bar{\Psi} \left( -i \gamma^i \pa_i + m \right) \Psi = \int \dd^3x\, \bar{\Psi} \left( -i \bs{\gamma} \cdot \bs{\nabla} + m \right) \Psi \\
        &= \mathcal{N} \int \dd^3x \int \frac{\dd^3p}{(2\pi)^3 \sqrt{2E_\ve{p}}} \int \frac{\dd^3q}{(2\pi)^3 \sqrt{2E_\ve{q}}} \sum_{s = 1,2} \sum_{r = 1,2} \left[ a_{\ve{p},s}^\dagger \bar{u}^s(p) e^{ip_\mu x^\mu} + b_{\ve{p},s} \bar{v}^s(p) e^{-ip_\mu x^\mu} \right] \times \\
        & \qquad \qquad \qquad \qquad \qquad \qquad \qquad \qquad \times \left( -i \bs{\gamma} \cdot \bs{\nabla} + m \right) \left[ a_{\ve{q},r} u^r(q) e^{-iq_\mu x^\mu} + b_{\ve{q},r}^\dagger v^r(q) e^{iq_\mu x^\mu} \right] \\
        &= \mathcal{N} \int \dd^3x \int \frac{\dd^3p}{(2\pi)^3 \sqrt{2E_\ve{p}}} \int \frac{\dd^3q}{(2\pi)^3 \sqrt{2E_\ve{q}}} \sum_{s = 1,2} \sum_{r = 1,2} \left[ a_{\ve{p},s}^\dagger \bar{u}^s(p) e^{i p_\mu x^\mu} + b_{\ve{p},s} \bar{v}^s(p) e^{-i p_\mu x^\mu} \right] \times \\
        & \qquad \qquad \qquad \qquad \qquad \qquad \times \left[ \left( \bs{\gamma} \cdot \ve{q} + m \right) a_{\ve{q},r} u^r(q) e^{-i q_\mu x^\mu} + \left( - \bs{\gamma} \cdot \ve{q} + m \right) b_{\ve{q},r}^\dagger v^r(q) e^{i q_\mu x^\mu} \right]
      \end{split}
    \end{equation*}
    Using the Dirac equation in the form $ (\slashed{p} - m) u(p) = (\slashed{p} + m) v(p) = 0 $:
    \begin{equation*}
      \left( \bs{\gamma} \cdot \ve{q} + m \right) u(q) = \gamma^0 E_\ve{q} u(q)
      \qquad
      \left( -\bs{\gamma} \cdot \ve{q} + m \right) v(q) = - \gamma^0 E_\ve{q} v(q)
    \end{equation*}
    Therefore, omitting the constraint $ p^0 = E_\ve{p} $ in the spinors' arguments and using Eq. \ref{eq:2.8}:
    \begin{equation*}
      \begin{split}
        H
        &= \mathcal{N} \int \dd^3x \int \frac{\dd^3p}{(2\pi)^3 \sqrt{2E_\ve{p}}} \int \frac{\dd^3q}{(2\pi)^3 \sqrt{2E_\ve{q}}} \sum_{s = 1,2} \sum_{r = 1,2} \left[ a_{\ve{p},s}^\dagger \bar{u}^s(\ve{p}) e^{i E_\ve{p} t} + b_{-\ve{p},s} \bar{v}^s(-\ve{p}) e^{-i E_\ve{p} t} \right] \times \\
        & \qquad \qquad \qquad \qquad \qquad \qquad \qquad \qquad \times \gamma^0 E_\ve{q} \left[ a_{\ve{q},r} u^r(\ve{q}) e^{-i E_\ve{q} t} - b_{-\ve{q},r}^\dagger v^r(-\ve{q}) e^{i E_\ve{q} t} \right] e^{i (\ve{q} - \ve{p}) \cdot \ve{x}} \\
        &= \mathcal{N} \int \frac{\dd^3p}{2(2\pi)^3} \sum_{s = 1,2} \sum_{r = 1,2} \left[ a_{\ve{p},s}^\dagger \bar{u}^s(\ve{p}) e^{i E_\ve{p} t} + b_{-\ve{p},s} \bar{v}^s(-\ve{p}) e^{-i E_\ve{p} t} \right] \times \\
        & \qquad \qquad \qquad \qquad \qquad \qquad \qquad \qquad \times \gamma^0 \left[ a_{\ve{p},r} u^r(\ve{p}) e^{-i E_\ve{p} t} - b_{-\ve{p},s}^\dagger v^r(-\ve{p}) e^{i E_\ve{p} t} \right] \\
        &= \mathcal{N} \int \frac{\dd^3p}{2(2\pi)^3} \sum_{s = 1,2} \sum_{r = 1,2} \Big[ a_{\ve{p},s}^\dagger a_{\ve{p},r} u^{s\dagger}(\ve{p}) u^r(\ve{p}) + b_{-\ve{p},s} a_{\ve{p},r} v^{s\dagger}(-\ve{p}) u^r(\ve{p}) e^{-2i E_\ve{p} t} + \\
        & \qquad \qquad \qquad \qquad \qquad \quad - a_{\ve{p},s}^\dagger b_{-\ve{p},r}^\dagger u^{s\dagger}(\ve{p}) v^r(-\ve{p}) e^{2i E_\ve{p} t} - b_{-\ve{p},s} b_{-\ve{p},r}^\dagger v^{s\dagger}(-\ve{p}) v^r(-\ve{p}) \Big]
      \end{split}
    \end{equation*}

    \begin{lemma}{}{haml-lem}
      \begin{equation*}
        u^{s\dagger}(\ve{p}) v^r(-\ve{p}) = v^{s\dagger}(-\ve{p}) u^r(\ve{p}) = 0
      \end{equation*}
    \end{lemma}

    Using Lemma \ref{lemma:haml-lem}, Eq. \ref{eq:1.93} and the antisymmetry $ \mathcal{N} b_{\ve{p},s} b_{\ve{p},s}^\dagger = - b_{\ve{p},s}^\dagger b_{\ve{p},s} $:
    \begin{equation*}
      H = \int \frac{\dd^3p}{(2\pi)^3} \sum_{s = 1,2} E_\ve{p} \mathcal{N} \left[ a_{\ve{p},s}^\dagger a_{\ve{p},s} - b_{\ve{p},s} b_{\ve{p},s}^\dagger \right] = \int \frac{\dd^3p}{(2\pi)^3} \sum_{s = 1,2} E_\ve{p} \left[ a_{\ve{p},s}^\dagger a_{\ve{p},s} + b_{\ve{p},s}^\dagger b_{\ve{p},s} \right]
    \end{equation*}
  \end{proof}
\end{proposition}

It can be seen that, using anticommutators, the Hamiltonian and its interpretation are analogous to that of the complex scalar field: if commutators were used, instead, one would get a final $ - b_{\ve{p},s}^\dagger b_{\ve{p},s} $ term, which is problematic as it yields an energy unbounded from below.









