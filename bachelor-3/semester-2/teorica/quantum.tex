\selectlanguage{english}

\section{Scalar fields}

As for the quantization of a classical system in Quantum Mechanics, the quantization of a scalar field theory is performed promoting $ \phi(t,\ve{x}) $ and $ \Pi(t,\ve{x}) $ to hermitian operators in the Heisenberg picture and imposing the canonical equal-time commutation relation:
\begin{equation}
  [\phi(t,\ve{x}) , \Pi(t,\ve{y})] = i \delta^{(3)}(\ve{x} - \ve{y})
  \label{eq:2.1}
\end{equation}
while of course $ [\phi(t,\ve{x}) , \phi(t,\ve{y})] = [\Pi(t,\ve{x}) , \Pi(t,\ve{y})] = 0 $.

\subsection{Real scalar fields}

A real scalar field is promoted to a real hermitian operator. In particular, by Eq. \ref{eq:1.70}:
\begin{equation}
  \phi(x) = \int \frac{\dd^3p}{(2\pi)^3 \sqrt{2E_\ve{p}}} \left[ a_\ve{p} e^{-i p_\mu x^\mu} + a_\ve{p}^\dagger e^{i p_\mu x^\mu} \right]_{p^0 = E_\ve{p}}
  \label{eq:2.2}
\end{equation}
In terms of creation and annihilation operators, the commutator Eq. \ref{eq:2.1} reads:
\begin{equation}
  [a_\ve{p} , a_\ve{p}^\dagger] = (2\pi)^3 \delta^{(3)}(\ve{p} - \ve{q})
  \label{eq:2.3}
\end{equation}
while $ [a_\ve{p} , a_\ve{q}] = [a_\ve{p}^\dagger , a_\ve{q}^\dagger] = 0 $. These can be regarded as the creation and annihilation operators of a collection of harmonic oscillators, one for each value of the momentum $ \ve{p} $: the \textit{Fock space} of the real scalar field can thus be constructed analogously to the Hilbert space of the harmonic oscillator.\\
Defining the \textit{vacuum state} $ \ket{0} : a_\ve{p} \ket{0} = 0 \,\,\forall \ve{p} $, suitably normalized as $ \braket{0 | 0} = 1 $, the generic state of the Fock space is:
\begin{equation}
  \ket{\ve{p}_1 , \dots , \ve{p}_n} = \sqrt{2E_{\ve{p}_1}} \dots \sqrt{2E_{\ve{p}_n}}\, a_{\ve{p}_1}^\dagger \dots a_{\ve{p}_n}^\dagger \ket{0}
  \label{eq:2.4}
\end{equation}

\begin{proposition}{Normalization}{}
  For one-particle states:
  \begin{equation}
    \braket{\ve{p}_1 | \ve{p}_2} = 2E_{\ve{p}_1} (2\pi)^3 \delta^{(3)}(\ve{p}_1 - \ve{p}_2)
    \label{eq:2.5}
  \end{equation}

  \begin{proof}
    By Eq. \ref{eq:2.3}:
    \begin{equation*}
      \braket{\ve{p}_1 | \ve{p}_2} = \sqrt{2E_{\ve{p}_1}} \sqrt{2E_{\ve{p}_2}} \braket{0 | a_{\ve{p}_1} a_{\ve{p}_2}^\dagger | 0}
      = \sqrt{2E_{\ve{p}_1}} \sqrt{2E_{\ve{p}_2}} \braket{0 | [a_{\ve{p}_1} , a_{\ve{p}_2}^\dagger] | 0} = 2E_{\ve{p}_1} (2\pi)^3 \delta^{(3)}(\ve{p}_1 - \ve{p}_2)
    \end{equation*}
  \end{proof}
\end{proposition}

\begin{lemma}{}{}
  The combination $ E_\ve{p} \delta^{(3)}(\ve{p} - \ve{q}) $ is Lorentz invariant.

  \tcblower

  \begin{proof}
    Recall that:
    \begin{equation}
      \delta(f(x) - f(x_0)) = \frac{1}{\abs{f'(x_0)}} \delta(x - x_0)
      \label{eq:2.6}
    \end{equation}
    A Lorentz boost along $ \ve{e}_i $ yields $ p'_i = \gamma (p_i + \beta E) $ and $ E' = \gamma (E + \beta p_i) $, so:
    \begin{equation*}
        \delta^{(3)}(\ve{p} - \ve{q})
        = \delta^{(3)}(\ve{p}' - \ve{q}') \frac{\dd p'_i}{\dd p_i}
        = \delta^{(3)}(\ve{p}' - \ve{q}') \gamma \left( 1 + \beta \frac{\dd E}{\dd p_i} \right)
    \end{equation*}
    Note that $ p^2 = E^2 - \ve{p}^2 $ is a Lorentz invariant, thus $ E \dd E = p_i \dd p_i $, so:
    \begin{equation*}
      \delta^{(3)}(\ve{p} - \ve{q})
      = \delta^{(3)}(\ve{p}' - \ve{q}') \frac{\gamma (E + \beta p_i)}{E}
      = \delta^{(3)}(\ve{p}' - \ve{q}') \frac{E'}{E}
    \end{equation*}
  \end{proof}
\end{lemma}

This explains the choice of normalization.

\begin{proposition}{KG Hamiltonian}{}
  The Hamiltonian of a real scalar field can be written as:
  \begin{equation}
    H = \int \frac{\dd^3 p}{(2\pi)^3} E_\ve{p} \left( a_\ve{p}^\dagger a_\ve{p} + \frac{1}{2} [a_\ve{p} , a_\ve{p}^\dagger] \right)
    \label{eq:2.7}
  \end{equation}

  \tcblower

  \begin{proof}
    First of all, from Eq. \ref{eq:2.2}:
    \begin{equation*}
      \begin{split}
        \Pi(t,\ve{x}) = \pa_0 \phi(t,\ve{x})
        &= -i \int \frac{\dd^3 p}{(2\pi)^3} \sqrt{\frac{E_\ve{p}}{2}} \left[ a_\ve{p} e^{-i p_\mu x^\mu} - a_\ve{p}^\dagger e^{i p_\mu x^\mu} \right]_{p^0 = E_\ve{p}} \\
        &= -i \int \frac{\dd^3 p}{(2\pi)^3} \sqrt{\frac{E_\ve{p}}{2}} \left[ a_\ve{p} e^{-i E_\ve{p} t} - a_{-\ve{p}}^\dagger e^{i E_\ve{p} t} \right] e^{i \ve{p} \cdot \ve{x}}
      \end{split}
    \end{equation*}
    \begin{equation*}
      \bs{\nabla} \phi(t,\ve{x}) = i \int \frac{\dd^3 p}{(2\pi)^3} \frac{\ve{p}}{\sqrt{2E_\ve{p}}} \left[ a_\ve{p} e^{-i E_\ve{p} t} + a_{-\ve{p}}^\dagger e^{i E_\ve{p} t} \right] e^{i \ve{p} \cdot \ve{x}}
    \end{equation*}
    Inserting these expressions in Eq. \ref{eq:1.71}:
    \begin{equation*}
      \begin{split}
        \mathcal{H}
        &= \frac{1}{2} \int \frac{\dd^3 p}{(2\pi)^3} \int \frac{\dd^3 q}{(2\pi)^3} \bigg\{ - \frac{\sqrt{E_\ve{p} E_\ve{q}}}{2} \left[ a_\ve{p} e^{-i E_\ve{p} t} - a_{-\ve{p}}^\dagger e^{i E_\ve{p} t} \right] \left[ a_\ve{q} e^{-i E_\ve{q} t} - a_{-\ve{q}}^\dagger e^{i E_{\ve{q}} t} \right] + \\
        & \qquad + \frac{- \ve{p} \cdot \ve{q} + m^2}{2\sqrt{E_\ve{p} E_\ve{q}}} \left[ a_\ve{p} e^{-i E_\ve{p} t} + a_{-\ve{p}}^\dagger e^{i E_\ve{p} t} \right] \left[ a_\ve{q} e^{-i E_\ve{q} t} + a_{-\ve{q}}^\dagger e^{i E_\ve{q} t} \right] \bigg\} e^{i (\ve{p} + \ve{q}) \cdot \ve{x}}
      \end{split}
    \end{equation*}
    Recall the identity:
    \begin{equation}
      \int \frac{\dd^3x}{(2\pi)^3} e^{i (\ve{p} - \ve{q}) \cdot \ve{x}} = \delta^{(3)}(\ve{p} - \ve{q})
      \label{eq:2.8}
    \end{equation}
    Then (using $ E_{-\ve{p}} = E_\ve{p} $):
    \begin{equation*}
      \begin{split}
        H
        &= \int \dd^3x\, \mathcal{H} \\
        &= \frac{1}{2} \int \frac{\dd^3 p}{(2\pi)^3} \int \dd^3 q \bigg\{ - \frac{\sqrt{E_\ve{p} E_\ve{q}}}{2} \left[ a_\ve{p} e^{-i E_\ve{p} t} - a_{-\ve{p}}^\dagger e^{i E_\ve{p} t} \right] \left[ a_\ve{q} e^{-i E_\ve{q} t} - a_{-\ve{q}}^\dagger e^{i E_\ve{q} t} \right] + \\
        & \qquad + \frac{- \ve{p} \cdot \ve{q} + m^2}{2\sqrt{E_\ve{p} E_\ve{q}}} \left[ a_\ve{p} e^{-i E_\ve{p} t} + a_{-\ve{p}}^\dagger e^{i E_\ve{p} t} \right] \left[ a_\ve{q} e^{-i E_\ve{q} t} + a_{-\ve{q}}^\dagger e^{i E_\ve{q} t} \right] \bigg\} \delta^{(3)}(\ve{p} + \ve{q}) \\
        &= \frac{1}{2} \int \frac{\dd^3 p}{(2\pi)^3} \bigg\{ - \frac{E_\ve{p}}{2} \left[ a_\ve{p} e^{-i E_\ve{p} t} - a_{-\ve{p}}^\dagger e^{i E_\ve{p} t} \right] \left[ a_{-\ve{p}} e^{-i E_\ve{p} t} - a_\ve{p}^\dagger e^{i E_\ve{p} t} \right] + \\
        & \qquad + \frac{\ve{p}^2 + m^2}{2E_\ve{p}} \left[ a_\ve{p} e^{-i E_\ve{p} t} + a_{-\ve{p}}^\dagger e^{i E_\ve{p} t} \right] \left[ a_{-\ve{p}} e^{-i E_\ve{p} t} + a_\ve{p}^\dagger e^{i E_\ve{p} t} \right] \bigg\} \\
        &= \frac{1}{2} \int \frac{\dd^3p}{(2\pi)^3} \frac{E_\ve{p}}{2} \bigg\{ - \left[ a_\ve{p} a_{-\ve{p}} e^{-2i E_\ve{p} t} - a_{-\ve{p}}^\dagger a_{-\ve{p}} - a_\ve{p} a_\ve{p}^\dagger + a_{-\ve{p}}^\dagger a_\ve{p}^\dagger e^{2i E_\ve{p} t} \right] \\
        & \qquad + \left[ a_\ve{p} a_{-\ve{p}} e^{-2i E_\ve{p} t} + a_{-\ve{p}}^\dagger a_{-\ve{p}} + a_\ve{p} a_\ve{p}^\dagger + a_{-\ve{p}}^\dagger a_\ve{p}^\dagger e^{2i E_\ve{p} t} \right] \bigg\} \\
        &= \frac{1}{2} \int \frac{\dd^3p}{(2\pi)^3} E_\ve{p} \left[ a_{-\ve{p}}^\dagger a_{-\ve{p}} + a_\ve{p} a_\ve{p}^\dagger \right] = \frac{1}{2} \int \frac{\dd^3p}{(2\pi)^3} E_\ve{p} \left( a_\ve{p}^\dagger a_\ve{p} + a_\ve{p} a_\ve{p}^\dagger \right)
      \end{split}
    \end{equation*}
  \end{proof}
\end{proposition}

The second term in the Hamiltonian Eq. \ref{eq:2.7} is the sum of the zero-point energy of all oscillators and is proportional to $ (2\pi)^3 \delta^{(3)}(0) \rightarrow V $, thus:
\begin{equation*}
  E_\text{vac} = \frac{V}{2} \int \frac{\dd^3p}{(2\pi)^3} E_\ve{p}
\end{equation*}
This energy shows two divergences: the one coming from the infinite-volume limit (i.e. small momentum), regularized introducing an \textit{infrared cutoff} in the form of a finite volume, and the one from the ultra-relativistic limit (i.e. large momentum), regularized introducing an \textit{ultraviolet cutoff} in the form of a maximum momentum $ \Lambda $. These divergences are retained in the expression for $ E_\text{vac} $, as $ E_\text{vac} \sim V $ and $ E_\text{vac} \sim \Lambda^4 $, but can be ignored (when ignoring gravity) since experiments are only sensitive to energy differences.\\
Discarding the zero-point energy, the Hamiltonian becomes:
\begin{equation}
  H = \int \frac{\dd^3p}{(2\pi)^3} E_\ve{p} a_\ve{p}^\dagger a_\ve{p} \equiv \mathfrak{N} \frac{1}{2} \int \frac{\dd^3p}{(2\pi)^3} E_\ve{p} \left( a_\ve{p}^\dagger a_\ve{p} + a_\ve{p} a_\ve{p}^\dagger \right)
  \label{eq:2.9}
\end{equation}
where the \textit{normal ordering operator} $ \mathfrak{N} $ was introduced, which acts by moving all creation operators to the left and all annihilation operators to the right (ex.: $ \mathfrak{N} a_\ve{p} a_\ve{p}^\dagger = a_\ve{p}^\dagger a_\ve{p} $). It is now straightforward to compute the energy of a generic state in the Fock space, as $ a_\ve{p}^\dagger a_\ve{p} $ is just a number operator:
\begin{equation}
  H \ket{\ve{p}_1 , \dots , \ve{p}_n} = \left( E_{\ve{p}_1} + \dots + E_{\ve{p}_n} \right) \ket{\ve{p}_1 , \dots , \ve{p}_n}
  \label{eq:2.10}
\end{equation}
Computing the spatial momentum from Eq. \ref{eq:1.67} as $ P^i = \mathfrak{N} \int \dd^3x\, \theta^{0i} = \int \dd^3x\, \mathfrak{N} \pa_0 \phi \pa^i \phi $:
\begin{equation}
  P^i = \int \frac{\dd^3p}{(2\pi)^3} p^i a_\ve{p}^\dagger a_\ve{p}
  \label{eq:2.11}
\end{equation}
Therefore, the state $ a_\ve{p}^\dagger \ket{0} $ can correctly be interpreted as a one-particle state with momentum $ \ve{p} $, mass $ m $ and energy $ E_\ve{p} = \sqrt{\ve{p}^2 + m^2} $. The generic state in the Fock space is a multiparticle state, with total energy and momentum the sum of the individual energies and momenta.\\
Finally, note that creation operators commute between themselves, hence multiparticle states are symmetric under exchange of pairs of particles, i.e. they obey the Bose-Einstein statistics: this agrees with the fact that quanta of a scalar field have no intrinsic spin. i.e. are spin-0 particles.

\subsection{Complex scalar fields}

When considering a complex scalar field, Eq. \ref{eq:1.82} becomes:
\begin{equation}
  \phi(x) = \int \frac{\dd^3p}{(2\pi)^3 \sqrt{2E_\ve{p}}} \left[ a_\ve{p} e^{-i p_\mu x^\mu} + b_\ve{p}^\dagger e^{i p_\mu x^\mu} \right]_{p^0 = E_\ve{p}}
  \label{eq:2.12}
\end{equation}
\begin{equation}
  \phi^\dagger(x) = \int \frac{\dd^3p}{(2\pi)^3 \sqrt{2E_\ve{p}}} \left[ a_\ve{p}^\dagger e^{i p_\mu x^\mu} + b_\ve{p} e^{-i p_\mu x^\mu} \right]_{p^0 = E_\ve{p}}
  \label{eq:2.13}
\end{equation}
Now there are two independent sets of creation/annihilation operators, which obey the canonical commutation relation:
\begin{equation}
  [a_\ve{p} , a_\ve{q}^\dagger] = [b_\ve{p} , b_\ve{q}^\dagger] = (2\pi)^3 \delta^{(3)}(\ve{p} - \ve{q})
  \label{eq:2.14}
\end{equation}
with all other commutators vanishing. The Fock space is constructed by defining a vacuum state $ \ket{0} : a_\ve{p} \ket{0} = b_\ve{p} \ket{0} = 0 $ and then acting repeatedly with both creation operators. With normal ordering, one finds:
\begin{equation}
  H = \int \frac{\dd^3p}{(2\pi)^3} E_\ve{p} \left( a_\ve{p}^\dagger a_\ve{p} + b_\ve{p}^\dagger b_\ve{p} \right)
  \label{eq:2.15}
\end{equation}
\begin{equation}
  P^i = \int \frac{\dd^3p}{(2\pi)^3} p^i \left( a_\ve{p}^\dagger a_\ve{p} + b_\ve{p}^\dagger b_\ve{p} \right)
  \label{eq:2.16}
\end{equation}
The quanta of a complex scalar field are given by two different species of particles with the same mass.

\begin{proposition}{$ \Un{1} $ charge}{}
  The $ \Un{1} $ charge of the quantized complex scalar field is:
  \begin{equation}
    Q_{\Un{1}} = \int \frac{\dd^3p}{(2\pi)^3} \left( a_\ve{p}^\dagger a_\ve{p} - b_\ve{p}^\dagger b_\ve{p} \right)
    \label{eq:2.17}
  \end{equation}

  \tcblower

  \begin{proof}
    By Eq. \ref{eq:1.83}:
    \begin{equation*}
      \begin{split}
        Q_{\Un{1}}
        &= i \int \dd^3x\, \phi^\dagger \smlra{\pa_0} \phi = i \int \dd^3x \int \frac{\dd^3q}{(2\pi)^3 \sqrt{2E_\ve{q}}} \int \frac{\dd^3p}{(2\pi)^3 \sqrt{2E_\ve{p}}} \,\times \\
        & \quad \times \bigg\{ \left[ a_\ve{q}^\dagger e^{i q_\mu x^\mu} + b_\ve{q} e^{-i q_\mu x^\mu} \right] \pa_0 \left( a_\ve{p} e^{-i p_\mu x^\mu} + b_\ve{p}^\dagger e^{i p_\mu x^\mu} \right) + \\
        & \quad - \pa_0 \left( a_\ve{q}^\dagger e^{i q_\mu x^\mu} + b_\ve{q} e^{-i q_\mu x^\mu} \right) \left[ a_\ve{p} e^{-i p_\mu x^\mu} + b_\ve{p}^\dagger e^{i p_\mu x^\mu} \right] \bigg\}
      \end{split}
    \end{equation*}
    \begin{equation*}
      \begin{split}
        Q_{\Un{1}}
        &= \int \dd^3x \int \frac{\dd^3q}{(2\pi)^3 \sqrt{2E_\ve{q}}} \int \frac{\dd^3p}{(2\pi)^3 \sqrt{2E_\ve{p}}} \,\times \\
        & \quad \times \bigg\{ E_\ve{p} \left[ a_\ve{q}^\dagger e^{i q_\mu x^\mu} + b_\ve{q} e^{-i q_\mu x^\mu} \right] \left[ a_\ve{p} e^{-i p_\mu x^\mu} - b_\ve{p}^\dagger e^{i p_\mu x^\mu} \right] + \\
        & \quad + E_\ve{q} \left[ a_\ve{q}^\dagger e^{i q_\mu x^\mu} - b_\ve{q} e^{-i q_\mu x^\mu} \right] \left[ a_\ve{p} e^{-i p_\mu x^\mu} + b_\ve{p}^\dagger e^{i p_\mu x^\mu} \right] \bigg\} \\
        &= \int \dd^3x \int \frac{\dd^3q}{(2\pi)^3 \sqrt{2E_\ve{q}}} \int \frac{\dd^3p}{(2\pi)^3 \sqrt{2E_\ve{p}}} \,\times \\
        & \quad \times \bigg\{ E_\ve{p} \left[ a_\ve{q}^\dagger e^{i E_\ve{q} t} + b_{-\ve{q}} e^{-i E_\ve{q} t} \right] \left[ a_\ve{p} e^{-i E_\ve{p} t} - b_{-\ve{p}}^\dagger e^{i E_\ve{p} t} \right] + \\
        & \quad + E_\ve{q} \left[ a_\ve{q}^\dagger e^{i E_\ve{q} t} - b_{-\ve{q}} e^{-i E_\ve{q} t} \right] \left[ a_\ve{p} e^{-i E_\ve{p} t} + b_{-\ve{p}}^\dagger e^{i E_\ve{p} t} \right] \bigg\} e^{i (\ve{p} - \ve{q}) \cdot \ve{x}} \\
        &= \frac{1}{2} \int \frac{\dd^3p}{(2\pi)^3} \bigg\{ \left[ a_\ve{p}^\dagger e^{i E_\ve{p} t} + b_{-\ve{p}} e^{-i E_\ve{p} t} \right] \left[ a_\ve{p} e^{-i E_\ve{p} t} - b_{-\ve{p}}^\dagger e^{i E_\ve{p} t} \right] + \\
        & \quad + \left[ a_\ve{p}^\dagger e^{i E_\ve{p} t} - b_{-\ve{p}} e^{-i E_\ve{p} t} \right] \left[ a_\ve{p} e^{-i E_\ve{p} t} + b_{-\ve{p}}^\dagger e^{i E_\ve{p} t} \right] \bigg\} \\
        &= \int \frac{\dd^3p}{(2\pi)^3} \left[ a_\ve{p}^\dagger a_\ve{p} - b_{-\ve{p}} b_{-\ve{p}}^\dagger \right] = \int \frac{\dd^3p}{(2\pi)^3} \left( a_\ve{p}^\dagger a_\ve{p} - b_\ve{p} b_\ve{p}^\dagger \right)
      \end{split}
    \end{equation*}
    Applying normal ordering yields the thesis.
  \end{proof}
\end{proposition}

While normal ordering was justified when considering the Hamiltonian on the grounds that the vacuum energy is unobservable, a charged vacuum would have observable effects; however when promoting $ \phi $ to a quantum operator, the expression $ \phi^\dagger \smlra{\pa_0} \phi $ presents an ordering ambiguity (ex.: $ \phi^\dagger (\pa_0 \phi) $ or $ (\pa_0 \phi) \phi^\dagger $), which is removed requiring the charge of the vacuum to vanish.\\
Being $ a_\ve{p}^\dagger a_\ve{p} $ and $ b_\ve{p}^\dagger b_\ve{p} $ number operators, the $ \Un{1} $ charge is equal to the number of quanta created by $ a_\ve{p}^\dagger $ minus the number of quanta created by $ b_\ve{p}^\dagger $, integrated over all momenta: in particular, $ a_\ve{p}^\dagger \ket{0} $ and $ b_\ve{p}^\dagger \ket{0} $ are both spin-zero particles of mass $ m $ and momentum $ \ve{p} $, but they respectively have charges $ Q_{\Un{1}} = +1 $ and $ Q_{\Un{1}} = -1 $. This allows to properly interpret the negative-energy solutions of the KG equations: they are positive-energy particles with opposite $ \Un{1} $ charge and are called \textit{antiparticles}.\\
For a real scalar field, the reality condition reads $ a_\ve{p} = b_\ve{p} $, thus it describes a field whose particle is its own antiparticle, and it is symmetric under any $ \Un{1} $ symmetry.

\section{Spinor fields}

A principle of QFT is the \textit{spin-statistic theorem}: integer-spin fields are to be quantized imposing equal-time commutation relations, while half-integer-spin with equal-time anticommutation relations.

\subsection{Dirac fields}

From the Dirac Lagrangian Eq. \ref{eq:1.91}, the conjugate momentum to the Dirac field $ \Psi $ is computed as:
\begin{equation}
  \Pi_\Psi = i \bar{\Psi} \gamma^0 = i \Psi^\dagger
  \label{eq:2.18}
\end{equation}
Imposing the canonical anticommutation relation, according to the spin-statistic theorem:
\begin{equation}
  \{\Psi_a(t,\ve{x}) , \Psi_b^\dagger(t,\ve{y})\} = \delta^{(3)}(\ve{x} - \ve{y}) \delta_{ab}
  \label{eq:2.19}
\end{equation}
where $ a,b = 1,2,3,4 $ are Dirac indices. Expanding the free Dirac field in plane waves:
\begin{equation}
  \Psi(x) = \int \frac{\dd^3p}{(2\pi)^3 \sqrt{2E_\ve{p}}} \sum_{s = 1,2} \left[ a_{\ve{p},s} u^s(p) e^{-i p_\mu x^\mu} + b_{\ve{p},s}^\dagger v^s(p) e^{i p_\mu x^\mu} \right]_{p^0 = E_\ve{p}}
  \label{eq:2.20}
\end{equation}
\begin{equation}
  \bar{\Psi}(x) = \int \frac{\dd^3p}{(2\pi)^3 \sqrt{2E_\ve{p}}} \sum_{s = 1,2} \left[ a_{\ve{p},s}^\dagger \bar{u}^s(p) e^{i p_\mu x^\mu} + b_{\ve{p},s} \bar{v}^s(p) e^{-i p_\mu x^\mu} \right]_{p^0 = E_\ve{p}}
  \label{eq:2.21}
\end{equation}
where the spinor wave functions $ u^s(p) , v^s(p) $ are given by Eqq. \ref{eq:1.94}-\ref{eq:1.95}. Translating Eq. \ref{eq:2.19} in terms of creation/annihilation operators:
\begin{equation}
  \{a_\ve{p},s , a_{\ve{q},r}^\dagger\} = \{b_{\ve{p},s} , b_{\ve{q},r}^\dagger\} = (2\pi)^3 \delta^{(3)}(\ve{p} - \ve{q}) \delta_{sr}
  \label{eq:2.22}
\end{equation}
The Fock space is again constructed defining a vacuum state $ \ket{0} : a_{\ve{p},s} \ket{0} = b_{\ve{p},s} \ket{0} = 0 $ and then acting repeatedly on it with $ a_{\ve{p}.s}^\dagger , b_{\ve{p},s}^\dagger $:
\begin{equation*}
  \begin{split}
    &\ket{(\ve{p}_1,s_1) , \dots , (\ve{p}_n,s_n) ; (\ve{q}_1,r_1) , \dots , (\ve{q}_m,r_m)} \\
    & \qquad \qquad \qquad \qquad \qquad = \sqrt{2E_{\ve{p}_1}} \dots \sqrt{2E_{\ve{p}_n}} \sqrt{2E_{\ve{q}_1}} \dots \sqrt{2E_{\ve{q}_m}} a_{\ve{p}_1,s_1}^\dagger \dots a_{\ve{p}_n,s_n}^\dagger b_{\ve{q}_1,r_1}^\dagger \dots b_{\ve{q}_m,r_m}^\dagger \ket{0}
  \end{split}
\end{equation*}
As these operators anticommute, states in this Fock space are antisymmetric under the exchange of particles, therefore spin-$ \frac{1}{2} $ obey the Fermi-Dirac statistics (as of the spin-statistic theorem).

\begin{proposition}{Dirac Hamiltonian}{}
  The Hamiltoniana for a Dirac field $ \Psi $ is:
  \begin{equation}
    H = \int \frac{\dd^3p}{(2\pi)^3} \sum_{s = 1,2} E_\ve{p} \left( a_{\ve{p},s}^\dagger a_{\ve{p},s} + b_{\ve{p},s}^\dagger b_{\ve{p},s} \right)
    \label{eq:2.23}
  \end{equation}

  \tcblower

  \begin{proof}
    By Eqq. \ref{eq:2.18}, the Hamiltonian density is:
    \begin{equation*}
      \mathcal{H} = \Pi_\Psi \pa_0 \Psi - \mathcal{L}_\text{D} = i \Psi^\dagger \pa_0 \Psi - \bar{\Psi} \left( i \gamma^0 \pa_0 + i \gamma^i \pa_i - m \right) \Psi = \bar{\Psi} \left( -i \gamma^i \pa_i + m \right) \Psi
    \end{equation*}
    Therefore, using Eqq. \ref{eq:2.20}-\ref{eq:2.21} and Eq. \ref{eq:2.8}:
    \begin{equation*}
      \begin{split}
        H
        &= \int \dd^3x\, \bar{\Psi} \left( -i \gamma^i \pa_i + m \right) \Psi = \int \dd^3x\, \bar{\Psi} \left( -i \bs{\gamma} \cdot \bs{\nabla} + m \right) \Psi \\
        &= \mathfrak{N} \int \dd^3x \int \frac{\dd^3p}{(2\pi)^3 \sqrt{2E_\ve{p}}} \int \frac{\dd^3q}{(2\pi)^3 \sqrt{2E_\ve{q}}} \sum_{s = 1,2} \sum_{r = 1,2} \left[ a_{\ve{p},s}^\dagger \bar{u}^s(p) e^{ip_\mu x^\mu} + b_{\ve{p},s} \bar{v}^s(p) e^{-ip_\mu x^\mu} \right] \times \\
        & \qquad \qquad \qquad \qquad \qquad \qquad \qquad \qquad \times \left( -i \bs{\gamma} \cdot \bs{\nabla} + m \right) \left[ a_{\ve{q},r} u^r(q) e^{-iq_\mu x^\mu} + b_{\ve{q},r}^\dagger v^r(q) e^{iq_\mu x^\mu} \right] \\
        &= \mathfrak{N} \int \dd^3x \int \frac{\dd^3p}{(2\pi)^3 \sqrt{2E_\ve{p}}} \int \frac{\dd^3q}{(2\pi)^3 \sqrt{2E_\ve{q}}} \sum_{s = 1,2} \sum_{r = 1,2} \left[ a_{\ve{p},s}^\dagger \bar{u}^s(p) e^{i p_\mu x^\mu} + b_{\ve{p},s} \bar{v}^s(p) e^{-i p_\mu x^\mu} \right] \times \\
        & \qquad \qquad \qquad \qquad \qquad \qquad \times \left[ \left( \bs{\gamma} \cdot \ve{q} + m \right) a_{\ve{q},r} u^r(q) e^{-i q_\mu x^\mu} + \left( - \bs{\gamma} \cdot \ve{q} + m \right) b_{\ve{q},r}^\dagger v^r(q) e^{i q_\mu x^\mu} \right]
      \end{split}
    \end{equation*}
    Using the Dirac equation in the form $ (\slashed{p} - m) u(p) = (\slashed{p} + m) v(p) = 0 $:
    \begin{equation*}
      \left( \bs{\gamma} \cdot \ve{q} + m \right) u(q) = \gamma^0 E_\ve{q} u(q)
      \qquad
      \left( -\bs{\gamma} \cdot \ve{q} + m \right) v(q) = - \gamma^0 E_\ve{q} v(q)
    \end{equation*}
    Therefore, omitting the constraint $ p^0 = E_\ve{p} $ in the spinors' arguments and using Eq. \ref{eq:2.8}:
    \begin{equation*}
      \begin{split}
        H
        &= \mathfrak{N} \int \dd^3x \int \frac{\dd^3p}{(2\pi)^3 \sqrt{2E_\ve{p}}} \int \frac{\dd^3q}{(2\pi)^3 \sqrt{2E_\ve{q}}} \sum_{s = 1,2} \sum_{r = 1,2} \left[ a_{\ve{p},s}^\dagger \bar{u}^s(\ve{p}) e^{i E_\ve{p} t} + b_{-\ve{p},s} \bar{v}^s(-\ve{p}) e^{-i E_\ve{p} t} \right] \times \\
        & \qquad \qquad \qquad \qquad \qquad \qquad \qquad \qquad \times \gamma^0 E_\ve{q} \left[ a_{\ve{q},r} u^r(\ve{q}) e^{-i E_\ve{q} t} - b_{-\ve{q},r}^\dagger v^r(-\ve{q}) e^{i E_\ve{q} t} \right] e^{i (\ve{q} - \ve{p}) \cdot \ve{x}} \\
        &= \mathfrak{N} \int \frac{\dd^3p}{2(2\pi)^3} \sum_{s = 1,2} \sum_{r = 1,2} \left[ a_{\ve{p},s}^\dagger \bar{u}^s(\ve{p}) e^{i E_\ve{p} t} + b_{-\ve{p},s} \bar{v}^s(-\ve{p}) e^{-i E_\ve{p} t} \right] \times \\
        & \qquad \qquad \qquad \qquad \qquad \qquad \qquad \qquad \times \gamma^0 \left[ a_{\ve{p},r} u^r(\ve{p}) e^{-i E_\ve{p} t} - b_{-\ve{p},s}^\dagger v^r(-\ve{p}) e^{i E_\ve{p} t} \right] \\
        &= \mathfrak{N} \int \frac{\dd^3p}{2(2\pi)^3} \sum_{s = 1,2} \sum_{r = 1,2} \Big[ a_{\ve{p},s}^\dagger a_{\ve{p},r} u^{s\dagger}(\ve{p}) u^r(\ve{p}) + b_{-\ve{p},s} a_{\ve{p},r} v^{s\dagger}(-\ve{p}) u^r(\ve{p}) e^{-2i E_\ve{p} t} + \\
        & \qquad \qquad \qquad \qquad \qquad \quad - a_{\ve{p},s}^\dagger b_{-\ve{p},r}^\dagger u^{s\dagger}(\ve{p}) v^r(-\ve{p}) e^{2i E_\ve{p} t} - b_{-\ve{p},s} b_{-\ve{p},r}^\dagger v^{s\dagger}(-\ve{p}) v^r(-\ve{p}) \Big]
      \end{split}
    \end{equation*}

    \begin{lemma}{}{haml-lem}
      \begin{equation*}
        u^{s\dagger}(\ve{p}) v^r(-\ve{p}) = v^{s\dagger}(-\ve{p}) u^r(\ve{p}) = 0
      \end{equation*}
    \end{lemma}

    Using Lemma \ref{lemma:haml-lem}, Eq. \ref{eq:1.97} and the antisymmetry $ \mathfrak{N} b_{\ve{p},s} b_{\ve{p},s}^\dagger = - b_{\ve{p},s}^\dagger b_{\ve{p},s} $:
    \begin{equation*}
      H = \int \frac{\dd^3p}{(2\pi)^3} \sum_{s = 1,2} E_\ve{p} \mathfrak{N} \left[ a_{\ve{p},s}^\dagger a_{\ve{p},s} - b_{\ve{p},s} b_{\ve{p},s}^\dagger \right] = \int \frac{\dd^3p}{(2\pi)^3} \sum_{s = 1,2} E_\ve{p} \left( a_{\ve{p},s}^\dagger a_{\ve{p},s} + b_{\ve{p},s}^\dagger b_{\ve{p},s} \right)
    \end{equation*}
  \end{proof}
\end{proposition}

It can be seen that, using anticommutators, the Hamiltonian and its interpretation are analogous to that of the complex scalar field: if commutators were used, instead, one would get a final $ - b_{\ve{p},s}^\dagger b_{\ve{p},s} $ term, which is problematic as it yields an energy unbounded from below\footnotemark{}.\\
\footnotetext{This hints to a more profound meaning of the spin-statistic theorem: as shown by Pauli, this theorem is implied by the conditions of Lorentz invariance, positive energies and causality.}
The momentum operator too is analogous to that of the complex scalar field, with the additional spin degree of freedom:

\begin{proposition}{Momentum operator}{}
  The momentum operator of a Dirac field $ \Psi $ is:
  \begin{equation}
    \ve{P} = \int \frac{\dd^3p}{(2\pi)^3} \sum_{s = 1,2} \ve{p} \left( a_{\ve{p},s}^\dagger a_{\ve{p},s} + b_{\ve{p},s}^\dagger b_{\ve{p},s} \right)
    \label{eq:2.24}
  \end{equation}

  \tcblower

  \begin{proof}
    By Eq. \ref{eq:1.70}, the $ 0i $-component of energy-momentum tensor of the Dirac Lagrangian is:
    \begin{equation*}
      \theta^{0i} = \frac{\pa \mathcal{L}_\text{D}}{\pa (\pa_0 \Psi)} \pa^i \Psi = \bar{\Psi} i \gamma^0 \pa^i \Psi = - \Psi^\dagger i \pa_i \Psi
    \end{equation*}
    Thus, according to Eq. \ref{eq:1.71}:
    \begin{equation*}
      \begin{split}
        \ve{P}
        &= \int \dd^3x\, \Psi^\dagger (-i \bs{\nabla}) \Psi \\
        &= \mathfrak{N} \int \dd^3x \int \frac{\dd^3p}{(2\pi)^3 \sqrt{2E_\ve{p}}} \int \frac{\dd^3q}{(2\pi)^3 \sqrt{2E_\ve{q}}} \sum_{s = 1,2} \sum_{r = 1,2} \left[ a_{\ve{p},s}^\dagger u^{s\dagger}(p) e^{i p_\mu x^\mu} + b_{\ve{p},s} v^{s\dagger}(p) e^{-ip_\mu x^\mu} \right] \times \\
        & \qquad \qquad \qquad \qquad \qquad \qquad \qquad \qquad \qquad \qquad \times (-i \bs{\nabla}) \left[ a_{\ve{q},r} u^r(q) e^{-i q_\mu x^\mu} + b_{\ve{q},r}^\dagger v^r(q) e^{i q_\mu x^\mu} \right] \\
        &= \mathfrak{N} \int \dd^3x \int \frac{\dd^3p}{(2\pi)^3 \sqrt{2E_\ve{p}}} \int \frac{\dd^3q}{(2\pi)^3 \sqrt{2E_\ve{q}}} \sum_{s = 1,2} \sum_{r = 1,2} \left[ a_{\ve{p},s}^\dagger u^{s\dagger}(p) e^{i p_\mu x^\mu} + b_{\ve{p},s} v^{s\dagger}(p) e^{-ip_\mu x^\mu} \right] \times \\
        & \qquad \qquad \qquad \qquad \qquad \qquad \qquad \qquad \qquad \qquad \qquad \times \ve{q} \left[ a_{\ve{q},r} u^r(q) e^{-i q_\mu x^\mu} - b_{\ve{q},r}^\dagger v^r(q) e^{i q_\mu x^\mu} \right] \\
        &= \mathfrak{N} \int \frac{\dd^3p}{(2\pi)^3} \frac{\ve{p}}{2E_\ve{p}} \sum_{s = 1,2} \sum_{r = 1,2} \left[ a_{\ve{p},s}^\dagger u^{s\dagger}(\ve{p}) e^{i E_\ve{p} t} + b_{-\ve{p},s} v^{s\dagger}(-\ve{p}) e^{-i E_\ve{p} t} \right] \times \\
        & \qquad \qquad \qquad \qquad \qquad \qquad \qquad \qquad \qquad \qquad \qquad \times \left[ a_{\ve{p},r} u^r(\ve{p}) e^{-i E_\ve{p} t} - b_{-\ve{p},r}^\dagger v^r(-\ve{p}) e^{i E_\ve{p} t} \right]
      \end{split}
    \end{equation*}
    Using Lemma \ref{lemma:haml-lem}, Eq. \ref{eq:1.97} and the antisymmetry $ \mathfrak{N} b_{\ve{p},s} b_{\ve{p},s}^\dagger = - b_{\ve{p},s}^\dagger b_{\ve{p},s} $:
    \begin{equation*}
      \ve{P} = \int \frac{\dd^3p}{(2\pi)^3} \sum_{s = 1,2} \ve{p} \mathfrak{N} \left[ a_{\ve{p},s}^\dagger a_{\ve{p},s} - b_{\ve{p},s} b_{\ve{p},s}^\dagger \right] = \int \frac{\dd^3p}{(2\pi)^3} \sum_{s = 1,2} \ve{p} \left( a_{\ve{p},s}^\dagger a_{\ve{p},s} + b_{\ve{p},s}^\dagger b_{\ve{p},s} \right)
    \end{equation*}
  \end{proof}
\end{proposition}

Thus, both $ a_{\ve{p},s}^\dagger $ and $ b_{\ve{p},s}^\dagger $ create particles with energy $ +E_\ve{p} $ and momentum $ \ve{p} $: these are respectively called \textit{fermions} and \textit{antifermions}.

\subsubsection{Quantum numbers}

Under a generic Lorentz transformation, the Dirac field $ \Psi $ transforms according to Eq. \ref{eq:1.50}:
\begin{equation}
  \Psi'(x) = e^{-\frac{i}{2} \omega_{\mu \nu} J^{\mu \nu}} \Psi(x) \simeq \left[ \tens{I}_4 - \frac{i}{2} \omega_{\mu \nu} J^{\mu \nu} \right] \Psi(x)
  \label{eq:2.25}
\end{equation}
where the Lorentz generator $ J^{\mu \nu} $ is defined in Eq. \ref{eq:1.51}.

\begin{proposition}{Angular momentum}{}
  The conserved charge associated to the infinitesimal rotation Eq. \ref{eq:2.25} is:
  \begin{equation}
    \ve{J} = \int \dd^3x\, \Psi^\dagger \left( \ve{x} \times (-i \bs{\nabla}) + \tfrac{1}{2} \bs{\Sigma} \right) \Psi
    \label{eq:2.26}
  \end{equation}

  \tcblower

  \begin{proof}
    Consider a rotation of $ \theta $ around the $ z $-axis: it is described by $ \omega_{12} = - \omega_{21} = \theta $, so, by Eq. \ref{eq:1.50}:
    \begin{equation*}
      \delta_0 \Psi = \theta \left( x^1 \pa^2 - x^2 \pa^1 - \tfrac{i}{2} \Sigma^3 \right) \Psi = - \theta \left( x^1 \pa_2 - x^2 \pa_1 + \tfrac{i}{2} \Sigma^3 \right) \Psi \equiv \theta \Delta \Psi
    \end{equation*}
    Using the same notation of Def. \ref{def:inf-trans}, $ \epsilon \equiv \theta $ and $ F = \Delta \Psi $, therefore, by Eq. \ref{eq:1.68}, the temporal component of the conserved Noether current is (without negative sign, as it is mathematically equivalent):
    \begin{equation*}
      j^0 = \frac{\pa \mathcal{L}_\text{D}}{\pa (\pa_0 \Psi)} \Delta \Psi = - i \Psi^\dagger \left( (\ve{x} \times \bs{\nabla})^3 + \tfrac{i}{2} \Sigma^3 \right) \Psi
    \end{equation*}
    As the associated Noether charge is $ J^3 = \int \dd^3x\, j^0 $, this can be generalized to rotations around the $ x $-axis and the $ y $-axis, yielding:
    \begin{equation*}
      \ve{J} = \int \dd^3\, \Psi^\dagger \left( \ve{x} \times (-i \bs{\nabla}) + \tfrac{1}{2} \bs{\Sigma} \right) \Psi
    \end{equation*}
  \end{proof}
\end{proposition}

For non-relativistic fermions, the first term gives the orbital angular momentum, while the second term gives the spin angular momentum. For relativistic fermions, this division is not straightforward.\\
To determine the spin of fermions, it is sufficient to consider them at rest. The spin along the $ z $-axis is given by the $ S_z $ operator (at $ t = 0 $):
\begin{equation*}
  \begin{split}
    S_z
    &= \int \dd^3x \int \int \frac{\dd^3p\, \dd^3q}{(2\pi)^6 \sqrt{4 E_\ve{p} E_\ve{q}}} e^{i (\ve{q} - \ve{p}) \cdot \ve{x}} \times \\
    & \qquad \qquad \times \sum_{t = 1,2} \sum_{r = 1,2} \left[ a_{\ve{p},t}^\dagger u^{t\dagger}(\ve{p}) + b_{-\ve{p},t} v^{t\dagger}(-\ve{p}) \right] \frac{\Sigma^3}{2} \left[ a_{\ve{q},r} u^r(\ve{q}) + b_{-\ve{q},r}^\dagger v^r(-\ve{q}) \right] \\
    &= \int \frac{\dd^3p}{(2\pi)^3 2E_\ve{p}} \sum_{t = 1,2} \sum_{r = 1,2} \left[ a_{\ve{p},t}^\dagger u^{t\dagger}(\ve{p}) + b_{-\ve{p},t} v^{t\dagger}(-\ve{p}) \right] \frac{\Sigma^3}{2} \left[ a_{\ve{p},r} u^r(\ve{p}) + b_{-\ve{p},r}^\dagger v^r(-\ve{p}) \right] \\
  \end{split}
\end{equation*}
Noting that $ S_z \ket{0} = 0 $ (by definition of vacuum), then $ S_z a_{\ve{0},s}^\dagger \ket{0} = \{S_z , a_{\ve{0},s}^\dagger\} \ket{0} $; the only non-zero terms are those proportional to $ u^\dagger \Sigma^3 u $ and $ v^\dagger \Sigma^3 v $, but the latter vanishes as $ \{a,b\} = 0 $, so the only term remaining is:
\begin{equation*}
  \{a_{\ve{p},t}^\dagger a_{\ve{p},r} , a_{\ve{0},s}^\dagger\} = a_{\ve{p},t}^\dagger \{a_{\ve{p},r} , a_{\ve{0},s}^\dagger\} = (2\pi)^3 \delta^{(3)}(\ve{p}) \delta_{rs} a_{\ve{p},t}^\dagger
\end{equation*}
The $ S_z $ operator thus acts as (recall Eq. \ref{eq:1.94} with $ p^\mu = (m,0,0,0) $):
\begin{equation*}
  \begin{split}
    S_z a_{\ve{0},s}^\dagger \ket{0}
    &= \frac{1}{2E_\ve{p}} \sum_{t = 1,2} u^{t\dagger}(\ve{0}) \frac{\Sigma^3}{2} u^s(\ve{0}) a_{\ve{0},t}^\dagger \ket{0} = \frac{1}{4} \left( 2\xi^{1\dagger} \sigma^3 \xi^s a_{\ve{0},1}^\dagger + 2\xi^{2\dagger} \sigma^3 \xi^s a_{\ve{0},2}^\dagger \right) \ket{0}
  \end{split}
\end{equation*}
Using $ \sigma^3 \xi^1 = + \xi^1 $ and $ \sigma^3 \xi^2 = - \xi^2 $, by $ \xi^{r\dagger} \xi^s = \delta^{rs} $ one gets:
\begin{equation*}
  S_z a_{\ve{0},1}^\dagger \ket{0} = +\frac{1}{2} a_{\ve{0},1}^\dagger \ket{0}
  \qquad \qquad
  S_z a_{\ve{0},2}^\dagger \ket{0} = -\frac{1}{2} a_{\ve{0},2}^\dagger \ket{0}
\end{equation*}
which means that fermions are spin-$ \frac{1}{2} $ particles, with $ a_{\ve{0},1}^\dagger $ creating $ s = +\frac{1}{2} $ fermions and $ a_{\ve{0},2}^\dagger $ creating $ s = -\frac{1}{2} $ fermions. Conversely, it is equivalent to show that $ b_{\ve{0},1}^\dagger $ creates $ s = -\frac{1}{2} $ antifermions and $ b_{\ve{0},2}^\dagger $ creates $ s = +\frac{1}{2} $ antifermions (as $ b $ and $ b^\dagger $ are not in normal order in $ S_z $, so there's an extra negative sign due to antisymmetry).\\
Another important conserved Noether charge of Dirac theory is that associated to the vector current $ j^\mu_\text{V} = \bar{\Psi} \gamma^\mu \Psi $ (recall Eq. \ref{eq:1.102}), which is (using $ j^0_\text{V} = \Psi^\dagger \Psi $):
\begin{equation}
  Q_{\Un{1}_\text{V}} = \int \frac{\dd^3p}{(2\pi)^3} \sum_{s = 1,2} \left( a_{\ve{p},s}^\dagger a_{\ve{p},s} - b_{\ve{p},s}^\dagger b_{\ve{p},s} \right)
  \label{eq:2.27}
\end{equation}
This means that fermions have $ \Un{1}_\text{V} $ charge of $ +1 $, while antifermions of $ -1 $.

\begin{table}[h!]
  \centering
  \begin{tabular}{ccc}
    \hline
    state & $ S_z $ & $ Q_{\Un{1}_\text{V}} $ \\
    \hline
    \rule{0pt}{3ex} $ a_{\ve{p},1}^\dagger \ket{0} $ & $ +\tfrac{1}{2} $ & $ +1 $ \\
    \rule{0pt}{3ex} $ a_{\ve{p},2}^\dagger \ket{0} $ & $ -\tfrac{1}{2} $ & $ +1 $ \\
    \rule{0pt}{3ex} $ b_{\ve{p},1}^\dagger \ket{0} $ & $ -\tfrac{1}{2} $ & $ -1 $ \\
    \rule{0pt}{3ex} $ b_{\ve{p},2}^\dagger \ket{0} $ & $ +\tfrac{1}{2} $ & $ -1 $ \\
  \end{tabular}
  \caption{Quantum numbers for fermions in the Dirac theory.}
  \label{tab:dirac-qn}
\end{table}

\subsection{Massless Weyl fields}

The quantization of massless Weyl fields follows immediately from that of Dirac fields, and its useful to use Dirac notation:
\begin{equation*}
  \Psi_\text{L} \equiv
  \begin{pmatrix}
    \psi_\text{L} \\ 0
  \end{pmatrix}
  \qquad \qquad
  \Psi_\text{R} \equiv
  \begin{pmatrix}
    0 \\ \psi_\text{R}
  \end{pmatrix}
\end{equation*}
Consider $ \Psi_\text{L} $. As for Eq. \ref{eq:2.20}:
\begin{equation*}
  \Psi_\text{L}(x) = \int \frac{\dd^3p}{(2\pi)^3 \sqrt{2E_\ve{p}}} \sum_{s = 1,2} \left[ a_{\ve{p},s} u_\text{L}^s(p) e^{-i p_\mu x^\mu} + b_{\ve{p},s}^\dagger v_\text{L}^s(p) e^{i p_\mu x^\mu} \right]_{p^0 = E_\ve{p}}
\end{equation*}
where the Dirac spinors now have the right-hand component vanishing, in the chiral representation. By Eqq. \ref{eq:1.93}-\ref{eq:1.94}-\ref{eq:1.95}, in the massless (ultra-relativistic) limit $ p^\mu = (E,0,0,E) $, so spinors with $ s = 1 $ have only the right-handed component, while those with $ s = 2 $ only the left-handed one: therefore, only $ s = 2 $ spinors contribute to $ \Psi_\text{L} $ (and only $ s = 1 $ to $ \Psi_\text{R} $).\\
By Tab. \ref{tab:dirac-qn}, it is clear that in this context $ a_{\ve{p},2}^\dagger $ creates a particle with helicity $ h = - \frac{1}{2} $, while $ b_{\ve{p},2}^\dagger $ creates an antiparticle with $ h = + \frac{1}{2} $: in general, a left-handed massless Weyl field describes particles with $ h = - \frac{1}{2} $ and antiparticles with $ h = + \frac{1}{2} $, while a right-handed one describes particles with $ h = + \frac{1}{2} $ and antiparticles with $ h = - \frac{1}{2} $.

\subsection{Discrete symmetries of fermionic fields}

\subsubsection{Parity}

Under parity $ \mathcal{P} \ve{p} = - \ve{p} $ and $ \mathcal{P} s = s $, so, for a generic particle of type $ a $:
\begin{equation}
  \mathcal{P} \ket{a; \ve{p},s} = \eta_a \ket{a; -\ve{p},s}
  \label{eq:2.28}
\end{equation}
where $ \eta_a $ is a generic constant phase factor, since states in the Fock space which differ by a phase still represent the same physical state. As $ \mathcal{P}^2 = \id $, this means that $ \eta_a^2 = \pm 1 $, as observables are built from an even number of fermionic ladder operators.

\paragraph{Non-Majorana fermions}

It is possible to prove that, for non-Majorana spin-$ \frac{1}{2} $ fermions, it is possible to redefine $ \mathcal{P} $ so that $ \eta_a = +1 $, i.e. $ \eta_a = \pm 1 $.

\begin{proposition}{}{}
  \begin{equation}
    \mathcal{P} a_{\ve{p},s}^\dagger \mathcal{P} = \eta_a a_{-\ve{p},s}^\dagger
    \qquad \qquad
    \mathcal{P} b_{\ve{p},s}^\dagger \mathcal{P} = \eta_b b_{-\ve{p},s}^\dagger
    \label{eq:2.29}
  \end{equation}

  \tcblower

  \begin{proof}
    For a multiparticle state one must have:
    \begin{equation*}
      \mathcal{P} a_{\ve{p},s}^\dagger b_{\ve{q},r}^\dagger \ket{0} = \eta_a \eta_b a_{-\ve{p},s}^\dagger b_{-\ve{q},r}^\dagger \ket{0}
    \end{equation*}
    Therefore:
    \begin{equation*}
      \mathcal{P} a_{\ve{p},s}^\dagger = \eta_a a_{-\ve{p},s}^\dagger \mathcal{P}
      \qquad \qquad
      \mathcal{P} b_{\ve{p},s}^\dagger = \eta_b b_{-\ve{p},s}^\dagger \mathcal{P}
    \end{equation*}
    Using $ \mathcal{P}^2 = \id $ yields the thesis.
  \end{proof}
\end{proposition}

According to Wigner's theorem (Th. \ref{th:wigner}), this symmetry can be represented by a unitary operator, i.e. $ \mathcal{P}^\dagger = \mathcal{P}^{-1} $, but $ \mathcal{P}^2 = \id $, therefore $ \mathcal{P}^\dagger = \mathcal{P} $. Then, Eq. \ref{eq:2.29} is valid for $ a_{\ve{p},s} , b_{\ve{p},s} $ too, thus parity acts as:
\begin{equation}
  \Psi(x) \mapsto \Psi'(x') = \mathcal{P} \Psi(x) \mathcal{P}
  \label{eq:2.30}
\end{equation}
Explicitly:
\begin{equation*}
  \begin{split}
    \mathcal{P} \Psi(x) \mathcal{P}
    &= \int \frac{\dd^3p}{(2\pi)^3 \sqrt{2E_\ve{p}}} \sum_{s = 1,2} \left[ \eta_a a_{-\ve{p},s} u^s(p) e^{-i p_\mu x^\mu} + \eta_b b_{-\ve{p},s}^\dagger v^s(p) e^{i p_\mu x^\mu} \right]_{p^0 = E_\ve{p}} \\
    &= \int \frac{\dd^3p}{(2\pi)^3 \sqrt{2E_\ve{p}}} \sum_{s = 1,2} \left[ \eta_a a_{-\ve{p},s} u^s(p) e^{-i E_\ve{p} t + i \ve{p} \cdot \ve{x}} + \eta_b b_{-\ve{p},s}^\dagger v^s(p) e^{i E_\ve{p} t - i \ve{p} \cdot \ve{x}} \right]_{p^0 = E_\ve{p}} \\
  \end{split}
\end{equation*}
Setting $ \ve{p}' \equiv - \ve{p} , \ve{x}' \equiv -\ve{x} $ and noting that, by Eqq. \ref{eq:1.94}-\ref{eq:1.95} $ \ve{p} \mapsto -\ve{p} $ exchanges the left-handed and the right-handed compontents of the spinors, i.e. $ u^s(p) \mapsto \gamma^0 u^s(p) $ and $ v^s(p) \mapsto - \gamma^0 v^s(p) $:
\begin{equation*}
  \begin{split}
    \mathcal{P} \Psi(x) \mathcal{P}
    &= \int \frac{\dd^3p'}{(2\pi)^3 \sqrt{E_{\ve{p}'}}} \sum_{s = 1,2} \left[ \eta_a a_{\ve{p}',s} \gamma^0 u^s(p') e^{-i E_{\ve{p}'} t + i \ve{p}' \cdot \ve{x}'} - \eta_b b_{\ve{p}',s}^\dagger \gamma^0 v^s(p') e^{i E_{\ve{p}'} t - i \ve{p}' \cdot \ve{x}'} \right]_{p'^0 = E_{\ve{p}'}} \\
    &= \gamma^0 \int \frac{\dd^3p}{(2\pi)^3 \sqrt{2E_\ve{p}}} \sum_{s = 1,2} \left[ \eta_a a_{\ve{p},s} u^s(p) e^{-i p_\mu x'^\mu} - \eta_b b_{\ve{p},s}^\dagger v^s(p) e^{i p_\mu x'^\mu} \right]_{p^0 = E_\ve{p}}
  \end{split}
\end{equation*}
Requiring that $ \Psi $ is a representation of parity, up to a phase, means that:
\begin{equation}
  \eta_a = - \eta_b
  \label{eq:2.31}
\end{equation}
so that:
\begin{equation}
  \mathcal{P} \Psi(t,\ve{x}) \mathcal{P} = \eta_a \gamma^0 \Psi(t,-\ve{x})
  \label{eq:2.32}
\end{equation}
This, in the chiral representation, agrees with the fact that parity exchanges left-handed and right-handed Weyl spinors. The $ \eta_a $ factor cancels in any fermion bilinear involving only one type of particles; however, the relative phase factors of different particles can be observed: in particular, the opposite intrinsic parity of fermions and antifermions.

\paragraph{Spin-$ 0 $ bosons}

As already noted, the scalar complex field is similar to the Dirac field, apart for the absence of spinors in the expansions of $ \phi(x) $: in particular, this means that scalar fields have no relative negative signe between $ \eta_a $ and $ \eta_b $, so a quantized complex scalar field gives a representation of parity if $ \eta_a = \eta_b $, and the intrinsic parity of spin-$ 0 $ particle and antiparticles is equal.

\subsubsection{Charge conjugation}

Recall Eq. \ref{eq:1.43}: charge conjugation acts on the classical Dirac field as $ \Psi \mapsto -i \gamma^2 \Psi^* $.

\begin{definition}{Quantized charge conjugation}{}
  The \textit{charge conjugation operator} is defined as:
  \begin{equation}
    \mathcal{C} a_{\ve{p},s} \mathcal{C} = \eta_c b_{\ve{p},s}
    \qquad \qquad
    \mathcal{C} b_{\ve{p},s} \mathcal{C} = \eta_c a_{\ve{p},s}
    \label{eq:2.33}
  \end{equation}
  with $ \eta_c = \pm 1 $ for simplicity.
\end{definition}

Thus $ \mathcal{C}^2 = \id $ too, and its physical interpretation is the exchange of particles and antiparticles, while leaving $ \ve{p} $ and $ s $ unchanged: as $ a_{\ve{p},s} $ and $ b_{\ve{p},s} $ create particles with opposite spin, this means that charge conjugation reverses the helicity of particles.

\begin{lemma}{Charge conjugation on spinors}{}
  \begin{equation}
    \mathcal{C} u^2(p) \mathcal{C} = -i \gamma^2 \left[ v^s(p) \right]^*
    \qquad \qquad
    \mathcal{C} v^2(p) \mathcal{C} = -i \gamma^2 \left[ u^s(p) \right]^*
    \label{eq:2.34}
  \end{equation}
\end{lemma}

\begin{proposition}{Charge conjugation on Dirac fields}{}
  Given a Dirac field $ \Psi(x) $:
  \begin{equation}
    \mathcal{C} \Psi(x) \mathcal{C} = -i \eta_c \gamma^2 [\Psi(x)]^*
    \label{eq:2.35}
  \end{equation}

  \tcblower

  \begin{proof}
    Using Eqq. \ref{eq:2.33}-\ref{eq:2.34}:
    \begin{equation*}
      \begin{split}
        \mathcal{C} \Psi(x) \mathcal{C}
        &= \eta_c \int \frac{\dd^3p}{(2\pi)^3 \sqrt{2E_\ve{p}}} \sum_{s = 1,2} \left[ b_{\ve{p},s} \mathcal{C} u^s(p) \mathcal{C} e^{-i p_\mu x^\mu} + a_{\ve{p},s}^\dagger \mathcal{C} v^s(p) \mathcal{C} e^{i p_\mu x^\mu} \right] \\
        &= -i \eta_c \gamma^2 \int \frac{\dd^3p}{(2\pi)^3 \sqrt{2E_\ve{p}}} \sum_{s = 1,2} \left[ b_{\ve{p},s} [v^s(p)]^* e^{-i p_\mu x^\mu} + a_{\ve{p},s}^\dagger [u^s(p)]^* e^{i p_\mu x^\mu} \right] = -i \eta_c \gamma^2 [\Psi(x)]^*
      \end{split}
    \end{equation*}
  \end{proof}
\end{proposition}

As for parity, the transformation of quantized fields is analogous to that of classical fields, with an additional quantum phase factor which depends on the particle type.

\begin{proposition}{Charge conjugation of the vector current}{}
  \begin{equation}
    \mathcal{C} \left( \bar{\Psi} \gamma^\mu \Psi \right) \mathcal{C} = - \bar{\Psi} \gamma^\mu \Psi
    \label{eq:2.36}
  \end{equation}
\end{proposition}

\subsubsection{Time reversal}

\begin{theorem}{Anti-unitary time reversal}{anti-uni-time-rev}
  Time reversal cannot be implemented as a linear unitary operator.

  \tcblower

  \begin{proof}
    Assume that the time reversal operator $ \mathcal{T} $ is linear and unitary; then, as it must be a symmetry of the free Dirac Lagrangian, $ [\mathcal{T} , H] = 0 $, so:
    \begin{equation*}
      \begin{split}
        \mathcal{T} \Psi(t,\ve{x}) \mathcal{T} \ket{0}
        &= \mathcal{T} e^{i H t} \Psi(\ve{x}) e^{-i H t} \mathcal{T} \ket{0} = e^{i H t} \mathcal{T} \Psi(\ve{x}) \mathcal{T} e^{-i H t} \ket{0} = e^{i H t} \mathcal{T} \Psi(\ve{x}) \mathcal{T} \ket{0} \\
        &= \Psi(-t,\ve{x}) \ket{0} = e^{-i H t} \Psi(\ve{x}) e^{i H t} \ket{0} = e^{-i H t} \Psi(\ve{x}) \ket{0}
      \end{split}
    \end{equation*}
    assuming $ H \ket{0} = 0 $. But the first line is a sum of negative frequencies only, while the second line is a sum of positive frequencies only, yielding an absurdum.
  \end{proof}
\end{theorem}

Time reversal is an example of operator which, according to Wigner's theorem (Th. \ref{th:wigner}), is represented as an anti-linear anti-unitary operator, i.e. an operator such that:
\begin{equation*}
  \braket{\mathcal{T} a | \mathcal{T} b} = \braket{a | b}^*
  \qquad \qquad
  \mathcal{T} \lambda \ket{a} = \lambda^* \mathcal{T} \ket{a}
  \qquad \qquad
  \forall \ket{a},\ket{b} \in \mathscr{H}\,,\,\forall \lambda \in \C
\end{equation*}
In particular, anti-linearity (second property) solves the absurdum in the proof of Th. \ref{th:anti-uni-time-rev}.\\
Time reversal is expected to reverse the spin of particles, and this can be used to construct $ \mathcal{T} $. First, define:
\begin{equation}
  \xi^{-s} \equiv -i \sigma^2 [\xi^s]^*
  \label{eq:2.37}
\end{equation}
that is, $ \xi^{-1} = \binom{0}{1} $ and $ \xi^{-2} = \binom{-1}{0} $. This allows redefining $ \eta^s \equiv \xi^{-s} $, and also $ \xi^{-(-s)} = - \xi^s $.

\begin{lemma}{Reversed spinors}{rev-spinors}
  \begin{equation}
    u^{-s}(-\ve{p}) = - \gamma^1 \gamma^3 [u^s(\ve{p})]^*
    \qquad \qquad
    v^{-s}(-\ve{p}) = - \gamma^1 \gamma^3 [v^s(\ve{p})]^*
    \label{eq:2.38}
  \end{equation}

  \tcblower

  \begin{proof}
    Defining $ \tilde{p}^\mu \equiv (p^0, -\ve{p}) $ and recalling that $ \bs{\sigma} \sigma^2 = - \sigma^2 \bs{\sigma}^* $:
    \begin{equation*}
      u^{-s}(-\ve{p}) =
      \begin{pmatrix}
        \sqrt{\tilde{p}^\mu \sigma_\mu} (-i \sigma^2 [\xi^s]^*) \\
        \sqrt{\tilde{p}^\mu \bar{\sigma}_\mu} (-i \sigma^2 [\xi^s]^*)
      \end{pmatrix}
      =
      \begin{pmatrix}
        -i \sigma^2 \sqrt{p^\mu \sigma_\mu^*} [\xi^s]^* \\
        -i \sigma^2 \sqrt{p^\mu \bar{\sigma}_\mu^*} [\xi^s]^*
      \end{pmatrix}
      = -i \begin{pmatrix} \sigma^2 & 0 \\ 0 & \sigma^2 \end{pmatrix} [u^s(\ve{p})]^*
    \end{equation*}
    noting that $ i \diag(\sigma^2,\sigma^2) = \gamma^1 \gamma^3 $ completes the proof. On the other hand:
    \begin{equation*}
      \begin{split}
        v^{-s}(-\ve{p})
        &=
        \begin{pmatrix}
          \sqrt{\tilde{p}^\mu \sigma_\mu} (-\xi^s) \\
          - \sqrt{\tilde{p}^\mu \bar{\sigma}_\mu} (-\xi^s)
        \end{pmatrix}
        = 
        \begin{pmatrix}
          \sigma^2 \sqrt{p^\mu \sigma_\mu^*} \sigma^2 (-\xi^s) \\
          - \sigma^2 \sqrt{p^\mu \bar{\sigma}_\mu^*} \sigma^2 (-\xi^s)
        \end{pmatrix}\\
        &= -i \begin{pmatrix} \sigma^2 & 0 \\ 0 & \sigma^2 \end{pmatrix}
        \begin{pmatrix}
          \sqrt{p^\mu \sigma_\mu^*} (-i \sigma^2 [\xi^s]^*) \\
          - \sqrt{p^\mu \bar{\sigma}_\mu^*} (-i \sigma^2 [\xi^s]^*)
        \end{pmatrix}
        = - \gamma^1 \gamma^3 [v^s(-\ve{p})]^*
      \end{split}
    \end{equation*}
  \end{proof}
\end{lemma}

It is useful to define the reversed ladder operators:
\begin{equation}
  a_{\ve{p},-s} \equiv (a_{\ve{p},2} , -a_{\ve{p},1})
  \qquad \qquad
  b_{\ve{p},-s} \equiv (b_{\ve{p},2} , -b_{\ve{p},1})
  \label{eq:2.39}
\end{equation}

\begin{definition}{Time reversal}{}
  The \textit{time reversal operator} is defined as:
  \begin{equation}
    \mathcal{T} a_{\ve{p},s} \mathcal{T} = a_{-\ve{p},-s}
    \qquad \qquad
    \mathcal{T} b_{\ve{p},s} \mathcal{T} = b_{-\ve{p},-s}
    \label{eq:2.40}
  \end{equation}
\end{definition}

An additional overall phase is irrelevant. Moreover, as other discrete symmetries, $ \mathcal{T}^2 = \id $.

\begin{proposition}{Time reversal on Dirac fields}{}
  Given a Dirac field $ \Psi(x) $:
  \begin{equation}
    \mathcal{T} \Psi(t,\ve{x}) \mathcal{T} = - \gamma^1 \gamma^3 \Psi(-t,\ve{x})
    \label{eq:2.41}
  \end{equation}

  \begin{proof}
    Using Lemma \ref{lemma:rev-spinors} and defining $ \tilde{p}^\mu \equiv (p^0,-\ve{p}) , \tilde{x}^\mu \equiv (-t,\ve{x}) $, so that $ p^\mu x_\mu = -\tilde{p}^\mu \tilde{x_\mu} $:
    \begin{equation*}
      \begin{split}
        \mathcal{T} \Psi(x) \mathcal{T}
        &= \int \frac{\dd^3p}{(2\pi)^3 \sqrt{2E_\ve{p}}} \sum_{s = 1,2} \mathcal{T} \left[ a_{\ve{p},s} u^s(p) e^{-i p_\mu x^\mu} + b_{\ve{p},s}^\dagger v^s(p) e^{i p_\mu x^\mu} \right] \mathcal{T} \\
        &= \int \frac{\dd^3p}{(2\pi)^3 \sqrt{2E_\ve{p}}} \sum_{s = 1,2} \left[ a_{-\ve{p},-s} [u^s(p)]^* e^{i p_\mu x^\mu} + b_{-\ve{p},-s}^\dagger [v^s(p)]^* e^{-i p_\mu x^\mu} \right] \\
        &= - \gamma^3 \gamma^1 \int \frac{\dd^3p}{(2\pi)^3 \sqrt{2E_\ve{p}}} \sum_{s = 1,2} \left[ a_{\tilde{\ve{p}},-s} u^{-s}(\tilde{p}) e^{-i \tilde{p}_\mu \tilde{x}^\mu} + b_{\tilde{\ve{p}},-s}^\dagger v^{-s}(\tilde{p}) e^{i \tilde{p}_\mu \tilde{x}^\mu} \right] \\
        &= \gamma^1 \gamma^3 \int \frac{-\dd^3\tilde{p}}{(2\pi)^3 \sqrt{2E_{\tilde{\ve{p}}}}} \sum_{s = 1,2} \left[ a_{\tilde{\ve{p}},s} u^s(\tilde{p}) e^{-i \tilde{p}_\mu \tilde{x}^\mu} + b_{\tilde{\ve{p}},s}^\dagger v^s(\tilde{p}) e^{i \tilde{p}_\mu \tilde{x}^\mu} \right]
        = - \gamma^1 \gamma^3 \Psi(\tilde{x})
      \end{split}
    \end{equation*}
  \end{proof}
 \end{proposition}

 \subsubsection{CPT symmetry}

 In order to study the invariance properties of fermionic Lagrangians, it is necessary to state the transformation relations of fermion bilinears.

 \begin{table}[h!]
  \centering
  \begin{tabular}{ccccccc}
    \hline
    \rule{0pt}{2.5ex} & $ \bar{\Psi} \Psi $ & $ i \bar{\Psi} \gamma^5 \Psi $ & $ \bar{\Psi} \gamma^\mu \Psi $ & $ \bar{\Psi} \gamma^\mu \gamma^5 \Psi $ & $ \bar{\Psi} \sigma^{\mu \nu} \Psi $ & $ \pa_\mu $ \\
    \hline
    \rule{0pt}{2.5ex} $ \mathcal{C} $ & $ +1 $ & $ +1 $ & $ -1 $ & $ +1 $ & $ -1 $ & $ +1 $ \\
    \rule{0pt}{2.5ex} $ \mathcal{P} $ & $ +1 $ & $ -1 $ & $ (-1)^\mu $ & $ - (-1)^\mu $ & $ (-1)^\mu (-1)^\nu $ & $ (-1)^\mu $ \\
    \rule{0pt}{2.5ex} $ \mathcal{T} $ & $ +1 $ & $ -1 $ & $ (-1)^\mu $ & $ (-1)^\mu $ & $ - (-1)^\mu (-1)^\nu $ & $ - (-1)^\mu $ \\
    \rule{0pt}{2.5ex} $ \mathcal{CPT} $ & $ +1 $ & $ +1 $ & $ -1 $ & $ -1 $ & $ +1 $ & $ -1 $
  \end{tabular}
  \caption{Eigenvalues of fermion bilinears and derivative operator, with $ (-1)^\mu \equiv (+1,-1,-1,-1) $.}
  \label{tab:ferm-bil-eigen}
 \end{table}

 This shows that it is not possible to construct a Lorentz-invariant Lagrangian which violates CPT simmetry: this is an example of the \textit{CPT theorem}, which states that, independently of the spin of the particle, a (locale) Lorentz-invariant field theory with a hermitian Hamiltonian cannot violate CPT symmetry.\\
 A consequence of this theorem is that particles and antiparticles have exactly the same mass, which has been so far empirically confirmed.










