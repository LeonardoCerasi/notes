\selectlanguage{english}

\section{Tree-level amplitudes}

\subsection{\texorpdfstring{$ e^+ e^- \rightarrow \mu^+ \mu^- $}{e+e- --> µ+µ-} scattering}

The $ e^+ e^- \rightarrow \mu^+ \mu^- $ scattering is the simplest process in QED and one of the most important in high-energy physics, as it is used to calibrate $ e^+ e^- $ colliders. \\
Recalling the Feynman rules for QED from \secref{sssec:qed-feyn}, the amplitude can be found from the only Feynman diagram of the process\footnotemark:
\begin{equation*}
  \begin{tikzpicture}[baseline=(r.base)]
    \begin{feynman}[inline=(r.base)]

      \vertex[dot] (a1) {};
      \vertex (l1) at ($(a1)+(-0.4,0)$) {\(e^-\)};

      \vertex[right=8cm of a1, dot] (a2) {};
      \vertex (l2) at ($(a2)+(+0.4,0)$) {\(\mu^-\)};

      \vertex[below=6em of a1, dot] (b1) {};
      \vertex (l3) at ($(b1)+(-0.4,0)$) {\(e^+\)};

      \vertex[below=6em of a2, dot] (b2) {};
      \vertex (l4) at ($(b2)+(+0.4,0)$) {\(\mu^+\)};

      \vertex[below=3em of a1] (c1) {};

      \vertex[right=2cm of c1, dot] (c2) {};

      \vertex[right=4cm of c2, dot] (c3) {};

      \vertex[below=1.1em of a1] (r);

      \diagram* {
        (a1) -- [fermion, momentum = \(p_1\)] (c2),
        (b1) -- [anti fermion, momentum' = \(p_2\)] (c2),

        (c3) -- [fermion, momentum = \(k_1\)] (a2),
        (c3) -- [anti fermion, momentum' = \(k_2\)] (b2),

        (c2) -- [photon, momentum = \(q\)] (c3),
      };
    \end{feynman}
  \end{tikzpicture}
\end{equation*}
with $ q = p_1 + p_2 = k_1 + k_2 $ by momentum conservation. By \eref{eq:mat-el}, then:
\begin{equation*}
  i \mathcal{M} = \bar{u}^{s_1}(k_1) (-i e \gamma^\mu) v^{s_2}(k_2) \frac{-i \eta_{\mu \nu}}{q^2 + i \epsilon} \bar{v}^{r_2}(p_2) (-i e \gamma^\nu) u^{r_1}(p_1) = \frac{i e^2}{q^2} \bar{u}^{s_1}(k_1) \gamma^\mu v^{s_2}(k_2) \bar{v}^{r_2}(p_2) \gamma_\mu u^{r_1}(p_1)
\end{equation*}

\footnotetext{Note that, if $ \ell \bar{\ell} \rightarrow \ell \bar{\ell} $, with $ \ell = e^-, \mu^- $, was considered instead, there would be a second Feynman diagram to computed:
\begin{equation*}
  \begin{tikzpicture}[baseline=(r.base)]
    \begin{feynman}[inline=(r.base)]

      \vertex[dot] (a1) {};
      \vertex (l1) at ($(a1)+(-0.4,0)$) {\(\ell\)};

      \vertex[right=8cm of a1, dot] (a2) {};
      \vertex (l2) at ($(a2)+(+0.4,0)$) {\(\bar{\ell}\)};

      \vertex[below=6em of a1, dot] (b1) {};
      \vertex (l3) at ($(b1)+(-0.4,0)$) {\(\ell\)};

      \vertex[below=6em of a2, dot] (b2) {};
      \vertex (l4) at ($(b2)+(+0.4,0)$) {\(\bar{\ell}\)};

      \vertex[below=3em of a1] (c1) {};

      \vertex[right=2cm of c1, dot] (c2) {};

      \vertex[right=4cm of c2, dot] (c3) {};

      \vertex[below=1.1em of a1] (r);

      \diagram* {
        (a1) -- [fermion, momentum = \(p_1\)] (c2),
        (c2) -- [fermion, momentum = \(k_1\)] (b1),

        (a2) -- [anti fermion, momentum' = \(p_2\)] (c3),
        (c3) -- [anti fermion, momentum' = \(k_2\)] (b2),

        (c2) -- [photon, momentum = \(q\)] (c3),
      };
    \end{feynman}
  \end{tikzpicture}
\end{equation*}
with $ q = p_1 - k_1 = k_2 - p_2 $.
}

\begin{lemma}[before upper = {\tcbtitle}]{Bi-spinor products}{}
  \begin{equation}
    (\bar{u} \gamma^\mu v)^* = \bar{v} \gamma^\mu u
  \end{equation}
\end{lemma}

\begin{proofbox}
  \begin{proof}
    $ (\bar{u} \gamma^\mu v)^* = v\dg (\gamma^\mu)\dg \bar{u}\dg = v\dg (\gamma^\mu)\dg (u\dg \gamma^0)\dg = v\dg (\gamma^\mu)\dg \gamma^0 u = v\dg \gamma^0 \gamma^\mu u = \bar{v} \gamma^\mu u $.
  \end{proof}
\end{proofbox}

With this lemma, it is easy to see that (as $ \bar{u} \gamma^\mu v \in \C $):
\begin{equation*}
  \begin{split}
    \abs{\mathcal{M}_{r_1,r_2,s_1,s_2}}^2
    & = \frac{e^4}{q^4} \bar{u}^{s_1}(k_1) \gamma^\mu v^{s_2}(k_2) \bar{v}^{r_2}(p_2) \gamma_\mu u^{r_1}(p_1) \bar{v}^{s_2}(k_2) \gamma^\nu u^{s_1}(k_1) \bar{u}^{r_1}(p_1) \gamma_\nu v^{r_2}(p_2) \\
    & = \frac{e^4}{q^4} \bar{u}^{s_1}(k_1) \gamma^\mu v^{s_2}(k_2) \bar{v}^{s_2}(k_2) \gamma^\nu u^{s_1}(k_1) \bar{v}^{r_2}(p_2) \gamma_\mu u^{r_1}(p_1) \bar{u}^{r_1}(p_1) \gamma_\nu v^{r_2}(p_2)
  \end{split}
\end{equation*}
This matrix element has explicit dependence on spin states: although not impossible, it is difficult to experimentally retain control over them, both in the initial- and final-state. For this reason, it is convenient to consider a matrix element which is averaged over initial spin states and summed over final spin states (as usually the detector does not distinguish them):
\begin{equation}
  \abs{\mathcal{M}}^2 = \frac{1}{2} \sum_{r_1 = 1,2} \frac{1}{2} \sum_{r_2 = 1,2} \sum_{s_1} \sum_{s_2 = 1,2} \abs{\mathcal{M}_{r_1,r_2,s_1,s_2}}^2
\end{equation}
Recalling \eref{eq:spinor-sums} and writing spinor indices explicitly, the first half of the matrix element is:
\begin{multline*}
  \frac{e^4}{4q^4} \sum_{s_1,s_2 = 1,2} \sum_{a,b,c,d = 1}^{4} [\bar{u}^{s_1}(k_1)]_a \gamma^\mu_{ab} [v^{s_2}(k_2)]_b [\bar{v}^{s_2}(k_2)]_c \gamma^\nu_{cd} [u^{s_1}(k_1)]_d \\
  = \frac{e^4}{4q^4} \sum_{s_1,s_2 = 1,2} \sum_{a,b,c,d = 1}^{4} [\slashed{k}_1 + m_\mu]_{da} \gamma^\mu_{ab} [\slashed{k}_2 - m_\mu]_{bc} \gamma^\nu_{cd} = \frac{e^4}{4q^4} \sum_{s_1,s_2 = 1,2} \tr\{(\slashed{k}_1 + m_\mu) \gamma^\mu (\slashed{k}_2 - m_\mu) \gamma^\nu\}
\end{multline*}
The unpolarized matrix element then is:
\begin{equation*}
  \abs{\mathcal{M}}^2 = \frac{e^4}{4q^4} \sum_{r_1,r_2,s_1,s_2 = 1,2} \tr\{(\slashed{k}_1 + m_\mu) \gamma^\mu (\slashed{k}_2 - m_\mu) \gamma^\nu\} \tr\{(\slashed{p}_1 + m_e) \gamma_\mu (\slashed{p}_2 - m_e) \gamma_\nu\}
\end{equation*}
This approach is general: any QED amplitude involving external fermions can be simplified into traces of products of gamma matrices, when squared and summed/averaged over spin states, hence eliminating explicit dependence on spinors.

\subsubsection{Traces}

Due to their importance in QED, traces of gamma matrices should be studied systematically.

\begin{proposition}{Odd number of gamma matrices}{odd-gamma}
  For any $ n \in \N $ odd:
  \begin{equation}
    \tr\{\gamma^{\mu_1} \dots \gamma^{\mu_n}\} = 0
  \end{equation}
\end{proposition}

\begin{proofbox}
  \begin{proof}
    As $ \{\gamma^\mu , \gamma^5\} = 0 $ and given the cyclic property of the trace:
    \begin{equation*}
      \begin{split}
        \tr\{\gamma^{\mu_1} \dots \gamma^{\mu_n}\}
        & = \tr\{\gamma^5 \gamma^5 \gamma^{\mu_1} \dots \gamma^{\mu_n}\} = - \tr\{\gamma^5 \gamma^{\mu_1} \gamma^5 \dots \gamma^{\mu_n}\} \\
        & = (-1)^n \tr\{\gamma^5 \gamma^{\mu_1} \dots \gamma^{\mu_n} \gamma^5\} = (-1)^n \tr\{\gamma^5 \gamma^5 \gamma^{\mu_1} \dots \gamma^{\mu_n}\} = (-1)^n \tr\{\gamma^{\mu_1} \dots \gamma^{\mu_n}\}
      \end{split}
    \end{equation*}
    Therefore, the trace vanishes for any odd $ n \in \N $.
  \end{proof}
\end{proofbox}

\begin{proposition}[before upper = {\tcbtitle}]{Even number of gamma matrices}{}
  \begin{equation}
    \tr\{\gamma^\mu \gamma^\nu\} = 4 \eta^{\mu \nu}
  \end{equation}
  \begin{equation}
    \tr\{\gamma^\mu \gamma^\nu \gamma^\sigma \gamma^\rho\} = 4 \left( \eta^{\mu \nu} \eta^{\sigma \rho} - \eta^{\mu \sigma} \eta^{\nu \rho} + \eta^{\mu \rho} \eta^{\nu \sigma} \right)
  \end{equation}
\end{proposition}

\begin{proofbox}
  \begin{proof}
    Using $ \{\gamma^\mu , \gamma^\nu\} = 2 \eta^{\mu \nu} \tens{I}_4 $:
    \begin{equation*}
      \tr\{\gamma^\mu \gamma^\nu\} = \tr\{ 2\eta^{\mu \nu} \tens{I}_4 - \gamma^\nu \gamma^\mu\} = 8 \eta^{\mu \nu} - \tr\{\gamma^\mu \gamma^\nu\}
    \end{equation*}
    \begin{equation*}
      \begin{split}
        \tr\{\gamma^\mu \gamma^\nu \gamma^\sigma \gamma^\rho\}
        & = \tr\{2\eta^{\mu \nu} \gamma^\sigma \gamma^\rho - \gamma^\nu \gamma^\mu \gamma^\sigma \gamma^\rho\} \\
        & = \tr\{2\eta^{\mu \nu} \gamma^\sigma \gamma^\rho - \gamma^\nu 2 \eta^{\mu \sigma} \gamma^\rho + \gamma^\nu \gamma^\sigma 2 \eta^{\mu \rho} - \gamma^\nu \gamma^\sigma \gamma^\rho \gamma^\mu\} \\
        & = 4 \left( \eta^{\mu \nu} \eta^{\sigma \rho} - \eta^{\mu \sigma} \eta^{\nu \rho} + \eta^{\mu \rho} \eta^{\nu \sigma} \right) - \tr\{\gamma^\mu \gamma^\nu \gamma^\sigma \gamma^\rho\}
      \end{split}
    \end{equation*}
  \end{proof}
\end{proofbox}

In general, the trace of the product of $ n $ gamma matrices can be reduced to traces of products of $ n-2 $ gamma matrices. Traces involving $ \gamma^5 \defeq i \gamma^0 \gamma^1 \gamma^2 \gamma^3 $ are of interest too.

\begin{proposition}{Odd number of gamma matrices}{}
  For any $ n \in \N $ odd:
  \begin{equation}
    \tr\{\gamma^{\mu_1} \dots \gamma^{\mu_n} \gamma^5\} = 0
  \end{equation}
\end{proposition}

\begin{proofbox}
  \begin{proof}
    As in the proof of \pref{prop:odd-gamma}:
    \begin{equation*}
      \tr\{\gamma^{\mu_1} \dots \gamma^{\mu_n} \gamma^5\} = (-1)^n \tr\{\gamma^5 \gamma^{\mu_1} \dots \gamma^{\mu_n} \gamma^5 \gamma^5\} = (-1)^n \tr\{\gamma^{\mu_1} \dots \gamma^{\mu_n} \gamma^5\}
    \end{equation*}
    which vanishes for any odd $ n \in \N $.
  \end{proof}
\end{proofbox}

\begin{proposition}[before upper = {\tcbtitle}]{Even number of gamma matrices}{}
  \begin{equation}
    \tr \gamma^5 = \tr\{\gamma^\mu \gamma^\nu \gamma^5\} = 0
  \end{equation}
  \begin{equation}
    \tr\{\gamma^\mu \gamma^\nu \gamma^\sigma \gamma^\rho \gamma^5\} = -4i \epsilon^{\mu \nu \sigma \rho}
  \end{equation}
\end{proposition}










