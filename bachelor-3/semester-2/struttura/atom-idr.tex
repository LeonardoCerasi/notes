\selectlanguage{italian}

\section{Principio variazionale di Ritz}

Per lo studio di un sistema quantistico, è utile dimostrare un principio variazionale sul valore di aspettazione dell'Hamiltoniana $ \mathcal{H} $ del sistema, ovvero sull'energia:
\begin{equation}
	E[\psi] \defeq \frac{\braket{\psi | \mathcal{H} | \psi}}{\braket{\psi | \psi}} \in \R
	\label{eq:1.1}
\end{equation}

\begin{proposition}{}{}
	Il valore di aspettazione di una Hamiltoniana su un suo autostato è stazionario.

	\tcblower

	\begin{proof}
		Prendendo una variazione infinitesima $ \ket{\psi + \delta\psi} $ ed usando $ \mathcal{H} \ket{\psi} = E[\psi] \ket{\psi} $:
		\begin{equation*}
			\begin{split}
				\delta E
				&= E[\psi + \delta\psi] - E[\psi] = \frac{\braket{\psi + \delta\psi | \mathcal{H} | \psi + \delta\psi}}{\braket{\psi + \delta\psi | \psi + \delta\psi}} - \frac{\braket{\psi | \mathcal{H} | \psi}}{\braket{\psi | \psi}} \\
				&\simeq \frac{\braket{\psi | \mathcal{H} | \psi} + \braket{\psi | \mathcal{H} | \delta\psi} + \braket{\delta\psi | \mathcal{H} | \psi}}{\braket{\psi | \psi} + \braket{\psi | \delta\psi} + \braket{\delta\psi | \psi}} - \frac{\braket{\psi | \mathcal{H} | \psi}}{\braket{\psi | \psi}} \\
				&= \frac{\braket{\psi | \mathcal{H} | \delta\psi} + \braket{\delta\psi | \mathcal{H} | \psi} - E[\psi] \braket{\psi | \delta\psi} - E[\psi] \braket{\delta\psi | \psi}}{\braket{\psi | \psi} + \braket{\psi | \delta\psi} + \braket{\delta\psi | \psi}} \\
				&= \frac{2\Re \braket{\delta\psi | (\mathcal{H} - E[\psi]) | \psi}}{\braket{\psi | \psi} + \braket{\psi | \delta\psi} + \braket{\delta\psi | \psi}} = 0
			\end{split}
		\end{equation*}
	\end{proof}
\end{proposition}

\begin{theorem}{Principio variazionale di Ritz}{}
	Detto $ \ket{\psi_0} $ lo stato fondamentale di $ \mathcal{H} $, allora $ E[\psi] \ge E[\psi_0] \equiv E_0 \,\,\forall \ket{\psi} \in \mathscr{H} $.

	\tcblower
	\begin{proof}
		Data una base di autostati $ \{u_n\} : \mathcal{H} \ket{u_n} = E_n \ket{u_n} \land \braket{u_n | u_m} = \delta_{nm} $, dove $ u_0 $ è il ground state con $ E_0 \le E_1 \le E_2 \le \dots $, per il generico stato $ \ket{\psi} = \sum_n A_n \ket{u_n} $:
		\begin{equation*}
			E[\psi] - E_0
			= \frac{\abs{A_0}^2 E_0 + \sum_{i \neq 0} \abs{A_i}^2 E_i}{\abs{A_0}^2 + \sum_{i \neq 0} \abs{A_i}^2} - E_0 = \frac{\sum_{i \neq 0} \left( E_i - E_0 \right) \abs{A_i}^2}{\abs{A_0}^2 + \sum_{i \neq 0} \abs{A_i}^2} \ge 0
		\end{equation*}
	\end{proof}
\end{theorem}

Questo risultato è utile poiché permette di trovare il ground state minimizzando l'energia: se si parametrizza la funzione d'onda, il ground state sarà dato dal set di parametri per cui si ha il minimo dell'energia.\\
Una possibile applicazione è quella di ottimizzare i coefficienti di uno sviluppo di una funzione d'onda su una base di funzioni d'onda fissate: vedendo i coefficienti della combinazione lineare come parametri, minimizzando l'energia si ottiene un sistema lineare di equazioni la cui risoluzione fornisce i parametri che meglio approssimano la funzione d'onda reale. Questo è il caso, ad esempio, della Linear Combination of Atomic Orbital, nel quale si esprime la funzione d'onda di una molecola come combinazione lineare delle funzioni d'onda dei suoi atomi costituenti.\\
Questo metodo può essere ulteriormente generalizzato facendo variare parametricamente anche le funzioni d'onda di base sulle quali si effettua lo sviluppo (es.: metodo di Hartree-Fock).

\section{Soluzione analitica}

L'atomo a singolo elettrone (o idrogenoide) è uno dei pochi casi in cui l'equazione di Schrödinger può essere risolta analiticamente. Essendo il potenziale coulombiano un potenziale radiale a simmetria sferica, si può separare il problema in moto del centro di massa e moto radiale: essendo la massa nucleo $ M $ migliaia di volte quella dell'elettrone $ m_e $, il centro di massa può essere approssimato con la posizione stessa del nucleo. La correzione di massa ridotta diventa importante quando si considerano atomi esotici, come ad esempio l'idrogeno muonico (sistema legato protone-muone).\\
Concentrandosi sull'Hamiltoniana di moto relativo (quella del centro di massa è semplicemente l'Hamiltoniana di particella libera):
\begin{equation}
	\mathcal{H} = - \frac{\hbar^2}{2\mu} \nabla_\ve{r}^2 - \frac{Ze}{r}
	\label{eq:1.2}
\end{equation}
dove $ \ve{r} \equiv \ve{R} - \ve{r}_e $ e $ \mu \equiv (M^{-1} + m_e^{-1})^{-1} $. In coordinate sferiche si trova:
\begin{equation*}
	\nabla^2 = \frac{2}{r} \frac{\pa}{\pa r} + \frac{\pa^2}{\pa r^2} - \frac{\hat{L}^2}{\hbar^2 r^2}
\end{equation*}
Dato che $ [\mathcal{H} , \hat{L}^2] = 0 $, si può cercare una soluzione $ \psi = \psi(r,\vartheta,\varphi) $ in funzione delle autofunzioni di $ \hat{L}^2 $: questi sono le armoniche sferiche $ Y_{\ell,m} : \hat{L}^2 Y_{\ell,m} = \hbar^2 \ell (\ell + 1) Y_{\ell,m} $, con $ \ell \in \N_0 $ e $ -\ell \le m \le \ell $. Scrivendo $ \psi(r,\vartheta,\varphi) = R(r) Y_{\ell,m}(\vartheta,\varphi) $ si ha:
\begin{equation*}
	\left[ - \frac{\hbar^2}{2\mu} \left( \frac{2}{r} \frac{\dd}{\dd r} + \frac{\dd^2}{\dd r^2} \right) + V_\ell(r) \right] R(r) = E R(r)
	\qquad \qquad
	V_\ell(r) \equiv - \frac{Ze}{r} + \frac{\hbar^2 \ell(\ell + 1)}{2\mu r^2}
\end{equation*}
Risolvendo questa equazione si trova che la funzione d'onda radiale dipende da un numero quantico, il \textit{numero quantico radiale} $ n_r $, tale per cui:
\begin{equation*}
	E_{n_r,\ell} = - \frac{Z^2 e^4 \mu}{2\hbar^2 (n_r + \ell + 1)^2}
\end{equation*}
Questo numero quantico rappresenta il numero di nodi nella funzione d'onda radiale $ R_{n_r,\ell}(r) $, nel caso di un potenziale radiale a simmetria sferica. Conviene definire un numero quantico equivalente, il \textit{numero quantico principale} $ n \equiv n_r + \ell + 1 $, così che:
\begin{equation}
	E_n = - \frac{Z^2 e^4 \mu}{2\hbar^2} \frac{1}{n^2} = - \frac{Z^2}{2} E_\text{Ha} \frac{\mu}{m_e} \frac{1}{n^2}
	\label{eq:1.3}
\end{equation}
Vale inoltre che $ 0 \le \ell \le n-1 $.

\paragraph{Degenerazione}

Dallo spettro energetico Eq. \ref{eq:1.3} si vede che l'energia non dipende né da $ m $ né da $ \ell $: nel primo caso si parla di \textit{degenerazione necessaria}, in quanto è una degenerazione dovuta alla simmetria sferica del problema e comporta $ d(\ell) = 2\ell + 1 $, mentre nel secondo caso di ha una \textit{degenerazione accidentale}, legata alla forma particolare del potenziale coulombiano. La presenza di quest'ultima degenerazione accidentale permette di definire il numero quantico principale, e si ha una degenerazione overall di $ d(n) = \sum_{\ell = 0}^{n-1} (2\ell + 1) = n^2 $.

\subsection{Funzione d'onda angolare}

L'armonica sferica $ Y_{\ell,m}(\vartheta,\varphi) $ contiene informazione esatta sul momento angolare totale e sulla sua proiezione sull'asse $ z $, in quanto $ \hat{L}^2 Y_{\ell,m} = \hbar^2 \ell (\ell + 1) $ e $ \hat{L}_z Y_{\ell,m} = \hbar m $, mentre quella sul momento angolare lungo le altre direzioni è di natura probabilistica, poiché $ [\hat{L}_x , \hat{L}_z],[\hat{L}_y , \hat{L}_z] \neq 0 $. Si noti, però, che grazie alla simmetria sferica la definizione di $ \ve{e}_z $ è arbitraria.\\
La forma generica di un'armonica sferica è data da:
\begin{equation}
	Y_{\ell,m}(\vartheta,\varphi) = \mathcal{N} e^{i m \varphi} P_\ell^{\abs{m}}(\cos \vartheta)
	\label{eq:1.4}
\end{equation}
dove $ P_\ell^{\abs{m}}(\cos \vartheta) $ è la funzione di Legendre. Si vede dunque che $ \abs{Y_{\ell,m}}^2 $ è indipendente da $ \varphi $, dunque la probabilità $ \abs{\braket{\vartheta,\varphi | n,\ell,m}}^2 $ dipende solo da $ \vartheta $. Inoltre, si trova che $ \ell - \abs{m} $ è pari al numero di nodi della funzione di Legendre, e che sotto operatore di parità $ \mathcal{P} Y_{\ell,m} = (-1)^\ell Y_{\ell,m} $.\\
Data la simmetria necessaria rispetto ad $ m $, si possono combinare orbitali con $ \pm m $ per ottenere degli orbitali reali: $ Y_{0,0} = \frac{1}{\sqrt{4\pi}} $ e $ Y_{\ell,0} \propto \cos{\vartheta} $ sono sempre reali, mentre ad esempio si definiscono gli orbitali reali $ \ch{p}_x , \ch{p}_y $ come:
\begin{equation*}
	\psi_{\ch{p}_x} = \frac{1}{\sqrt{2}} \left( Y_{1,-1} - Y_{1,1} \right)
	\qquad \qquad
	\psi_{\ch{p}_y} = \frac{i}{2} \left( Y_{1,-1} + Y_{1,1} \right)
\end{equation*}

\subsection{Funzione d'onda radiale}

Per quanto riguarda la funzione d'onda radiale $ R_{n_r,\ell}(r) $, essa ha la forma generica:
\begin{equation}
	R_{n,\ell}(r) = \mathcal{N} (kr)^\ell L_{n+\ell}^{2\ell + 1}(kr) e^{-\frac{1}{2} kr}
	\qquad
	k \equiv \frac{2Z \mu}{a_0 m_e} \frac{1}{n}
	\label{eq:1.5}
\end{equation}
dove $ L_{n+\ell}^{2\ell+1}(kr) $ è il polinomio di Laguerre, un polinomio di grado $ n - \ell - 1 $ (con termine noto non-nullo) che presenta un numero di nodi pari a $ n - \ell - 1 $. Si vedono subito i due limiti:
\begin{equation*}
	\lim_{r \rightarrow 0} R_{n,\ell}(r) \sim r^\ell
	\qquad \qquad
	\lim_{r \rightarrow \infty} R_{n,\ell}(r) \sim r^{n-1} e^{-\frac{1}{2} kr}
\end{equation*}
La distribuzione di probabilità spaziale è data dunque da:
\begin{equation}
	P_{n,\ell,m}(\ve{r}) \dd^3r = \abs{R_{n,\ell}(r)}^2 \abs{Y_{\ell,m}(\vartheta,\varphi)}^2 r^2 \sin{\vartheta} \dd r \dd \vartheta \dd \varphi
	\label{eq:1.6}
\end{equation}
La probabilità $ P_{n,0,0}(\ve{r}) $ ha un massimo assoluto in $ \ve{r} = \ve{0} $: la posizione più probabile per l'elettrone nello stato $ \ch{s} $ è il centro del potenziale, dove esso è più attrattivo. Per $ \ell \neq 0 $, invece, $ P_{n,\ell,m}(\ve{0}) = 0 $, e ciò evidenzia l'impossibilità per una particella dotata di momento angolare di cadere nel centro di un potenziale centrale.\\
Si osserva inoltre che, all'aumentare di $ Z $, il massimo di $ \abs{R_{n,\ell}(r)}^2 $ e dunque di $ P_{n,\ell,m}(\ve{r}) $ si sposta verso l'origine\footnote{In particolare $ R_{n,\ell}^{[Z]}(r) = Z^{3/2} R_{n,\ell}^{[1]}(rZ) $, quindi $ P_{n,\ell,m}^{[Z]}(r) = Z P^{[1]}(rZ) $.}: ciò implica che la distanza elettrone-nucleo è $ \propto Z^{-1} $, dunque, dato che $ V \propto Z/r $, si trova l'andamento dell'energia $ E \propto Z^2 $.\\
La funzione d'onda più semplice, quello dello stato $ 1\ch{s} $, si trova essere:
\begin{equation}
	\psi_{1,0,0}(\ve{r}) = \frac{1}{\sqrt{\pi}} \left( \frac{Z \mu}{a_0 m_e} \right)^{3/2} e^{- \frac{Z \mu}{a_0 m_e} r}
	\label{eq:1.7}
\end{equation}

\section{Spettro energetico}

Quando si parla di spettro d'eccitazione si intende lo spettro delle differenze di autovalori di energia: dati due stati $ \ket{i} , \ket{f} $, si ha $ \Delta E = E_f - E_i $. Dall'Eq. \ref{eq:1.3} si trova:
\begin{equation}
	\Delta E = - \frac{\mu}{m_e} \frac{E_\text{Ha}}{2} Z^2 \left( \frac{1}{n_f^2} - \frac{1}{n_i^2} \right)
	\label{eq:1.8}
\end{equation}
Queste sono quantità misurabili spettroscopicamente.

\subsection{Atomo di idrogeno}

Per l'atomo di idrogeno si ha il ground state $ E_1 = - \frac{1}{2} E_\text{Ha} \frac{\mu}{m_e} = - 13.5983\ev $. Le transizioni $ n_i \rightarrow n_f $ si raggruppano in serie, ciascuna caratterizzata dallo stesso stato finale $ n_f $ ad energia più bassa: per l'atomo di idrogeno, ciascuna serie è osservata in una presisa regione caratteristica dello spettro elettromagnetico, ed in particolare le serie di Lyman ($ n_f = 1 $) e Balmer ($ n_f = 2 $) non presentano alcun overlap con altre serie, dato che la distanza energetica tra $ E_1 \simeq - 13.60\ev $ o $ E_2 \simeq -3.39\ev $ ed il successivo stato eccitato supera l'intero range energetico tra quest'ultimo e l'ionization threshold ($ E = 0 $).\\
Ricordando che i fotoni nel visibile si trovano nel range $ 1.8\ev - 3\ev $ ($ 400\,\text{nm} - 700\,\text{nm} $), si trova che la serie di Lyman ($ 10\ev < E < 10\ev $) è nell'ultravioletto, la serie di Balmer ($ 1.8\ev < E < 3.2\ev $) è nel visibile e quella di Paschen ($ E < 1.5\ev $) nell'infrarosso.\\
Come da Eq. \ref{eq:1.8}, lo spettro atomico risente di un prefattore $ \mu / m_e $ che determina una debole dipendenza dalla massa del nucleo $ M $: di conseguenza, miscele di isotopi presenteranno delle duplicazioni di linee spettrali, sebbene energeticamente estremamente vicine. Inoltre, la dipendenza dell'energia da $ Z^2 $ può portare a sovrapposizioni parziali degli spettri di elementi diversi: as esempio, metà delle righe di metà delle serie dell'elio si sovrappongono a quelle dell'idrogeno.

\subsection{Modello di Bohr}

Gli spettri atomici furono osservati prima della formulazione della meccanica quantistica, dunque Bohr propose un modello per spiegarli.\\
Il modello di Bohr assume soltanto che gli elettroni si muovano attorno al nucleo in orbite circolari e che il loro momento angolare sia quantizzato in unità di $ \hbar $: $ mvr = n\hbar $, con $ n \in \N_0 $. La circolarità dell'orbita fa sì che la forza coulombiana sia di natura centripeta, ovvero:
\begin{equation*}
	\frac{mv^2}{r} = \frac{Ze^2}{r^2}
	\qquad \Rightarrow \qquad
	r = \frac{n^2 \hbar^2}{m_e Z e} = \frac{a_0}{Z} n^2
\end{equation*}
Si vede dunque che la distanza dell'elettrone dal nucleo è anch'essa quantizzata da $ n $, ed inoltre dipende da $ Z^{-1} $. L'energia cinetica e quella potenziale dell'elettrone sono:
\begin{equation*}
	T = \frac{1}{2} m v^2 = \frac{1}{2} \frac{Z^2 e^2}{a_0} \frac{1}{n^2} = \frac{1}{2} E_\text{Ha} \frac{Z^2}{n^2}
	\qquad \qquad
	U = - \frac{Z e^2}{r^2} = - E_\text{Ha} \frac{Z}{n^2}
\end{equation*}
Ciò è consistente col teorema del viriale\footnote{Il teorema del viriale stabilisce che, se $ U \propto r^\alpha $, allora $ \braket{T} = \frac{\alpha}{2} \braket{U} $.}, e per l'energia dell'elettrone si trova quindi:
\begin{equation*}
	E = T + U = - \frac{1}{2} E_\text{Ha} \frac{Z}{n^2} \equiv E_n
\end{equation*}
Si ha dunque un accordo perfetto con lo spettro energetico osservato. Risulta però errato lo spettro del momento angolare: ad esempio, nel ground state il modello di Bohr richiede un momento angolare pari a $ \hbar $, mentre un elettrone nello stato $ 1\ch{s} $ ha momento angolare nullo. Ciò è anche evidente considerando che l'elettrone ha una probabilità non-nulla di trovarsi in $ \ve{r} = \ve{0} $, ovvero all'interno del nucleo (es.: cattura elettronica), il che sarebbe impossibile se esso fosse dotato di momento angolare non-nullo.











