\selectlanguage{italian}

Il problema generale che si va ad analizzare è il sistema di $ N_n $ elettroni ed $ N_n $ nuclei atomici, il cui moto non-relativistico è affetto solo dall'interazione elettromagnetica ed è dunque descritto dall'Hamiltoniana:
\begin{equation*}
	\mathcal{H} = T_n + T_e + V_{ne} + V_{nn} + V_{ee}
\end{equation*}
L'energia cinetica totale dei nuclei è:
\begin{equation*}
	T_n = \frac{1}{2} \sum_\alpha \frac{\ve{P}_{\ve{R}_\alpha}^2}{M_\alpha}
\end{equation*}
dove $ \ve{P}_{\ve{R}_\alpha} $ è il momento coniugato alla posizione $ \ve{R}_\alpha $ dell'$ \alpha $-esimo nucleo, mentre l'energia cinetica degli elettroni è:
\begin{equation*}
	T_e = \frac{1}{2m_e} \sum_i \ve{P}_{\ve{r}_i}^2
\end{equation*}
dove $ \ve{P}_{\ve{r}_i} $ è il momento coniugato alla posizione $ \ve{r}_i $ dell'$ i $-esimo elettrone. I potenziali d'interazione elettromagnetica sono invece:
\begin{equation*}
	V_{ne} = - \frac{q_e^2}{4\pi \epsilon_0} \sum_\alpha \sum_i \frac{Z_\alpha}{\abs{\ve{R}_\alpha - \ve{r}_i}}
\end{equation*}
\begin{equation*}
	V_{nn} = \frac{q_e^2}{4\pi \epsilon_0} \frac{1}{2} \sum_\alpha \sum_{\beta \neq \alpha} \frac{Z_\alpha Z_\beta}{\abs{\ve{r}_\alpha - \ve{r}_\beta}}
\end{equation*}
\begin{equation*}
	V_{ee} = \frac{q_e^2}{4\pi \epsilon_0} \frac{1}{2} \sum_i \sum_{j \neq i} \frac{1}{\abs{\ve{r}_i - \ve{r}_j}}
\end{equation*}
La risoluzione analitica di questo problema è possibile solo in un numero limitato di casi, mentre la sua integrazione numerica scala in complessità esponenzialmente con $ N = N_e + N_n $.\\
Nella formulazione di tale Hamiltoniana, si sono applicate alcune approssimazioni:
\begin{enumerate}
	\item corpi puntiformi: mentre per l'elettrone, in quanto particella fondamentale, questa assunzione è sempre lecita, per il nucleo atomico essa è possibile visto che il rapporto tra raggio nucleare e raggio atomico è dell'ordine di $ 10^{-3} $, oltre al fatto che le energie in gioco nei processi atomici ($ \sim 1\ev $) non sono sufficienti ad eccitare i gradi di libertà interni del nucleo ($ \sim 1\mev $);
	\item moto non-relativistico: alcune correzioni relativistiche (es.: interazione spin-orbita) possono essere trattate perturbativamente;
	\item sistema isolato: si assume il sistema non-interagente con l'ambiente esterno ed in assenza di campi esterni.
\end{enumerate}
Si noti che mentre i sistemi a singola particella godono di determinate simmetrie, quelli a molti corpi possono presentare delle rotture spontanee di simmetria: ciò è particolarmente evidente nei sistemi molecolari, mentre in quelli atomici è presente ma in misura minore, ed avviene poiché nei casi in cui una rottura di simmetria permetta di abbassare l'energia totale del sistema.

\paragraph{Ordini di grandezza}

Innanzitutto, conviene definire la coupling constant dell'interazione elettromagnetica:
\begin{equation*}
	e^2 \equiv \frac{q_e^2}{4\pi \epsilon_0} = 2.3071 \cdot 10^{-28} \,\text{J} \,\text{m} = 14.3387 \ev \ang
\end{equation*}
La scala del moto elettronico nell'atomo è data dal \textit{raggio di Bohr}:
\begin{equation*}
	a_0 \equiv \frac{\hbar^2}{m_e e^2} = 0.529177 \cdot 10^{-10} \,\text{m} = 0.529177 \ang
\end{equation*}
Evidenze sperimentali mostrano che gli atomi nella materia sono distanziati nell'ordine di $ 2a_0 - 10a_0 $. La scala delle interazioni elettromagnetiche in ambito atomico/molecolare è dunque data dall'\textit{energia di Hartree}:
\begin{equation*}
	E_\text{Ha} \equiv \frac{e^2}{a_0} = 4.35974 \cdot 10^{-18} \,\text{J} = 27.2114 \ev
\end{equation*}
La tipica timescale del moto elettronico si ottiene dal principio d'indeterminazione:
\begin{equation*}
	t_0 = \frac{\hbar}{E_\text{Ha}} = 2.4189 \cdot 10^{-17} \,\text{s}
\end{equation*}
Ciò permette di calcolare la scala delle velocità elettroniche, confermando l'approssimazione non-relativistica:
\begin{equation*}
	v_0 = \frac{a_0}{t_0} = 2.1877 \cdot 10^6 \,\text{m} \,\text{s}^{-1} \simeq 0.01 c
\end{equation*}
La scala dei fenomeni relativistici nella dinamica elettronica è data dalla \textit{costante di struttura fine}:
\begin{equation*}
	\alpha = \frac{v_0}{c} = \frac{e^2}{\hbar c} = 7.29734 \cdot 10^{-3} \simeq \frac{1}{137.036}
\end{equation*}
Comparando con le onde elettromagnetiche, la scala delle distanze interatomiche ($ \sim 1\ang $) corrisponde alla regione dei raggi X, mentre quella delle frequenze elettroniche (e dunque di $ E_\text{Ha} $) corrisponde alla fascia UV ($ \lambda \sim 10^3 a_0 $); le frequenze tipiche (e dunque le energie) del moto nucleare sono invece associate alla regione IR ($ \nu \sim 5 \,\text{THz} $).

\paragraph{Spettroscopia}

Gli esperimenti spettroscopici sono quelli in cui una proprietà caratterizzante l'interazione tra radiazione e materia è misurata in funzione della frequenza della radiazione incidente sul campione che si vuole studiare. I principali tipi di spettroscopia sono due:
\begin{enumerate}
	\item assorbimento: un fascio collimato di luce monocromatica incide sul bersaglio; se la frequenza della radiazione coincide con quella di una transizione specifica del campione, ci sarà un importante assorbimento di fotoni: questo sarà visibile plottando l'intensità della radiazione emergente dal campione $ I(\omega) $, ottenendo il cosiddetto spettro d'assorbimento;
	\item emissione: il campione viene portato in uno stato eccitato (es.: bombardandolo di elettroni o fotoni alto-energetici), dunque emetterà della radiazione ad ogni transizione di diseccitamento, la quale va a formare il cosiddetto spettro d'emissione.
\end{enumerate}
Gli spettri atomici e molecolari sono dunque caratterizzati da picchi monocromatici, detti linee, associate a transizioni risonanti tra stati $ \ket{i} , \ket{f} $ che determinano linee a $ \omega_{if} = \frac{1}{\hbar} \abs{E_i - E_f} $. Sebbene a livello teorico queste linee sarebbero delle $ \delta $ di Dirac, sperimentalmente si misurano sempre delle righe più o meno strette; le cause dell'allargamento delle linee spettrali sono sia intrinseche che estrinseche, e principalmente sono:
\begin{enumerate}
	\item risoluzione sperimentale: tipicamente determinata da vari effetti aleatori, dunque determina una forma gaussiana; può essere migliorata con accorgimenti tecnici;
	\item allargamento naturale: dovuto al fatto che gli stati eccitati, sebbene stazionari in prima approssimazione, vengono resi instabili dall'interazione col le fluttuazioni di punto-zero del campo elettromagnetico (quantistico); si determina dunque un decadimento spontaneo di tutti gli autostati d'energia (eccetto il ground state) che, sebbene randomico per un singolo atomo, segue una legge statistica per un sistema a molti atomi:
	\begin{equation*}
		N(t) = N_0 e^{- \gamma t} = N_0 e^{- t / \tau}
	\end{equation*}
	dove $ \gamma $ è la costante di decadimento e $ \tau $ la vita media dello stato eccitato, la quale setta la durata tipica della spettroscopia. Dal principio d'indeterminazione, si trova che l'energia di uno stato eccitato non è misurabile con precisione migliore di:
	\begin{equation*}
		\Delta E = \frac{\hbar}{\tau} = \hbar \gamma
	\end{equation*}
	Ciò causa dunque l'allargamento naturale delle linee spettrali secondo la Lorentziana:
	\begin{equation*}
		I(\omega) = I_0 \frac{\gamma^2}{(\omega - \omega_{if})^2 + \gamma^2}
	\end{equation*}
	Gli stati atomici eccitati hanno $ \tau \sim 1\,\text{ns} $, dunque l'allargamento è di $ \Delta E \sim 1\,\mu\text{eV} $.
	\item allargamento Doppler: nel caso di un campione in fase gassosa, il moto termico randomico degli atomi/molecole determina un red/blue-shift delle frequenze di transizione, a seconda della velocità casuale dell'atomo/molecola che decade; si ha un allargamento gaussiano delle righe spettrali, che determina:
	\begin{equation*}
		\Delta \omega_\text{Doppler} = \omega_{if} \sqrt{8 \ln(2) \frac{k_B T}{M c^2}}
	\end{equation*}
	A temperatura fissata, gli atomi/molecole più leggeri si muoveranno più velocemente, determinando un allargamento maggiore (sempre nell'ordine dei $ \mu\text{eV} $).
\end{enumerate}
Si ha dunque un allrgamento totale pari alla somma in quadratura di questi allargamenti singoli.










