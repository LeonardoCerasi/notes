\documentclass[a4paper, 12pt]{book}
\usepackage[english]{babel}

\usepackage{fontspec}
\setmainfont{Latin Modern Roman}
\newfontfamily\cyrillicfont{DejaVu Serif}[Script=Cyrillic]

% page layout
\usepackage[left=20mm, right=20mm, top=25mm]{geometry}
\geometry{a4paper}

% appendices
\usepackage[toc,page]{appendix}

% images
\usepackage{graphicx}

% colored headings
\usepackage{xcolor}

% bookmarks
\usepackage{bookmark}

% dedication
\newenvironment{dedication}
{\clearpage             % new page
 \thispagestyle{empty}  % no header and footer
 \vspace*{\stretch{1}}  % top space
 %\itshape               % italics text
 \raggedleft            % flush to the right margin
}
{\par                   % end paragraph
 \vspace{\stretch{3}}   % bottom space (3x top space)
 \clearpage             % finish page
}

% custom title page
\newcommand{\customtitlepage}[7]{%
  \begin{titlepage}
    \centering
    % department logo at the top of the page
    \includegraphics[width=0.75\textwidth]{imgs/unimi.jpg}\\
    \vspace{1cm} % ddjust the vertical space as needed
    {\large Bachelor Degree in #1 \par} % degree title
    \vspace{1cm}
    \hrule % horizontal line
    \vspace{1cm} % adjust the vertical space as needed
    {\large \sffamily \textbf{#2} \par} % thesis title (bold)
    \vspace{7cm} % adjust the vertical space as needed

    \begin{minipage}[H]{\textwidth}
      \small Supervisor: \\
      \normalsize #3 \\
      \vspace{1cm}
    \end{minipage}
    \begin{minipage}[t]{\textwidth}
      \raggedleft
      \small Student Name: \\
      \normalsize #4\\ % student name
      \normalsize Matr.: #5 % student number
    \end{minipage}

    % academic year section at the bottom of the page
    \vspace{\fill}
    {\normalsize Academic Year #6 \par}

  \end{titlepage}
}

% equation numbering
\makeatletter
\def\tagform@#1{\maketag@@@{\ignorespaces\sffamily(#1)\unskip\@@italiccorr}}
\makeatother

% sectioning
\usepackage{titlesec}

% titleformat{name}
% {title style}
% {number style}
% {space between number and title}
% {before-title code}

\titleformat{\part}[display]
  {\normalfont \sffamily \bfseries \centering}
  {\color{blue!70!black} \Large \partname \, \thepart}
  {10pt} % vertical spacing
  {\Huge}

\titleformat{\chapter}[display]
  {\normalfont \sffamily \bfseries}
  {\color{blue!70!black} \large \chaptertitlename \, \thechapter}
  {1pt}
  {\titlerule[1pt] \vspace{7pt} \Huge}

\titleformat{\section}
  {\normalfont \sffamily \Large}
  {\color{red!70!black} §\bfseries\thesection}
  {10pt}
  {\bfseries}

\titleformat{\subsection}
  {\normalfont \sffamily \large}
  {\color{red!70!black} §\bfseries\thesubsection}
  {10pt}
  {\bfseries}

\titleformat{\subsubsection}
  {\normalfont \sffamily}
  {\color{red!70!black} §\bfseries\thesubsubsection}
  {10pt}
  {\bfseries}

\renewcommand{\appendixtocname}{\sffamily Appendices}
\renewcommand{\appendixpagename}{\sffamily \Huge \bfseries Appendices}

% custom table of contents
\usepackage{tocloft}

\renewcommand{\cfttoctitlefont}{\sffamily \Huge \bfseries} % title font
% part
\renewcommand{\cftpartfont}{\sffamily \large \bfseries}
\renewcommand{\cftpartpagefont}{\sffamily \large \bfseries}
\renewcommand{\cftpartpresnum}{\sffamily \color{blue!70!black}}
% chapter
\renewcommand{\cftchapfont}{\sffamily \bfseries}
\renewcommand{\cftchappagefont}{\sffamily \bfseries}
\renewcommand{\cftchappresnum}{\sffamily \color{blue!70!black}}
% section
\renewcommand{\cftsecfont}{\sffamily}
\renewcommand{\cftsecpagefont}{\sffamily}
\renewcommand{\cftsecpresnum}{\sffamily \color{red!70!black}}
% subsection
\renewcommand{\cftsubsecfont}{\sffamily}
\renewcommand{\cftsubsecpagefont}{\sffamily}
\renewcommand{\cftsubsecpresnum}{\sffamily \color{red!70!black}}

% unnumbered chapter
\titleformat{name = \chapter, numberless}[block]
  {\centering \sffamily \bfseries \Large}
  {}
  {0pt}
  {}

% headers and foooters
\usepackage{fancyhdr}
\setlength{\headheight}{15pt}

% part name
\let\Oldpart\part
\newcommand{\parttitle}{}
\renewcommand{\part}[1]{\Oldpart{#1}\renewcommand{\parttitle}{#1}}

% chapter and section title with \leftmark and \rightmark
\renewcommand{\sectionmark}[1]{\markright{#1}}
\renewcommand{\chaptermark}[1]{\markboth{#1}{}}

% no random page numbers
\fancypagestyle{plain}{%
  \fancyhf{} % clear all header and footer fields
  \renewcommand{\headrulewidth}{0pt}
  \renewcommand{\footrulewidth}{0pt}
}

% custom headers and footers
\pagestyle{fancy}
\fancyhead{}
\fancyfoot{}

\fancypagestyle{body}{%
  \fancyhead[LE,RO]{\sffamily\thepage}%
  \fancyhead[LO]{\sffamily \color{blue!70!black}\chaptername\ \thechapter\color{black} :\ \leftmark}%
  \fancyhead[RE]{\sffamily \color{blue!70!black}\partname\ \thepart\color{black} :\ \parttitle}%
}

% special header for the thesis (without parts) --- SECTION NAME HARDCODED
\fancypagestyle{body-thesis}{%
  \fancyhead[LE,RO]{\sffamily\thepage}%
  \fancyhead[LO]{\sffamily \color{blue!70!black}\chaptername\ \thechapter\color{black} :\ \leftmark}%
  \fancyhead[RE]{\sffamily \color{red!70!black}Section\ \thesection\color{black} :\ \rightmark}%
}

% special header for Introduction
\fancypagestyle{introd}{%
  \fancyhead[LE,RO]{\sffamily \thepage}%
  \fancyhead[RE,LO]{\sffamily \leftmark}%
}

% special header for Contents
\fancypagestyle{contents}{%
  \fancyhead[LE,RO]{\sffamily \thepage}%
  \fancyhead[RE,LO]{\sffamily Contents}%
}

% special header for Appendix
\fancypagestyle{append}{%
  \fancyhead[LE,RO]{\sffamily \thepage}%
  \fancyhead[LO]{\sffamily \color{blue!70!black}\appendixname\ \thechapter \color{black}:\ \leftmark}%
  \fancyhead[RE]{\sffamily Appendices}%
}

% special header for Bibliography
\fancypagestyle{biblio}{%
  \fancyhead[LE,RO]{\sffamily \thepage}%
  \fancyhead[RE,LO]{\sffamily Bibliography}%
}

% special header for Abstract
\fancypagestyle{abstract}{%
  \fancyhead[LE,RO]{\sffamily \thepage}%
  \fancyhead[RE,LO]{\sffamily Abstract}%
}

% no blank pages
\let\cleardoublepage\clearpage
% removes indentation
\setlength{\parindent}{0pt}
% add subsubsection numbering
\setcounter{secnumdepth}{3}

% bold text

\newcommand{\bctxt}[1]{\textcolor{blue!70!black}{\sffamily \textbf{#1}}}

\newcommand{\bcdef}[1]{\textcolor{green!50!black}{\sffamily \textbf{#1}}}

\newcommand{\bcth}[1]{\textcolor{red!50!black}{\sffamily \textbf{#1}}}

\newcommand{\bclemma}[1]{\textcolor{blue!35!black}{\sffamily \textbf{#1}}}

\newcommand{\bcprop}[1]{\textcolor{cyan!35!black}{\sffamily \textbf{#1}}}

\newcommand{\bcex}[1]{\textcolor{yellow!35!black}{\sffamily \textbf{#1}}}

\newcommand{\bcobs}[1]{\textcolor{green!35!black!80}{\sffamily \textbf{#1}}}

% fancy environments
\usepackage[many]{tcolorbox}
\BeforeBeginEnvironment{tcolorbox}{\savenotes}
\AfterEndEnvironment{tcolorbox}{\spewnotes}

\tcbuselibrary{theorems}

\NewTcbTheorem[number within=section]{definition}{Definition}{%
  enhanced,%
  breakable,%
  colback = green!10,%
  colframe = green!5,%
  coltitle = green!50!black,%
  fonttitle = \sffamily\bfseries,%
  sharp corners,%
  boxrule = 0pt,%
  detach title,%
  before upper = {\tcbtitle\\[4pt]},%
  left = 5pt,%
  right = 5pt,%
  top = 5pt,%
  bottom = 5pt,%
  separator sign none,%
  description delimiters parenthesis,%
  description font = \mdseries%
}{def}

\newtcbtheorem[number within=section]{theorem}{Theorem}{%
  enhanced,%
  breakable,%
  colback = red!5,%
  colframe = red!5,%
  coltitle = red!50!black,%
  fonttitle = \sffamily\bfseries,%
  sharp corners,%
  boxrule = 0pt,%
  detach title,%
  before upper = {\tcbtitle\\[4pt]},%
  left = 5pt,%
  right = 5pt,%
  top = 5pt,%
  bottom = 5pt,%
  separator sign none,%
  description delimiters parenthesis,%
  description font = \mdseries%
}{th}

\newtcbtheorem[number within=tcb@cnt@theorem]{corollary}{Corollary}{%
  enhanced,%
  breakable,%
  colback = red!5,%
  colframe = red!5,%
  coltitle = red!35!black,%
  fonttitle = \sffamily\bfseries,%
  sharp corners,%
  boxrule = 0pt,%
  detach title,%
  before upper = {\tcbtitle\\[4pt]},%
  left = 5pt,%
  right = 5pt,%
  top = 5pt,%
  bottom = 5pt,%
  separator sign none,%
  description delimiters parenthesis,%
  description font = \mdseries%
}{cor}

\newtcbtheorem[number within=section]{lemma}{Lemma}{%
  enhanced,%
  breakable,%
  colback = blue!5,%
  colframe = blue!5,%
  coltitle = blue!35!black,%
  fonttitle = \sffamily\bfseries,%
  sharp corners,%
  boxrule = 0pt,%
  detach title,%
  before upper = {\tcbtitle\\[4pt]},%
  left = 5pt,%
  right = 5pt,%
  top = 5pt,%
  bottom = 5pt,%
  separator sign none,%
  description delimiters parenthesis,%
  description font = \mdseries%
}{lemma}

\newtcbtheorem[number within=tcb@cnt@lemma]{lemcorollary}{Corollary}{%
  enhanced,%
  breakable,%
  colback = blue!5,%
  colframe = blue!5,%
  coltitle = blue!35!black,%
  fonttitle = \sffamily\bfseries,%
  sharp corners,%
  boxrule = 0pt,%
  detach title,%
  before upper = {\tcbtitle\\[4pt]},%
  left = 5pt,%
  right = 5pt,%
  top = 5pt,%
  bottom = 5pt,%
  separator sign none,%
  description delimiters parenthesis,%
  description font = \mdseries%
}{cor}

\newtcbtheorem[number within=section]{proposition}{Proposition}{%
  enhanced,%
  breakable,%
  colback = cyan!5,%
  colframe = cyan!5,%
  coltitle = cyan!35!black,%
  fonttitle = \sffamily\bfseries,%
  sharp corners,%
  boxrule = 0pt,%
  detach title,%
  before upper = {\tcbtitle\\[4pt]},%
  left = 5pt,%
  right = 5pt,%
  top = 5pt,%
  bottom = 5pt,%
  separator sign none,%
  description delimiters parenthesis,%
  description font = \mdseries%
}{prop}

\newtcbtheorem[number within=tcb@cnt@proposition]{propcorollary}{Corollary}{%
  enhanced,%
  breakable,%
  colback = cyan!5,%
  colframe = cyan!5,%
  coltitle = cyan!35!black,%
  fonttitle = \sffamily\bfseries,%
  sharp corners,%
  boxrule = 0pt,%
  detach title,%
  before upper = {\tcbtitle\\[4pt]},%
  left = 5pt,%
  right = 5pt,%
  top = 5pt,%
  bottom = 5pt,%
  separator sign none,%
  description delimiters parenthesis,%
  description font = \mdseries%
}{cor}

\newtcbtheorem[number within=section]{example}{Example}{%
  enhanced,%
  breakable,%
  colback = yellow!10,%
  colframe = yellow!5,%
  coltitle = yellow!35!black,%
  fonttitle = \sffamily\bfseries,%
  sharp corners,%
  boxrule = 0pt,%
  detach title,%
  before upper = {\tcbtitle\\[4pt]},%
  left = 5pt,%
  right = 5pt,%
  top = 5pt,%
  bottom = 5pt,%
  separator sign none,%
  description delimiters parenthesis,%
  description font = \mdseries%
}{ex}

\newtcbtheorem[number within=section]{observation}{Observation}{%
  enhanced,%
  breakable,%
  colback = green!5,%
  colframe = green!5,%
  coltitle = green!35!black!80,%
  fonttitle = \sffamily\bfseries,%
  sharp corners,%
  boxrule = 0pt,%
  detach title,%
  before upper = {\tcbtitle\\[4pt]},%
  left = 5pt,%
  right = 5pt,%
  top = 5pt,%
  bottom = 5pt,%
  separator sign none,%
  description delimiters parenthesis,%
  description font = \mdseries%
}{obs}

\newtcolorbox{proofbox}{%
  enhanced,%
  breakable,%
  colback = black!5,%
  colframe = black!5,%
  sharp corners,%
  boxrule = 0pt,%
  left = 5pt,%
  right = 5pt,%
  top = 5pt,%
  bottom = 5pt,%
  borderline west = {1pt}{0pt}{black!70},%
}

% custom references
\newcommand{\eref}[1]{\textsf{Eq.\,\ref{#1}}}
\newcommand{\eeref}[2]{\textsf{Eq.\,\ref{#1}-\ref{#2}}}
\newcommand{\ceref}[2]{\textsf{Eq.\,\ref{#1},\ref{#2}}}
\newcommand{\dref}[1]{\textsf{Def.\,\ref{#1}}}
\newcommand{\ddref}[2]{\textsf{Def.\,\ref{#1}-\ref{#2}}}
\newcommand{\tref}[1]{\textsf{Th.\,\ref{#1}}}
\newcommand{\cref}[1]{\textsf{Cor.\,\ref{#1}}}
\newcommand{\pref}[1]{\textsf{Prop.\,\ref{#1}}}
\newcommand{\lref}[1]{\textsf{Lemma\,\ref{#1}}}
\newcommand{\exref}[1]{\textsf{Ex.\,\ref{#1}}}
\newcommand{\obsref}[1]{\textsf{Obs.\,\ref{#1}}}

\newcommand{\figref}[1]{\textsf{Fig.\,\ref{#1}}}

\newcommand{\secref}[1]{\textcolor{red!70!black}{\textsf{§\ref{#1}}}}

% hyper-references
\usepackage{hyperref}
\hypersetup{
  colorlinks = true,
  urlcolor = cyan,
  % linkcolor defined directly in \toc environment
  citecolor = green!70!black,
}

\newcommand{\toc}{%
  \hypersetup{linkcolor = black}%
  \tableofcontents%
  \hypersetup{linkcolor = red!70!black}%
}

% extended integral symbols
\usepackage{esint}

% spacing
\newcommand{\ensp}{\vspace{-2em}}

% text
\newcommand{\tildetext}{\raise.17ex\hbox{$\scriptstyle\mathtt{\sim}$}}

% greek
\newcommand{\Chi}{\text{X}}

% custom math symbols
\newcommand{\abs}[1]{\left\lvert#1\right\rvert}
\newcommand{\norm}[1]{\left\lVert#1\right\rVert}
\newcommand{\sgn}[1]{\mathrm{sgn}\,#1}
\newcommand{\mt}[1]{\mathrm{#1}}
\newcommand{\defeq}{\mathrel{\vcenter{\baselineskip0.5ex \lineskiplimit0pt
                     \hbox{\scriptsize.}\hbox{\scriptsize.}}}%
                     =}
\newcommand{\eqdef}{=%
                     \mathrel{\vcenter{\baselineskip0.5ex \lineskiplimit0pt
                     \hbox{\scriptsize.}\hbox{\scriptsize.}}}}
\newcommand{\ve}[1]{\mathbf{#1}}
\newcommand{\bs}[1]{\boldsymbol{#1}}
\newcommand{\tsp}{^\intercal}
\newcommand{\dg}{^\dagger}
\newcommand{\smo}{o}
% \newcommand{\pwst}[1]{\euscr{P}(#1)}
\newcommand{\pwst}[1]{\raisebox{.15\baselineskip}{\Large\ensuremath{\wp}}(#1)}
\newcommand{\ra}{\rightarrow}
\newcommand{\lra}{\leftrightarrow}

\let\oldemptyset\emptyset
\let\emptyset\varnothing

\newcommand{\absurd}{\rightarrow \hspace{-0.15em} \leftarrow}
\newcommand{\img}{\mathrm{i}}
\newcommand{\jmg}{\mathrm{j}}
\newcommand{\kmg}{\mathrm{k}}
\newcommand{\bas}{\mathcal{B}}
\newcommand{\ssq}{\subseteq}
\newcommand{\ssnq}{\subsetneq}
\newcommand{\spi}{{\scriptstyle \Pi}}
\newcommand{\dist}{\mathrm{d}}
\newcommand{\cc}[1]{\overline{#1}}
\newcommand{\id}{\mathds{1}}

% multilinear algebra
\DeclareMathOperator{\diag}{diag}
\DeclareMathOperator{\lspan}{span}
\DeclareMathOperator{\ran}{ran}
\DeclareMathOperator{\tr}{tr}
\DeclareMathOperator{\Tr}{Tr}
\DeclareMathOperator{\End}{End}
\DeclareMathOperator{\Hom}{Hom}
\DeclareMathOperator{\im}{Im}
\DeclareMathOperator{\Aut}{Aut}
\DeclareMathOperator{\ad}{ad}
\DeclareMathOperator{\Ad}{Ad}
\DeclareMathOperator{\cof}{cof}
\DeclareMathOperator{\rk}{rk}

% spaces
\newcommand{\hilb}{\rsscript{H}}
\newcommand{\fock}{\rsscript{F}}
\newcommand{\cm}{\mathcal{C}^{\infty}(\mathcal{M})}
\newcommand{\xm}{\mathfrak{X}(\mathcal{M})}
\newcommand{\ld}{\mathcal{L}}
\newcommand{\lm}[1]{{\bigwedge}^{#1}(\mathcal{M})}
\newcommand{\hrm}[1]{\mathrm{Harm}^{#1}(\mathcal{M})}
\newcommand{\Vect}{\mathrm{Vect}}

% general relativity
\newcommand{\sqg}{\sqrt{\mt{g}}}
\newcommand{\sqgm}{\sqrt{-\mt{g}}}
\newcommand{\tors}{\textsf{T}}
\newcommand{\riem}{\textsf{R}}
\newcommand{\ricc}{\textsf{Ric}}

% qft
\newcommand{\lag}{\mathcal{L}}
\newcommand{\ham}{\mathcal{H}}
\newcommand{\act}{\mathcal{S}}
\newcommand{\mat}{\mathcal{M}}
\newcommand{\ampl}{\mathcal{A}}
\newcommand{\smat}{S}
\newcommand{\con}{\EuScript{A}}
\newcommand{\fs}{\EuScript{F}}

% propagators
\newcommand{\fpr}{\Delta}
\newcommand{\fprp}{\tilde{\Delta}}
\newcommand{\gpr}{\Delta}
\newcommand{\gprp}{\tilde{\Delta}}
\newcommand{\dpr}{\Sigma}
\newcommand{\dprp}{\tilde{\Sigma}}

% operators
\newcommand{\normord}{\mathfrak{N}}
\newcommand{\tempord}{\mathfrak{T}}
\newcommand{\parity}{\mathcal{P}}
\newcommand{\chargec}{\mathcal{C}}
\newcommand{\timer}{\mathcal{T}}

% symbols and constants
\newcommand{\msb}{\overline{\text{MS}}}
\newcommand{\eg}{\gamma_\text{E}}

% differential operators
\newcommand{\dd}{\mathrm{d}}
\newcommand{\pa}{\partial}
\newcommand{\covder}{\EuScript{D}}
\newcommand{\na}{\nabla}
\newcommand{\grad}{\boldsymbol{\nabla}}
\newcommand{\dive}{\boldsymbol{\nabla}\cdot}
\newcommand{\rot}{\boldsymbol{\nabla}\times}
\newcommand{\lap}{\triangle}

% small \over..arrow
\makeatletter
\newcommand{\overleftrightsmallarrow}{\mathpalette{\overarrowsmall@\leftrightarrowfill@}}
\newcommand{\overrightsmallarrow}{\mathpalette{\overarrowsmall@\rightarrowfill@}}
\newcommand{\overleftsmallarrow}{\mathpalette{\overarrowsmall@\leftarrowfill@}}
\newcommand{\overarrowsmall@}[3]{%
  \vbox{%
    \ialign{%
      ##\crcr
      #1{\smaller@style{#2}}\crcr
      \noalign{\nointerlineskip}%
      $\m@th\hfil#2#3\hfil$\crcr
    }%
  }%
}
\def\smaller@style#1{%
  \ifx#1\displaystyle\scriptstyle\else
    \ifx#1\textstyle\scriptstyle\else
      \scriptscriptstyle
    \fi
  \fi
}
\makeatother
\newcommand{\smlra}[1]{\overleftrightsmallarrow{#1}}
\newcommand{\smla}[1]{\overleftsmallarrow{#1}}
\newcommand{\smra}[1]{\overrightsmallarrow{#1}}

% functions
\DeclareMathOperator{\sech}{sech}
\DeclareMathOperator{\Exp}{Exp}
\DeclareMathOperator{\li}{Li_2}

% number sets & fields
\newcommand{\N}{\mathbb{N}}
\newcommand{\Z}{\mathbb{Z}}
\newcommand{\Q}{\mathbb{Q}}
\newcommand{\R}{\mathbb{R}}
\newcommand{\C}{\mathbb{C}}
\newcommand{\K}{\mathbb{K}}

\newcommand{\Sn}[1]{\mathbb{S}^{#1}}
\newcommand{\Dn}[1]{\mathbb{D}^{#1}}

% groups
\newcommand{\sy}[1]{S_{#1}} % symmetric group

% Lie groups
\newcommand{\Ot}{\mathrm{O}(3)}
\newcommand{\SOt}{\mathrm{SO}(3)}
\newcommand{\On}[1]{\mathrm{O}(#1)}
\newcommand{\SOn}[1]{\mathrm{SO}(#1)}
\newcommand{\Un}[1]{\mathrm{U}(#1)}
\newcommand{\SUn}[1]{\mathrm{SU}(#1)}
\newcommand{\GL}[1]{\mathrm{GL}(#1)}
\newcommand{\SL}[1]{\mathrm{SL}(#1)}

% Lie algebras
\newcommand{\sua}[1]{\mathfrak{su}(#1)}
\newcommand{\sla}[1]{\mathfrak{sl}(#1)}

% Clifford algebras
\newcommand{\cla}[1]{\mathfrak{cl}(#1)}
\newcommand{\clar}[1]{\mathfrak{cl}_{#1}(\R)}
\newcommand{\clac}[1]{\mathfrak{cl}_{#1}(\C)}

% Lorentz/Poincaré group/algebra
\newcommand{\lorg}{\mathrm{SO}^+(1,3)}
\newcommand{\lora}{\mathfrak{so}^+(1,3)}
\newcommand{\pog}{\mathrm{ISO}^+(1,3)}
\newcommand{\poa}{\mathfrak{iso}^+(1,3)}

% Dirac algebra
\newcommand{\dira}{\mathfrak{cl}_{1,3}(\C)}

% spin group
\DeclareMathOperator{\spin}{Spin}

% SU(n) Casimirs
\newcommand{\caf}{C_\mathtt{F}}
\newcommand{\caa}{C_\mathtt{A}}
\newcommand{\ttr}{T_\text{R}}

% units of measure
\newcommand{\m}{\,\mathrm{m}}
\newcommand{\ang}{\,\mathrm{\AA}}
\newcommand{\fm}{\,\mathrm{fm}}

\newcommand{\barn}{\,\mathrm{barn}}

\newcommand{\cels}{\,^{\circ}\mathrm{C}}

\newcommand{\ev}{\,\mathrm{eV}}
\newcommand{\kev}{\,\mathrm{keV}}
\newcommand{\mev}{\,\mathrm{MeV}}
\newcommand{\gev}{\,\mathrm{GeV}}
\newcommand{\tev}{\,\mathrm{TeV}}


\usepackage{overarrows}
\usepackage{slashed}
\usepackage{simpler-wick}

\usepackage{tikz}
\usepackage[compat=1.1.0]{tikz-feynman}

\graphicspath{{./imgs/}}
\captionsetup[figure]{labelfont={sf}}

% bibliography

\usepackage{csquotes}
\usepackage[
backend = biber,
style = numeric-comp,
sorting = none
]{biblatex}

\defbibentryset{el-weak}{Glashow-1961,Salam-1964,Weinberg-1967}
\defbibentryset{strong}{Fritzsch-1972,Fritzsch-1973}
\defbibentryset{higgs}{Higgs-1964-1,Higgs-1964-2,Englert-1964,Guralnik-1964}

\defbibentryset{dim-reg}{Thooft-1972,Bollini-1972}

\addbibresource{sm.bib}
\addbibresource{bibl.bib}

\renewbibmacro{in:}{}

\AtEveryBibitem{%
	\clearfield{publisher}%
	\clearfield{month}%
	\clearfield{numpages}%
	\clearfield{issue}%
	\clearfield{isbn}%
	\clearfield{issn}%
}

\DeclareFieldFormat[book,article]{journaltitle}{#1}
\DeclareFieldFormat[book,article]{volume}{\mkbibbold{#1}}

%macros

\newcommand{\qcd}{\Lambda_\text{QCD}}
\newcommand{\tm}{p_\text{T}}

\newcommand{\ren}{\mu_\text{R}}
\newcommand{\fac}{\mu_\text{F}}
\newcommand{\reg}{\mu_0}

\newcommand{\pdfb}[2]{f_{#1}^{(#2)}}
\newcommand{\pdfr}[2]{\bar{f}_{#1}^{(#2)}}

\newcommand{\rc}{\alpha_\text{s}}
\newcommand{\rcr}{\alpha_\text{s}(\ren^2)}
\newcommand{\bc}{\alpha_\text{s,b}}

\newcommand{\lps}{\dd \bs{\Phi}}

\newcommand{\pcs}{\hat{\sigma}}
\newcommand{\hcs}{\sigma}

\newcommand{\pcsr}{\hat{\sigma}^\text{R}}
\newcommand{\pcsv}{\hat{\sigma}^\text{V}}
\newcommand{\pcspdf}{\hat{\sigma}^\text{pdf}}

\newcommand{\de}{\epsilon}

\newcommand{\tco}{\ve{T}} % colour operator
\newcommand{\tcc}{T} % colour charge
\newcommand{\cc}{\tco_i \cdot \tco_j} % colour-correlation term

\newcommand{\ema}{\mathcal{E}}
\newcommand{\zm}{z_\text{m}}

\newcommand{\leg}{L}
\newcommand{\ler}{\mathscr{L}}
\newcommand{\lmx}{\mathscr{L}_\text{m}}

\newcommand{\en}{\mathcal{G}}

\newcommand{\mq}{Q}
\newcommand{\maq}{\bar{Q}}

\newcommand{\prt}[2]{\mathcal{H}^{#1}_{#2,0}}
\newcommand{\fs}[2]{\mathcal{X}^{#1}_{#2}} % #1 amplitude index, #2 parton multiplicity
\newcommand{\fsp}[2]{\mathcal{X}^{#1}_{#2,0}}
\newcommand{\fsg}[2]{\mathcal{X}^{#1}_{#2}(g)}
\newcommand{\fsq}[2]{\mathcal{X}^{#1}_{#2}(q)}
\newcommand{\fsqm}[2]{\mathcal{X}^{#1}_{#2}(\mq)}
\newcommand{\fsbq}[2]{\mathcal{X}^{#1}_{#2}(\bar{q})}
\newcommand{\fsbqm}[2]{\mathcal{X}^{#1}_{#2}(\maq)}
\newcommand{\fsu}[2]{\mathcal{H}^{#1}_{#2}}
\newcommand{\fsm}[2]{\mathcal{X}^{#1}_{#2,\text{m}}}

\newcommand{\ol}[1]{\overline{#1}}
\newcommand{\pd}[1]{\left[ #1 \right]_+}

\newcommand{\ur}{\mathfrak{m}}

\newcommand{\spl}{P}
\newcommand{\ap}{\hat{P}^{(0)}}
\newcommand{\nspl}{\mathcal{P}}
\newcommand{\gspl}{\mathcal{P}^\text{gen}}
\newcommand{\fspl}{\mathcal{P}^\text{fin}}
\newcommand{\fap}{P^\text{gen}}

\newcommand{\flm}{\mathcal{F}}
\newcommand{\op}{\mathcal{O}}
\newcommand{\ips}[1]{\left\langle #1 \right\rangle}

\newcommand{\nsym}{\mathcal{N}_\text{sym}}
\newcommand{\navg}{\mathcal{N}_\text{avg}}

\newcommand{\so}{\mathsf{S}}
\newcommand{\co}{\mathsf{C}}
\newcommand{\oso}{\overline{\mathsf{S}}}
\newcommand{\oco}{\overline{\mathsf{C}}}
\newcommand{\nlo}{\mathsf{O}_\text{NLO}}

\newcommand{\iso}{I_\text{S}}
\newcommand{\ico}{I_\text{C}}
\newcommand{\ioo}{I_1}
\newcommand{\ivo}{I_\text{V}}
\newcommand{\ito}{I_\text{T}}
\newcommand{\irs}{I_\text{CDR}}
\newcommand{\isvo}{I_{\text{S}+\text{V}}}

\newcommand{\eik}{\mathcal{S}}
\newcommand{\hyp}{{_2F_1}}

\newcommand{\rop}{\mathcal{Z}}

\newcommand{\mm}[1]{\hat{#1}}
\newcommand{\eit}{I}
\newcommand{\eitt}[2]{I_{#1}^{(#2)}}
\newcommand{\ns}{\mathcal{V}}
\newcommand{\nss}[2]{\mathcal{V}_{#1}^{(#2)}}

\newcommand{\ci}[1]{\chi_{#1}}
\newcommand{\finc}[1]{\text{\cyrillicfont ж}_{#1}}
\newcommand{\finci}[1]{\text{\cyrillicfont и}_{#1}}
\newcommand{\finm}[1]{\text{\cyrillicfont д}_{#1}}
\newcommand{\fing}[1]{\text{\cyrillicfont ц}_{#1}}

\title{Thesis}
\author{Leonardo Cerasi}

\begin{document}

\frontmatter

\customtitlepage{Physics}{Infrared-Safe NLO Calculations with Massive Quarks: An Extension of the Nested Soft-Collinear Subtraction Formalism}{Prof. Raoul Horst Röntsch}{Leonardo Cerasi}{11410A}{2024--2025}
\clearpage

%\begin{dedicationpage}
%  «\textit{What we know is a drop,} \\
%  \textit{what we don't know is an ocean.}» \\
%  I. Newton
%\end{dedicationpage}

%\begin{dedication}
%  «\textit{We live on a placid island of ignorance} \\
%  \textit{in the midst of black seas of infinity,} \\
%  \textit{and it was not meant that we should voyage far.}» \\
%  H. P. Lovecraft
%\end{dedication}

\newpage

\chapter*{Abstract}
\selectlanguage{english}

The treatment of infrared divergences in Next-to-Leading Order (NLO) QCD calculations becomes significantly more complex when accounting for massive quarks, particularly in processes where mass effects cannot be neglected. We present a generalization of the Nested Soft-Collinear (NSC) subtraction scheme to incorporate arbitrary massive quark flavours, preserving the original framework’s efficiency while systematically addressing mass-dependent divergences. \\
By removing the need for massless approximations, this work enables precision calculations in particle production processes where quark mass effects are theoretically or phenomenologically relevant.


\newpage

\toc

\pagestyle{contents}

\mainmatter

\pagestyle{body-thesis}

\chapter{Introduction}
\selectlanguage{english}

The Standard Model of Particle Physics (SM) is, as of now, the most complete theoretical framework in subatomic physics, describing all known elementary particles and fundamental interactions \cite{Glashow-1961, Salam-1964, Weinberg-1967, Fritzsch-1972, Fritzsch-1973, Higgs-1964-1, Higgs-1964-2, Englert-1964, Guralnik-1964}, except for the very weak gravitational force. Over the last fifty years, the SM has been continuously tested via experiments, mainly in the context of particle colliders, and its validity has been confirmed by the agreement of its predictions with experimental observations, culminating in 2012 with the discovery of the Higgs boson \cite{ATLAS-2012, CMS-2012} at the Large Hadron Collider (LHC) at CERN.

Despite its success, there is strong evidence for the existence of Physics Beyond the SM (BSM): the most prominent indications include the existence of dark matter and dark energy, the observed matter-antimatter asymmetry and the non-vanishing neutrino masses. Contrary to earlier expectations, though, since its first run in 2009 the LHC has not yet detected any new particle, nor any confirmation of BSM physics: on the contrary, the huge amount of data collected in its three runs (Run 3 is currently ongoing) puts increasingly stricter exclusion limits to BSM models \cite{CMS-ATLAS-SUSY, Bsekidt-2012, Ghosh-2025, Crivellin-2015}. As a consequence, the masses of hypothesized new particles become so large that, although still not excluded, their frequent production at the LHC is hardly possible.

The lack of any observation of BSM physics at the LHC has sparked a change in the research paradigm in High-Energy Particle Physics. Substantial further increase in the energy of colliding particles at the LHC (or anywhere else) is currently not feasible, hence it is clear that BSM physics searches based on the idea of detectable resonant-like structures on top of flat backgrounds has to be supplemented by new research strategies. Indeed, new particles can still be produced at the LHC, though in a way which does not allow for their direct detection: undetected light particles could be hidden in complex final states, while heavy particles could be virtually produced for extremely short periods of time, before disappearing back into the quantum vacuum. In the latter case, these virtual particles could affect measurable properties, prompting their indirect detection as deviations from SM predictions.

Given this shift of focus towards higher experimental precision in collider physics, it is clear that reliable theoretical predictions of hadron-collision processes are needed.

\section{QCD in collider physics}

Systematic searches for BSM physics through precision studies at hadron colliders are difficult to perform, given the poorly-understood nature of the strong force which keeps hadrons together. In fact, the strong interaction is described by Quantum Chromodynamics (QCD), which has the complicated mathematical structure of a non-Abelian gauge theory (\secref{ssec:gauge-th} for details).

Although it has not been possible, so far, to describe the properties of a single proton from first principles, in the context of hadron collisions a first-principle description is made possible for a particular class of processes: hard scattering processes.

Even though hard scattering processes have a lower probability of happening, with respect, for example, to elastic scattering processes, they are of great interest to modern particle physics. To understand why, a remarkable property of non-Abelian gauge theory needs to be states: \bctxt{asymptotic freedom}. The evolution of the coupling $ \alpha(\mu^2) $ of a quantum field theory as a function of the energy scale $ \mu $ is described by the renormalization group equation (see e.g. Chapter 12 of \cite{Peskin-1995}):
\begin{equation}
  \mu^2 \frac{\dd \alpha(\mu^2)}{\dd \mu^2} = - 2 \beta(\alpha(\mu^2)) \alpha(\mu^2)
  \label{eq:ren-gr}
\end{equation}
where the $ \beta $-function has a power-series expansion like:
\begin{equation}
  \beta(\alpha) = \sum_{n \in \N_0} \beta_n \left( \frac{\alpha}{4\pi} \right)^{n+1} = \beta_0 \frac{\alpha}{4\pi} + \smo(\alpha^2)
\end{equation}
For non-Abelian gauge theories $ \beta_0 > 0 $ (for QCD, see \cite{Gross-1973, Politzer-1973}), hence the coupling becomes small at high energies (small distances). This allows for a perturbative description of hard scattering processes, which are characterized by a large momentum transfer: this kind of events happen at small distances, hence the hadronic scattering can be studied through the interaction between single partons (see \figref{fig:part-scatt}), i.e. the quarks and gluons which compose the hadrons.

\begin{figure}
  \centering
  \includegraphics[width = 0.90 \textwidth]{imgs/part-scatt.png}
  \caption{Schematics of hard hadronic scattering. Due to asymptotic freedom, individual partons can be assumed to be free particles, so that their (hard) scattering can be computed via perturbative QCD. Initial- and final-state radiation accounts for beyond-leading-order effects. Figure from \cite{Asteriadis-2020}.}
  \label{fig:part-scatt}
\end{figure}

\subsection{Hadronic scattering}

Theoretical predictions for hard hadronic scattering are based on the factorization theorem \cite{Collins-1989}, which states that hadronic cross-sections can be computed from partonic cross-sections as:
\begin{equation}
  \dd\hcs_{h_1,h_2}(P_1 , P_2) = \sum_{a,b} \int_{[0,1]^2} \dd \xi_1 \dd \xi_2 \, f_a^{(h_1)}(\xi_1, \fac^2) f_b^{(h_2)}(\xi_2, \fac^2) \, \dd\pcs_{a,b}(\xi_1 P_1, \xi_2 P_2, \rc, \ren^2, \fac^2)
\end{equation}
Here, the two scattering hadrons $ h_1 , h_2 $ have momenta $ P_1 , P_2 $, while the scattering partons $ a , b $ have momentum fractions $ x_1 P_1 , x_2 P_2 $. The factorization scale $ \fac $ is taken to be equal to the renormalization scale $ \ren $ for the rest of this work.

The link between hadron-scale physics and parton-scale physics is given by \bctxt{parton distribution functions} (PDFs): in general, $ f_a^{(h)}(x) $ is the numerical probability of finding a parton $ a $ inside the hadron $ h $ with a definite energy fraction $ x : p_a = x P_h $, where $ p_a $ and $ P_h $ are the momenta of the parton and of the hadron, respectively. A crucial property of PDFs is their universality, as they are energy-independent: this means that they can be measured in a particular process and then used in many others. However, they incapsulate non-perturbative effects which are poorly understood, thus they have not been computed from first principles so far.

Another instance of non-perturbative effects arises when considering that, after the partonic interaction, final-state partons can be clustered in the so-called jets: despite the difficulty in formally defining jets (for a review of various jet algorithms, see \cite{Salam-2010}), they can intuitively be pictured as seeds of hadronic energy flow with are barely affected by non-perturbative QCD effects. While on short time-scales QCD can be treated perturbatively, on long time-scales QCD partons (and so jets too) are subject to the phenomenon of \bctxt{hadronization}.

\begin{figure}
  \centering
  \includegraphics[width = 1.00 \textwidth]{imgs/hadr-scatt.pdf}
  \caption{Hadronization of jets produced in a hard hadronic scattering. Incoming hadrons produce initial-state radiation (blue), which determines two hard scattering events (red and purple blobs): these scatterings give rise to partonic jets (red and purple) which undergo hadronization (light green blobs), eventually decaying into heavy hadrons (dark green blobs) and soft radiation (yellow). Figure from \cite{Hoche-2014}.}
  \label{fig:hadr-scatt}
\end{figure}

Hadronization can be explained by considering a solution to \eref{eq:ren-gr}, found introducing a reference scale $ Q $:
\begin{equation}
  \rcr = \frac{\rc(Q^2)}{1 + 2 \rc(Q^2) \frac{\beta_0}{4\pi} \log \frac{\ren^2}{Q^2}}
\end{equation}
For e.g. $ \rc(m_Z^2) \approx 0.118 $ \cite{PDG-2024}. It is customary to introduce a QCD scale $ \qcd \approx 300 \mev $, so that:
\begin{equation}
  \rcr = \frac{1}{2\frac{\beta_0}{4\pi} \log \frac{\ren^2}{\qcd^2}}
\end{equation}
This expression shows that $ \ren \gg \qcd $ is the perturbative region, where asymptotic freedom makes $ \rc $ small enough for perturbative techniques. On the other hand, for $ \ren \rightarrow \qcd $ a Landau pole is present: this pole signals the breakdown of perturbation theory and the hadronization of partons, i.e. their confinement into bound states (hadrons).

As illustrated in \figref{fig:hadr-scatt}, the hard scattering process occurs at high energy $ Q \gg \qcd $ (typically $ Q \sim 1\tev $ at the LHC), resulting in jets which are unaffected by non-perturbative QCD, as their energy is well above the QCD scale; however, this energy is radiated off in the form of parton showers, and, when the threshold energy $ \qcd $ is reached, non-perturbative effects come into play, resulting in the hadronization of jets.

\subsection{Partonic scattering}

For the rest of this work, the analysis is restricted to perturbative effects only. As the partonic scattering can be treated with perturabtion theory, the partonic cross section for the scattering of two partons $ a , b $ with momenta $ p_1 , p_2 $ can be expressed as a power series in the running coupling:
\begin{equation}
  \dd\pcs_{a,b}(p_1 , p_2) = \sum_{n \in \N_0} \dd\pcs_{a,b}^{(n)}(p_1 , p_2)
\end{equation}
where each term is $ \dd\pcs^{(n)} \sim \rc^{n_0 + n} $, with $ n_0 \in \N $ giving the leading-order-dependence on $ \rcr $ due to the leading-order (LO) process, which is usually (but not always) a tree-level process.

The $ n \ge 1 $ terms form what are denoted by QCD corrections. Focusing on next-to-leading-order (NLO) corrections, they can be of two kinds: real corrections and virtual corrections. Real corrections consist in the emission of an additional parton as initial- or final-state radiation, while virtual corrections present an additional partonic loop. Examples of a real and a virtual correction to the Drell-Yan process may be:
\begin{equation*}
  \begin{tikzpicture}
    \begin{feynman}

      \vertex (a1) {\(q\)};
      \vertex[below = 3cm of a1] (a2) {\(\bar{q}\)};

      \vertex[below = 1.5cm of a1] (b1) {};
      \vertex[right = 2.25cm of b1, dot] (v1) {};

      \vertex[right = 2.25cm of v1, dot] (v2) {};
      \vertex[right = 2.25cm of v2] (b2) {};

      \vertex[above = 1.5cm of b2] (a3) {\(e^-\)};
      \vertex[below = 1.5cm of b2] (a4) {\(e^+\)};

      \vertex[dot] (c1) at ($(a1) + (1.5,-1)$) {};
      \vertex (c2) at ($(c1) + (1.5,1)$) {\(g\)};

      \diagram* {
	(a1) -- [fermion] (v1),
	(a2) -- [anti fermion] (v1),

	(v1) -- [photon, edge label = \(\gamma^*\)] (v2),

	(v2) -- [fermion] (a3),
	(v2) -- [anti fermion] (a4),

	(c1) -- [gluon] (c2),
      };
    \end{feynman}
  \end{tikzpicture}
  \qquad \qquad
  \begin{tikzpicture}
    \begin{feynman}

      \vertex (a1) {\(q\)};
      \vertex[below = 3cm of a1] (a2) {\(\bar{q}\)};

      \vertex[below = 1.5cm of a1] (b1) {};
      \vertex[right = 2.25cm of b1, dot] (v1) {};

      \vertex[right = 2.25cm of v1, dot] (v2) {};
      \vertex[right = 2.25cm of v2] (b2) {};

      \vertex[above = 1.5cm of b2] (a3) {\(e^-\)};
      \vertex[below = 1.5cm of b2] (a4) {\(e^+\)};

      \vertex[dot] (c1) at ($(a1) + (0.75,-0.5)$) {};
      \vertex[dot] (c2) at ($(a2) + (0.75,0.5)$) {};

      \diagram* {
	(a1) -- [fermion] (v1),
	(a2) -- [anti fermion] (v1),

	(v1) -- [photon, edge label = \(\gamma^*\)] (v2),

	(v2) -- [fermion] (a3),
	(v2) -- [anti fermion] (a4),

	(c1) -- [gluon, edge label' = \(g^*\)] (c2),
      };
    \end{feynman}
  \end{tikzpicture}
\end{equation*}
In general, then:
\begin{equation}
  \dd\pcs_{a,b}^{(1)}(p_1 , p_2) = \dd\pcsr_{a,b}(p_1 , p_2) + \dd\pcsv_{a,b}(p_1 , p_2) + \dd\pcspdf_{a,b}(p_1 , p_2)
\end{equation}
where $ \dd\pcsr_{a,b} $ and $ \dd\pcsv_{a,b} $ are the single-real and 1-loop corrections. The additional correction $ \dd\pcspdf_{a,b} $ is due to the collinear renormalization of PDFs.

\section{Singularities in QCD amplitudes}

show how divergences arise in the soft and collinear limits for the real case

cite UV-singularities for the virtual case (link to section about renormalization), then say that IR-singularities cancel when combined with real ones due to the kinoshita-lee-nauenberg theorem

prove the general form of collinear pdf renormalization counterterms












\chapter{Preliminaries}
\selectlanguage{english}

\section{Renormalization scheme}

The computation of NLO corrections to scattering processes often involves diverging loop amplitudes. In order to obtain finite results from these divergences, a renormalization scheme must be implemented.

As the generalized Catani's formula for virtual corrections is provided in \cite{Catani-2001} in a charge-unrenormalized (but mass-renormalized) way, it is necessary to carry out the renormalization procedureexplicitly. To this end, we formally state the renormalization scheme adopted in this work.

\subsection{Dimensional regularization}

In the evaluation of loop amplitudes, both UV- and IR-singularities are encountered. The most efficient way to simultaneously regularize both types of divergences is dimensional regularization, a regularization scheme first introduced by 't Hooft and Veltman in \cite{thooft-1972}.

In general, the dimensional regularization scheme consists in the analytic continuation of loop momenta to $ d = 4 - 2 \de $ dimensions, with $ \de \in \C : \Re\de < 0 $. This procedure turns loop integrals into meromorphic\footnotemark functions of $ \de \in \C $, allowing for the isolation of divergences as poles in $ \de $.

\footnotetext{Given an open set $ D \subset \C $, then $ f : D \rightarrow \C $ is \textit{meromorphic} if it is holomorphic on $ D - P $, where $ P \subset D $ is a set of isolated points called \textit{poles}. Recall that a function $ f : D \rightarrow \C $ is \textit{holomorphic} on $ D $ if it is complex differentiable at every point in $ D $.}

The dimensional regularization prescription leaves freedom in choosing the dimensionality of external momenta, as well as the number of polarizations of both external and internal particles, thus allowing for the definition of different regularization schemes. We choose to work with \bctxt{conventional dimensional regularization} (CDR), in which all momenta and polarization are analytically continued to $ d $ dimensions, as opposed to the 't Hooft--Veltman scheme (HV), in which only internal momenta and polarizations are.

When considering non-chiral gauge theories like QCD, CDR is the most natural choice, as the main difference between CDR and HV is the treatment of purely $ 4 $-dimensional objects, i.e. $ \gamma^5 $ and $ \epsilon_{\mu \nu \sigma \rho} $. In particular, in CDR both the Dirac algebra and Lorentz indices are analytically continued to $ d $ dimensions, leading to a mathematical inconsistency when $ d \neq 4 $.

\begin{observation}{Inconsistency in CDR}{}
  In $ 4 $-dimensional Minkowski space $ \R^{1,3} $ with metric signature $ \eta = (+,-,-,-) $, the Dirac algebra is defined as $ \dira \cong \clar{1,3} \otimes \C $ (complexification\footnotemark). This Clifford algebra admits a matrix representation with generators $ \{\gamma^\mu\}_{\mu = 0, 1, 2, 3} \subset \C^{4 \times 4} $ such that:
  \begin{equation}
    \{\gamma^\mu , \gamma^\nu\} = 2 \eta^{\mu \nu} \tens{I}_4
  \end{equation}
  As in CDR Lorentz indices too are $ d $-dimensional, the Dirac algebra becomes $ \clac{1,d-1} \simeq \clar{1,d-1} \otimes \C $, where usual Minkowski space $ \R^{1,3} $ is replaced by the pseudo-Euclidean space $ \R^{1,d-1} $ with metric signature $ \eta = (+,-,\dots,-) $. This structure, however, is ill-defined, as for $ d \in \N $ then $ \dim_\C \clac{1,d-1} = 2^d $, but a finite-dimensional algebra of dimension $ 2^d $ with $ d \notin \N $ is meaningless.

  To solve this issue, in CDR spinor indices remain $ 4 $-dimensional, i.e. we consider a matrix representation generated by $ \{\gamma^\mu\}_{\mu = 0, \dots, d-1} \subset \C^{4 \times 4} $ and impose the formal relations:
  \begin{equation}
    \{\gamma^\mu , \gamma^\nu\} = 2 \eta^{\mu \nu} \tens{I}_d
    \label{eq:gen-dir-alg}
  \end{equation}
  Consistency is achieved by analytical continuation of trace identities: indeed, traces obtained by recursively applying \eref{eq:gen-dir-alg} (like $ \tr\{\gamma^\mu \gamma^\nu\} = 4 \eta^{\mu \nu} $) are still valid in $ d $-dimensions, as the only dependence on dimension comes from contractions such as $ \tensor{\eta}{^\mu_\mu} = d $.

  A fatal inconsistency of CDR arises when considering $ \gamma^5 $. In $ \dira $, this matrix is defined as $ \gamma^5 \defeq \frac{i}{4!} \epsilon_{\mu \nu \rho \sigma} \gamma^\mu \gamma^\nu \gamma^\rho \gamma^\sigma = i \gamma^0 \gamma^1 \gamma^2 \gamma^3 $ and has the property $ \{\gamma^5 , \gamma^\mu\} = 0 \,\, \forall \mu = 0,1,2,3 $, which allows to prove this identity:
  \begin{equation}
    \tr\{\gamma^5 \gamma^\mu \gamma^\nu \gamma^\rho \gamma^\sigma\} = -4 i \epsilon^{\mu \nu \rho \sigma}
    \label{eq:top-form-gamma-5}
  \end{equation}
  This construction cannot be generalized consistently to $ d \notin \N $. To show this, assume a $ d $-dimensional generalization $ \gamma^5 \in \C^{4 \times 4} : \{\gamma^5 , \gamma^\mu\} = 0 \,\,\forall \mu = 0, \dots, d-1 $, so that $ \gamma^5 \gamma^{\mu_1} \dots \gamma^{\mu_n} = \left( -1 \right)^n \gamma^{\mu_1} \dots \gamma^{\mu_n} \gamma^5 $. By the cyclicity of the trace, then:
  \begin{equation}
    \left( 1 - \left( -1 \right)^n \right) \tr\{\gamma^5 \gamma^{\mu_1} \dots \gamma^{\mu_n}\} = 0
  \end{equation}
  Consider $ n = d $: clearly $ 1 - \left( -1 \right)^d = 1 - e^{i \pi d} \neq 0 $ for $ d \notin \N $, so $ \tr\{\gamma^5 \gamma^{\mu_1} \dots \gamma^{\mu_d}\} = 0 $. This is an open contradiction to \eref{eq:top-form-gamma-5}, as analytic continuation should continuously preserve the top product of the algebra as $ \de \rightarrow 0 $.

  This contradiction is the explicit manifestation of a more profound topological issue of analytically continuing the number of dimensions: the Levi-Civita symbol in $ d = 4 $ is linked\footnotemark to the Grassmann algebra $ \bigwedge(\R^{1,3}) $, and in particular to its top-form, but $ \bigwedge^k(\R^d) $ is only defined for $ d \in \N $, so the top exterior subspace $ \bigwedge^d(\R^{1,d-1}) $ is meaningless for $ d \notin \N $ and the Levi-Civita symbol cannot be analytically continued to $ d = 4 - 2 \de $ dimensions.
\end{observation}

\footnotetext{Given an $ n $-dimensional vector space $ V(\K) $ with a quadratic form $ q $, associated linear form $ \, \omega $ and orthogonal basis $ \{e_i\}_{i = 1,\dots,n} $, and a unital associative $ \K $-algebra $ \mathcal{A} $, a \textit{Clifford mapping} is an injective $ \K $-linear map $ \rho : V \rightarrow \mathcal{A} : \mathit{1} \notin \rho(V) \land \rho(x)^2 = - q(x) \mathit{1} \,\,\forall x \in V $. If $ \rho(V) $ generates $ \mathcal{A} $, then $ (\mathcal{A},\rho) $ is a \textit{Clifford algebra} for $ (V,q) $, and is denoted by $ \cla{V} $. It can be shown with simple algebraic manipulation that $ \{\rho(x),\rho(y)\} = 2 \omega(x,y)\mathit{1} \,\, \forall x,y \in V $.}
\footnotetext{Given an $ n $-dimensional vector space $ v(\K) $, its \textit{Grassmann algebra} (or exterior algebra) is the $ \N_0 $-graded algebra $ \bigwedge(V) = \bigoplus_{k = 0}^n \bigwedge^k(V) $ of $ k $-forms. It can be shown that $ \dim_\K \bigwedge^k(V) = \binom{n}{k} $, hence the top exterior subspace $ \bigwedge^n(V) $ is $ 1 $-dimensional: indeed, given a basis $ \{e_i\}_{i = 1,\dots,n} \subset V $, it is $ \bigwedge^n(V) = \braket{e_1 \wedge \dots \wedge e_n} $, and the Levi-Civita symbol is normalized so that $ \epsilon^{1 \dots n} $ has the same sign of $ e_1 \wedge \dots \wedge e_n $.
}














\chapter{NSC Subtraction Scheme}
\selectlanguage{english}

The aim of the NSC subtraction scheme (SS) is to compute integrated subtraction terms which account for QCD corrections to the inclusive\footnotemark production of jets in a hadron collider, i.e. to the process:
\begin{equation}
  p + p \rightarrow X + N \,\text{jets}
\end{equation}
Here, $ X $ is a colour-neutral system. The hadron-scale physics is known to be separated from the parton-scale physics (see Section 1.1 of \cite{Collins-2011}): this makes it possible for us to only manipulate partonic cross-sections according to \eref{eq:fact-th}, where now the sum runs over all intial-state massless partons $ a $ and $ b $ which contribute to the production of the considered final state. Moreover, for the rest of this work we set $ \ren = \fac = \mu $, where $ \mu $ is the typical energy scale of the considered process.

\footnotetext{Inclusive jet production denotes the theoretical prediction (or experimental measurement) of the cross-section for the production of jets of given kinematics, while summing/integrating over all other final-state radiation and particles.}

Denoting the partons' momenta as $ p_i \equiv \xi_i P_i $, $ i = 1,2 $, and suppressing the explicit dependence on the running coupling and the renormalization scale, it is possible to express the LO term of \eref{eq:part-ser-exp} as (see \secref{sec:ph-sp-p}):
\begin{equation}
  \dd\pcs_{a,b}^{(0)}(p_1 , p_2) \defeq \sum_\text{f} \frac{\mathcal{N}}{2\hat{s}} \int \lps_n \, \abs{\ampl^{(0)}_m(p_1, p_2, p_X, p_\text{f})}^2 \op_m(p_X, p_\text{f})
  \label{eq:ds-lo}
\end{equation}
where $ \hat{s} \equiv 2 p_1 \cdot p_2 $ is the partonic center-of-mass (CM) energy squared, $ p_\text{f} $ is the total final-state momentum and the normalization factor $ \mathcal{N} $ includes all necessary symmetry factors (e.g. $ (\ngl!)^{-1} $, with $ \ngl $ number of resolved gluons in the final state), as well as averaging factors for initial-state colours and helicities. The sum runs over all possible partonic final states for the considered process (formalized in the next section).

Note that $ \op_m $ is an IR-finite measurement function defining the observable, which ensures that the final state contains at least $ N $ resolved jets: in particular, if the energy of a final-state gluon vanishes (soft limit), or if two partons become collinear to one another (collinear limit), then $ \op_{m + n} \rightarrow \op_{m + n - 1} $ for $ n \in \N $, and $ \op_m \rightarrow 0 $.

Similarly, it is possible to write the NLO corrections in \eref{eq:part-ser-exp} as:
\begin{equation}
  \dd\pcs_{a,b}^\text{R}(p_1 , p_2) \defeq \sum_\text{f} \frac{\mathcal{N}_\text{R}}{2\hat{s}} \int \lps_{m+1} \, \abs{\ampl^{(0)}_{m+1}(p_1, p_2, p_X, p_\text{f})}^2 \op_{m+1}(p_X, p_\text{f})
  \label{eq:real-cs}
\end{equation}
\begin{equation}
  \dd\pcs_{a,b}^\text{V}(p_1 , p_2) \defeq \sum_\text{f} \frac{\mathcal{N}_\text{V}}{2\hat{s}} \int \lps_m \, 2\Re \braket{\ampl_m^{(0)} | \ampl_m^{(1)}} \op_m(p_X, p_\text{f})
  \label{eq:virt-cs}
\end{equation}
\begin{equation}
  \dd\pcs_{a,b}^\text{C}(p_1 , p_2) \defeq \frac{\rcr}{2\pi} \frac{1}{\de} \sum_c \int_0^1 \frac{\dd z}{z} \left[ \ap_{c,a}(z) \dd\pcs_{c,b}^{(0)}(z p_1 , p_2) + \ap_{c,b}(z) \dd\pcs_{a,c}^{(0)}(p_1 , z p_2) \right]
  \label{eq:pdf-cs}
\end{equation}
Note that in \eref{eq:real-cs} the final state contains $ m + 1 $ partons. The \bctxt{Altarelli-Parisi splitting kernels} are listed in \secref{sec:spl-func}, and proof of \eref{eq:pdf-cs} is provided in \secref{ssec:coll-ren}.

The rest of this chapter is devoted to the extrapolation of IR-singularities from \eeref{eq:real-cs}{eq:pdf-cs}, proving their cancellation and providing the associated integrated counterterms.

\section{Nested subtraction}

As suggested by the name, in the NSC SS the IR-poles of real corrections are removed sequentially, starting from those arising from soft limits and then subtracting the collinear ones from the soft-regulated terms.

To show this procedure, we introduce some notation. First of all, we define the integrand function in \eeref{eq:ds-lo}{eq:real-cs} as:
\begin{equation}
  \flm_{a,b}[\fs{n}{m}] \equiv \nsym \abs{\ampl_m^{(0)}(p_1, p_2, p_X, p_\text{f})}^2 \op_m(p_X, p_\text{f})
\end{equation}
where $ \fs{n}{m} $ is the set of $ m $ final-state partons and $ \mathcal{N}_\text{sym} $ is the relative symmetry factor. The index $ n \in \en_m(a,b,X) $ enumerates all the possible QCD final states which may contribute to the partonic process: these include all combinations of flavours $ \{f_i\}_{i = 1,\dots,m} $ consistent with the initial state $ (a,b) $ and the color-singlet $ X $. In the following, we suppress the arguments of $ \en_m $. The integration on the $ m $-parton final-state phase space is instead defined as:
\begin{equation}
  \ips{\flm_{a,b}[\fs{n}{m}]} \defeq \navg \int \lps_m \, \flm_m^{a,b}[\fs{n}{m}]
\end{equation}
where is the appropriate initial-state averaging factor. Then, we can rewrite \eeref{eq:ds-lo}{eq:real-cs} as:
\begin{equation}
  2\hat{s}\, \dd\pcs_{a,b}^{(0)} = \sum_{n \in \en_m} \ips{\flm_{a,b}[\fs{n}{m}]}
  \qquad \qquad
  2\hat{s}\, \dd\pcs_{a,b}^\text{R} = \sum_{n \in \en_{m+1}} \ips{\flm_{a,b}[\fs{n}{m+1}]}
  \label{eq:lo-real}
\end{equation}

Soft and collinear singularities are isolated through operators acting on $ \flm $ functions: $ \so_i $ denotes the limits in which the parton $ i $ becomes soft, while $ \co_{ij} $ that in which the partons $ i $ and $ j $ become collinear to each other. In particular, these operators extract only the leading aymptotic behaviour of $ \flm $ which is non-integrable in $ d = 4 $ dimensions, hence, if they act on quantities without non-integrable singularities, then they identically vanish (e.g. $ \so_i \equiv 0 $ if $ i $ is a (anti)quark).

\subsection{Partonic sets}

A delicate step is the determination of which final-state partons can become unresolved: indeed, in fixed-order perturbative QCD, the number of final-state hard partons cannot drop below the number of jets in the LO process. This means that at NLO no more than one parton can become unresolved, and this is ensured by the $ \op $ operators. In order to use symmetry arguments to minimize the number of unresolved partons that need to be considered, we can partition the set of final-state partons as:
\begin{equation*}
  \fs{n}{m} = \fsg{n}{m} \cup \fsq{n}{m} \cup \fsbq{n}{m} \cup \fsqm{n}{m} \cup \fsbqm{n}{m}
\end{equation*}
which are respectively the subsets of final-state gluons, massless quarks, massless antiquarks, massive quarks and massive antiquarks. It is also usefull to define the set of all massless partons:
\begin{equation*}
  \prt{n}{m} \equiv \{a,b\} \cup \fsg{n}{m} \cup \fsq{n}{m} \cup \fsbq{n}{m} \equiv \{a,b\} \cup \fsu{n}{m}
\end{equation*}
as we only consider massless initial-state partons. Note that $ \fsq{n}{m} $ and $ \fsbq{n}{m} $ can be further partitioned into sets of definite massless quark flavours, and the same can be done with $ \fsqm{n}{m} $ and $ \fsbqm{n}{m} $ with massive quark flavours.

For the remainder of this chapter, we set $ \fsqm{n}{m} = \fsbqm{n}{m} = \emptyset $: this means that $ \fsu{n}{m} \equiv \fs{n}{m} $, however we keep this redundant notation in all the equations, as some will be used in the next chapter where massive partons will be included.

For NLO real emissions we consider an additional parton, i.e. a final state $ \fs{n}{m+1} $ with $ n \in \en_{m+1} $. To extract soft singularities from \eref{eq:lo-real}, we first consider a partition of unity such that:
\begin{equation}
  \sum_{i \in \fsu{n}{m+1}} \Delta^i = 1
  \quad : \quad
  \so_i \Delta^j = \delta_i^j
  \quad \land \quad
  \co_{ij} \Delta^k =
  \begin{cases}
    0 & i,j \neq k \\
    1 & i = k \,,\, j \in \{a,b\} \\
    z_{k,j} & i = k \,,\, j \in \fsu{n}{m+1} - \{i\}
  \end{cases}
  \label{eq:delta-part}
\end{equation}
with $ z_{k,j} \equiv E_k / (E_k + E_j) $. An explicit construction of these damping factors is given in \secref{sec:unit-part}. It is clear that a term multiplied by $ \Delta^i $ vanishes if any parton other than $ i $ becomes unresolved, thus this partition allows for the extraction of single unresolved partons:
\begin{equation*}
  2\hat{s}\, \dd\pcs_{a,b}^\text{R} = \sum_{n \in \en_{m+1}} \sum_{i \in \fsu{n}{m+1}} \ips{\Delta^i \flm_{ab}[\fs{n}{m+1}]}
\end{equation*}
We can relabel the potentially-unresolved parton $ i $ in each term as $ \ur_{f_i} $: then, for each allowed massless\footnotemark flavour $ f $, there are $ N_f $ equal terms, where $ N_f $ is the number of final-state partons of flavour $ f $. We can account for the cancellation of these factors with symmetry factors defining $ \fs{n}{m}(\ur_f) \equiv \fs{n}{m+1} - \{\ur_f\} $, thus imposing that the symmetry factors of $ \flm_{a,b}[\fs{n}{m}(\ur_f)] $ are determined ignoring $ \ur_f $ (i.e. implicitly multiplying by $ N_f $), but with the convention that the amplitude in $ \flm_{a,b}[\fs{n}{m}(\ur_f)] $ still contains the potentially-unresolved parton $ \ur_f $. Therefore:
\begin{equation}
  \begin{split}
    2\hat{s}\, \dd\pcs_{a,b}^\text{R}
    & = \sum_{n \in \en_{m+1}(g)} \ips{\Delta^\ur \flm_{a,b}[\fs{n}{m}(\ur_g)]} + \sum_{\rho = 1}^{n_f} \sum_{n \in \en_{m+1}(q_\rho)} \ips{\Delta^\ur \flm_{a,b}[\fs{n}{m}(\ur_{q_\rho})]} \\
    & \quad\, + \sum_{\rho = 1}^{n_f} \sum_{n \in \en_{m+1}(\bar{q}_\rho)} \ips{\Delta^\ur \flm_{a,b}[\fs{n}{m}(\ur_{\bar{q}_\rho})]}
  \end{split}
  \label{eq:part-chan}
\end{equation}
where $ \en_m(f) \subset \en_m : N_f \ge 1 $ denotes the subset of possible final states with at least one parton of flavour $ f $. The subscript $ f $ in $ \ur_f $ is suppressed when implicitly understood.

\footnotetext{Massive partons cannot go unresolved, as they do not determine neither soft nor collinear singularities.}

Now, the nested subtraction procedure introduced in \cite{rontsch-2017} can be applied. In particular, for each term we rewrite the identity operator as:
\begin{equation}
  \id = \so_\ur + \sum_{i \in \prt{n}{m}(\ur)} \oso_\ur \co_{i\ur} + \nlo^\ur
  \qquad \qquad
  \nlo^\ur \defeq \sum_{i \in \prt{n}{m}(\ur)} \oso_\ur \oco_{i\ur} \omega^{\ur i}
  \label{eq:nest-sub}
\end{equation}
where we defined the notation for generic operators $ \overline{\mathcal{O}} \equiv \id - \mathcal{O} $ and introduced an angular partition of unity (see \secref{sec:unit-part}):
\begin{equation}
  \sum_{i \in \prt{n}{m}(\ur)} \omega^{\ur i} = 1
  \quad : \quad
  \co_{j\ur} \omega^{\ur i} = \delta_j^i
  \label{eq:omega-part}
\end{equation}
There now remains to understand how the operators in \eref{eq:nest-sub} act on the $ \flm $ functions and how the a parton set $ \fs{n}{m}(\ur) $ changes when $ \ur $ effectively becomes unresolved.

\subsection{Soft limits}

We first consider the soft limit of each term in \eref{eq:part-chan}. For the last two terms this is trivial: as quarks do not determine soft singularities, then:
\begin{equation*}
  \so_{\ur_q} = \so_{\ur_{\bar{q}}} \equiv 0
\end{equation*}
The only soft singularities come from the first term, when a gluon becomes unresolved. In this limit, the factorization of the amplitude is found to be (see \cite{Catani-1997}):
\begin{equation}
  \so_{\ur_g} \flm_{a,b}[\fs{n}{m}(\ur_g)] = - 4\pi \bc \sum_{i,j \in \fs{n}{m}} \eik_{i,j} (\cc) \flm_{a,b}[\fs{n}{m}]
\end{equation}
where the \bctxt{eikonal factor} reads:
\begin{equation}
  \eik_{i,j} \equiv \frac{p_i \cdot p_j}{(p_i \cdot p_\ur) (p_j \cdot p_\ur)}
\end{equation}
To perform the integration on the unresolved phase space, we extract the $ [\dd p_\ur] $ measure from $ \lps_{m+1} $:
\begin{equation*}
  \ips{\so_\ur \Delta^\ur \flm_{a,b}[\fs{n}{m}(\ur_g)]} = - 4\pi \bc \sum_{i,j \in \fs{n}{m}} (\cc) \ips{\int [\dd p_\ur] \, \eik_{i,j} \flm_{a,b}[\fs{n}{m}]}
\end{equation*}
The only factor dependent on $ p_\ur $ is the eikonal factor, hence we can perform the integration explicitly:
\begin{equation*}
  \begin{split}
    \int [\dd p_\ur] \, \eik_{i,j}
    & = \int [\dd p_\ur] \frac{p_i \cdot p_j}{(p_i \cdot p_j) (p_j \cdot p_\ur)} = \int_0^{\ema} \frac{\dd E_\ur}{E_\ur^{1+2\de}} \int_{\mathbb{S}^{2-2\de}} \frac{\dd \Omega_{2-2\de}}{2(2\pi)^{3-2\de}} \frac{\rho_{ij}}{\rho_{i\ur} \rho_{j\ur}} \\
    & = - \frac{\ema^{-2\de}}{2\de} \rho_{ij} \frac{1}{8\pi^2} \frac{(4\pi)^\de}{\Gamma(1-\de)} 2^{-1-2\de} \frac{\Gamma^2(-\de)}{\Gamma(-2\de)} \,\hyp(1,1,1-\de,1-\eta_{ij}) \\
    & = \frac{1}{\de^2} \frac{[\rc]}{4\pi \bc} \left( \frac{2\ema}{\mu} \right)^{-2\de} \frac{\Gamma^2(1-\de)}{\Gamma(1-2\de)} \eta_{ij} \,\hyp(1,1,1-\de,1-\eta_{ij})
  \end{split}
\end{equation*}
where we made use of the short-hands defined in \eref{eq:short-hand} and the angular integral from Appendix G.3 of \cite{Asteriadis-2020}. It is thus possible to write the integrated soft-counterterm as:
\begin{equation}
  \sum_{n \in \en_{m+1}(g)} \ips{\so_\ur \Delta^\ur \flm_{a,b}[\fs{n}{m}(\ur_g)]} = \sum_{n \in \en_m} [\rc] \ips{\iso(\de) \flm_{a,b}[\fs{n}{m}]}
\end{equation}
where the \bctxt{integrated soft operator} is defined as:
\begin{equation}
  \iso(\de) \defeq - \frac{1}{\de^2} \left( \frac{2\ema}{\mu} \right)^{-2\de} \frac{\Gamma^2(1-\de)}{\Gamma(1-2\de)} \sum_{i,j \in \fs{n}{m}} \eta_{ij} (\cc) \,\hyp(1,1,1-\de,1-\eta_{ij})
\end{equation}

\subsection{Collinear limits}

Collinear limits are more delicate to analyze. In particular, we are interested in the extraction of hard-collinear singularities stemming from terms of the form $ \oso_\ur \co_{i\ur} $, which in the case of unresolved quarks coincide with $ \co_{i\ur} $.

\subsubsection{Generalized anomalous dimensions}

The factorization of the amplitude in a collinear limit can be found in \cite{Catani-1997}, and it is depends on whether the unresolved final-state parton becomes collinear to an initial- or final-state parton.

\paragraph{Final-state collinear limit}

Consider the case of a final-state parton $ \ur $ of flavour $ f_\ur $ becoming collinear to another final-state parton $ i $ of flavour $ f_i $. Then, we set:
\begin{equation*}
  \ampl_\ur \ : \
  \begin{tikzpicture}[baseline = (r.base)]
    \begin{feynman}[inline = (r.base)]
      \vertex[blob, minimum size = 0.8cm] (v) {};

      \vertex[left = 1.2cm of v] (r1) {};
      \vertex[above = 0.6cm of r1] (a) {};
      \vertex (la) at ($(a) + (-0.1,0)$) {$ a $};
      \vertex[below = 0.6cm of r1] (b) {};
      \vertex (lb) at ($(b) + (-0.1,0)$) {$ b $};

      \vertex[right = 1.2cm of v] (r2) {};
      \vertex[above = 0.84cm of r2] (c) {};
      \vertex[below = 0.84cm of r2] (d) {};
      \vertex (vd1) at ($ (v) + (0.8,0.3) $) {$ \vdots $};
      \vertex (vd2) at ($ (v) + (0.8,-0.12) $) {$ \vdots $};

      \vertex[right = 1.5cm of v] (r3) {};
      \vertex[above = 1.5cm of r3] (e);

      \vertex[right = 0.7cm of e] (r4) {};
      \vertex[above = 0.4cm of r4] (f) {};
      \vertex (lf) at ($(f) + (0.1,0)$) {\color{red} $ \ur $};
      \vertex[below = 0.4cm of r4] (g) {};
      \vertex (lg) at ($(g) + (0.1,0)$) {\color{red} $ i $};

      \vertex[below = 0.25em of v] (r);

      \diagram* {
        (a) -- (v) -- (c),
        (b) -- (v) -- (d),
        (v) -- [red, edge label = {$ [i\ur] $}, sloped] (e),
        (e) -- [red] (f),
        (e) -- [red] (g),
      };
    \end{feynman}
  \end{tikzpicture}
  \qquad \qquad
  \ampl_0 \ : \
  \begin{tikzpicture}[baseline = (r.base)]
    \begin{feynman}[inline = (r.base)]
      \vertex[blob, minimum size = 0.8cm] (v) {};

      \vertex[left = 1.2cm of v] (r1) {};
      \vertex[above = 0.6cm of r1] (a) {};
      \vertex (la) at ($(a) + (-0.1,0)$) {$ a $};
      \vertex[below = 0.6cm of r1] (b) {};
      \vertex (lb) at ($(b) + (-0.1,0)$) {$ b $};

      \vertex[right = 1.2cm of v] (r2) {};
      \vertex[above = 0.84cm of r2] (c) {};
      \vertex[below = 0.84cm of r2] (d) {};
      \vertex (vd1) at ($ (v) + (0.8,0.3) $) {$ \vdots $};
      \vertex (vd2) at ($ (v) + (0.8,-0.12) $) {$ \vdots $};

      \vertex[right = 1.5cm of v] (r3) {};
      \vertex[above = 1.5cm of r3] (e);
      \vertex (le) at ($(e) + (0.4,0)$) {\color{red} $ [i\ur] $};

      \vertex[below = 0.25em of v] (r);

      \diagram* {
        (a) -- (v) -- (c),
        (b) -- (v) -- (d),
        (v) -- [red] (e),
      };
    \end{feynman}
  \end{tikzpicture}
\end{equation*}
The factorization of the amplitude reads:
\begin{equation}
  \co_{i\ur} \ampl_\ur = - \frac{8\pi \bc}{(p_i - p_\ur)^2} \spl_{f_{[i\ur]}f_i}(z) \ampl_0
  \label{eq:catani-spl}
\end{equation}
where $ \spl_{f_{[i\ur]f_i}f_i}(z) $ is the \bctxt{Altarelli-Parisi splitting function} associated to the splitting process $ [i\ur] \rightarrow i + \ur $ and the $ z $ is the momentum fraction carried by the parton $ i $, i.e.:
\begin{equation}
  z \equiv 1 - \frac{E_\ur}{E_{[i\ur]}}
\end{equation}
The possible splittings are determined by the QCD interaction vertices (see \secref{ssec:qcd-quant}) and are listed in \eeref{eq:spl-1}{eq:spl-4}:
\begin{equation*}
  \begin{tikzpicture}
    \begin{feynman}
      \vertex (a) {};
      \vertex (la) at ($ (a) + (-0.1,0) $) {$ g $};

      \vertex[right = 1.5cm of a, dot] (v) {};
      \vertex[right = 1cm of v] (r1) {};

      \vertex[above = 0.75cm of r1] (b) {};
      \vertex (lb) at ($ (b) + (0.1,0) $) {$ g $};

      \vertex[below = 0.75cm of r1] (c) {};
      \vertex (lc) at ($ (c) + (0.1,0) $) {$ g $};

      \diagram* {
        (a) -- [gluon] (v),
        (v) -- [gluon] (b),
        (v) -- [gluon] (c),
      };
    \end{feynman}
  \end{tikzpicture}
  \qquad
  \begin{tikzpicture}
    \begin{feynman}
      \vertex (a) {};
      \vertex (la) at ($ (a) + (-0.1,0) $) {$ g $};

      \vertex[right = 1.5cm of a, dot] (v) {};
      \vertex[right = 1cm of v] (r1) {};

      \vertex[above = 0.75cm of r1] (b) {};
      \vertex (lb) at ($ (b) + (0.1,0) $) {$ q $};

      \vertex[below = 0.75cm of r1] (c) {};
      \vertex (lc) at ($ (c) + (0.1,0) $) {$ \bar{q} $};

      \diagram* {
        (a) -- [gluon] (v),
        (v) -- [fermion] (b),
        (v) -- [anti fermion] (c),
      };
    \end{feynman}
  \end{tikzpicture}
  \qquad
  \begin{tikzpicture}
    \begin{feynman}
      \vertex (a) {};
      \vertex (la) at ($ (a) + (-0.1,0) $) {$ q $};

      \vertex[right = 1.5cm of a, dot] (v) {};
      \vertex[right = 1cm of v] (r1) {};

      \vertex[above = 0.75cm of r1] (b) {};
      \vertex (lb) at ($ (b) + (0.1,0) $) {$ q $};

      \vertex[below = 0.75cm of r1] (c) {};
      \vertex (lc) at ($ (c) + (0.1,0) $) {$ g $};

      \diagram* {
        (a) -- [fermion] (v),
        (v) -- [fermion] (b),
        (v) -- [gluon] (c),
      };
    \end{feynman}
  \end{tikzpicture}
  \qquad
  \begin{tikzpicture}
    \begin{feynman}
      \vertex (a) {};
      \vertex (la) at ($ (a) + (-0.1,0) $) {$ q $};

      \vertex[right = 1.5cm of a, dot] (v) {};
      \vertex[right = 1cm of v] (r1) {};

      \vertex[above = 0.75cm of r1] (b) {};
      \vertex (lb) at ($ (b) + (0.1,0) $) {$ g $};

      \vertex[below = 0.75cm of r1] (c) {};
      \vertex (lc) at ($ (c) + (0.1,0) $) {$ q $};

      \diagram* {
        (a) -- [fermion] (v),
        (v) -- [gluon] (b),
        (v) -- [fermion] (c),
      };
    \end{feynman}
  \end{tikzpicture}
\end{equation*}
and respective charge-conjugates (splitting functions do not distinguish between quarks and antiquarks).

Using \eref{eq:catani-spl} we can derive a general expression for the integrated final-state collinear counterterm:
\begin{equation*}
  \begin{split}
    & \ips{\co_{i\ur} \Delta^\ur \flm_{a,b}[\fs{n}{m}(\ur)]} = \ips{\int [\dd p_i] [\dd p_\ur] \frac{4\pi \bc}{p_i \cdot p_\ur} z \spl_{f_{[i\ur]} f_i}(z) \flm_{a,b}[\fs{n'}{m}](p_{[i\ur]})} \\
    & \qquad \ = \ips{\int_{(\mathbb{S}^{2-2\de})^2} \frac{\dd \Omega_{2-2\de}^2}{(2(2\pi)^{3-2\de})^2} \int_0^{\ema} \frac{\dd E_i}{E_i^{-1+2\de}} \int_0^{\ema} \frac{\dd E_\ur}{E_\ur^{-1+2\de}} \frac{4\pi \bc}{E_{[i\ur]}^2 \rho_{i\ur}} \frac{\spl_{f_{[i\ur]} f_i}(z)}{1-z} \flm_{a,b}[\fs{n'}{m}](p_{[i\ur]})} \\
    & \qquad \ = - \frac{[\rc]}{\de} \frac{\Gamma^2(1-\de)}{\Gamma(1-2\de)} \ips{\int [\dd p_{[i\ur]}] \left( \frac{2E_{[i\ur]}}{\mu} \right)^{-2\de} \int_{\zm}^1 \dd z\, \frac{\spl_{f_{[i\ur]} f_i}(z)}{z^{-1 + 2\de} (1-z)^{2\de}} \flm_{a,b}[\fs{n'}{m}](p_{[i\ur]})}
  \end{split}
\end{equation*}
Note that we made the dependence of $ \flm_{a,b}[\fs{n'}{m}] $ on $ p_{[i\ur]} = z^{-1} p_i $ explicit, making it clear that the $ \flm $ function vanishes when $ z < 0 $, which is the case for $ z \in [\zm,0) $ as $ \zm \equiv 1 - \ema / E_{[i\ur]} < 0 $. The hard-collinear counterterm reads:
\begin{equation*}
  \begin{split}
    & \ips{\oso_\ur \co_{i\ur} \Delta^\ur \flm_{a,b}[\fs{n}{m}(\ur)]} \\
    & \qquad \qquad = - \frac{[\rc]}{\de} \frac{\Gamma^2(1-\de)}{\Gamma(1-2\de)} \ips{\int [\dd p_{[i\ur]}] \left( \frac{2E_{[i\ur]}}{\mu} \right)^{-2\de} \int_{\zm}^1 \dd z\, \oso_z \frac{\spl_{f_{[i\ur]} f_i}(z)}{z^{-1 + 2\de} (1-z)^{2\de}} \flm_{a,b}[\fs{n'}{m}](p_{[i\ur]})}
  \end{split}
\end{equation*}
where $ \so_z \equiv \lim_{z \rightarrow 1} $. Since $ \so_z $ only extracts the singular part, by \eeref{eq:spl-1}{eq:spl-4} it is clear that:
\begin{equation}
  \so_z \spl_{f_{[i\ur]}f_i}(z) = \frac{2}{1 - z} \tco^2_{f_{[i\ur]}} \delta_{f_{[i\ur]},f_i}
\end{equation}
Then, writing $ [\zm,1] = [\zm,0) \cup [0,1] $, there only remains to evaluate the following integral:
\begin{equation*}
  \begin{split}
    \int_{\zm}^0 \dd z\, \oso_z  \frac{\spl_{f_{[i\ur]}f_i}(z)}{z^{-1+2\de} (1-z)^{2\de}} \flm_{a,b}[\fs{n'}{m}](z^{-1}p_i)
    & = - 2 \tco^2_{f_{[i\ur]}} \delta_{f_{[i\ur]},f_i} \int_0^{\zm} \dd z\, (1 - z)^{-1-2\de} \flm_{a,b}[\fs{n'}{m}](p_i) \\
    & = \tco^2_{f_{[i\ur]}} \delta_{f_{[i\ur]},f_i} \frac{1 - e^{-2 \de \leg_i}}{\de} \flm_{a,b}[\fs{n'}{m}]
  \end{split}
\end{equation*}
where we set $ \leg_i \equiv \log (\ema / E_i) $ and suppressed the depedance of the $ \flm $ function on $ p_i = \so_z p_{[i\ur]} $. Finally, the integrated final-state hard-collinear counterterm can be expressed as:
\begin{equation}
  \ips{\oso_\ur \co_{i\ur} \Delta^\ur \flm_{a,b}[\fs{n}{m}](\ur)} = [\rc] \ips{\frac{\Gamma_{i,f_{[i\ur]} \rightarrow f_i f_\ur}}{\de} \flm_{a,b}[\fs{n'}{m}]}
\end{equation}
where $ n' \in \en_m $ and the \bctxt{generalized final-state anomalous dimension} is defined as:
\begin{equation}
  \Gamma_{i,f_{[i\ur]} \rightarrow f_i f_\ur} \defeq - \left( \frac{2E_i}{\mu} \right)^{-2\de} \frac{\Gamma^2(1-\de)}{\Gamma(1-2\de)} \left[ \int_0^1 \dd z\, \oso_z \frac{\spl_{f_{[i\ur]}f_i}}{z^{-1+2\de} (1-z)^{2\de}} - \delta_{f_{[i\ur]},f_i} \tco^2_{f_{[i\ur]}} \frac{1 - e^{-2\de L_i}}{\de} \right]
\end{equation}

\paragraph{Initial-state collinear limit}

Now, consider a final-state parton $ \ur $ of flavour $ f_\ur $ becoming collinear to an initial-state parton $ a $ of flavour $ f_a $. In this case, we set:
\begin{equation*}
  \ampl_\ur \ : \
  \begin{tikzpicture}[baseline = (r.base)]
    \begin{feynman}[inline = (r.base)]
      \vertex[blob, minimum size = 0.8cm] (v) {};

      \vertex[left = 1.2cm of v] (r1) {};
      \vertex[below = 0.6cm of r1] (b) {};
      \vertex (lb) at ($(b) + (-0.1,0)$) {$ b $};

      \vertex[right = 1.2cm of v] (r2) {};
      \vertex[above = 0.84cm of r2] (c) {};
      \vertex[below = 0.84cm of r2] (d) {};
      \vertex (vd1) at ($ (v) + (0.8,0.3) $) {$ \vdots $};
      \vertex (vd2) at ($ (v) + (0.8,-0.12) $) {$ \vdots $};

      \vertex[below = 0.25em of v] (r);

      \vertex[above = 2cm of b] (a) {};
      \vertex (la) at ($ (a) + (-0.1,0) $) {\color{red} $ a $};
      \vertex[right = 1cm of a] (e);
      \vertex[right = 1cm of e] (r3) {};
      \vertex[above = 0.5cm of r3] (f) {};
      \vertex (lf) at ($ (f) + (0.1,0) $) {\color{red} $ \ur $};

      \diagram* {
        (b) -- (v) -- (d),
        (v) -- (c),
        (a) -- [red] (e) -- [red, edge label = {\color{red} $ [a\ur] $}, sloped] (v),
        (e) -- [red] (f),
      };
    \end{feynman}
  \end{tikzpicture}
  \qquad \qquad
  \ampl_0 \ : \
  \begin{tikzpicture}[baseline = (r.base)]
    \begin{feynman}[inline = (r.base)]
      \vertex[blob, minimum size = 0.8cm] (v) {};

      \vertex[left = 1.2cm of v] (r1) {};
      \vertex[below = 0.6cm of r1] (b) {};
      \vertex (lb) at ($(b) + (-0.1,0)$) {$ b $};

      \vertex[right = 1.2cm of v] (r2) {};
      \vertex[above = 0.84cm of r2] (c) {};
      \vertex[below = 0.84cm of r2] (d) {};
      \vertex (vd1) at ($ (v) + (0.8,0.3) $) {$ \vdots $};
      \vertex (vd2) at ($ (v) + (0.8,-0.12) $) {$ \vdots $};

      \vertex[above = 1.7cm of b] (e);
      \vertex (le) at ($(e) + (-0.4,0)$) {\color{red} $ [a\ur] $};

      \vertex[below = 0.25em of v] (r);

      \diagram* {
        (e) -- [red] (v) -- (c),
        (b) -- (v) -- (d),
      };
    \end{feynman}
  \end{tikzpicture}
\end{equation*}
The factorization of the amplitude reads:
\begin{equation}
\co_{a\ur} \ampl_m = - \frac{8\pi \bc}{(p_a - p_\ur)^2} \frac{1}{z} P_{f_{[a\ur]}f_a , \text{i}}(z) \ampl_0
\end{equation}
where the Altarelli-Parisi initial-state splitting functions are defined in \eeref{eq:spli-1}{eq:spli-4} and depend on the energy fraction:
\begin{equation}
  z \equiv 1 - \frac{E_\ur}{E_a}
\end{equation}
associated to the splitting process $ a \rightarrow [a\ur] + \ur $. The integrated initial-state collinear counterterm is found analogously to the final-state one:
\begin{equation*}
  \begin{split}
    & \ips{\co_{a\ur} \Delta^\ur \flm_{a,b}[\fs{n}{m}(\ur)]} = - \frac{[\rc]}{\de} \frac{\Gamma^2(1-\de)}{\Gamma(1-2\de)} \ips{\left( \frac{2E_{[a\ur]}}{\mu} \right)^{-2\de} \int_{\zm}^1 \dd z\, \frac{\spl_{f_{[a\ur]} f_a , \text{i}}(z)}{z (1-z)^{2\de}} \flm_{[a\ur],b}[\fs{n'}{m}](p_{[a\ur]})}
  \end{split}
\end{equation*}
where $ p_{[a\ur]} = z p_a $. As in Appendix A.3 of \cite{rontsch-2503}, we define new splitting functions:
\begin{equation}
  \nspl_{f_{[a\ur]}f_a}(z, E_a) \defeq \oso_z \frac{P_{f_{[a\ur]}f_a , \text{i}}(z)}{(1-z)^{2\de}} - \delta_{f_{[a\ur]}f_a} \tco_{f_{[a\ur]}}^2 \frac{1 - e^{-2\de L_a}}{\de} \delta(1-z)
\end{equation}
which can be conveniently rewritten as:
\begin{equation}
  - \nspl_{\alpha \beta}(z, E_a) = \delta_{\alpha \beta} \delta(1-z) \left[ \gamma_\alpha + 2 \tco_\alpha^2 \frac{1 - e^{-2\de L_a}}{\de} \right] + \left[ - \ap_{\alpha \beta}(z) + \de \fspl_{\alpha \beta}(z) \right]
\end{equation}
where $ \gamma_\alpha $ is the anomalous dimension of the partonic flavour $ \alpha $ and $ \fspl_{\alpha \beta} $ is the $ \de $-expansion of $ - \nspl(z, 0) $ starting at $ \smo(\de) $. This relation allows us to get an expllicit expression for the integrated initial-state hard-collinear counterterm:
\begin{equation}
  \ips{\oso_\ur \co_{a\ur} \Delta^\ur \flm_{a,b}[\fs{n}{m}(\ur)]} = \frac{[\rc]}{\de} \ips{\Gamma_{a,f_a} \flm_{[a\ur],b}[\fs{n'}{m}]} + \frac{\rc}{\de} \ips{\gspl_{f_{[a\ur]}f_a} \otimes \flm_{f_{[a\ur]},b}[\fs{n'}{m}]}
\end{equation}
where $ n' \in \en_m([a\ur],b,X) $ and the \bctxt{generalized initial-state anomalous dimension} and the generalized splitting functions are defined as:
\begin{equation}
  \Gamma_{a,\alpha} \defeq \left( \frac{2E_a}{\mu} \right)^{-2\de} \frac{\Gamma^2(1-\de)}{\Gamma(1-2\de)} \left[ \gamma_\alpha + \tco^2_\alpha \frac{1 - e^{-2\de L_a}}{\de} \right]
\end{equation}
\begin{equation}
  \gspl_{\alpha \beta}(z) \defeq \left( \frac{2E_a}{\mu} \right)^{-2\de} \frac{\Gamma^2(1-\de)}{\Gamma(1-2\de)} \left[ - \ap_{\alpha \beta}(z) + \de \fspl_{\alpha \beta}(z) \right]
\end{equation}
and the Mellin convolution is defined as:
\begin{equation}
  \fspl_{f_{[a\ur]}f_a} \otimes \flm_{[a\ur],b}[\fs{n'}{m}](p_a) \equiv \int_0^1 \frac{\dd z}{z}\, \fspl_{f_{[a\ur]}f_a}(z) \flm_{[a\ur],b}[\fs{n'}{m}](z p_a)
\end{equation}

\subsubsection{Hard-collinear limits}

The only difficulty left is how to combine the generalized anomalous dimensions arising from the various terms in \eref{eq:part-chan}.

\paragraph{Unresolved gluon}

\paragraph{Unresolved quark}

\paragraph{Unresolved antiquark}

\subsection{Real operators}

\section{Non-real corrections}

\subsection{Virtual corrections}

\subsection{Collinear renormalization}
\label{ssec:coll-ren}

\section{Integrated counterterms}












\chapter{NSC SS with Massive Quarks}
\selectlanguage{english}

The NSC SS, as developed in \cite{rontsch-2023, rontsch-2503, rontsch-2509} and described in the previous chapter, does not include massive final-state partons. In this chapter, we consider the inclusion of these massive partons. We prove the cancellation of poles in this case, i.e. with $ \fsqm{n}{m} \neq \emptyset , \fsbqm{n}{m} \neq \emptyset $, and we derive the integrated counterterms in an explicit form.

The general structure of the NSC SS as illustrated in the previous chapter remains valid: in particular, massive partons do not change collinear singularities (as shown in \secref{ssec:ir-poles}), hence the $ \ico(\de) $ operator, as well as the Mellin convolutions of generalized splitting functions, remain unchanged. The only operators that do change are $ \iso(\de) $ and $ \ivo(\de) $.

\section{Generalized operators}

\subsection{Generalized soft operator}

The factorization formula for amplitudes in the soft limit, given in \eref{eq:soft-lim}, holds in general, thus to generalize the soft operator we only have to compute integrated eikonal factors in the case of massive partons. To do so, we choose a different parametrization of the phase space of the parton $ \ur_g $, with $ p = (p_1 , p_2 , p_\perp) $:
\begin{equation}
  \dd^{d-1} p = \dd p_1 \, \dd p_2 \, \dd^{d-3} p_\perp = \dd p_1 \, \dd p_2 \, \dd p_\perp \, p_\perp^{d-4} \dd \Omega_{d-4}
\end{equation}
Then, we perform a transformation to polar coordinates $ (p_1 , p_2 , p_\perp) \mapsto (p_0 , \vartheta , \varphi) $ such that:
\begin{equation}
  p_1 = p_0 \cos \vartheta
  \qquad \qquad
  p_2 = p_0 \sin \vartheta \cos \varphi
  \qquad \qquad
  p_\perp = p_0 \sin \vartheta \sin \varphi
\end{equation}
The condition $ p_\perp \in \R^+_0 $ implies $ \varphi \in [0,\pi] $, while $ \vartheta \in [0,\pi] $ by definition. As $ p_0 = E $ for gluons, the phase-space measure becomes:
\begin{equation}
  [\dd p] \equiv \frac{\dd^{3-2\de}p}{(2\pi)^{3-2\de}2E} \theta(\ema - E) = \frac{\dd \Omega_{-2\de}}{2(2\pi)^{3-2\de}} \dd E\, E^{1-2\de} \theta(\ema - E) \dd \cos \vartheta \dd \varphi \left( \sin \vartheta \sin \varphi \right)^{-2\de}
\end{equation}
Now we can perform the integration of the eikonal factor $ \eik_{i,j} $:
\begin{equation*}
  \begin{split}
    \int [\dd p_\ur] \eik_{i,j}
    & = \int_{\mathbb{S}^{-2\de}} \frac{\dd \Omega_{-2\de}}{2(2\pi)^{3-2\de}} \int_0^{\ema} \frac{\dd E_\ur}{E_\ur^{1-2\de}} \int_{-1}^1 \dd \cos \vartheta \int_0^\pi \dd \varphi \left( \sin \vartheta \sin \varphi \right)^{-2\de} \frac{\mm{p}_i \cdot \mm{p}_j}{(\mm{p}_i \cdot \mm{p}_\ur) (\mm{p}_j \cdot \mm{p}_\ur)} \\
    & = - \frac{2^{1-2\de} \pi^{-\de}}{2(2\pi)^{3-2\de}} \frac{\Gamma(1-\de)}{\Gamma(1-2\de)} \frac{\ema^{-2\de}}{2\de} \pi \eit_{i,j} = - \frac{1}{\de} \frac{\pi^\de}{16\pi^2} \frac{\Gamma(1-\de)}{\Gamma(1-2\de)} \ema^{-2\de} \eit_{i,j}(\de)
  \end{split}
\end{equation*}
where we used the following identity:
\begin{equation}
  \int_{\mathbb{S}^{d-1}} \dd \Omega_{d-1} = \frac{2 \pi^{\frac{d}{2}}}{\Gamma(\frac{d}{2})} = 2^d \pi^{\frac{d}{2} - \frac{1}{2}} \frac{\Gamma\left( \frac{d+1}{2} \right)}{\Gamma(d)}
\end{equation}
and defined the generalized angular integral:
\begin{equation}
  \eit_{i,j} \equiv \frac{1}{\pi} \int_{-1}^1 \dd \cos \vartheta \int_0^\pi \dd \varphi \left( \sin \vartheta \sin \varphi \right)^{-2\de} \frac{\mm{p}_i \cdot \mm{p}_j}{(\mm{p}_i \cdot \mm{p}_\ur) (\mm{p}_j \cdot \mm{p}_\ur)}
  \label{eq:gen-int}
\end{equation}
with $ \mm{p}_k : p_k^\mu = E_k \mm{p}_k^\mu \,\,\forall k \in \fsp{n}{m}\cup\{\ur\} $. It only remains to evaluate $ \eit_{i,j}(\de) $ in the three possible cases for $ i $ and $ j $: massive--massive, massive--massless (note that $ \eit_{i,j}(\de) = \eit_{j,i}(\de) $) and massless--massless. The former two cases are known in the literature, e.g. Appendix D.2 of \cite{Behring-2020}, and are reported in \secref{sec:mass-int}.

To determine $ \eit_{i,j}(\de) $ in the massless--massless case, we just have to compare the expression for the integrated eikonal factor computed in \secref{ssec:soft-massless} with the above expression:
\begin{equation*}
  \frac{\ema^{-2\de}}{\de^2} \eta_{ij} \frac{\pi^\de}{8\pi^2} \frac{\Gamma(1-\de)}{\Gamma(1-2\de)} \,\hyp(1,1,1-\de,1-\eta_{ij}) = - \frac{1}{\de} \frac{\pi^\de}{16\pi^2} \frac{\Gamma(1-\de)}{\Gamma(1-2\de)} \ema^{-2\de} \eit_{i,j}(\de)
\end{equation*}
We can now make use of the following expansion:
\begin{equation}
  \eta \,\hyp(1,1,1-\de,1-\eta) = 1 - \de \log \abs{\eta} - \frac{1}{2} \left( \log^2 \abs{\eta} + 2 \li(1-\eta) \right) \de^2 + \smo(\de^3)
\end{equation}
Hence, with the same notation from \secref{sec:mass-int}, we can write:
\begin{equation}
  \eitt{i,j}{-1} = - 2
  \qquad
  \eitt{i,j}{0} = 2 \log \abs{\eta_{ij}}
  \qquad
  \eitt{i,j}{1} = \log^2 \abs{\eta_{ij}} + 2 \li(1 - \eta_{ij})
  \label{eq:i-nmnm}
\end{equation}
Inserting everything in \eref{eq:soft-lim}, we get:
\begin{equation}
  \sum_{n \in \en_{m+1}(g)} \ips{\so_\ur \Delta^\ur \flm_{a,b}[\fs{n}{m}(\ur_g)]} = [\rc] \ips{\iso(\de) \flm_{a,b}[\fs{n}{m}]}
\end{equation}
with the generalized soft operator:
\begin{equation}
  \iso(\de) \defeq \frac{1}{2\de} \left( \frac{2\ema}{\mu} \right)^{-2\de} \frac{\Gamma^2(1-\de)}{\Gamma(1-2\de)} \sum_{i,j \in \fsp{n}{m}} (\cc) \eit_{i,j}(\de)
  \label{eq:iso-gen}
\end{equation}

\subsection{Generalized virtual operator}

Recall that IR poles of the 1-loop amplitude for a process involving massless partons can be expressed in terms of Catani's operator $ \ioo(\de) $ (see \eref{eq:ioo}): a generalization of this operator in the case of massive partons is given in \cite{Catani-2001}. However, as already stated, this operator is written in a charge-unrenormalized way, hence we have to perform the renormalization of the results of this reference. In the following, we will denote charge-unrenormalized amplitudes as $ \mat $ and charge-renormalized amplitudes as $ \ampl $.

Loop corrections to the LO amplitude can be written as:
\begin{equation}
  \mat_m = \left( \frac{\bc}{4\pi \mu^{2\de}} \right)^q \left[ \mat_m^{(0)} + \frac{\bc}{4\pi \mu^{2\de}} \mat^{(1)} + \smo(\bc^2) \right]
  \label{eq:ampl-non-ren}
\end{equation}
where $ q \in \frac{1}{2} \N_0 $. Note that this equation fixes the normalization of the charge-unrenormalized amplitudes. Then, the IR poles can be written as:
\begin{equation}
  \mat_m^{(1)} = \irs(\de) \mat_m^{(0)} + \mat_m^\text{fin}
  \label{eq:gen-cat-form}
\end{equation}
where $ \mat_m^\text{fin} $ is the non-singular part of the 1-loop amplitude and Catani's operator\footnotemark is define as (in CDR):
\begin{equation}
  \irs(\de) \defeq \frac{(4\pi)^\de}{\Gamma(1-\de)} \left[ q \frac{\beta_0}{\de} + \ioo(\de) \right]
\end{equation}
\begin{equation}
  \ioo(\de) \defeq \sum_{i \neq j \in \fsp{n}{m}} (\cc) \left( \frac{\mu^2}{\abs{s_{ij}}} \right)^\de \left[ \ns_{i,j}(\de) + \frac{1}{v_{ij}} \left( \frac{i\pi}{\de} - \frac{\pi^2}{2} \right) \theta(s_{ij}) \right] - \sum_{i \in \fsp{n}{m}} \Gamma_i(\de)
  \label{eq:gen-cat}
\end{equation}
with $ v_{ij} $ defined in \eref{eq:v-def}, $ s_{ij} \equiv 2 p_i \cdot p_j $ and the singular functions defined in \secref{sec:cat-fun}. Inserting \eeref{eq:gen-cat-form}{eq:gen-cat} into \eref{eq:ampl-non-ren}:
\begin{multline*}
  \mat_m = \left( \frac{\bc}{4\pi \mu^{2\de}} \right)^q \bigg[ \left( 1 + \frac{(4\pi)^\de}{\Gamma(1-\de)} \frac{\bc}{4\pi \mu^{2\de}} q \frac{\beta_0}{\de} \right) \mat_m^{(0)} \\
  + \frac{\bc}{4\pi \mu^{2\de}} \left( \frac{(4\pi)^\de}{\Gamma(1-\de)} \ioo(\de) \mat_m^{(0)} + \mat_m^\text{fin} \right) + \smo(\bc^2) \bigg]
\end{multline*}
Renormalization in the $ \msb $ scheme is performed via \eref{eq:msb-def}. Substituting this to $ \bc $ yields:
\begin{equation*}
  \begin{split}
    \mat_m
    & = \left( \frac{\rc}{4\pi S_\de} \right)^q \left[ 1 - \frac{1}{2} \frac{\rc}{2\pi} \frac{\beta_0}{\de} + \smo(\rc^2) \right]^q \times \\
    & \qquad \quad \times \left[ \left( 1 + \frac{(4\pi)^\de}{\Gamma(1-\de)} \frac{\rc}{4\pi S_\de} q \frac{\beta_0}{\de} \right) \mat_m^{(0)} + \frac{\rc}{4\pi S_\de} \left( \frac{(4\pi)^\de}{\Gamma(1-\de)} \ioo(\de) \mat_m^{(0)} + \mat_m^\text{fin} \right) + \smo(\bc^2) \right] \\
    & = \left( \frac{\rc}{4\pi S_\de} \right)^q \bigg[ \left( 1 - \frac{\rc}{4\pi} q \frac{\beta_0}{\de} \left( 1 - \frac{e^{\eg\de}}{\Gamma(1-\de)} \right) \right) \mat_m^{(0)} + \\
    & \qquad \qquad \qquad \qquad \qquad \qquad \qquad \qquad \qquad + \frac{\rc}{4\pi} \frac{e^{\eg\de}}{\Gamma(1-\de)} \ioo(\de) \mat_m^{(0)} + \frac{\rc}{4\pi S_\de} \mat_m^\text{fin} + \smo(\rc^2) \bigg] \\
    & = \left( \frac{\rc}{4\pi S_\de} \right)^q \left[ \mat_m^{(0)} + \frac{\rc}{4\pi} \frac{e^{\eg\de}}{\Gamma(1-\de)} \ioo(\de) \mat_m^{(0)} + \frac{\rc}{4\pi S_\de} \mat_m^\text{fin} + \smo(\de) + \smo(\rc^2) \right]
  \end{split}
\end{equation*}
Hence, we can write the renormalized relation:
\begin{equation}
  \ampl_m = \left( \frac{\rc}{2\pi} \right)^q \left[ \ampl_m^{(0)} + \frac{\rc}{2\pi} \ampl_m^{(1)} + \smo(\rc^2) \right]
\end{equation}
with:
\begin{equation}
  \ampl_m \equiv \mat_m
  \qquad \qquad
  \ampl_m^{(0)} \equiv (2S_\de)^{-q} \mat_m^{(0)}
  \qquad \qquad
  \ampl_m^\text{fin} \equiv (2S_\de)^{-(1+q)} \mat_m^\text{fin}
\end{equation}
\begin{equation}
  \ampl_m^{(1)} = \frac{1}{2} \frac{e^{\eg\de}}{\Gamma(1-\de)} \ioo(\de) \ampl_m^{(0)} + \ampl_m^\text{fin}
\end{equation}
It is then clear, from \eref{eq:virt-cs}, that:
\begin{equation}
  2\hat{s}\, \dd\pcs_{a,b}^\text{V} = [\rc] \sum_{n \in \en_m} \ips{\ivo(\de) \flm_{a,b}[\fs{n}{m}]} + \sum_{n \in \en_m} \ips{\flm_{a,b}^\text{fin}[\fs{n}{m}]}
\end{equation}
where the generalized virtual operator is:
\begin{equation}
  \ivo(\de) \defeq \Re \ioo(\de) = \sum_{i \neq j \in \fsp{n}{m}} (\cc) \left( \frac{\mu^2}{\abs{s_{ij}}} \right)^\de \left[ \ns_{i,j}(\de) - \frac{1}{v_{ij}} \frac{\pi^2}{2} \theta(s_{ij}) \right] - \sum_{i \in \fsp{n}{m}} \Gamma_i(\de)
  \label{eq:ivo-gen}
\end{equation}

\footnotetext{Note that the notation is slightly different from \cite{Catani-2001}.}

\section{Integrated counterterms}
\label{sec:int-count}

The generalized expressions for the soft and virtual operators with massive partons derived in the previous section can be combined with the unchanged collinear operator to form $ \ito \equiv \iso + \ivo + \ico $. It now only remains to show that $ \ito(\de) $ is $ \de $-finite and find its $ \smo(\de^0) $ term $ \ito^{(0)} $. These constitute the main results of this work. First, setting $ \isvo \equiv \iso + \ivo $, we show the cancellation of colour-correlated terms between the soft and virtual operators (\ceref{eq:iso-gen}{eq:ivo-gen}):
\begin{equation*}
  \begin{split}
    \isvo(\de)
    & = \sum_{i \neq j \in \fsp{n}{m}} (\cc) \bigg[ \frac{1}{2\de} \left( 1 - 2\lmx \de + \left( 2\lmx^2 - \frac{\pi^2}{6} \right) \de^2 + \smo(\de^3) \right) \eit_{i,j}(\de) \,+ \\
    & \quad + \left( 1 + \de \log \frac{\mu^2}{\abs{s_{ij}}} + \frac{1}{2} \de^2 \log^2 \frac{\mu^2}{\abs{s_{ij}}} + \smo(\de^3) \right) \left( \ns_{ij}(\de) - \frac{1}{v_{ij}} \frac{\pi^2}{2} \theta(s_{ij}) \right) \bigg] + \\
    & \quad + \sum_{i \in \fsp{n}{m}} \bigg[ \tco_i^2 \frac{1}{2\de} \left( 1 - 2\lmx \de + \smo(\de^2) \right) \eit_{i,i}(\de) - \Gamma_i(\de) \bigg] \\
    & = \sum_{i \neq j \in \fsp{n}{m}} (\cc) \bigg[ \left( \frac{1}{2\de} - \lmx + \left( \lmx^2 - \frac{\pi^2}{12} \right) \de \right) \left( \frac{\eitt{i,j}{-1}}{\de} + \eitt{i,j}{0} + \de \eitt{i,j}{1} \right) + \smo(\de) \\
    & \quad + \left( 1 + \de \log \frac{\mu^2}{\abs{s_{ij}}} + \frac{1}{2} \de^2 \log^2 \frac{\mu^2}{\abs{s_{ij}}} \right) \left( \frac{\nss{i,j}{-2}}{\de^2} + \frac{\nss{i,j}{-1}}{\de} + \nss{i,j}{0} - \frac{1}{v_{ij}} \frac{\pi^2}{2} \theta(s_{ij}) \right) + \smo(\de) \bigg] + \\
    & + \sum_{i \in \fsm{n}{m}} \left[ \tco_i^2 \left( \frac{1}{2\de} - \lmx \right) \left( \eitt{i,i}{0} + \de \eitt{i,i}{1} \right) - \Gamma_i(\de) \right] - \sum_{i \in \prt{n}{m}} \Gamma_i(\de) + \smo(\de^2)
  \end{split}
\end{equation*}
where we set $ \fsm{n}{m} \equiv \fsqm{n}{m} \cup \fsbqm{n}{m} $. To better understand the calculation, we consider each summation separately. The colour-correlated term can be written as:
\begin{multline*}
  \underbrace{\frac{1}{\de^2} \left( \frac{1}{2} \eitt{i,j}{-1} + \nss{i,j}{-2} \right) + \frac{1}{\de} \left( \frac{1}{2} \eitt{i,j}{0} + \nss{i,j}{-1} - \lmx \eitt{i,j}{-1} + \nss{i,j}{-2} \log \frac{\mu^2}{\abs{s_{ij}}} \right)}_{\ci{i,j}} + \,\smo(\de) \,+ \\
  + \underbrace{\frac{1}{2} \eitt{i,j}{1} - \lmx \eitt{i,j}{0} + \left( \lmx^2 - \frac{\pi^2}{12} \right) \eitt{i,j}{-1} + \nss{i,j}{0} - \frac{1}{v_{ij}} \frac{\pi^2}{2} \theta(s_{ij}) + \nss{i,j}{-1} \log \frac{\mu^2}{\abs{s_{ij}}} + \frac{1}{2} \nss{i,j}{-2} \log^2 \frac{\mu^2}{\abs{s_{ij}}}}_{\finc{i,j}}
\end{multline*}
where $ \ci{i,j} $ contains the $ \de $-poles and $ \finc{i,j} $ is $ \de $-finite (more precisely, $ \finc{i,j} \sim \smo(\de^0) $). The explicit expression of $ \ci{i,j} $ is derived in \secref{sec:poles}: in particular, it has $ \de $-poles only if $ i $ and/or $ j $ are massless partons, while it vanishes otherwise. The sum on $ i \in \fsm{n}{m} $ instead reads, using \ceref{eq:i-self}{eq:g-qm}:
\begin{equation*}
  \sum_{i \in \fsm{n}{m}} \left[ \caf \left( \frac{1}{\de} - \frac{1}{\kappa_i} \log \frac{1 - \kappa_i}{1 + \kappa_i} - 2 \lmx \right) - \caf \left( \frac{1}{\de} + \frac{1}{2} \log \frac{m_{\mq_i}^2}{\mu^2} - 2 \right) \right] \equiv \sum_{i \in \fsm{n}{m}} \finm{i}
\end{equation*}
which is manifestly $ \de $-finite. Finally, from \eeref{eq:g-g}{eq:g-q}, the last summation expands as:
\begin{equation*}
  \sum_{i \in \prt{n}{m}} \Gamma_i(\de) = \sum_{i \in \prt{n}{m}} \left( \frac{1}{\de} \gamma_i + \fing{i} \right)
\end{equation*}
where we have defined the $ \smo(\de^0) $ term as:
\begin{equation}
  \fing{g} \equiv - \frac{2}{3} \ttr \sum_{\rho = 1}^{n_F} \log \frac{m_{\mq_\rho}^2}{\mu^2}
  \qquad \qquad
  \fing{q} \equiv 0
\end{equation}
We can put all the above expressions together, yielding:
\begin{equation}
    \isvo(\de) = \sum_{i \neq j \in \fsp{n}{m}} (\cc) \left[ \ci{i,j} + \finc{i,j} \right] + \sum_{i \in \fsm{n}{m}} \finm{i} - \sum_{i \in \prt{n}{m}} \left( \frac{1}{\de} \gamma_i + \fing{i} \right)
\end{equation}
Note that this equation appears to have colour-correlated poles, contained in $ \ci{i,j} $. We now prove that these in fact cancel, as was the case for massless amplitudes. Recalling that $ \fsp{n}{m} $ is the set of all partons, $ \prt{n}{m} $ the set of all massless partons and $ \fsm{n}{m} $ the set of all final-state massive partons, writing $ \fsp{n}{m} \times \fsp{n}{m} = (\prt{n}{m} \cup \fsm{n}{m}) \times (\prt{n}{m} \cup \fsm{n}{m}) $, we find:
\begin{equation*}
  \sum_{i \neq j \in \fsp{n}{m}} (\cc) \ci{i,j} = \bigg[ \sum_{i \neq j \in \prt{n}{m}} + \sum_{i \neq j \in \fsm{n}{m}} + \sum_{i \in \prt{n}{m} \,,\, j \in \fsm{n}{m}} + \sum_{i \in \fsm{n}{m} \,,\, j \in \prt{n}{m}} \bigg] (\cc) \ci{i,j}
\end{equation*}
From \secref{sec:poles}, $ \ci{i,j} $ contributes with a $ \leg_i / \de $ term for each massless index, hence:
\begin{equation*}
  \begin{split}
    \sum_{i \neq j \in \fsp{n}{m}} (\cc) \ci{i,j}
    & = \sum_{i \neq j \in \prt{n}{m}} (\cc) \frac{\leg_i + \leg_j}{\de} \,+ \\
    & \qquad \qquad \qquad \qquad + \sum_{i \in \prt{n}{m} \,,\, j \in \fsm{n}{m}} (\cc) \frac{\leg_i}{\de} + \sum_{i \in \fsm{n}{m} \,,\, j \in \prt{n}{m}} (\cc) \frac{\leg_j}{\de} \\
    & = \sum_{i , j \in \prt{n}{m}} (\cc)\frac{\leg_i + \leg_j}{\de} - \sum_{i \in \prt{n}{m}} \tco_i^2 \frac{2\leg_i}{\de} \,+ \\
    & \qquad \qquad \qquad \qquad + \sum_{i \in \prt{n}{m} \,,\, j \in \fsm{n}{m}} (\cc) \frac{\leg_i}{\de} + \sum_{i \in \fsm{n}{m} \,,\, j \in \prt{n}{m}} (\cc) \frac{\leg_j}{\de} \\
    & = \sum_{i \in \prt{n}{m}} \frac{\leg_i}{\de} \tco_i \cdot \sum_{j \in \fsp{n}{m}} \tco_j + \sum_{i \in \fsp{n}{m}} \tco_i \cdot \sum_{j \in \prt{n}{m}} \tco_j \frac{\leg_j}{\de} - \sum_{i \in \prt{n}{m}} \tco_i^2 \frac{2\leg_i}{\de} \\
    & = - \sum_{i \in \prt{n}{m}} \tco_i^2 \frac{2\leg_i}{\de}
  \end{split}
\end{equation*}
where we used the colour-conservation condition \eref{eq:col-cons}. Finally, we find an expression for the sum of soft and virtual operators without colour-correlated poles:
\begin{equation}
  \isvo(\de) = - \frac{1}{\de} \sum_{i \in \prt{n}{m}} \left( \gamma_i + 2 \tco_i^2 \leg_i \right) + \isvo^{(0)} + \smo(\de)
  \label{eq:isvo}
\end{equation}
where the $ \smo(\de^0) $ term reads:
\begin{equation}
  \isvo^{(0)} \equiv \sum_{i \neq j \in \fsp{n}{m}} (\cc) \finc{i,j} + \sum_{i \in \fsm{n}{m}} \finm{i} - \sum_{i \in \prt{n}{m}} \fing{i}
  \label{eq:isvo-zero}
\end{equation}
The poles in \eref{eq:isvo} clearly cancel those in the collinear operator \eref{eq:ic-exp}, as it remains unchanged when including massive final-state partons. Thus, we confirm that the cancellation of poles holds in the extension of the NSC SS including massive final states in NLO corrections. We now only need to determine its Laurent series at $ \smo(\de) $. First, consider the generalized initial-state anomalous dimension \eref{eq:in-gen-an-dim}:
\begin{equation}
  \Gamma_{a , f_a} = \gamma_a + 2 \tco_a^2 \leg_a - \de \left[ 2 \ler_a \left( \gamma_a + 2 \tco_a^2 \leg_a \right) + 2 \tco_a^2 \leg_a^2 \right] + \smo(\de^2)
\end{equation}
Then, the generalized final-state anomalous dimension \ceref{eq:fin-gen-an-dim-pd}{eq:fin-gen-an-dim}:
\begin{equation}
  \Gamma_{i , q} = \gamma_q + 2 \caf \leg_i - \de \left[ 2 \ler_i \left( \gamma_q + 2 \caf \leg_i \right) + 2 \caf \leg_i^2 - \frac{\caf}{6} \left( 39 - 4\pi^2 \right) \right] + \smo(\de^2)
\end{equation}
\begin{equation}
  \Gamma_{i , g} = \gamma_g + 2 \caa \leg_i - \de \left[ 2 \ler_i \left( \gamma_g + 2 \caa \leg_i \right) + 2 \caa \leg_i^2 - \frac{\caa}{9} \left( 67 - 6\pi^2 \right) + \frac{23}{9} \ttr n_f \right] + \smo(\de^2)
\end{equation}
Therefore, the Laurent series of the collinear operator is:
\begin{equation}
  \ico(\de) = \frac{1}{\de} \sum_{i \in \prt{n}{m}} \left( \gamma_i + 2 \tco_i^2 \leg_i \right) + \ico^{(0)}
\end{equation}
where the $ \smo(\de^0) $ term reads:
\begin{multline}
  \ico^{(0)} = - \sum_{i \in \prt{n}{m}} \left[ 2 \ler_i \left( \gamma_i + 2 \tco_i^2 \leg_i \right) + 2 \tco_i^2 \leg_i^2 \right] + \\
  + \left( N_q + N_{\bar{q}} \right) \frac{\caf}{6} \left( 39 - 4\pi^2 \right) + N_g \left[ \frac{\caa}{9} \left( 67 - 6\pi^2 \right) - \frac{23}{9} \ttr n_f \right]
  \label{eq:ico-zero}
\end{multline}
where $ N_g \equiv \abs{\fsg{n}{m}} $ is the number of final-state gluons, $ N_q \equiv \abs{\fsq{n}{m}} $ the number of final-state massless quarks and $ N_{\bar{q}} \equiv \abs{\fsbq{n}{m}} $ the number of final-state massless antiquarks.

We see that, when including massive final-state partons, the general structure of the NLO correction \eref{eq:nlo-finite-massless} remains unchanged. Aside from the amplitudes themselves, the only term that does change, indeed, is the $ \ito^{(0)} $ term, whose expression is obtained combining \ceref{eq:isvo-zero}{eq:ico-zero}.

This result paves the way for the inclusion of massive final-state partons in NLO calculations performed with the NSC subtraction framework.












\clearpage

\bookmarksetupnext{level = -1}
\begin{appendices}
\pagestyle{append}

\chapter{Mathematical reference}
\selectlanguage{english}

\section{Phase-space parametrization}
\label{sec:ph-sp-p}

In dimensional regularization with $ d = 4 - 2 \de $, we define the measure on the phase space of a parton $ i $ to be:
\begin{equation}
  [\dd p_i] \equiv \frac{\dd^{d-1} p_i}{(2\pi)^{d-1} 2E_i} \theta(\ema - E_i)
\end{equation}
Note that $ \ema $ is an upper bound on the energies of individual partons: it is an arbitrary parameter to be taken sufficiently large as to be greater or equal to the maximal energy that a final-state parton can reach.

This measure can be cast in a more useful form introducing a suitable parametrization of the phase space: in particular, given that $ \R^n-\{\ve{0}\} \cong \R^+ \times \mathbb{S}^{n-1} $, it is convenient to introduce hyperspherical coordinates on the $ \mathbb{S}^{d-2} $ component of the phase space. In general, the \bctxt{hyperspherical measure} on $ \mathbb{S}^n $ is recursively defined as:
\begin{equation}
  \dd \Omega_n = \sin^{n-1} \varphi \, \dd \varphi \, \dd \Omega_{n-1}
  \label{eq:hyp-rec}
\end{equation}
Using \eref{eq:hyp-rec} (with $ \sin \varphi \, \dd \varphi = \dd \cos \varphi $), we can express the measure $ \dd^{d-1} p_i $ as:
\begin{equation}
  \dd^{d-1} p_i = \abs{\ve{p}_i}^{d-2} \, \dd \abs{\ve{p}_i} \, \sin^{d-4} \varphi \, \dd \cos \varphi \, \dd \Omega_{d-3}
\end{equation}
As we are only interested in integrations on phase spaces of real unresolved partons, which can only be massless, we can use the on-shell condition $ p_i^2 = 0 $ to express $ \abs{\ve{p}_i} = E_i $, so that the phase-space measure becomes:
\begin{equation}
  [\dd p_i] = \theta(\ema - E_i) E_i^{d-3} \dd E_i \, \sin^{d-4} \varphi \, \dd \cos \varphi \, \frac{\dd \Omega_{d-3}}{2(2\pi)^{d-1}}
\end{equation}
with $ E_i \in \R^+ $ and $ \varphi \in [0,\pi] $.

\subsection{Multi-particle phase space}

When considering scattering processes, in general the final state is a multi-particle state, hence the measure on the final-state phase space must account for energy conservation too.

Given a $ 2 \rightarrow m $ scattering process with well-defined initial momenta $ p_\mathcal{A} $ and $ p_\mathcal{B} $, then the differential cross-section is (see Chapter 4 of \cite{Peskin-1995}):
\begin{equation}
  \dd\sigma = \frac{1}{2E_a 2E_b \abs{\ve{v}_a - \ve{v}_b}} \prod_{k = 1}^m \int \frac{\dd^3p_k}{(2\pi)^3 2E_k} \abs{\ampl(ab \rightarrow \mathcal{H})}^2 (2\pi)^4 \delta^{(4)}(p_a + p_b - \textstyle\sum_{i = 1}^m p_i)
  \label{eq:scatt-cr-sec}
\end{equation}
where $ \ampl(ab \rightarrow \mathcal{H}) $ is the amplitude of the scattering process and $ \ve{v}_k \equiv \frac{\ve{p}_k}{E_k} $ is the velocity of the $ k^\text{th} $ particle.

As we are only interested in massless initial-state partons, in the center-of-mass (CM) frame $ p_{a,b} = (E, \pm\ve{p}) $, hence it is trivial to see that the flux factor in \eref{eq:scatt-cr-sec} is just $ 2\hat{s} \defeq 2 (p_a + p_b)^2 $. The differential cross-section can then be rewritten as:
\begin{equation}
  \dd \sigma = \frac{1}{2\hat{s}} \int \lps_m (2\pi)^4 \delta^{(4)}(p_a + p_b - \sum_{i = 1}^m p_i) \abs{\mat(ab \rightarrow \mathcal{H})}^2
\end{equation}
where the \bctxt{invariant $ m $-body phase space measure} is defined as:
\begin{equation}
  \lps_m \equiv \prod_{k = 1}^m [\dd p_k]
\end{equation}

\section{Partitions of unity}
\label{sec:unit-part}

To define a partition of unity such as \eref{eq:delta-part}, define $ \mathcal{H}^i \equiv \fsu{n}{m+1} - \{i\} $ and introduce the function:
\begin{equation}
  d^i \equiv \prod_{k \in \mathcal{H}^i} p_{k,\perp} \prod_{l < m \in \mathcal{H}^i} \rho_{lm}
\end{equation}
where $ p_{k,\perp} $ is the transverse momentum of the parton $ k $. Then, the partition is found as:
\begin{equation}
  \Delta^i \defeq \frac{d^i}{\sum_{j \in \fsu{n}{m+1}} d^j}
\end{equation}
These clearly provide a $ \fsu{n}{m+1} $-partition of unity. To prove its properties, note that:
\begin{equation*}
  \so_i d^j = \lim_{E_i \rightarrow 0} \prod_{k \in \mathcal{H}^i} p_{k,\perp} \prod_{l < m \in \mathcal{H}^i} \rho_{lm} =
  \begin{cases}
    0 & i \neq j \\
    d_j & i = j
  \end{cases}
\end{equation*}
Thus trivially $ \so_i \Delta^j = \delta_i^j $. The collinear limit is slightly more complex:
\begin{equation*}
  \co_{ij} d^k = \lim_{\rho_{ij} \rightarrow 0} \prod_{s \in \mathcal{H}^i} p_{s,\perp} \prod_{l < m \in \mathcal{H}^i} \rho_{lm} =
  \begin{cases}
    0 & i,j \neq k \\
    d^k & i = k \neq j
  \end{cases}
\end{equation*}
The latter case has two possibilities: either $ j \in \{a,b\} $ or $ j \in \mathcal{H}^i $. Then, clearly $ \co_{ia} d^i = \co_{ib} d^i = 1 $, while if $ j \in \mathcal{H}^i $:
\begin{equation*}
  \co_{ij} \Delta^i = \frac{d^i}{d^i + d^j} = \left[ 1 + \frac{d^j}{d^i} \right]^{-1}
\end{equation*}
With explicit calculation:
\begin{equation*}
  \frac{d^j}{d^i} = \frac{p_{i_\perp}}{p_{j,\perp}} \frac{\rho_{1,j} \dots \rho_{i-1,j} \rho_{i+1,j} \dots \rho_{j-1,j} \rho_{j,j+1} \dots \rho_{j,m+1}}{\rho_{1,i} \dots \rho_{i-1,i} \rho_{i,i+1} \dots \rho_{i,j-1} \rho_{i,j+1} \dots \rho_{i,m+1}} = \frac{p_{i,\perp}}{p_{j,\perp}} = \frac{E_i}{E_j}
\end{equation*}
where we have used the fact that $ \rho_{ik} \rightarrow \rho_{jk} \,\,\forall k \neq i,j $ as $ \rho_{ij} \rightarrow 0 $. This completes the proof.

To construct the angular partition of unity in \eref{eq:omega-part}, we set $ g_{kl} \equiv \rho_{kl}^{-1} $ and define the angular factors:
\begin{equation}
  \omega^{\ur i} \defeq \frac{g_{i\ur}}{\sum_{j \in \prt{n}{m}(\ur)} g_{j\ur}}
\end{equation}
These clearly provide a $ \prt{n}{m}(\ur) $-partition of unity. Moreover:
\begin{equation*}
  \co_{j\ur} \omega^{\ur i} = \lim_{\rho_{j\ur} \rightarrow 0} \frac{g_{i\ur}}{\sum_{k \in \prt{n}{m}(\ur)} g_{k\ur}} = \bigg[ \lim_{\rho_{j\ur} \rightarrow 0} \sum_{k \in \prt{n}{m}(\ur)} \frac{\rho_{i\ur}}{\rho_{k\ur}} \bigg]^{-1} =
  \begin{cases}
    1^{-1} & j = i \\
    \infty^{-1} & j \neq i
  \end{cases}
  = \delta^i_j
\end{equation*}
where we made an abuse of notation. This completes the proof.

\section{Quadratic Casimir operators of \texorpdfstring{$ \SUn{n_c} $}{SU(n)}}
\label{sec:cas-op}

To prove \eref{eq:cas-sun}, first consider the fundamental representation $ \mathtt{n} $ of $ \SUn{n_c} $.
Then, contracting \eref{eq:quad-cas} with $ \delta^{ab} $ (with $ a,b = 1,\dots, n^2 - 1 $, as they label the basis of $ \mathfrak{su}(n_c) $):
\begin{equation*}
  C_2(\mathtt{n}) n_c = \frac{1}{2} (n_c^2 - 1)
\end{equation*}
To compute the Casimir operator for the adjoint representation $ \mathtt{g} $, consider the decomposition of the direct product of two representations:
\begin{equation*}
  \mathtt{r}_1 \otimes \mathtt{r}_2 = \bigoplus_i \mathtt{r}_i
\end{equation*}
In this representation $ T^a_{\mathtt{r}_1 \otimes \mathtt{r}_2} = T^a_{\mathtt{r}_1} \otimes \id_{\mathtt{r}_2} + \id_{\mathtt{r}_1} \otimes T^a_{\mathtt{r}_2} $, and it acts on tensor objects $ \Xi_{pq} $ whose first index transforms according to $ \mathtt{r}_1 $ and the second index according to $ \mathtt{r}_2 $. Recalling that $ \tr{T^a} = 0 $:
\begin{equation*}
  \begin{split}
    \tr (T^a_{\mathtt{r}_1 \otimes \mathtt{r}_2})^2
    &= \tr ((T^a_{\mathtt{r}_1})^2 \otimes \id_{\mathtt{r}_2} + 2 T^a_{\mathtt{r}_1} \otimes T^a_{\mathtt{r}_2} + \id_{\mathtt{r}_1} \otimes (T^a_{\mathtt{r}_2})^2) \\
    &= \tr (C_2(\mathtt{r}_1) \id_{\mathtt{r}_1} \otimes \id_{\mathtt{r}_2}) + \tr (C_2(\mathtt{r}_2) \id_{\mathtt{r}_1} \otimes \id_{\mathtt{r}_2}) = (C_2(\mathtt{r}_1) + C_2(\mathtt{r}_2)) n_{\mathtt{r}_1} n_{\mathtt{r}_2}
  \end{split}
\end{equation*}
However, by the decomposition above:
\begin{equation*}
  \tr (T^a_{\mathtt{r}_1 \otimes \mathtt{r}_2})^2 = \sum_i C_2(\mathtt{r}_i) n_{\mathtt{r}_i}
\end{equation*}
Consider $ \mathtt{n} \otimes \mathtt{n}^* $, where $ \mathtt{n}^* $ is the complex conjugate of the fundamental representation (for complex representations, $ \mathtt{r} $ and $ \mathtt{r}^* $ are generally inequivalent representations): then $ \Xi_{pq} $ contains a term proportional to the invariant $ \delta_{pq} $, while the other $ n_c^2 - 1 $  independent components transform as a general $ n_c \times n_c $ traceless tensor, i.e. under the adjoint representation of $ \SUn{n_c} $ (as of \eeref{eq:sun-herm}{eq:sun-trace}), thus $ \mathtt{n} \otimes \mathtt{n}^* = \ve{1} \oplus \mathtt{g} $ and the above identity becomes:
\begin{equation*}
  (C_2(\ve{1}) + C_2(\mathtt{g})) (n_c^2 - 1) = (C_2(\mathtt{n}) + C_2(\mathtt{n}^*)) n_c^2
\end{equation*}
Using $ C_2(\ve{1}) = 0 $ (as all generators are trivially zero) and $ C_2(\mathtt{n}^*) = C_2(\mathtt{n}) $:
\begin{equation*}
  C_2(\mathtt{g}) (n_c^2 - 1) = \frac{n_c^2 - 1}{n_c} n_c^2
\end{equation*}
which completes the proof.












\chapter{Collection of relevant equations}
\selectlanguage{english}

In this Appendix, we provide definitions of relevant objects used in this work. To simply various formulas, we use a notation analogous to \cite{rontsch-2503}:
\begin{equation}
  \begin{gathered}
    \ol{z} \equiv 1 - z
    \qquad \qquad
    \mathcal{D}_n(z) \equiv \pd{\frac{\log^n(1-z)}{1-z}}
    \\
    L_i \equiv \log \frac{\ema}{E_i}
    \qquad \qquad
    \mathscr{L}_i \equiv \log \frac{2E_i}{\ren}
    \qquad \qquad
    L_\text{max} \equiv \log \frac{2\ema}{\ren}
  \end{gathered}
\end{equation}

\section{Useful constants}

Denoting the colour-charge operators $ \ve{T}_i $, with the conventional normalization $ \ttr = \frac{1}{2} $ for $ \SUn{n_c} $, the squares of these operators are the quadratic Casimir operators of the corresponding representations:
\begin{equation}
  \ve{T}_q^2 = \ve{T}_{\bar{q}}^2 = \caf = \frac{n_c^2 - 1}{2n_c}
  \qquad \qquad
  \ve{T}_g^2 = \caa = n_c
\end{equation}
The quark and gluon anomalous dimensions are:
\begin{equation}
  \gamma_q = \frac{3}{2} \caf
  \qquad \qquad
  \gamma_q = \frac{11}{6} \caa - \frac{2}{3} \ttr n_q
\end{equation}
where $ n_q $ is the number of active flavours.

The strong coupling is renormalized in the $ \msb $ scheme, so that the bare and running couplings are related by:
\begin{equation}
  \bc S_\de = \rcr \ren^{2\de} \left[ 1 - \frac{\rcr}{2\pi} \frac{\beta_0}{\de} + \smo(\rc^2) \right]
\end{equation}
where $ S_\de \equiv (4\pi)^\de e^{-\eg \de} $ and:
\begin{equation}
  \beta_0 = \frac{11}{6} \caa - \frac{2}{3} \ttr n_q = \gamma_g
\end{equation}
It is convenient to define a quantity related to the coupling constant:
\begin{equation}
  [\rc] \equiv \frac{\rcr}{2\pi} \frac{e^{\eg \de}}{\Gamma(1-\de)}
\end{equation}

\section{Splitting functions}
\label{sec:spl-func}

THIS TO BE PARTIALLY MOVED TO SECTION ON COLLINEAR SINGULARITIES

Consider the final-state splitting process $ [i\ur]^* \rightarrow i(z) + \ur(1-z) $, where $ i $ and $ \ur $ are two partons of flavours $ f_i $ and $ f_\ur $ and $ [i\ur] $ is the corresponding clustered parton of flavour $ f_{[i\ur]} $. Recall that, given the interaction vertices determined by the QCD Lagrangian \eref{eq:qcd-lag} (see FIGURE), a gluon clustered with any type of parton preserves the latter's flavours, while a quark clustered with an antiquark gives a gluon.

The energy fraction carried by the parton $ i $ is defined as $ z \equiv 1 - E_\ur / E_{[i\ur]} $. As a consequence, the parton $ \ur $ carries an energy fraction $ 1 - z $. Denoting the spin-averaged fintal-state splitting functions as $ P_{f_{[i\ur]}f_i}(z) $, they read:
\begin{align}
  P_{qq}(z) & = \caf \left[ \frac{1 + z^2}{1 - z} - \de (1 - z) \right] \\
  P_{qg}(z) & = \caf \left[ \frac{1 - (1 - z)^2}{z} - \de z \right] \equiv P_{qq}(1 - z) \\
  P_{gq}(z) & = \ttr \left[ 1 - \frac{2z (1 - z)}{1 - \de} \right] \\
  P_{gg}(z) & = 2 \caa \left[ \frac{z}{1 - z} + \frac{1 - z}{z} + z (1 - z) \right]
\end{align}
Now, consider instead the initial-state splitting process $ i \rightarrow [i\ur]^* + \ur $, where $ i $ and $ \ur $ are respectively an ingoing and outgoing parton, while the clustered parton $ [i\ur]^* $ enters the hard scattering process. In this case, we define the $ z $ variable as $ z \equiv 1 - E_\ur / E_i $. The spin- and color-averaged intial-state splitting functions, denoted as $ P_{f_{[i\ur]}f_i, \text{i}}(z) $, are:
\begin{align}
  P_{qq,\text{i}} & = - z P_{qq}(1/z) \equiv P_{qq}(z) \\
  P_{qg,\text{i}} & = \left[ \frac{2n_c}{2 (1 - \de) (n_c^2 - 1)} \right] z P_{qg}(1/z) \equiv P_{gq}(z) \\
  P_{gq,\text{i}} & = \left[ \frac{2 (1 - \de) (n_c^2 - 1)}{2n_c} \right] z P_{gq}(1/z) \equiv P_{qg}(z) \\
  P_{gg,\text{i}} & = - z P_{gg}(1/z) \equiv P_{gg}(z)
\end{align}
Finally, the LO Altarelli-Parisi splitting kernels are:
\begin{align}
  \hat{P}^{(0)}_{qq}(z) & = \caf \left[ 2 \mathcal{D}_0(z) - (1 + z) + \frac{3}{2} \delta(1 - z) \right] \\
  \hat{P}^{(0)}_{qq}(z) & = \ttr \left[ (1 - z)^2 + z^2 \right] \\
  \hat{P}^{(0)}_{qq}(z) & = \caf \left[ \frac{1 + (1 - z)^2}{z} \right] \\
  \hat{P}^{(0)}_{qq}(z) & = 2 \caa \left[ \mathcal{D}_0(z) + z (1 - z) + \frac{1}{z} - 2 \right] + \beta_0 \delta(1 - z)
\end{align}
All these splitting functions and kernels can be found in \cite{Ellis-1996}.












\clearpage
\end{appendices}

\bookmarksetupnext{level = -1}
\pagestyle{biblio}
\printbibliography[heading = bibintoc, title = {Bibliography}]

\end{document}
