\selectlanguage{english}

The aim of the NSC subtraction scheme (SS) is to compute integrated subtraction terms which account for QCD corrections to the inclusive\footnotemark production of jets in a hadron collider, i.e. to the process:
\begin{equation}
  p + p \rightarrow X + N \,\text{jets}
\end{equation}
Here, $ X $ is a colour-neutral system. The hadron-scale physics is known to be separated from the parton-scale physics (see Section 1.1 of \cite{Collins-2011}): this makes it possible for us to directly manipulate partonic cross-sections according to \eref{eq:fact-th}, where now the sum runs over all intial-state massless partons $ a $ and $ b $ which contribute to the production of the considered final state. Moreover, for the rest of this work we set $ \ren = \fac = \mu $, where $ \mu $ is the typical energy scale of the considered process.

\footnotetext{Inclusive jet production denotes the theoretical prediction (or experimental measurement) of the cross-section for the production of jets of given kinematics, while summing/integrating over all other final-state radiation and particles.}

Denoting the partons' momenta as $ p_i \equiv \xi_i P_i $, $ i = 1,2 $, and suppressing the explicit dependence on the running coupling and the renormalization scale, it is possible to express the LO term of \eref{eq:part-ser-exp} as (see \secref{sec:ph-sp-p} for the definition of the $ m $-particle phase-space measure $ \lps_m $):
\begin{equation}
  \dd\pcs_{a,b}^{(0)}(p_1 , p_2) \defeq \sum_\text{f} \frac{\mathcal{N}}{2\hat{s}} \int \lps_m (2\pi)^4 \delta^{(4)}(p_\text{in} - p_\text{out}) \abs{\ampl^{(0)}_m(p_1, p_2, p_X, p_\text{f})}^2 \op_m(p_X, p_\text{f})
  \label{eq:ds-lo}
\end{equation}
where $ \hat{s} \equiv 2 p_1 \cdot p_2 $ is the partonic center-of-mass (CM) energy squared, $ p_\text{f} $ is the total final-state partonic momentum and the normalization factor $ \mathcal{N}_m $ includes all necessary symmetry factors, as well as averaging factors for initial-state colours and helicities. The sum runs over all possible partonic final states for the considered process (formalized in the next section).

$ \op_m $ is an IR-finite measurement function defining the observable, which ensures that the final state contains at least $ N $ resolved jets: in particular, if the energy of a final-state gluon vanishes (soft limit), or if two partons become collinear to one another (collinear limit), then $ \op_{m + n} \rightarrow \op_{m + n - 1} $ for $ n \in \N $, and $ \op_m \rightarrow 0 $. Moreover, we assume that $ \op_m $ contains the integration measure $ \lps_X $ for the colour-singlet. Note that at LO $ m = N $.

Similarly, it is possible to write the NLO corrections in \eref{eq:part-ser-exp} as:
\begin{equation}
  \dd\pcs_{a,b}^\text{R}(p_1 , p_2) \defeq \sum_\text{f} \frac{\mathcal{N}'}{2\hat{s}} \int \lps_{m+1} (2\pi)^4 \delta^{(4)}(p_\text{in} - p_\text{out}) \abs{\ampl^{(0)}_{m+1}(p_1, p_2, p_X, p_\text{f})}^2 \op_{m+1}(p_X, p_\text{f})
  \label{eq:real-cs}
\end{equation}
\begin{equation}
  \dd\pcs_{a,b}^\text{V}(p_1 , p_2) \defeq \sum_\text{f} \frac{\mathcal{N}}{2\hat{s}} \frac{\rcr}{2\pi} \int \lps_m (2\pi)^4 \delta^{(4)}(p_\text{in} - p_\text{out}) 2\Re \braket{\ampl_m^{(0)} | \ampl_m^{(1)}} \op_m(p_X, p_\text{f})
  \label{eq:virt-cs}
\end{equation}
\begin{equation}
  \dd\pcs_{a,b}^\text{C}(p_1 , p_2) \defeq \frac{\rcr}{2\pi} \frac{1}{\de} \sum_c \int_0^1 \frac{\dd z}{z} \left[ \ap_{f_c,f_a}(z) \dd\pcs_{c,b}^{(0)}(z p_1 , p_2) + \ap_{f_c,f_b}(z) \dd\pcs_{a,c}^{(0)}(p_1 , z p_2) \right]
  \label{eq:pdf-cs}
\end{equation}
Note that in \eref{eq:real-cs} the final state contains $ m + 1 $ partons. The \bctxt{Altarelli-Parisi splitting kernels} are listed in \secref{sec:spl-func}, and proof of \eref{eq:pdf-cs} is provided in \secref{ssec:coll-ren}.

The rest of this chapter is devoted to the extraction of IR-singularities from \eeref{eq:real-cs}{eq:pdf-cs}, proving their cancellation and providing the associated integrated counterterms.

\section{Nested subtraction}

As suggested by the name, in the NSC SS the IR-poles of real corrections are removed sequentially, starting from those arising from soft limits and then subtracting the collinear ones from the soft-regulated terms.

To show this procedure, we introduce some notation\footnotemark. First of all, we define the integrand function in \eeref{eq:ds-lo}{eq:real-cs} as:
\begin{equation}
  \flm_{a,b}[\fs{n}{m}] \equiv \nsym (2\pi)^4 \delta^{(4)}(p_1 + p_2 - p_X - p_\text{f}) \abs{\ampl_m^{(0)}(p_1, p_2, p_X, p_\text{f})}^2 \op_m(p_X, p_\text{f})
\end{equation}
where $ \fs{n}{m} $ is the set of $ m $ final-state partons and $ \mathcal{N}_\text{sym} $ is the relative symmetry factor. The index $ n \in \en_m[a,b,X] $ enumerates all the possible QCD final states which may contribute to the partonic process: these include all combinations of flavours $ \{f_i\}_{i = 1,\dots,m} $ consistent with the initial state $ (a,b) $ and the color-singlet $ X $. In the following, we suppress the arguments of $ \en_m $. The integration on the $ m $-parton final-state phase space is instead defined as:
\begin{equation}
  \ips{\flm_{a,b}[\fs{n}{m}]} \defeq \navg \int \lps_m \, \flm_m^{a,b}[\fs{n}{m}]
\end{equation}
where $ \navg $ is the appropriate initial-state averaging factor. Then, we can rewrite \eeref{eq:ds-lo}{eq:real-cs} as:
\begin{equation}
  2\hat{s}\, \dd\pcs_{a,b}^{(0)} = \sum_{n \in \en_m} \ips{\flm_{a,b}[\fs{n}{m}]}
  \qquad \qquad
  2\hat{s}\, \dd\pcs_{a,b}^\text{R} = \sum_{n \in \en_{m+1}} \ips{\flm_{a,b}[\fs{n}{m+1}]}
  \label{eq:lo-real}
\end{equation}
Soft and collinear singularities are isolated through operators acting on $ \flm $ functions: $ \so_i $ denotes the limit in which the parton $ i $ becomes soft, while $ \co_{ij} $ that in which the partons $ i $ and $ j $ become collinear to each other. These operators extract only the leading aymptotic behaviour of $ \flm $ which is non-integrable in $ d = 4 $ dimensions, hence, if they act on quantities without non-integrable singularities, then they identically vanish (e.g. $ \so_i \equiv 0 $ if $ i $ is a (anti)quark).

\footnotetext{The notation is slightly different from that in \cite{rontsch-2023, rontsch-2503, rontsch-2509}: in particular, we set $ F_\text{LM} \mapsto \flm $ and $ \mathcal{B} \mapsto \fs{}{} $.}

\subsection{Partonic sets}

A delicate step is the determination of which final-state partons can become unresolved: indeed, in fixed-order perturbative QCD, the number of final-state hard partons cannot drop below the number of jets in the LO process. This means that at NLO no more than one parton can become unresolved, and this is ensured by the $ \op $ operators. In order to use symmetry arguments to minimize the number of unresolved partons that need to be considered, we can partition the set of final-state partons as:
\begin{equation*}
  \fs{n}{m} = \fsg{n}{m} \cup \fsq{n}{m} \cup \fsbq{n}{m} \cup \fsqm{n}{m} \cup \fsbqm{n}{m}
\end{equation*}
which are respectively the subsets of final-state gluons, massless quarks, massless antiquarks, massive quarks and massive antiquarks. It is also usefull to define the set of all massless partons:
\begin{equation*}
  \prt{n}{m} \equiv \{a,b\} \cup \fsg{n}{m} \cup \fsq{n}{m} \cup \fsbq{n}{m} \equiv \{a,b\} \cup \fsu{n}{m}
\end{equation*}
as we only consider massless initial-state partons. Note that $ \fsq{n}{m} $ and $ \fsbq{n}{m} $ can be further partitioned into sets of definite massless quark flavours, and the same can be done with $ \fsqm{n}{m} $ and $ \fsbqm{n}{m} $ with massive quark flavours.

For the remainder of this chapter, we set $ \fsqm{n}{m} = \fsbqm{n}{m} = \emptyset $: this means that $ \fsu{n}{m} \equiv \fs{n}{m} $, however we keep this redundant notation in all the equations, as some will be used in the next chapter where massive partons will be included.

For NLO real emissions we consider an additional parton, i.e. a final state $ \fs{n}{m+1} $ with $ n \in \en_{m+1} $. To extract soft singularities from \eref{eq:lo-real}, we first consider a partition of unity such that:
\begin{equation}
  \sum_{i \in \fsu{n}{m+1}} \Delta^i = 1
  \quad : \quad
  \so_i \Delta^j = \delta_i^j
  \quad \land \quad
  \co_{ij} \Delta^k =
  \begin{cases}
    0 & i,j \neq k \\
    1 & i = k \,,\, j \in \{a,b\} \\
    z_{k,j} & i = k \,,\, j \in \fsu{n}{m+1} - \{i\}
  \end{cases}
  \label{eq:delta-part}
\end{equation}
with $ z_{k,j} \equiv E_k / (E_k + E_j) $. An explicit construction of these damping factors is given in \secref{sec:unit-part}. It is clear that a term multiplied by $ \Delta^i $ vanishes if any parton other than $ i $ becomes unresolved, thus this partition allows for the extraction of single unresolved partons:
\begin{equation*}
  2\hat{s}\, \dd\pcs_{a,b}^\text{R} = \sum_{n \in \en_{m+1}} \sum_{i \in \fsu{n}{m+1}} \ips{\Delta^i \flm_{ab}[\fs{n}{m+1}]}
\end{equation*}
We can relabel the potentially-unresolved parton $ i $ in each term as $ \ur_{f_i} $: then, for each allowed massless\footnotemark flavour $ f $, there are $ N_f $ equal terms, where $ N_f $ is the number of final-state partons of flavour $ f $. We can account for the cancellation of these factors with symmetry factors defining $ \fs{n}{m}(\ur_f) \equiv \fs{n}{m+1} - \{\ur_f\} $, thus imposing that the symmetry factors of $ \flm_{a,b}[\fs{n}{m}(\ur_f)] $ are determined ignoring $ \ur_f $ (i.e. implicitly multiplying by $ N_f $), but with the convention that the amplitude in $ \flm_{a,b}[\fs{n}{m}(\ur_f)] $ still contains the potentially-unresolved parton $ \ur_f $. Therefore:
\begin{equation}
  \begin{split}
    2\hat{s}\, \dd\pcs_{a,b}^\text{R}
    & = \sum_{n \in \en_{m+1}(g)} \ips{\Delta^\ur \flm_{a,b}[\fs{n}{m}(\ur_g)]} + \sum_{\rho = 1}^{n_f} \sum_{n \in \en_{m+1}(q_\rho)} \ips{\Delta^\ur \flm_{a,b}[\fs{n}{m}(\ur_{q_\rho})]} \\
    & \quad\, + \sum_{\rho = 1}^{n_f} \sum_{n \in \en_{m+1}(\bar{q}_\rho)} \ips{\Delta^\ur \flm_{a,b}[\fs{n}{m}(\ur_{\bar{q}_\rho})]}
  \end{split}
  \label{eq:part-chan}
\end{equation}
where $ \en_m(f) \subset \en_m : N_f \ge 1 $ denotes the subset of possible final states with at least one parton of flavour $ f $. The subscript $ f $ in $ \ur_f $ is suppressed when implicitly understood.

\footnotetext{Massive partons cannot go unresolved, as they do not determine neither soft nor collinear singularities.}

Now, the nested subtraction procedure introduced in \cite{rontsch-2017} can be applied. In particular, for each term we rewrite the identity operator as:
\begin{equation}
  \id = \so_\ur + \sum_{i \in \prt{n}{m}(\ur)} \oso_\ur \co_{i\ur} + \nlo^\ur
  \qquad \qquad
  \nlo^\ur \defeq \sum_{i \in \prt{n}{m}(\ur)} \oso_\ur \oco_{i\ur} \omega^{\ur i}
  \label{eq:nest-sub}
\end{equation}
where we defined the notation for generic operators $ \overline{\mathcal{O}} \equiv \id - \mathcal{O} $ and introduced an angular partition of unity (see \secref{sec:unit-part}):
\begin{equation}
  \sum_{i \in \prt{n}{m}(\ur)} \omega^{\ur i} = 1
  \quad : \quad
  \co_{j\ur} \omega^{\ur i} = \delta_j^i
  \label{eq:omega-part}
\end{equation}
There now remains to understand how the operators in \eref{eq:nest-sub} act on the $ \flm $ functions and how the partonic set $ \fs{n}{m}(\ur) $ changes when $ \ur $ effectively becomes unresolved.

\subsection{Soft limits}

We first consider the soft limit of each term in \eref{eq:part-chan}. For the last two terms this is trivial, since quarks do not determine soft singularities:
\begin{equation*}
  \so_{\ur_q} = \so_{\ur_{\bar{q}}} \equiv 0
\end{equation*}
The only soft singularities come from the first term, when a gluon becomes unresolved. In this limit, the factorization of the amplitude is found to be (see \cite{Catani-1997}):
\begin{equation}
  \so_{\ur_g} \flm_{a,b}[\fs{n}{m}(\ur_g)] = - 4\pi \bc \sum_{i,j \in \fs{n}{m}} \eik_{i,j} (\cc) \flm_{a,b}[\fs{n}{m}]
\end{equation}
where the \bctxt{eikonal factor} reads:
\begin{equation}
  \eik_{i,j} \equiv \frac{p_i \cdot p_j}{(p_i \cdot p_\ur) (p_j \cdot p_\ur)}
\end{equation}
To perform the integration on the unresolved phase space, we extract the $ [\dd p_\ur] $ measure from $ \lps_{m+1} $:
\begin{equation*}
  \ips{\so_\ur \Delta^\ur \flm_{a,b}[\fs{n}{m}(\ur_g)]} = - 4\pi \bc \sum_{i,j \in \fs{n}{m}} (\cc) \ips{\int [\dd p_\ur] \, \eik_{i,j} \flm_{a,b}[\fs{n}{m}]}
\end{equation*}
The only factor dependent on $ p_\ur $ is the eikonal factor, hence we can perform the integration explicitly:
\begin{equation*}
  \begin{split}
    \int [\dd p_\ur] \, \eik_{i,j}
    & = \int [\dd p_\ur] \frac{p_i \cdot p_j}{(p_i \cdot p_j) (p_j \cdot p_\ur)} = \int_0^{\ema} \frac{\dd E_\ur}{E_\ur^{1+2\de}} \int_{\mathbb{S}^{2-2\de}} \frac{\dd \Omega_{2-2\de}}{2(2\pi)^{3-2\de}} \frac{\rho_{ij}}{\rho_{i\ur} \rho_{j\ur}} \\
    & = - \frac{\ema^{-2\de}}{2\de} \rho_{ij} \frac{1}{8\pi^2} \frac{(4\pi)^\de}{\Gamma(1-\de)} 2^{-1-2\de} \frac{\Gamma^2(-\de)}{\Gamma(-2\de)} \,\hyp(1,1,1-\de,1-\eta_{ij}) \\
    & = \frac{1}{\de^2} \frac{[\rc]}{4\pi \bc} \left( \frac{2\ema}{\mu} \right)^{-2\de} \frac{\Gamma^2(1-\de)}{\Gamma(1-2\de)} \eta_{ij} \,\hyp(1,1,1-\de,1-\eta_{ij})
  \end{split}
\end{equation*}
where we made use of the short-hands defined in \eref{eq:short-hand} and the angular integral from Appendix G.3 of \cite{Asteriadis-2020}. It is thus possible to write the integrated soft-counterterm as:
\begin{equation}
  \sum_{n \in \en_{m+1}(g)} \ips{\so_\ur \Delta^\ur \flm_{a,b}[\fs{n}{m}(\ur_g)]} = [\rc] \sum_{n \in \en_m} \ips{\iso(\de) \flm_{a,b}[\fs{n}{m}]}
\end{equation}
where the \bctxt{integrated soft operator} is defined as:
\begin{equation}
  \iso(\de) \defeq - \frac{1}{\de^2} \left( \frac{2\ema}{\mu} \right)^{-2\de} \frac{\Gamma^2(1-\de)}{\Gamma(1-2\de)} \sum_{i,j \in \fs{n}{m}} \eta_{ij} (\cc) \,\hyp(1,1,1-\de,1-\eta_{ij})
\end{equation}

\subsection{Collinear limits}

Collinear limits are more delicate to analyze. In particular, we are interested in the extraction of hard-collinear singularities stemming from terms of the form $ \oso_\ur \co_{i\ur} $, which in the case of unresolved quarks coincides with $ \co_{i\ur} $.

\subsubsection{Generalized anomalous dimensions}

The factorization of the amplitude in a collinear limit can be found in \cite{Catani-1997}, and it depends on whether the unresolved final-state parton becomes collinear to an initial- or final-state parton.

\paragraph{Final-state collinear limit}

Consider the case of a final-state parton $ \ur $ of flavour $ f_\ur $ becoming collinear to another final-state parton $ i $ of flavour $ f_i $. Then, we set:
\begin{equation*}
  \ampl_\ur \ : \
  \begin{tikzpicture}[baseline = (r.base)]
    \begin{feynman}[inline = (r.base)]
      \vertex[blob, minimum size = 0.8cm] (v) {};

      \vertex[left = 1.2cm of v] (r1) {};
      \vertex[above = 0.6cm of r1] (a) {};
      \vertex (la) at ($(a) + (-0.1,0)$) {$ a $};
      \vertex[below = 0.6cm of r1] (b) {};
      \vertex (lb) at ($(b) + (-0.1,0)$) {$ b $};

      \vertex[right = 1.2cm of v] (r2) {};
      \vertex[above = 0.84cm of r2] (c) {};
      \vertex[below = 0.84cm of r2] (d) {};
      \vertex (vd1) at ($ (v) + (0.8,0.3) $) {$ \vdots $};
      \vertex (vd2) at ($ (v) + (0.8,-0.12) $) {$ \vdots $};

      \vertex[right = 1.5cm of v] (r3) {};
      \vertex[above = 1.5cm of r3] (e);

      \vertex[right = 0.7cm of e] (r4) {};
      \vertex[above = 0.4cm of r4] (f) {};
      \vertex (lf) at ($(f) + (0.1,0)$) {\color{red} $ \ur $};
      \vertex[below = 0.4cm of r4] (g) {};
      \vertex (lg) at ($(g) + (0.1,0)$) {\color{red} $ i $};

      \vertex[below = 0.25em of v] (r);

      \diagram* {
        (a) -- (v) -- (c),
        (b) -- (v) -- (d),
        (v) -- [red, edge label = {$ [i\ur] $}, sloped] (e),
        (e) -- [red] (f),
        (e) -- [red] (g),
      };
    \end{feynman}
  \end{tikzpicture}
  \qquad \qquad
  \ampl_0 \ : \
  \begin{tikzpicture}[baseline = (r.base)]
    \begin{feynman}[inline = (r.base)]
      \vertex[blob, minimum size = 0.8cm] (v) {};

      \vertex[left = 1.2cm of v] (r1) {};
      \vertex[above = 0.6cm of r1] (a) {};
      \vertex (la) at ($(a) + (-0.1,0)$) {$ a $};
      \vertex[below = 0.6cm of r1] (b) {};
      \vertex (lb) at ($(b) + (-0.1,0)$) {$ b $};

      \vertex[right = 1.2cm of v] (r2) {};
      \vertex[above = 0.84cm of r2] (c) {};
      \vertex[below = 0.84cm of r2] (d) {};
      \vertex (vd1) at ($ (v) + (0.8,0.3) $) {$ \vdots $};
      \vertex (vd2) at ($ (v) + (0.8,-0.12) $) {$ \vdots $};

      \vertex[right = 1.5cm of v] (r3) {};
      \vertex[above = 1.5cm of r3] (e);
      \vertex (le) at ($(e) + (0.4,0)$) {\color{red} $ [i\ur] $};

      \vertex[below = 0.25em of v] (r);

      \diagram* {
        (a) -- (v) -- (c),
        (b) -- (v) -- (d),
        (v) -- [red] (e),
      };
    \end{feynman}
  \end{tikzpicture}
\end{equation*}
The factorization of the amplitude reads:
\begin{equation}
  \co_{i\ur} \ampl_\ur = - \frac{8\pi \bc}{(p_i - p_\ur)^2} \spl_{f_{[i\ur]}f_i}(z) \ampl_0
  \label{eq:catani-spl}
\end{equation}
where $ \spl_{f_{[i\ur]f_i}f_i}(z) $ is the \bctxt{Altarelli-Parisi splitting function} associated to the splitting process $ [i\ur] \rightarrow i + \ur $ and $ z $ is the momentum fraction carried by the parton $ i $, i.e.:
\begin{equation}
  z \equiv 1 - \frac{E_\ur}{E_{[i\ur]}}
\end{equation}
The possible splittings are determined by the QCD interaction vertices (see \secref{ssec:qcd-quant}) and are listed in \eeref{eq:spl-1}{eq:spl-4}:
\begin{equation*}
  \begin{tikzpicture}
    \begin{feynman}
      \vertex (a) {};
      \vertex (la) at ($ (a) + (-0.1,0) $) {$ g $};

      \vertex[right = 1.5cm of a, dot] (v) {};
      \vertex[right = 1cm of v] (r1) {};

      \vertex[above = 0.75cm of r1] (b) {};
      \vertex (lb) at ($ (b) + (0.1,0) $) {$ g $};

      \vertex[below = 0.75cm of r1] (c) {};
      \vertex (lc) at ($ (c) + (0.1,0) $) {$ g $};

      \diagram* {
        (a) -- [gluon] (v),
        (v) -- [gluon] (b),
        (v) -- [gluon] (c),
      };
    \end{feynman}
  \end{tikzpicture}
  \qquad
  \begin{tikzpicture}
    \begin{feynman}
      \vertex (a) {};
      \vertex (la) at ($ (a) + (-0.1,0) $) {$ g $};

      \vertex[right = 1.5cm of a, dot] (v) {};
      \vertex[right = 1cm of v] (r1) {};

      \vertex[above = 0.75cm of r1] (b) {};
      \vertex (lb) at ($ (b) + (0.1,0) $) {$ q $};

      \vertex[below = 0.75cm of r1] (c) {};
      \vertex (lc) at ($ (c) + (0.1,0) $) {$ \bar{q} $};

      \diagram* {
        (a) -- [gluon] (v),
        (v) -- [fermion] (b),
        (v) -- [anti fermion] (c),
      };
    \end{feynman}
  \end{tikzpicture}
  \qquad
  \begin{tikzpicture}
    \begin{feynman}
      \vertex (a) {};
      \vertex (la) at ($ (a) + (-0.1,0) $) {$ q $};

      \vertex[right = 1.5cm of a, dot] (v) {};
      \vertex[right = 1cm of v] (r1) {};

      \vertex[above = 0.75cm of r1] (b) {};
      \vertex (lb) at ($ (b) + (0.1,0) $) {$ q $};

      \vertex[below = 0.75cm of r1] (c) {};
      \vertex (lc) at ($ (c) + (0.1,0) $) {$ g $};

      \diagram* {
        (a) -- [fermion] (v),
        (v) -- [fermion] (b),
        (v) -- [gluon] (c),
      };
    \end{feynman}
  \end{tikzpicture}
  \qquad
  \begin{tikzpicture}
    \begin{feynman}
      \vertex (a) {};
      \vertex (la) at ($ (a) + (-0.1,0) $) {$ q $};

      \vertex[right = 1.5cm of a, dot] (v) {};
      \vertex[right = 1cm of v] (r1) {};

      \vertex[above = 0.75cm of r1] (b) {};
      \vertex (lb) at ($ (b) + (0.1,0) $) {$ g $};

      \vertex[below = 0.75cm of r1] (c) {};
      \vertex (lc) at ($ (c) + (0.1,0) $) {$ q $};

      \diagram* {
        (a) -- [fermion] (v),
        (v) -- [gluon] (b),
        (v) -- [fermion] (c),
      };
    \end{feynman}
  \end{tikzpicture}
\end{equation*}
and respective charge-conjugates (splitting functions do not distinguish between quarks and antiquarks).

Using \eref{eq:catani-spl} we can derive a general expression for the integrated final-state collinear counterterm:
\begin{equation*}
  \begin{split}
    & \ips{\co_{i\ur} \Delta^\ur \flm_{a,b}[\fs{n}{m}(\ur)]} = \ips{\int [\dd p_i] [\dd p_\ur] \frac{4\pi \bc}{p_i \cdot p_\ur} z \spl_{f_{[i\ur]} f_i}(z) \flm_{a,b}[\fs{n'}{m}](p_{[i\ur]})} \\
    & \qquad \ = \ips{\int_{(\mathbb{S}^{2-2\de})^2} \frac{\dd \Omega_{2-2\de}^2}{(2(2\pi)^{3-2\de})^2} \int_0^{\ema} \frac{\dd E_i}{E_i^{-1+2\de}} \int_0^{\ema} \frac{\dd E_\ur}{E_\ur^{-1+2\de}} \frac{4\pi \bc}{E_{[i\ur]}^2 \rho_{i\ur}} \frac{\spl_{f_{[i\ur]} f_i}(z)}{1-z} \flm_{a,b}[\fs{n'}{m}](p_{[i\ur]})} \\
    & \qquad \ = - \frac{[\rc]}{\de} \frac{\Gamma^2(1-\de)}{\Gamma(1-2\de)} \ips{\int [\dd p_{[i\ur]}] \left( \frac{2E_{[i\ur]}}{\mu} \right)^{-2\de} \int_{\zm}^1 \dd z\, \frac{\spl_{f_{[i\ur]} f_i}(z)}{z^{-1 + 2\de} (1-z)^{2\de}} \flm_{a,b}[\fs{n'}{m}](p_{[i\ur]})}
  \end{split}
\end{equation*}
Note that we made the dependence of $ \flm_{a,b}[\fs{n'}{m}] $ on $ p_{[i\ur]} = z^{-1} p_i $ explicit, making it clear that the $ \flm $ function vanishes when $ z < 0 $, which is the case for $ z \in [\zm,0) $ as $ \zm \equiv 1 - \ema / E_{[i\ur]} < 0 $. The hard-collinear counterterm reads:
\begin{equation*}
  \begin{split}
    & \ips{\oso_\ur \co_{i\ur} \Delta^\ur \flm_{a,b}[\fs{n}{m}(\ur)]} \\
    & \qquad \qquad = - \frac{[\rc]}{\de} \frac{\Gamma^2(1-\de)}{\Gamma(1-2\de)} \ips{\int [\dd p_{[i\ur]}] \left( \frac{2E_{[i\ur]}}{\mu} \right)^{-2\de} \int_{\zm}^1 \dd z\, \oso_z \frac{\spl_{f_{[i\ur]} f_i}(z)}{z^{-1 + 2\de} (1-z)^{2\de}} \flm_{a,b}[\fs{n'}{m}](p_{[i\ur]})}
  \end{split}
\end{equation*}
where $ \so_z \equiv \lim_{z \rightarrow 1} $. Since $ \so_z $ only extracts the singular part, by \eeref{eq:spl-1}{eq:spl-4} it is clear that:
\begin{equation}
  \so_z \spl_{f_{[i\ur]}f_i}(z) = \frac{2}{1 - z} \tco^2_{f_{[i\ur]}} \delta_{f_{[i\ur]},f_i}
\end{equation}
Then, writing $ [\zm,1] = [\zm,0) \cup [0,1] $, there only remains to evaluate the following integral:
\begin{equation*}
  \begin{split}
    \int_{\zm}^0 \dd z\, \oso_z  \frac{\spl_{f_{[i\ur]}f_i}(z)}{z^{-1+2\de} (1-z)^{2\de}} \flm_{a,b}[\fs{n'}{m}](z^{-1}p_i)
    & = - 2 \tco^2_{f_{[i\ur]}} \delta_{f_{[i\ur]},f_i} \int_0^{\zm} \dd z\, (1 - z)^{-1-2\de} \flm_{a,b}[\fs{n'}{m}](p_i) \\
    & = \tco^2_{f_{[i\ur]}} \delta_{f_{[i\ur]},f_i} \frac{1 - e^{-2 \de \leg_i}}{\de} \flm_{a,b}[\fs{n'}{m}]
  \end{split}
\end{equation*}
where we set $ \leg_i \equiv \log (\ema / E_i) $ and suppressed the depedance of the $ \flm $ function on $ p_i = \so_z p_{[i\ur]} $. Finally, the integrated final-state hard-collinear counterterm can be expressed as:
\begin{equation}
  \ips{\oso_\ur \co_{i\ur} \Delta^\ur \flm_{a,b}[\fs{n}{m}](\ur)} = [\rc] \ips{\frac{\Gamma_{i,f_{[i\ur]} \rightarrow f_i f_\ur}}{\de} \flm_{a,b}[\fs{n'}{m}]}
  \label{eq:hc-fs}
\end{equation}
where $ n' \in \en_m $ and the generalized process-dependent final-state anomalous dimension is defined as:
\begin{equation}
  \Gamma_{i,f_{[i\ur]} \rightarrow f_i f_\ur} \defeq - \left( \frac{2E_i}{\mu} \right)^{-2\de} \frac{\Gamma^2(1-\de)}{\Gamma(1-2\de)} \left[ \int_0^1 \dd z\, \oso_z \frac{\spl_{f_{[i\ur]}f_i}}{z^{-1+2\de} (1-z)^{2\de}} - \delta_{f_{[i\ur]},f_i} \tco^2_{f_{[i\ur]}} \frac{1 - e^{-2\de L_i}}{\de} \right]
\end{equation}

\paragraph{Initial-state collinear limit}

Now, consider a final-state parton $ \ur $ of flavour $ f_\ur $ becoming collinear to an initial-state parton $ a $ of flavour $ f_a $. In this case, we set:
\begin{equation*}
  \ampl_\ur \ : \
  \begin{tikzpicture}[baseline = (r.base)]
    \begin{feynman}[inline = (r.base)]
      \vertex[blob, minimum size = 0.8cm] (v) {};

      \vertex[left = 1.2cm of v] (r1) {};
      \vertex[below = 0.6cm of r1] (b) {};
      \vertex (lb) at ($(b) + (-0.1,0)$) {$ b $};

      \vertex[right = 1.2cm of v] (r2) {};
      \vertex[above = 0.84cm of r2] (c) {};
      \vertex[below = 0.84cm of r2] (d) {};
      \vertex (vd1) at ($ (v) + (0.8,0.3) $) {$ \vdots $};
      \vertex (vd2) at ($ (v) + (0.8,-0.12) $) {$ \vdots $};

      \vertex[below = 0.25em of v] (r);

      \vertex[above = 2cm of b] (a) {};
      \vertex (la) at ($ (a) + (-0.1,0) $) {\color{red} $ a $};
      \vertex[right = 1cm of a] (e);
      \vertex[right = 1cm of e] (r3) {};
      \vertex[above = 0.5cm of r3] (f) {};
      \vertex (lf) at ($ (f) + (0.1,0) $) {\color{red} $ \ur $};

      \diagram* {
        (b) -- (v) -- (d),
        (v) -- (c),
        (a) -- [red] (e) -- [red, edge label = {\color{red} $ [a\ur] $}, sloped] (v),
        (e) -- [red] (f),
      };
    \end{feynman}
  \end{tikzpicture}
  \qquad \qquad
  \ampl_0 \ : \
  \begin{tikzpicture}[baseline = (r.base)]
    \begin{feynman}[inline = (r.base)]
      \vertex[blob, minimum size = 0.8cm] (v) {};

      \vertex[left = 1.2cm of v] (r1) {};
      \vertex[below = 0.6cm of r1] (b) {};
      \vertex (lb) at ($(b) + (-0.1,0)$) {$ b $};

      \vertex[right = 1.2cm of v] (r2) {};
      \vertex[above = 0.84cm of r2] (c) {};
      \vertex[below = 0.84cm of r2] (d) {};
      \vertex (vd1) at ($ (v) + (0.8,0.3) $) {$ \vdots $};
      \vertex (vd2) at ($ (v) + (0.8,-0.12) $) {$ \vdots $};

      \vertex[above = 1.7cm of b] (e);
      \vertex (le) at ($(e) + (-0.4,0)$) {\color{red} $ [a\ur] $};

      \vertex[below = 0.25em of v] (r);

      \diagram* {
        (e) -- [red] (v) -- (c),
        (b) -- (v) -- (d),
      };
    \end{feynman}
  \end{tikzpicture}
\end{equation*}
The factorization of the amplitude reads:
\begin{equation}
\co_{a\ur} \ampl_m = - \frac{8\pi \bc}{(p_a - p_\ur)^2} \frac{1}{z} P_{f_{[a\ur]}f_a , \text{i}}(z) \ampl_0
\end{equation}
where the Altarelli-Parisi initial-state splitting functions are defined in \eeref{eq:spli-1}{eq:spli-4} and depend on the energy fraction:
\begin{equation}
  z \equiv 1 - \frac{E_\ur}{E_a}
\end{equation}
associated to the splitting process $ a \rightarrow [a\ur] + \ur $. The integrated initial-state collinear counterterm is found analogously to the final-state one:
\begin{equation*}
  \begin{split}
    & \ips{\co_{a\ur} \Delta^\ur \flm_{a,b}[\fs{n}{m}(\ur)]} = - \frac{[\rc]}{\de} \frac{\Gamma^2(1-\de)}{\Gamma(1-2\de)} \ips{\left( \frac{2E_{[a\ur]}}{\mu} \right)^{-2\de} \int_{\zm}^1 \dd z\, \frac{\spl_{f_{[a\ur]} f_a , \text{i}}(z)}{z (1-z)^{2\de}} \flm_{[a\ur],b}[\fs{n'}{m}](p_{[a\ur]})}
  \end{split}
\end{equation*}
where $ p_{[a\ur]} = z p_a $. As in Appendix A.3 of \cite{rontsch-2503}, we define new splitting functions:
\begin{equation}
  \nspl_{f_{[a\ur]}f_a}(z, E_a) \defeq \oso_z \frac{P_{f_{[a\ur]}f_a , \text{i}}(z)}{(1-z)^{2\de}} - \delta_{f_{[a\ur]}f_a} \tco_{f_{[a\ur]}}^2 \frac{1 - e^{-2\de L_a}}{\de} \delta(1-z)
\end{equation}
which can be conveniently rewritten as:
\begin{equation}
  - \nspl_{\alpha \beta}(z, E_a) = \delta_{\alpha \beta} \delta(1-z) \left[ \gamma_\alpha + \tco_\alpha^2 \frac{1 - e^{-2\de L_a}}{\de} \right] + \left[ - \ap_{\alpha \beta}(z) + \de \fspl_{\alpha \beta}(z) \right]
\end{equation}
where $ \gamma_\alpha $ is the anomalous dimension of the partonic flavour $ \alpha $ and $ \fspl_{\alpha \beta} $ is the $ \de $-expansion of $ - \nspl(z, 0) $ starting at $ \smo(\de) $. This relation allows us to get an explicit expression for the integrated initial-state hard-collinear counterterm:
\begin{equation}
  \ips{\oso_\ur \co_{a\ur} \Delta^\ur \flm_{a,b}[\fs{n}{m}(\ur)]} = \frac{[\rc]}{\de} \ips{\delta_{f_{[a\ur]}f_a} \Gamma_{a,f_a} \flm_{[a\ur],b}[\fs{n'}{m}]} + \frac{[\rc]}{\de} \ips{\gspl_{f_{[a\ur]}f_a} \otimes \flm_{f_{[a\ur]},b}[\fs{n'}{m}]}
  \label{eq:hc-is}
\end{equation}
where $ n' \in \en_m[[a\ur],b,X] $ and the \bctxt{generalized initial-state anomalous dimension} and the generalized splitting functions are defined as:
\begin{equation}
  \Gamma_{a,\alpha} \defeq \left( \frac{2E_a}{\mu} \right)^{-2\de} \frac{\Gamma^2(1-\de)}{\Gamma(1-2\de)} \left[ \gamma_\alpha + \tco^2_\alpha \frac{1 - e^{-2\de L_a}}{\de} \right]
\end{equation}
\begin{equation}
  \gspl_{\alpha \beta}(z) \defeq \left( \frac{2E_a}{\mu} \right)^{-2\de} \frac{\Gamma^2(1-\de)}{\Gamma(1-2\de)} \left[ - \ap_{\alpha \beta}(z) + \de \fspl_{\alpha \beta}(z) \right]
\end{equation}
and the Mellin convolution is defined as:
\begin{equation}
  \gspl_{f_{[a\ur]}f_a} \otimes \flm_{[a\ur],b}[\fs{n'}{m}](p_a) \equiv \int_0^1 \frac{\dd z}{z}\, \gspl_{f_{[a\ur]}f_a}(z) \flm_{[a\ur],b}[\fs{n'}{m}](z p_a)
\end{equation}

\subsubsection{Hard-collinear limits}

The only difficulty left is how to combine the generalized anomalous dimensions arising from the various terms in \eref{eq:part-chan}.

\paragraph{Unresolved gluon}

Consider the hard-collinear operator $ \oso_{\ur_g} \co_{i\ur_g} $. Since the potentially-unresolved gluon can be clustered with, or emitted from, any parton without changing the latter's flavour, under the action of this hard-collinear operator $ \fs{n}{m}(\ur_g) \mapsto \fs{n}{m} $, i.e. the resolved partons remain unchanged. The symmetry factors in the $ \flm $ function thus do not change, so, from \ceref{eq:hc-fs}{eq:hc-is}:
\begin{equation}
  \sum_{n \in \en_{m+1}(g)} \ips{\oso_\ur \co_{a\ur} \Delta^\ur \flm_{a,b}[\fs{n}{m}(\ur_g)]} = \sum_{n \in \en_m} \frac{[\rc]}{\de} \left[ \ips{\Gamma_{a,f_a} \flm_{a,b}[\fs{n}{m}]} + \ips{\gspl_{f_a f_a} \otimes \flm_{a,b}[\fs{n}{m}]} \right]
\end{equation}
\begin{equation}
  \sum_{n \in \en_{m+1}(g)} \sum_{i \in \fsu{n}{m}(\ur_g)} \ips{\oso_\ur \co_{i\ur} \Delta^\ur \flm_{a,b}[\fs{n}{m}]} = \sum_{n \in \en_m} \sum_{i \in \fsu{n}{m}} \frac{[\rc]}{\de} \ips{\Gamma_{i, f_i \rightarrow f_i g} \,\flm_{a,b}[\fs{n}{m}]}
\end{equation}

\paragraph{Unresolved quark}

Now, consider the hard-collinear operator $ \oso_{\ur_{q_\rho}} \co_{i\ur_{q_\rho}} \equiv \co_{i\ur_{q_\rho}} $. In this case, IR singularities arise when $ \ur_{q_\rho} $ becomes collinear to:
\begin{itemize}
  \item an initial-state gluon ($ a_g \rightarrow \ur_{q_\rho} + [a\ur]_{\bar{q}_\rho} $), in which case $ \en_{m+1}(q_\rho) \mapsto \en_m[\bar{q}_\rho , b , X] $ and $ \fs{n}{m}(\ur_{q_\rho}) \mapsto \fs{n'}{m} $ composed of the same resolved partons, so from \eref{eq:hc-is}:
  \begin{equation*}
    \sum_{n \in \en_{m+1}(q_\rho)} \delta_{f_a , g} \ips{\oso_{\ur} \co_{a\ur} \Delta^\ur \flm_{a,b}[\fs{n}{m}(\ur_{q_\rho})]} = \sum_{n \in \en_m} \frac{[\rc]}{\de} \ips{\gspl_{\bar{q} g} \otimes \flm_{\bar{q}_\rho , b}[\fs{n}{m}]}
  \end{equation*}
  where we adopt the convention that the arguments of $ \en_m $ are determined by the $ \flm $ function in the sum. Note that the symmetry factors of $ \flm $ functions are solely determined by final-state resolved partons, hence initial-state hard-collinear limits always leave symmetry factors unchanged (independently of $ f_a $ and $ f_\ur $); however, initial-state spin degrees of freedom and colour factors do change, and this is absorbed in the $ \gspl $ function.
  \item an initial-state quark of the same flavour ($ a_{q_\rho} \rightarrow \ur_{q_\rho} + [a\ur]_g $), in which case $ \en_{m+1}(q_\rho) \mapsto \en_m[g,b,X] $ and $ \fs{n}{m}(\ur_{q_\rho}) \mapsto \fs{n'}{m} $ composed of the same resolved partons, so from \eref{eq:hc-is}:
  \begin{equation*}
    \sum_{n \in \en_{m+1}(q_\rho)} \delta_{f_a , q_\rho} \ips{\oso_\ur \co_{a\ur} \Delta^\ur \flm_{a,b}[\fs{n}{m}(\ur_{q_\rho})]} = \sum_{n \in \en_m} \frac{[\rc]}{\de} \ips{\gspl_{gq} \otimes \flm_{g,b}[\fs{n}{m}]}
  \end{equation*}
  \item a final-state gluon ($ [i\ur]_{q_\rho} \rightarrow \ur_{q_\rho} + i_g $), in which case $ \en_{m+1} (q_\rho) \mapsto \en_{m} $ and $ \fs{n}{m}(\ur_{q_\rho}) \mapsto \fs{n'}{m} $ with one less gluon and one more quark of flavour $ \rho $, hence from \eref{eq:hc-fs}:
  \begin{equation*}
    \sum_{n \in \en_{m+1}(q_\rho)} \sum_{i \in \fsu{n}{m}(\ur_{q_\rho})} \delta_{f_i , g} \ips{\oso_\ur \co_{i\ur} \Delta^\ur \flm_{a,b}[\fs{n}{m}(\ur_{q_\rho})]} = \sum_{n \in \en_m} \sum_{i \in \fsu{n}{m}} \delta_{f_i , q_\rho} \frac{[\rc]}{\de} \ips{\Gamma_{i , q \rightarrow gq} \,\flm_{a,b}[\fs{n}{m}]}
  \end{equation*}
  A clarification on symmetry factors is now needed. Assuming that $ \fs{n}{m}(\ur_{q_\rho}) $ contains $ N_g $ gluons and $ N_{q_\rho} $ quarks of flavour $ \rho $ (always not including $ \ur_{q_\rho} $), then the symetry factor is $ \nsym \sim 1/(N_g! N_{q_\rho}!) $. On the other hand, $ \fs{n'}{m} $ contains $ N_g - 1 $ gluons and $ N_{q_\rho} + 1 $ quarks of flavour $ \rho $, hence there is a mismatch of $ \nsym' = \nsym \cdot N_g / (N_{q_\rho} + 1) $: this is easily understood, as the $ i $-sum on the left-hand side gives an $ N_g $ factor, while that on the right-hand side gives an $ N_{q_\rho} + 1 $ factor.
  \item a final-state antiquark of the same flavour ($ [i\ur]_g \rightarrow \ur_{q_\rho} + i_{\bar{q}_\rho} $), in which case $ \en_{m+1}(q_\rho) \mapsto \en_m $ and $ \fs{n}{m}(\ur_{q_\rho}) \mapsto \fs{n'}{m} $ with one less antiquark of flavour $ \rho $ and one more gluon, hence from \eref{eq:hc-fs}:
  \begin{equation*}
    \sum_{n \in \en_{m+1}(q_\rho)} \sum_{i \in \fsu{n}{m}(\ur_{q_\rho})} \delta_{f_i , \bar{q}_\rho} \ips{\oso_\ur \co_{i\ur} \Delta^\ur \flm_{a,b}[\fs{n}{m}(\ur_{q_\rho})]} = \sum_{n \in \en_m} \sum_{i \in \fsu{n}{m}} \delta_{f_i , g} \ips{\Gamma_{i , g \rightarrow \bar{q} q} \,\flm_{a,b}[\fs{n}{m}]}
  \end{equation*}
\end{itemize}
Putting everything together:
\begin{multline}
  \sum_{n \in \en_{m+1}(q_\rho)} \sum_{i \in \prt{n}{m}(\ur_{q_\rho})} \ips{\oso_\ur \co_{i\ur} \Delta^\ur \flm_{a,b}[\fs{n}{m}(\ur_{q_\rho})]} = \sum_{n \in \en_m} \frac{[\rc]}{\de} \ips{\delta_{f_a , g} \gspl_{\bar{q} g} \otimes \flm_{\bar{q}_\rho , b}[\fs{n}{m}]} + \\
  + \sum_{n \in \en_m} \frac{[\rc]}{\de} \ips{\delta_{f_a , q_\rho} \gspl_{gq} \otimes \flm_{g,b}[\fs{n}{m}]} + \sum_{n \in \en_m} \sum_{i \in \fsu{n}{m}} \frac{[\rc]}{\de} \ips{\left( \delta_{f_i , q_\rho} \Gamma_{i , q \rightarrow g q} + \delta_{f_i , g} \Gamma_{i , g \rightarrow \bar{q} q} \right) \flm_{a,b}[\fs{n}{m}]}
\end{multline}

\paragraph{Unresolved antiquark}

Finally, consider the hard-collinear operator $ \oso_{\ur_{\bar{q}_\rho}} \co_{i\ur_{\bar{q}_\rho}} \equiv \co_{i\ur_{\bar{q}_\rho}} $. The analysis is equivalent to that of a potentially-unresolved quark, with the charge-conjugated splittings:
\begin{multline}
  \sum_{n \in \en_{m+1}(\bar{q}_\rho)} \sum_{i \in \prt{n}{m}(\ur_{\bar{q}_\rho})} \ips{\oso_\ur \co_{i\ur} \Delta^\ur \flm_{a,b}[\fs{n}{m}(\ur_{\bar{q}_\rho})]} = \sum_{n \in \en_m} \frac{[\rc]}{\de} \ips{\delta_{f_a , g} \gspl_{q g} \otimes \flm_{q_\rho , b}[\fs{n}{m}]} + \\
  + \sum_{n \in \en_m} \frac{[\rc]}{\de} \ips{\delta_{f_a , \bar{q}_\rho} \gspl_{g \bar{q}} \otimes \flm_{g,b}[\fs{n}{m}]} + \sum_{n \in \en_m} \sum_{i \in \fsu{n}{m}} \frac{[\rc]}{\de} \ips{\left( \delta_{f_i , \bar{q}_\rho} \Gamma_{i , \bar{q} \rightarrow g \bar{q}} + \delta_{f_i , g} \Gamma_{i , g \rightarrow q \bar{q}} \right) \flm_{a,b}[\fs{n}{m}]}
\end{multline}

\subsubsection{Collinear operator}

All the integrated hard-collinear counterterms can be combined into a convenient expression. First of all, putting everything together:
\begin{equation*}
  \begin{split}
    & \sum_{n \in \en_{m+1}(g)} \sum_{i \in \prt{n}{m}(\ur_g)} \ips{\oso_\ur \co_{i\ur} \Delta^\ur \flm_{a,b}[\fs{n}{m}(\ur_g)]} + \sum_{\rho = 1}^{n_f} \sum_{n \in \en_{m+1}(q_\rho)} \sum_{i \in \prt{n}{m}(\ur_{q_\rho})} \ips{\oso_\ur \co_{i\ur} \Delta^\ur \flm_{a,b}[\fs{n}{m}(\ur_{q_\rho})]} \\
    & \qquad \qquad \qquad \qquad \qquad \qquad \qquad \qquad \qquad \,\, + \sum_{\rho = 1}^{n_f} \sum_{n \in \en_{m+1}(\bar{q}_\rho)} \sum_{i \in \prt{n}{m}(\ur_{\bar{q}_\rho})} \ips{\oso_\ur \co_{i\ur} \Delta^\ur \flm_{a,b}[\fs{n}{m}(\ur_{\bar{q}_\rho})]} \\
    %
    %
    & = \frac{[\rc]}{\de} \sum_{n \in \en_m} \bigg[ \left[ \ips{\Gamma_{a,f_a} \flm_{a,b}[\fs{n}{m}]} + \ips{\gspl_{f_a f_a} \otimes \flm_{a,b}[\fs{n}{m}]} + \left( a \leftrightarrow b \right) \right] + \\
    & \qquad \quad + \sum_{\rho = 1}^{n_f} \big[ \ips{\delta_{f_a , g} \gspl_{\bar{q} g} \otimes \flm_{\bar{q}_\rho , b}[\fs{n}{m}]} + \ips{\delta_{f_a , q_\rho} \gspl_{gq} \otimes \flm_{g,b}[\fs{n}{m}]} + \ips{\delta_{f_a , g} \gspl_{q g} \otimes \flm_{q_\rho , b}[\fs{n}{m}]} \\
    & \qquad \qquad \qquad \qquad \qquad \qquad \qquad \qquad \qquad \qquad + \ips{\delta_{f_a , \bar{q}_\rho} \gspl_{g \bar{q}} \otimes \flm_{g,b}[\fs{n}{m}]} + \left( a \leftrightarrow b\right) \big] \\
    & \qquad \quad + \sum_{i \in \fsg{n}{m}} \ips{\Gamma_{i, g \rightarrow g g} \,\flm_{a,b}[\fs{n}{m}]} + \sum_{i \in \fsq{n}{m}} \ips{\Gamma_{i, q \rightarrow q g} \,\flm_{a,b}[\fs{n}{m}]} + \sum_{i \in \fsbq{n}{m}} \ips{\Gamma_{i, q \rightarrow q g} \,\flm_{a,b}[\fs{n}{m}]} + \\
    & \qquad \quad + 2n_f \sum_{i \in \fsg{n}{m}} \ips{\Gamma_{i , g \rightarrow \bar{q} q} \,\flm_{a,b}[\fs{n}{m}]} + \sum_{i \in \fsq{n}{m}} \ips{\Gamma_{i , q \rightarrow g q} \,\flm_{a,b}[\fs{n}{m}]} + \sum_{i \in \fsbq{n}{m}} \ips{\Gamma_{i , q \rightarrow g q} \,\flm_{a,b}[\fs{n}{m}]} \bigg]
  \end{split}
\end{equation*}
where we used the fact that generalized splitting functions and anomalous dimensions do not depend on quark/antiquark flavours. Collectively, Mellin convolutions can be grouped as:
\begin{equation*}
  \frac{[\rc]}{\de} \sum_c \sum_{n \in \en_m} \ips{\gspl_{f_c,f_a} \otimes \flm_{c,b}[\fs{n}{m}] + \gspl_{f_c,f_b} \otimes \flm_{a,c}}
\end{equation*}
where $ c $ spans the same set as $ a $ and $ b $. Note that the Mellin convolution always acts on the momentum associated to the parton $ c $, i.e. $ p_1 $ in the first term and $ p_2 $ in the second term. Moreover, we define the \bctxt{generalized final-state anomalous dimensions} as:
\begin{equation}
  \Gamma_{i,q} \defeq \Gamma_{i , q \rightarrow q g} + \Gamma_{i , q \rightarrow g q}
  \qquad \qquad
  \Gamma_{i,g} \defeq \Gamma_{i , g \rightarrow g g} + 2n_f \Gamma_{i , g \rightarrow \bar{q} q}
\end{equation}
Finally, we can then write the integrated hard-collinear counterterms as:
\begin{equation*}
  \begin{split}
    & \sum_{n \in \en_{m+1}(g)} \sum_{i \in \prt{n}{m}(\ur_g)} \ips{\oso_\ur \co_{i\ur} \Delta^\ur \flm_{a,b}[\fs{n}{m}(\ur_g)]} + \sum_{\rho = 1}^{n_f} \sum_{n \in \en_{m+1}(q_\rho)} \sum_{i \in \prt{n}{m}(\ur_{q_\rho})} \ips{\oso_\ur \co_{i\ur} \Delta^\ur \flm_{a,b}[\fs{n}{m}(\ur_{q_\rho})]} \\
    & \qquad \qquad \qquad \qquad \qquad \qquad \qquad \qquad \qquad \,\, + \sum_{\rho = 1}^{n_f} \sum_{n \in \en_{m+1}(\bar{q}_\rho)} \sum_{i \in \prt{n}{m}(\ur_{\bar{q}_\rho})} \ips{\oso_\ur \co_{i\ur} \Delta^\ur \flm_{a,b}[\fs{n}{m}(\ur_{\bar{q}_\rho})]} \\
    & = \frac{[\rc]}{\de} \sum_c \sum_{n \in \en_m} \ips{\gspl_{f_c,f_a} \otimes \flm_{c,b}[\fs{n}{m}] + \gspl_{f_c,f_b} \otimes \flm_{a,c}} + [\rc] \sum_{n \in \en_m} \ips{\ico(\de) \flm_{a,b}[\fs{n}{m}]}
  \end{split}
\end{equation*}
where the \bctxt{integrated collinear operator} is defined as:
\begin{equation}
  \ico(\de) \defeq \sum_{i \in \prt{n}{m}} \frac{\Gamma_{i,f_i}}{\de}
\end{equation}
This allows us to rewrite \eref{eq:part-chan} as:
\begin{multline}
  2\hat{s}\, \dd\pcs_{a,b}^\text{R} = 2\hat{s}\, \dd\pcs_{a,b}^\text{NLO} + [\rc] \sum_{n \in \en_m} \ips{[\iso(\de) + \ico(\de)] \flm_{a,b}[\fs{n}{m}]} + \\
  + \frac{[\rc]}{\de} \sum_c \sum_{n \in \en_m} \ips{\gspl_{f_c,f_a} \otimes \flm_{c,b}[\fs{n}{m}] + \gspl_{f_c,f_b} \otimes \flm_{a,c}}
  \label{eq:real-fin}
\end{multline}
where the finite reminder is:
\begin{equation}
  \dd\pcs_{a,b}^\text{NLO} \equiv \sum_{n \in \en_{m+1}} \ips{\nlo^\ur \Delta^\ur \bigg[ \flm_{a,b}[\fs{n}{m}(\ur_g)] + \sum_{\rho = 1}^{n_f} \left( \flm_{a,b}[\fs{n}{m}(\ur_{q_\rho})] + \flm_{a,b}[\fs{n}{m}(\ur_{\bar{q}_\rho})] \right) \bigg]}
\end{equation}

\section{Non-real corrections}

The counterterms resulting from the extraction of IR singularities from real corrections, which have an explicit expression in \eref{eq:real-fin}, are not $ \de $-finite: their poles must cancel against those resulting from non-real corrections, i.e. from virtual corrections and from the collinear renormalization of PDFs.

\subsection{Virtual corrections}

An expression with clear $ \de $-poles can be derived from \eref{eq:virt-cs} by expressing the 1-loop amplitude in \eref{eq:mat-exp} using Catani's formula:
\begin{equation}
  \ket{\ampl_m^{(1)}(\mu^2 ; \{p\})} = \ioo(\de) \ket{\ampl_m^{(0)}(\mu^2 ; \{p\})} + \ket{\ampl_m^\text{fin}(\mu^2 ; \{p\})}
\end{equation}
where $ \ampl_m^\text{fin} $ is the finite part of the 1-loop amplitude. Catani's operator can be expressed as in \cite{Catani-1998}:
\begin{equation}
  \ioo(\de) = \frac{1}{2} \frac{e^{\eg\de}}{\Gamma(1-\de)} \sum_{i \neq j \in \fsp{n}{m}} \frac{\tco_i \cdot \tco_j}{\tco_i^2} \left( \frac{\mu^2 e^{-i \lambda_{ij} \pi}}{2p_i \cdot p_j} \right)^\de \ns_i(\de)
\end{equation}
with $ \lambda_{ij} = +1 $ if $ i $ and $ j $ are both incoming/outgoing and $ \lambda_{ij} = 0 $ otherwise. The singular function $ \ns_i(\de) $ depends only on the parton's flavour:
\begin{equation}
  \ns_i(\de) \equiv \frac{\tco_i^2}{\de^2} + \frac{\gamma_i}{\de}
\end{equation}
The $ \ioo $ operator extracts the $ \de $-poles from \eref{eq:virt-cs} as:
\begin{equation*}
  2\Re \braket{\ampl_m^{(0)} | \ampl_m^{(1)}} = \frac{e^{\eg\de}}{\Gamma(1-\de)} \ivo(\de) \abs{\ampl_m^{(0)}}^2 + 2\Re \braket{\ampl_m^{(0)} | \ampl_m^\text{fin}}
\end{equation*}
where the \bctxt{virtual operator} is defined as:
\begin{equation}
  \ivo(\de) \equiv \frac{\Gamma(1-\de)}{e^{\eg\de}} \left( \ioo(\de) + \ioo\dg(\de) \right) = \sum_{i \neq j \in \fsp{n}{m}} \frac{\tco_i \cdot \tco_j}{\tco_i^2} \left( \frac{\mu^2}{2p_i \cdot p_j} \right)^\de \cos \left( \lambda_{ij} \pi \de \right) \ns_i(\de)
\end{equation}
Then, \eref{eq:virt-cs} can be rewritten as:
\begin{equation}
  2\hat{s} \dd\pcs_{a,b}^\text{V} = [\rc] \ips{\ivo(\de) \flm_{a,b}[\fs{n}{m}]} + \ips{\flm_{a,b}^\text{fin}[\fs{n}{m}]}
\end{equation}
where the second term is the $ \de $-finite remainder of the 1-loop amplitude.

\subsection{Collinear renormalization}
\label{ssec:coll-ren}

\section{Integrated counterterms}










