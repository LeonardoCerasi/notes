\begin{frame}

  need for precision estimates at LHC

\end{frame}

%=======================================================================

\begin{frame}

  factorization theorem and perturbative expansion of $ \mathrm{d} \hat{\sigma}_{a,b} $

\end{frame}

%=======================================================================

\begin{frame}

  real and virtual corrections

\end{frame}

%=======================================================================

\begin{frame}

  soft and collinear singularities (in CDR, show in real corrections)

\end{frame}

%=======================================================================

\begin{frame}

  subtraction scheme to regulate divergences

\end{frame}

%=======================================================================

\begin{frame}

  introduce the NSC SS

\end{frame}

%=======================================================================

\begin{frame}

  briefly show pole cancellation in the NSC SS

\end{frame}

%=======================================================================

\begin{frame}

  explain why massive quarks change $ I_\text{S}(\epsilon) $ and $ I_\text{V}(\epsilon) $, but not $ I_\text{C}(\epsilon) $

\end{frame}

%=======================================================================

\begin{frame}

  show how $ I_\text{S}(\epsilon) $ changes (in particular massive angular integrals)

\end{frame}

%=======================================================================

\begin{frame}

  show how $ I_\text{V}(\epsilon) $ changes (in particular, colour-correlated $ \epsilon^{-2} $-poles in $ \mathcal{V}_{i,j}(\epsilon) $ coefficients)

\end{frame}

%=======================================================================

\begin{frame}

  highlights of pole cancellation in $ I_{\text{S}+\text{V}}(\epsilon) $, define $ \chi_{i,j}(\epsilon) $ coefficients and explain their property

\end{frame}

%=======================================================================

\begin{frame}

  show pole cancellation in the colour-correlated sum of $ I_{\text{S}+\text{V}}(\epsilon) $, leaving the same (and opposite) pole terms of $ I_\text{C}(\epsilon) $

\end{frame}

%=======================================================================

\begin{frame}

  show integrated counterterms, highlighting massive logs, and draw conclusions

\end{frame}










