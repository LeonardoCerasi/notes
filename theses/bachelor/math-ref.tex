\selectlanguage{english}

\section{Phase-space parametrization}
\label{sec:ph-sp-p}

In dimensional regularization with $ d = 4 - 2 \de $, we define the measure on the phase space of a parton $ i $ to be:
\begin{equation}
  [\dd p_i] \equiv \frac{\dd^{d-1} p_i}{(2\pi)^{d-1} 2E_i} \theta(\ema - E_i)
\end{equation}
Note that $ \ema $ is an upper bound on the energies of individual partons: it is an arbitrary parameter to be taken sufficiently large as to be greater or equal to the maximal energy that a final-state parton can reach.

This measure can be cast in a more useful form introducing a suitable parametrization of the phase space: in particular, given that $ \R^n-\{\ve{0}\} \cong \R^+ \times \mathbb{S}^{n-1} $, it is convenient to introduce hyperspherical coordinates on the $ \mathbb{S}^{d-2} $ component of the phase space. In general, the \bctxt{hyperspherical measure} on $ \mathbb{S}^n $ is recursively defined as:
\begin{equation}
  \dd \Omega_n = \sin^{n-1} \varphi \, \dd \varphi \, \dd \Omega_{n-1}
  \label{eq:hyp-rec}
\end{equation}
Using \eref{eq:hyp-rec} (with $ \sin \varphi \, \dd \varphi = \dd \cos \varphi $), we can express the measure $ \dd^{d-1} p_i $ as:
\begin{equation}
  \dd^{d-1} p_i = \abs{\ve{p}_i}^{d-2} \, \dd \abs{\ve{p}_i} \, \sin^{d-4} \varphi \, \dd \cos \varphi \, \dd \Omega_{d-2}
\end{equation}
As we are only interested in integrations on phase spaces of real unresolved partons, which can only be massless, we can use the on-shell condition $ p_i^2 = 0 $ to express $ \abs{\ve{p}_i} = E_i $, so that the phase-space measure becomes:
\begin{equation}
  [\dd p_i] = \theta(\ema - E_i) E_i^{d-3} \dd E_i \, \sin^{d-4} \varphi \, \dd \cos \varphi \, \frac{\dd \Omega_{d-2}}{2(2\pi)^{d-1}}
\end{equation}
with $ E_i \in \R^+ $ and $ \varphi \in [0,\pi] $.

\subsection{Multi-particle phase space}

When considering scattering processes, in general the final state is a multi-particle state, hence the measure on the final-state phase space must account for energy conservation too.

Given a $ 2 \rightarrow m $ scattering process with well-defined initial momenta $ p_\mathcal{A} $ and $ p_\mathcal{B} $, then the differential cross-section is (see Chapter 4 of \cite{Peskin-1995}):
\begin{equation}
  \dd\sigma = \frac{1}{2E_\mathcal{A} 2E_\mathcal{B} \abs{\ve{v}_\mathcal{A} - \ve{v}_\mathcal{B}}} \prod_{k = 1}^m \int \frac{\dd^3p_k}{(2\pi)^3 2E_k} \abs{\mathcal{M}(\mathcal{A} \mathcal{B} \rightarrow \{f\})}^2 (2\pi)^4 \delta^{(4)}(p_\mathcal{A} + p_\mathcal{B} - \textstyle\sum_{i = 1}^m p_i)
  \label{eq:scatt-cr-sec}
\end{equation}
where $ \mathcal{M}(\mathcal{A} \mathcal{B} \rightarrow \{f\}) $ is the matrix element of the scattering process and $ \ve{v}_k \equiv \frac{\ve{p}_k}{E_k} $ is the velocity of the $ k^\text{th} $ particle.

As we are only interested in massless initial-state partons, in the center-of-mass (CM) frame $ p_{\mathcal{A},\mathcal{B}} = (E, \pm\ve{p}) $, hence it is trivial to see that the flux factor in \eref{eq:scatt-cr-sec} is just $ 2\hat{s} \defeq 2 (p_\mathcal{A} + p_\mathcal{B})^2 $. The differential cross-section can then be rewritten as:
\begin{equation}
  \dd \sigma = \frac{1}{2\hat{s}} \int \lps_m \abs{\mat(\mathcal{A}\mathcal{B} \rightarrow \{f\})}^2
\end{equation}
where the \bctxt{invariant $ m $-body phase space measure} is defined as:
\begin{equation}
  \lps_m \equiv \prod_{k = 1}^m [\dd p_k] (2\pi)^4 \delta^{(4)}(p_\mathcal{A} + p_\mathcal{B} - \sum_{i = 1}^m p_i)
\end{equation}

\section{Angular integrals}

\section{Quadratic Casimir operators of \texorpdfstring{$ \SUn{n_c} $}{SU(n)}}
\label{sec:cas-op}

To prove \eref{eq:cas-sun}, first consider the fundamental representation $ \mathtt{n} $ of $ \SUn{n_c} $.
Then, contracting \eref{eq:quad-cas} with $ \delta^{ab} $ (with $ a,b = 1,\dots, n^2 - 1 $, as they label the basis of $ \mathfrak{su}(n_c) $):
\begin{equation*}
  C_2(\mathtt{n}) n = \frac{1}{2} (n_c^2 - 1)
\end{equation*}
To compute the Casimir operator for the adjoint representation $ \mathtt{g} $, consider the decomposition of the direct product of two representations:
\begin{equation*}
  \mathtt{r}_1 \otimes \mathtt{r}_2 = \bigoplus_i \mathtt{r}_i
\end{equation*}
In this representation $ T^a_{\mathtt{r}_1 \otimes \mathtt{r}_2} = T^a_{\mathtt{r}_1} \otimes \id_{\mathtt{r}_2} + \id_{\mathtt{r}_1} \otimes T^a_{\mathtt{r}_2} $, and it acts on tensor objects $ \Xi_{pq} $ whose first index transforms according to $ \mathtt{r}_1 $ and the second index according to $ \mathtt{r}_2 $. Recalling that $ \tr{T^a} = 0 $:
\begin{equation*}
  \begin{split}
    \tr (T^a_{\mathtt{r}_1 \otimes \mathtt{r}_2})^2
    &= \tr ((T^a_{\mathtt{r}_1})^2 \otimes \id_{\mathtt{r}_2} + 2 T^a_{\mathtt{r}_1} \otimes T^a_{\mathtt{r}_2} + \id_{\mathtt{r}_1} \otimes (T^a_{\mathtt{r}_2})^2) \\
    &= \tr (C_2(\mathtt{r}_1) \id_{\mathtt{r}_1} \otimes \id_{\mathtt{r}_2}) + \tr (C_2(\mathtt{r}_2) \id_{\mathtt{r}_1} \otimes \id_{\mathtt{r}_2}) = (C_2(\mathtt{r}_1) + C_2(\mathtt{r}_2)) n_{\mathtt{r}_1} n_{\mathtt{r}_2}
  \end{split}
\end{equation*}
However, by the decomposition above:
\begin{equation*}
  \tr (T^a_{\mathtt{r}_1 \otimes \mathtt{r}_2})^2 = \sum_i C_2(\mathtt{r}_i) n_{\mathtt{r}_i}
\end{equation*}
Consider $ \mathtt{n} \otimes \mathtt{n}^* $, where $ \mathtt{n}^* $ is the complex conjugate of the fundamental representation (for complex representations, $ \mathtt{r} $ and $ \mathtt{r}^* $ are generally inequivalent representations): then $ \Xi_{pq} $ contains a term proportional to the invariant $ \delta_{pq} $, while the other $ n_c^2 - 1 $  independent components transform as a general $ n_c \times n_c $ traceless tensor, i.e. under the adjoint representation of $ \SUn{n_c} $ (as of \eeref{eq:sun-herm}{eq:sun-trace}), thus $ \mathtt{n} \otimes \mathtt{n}^* = \ve{1} \oplus \mathtt{g} $ and the above identity becomes:
\begin{equation*}
  (C_2(\ve{1}) + C_2(\mathtt{g})) (n_c^2 - 1) = (C_2(\mathtt{n}) + C_2(\mathtt{n}^*)) n_c^2
\end{equation*}
Using $ C_2(\ve{1}) = 0 $ (as all generators are trivially zero) and $ C_2(\mathtt{n}^*) = C_2(\mathtt{n}) $:
\begin{equation*}
  C_2(\mathtt{g}) (n_c^2 - 1) = \frac{n_c^2 - 1}{n} n_c^2
\end{equation*}
which completes the proof.










