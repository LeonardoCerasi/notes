\selectlanguage{english}

\section{Phase-space parametrization}
\label{sec:ph-sp-p}

In dimensional regularization with $ d = 4 - 2 \de $, we define the measure on the phase space of a parton $ i $ to be:
\begin{equation}
  [\dd p_i] \equiv \frac{\dd^{d-1} p_i}{(2\pi)^{d-1} 2E_i} \theta(\ema - E_i)
\end{equation}
Note that $ \ema $ is an upper bound on the energies of individual partons: it is an arbitrary parameter to be taken sufficiently large as to be greater or equal to the maximal energy that a final-state parton can reach.

This measure can be cast in a more useful form introducing a suitable parametrization of the phase space: in particular, given that $ \R^n-\{\ve{0}\} \cong \R^+ \times \mathbb{S}^{n-1} $, it is convenient to introduce hyperspherical coordinates on the $ \mathbb{S}^{d-2} $ component of the phase space. In general, the \bctxt{hyperspherical measure} on $ \mathbb{S}^n $ is recursively defined as:
\begin{equation}
  \dd \Omega_n = \sin^{n-1} \varphi \, \dd \varphi \, \dd \Omega_{n-1}
  \label{eq:hyp-rec}
\end{equation}
Using \eref{eq:hyp-rec} (with $ \sin \varphi \, \dd \varphi = \dd \cos \varphi $), we can express the measure $ \dd^{d-1} p_i $ as:
\begin{equation}
  \dd^{d-1} p_i = \abs{\ve{p}_i}^{d-2} \, \dd \abs{\ve{p}_i} \, \sin^{d-4} \varphi \, \dd \cos \varphi \, \dd \Omega_{d-3}
\end{equation}
As we are only interested in integrations on phase spaces of real unresolved partons, which can only be massless, we can use the on-shell condition $ p_i^2 = 0 $ to express $ \abs{\ve{p}_i} = E_i $, so that the phase-space measure becomes:
\begin{equation}
  [\dd p_i] = \theta(\ema - E_i) E_i^{d-3} \dd E_i \, \sin^{d-4} \varphi \, \dd \cos \varphi \, \frac{\dd \Omega_{d-3}}{2(2\pi)^{d-1}}
\end{equation}
with $ E_i \in \R^+ $ and $ \varphi \in [0,\pi] $.

\subsection{Multi-particle phase space}

When considering scattering processes, in general the final state is a multi-particle state, hence the measure on the final-state phase space must account for energy conservation too.

Given a $ 2 \rightarrow m $ scattering process with well-defined initial momenta $ p_\mathcal{A} $ and $ p_\mathcal{B} $, then the differential cross-section is (see Chapter 4 of \cite{Peskin-1995}):
\begin{equation}
  \dd\sigma = \frac{1}{2E_a 2E_b \abs{\ve{v}_a - \ve{v}_b}} \prod_{k = 1}^m \int \frac{\dd^3p_k}{(2\pi)^3 2E_k} \abs{\ampl(ab \rightarrow \mathcal{H})}^2 (2\pi)^4 \delta^{(4)}(p_a + p_b - \textstyle\sum_{i = 1}^m p_i)
  \label{eq:scatt-cr-sec}
\end{equation}
where $ \ampl(ab \rightarrow \mathcal{H}) $ is the amplitude of the scattering process and $ \ve{v}_k \equiv \frac{\ve{p}_k}{E_k} $ is the velocity of the $ k^\text{th} $ particle.

As we are only interested in massless initial-state partons, in the center-of-mass (CM) frame $ p_{a,b} = (E, \pm\ve{p}) $, hence it is trivial to see that the flux factor in \eref{eq:scatt-cr-sec} is just $ 2\hat{s} \defeq 2 (p_a + p_b)^2 $. The differential cross-section can then be rewritten as:
\begin{equation}
  \dd \sigma = \frac{1}{2\hat{s}} \int \lps_m (2\pi)^4 \delta^{(4)}(p_a + p_b - \sum_{i = 1}^m p_i) \abs{\mat(ab \rightarrow \mathcal{H})}^2
\end{equation}
where the \bctxt{invariant $ m $-body phase space measure} is defined as:
\begin{equation}
  \lps_m \equiv \prod_{k = 1}^m [\dd p_k]
\end{equation}

\section{Partitions of unity}
\label{sec:unit-part}

To define a partition of unity such as \eref{eq:delta-part}, define $ \mathcal{H}^i \equiv \fsu{n}{m+1} - \{i\} $ and introduce the function:
\begin{equation}
  d^i \equiv \prod_{k \in \mathcal{H}^i} p_{k,\perp} \prod_{l < m \in \mathcal{H}^i} \rho_{lm}
\end{equation}
where $ p_{k,\perp} $ is the transverse momentum of the parton $ k $. Then, the partition is found as:
\begin{equation}
  \Delta^i \defeq \frac{d^i}{\sum_{j \in \fsu{n}{m+1}} d^j}
\end{equation}
These clearly provide a $ \fsu{n}{m+1} $-partition of unity. To prove its properties, note that:
\begin{equation*}
  \so_i d^j = \lim_{E_i \rightarrow 0} \prod_{k \in \mathcal{H}^i} p_{k,\perp} \prod_{l < m \in \mathcal{H}^i} \rho_{lm} =
  \begin{cases}
    0 & i \neq j \\
    d_j & i = j
  \end{cases}
\end{equation*}
Thus trivially $ \so_i \Delta^j = \delta_i^j $. The collinear limit is slightly more complex:
\begin{equation*}
  \co_{ij} d^k = \lim_{\rho_{ij} \rightarrow 0} \prod_{s \in \mathcal{H}^i} p_{s,\perp} \prod_{l < m \in \mathcal{H}^i} \rho_{lm} =
  \begin{cases}
    0 & i,j \neq k \\
    d^k & i = k \neq j
  \end{cases}
\end{equation*}
The latter case has two possibilities: either $ j \in \{a,b\} $ or $ j \in \mathcal{H}^i $. Then, clearly $ \co_{ia} d^i = \co_{ib} d^i = 1 $, while if $ j \in \mathcal{H}^i $:
\begin{equation*}
  \co_{ij} \Delta^i = \frac{d^i}{d^i + d^j} = \left[ 1 + \frac{d^j}{d^i} \right]^{-1}
\end{equation*}
With explicit calculation:
\begin{equation*}
  \frac{d^j}{d^i} = \frac{p_{i_\perp}}{p_{j,\perp}} \frac{\rho_{1,j} \dots \rho_{i-1,j} \rho_{i+1,j} \dots \rho_{j-1,j} \rho_{j,j+1} \dots \rho_{j,m+1}}{\rho_{1,i} \dots \rho_{i-1,i} \rho_{i,i+1} \dots \rho_{i,j-1} \rho_{i,j+1} \dots \rho_{i,m+1}} = \frac{p_{i,\perp}}{p_{j,\perp}} = \frac{E_i}{E_j}
\end{equation*}
where we have used the fact that $ \rho_{ik} \rightarrow \rho_{jk} \,\,\forall k \neq i,j $ as $ \rho_{ij} \rightarrow 0 $. This completes the proof.

To construct the angular partition of unity in \eref{eq:omega-part}, we set $ g_{kl} \equiv \rho_{kl}^{-1} $ and define the angular factors:
\begin{equation}
  \omega^{\ur i} \defeq \frac{g_{i\ur}}{\sum_{j \in \prt{n}{m}(\ur)} g_{j\ur}}
\end{equation}
These clearly provide a $ \prt{n}{m}(\ur) $-partition of unity. Moreover:
\begin{equation*}
  \co_{j\ur} \omega^{\ur i} = \lim_{\rho_{j\ur} \rightarrow 0} \frac{g_{i\ur}}{\sum_{k \in \prt{n}{m}(\ur)} g_{k\ur}} = \bigg[ \lim_{\rho_{j\ur} \rightarrow 0} \sum_{k \in \prt{n}{m}(\ur)} \frac{\rho_{i\ur}}{\rho_{k\ur}} \bigg]^{-1} =
  \begin{cases}
    1^{-1} & j = i \\
    \infty^{-1} & j \neq i
  \end{cases}
  = \delta^i_j
\end{equation*}
where we made an abuse of notation. This completes the proof.

\section{Quadratic Casimir operators of \texorpdfstring{$ \SUn{n_c} $}{SU(n)}}
\label{sec:cas-op}

To prove \eref{eq:cas-sun}, first consider the fundamental representation $ \mathtt{n} $ of $ \SUn{n_c} $.
Then, contracting \eref{eq:quad-cas} with $ \delta^{ab} $ (with $ a,b = 1,\dots, n^2 - 1 $, as they label the basis of $ \mathfrak{su}(n_c) $):
\begin{equation*}
  C_2(\mathtt{n}) n_c = \frac{1}{2} (n_c^2 - 1)
\end{equation*}
To compute the Casimir operator for the adjoint representation $ \mathtt{g} $, consider the decomposition of the direct product of two representations:
\begin{equation*}
  \mathtt{r}_1 \otimes \mathtt{r}_2 = \bigoplus_i \mathtt{r}_i
\end{equation*}
In this representation $ T^a_{\mathtt{r}_1 \otimes \mathtt{r}_2} = T^a_{\mathtt{r}_1} \otimes \id_{\mathtt{r}_2} + \id_{\mathtt{r}_1} \otimes T^a_{\mathtt{r}_2} $, and it acts on tensor objects $ \Xi_{pq} $ whose first index transforms according to $ \mathtt{r}_1 $ and the second index according to $ \mathtt{r}_2 $. Recalling that $ \tr{T^a} = 0 $:
\begin{equation*}
  \begin{split}
    \tr (T^a_{\mathtt{r}_1 \otimes \mathtt{r}_2})^2
    &= \tr ((T^a_{\mathtt{r}_1})^2 \otimes \id_{\mathtt{r}_2} + 2 T^a_{\mathtt{r}_1} \otimes T^a_{\mathtt{r}_2} + \id_{\mathtt{r}_1} \otimes (T^a_{\mathtt{r}_2})^2) \\
    &= \tr (C_2(\mathtt{r}_1) \id_{\mathtt{r}_1} \otimes \id_{\mathtt{r}_2}) + \tr (C_2(\mathtt{r}_2) \id_{\mathtt{r}_1} \otimes \id_{\mathtt{r}_2}) = (C_2(\mathtt{r}_1) + C_2(\mathtt{r}_2)) n_{\mathtt{r}_1} n_{\mathtt{r}_2}
  \end{split}
\end{equation*}
However, by the decomposition above:
\begin{equation*}
  \tr (T^a_{\mathtt{r}_1 \otimes \mathtt{r}_2})^2 = \sum_i C_2(\mathtt{r}_i) n_{\mathtt{r}_i}
\end{equation*}
Consider $ \mathtt{n} \otimes \mathtt{n}^* $, where $ \mathtt{n}^* $ is the complex conjugate of the fundamental representation (for complex representations, $ \mathtt{r} $ and $ \mathtt{r}^* $ are generally inequivalent representations): then $ \Xi_{pq} $ contains a term proportional to the invariant $ \delta_{pq} $, while the other $ n_c^2 - 1 $  independent components transform as a general $ n_c \times n_c $ traceless tensor, i.e. under the adjoint representation of $ \SUn{n_c} $ (as of \eeref{eq:sun-herm}{eq:sun-trace}), thus $ \mathtt{n} \otimes \mathtt{n}^* = \ve{1} \oplus \mathtt{g} $ and the above identity becomes:
\begin{equation*}
  (C_2(\ve{1}) + C_2(\mathtt{g})) (n_c^2 - 1) = (C_2(\mathtt{n}) + C_2(\mathtt{n}^*)) n_c^2
\end{equation*}
Using $ C_2(\ve{1}) = 0 $ (as all generators are trivially zero) and $ C_2(\mathtt{n}^*) = C_2(\mathtt{n}) $:
\begin{equation*}
  C_2(\mathtt{g}) (n_c^2 - 1) = \frac{n_c^2 - 1}{n_c} n_c^2
\end{equation*}
which completes the proof.

\section{Pole coefficients}
\label{sec:poles}

In this section, we derive an explicit expression for the $ \ci{i,j} $ coefficients defined in \secref{sec:int-count}. Their expression only depends on whether $ i $ and $ j $ are massive--massive, massive--massless or massless--massless.

\paragraph{Massive--massive}

The needed coefficients are for $ \eit_{i,j} $ and $ \ns_{i,j} $ in the massive--massive case are listed in \ceref{eq:i-mm}{eq:n-mm}, so:
\begin{equation*}
  \ci{i,j} = \frac{1}{\de} \left( \frac{1}{2} \frac{1}{v_{ij}} \log \frac{1 + v_{ij}}{1 - v_{ij}} + \frac{1}{2} \frac{1}{v_{ij}} \log \frac{1 - v_{ij}}{1 + v_{ij}} \right)
\end{equation*}
Hence, trivially:
\begin{equation}
  \ci{i,j} = 0
  \label{eq:ci-mm}
\end{equation}

\paragraph{Massive--massless}

In the case of $ i $ massive and $ j $ massless, we use \ceref{eq:i-mnm}{eq:n-mnm}:
\begin{equation*}
  \begin{split}
    \ci{i,j}
    & = \frac{1}{\de} \left( \frac{1}{2} \log \frac{(\mm{p}_i \cdot \mm{p}_j)^2}{\mm{p}_i \cdot \mm{p}_i} + \frac{1}{2} \log \frac{m_i^2}{\abs{s_{ij}}} + \log \frac{2\ema}{\mu} + \frac{1}{2} \log \frac{\mu^2}{\abs{s_{ij}}} \right) \\
    & = \frac{1}{2\de} \left( \log \frac{4 E_i^2 \eta_{ij}^2}{m_i^2} + \log \frac{m_i^2}{4 E_i E_j \abs{\eta_{ij}}} + \log \frac{4\ema^2}{\mu^2} + \log \frac{\mu^2}{4 E_i E_j \abs{\eta_{ij}}} \right) = \frac{1}{2\de} \log \frac{\ema^2}{E_j^2}
  \end{split}
\end{equation*}
Thus:
\begin{equation}
  \ci{i,j} = \frac{1}{\de} \leg_j
  \label{eq:ci-mnm}
\end{equation}
The case of $ i $ massless and $ j $ massive it trivially found with $ i \leftrightarrow j $.

\paragraph{Massless--massless}

Finally, for the massless--massless case we have \ceref{eq:i-nmnm}{eq:n-nmnm}, i.e.:
\begin{equation*}
  \begin{split}
    \ci{i,j}
    & = \frac{1}{\de^2} \left( - \eta_{ij} \,\hyp(1,1,1-\de,1-\eta_{ij}) + 1 \right) + \frac{1}{\de} \left( 2 \eta_{ij} \,\hyp(1,1,1-\de,1-\eta_{ij}) \log \frac{2\ema}{\mu} + \log \frac{\mu^2}{\abs{s_{ij}}} \right) \\
    & = \frac{1}{\de} \left( \log \abs{\eta_{ij}} + \log \frac{4\ema^2}{\mu^2} + \log \frac{\mu^2}{4 E_i E_j \abs{\eta_{ij}}} \right) \underbrace{- \lmx \log \abs{\eta_{ij}} - \frac{1}{2} \left( \log^2 \abs{\eta_{ij}} + 2 \li(1 - \eta_{ij}) \right)}_{\finci{i,j}}
  \end{split}
\end{equation*}
where we used the expansion:
\begin{equation}
  \eta \,\hyp(1,1,1-\de,1-\eta) = 1 - \de \log \abs{\eta} - \frac{1}{2} \left( \log^2 \abs{\eta} + 2 \li(1-\eta) \right) \de^2 + \smo(\de^3)
\end{equation}
Therefore:
\begin{equation}
  \ci{i,j} = \frac{1}{\de} \left( \leg_i + \leg_j \right) + \finci{i,j} + \smo(\de)
  \label{eq:ci-nmnm}
\end{equation}
We extend the definition of $ \finci{i,j} $ so that $ \finci{i,j} \equiv 0 $ if either $ i $ or $ j $ (or both) is massive.










