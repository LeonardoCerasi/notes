\selectlanguage{english}

\section{Phase-space parametrization}

In dimensional regularization with $ d = 4 - 2 \de $, we define the measure on the phase space of a parton $ i $ to be:
\begin{equation}
  [\dd p_i] \equiv \frac{\dd^{d-1} p_i}{(2\pi)^{d-1} 2E_i} \theta(\ema - E_i)
\end{equation}
Note that $ \ema $ is an upper bound on the energies of individual partons: it is an arbitrary parameter to be taken sufficiently large as to be greater or equal to the maximal energy that a final-state parton can reach.

This measure can be cast in a more useful form introducing a suitable parametrization of the phase space: in particular, given that $ \R^n-\{\ve{0}\} \cong \R^+ \times \mathbb{S}^{n-1} $, it is convenient to introduce hyperspherical coordinates on the $ \mathbb{S}^{d-2} $ component of the phase space.

\begin{observation}{Hyperspherical coordinates}{}
  In general, $ \mathbb{S}^n $ can be described as a surface embedded in $ \R^{n+1} $, defined by:
  \begin{equation}
    \sum_{i = 1}^{n+1} x_i^2 = 1
  \end{equation}
  Therefore, $ \mathbb{S}^n $ can be parametrized by \bcobs{hyperspherical coordinates} $ \{\varphi_1 , \dots, \varphi_{n-1}, \varphi_n\} \subset [0,\pi]^{n-1} \times [0,2\pi) $ as:
  \begin{equation}
    \begin{split}
      x_1 &= \cos \varphi_1 \\
      x_2 &= \sin \varphi_1 \cos \varphi_2 \\
      x_3 &= \sin \varphi_1 \sin \varphi_2 \cos \varphi_3 \\
          &\,\,\,\vdots \\
      x_{n-1} &= \sin \varphi_1 \dots \sin \varphi_{n-2} \cos \varphi_{n-1} \\
      x_n &= \sin \varphi_1 \dots \sin \varphi_{n-2} \sin \varphi_{n-1} \cos \varphi_n \\
      x_{n+1} &= \sin \varphi_1 \dots \sin \varphi_{n-2} \sin \varphi_{n-1} \sin \varphi_n
    \end{split}
  \end{equation}
  It is then possible to define the hyperspherical measure on $ \mathbb{S}^n $:
  \begin{equation}
    \dd \Omega_n \equiv \sin^{n-1} \varphi_1 \sin^{n-2} \varphi_2 \dots \sin^2 \varphi_{n-2} \sin \varphi_{n-1} \, \dd \varphi_1 \dd \varphi_2 \dots \dd \varphi_{n-2} \dd \varphi_{n-1} \dd \varphi_n
  \end{equation}
  It is clear that this is a recursive relation, as:
  \begin{equation}
    \dd \Omega_n = \sin^{n-1} \varphi \, \dd \varphi \, \dd \Omega_{n-1}
    \label{eq:hyp-rec}
  \end{equation}
\end{observation}

Using \eref{eq:hyp-rec} (with $ \sin \varphi \, \dd \varphi = \dd \cos \varphi $), we can express the measure $ \dd^{d-1} p_i $ as:
\begin{equation}
  \dd^{d-1} p_i = \abs{\ve{p}_i}^{d-2} \, \dd \abs{\ve{p}_i} \, \sin^{d-4} \varphi \, \dd \cos \varphi \, \dd \Omega_{d-2}
\end{equation}
As we are only interested in integrations on phase spaces of real unresolved partons, which can only be massless, we can use the on-shell condition $ p_i^2 = 0 $ to express $ \abs{\ve{p}_i} = E_i $, so that the phase-space measure becomes:
\begin{equation}
  [\dd p_i] = \theta(\ema - E_i) E_i^{d-3} \dd E_i \, \sin^{d-4} \varphi \, \dd \cos \varphi \, \frac{\dd \Omega_{d-2}}{2(2\pi)^{d-1}}
\end{equation}
with $ E_i \in \R^+_0 $ and $ \varphi \in [0,\pi] $.

\subsection{Multi-particle phase space}
\label{ssec:multi-part-ps}

When considering scattering processes, in general the final state is a multi-particle state, hence the measure on the final-state phase space must account for energy conservation too.

\begin{theorem}{Scattering cross-section}{}
  Given a $ 2 \rightarrow m $ scattering process with well-defined initial momenta $ p_\mathcal{A} $ and $ p_\mathcal{B} $, then the \bcth{differential cross-section} is:
  \begin{equation}
    \dd\sigma = \frac{1}{2E_\mathcal{A} 2E_\mathcal{B} \abs{\ve{v}_\mathcal{A} - \ve{v}_\mathcal{B}}} \prod_{k = 1}^m \int \frac{\dd^3p_k}{(2\pi)^3 2E_k} \abs{\mathcal{M}(\mathcal{A} \mathcal{B} \rightarrow \{f\})}^2 (2\pi)^4 \delta^{(4)}(p_\mathcal{A} + p_\mathcal{B} - \textstyle\sum_{i = 1}^m p_i)
    \label{eq:scatt-cr-sec}
  \end{equation}
  where $ \mathcal{M}(\mathcal{A} \mathcal{B} \rightarrow \{f\}) $ is the matrix element of the scattering process and $ \ve{v}_k \equiv \frac{\ve{p}_k}{E_k} $ is the velocity of the $ k^\text{th} $ particle.
\end{theorem}

\begin{proofbox}
  \begin{proof}
    See Chapter 4 of \cite{Peskin-1995}.
  \end{proof}
\end{proofbox}

As we are only interested in massless initial-state partons, in the center-of-mass (CM) frame $ p_{\mathcal{A},\mathcal{B}} = (E, \pm\ve{p}) $, hence it is trivial to see that the flux factor in \eref{eq:scatt-cr-sec} is just $ 2\hat{s} \defeq 2 (p_\mathcal{A} + p_\mathcal{B})^2 $. The differential cross-section can then be rewritten as:
\begin{equation}
  \dd \sigma = \frac{1}{2\hat{s}} \int \lps_m \abs{\mat(\mathcal{A}\mathcal{B} \rightarrow \{f\})}^2
\end{equation}
where the \bctxt{invariant $ m $-body phase space measure} is defined as:
\begin{equation}
  \lps_m \equiv \prod_{k = 1}^m [\dd p_k] (2\pi)^4 \delta^{(4)}(p_\mathcal{A} + p_\mathcal{P} - \sum_{i = 1}^m p_i)
\end{equation}

\section{Angular integrals}










