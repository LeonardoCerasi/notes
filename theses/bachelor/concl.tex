\selectlanguage{english}

The generalized soft and virtual operators \ceref{eq:iso-gen}{eq:ivo-gen}, as well as the $ \sim \smo(\de^0) $ integrated counterterms \eref{eq:isvo-zero}, constitute the main results of this work. Indeed, they correctly account for the presence of generic massive partons in the final state: as expected, the integrated counterterms contain terms of the kind $ \sim \log m^2 / \mu^2 $ (in particular, $ \fing{i} $ and $ \finm{i} $ have this explicit logarithmic dependence on quark masses, while $ \finc{i,j} $ contains these terms in the massive--massive and massive--massless $ \ns_{i,j} $ coefficients).

These results pave the way for the inclusion of massive final-state partons in NLO calculations performed within the NSC subtraction framework. The explicit form of the counterterms is reported in \secref{sec:gen-count}: $ \fing{i} $ and $ \finm{i} $ are expressed in a form which can be readily integrated on the LO-like multi-parton final-state phase space through Monte Carlo techniques, while $ \finc{i,j} $ still appears in a colour-correlated sum in the final $ \sim \smo(\de^0) $ Laurent coefficient. A further improvement of these results would then be the (partial) expression of the colour-correlated sum in \eref{eq:isvo-zero} as a function of Casimir operators (i.e. $ \sim \tco_i^2 $ terms), in order to facilitate its numerical integration: indeed, integrating colour-correlated terms requires both more effort in writing code to handle colour correlation and more computing time to run it.

Moreover, this work lays the foundations for a future generalization of the NSC SS with massive partons at NNLO: as explained in \cite{rontsch-2509}, many of the singularities encountered in NNLO calculations are captured by the $ \iso $, $ \ico $ and $ \ivo $ operators or by their iterations, hence having already generalized their expressions to include massive final-state partons should significantly reduce the work needed for a generalization of the NNLO calculation including massive partons. In particular, since final-state massive partons do not affect collinear limits or the collinear renormalization of PDFs, and given that double-virtual corrections at $ \smo(\de^{-2}) $ can be written in terms of $ \ivo(\de) $, $ \ivo(2\de) $ and $ [\ioo(\de) , \ioo\dg(\de)] $ (see \cite{Catani-1998}), the only difficulty left is generalizing the various soft limits at NNLO to include both massless and massive partons, as we did in \secref{ssec:soft-gen}. Indeed, work in this direction has already begun: see e.g. \cite{Horstmann-2025, Long-2025}.

Finally, we note that we only considered final-state massive partons in this work. The reason for this limitation is that the inclusion of initial-state massive partons is currently poorly understood: indeed, parton distribution functions are generally defined for massless partons, but their definition does not strictly require this condition, hence the factorization theorem \eref{eq:fact-th} is still valid in presence of initial-state massive partons. However, the collinear renormalization of PDFs changes, as the masses of initial-state partons regulate the corresponding collinear divergences: fixed-order diagrams that would be collinearly divergent for $ m \rightarrow 0 $ instead produce a logarithmic dependence in the form of powers $ \log^n Q^2 / m^2 $, with $ Q $ the energy scale of the hard process. We see that, for $ Q \gg m $, these logarithms can spoil the convergence of the perturbative expansion \eref{eq:part-ser-exp}, as each term is now $ \sim \rc^n \log^n Q^2 / m^2 $: the best strategy in this case would be the resummation of massive logarithms, which is an area of active research.
