\selectlanguage{english}

In this Appendix, we provide definitions of relevant objects used in this work. To simply various formulas, we use a notation analogous to \cite{rontsch-2503}:
\begin{equation}
  \begin{gathered}
    \ol{z} \equiv 1 - z
    \qquad \qquad
    \mathcal{D}_n(z) \equiv \pd{\frac{\log^n(1-z)}{1-z}}
    \\
    L_i \equiv \log \frac{\ema}{E_i}
    \qquad \qquad
    \mathscr{L}_i \equiv \log \frac{2E_i}{\ren}
    \qquad \qquad
    L_\text{max} \equiv \log \frac{2\ema}{\ren}
  \end{gathered}
\end{equation}

\section{Useful constants}

Denoting the colour-charge operators $ \ve{T}_i $, with the conventional normalization $ \ttr = \frac{1}{2} $ for $ \SUn{n_c} $, the squares of these operators are the quadratic Casimir operators of the corresponding representations:
\begin{equation}
  \ve{T}_q^2 = \ve{T}_{\bar{q}}^2 = \caf = \frac{n_c^2 - 1}{2n_c}
  \qquad \qquad
  \ve{T}_g^2 = \caa = n_c
\end{equation}
The quark and gluon anomalous dimensions are:
\begin{equation}
  \gamma_q = \frac{3}{2} \caf
  \qquad \qquad
  \gamma_q = \frac{11}{6} \caa - \frac{2}{3} \ttr n_q
\end{equation}
where $ n_q $ is the number of active flavours.

The strong coupling is renormalized in the $ \msb $ scheme, so that the bare and running couplings are related by:
\begin{equation}
  \bc S_\de = \rcr \ren^{2\de} \left[ 1 - \frac{\rcr}{2\pi} \frac{\beta_0}{\de} + \smo(\rc^2) \right]
\end{equation}
where $ S_\de \equiv (4\pi)^\de e^{-\eg \de} $ and:
\begin{equation}
  \beta_0 = \frac{11}{6} \caa - \frac{2}{3} \ttr n_q = \gamma_g
\end{equation}
It is convenient to define a quantity related to the coupling constant:
\begin{equation}
  [\rc] \equiv \frac{\rcr}{2\pi} \frac{e^{\eg \de}}{\Gamma(1-\de)}
\end{equation}

\section{Splitting functions}

THIS TO BE PARTIALLY MOVED TO SECTION ON COLLINEAR SINGULARITIES

Consider the final-state splitting process $ [i\ur]^* \rightarrow i(z) + \ur(1-z) $, where $ i $ and $ \ur $ are two partons of flavours $ f_i $ and $ f_\ur $ and $ [i\ur] $ is the corresponding clustered parton of flavour $ f_{[i\ur]} $. Recall that, given the interaction vertices determined by the QCD Lagrangian \eref{eq:qcd-lag} (see FIGURE), a gluon clustered with any type of parton preserves the latter's flavours, while a quark clustered with an antiquark gives a gluon.

The energy fraction carried by the parton $ i $ is defined as $ z \equiv 1 - E_\ur / E_{[i\ur]} $. As a consequence, the parton $ \ur $ carries an energy fraction $ 1 - z $. Denoting the spin-averaged fintal-state splitting functions as $ P_{f_{[i\ur]}f_i}(z) $, they read:
\begin{align}
  P_{qq}(z) & = \caf \left[ \frac{1 + z^2}{1 - z} - \de (1 - z) \right] \\
  P_{qg}(z) & = \caf \left[ \frac{1 - (1 - z)^2}{z} - \de z \right] \equiv P_{qq}(1 - z) \\
  P_{gq}(z) & = \ttr \left[ 1 - \frac{2z (1 - z)}{1 - \de} \right] \\
  P_{gg}(z) & = 2 \caa \left[ \frac{z}{1 - z} + \frac{1 - z}{z} + z (1 - z) \right]
\end{align}
Now, consider instead the initial-state splitting process $ i \rightarrow [i\ur]^* + \ur $, where $ i $ and $ \ur $ are respectively an ingoing and outgoing parton, while the clustered parton $ [i\ur]^* $ enters the hard scattering process. In this case, we define the $ z $ variable as $ z \equiv 1 - E_\ur / E_i $. The spin- and color-averaged intial-state splitting functions, denoted as $ P_{f_{[i\ur]}f_i, \text{i}}(z) $, are:
\begin{align}
  P_{qq,\text{i}} & = - z P_{qq}(1/z) \equiv P_{qq}(z) \\
  P_{qg,\text{i}} & = \left[ \frac{2n_c}{2 (1 - \de) (n_c^2 - 1)} \right] z P_{qg}(1/z) \equiv P_{gq}(z) \\
  P_{gq,\text{i}} & = \left[ \frac{2 (1 - \de) (n_c^2 - 1)}{2n_c} \right] z P_{gq}(1/z) \equiv P_{qg}(z) \\
  P_{gg,\text{i}} & = - z P_{gg}(1/z) \equiv P_{gg}(z)
\end{align}
Finally, the LO Altarelli-Parisi splitting kernels are:
\begin{align}
  \hat{P}^{(0)}_{qq}(z) & = \caf \left[ 2 \mathcal{D}_0(z) - (1 + z) + \frac{3}{2} \delta(1 - z) \right] \\
  \hat{P}^{(0)}_{qq}(z) & = \ttr \left[ (1 - z)^2 + z^2 \right] \\
  \hat{P}^{(0)}_{qq}(z) & = \caf \left[ \frac{1 + (1 - z)^2}{z} \right] \\
  \hat{P}^{(0)}_{qq}(z) & = 2 \caa \left[ \mathcal{D}_0(z) + z (1 - z) + \frac{1}{z} - 2 \right] + \beta_0 \delta(1 - z)
\end{align}
All these splitting functions and kernels can be found in \cite{Ellis-1996}.










