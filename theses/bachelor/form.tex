\selectlanguage{english}

In this Appendix, we provide definitions of relevant objects used in this work. To simply various formulas, we use a notation analogous to \cite{rontsch-2503}:
\begin{equation}
  \begin{gathered}
    \ol{z} \equiv 1 - z
    \qquad \qquad
    \mathcal{D}_n(z) \equiv \pd{\frac{\log^n(1-z)}{1-z}}
    \\
    \leg_i \equiv \log \frac{\ema}{E_i}
    \qquad \qquad
    \ler_i \equiv \log \frac{2E_i}{\ren}
    \qquad \qquad
    \lmx \equiv \log \frac{2\ema}{\ren}
  \end{gathered}
  \label{eq:log-short}
\end{equation}
Moreover, we adopt the following short-hands:
\begin{equation}
  \rho_{ij} \equiv 1 - \cos \theta_{ij}
  \qquad \qquad
  \eta_{ij} \equiv \frac{\rho_{ij}}{2}
  \label{eq:short-hand}
\end{equation}
where $ i $ and $ j $ are two partons.

\section{Useful constants}

Denoting the colour-charge operators $ \ve{T}_i $, with the conventional normalization $ \ttr = \frac{1}{2} $ for $ \SUn{n_c} $, the squares of these operators are the quadratic Casimir operators of the corresponding representations:
\begin{equation}
  \ve{T}_q^2 = \ve{T}_{\bar{q}}^2 = \caf = \frac{n_c^2 - 1}{2n_c}
  \qquad \qquad
  \ve{T}_g^2 = \caa = n_c
\end{equation}
The quark and gluon anomalous dimensions are:
\begin{equation}
  \gamma_q = \frac{3}{2} \caf
  \qquad \qquad
  \gamma_q = \frac{11}{6} \caa - \frac{2}{3} \ttr n_q
\end{equation}
where $ n_q $ is the number of active flavours.

The strong coupling is renormalized in the $ \msb $ scheme, so that the bare and running couplings are related by:
\begin{equation}
  \bc S_\de = \rcr \ren^{2\de} \left[ 1 - \frac{\rcr}{2\pi} \frac{\beta_0}{\de} + \smo(\rc^2) \right]
\end{equation}
where $ S_\de \equiv (4\pi)^\de e^{-\eg \de} $ and:
\begin{equation}
  \beta_0 = \frac{11}{6} \caa - \frac{2}{3} \ttr n_q = \gamma_g
\end{equation}
It is convenient to define a quantity related to the running coupling:
\begin{equation}
  [\rc] \equiv \frac{\rcr}{2\pi} \frac{e^{\eg \de}}{\Gamma(1-\de)}
\end{equation}

\section{Splitting functions}
\label{sec:spl-func}

Consider the final-state splitting process $ [i\ur]^* \rightarrow i(z) + \ur(1-z) $, where $ i $ and $ \ur $ are two partons of flavours $ f_i $ and $ f_\ur $ and $ [i\ur] $ is the corresponding clustered parton of flavour $ f_{[i\ur]} $. Recall that, given the interaction vertices determined by the QCD Lagrangian \eref{eq:qcd-lag} (see \secref{ssec:qcd-quant}), a gluon clustered with any type of parton preserves the latter's flavours, while a quark clustered with an antiquark gives a gluon.

The energy fraction carried by the parton $ i $ is defined as $ z \equiv 1 - E_\ur / E_{[i\ur]} $. As a consequence, the parton $ \ur $ carries an energy fraction $ 1 - z $. Denoting the spin-averaged final-state splitting functions as $ P_{f_{[i\ur]}f_i}(z) $, they read:
\begin{align}
  \spl_{qq}(z) & = \caf \left[ \frac{1 + z^2}{1 - z} - \de (1 - z) \right] \label{eq:spl-1} \\
  \spl_{qg}(z) & = \caf \left[ \frac{1 - (1 - z)^2}{z} - \de z \right] \equiv P_{qq}(1 - z) \\
  \spl_{gq}(z) & = \ttr \left[ 1 - \frac{2z (1 - z)}{1 - \de} \right] \\
  \spl_{gg}(z) & = 2 \caa \left[ \frac{z}{1 - z} + \frac{1 - z}{z} + z (1 - z) \right] \label{eq:spl-4}
\end{align}
Now, consider instead the initial-state splitting process $ i \rightarrow [i\ur]^* + \ur $, where $ i $ and $ \ur $ are respectively an ingoing and outgoing parton, while the clustered parton $ [i\ur]^* $ enters the hard scattering process. In this case, we define the $ z $ variable as $ z \equiv 1 - E_\ur / E_i $. The spin- and color-averaged intial-state splitting functions, denoted as $ P_{f_{[i\ur]}f_i, \text{i}}(z) $, are:
\begin{align}
  \spl_{qq,\text{i}} & = - z P_{qq}(1/z) \equiv P_{qq}(z) \label{eq:spli-1} \\
  \spl_{qg,\text{i}} & = \left[ \frac{2n_c}{2 (1 - \de) (n_c^2 - 1)} \right] z P_{qg}(1/z) \equiv P_{gq}(z) \\
  \spl_{gq,\text{i}} & = \left[ \frac{2 (1 - \de) (n_c^2 - 1)}{2n_c} \right] z P_{gq}(1/z) \equiv P_{qg}(z) \\
  \spl_{gg,\text{i}} & = - z P_{gg}(1/z) \equiv P_{gg}(z) \label{eq:spli-4}
\end{align}
Finally, the LO Altarelli-Parisi splitting kernels are:
\begin{align}
  \ap_{qq}(z) & = \caf \left[ 2 \mathcal{D}_0(z) - (1 + z) + \frac{3}{2} \delta(1 - z) \right] \\
  \ap_{qq}(z) & = \ttr \left[ (1 - z)^2 + z^2 \right] \\
  \ap_{qq}(z) & = \caf \left[ \frac{1 + (1 - z)^2}{z} \right] \\
  \ap_{qq}(z) & = 2 \caa \left[ \mathcal{D}_0(z) + z (1 - z) + \frac{1}{z} - 2 \right] + \beta_0 \delta(1 - z)
\end{align}
All these splitting functions and kernels can be found in \cite{Ellis-1996}.

\section{Massive angular integrals}
\label{sec:mass-int}

As in \cite{Behring-2020}, we can expand the $ \eit_{i,j} $ integral defined in \eref{eq:gen-int} in its Laurent series in $ \de $:
\begin{equation}
  \eit_{i,j} = \sum_{k = -1}^1 \de^k \eitt{i,j}{k} + \smo(\de^2)
\end{equation}
The expression of the coefficients $ \eitt{i,j}{k} $ depends on the nature of the partons $ i $ and $ j $.

\paragraph{Massive--massive}

If $ i $ and $ j $ are both massive, then $ p_i^2 = m_i^2 $ and $ p_j^2 = m_j^2 $. Set the notation:
\begin{equation}
  v_{ij} \equiv \sqrt{1 - \frac{m_i^2 m_j^2}{(p_i \cdot p_j)^2}}
  \qquad \qquad
  \ve{q}_i \equiv \frac{\ve{p}_i}{E_i}
  \qquad \qquad
  \ve{q}_j \equiv \frac{\ve{p}_j}{E_j}
  \label{eq:v-def}
\end{equation}
Moreover, define the shorthands:
\begin{align*}
  A^2 &= (\ve{q}_i - \ve{q}_j)^2 & X_1 &= \ve{q}_i^2 - (\ve{q}_i \cdot \ve{q}_j) \\
  B^2 &= \ve{q}_i^2 \ve{q}_j^2 - (\ve{q}_i \cdot \ve{q}_j)^2 & X_2 &= \ve{q}_j^2 - (\ve{q}_i \cdot \ve{q}_j)
\end{align*}
and the arguments:
\begin{equation*}
  \gamma_{\pm} = A \pm \sqrt{A^2 - B^2}
  \qquad \qquad
  z_k = \sqrt{X_k^2 + B^2} - X_k
\end{equation*}
used in the function:
\begin{equation*}
  K(z) = - 2 \li \left( \frac{2\gamma_- (\gamma_+ - z)}{(\gamma_+ - \gamma_-) (\gamma_- + z)} \right) - 2 \li \left( \frac{-2\gamma_+(\gamma_- + z)}{(\gamma_+ - \gamma_-) (\gamma_+ - z)} \right) - \frac{1}{2} \log^2 \frac{(z - \gamma_-) (\gamma_+ - z)}{(z + \gamma_-) (\gamma_+ + z)}
\end{equation*}
With this notation, the coefficients $ \eitt{i,j}{k} $ in the massive--massive case read:
\begin{equation}
  \eitt{i,j}{-1} = 0
  \qquad \qquad
  \eitt{i,j}{0} = \frac{1}{v_{ij}} \log \frac{1 + v_{ij}}{1 - v_{ij}}
  \qquad \qquad
  \eitt{i,j}{1} = \frac{1 - (\ve{q}_i \cdot \ve{q}_j)}{\sqrt{A^2 - B^2}} \left[ K(z_2) - K(z_1) \right]
  \label{eq:i-mm}
\end{equation}

\paragraph{Massive--massless}

As $ \eit_{i,j} = \eit_{j,i} $, consider WLOG $ i $ massive and $ j $ massless, i.e. $ p_i^2 = m_i^2 $ and $ p_j^2 = 0 $. In this case, we define:
\begin{equation}
  \kappa \equiv \sqrt{1 - \frac{m_i^2}{E_i}}
  \qquad \qquad
  \mm{p}_i \equiv \frac{p_i}{E_i}
  \qquad \qquad
  \mm{p}_j \equiv \frac{p_j}{E_j}
  \label{eq:kappa-def}
\end{equation}
Then, the coefficients $ \eitt{i,j}{k} $ in the massive--massless case read:
\begin{equation}
  \begin{gathered}
    \eitt{i,j}{-1} = -1
    \qquad \qquad
    \eitt{i,j}{0} = \log \frac{(\mm{q}_i \cdot \mm{q}_j)^2}{\mm{q}_i \cdot \mm{q}_j}
    \\
    \eitt{i,j}{1} = -2 \left[ \frac{1}{4} \log^2 \frac{1 - \kappa}{1 + \kappa} + \log \frac{\mm{q}_i \cdot \mm{q}_j}{1 + \kappa} \log \frac{\mm{q}_i \cdot \mm{q}_j}{1 - \kappa} + \li \left( 1 - \frac{\mm{q}_i \cdot \mm{q}_j}{1 + \kappa} \right) + \li \left( 1 - \frac{\mm{q}_i \cdot \mm{q}_j}{1 - \kappa} \right) \right]
  \end{gathered}
  \label{eq:i-mnm}
\end{equation}

\paragraph{Self-correlated massive}

In the special case $ i = j $, if $ i $ is massless then $ \eit_{i,i} = 0 $ trivially from the definition of the eikonal factor, while if $ i $ is massive then the coefficients $ \eitt{i,i}{k} $ read:
\begin{equation}
  \eitt{i,i}{-1} = 0
  \qquad \qquad
  \eitt{i,i}{0} = 2
  \qquad \qquad
  \eitt{i,i}{1} = - \frac{2}{\kappa} \log \frac{1 - \kappa}{1 + \kappa}
  \label{eq:i-self}
\end{equation}
where $ \kappa $ is defined in \eref{eq:kappa-def}.

\section{Generalized Catani's functions}
\label{sec:cat-fun}

In the generalized Catani's operator \eref{eq:gen-cat} two divergent functions appear. Consider first the $ \ns_{i,j}(\de) $ functions. If $ i $ and $ j $ are both massive, then:
\begin{equation}
  \ns_{i,j}(\de) = \frac{1}{2\de} \frac{1}{v_{ij}} \log \frac{1 - v_{ij}}{1 + v_{ij}} - \frac{1}{4} \left( \log^2 \frac{m_i^2}{\abs{s_{ij}}} + \log^2 \frac{m_j^2}{\abs{s_{ij}}} \right) - \frac{\pi^2}{6}
  \label{eq:n-mm}
\end{equation}
If $ i $ is massive and $ j $ is massless, instead:
\begin{equation}
  \ns_{i,j}(\de) = \frac{1}{2\de^2} + \frac{1}{2\de} \log \frac{m_i^2}{\abs{s_{ij}}} - \frac{1}{4} \log^2 \frac{m_i^2}{\abs{s_{ij}}} - \frac{\pi^2}{12}
  \label{eq:n-mnm}
\end{equation}
Finally, it $ i $ and $ j $ are both massless, $ \ns_{i,j}(\de) $ reduces to:
\begin{equation}
  \ns_{i,j}(\de) = \frac{1}{\de^2}
  \label{eq:n-nmnm}
\end{equation}
Now, consider the $ \Gamma_i(\de) $ functions. They only depend on whether $ i $ is a gluon, a massless quark or a massive quark, and the corresponding expressions are:
\begin{equation}
  \Gamma_g(\de) = \frac{1}{\de} \gamma_g - \frac{2}{3} \ttr \sum_{\rho = 1}^{n_F} \log \frac{m_{\mq_\rho}^2}{\mu^2}
  \label{eq:g-g}
\end{equation}
\begin{equation}
  \Gamma_q = \frac{1}{\de} \gamma_q
  \label{eq:g-q}
\end{equation}
\begin{equation}
  \Gamma_Q = \caf \left[ \frac{1}{\de} + \frac{1}{2} \log \frac{m_\mq^2}{\mu^2} - 2 \right]
  \label{eq:g-qm}
\end{equation}

\section{Integrated finite remainders}

We report the various finite remainders present in the $ \smo(\de^0) $ term of the Laurent series of $ \isvo(\de) $ (\eref{eq:isvo-zero}). The colour-correlated remainders, defined for $ i \neq j $, are:
\begin{multline}
  \finc{i,j} = \frac{1}{2} \eitt{i,j}{1} - \lmx \eitt{i,j}{0} + \left( \lmx^2 - \frac{\pi^2}{12} \right) \eitt{i,j}{-1} + \nss{i,j}{0} \\
  - \frac{1}{v_{ij}} \frac{\pi^2}{2} \theta(s_{ij}) + \nss{i,j}{-1} \log \frac{\mu^2}{\abs{s_{ij}}} + \frac{1}{2} \nss{i,j}{-2} \log^2 \frac{\mu^2}{\abs{s_{ij}}}
\end{multline}
\begin{equation}
  \finci{i,j} = - \lmx \log \abs{\eta_{ij}} - \frac{1}{2} \left( \log^2 \abs{\eta_{ij}} + 2 \li(1 - \eta_{ij}) \right)
\end{equation}
with the convention that $ \finci{i,j} \equiv 0 $ if either $ i $ or $ j $ (or both) is massive. Then, the other finite remainders are:
\begin{equation}
  \finm{i} = - \caf \left[ \frac{1}{\kappa_i} \log \frac{1 - \kappa_i}{1 + \kappa_i} + \frac{1}{2} \log \frac{m_{\mq_i}^2}{\mu^2} - 2 \right]
\end{equation}
which is defined for massive partons, and:
\begin{equation}
  \fing{i} = - \delta_{f_i , g} \frac{2}{3} \ttr \sum_{\rho = 1}^{n_F} \log \frac{m_{\mq_\rho}^2}{\mu^2}
\end{equation}
which is defined for massless partons.











