\selectlanguage{english}

% Draft abstract

% The treatment of infrared divergences in Next-to-Leading Order (NLO) QCD calculations becomes significantly more complex when accounting for massive quarks, particularly in processes where mass effects cannot be neglected. We present a generalization of the Nested Soft-Collinear (NSC) subtraction scheme to incorporate arbitrary massive quark flavours, preserving the original framework’s efficiency while systematically addressing mass-dependent divergences.

% By removing the need for massless approximations, this work enables precision calculations in particle-production processes where quark mass effects are theoretically or phenomenologically relevant.

Precise predictions of hadronic scattering processes in particle colliders are gaining increasing importance in high-energy physics, as they both allow for testing the Standard Model (SM) and for probing potential signals of Beyond-SM physics (BSM). Given this growing experimental focus on precision measurements, especially at facilities like the Large Hadron Collider at CERN, there is a compelling need for precise theoretical estimates of observables sensitive to Quantum Chromodynamics (QCD) effects. Achieving greater precision requires the correction of leading-order (LO) calculations with (next-to-)$ ^n $leading (N$ ^n $LO) terms: these are generally challenging to compute, in part because they need to account for both real and virtual corrections to the considered hard processes, which both present infrared (IR) singularities stemming from unresolved partons (i.e. soft and/or collinear to other partons).

Although these individual corrections are divergent, the IR singularities cancel between them, ensuring that the total cross-section is finite even when including N$ ^n $LO corrections. One of the main difficulties of computing higher-order corrections is making the cancellation manifest. To do so, it is necessary to introduce subtraction methods for the regularization and extraction of these singularities in a general and local way on the unresolved phase space.

While a general solution has been found at NLO, most notably with the Catani--Seymour (CS) and the Frixione--Kunszt--Signer (FKS) subtractions schemes (SS), the problem remains open at NNLO: in the last two decades, a variety of schemes have been proposed, but none has reached the same level of generality as the CS and FKS schemes. One of the most promising ones is the Nested Soft-Collinear (NSC) SS, given its modularity and conceptual clarity: this SS regularizes IR divergences by factorizing soft and collinear singularities in a nested and sequential manner. This modular structure is particularly suited for an extension from NLO to NNLO, as well as for an implementation in Monte Carlo integration codes.

A limitation of the NSC SS is that it adopts the massless approximation for all quark flavours: this hinders the applicability of this SS as-is to processes involving heavy quarks (i.e. bottom and top quarks), which constitute one of the main areas of active research at particle colliders. The aim of this thesis is generalizing the NSC SS to consider completely generic partonic final states at NLO, with a general number of massless and massive quark flavours, in order to accommodate for the description of heavy-quark processes.

The analysis conducted in this thesis shows that the structure of the NSC SS is preserved when including massive final-state partons. Indeed, these partons only modify the soft and virtual operators $ \iso(\de) $ and $ \ivo(\de) $, which factorize the divergences resulting from soft limits of real corrections and from virtual corrections respectively, while leaving the collinear operator $ \ico(\de) $ and the counterterms introduced by the collinear renormalization of PDFs unchanged. In particular, it is shown that $ \de^{-2} $-poles in the sum $ \iso(\de) + \ivo(\de) $ cancel for general partonic processes, while $ \de^{-1} $-poles are independent of massive partons and indeed cancel against the $ \de^{-1} $-poles of $ \ico(\de) $, thus yielding an $ \de $-finite reminder. Moreover, this finite reminder contains massive logs of the kind $ \sim \log m^2 / Q^2 $, with $ Q $ the hard energy scale, exactly as expected: this suggests that for studies at colliders like the LHC, with $ Q \sim 100\gev - 1\tev $, the numerical stability of Monte Carlo codes restricts the quark flavours which can be considered massive to only the top and bottom flavours ($ m_t \approx 172 \gev $, $ m_b \approx 4.18 \gev $), as smaller masses can cause numerical instability due to large logarithms.

This thesis therefore lays the foundations for a future inclusion of general massive final states in NNLO QCD computations, thereby paving the way for precision studies of heavy-quark processes with the NSC SS.










