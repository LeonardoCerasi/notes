\selectlanguage{english}

\section{Renormalization scheme}

The computation of NLO corrections to scattering processes often involves diverging loop amplitudes. In order to obtain finite results from these divergences, a renormalization scheme must be implemented.

As the generalized Catani's formula for virtual corrections is provided in \cite{Catani-2001} in a charge-unrenormalized (but mass-renormalized) way, it is necessary to carry out the renormalization procedureexplicitly. To this end, we formally state the renormalization scheme adopted in this work.

\subsection{Dimensional regularization}

In the evaluation of loop amplitudes, both UV- and IR-singularities are encountered. The most efficient way to simultaneously regularize both types of divergences is dimensional regularization, a regularization scheme first introduced by 't Hooft and Veltman in \cite{thooft-1972}.

In general, the dimensional regularization scheme consists in the analytic continuation of loop momenta to $ d = 4 - 2 \de $ dimensions, with $ \de \in \C : \Re\de < 0 $. This procedure turns loop integrals into meromorphic\footnotemark functions of $ \de \in \C $, allowing for the isolation of divergences as poles in $ \de $.

\footnotetext{Given an open set $ D \subset \C $, then $ f : D \rightarrow \C $ is \textit{meromorphic} if it is holomorphic on $ D - P $, where $ P \subset D $ is a set of isolated points called \textit{poles}. Recall that a function $ f : D \rightarrow \C $ is \textit{holomorphic} on $ D $ if it is complex differentiable at every point in $ D $.}

The dimensional regularization prescription leaves freedom in choosing the dimensionality of external momenta, as well as the number of polarizations of both external and internal particles, thus allowing for the definition of different regularization schemes. We choose to work with \bctxt{conventional dimensional regularization} (CDR), in which all momenta and polarization are analytically continued to $ d $ dimensions, as opposed to the 't Hooft--Veltman scheme (HV), in which only internal momenta and polarizations are.

When considering non-chiral gauge theories like QCD, CDR is the most natural choice, as the main difference between CDR and HV is the treatment of purely $ 4 $-dimensional objects, i.e. $ \gamma^5 $ and $ \epsilon_{\mu \nu \sigma \rho} $. In particular, in CDR both the Dirac algebra and Lorentz indices are analytically continued to $ d $ dimensions, leading to a mathematical inconsistency when $ d \neq 4 $.

\begin{observation}{Inconsistency in CDR}{}
  In $ 4 $-dimensional Minkowski space $ \R^{1,3} $ with metric signature $ \eta = (+,-,-,-) $, the Dirac algebra is defined as $ \dira \cong \clar{1,3} \otimes \C $ (complexification\footnotemark). This Clifford algebra admits a matrix representation with generators $ \{\gamma^\mu\}_{\mu = 0, 1, 2, 3} \subset \C^{4 \times 4} $ such that:
  \begin{equation}
    \{\gamma^\mu , \gamma^\nu\} = 2 \eta^{\mu \nu} \tens{I}_4
  \end{equation}
  As in CDR Lorentz indices too are $ d $-dimensional, the Dirac algebra becomes $ \clac{1,d-1} \simeq \clar{1,d-1} \otimes \C $, where usual Minkowski space $ \R^{1,3} $ is replaced by the pseudo-Euclidean space $ \R^{1,d-1} $ with metric signature $ \eta = (+,-,\dots,-) $. This structure, however, is ill-defined, as for $ d \in \N $ then $ \dim_\C \clac{1,d-1} = 2^d $, but a finite-dimensional algebra of dimension $ 2^d $ with $ d \notin \N $ is meaningless.

  To solve this issue, in CDR spinor indices remain $ 4 $-dimensional, i.e. we consider a matrix representation generated by $ \{\gamma^\mu\}_{\mu = 0, \dots, d-1} \subset \C^{4 \times 4} $ and impose the formal relations:
  \begin{equation}
    \{\gamma^\mu , \gamma^\nu\} = 2 \eta^{\mu \nu} \tens{I}_d
    \label{eq:gen-dir-alg}
  \end{equation}
  Consistency is achieved by analytical continuation of trace identities: indeed, traces obtained by recursively applying \eref{eq:gen-dir-alg} (like $ \tr\{\gamma^\mu \gamma^\nu\} = 4 \eta^{\mu \nu} $) are still valid in $ d $-dimensions, as the only dependence on dimension comes from contractions such as $ \tensor{\eta}{^\mu_\mu} = d $.

  A fatal inconsistency of CDR arises when considering $ \gamma^5 $. In $ \dira $, this matrix is defined as $ \gamma^5 \defeq \frac{i}{4!} \epsilon_{\mu \nu \rho \sigma} \gamma^\mu \gamma^\nu \gamma^\rho \gamma^\sigma = i \gamma^0 \gamma^1 \gamma^2 \gamma^3 $ and has the property $ \{\gamma^5 , \gamma^\mu\} = 0 \,\, \forall \mu = 0,1,2,3 $, which allows to prove this identity:
  \begin{equation}
    \tr\{\gamma^5 \gamma^\mu \gamma^\nu \gamma^\rho \gamma^\sigma\} = -4 i \epsilon^{\mu \nu \rho \sigma}
    \label{eq:top-form-gamma-5}
  \end{equation}
  This construction cannot be generalized consistently to $ d \notin \N $. To show this, assume a $ d $-dimensional generalization $ \gamma^5 \in \C^{4 \times 4} : \{\gamma^5 , \gamma^\mu\} = 0 \,\,\forall \mu = 0, \dots, d-1 $, so that $ \gamma^5 \gamma^{\mu_1} \dots \gamma^{\mu_n} = \left( -1 \right)^n \gamma^{\mu_1} \dots \gamma^{\mu_n} \gamma^5 $. By the cyclicity of the trace, then:
  \begin{equation}
    \left( 1 - \left( -1 \right)^n \right) \tr\{\gamma^5 \gamma^{\mu_1} \dots \gamma^{\mu_n}\} = 0
  \end{equation}
  Consider $ n = d $: clearly $ 1 - \left( -1 \right)^d = 1 - e^{i \pi d} \neq 0 $ for $ d \notin \N $, so $ \tr\{\gamma^5 \gamma^{\mu_1} \dots \gamma^{\mu_d}\} = 0 $. This is an open contradiction to \eref{eq:top-form-gamma-5}, as analytic continuation should continuously preserve the top product of the algebra as $ \de \rightarrow 0 $.

  This contradiction is the explicit manifestation of a more profound topological issue of analytically continuing the number of dimensions: the Levi-Civita symbol in $ d = 4 $ is linked\footnotemark to the Grassmann algebra $ \bigwedge(\R^{1,3}) $, and in particular to its top-form, but $ \bigwedge^k(\R^d) $ is only defined for $ d \in \N $, so the top exterior subspace $ \bigwedge^d(\R^{1,d-1}) $ is meaningless for $ d \notin \N $ and the Levi-Civita symbol cannot be analytically continued to $ d = 4 - 2 \de $ dimensions.
\end{observation}

\footnotetext{Given an $ n $-dimensional vector space $ V(\K) $ with a quadratic form $ q $, associated linear form $ \, \omega $ and orthogonal basis $ \{e_i\}_{i = 1,\dots,n} $, and a unital associative $ \K $-algebra $ \mathcal{A} $, a \textit{Clifford mapping} is an injective $ \K $-linear map $ \rho : V \rightarrow \mathcal{A} : \mathit{1} \notin \rho(V) \land \rho(x)^2 = - q(x) \mathit{1} \,\,\forall x \in V $. If $ \rho(V) $ generates $ \mathcal{A} $, then $ (\mathcal{A},\rho) $ is a \textit{Clifford algebra} for $ (V,q) $, and is denoted by $ \cla{V} $. It can be shown with simple algebraic manipulation that $ \{\rho(x),\rho(y)\} = 2 \omega(x,y)\mathit{1} \,\, \forall x,y \in V $.}
\footnotetext{Given an $ n $-dimensional vector space $ v(\K) $, its \textit{Grassmann algebra} (or exterior algebra) is the $ \N_0 $-graded algebra $ \bigwedge(V) = \bigoplus_{k = 0}^n \bigwedge^k(V) $ of $ k $-forms. It can be shown that $ \dim_\K \bigwedge^k(V) = \binom{n}{k} $, hence the top exterior subspace $ \bigwedge^n(V) $ is $ 1 $-dimensional: indeed, given a basis $ \{e_i\}_{i = 1,\dots,n} \subset V $, it is $ \bigwedge^n(V) = \braket{e_1 \wedge \dots \wedge e_n} $, and the Levi-Civita symbol is normalized so that $ \epsilon^{1 \dots n} $ has the same sign of $ e_1 \wedge \dots \wedge e_n $.
}












