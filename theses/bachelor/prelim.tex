\selectlanguage{english}

\section{Quantum Chromodynamics}
We consider a generalized QCD with gauge group $ \SUn{n_c} $, with $ n_c $ colours and $ n = n_f + n_F $ total quark flavours ($ n_f $ massless and $ n_F $ massive quark flavours).

\subsection{Yang--Mills theories}
\label{ssec:gauge-th}

A quantum field theory can be built starting from its symmetry properties: in particular, specifying a group of local transformations, the \bctxt{gauge group}, under which the theory must be invariant. Historically, the idea of gauge theories was first explored by Yang and Mills in \cite{Yang-1954}, with the aim of studying isotopic gauge invariance for the nucleon, and then generalized by Utiyama in \cite{Utiyama-1956}. A modern treatment of gauge theories can be found in Chapter 15 of \cite{Peskin-1995}, which we follow for our discussion.

Consider $ n $ fermionic fields $ \{\psi_k(x)\}_{k = 1, \dots, n} $ and an $ n $-spinor $ \Psi(x) $ defined as:
\begin{equation}
  \Psi(x) =
  \begin{pmatrix}
    \psi_1(x) \\ \vdots \\ \psi_n(x)
  \end{pmatrix}
\end{equation}
As a gauge group, consider a $ d $-dimensional Lie group $ G $: in particular, take $ G $ to be a simply-connected, so that each element can be expressed via the exponential map, and compact, so that its representations are unitary. Then, consider $ \{T^a\}_{a = 1, \dots, d} \subset \C^{n \times n} $ a representation of the associated Lie algebra $ \mathfrak{g} $, so that the action of $ G $ on $ \Psi $ can be expressed as:
\begin{equation}
  \Psi(x) \mapsto V(x) \Psi(x)
  \qquad \qquad
  V(x) \defeq \exp \left[ i \theta_a(x) T^a \right]
  \label{eq:gauge-trans}
\end{equation}
where the Lie parameters $ \{\theta_a(x)\}_{a = 1, \dots, d} \subset \mathcal{C}^\infty(\R^{1,3}) $ define a local gauge transformation. The aim is to define a Lagrangian which is invariant under this transformation, i.e. the Lagrangian of a (local) gauge theory.

Simple terms invariant under global phase rotations, like the fermion mass term $ m \bar{\Psi} \Psi $, are of course invariant under \eref{eq:gauge-trans} too, but derivatives need a careful treatment: indeed, the limit-definition of a derivative involves fields at different spacetime points, which have different transformations according to \eref{eq:gauge-trans}. In order to define a derivative of \Psi, it is necessary to introduce a factor to subtract values of $ \Psi(x) $ in a meaningful way, so consider $ \tens{U}(y,x) \in \Un{n} : \tens{U}(x,x) = 1 $ and which transforms under the action of $ G $ as:
\begin{equation}
  \tens{U}(y,x) \mapsto V(y) \tens{U}(y,x) V\dg(x)
  \label{eq:fac-cov-der}
\end{equation}
By the unitarity of the representations of $ G $, it is clear that $ \tens{U}(y,x) \Psi(x) $ and $ \Psi(y) $ have the same transformation law, so they can be meaningfully subtracted. Then, given $ n^\mu \in \R^{1,3} $, the covariant derivative of a fermionic field $ \Psi(x) $ along $ n^\mu $ is defined as:
\begin{equation}
  n^\mu D_\mu \Psi(x) \defeq \lim_{\varepsilon \rightarrow 0} \frac{1}{\varepsilon} \left[ \Psi(x + \varepsilon n) - \tens{U}(x + \varepsilon n, x) \Psi(x) \right]
  \label{eq:cov-der-def}
\end{equation}
where $ \tens{U}(y,x) $ is defined through \eref{eq:fac-cov-der}. To make this definition explicit, it is necessary to get an expression of $ \tens{U}(y,x) $ at infinitesimally-separted points. Given the unitarity of $ \tens{U}(y,x) $, it can be expressed through the generators $ \{T^a\}_{a = 1, \dots, d} $ as:
\begin{equation}
  \tens{U}(x + \varepsilon n, x) = \tens{I}_n + i g \varepsilon n^\mu A_\mu^a(x) T_a + \smo(\varepsilon^2)
  \label{eq:fac-cov-der-exp}
\end{equation}
where $ g \in \R $ is a constant. The new vector field $ A_\mu^a(x) $ (actually, $ d $ different vector fields) is a \bctxt{connection}, and it allows us to express the covariant derivative as (directly from \eref{eq:cov-der-def}):
\begin{equation}
  D_\mu = \pa_\mu - i g A_\mu^a T_a
\end{equation}
To show that $ D_\mu \Psi $ transforms in the same way as $ \Psi $, note that, from \eeref{eq:fac-cov-der}{eq:fac-cov-der-exp}:
\begin{equation*}
  \tens{I}_n + i g \varepsilon n^\mu A_\mu^a(x) T_a \mapsto I_n - \varepsilon n^\mu V(x) \pa_\mu V\dg(x) + V(x) \left( i g \varepsilon n^\mu A_\mu^a(x) T_a \right) V\dg(x) + \smo(\varepsilon^2)
\end{equation*}
Hence, the connection transforms as:
\begin{equation*}
  A_\mu^a(x) T_a \mapsto V(x) \left[ A_\mu^a(x) T_a + \frac{i}{g} \pa_\mu \right] V\dg(x) = A_\mu^a(x) T_a - f^{abc} A_\mu^a(x) \theta^b(x) T_c + \frac{1}{g} \pa_\mu \theta^a(x) T_a + \smo(\theta^2)
\end{equation*}
The second term makes it clear that the connection transforms according to the adjoint representation. From this expression, it follows that:
\begin{equation*}
  D_\mu \Psi(x) \mapsto \left[ \tens{I}_n + i \theta^a(x) T_a + \smo(\theta^2) \right] \left( \pa_\mu - i g A_\mu^a(x) T_a \right) \Psi(x) = V(x) D_\mu \Psi(x)
\end{equation*}
where the relation $ T_a T_b - i f^{abc} T_c = T_b T_a $ of the associated Lie algebra was used.

The gauge-invariant Lagrangian can thus be built using covariant derivatives (minimal coupling prescription), but one also needs to include a kinetic term for the connection, i.e. a gauge-invariant term dependent on $ A_\mu^a(x) $ only. This term can be found considering the commutator of covariant derivatives:
\begin{equation}
  [D_\mu , D_\nu] = - i g F_{\mu \nu}^a T_a
\end{equation}
with the \bctxt{field-strength tensor} defined as:
\begin{equation}
  F_{\mu \nu}^a \defeq \pa_\mu A_\nu^a - \pa_\nu A_\mu^a + g f^{abc} A_\mu^b A_\nu^c
  \label{eq:field-str}
\end{equation}
Note that the field-strength tensor is not itself a gauge-invariant quantity, as really there are $ d $ different field-strength tensors; however, it is straightforward to construct gauge-invariant combinations of $ F_{\mu \nu}^a $. In fact, in general any globally-symmetric function of $ \Psi $, $ F_{\mu \nu}^a $ and their covariant derivatives is also locally-symmetric, i.e. gauge-invariant: this follows from the construction of the covariant derivative. For a complete discussion, see Chapter 15 of \cite{Peskin-1995}.

Usually, the following gauge-invariant term is taken as kinetic term for the gauge field (i.e. the connection $ A_\mu^a(x) $):
\begin{equation}
  \tr \{(F_{\mu \nu}^a T_a)^2\} = 2 F_{\mu \nu}^a F^{\mu \nu}_a
\end{equation}
This allows defining the simplest non-Abelian gauge theory, \bctxt{Yang--Mills theory} without fermionic species:
\begin{equation}
  \lag_\text{YM} = - \frac{1}{4} F_{\mu \nu}^a F^{\mu \nu}_a
\end{equation}
To account for fermions interacting with the gauge field $ A_\mu^a(x) $, the Dirac Lagrangian with minimal coupling is added (see Chapter 15 of \cite{Weinberg-1996}):
\begin{equation}
  \lag = - \frac{1}{4} F_{\mu \nu}^a F^{\mu \nu}_a + \bar{\Psi} \left( i \slashed{D} - m \right) \Psi
  \label{eq:qcd-lag}
\end{equation}

\subsection{Gauge group \texorpdfstring{$ \SUn{n_c} $}{SU(n)}}

The $ \SUn{n_c} $ group is the group of unitary transformations of $ n_c $-dimensional complex vectors. Its (faithful) fundamental representation thus is:
\begin{equation*}
  \SUn{n_c} = \{\tens{U} \in \C^{n_c \times n_c} : \tens{U}\tens{U}\dg = \tens{U}\dg\tens{U} = \tens{I}_{n_c} \land \det{\tens{U}} = +1 \}
\end{equation*}
The generators of $ \SUn{n_c} $ can be found setting $ \tens{U} = \exp \left( i \theta_a T^a \right) = \tens{I}_{n_c} + i \theta_a T^a + \smo(\theta^2) $ and using $ \tens{U}\dg\tens{U} = \tens{I}_{n_c} $:
\begin{equation}
  T^a = T^{a\dagger}
  \label{eq:sun-herm}
\end{equation}
Moreover, by the Jacobi formula $ (\det A(t)) \frac{\dd}{\dd t} (\det A(t)) = \tr (A(t)^{-1} \frac{\dd}{\dd t} A(t)) $ evaluated at $ t = 0 $:
\begin{equation}
  \tr T^a = 0
  \label{eq:sun-trace}
\end{equation}
The traceless condition can be generalized to all semi-simple Lie algebras.
Therefore, the generators of $ \SUn{n_c} $ are $ \C^{n_c \times n_c} $ Hermitian traceless matrices: the dimension of $ \mathfrak{su}(n_c) $ then is $ n_c^2 - 1 $.

In general, the adjoint representation of a Lie group is given by representing its generators (i.e. the basis of the Lie algebra) with the structure constants of the Lie algebra:
\begin{equation}
  (T^b_\text{ad})_{ac} \equiv \bar{T}^b_{ac} = i f^{abc}
\end{equation}
which, in the case of $ \SUn{n_c} $, are $ f^{abc} = \epsilon^{abc} $. Indeed, it can be shown by the Jacobi identity that the structure constants satisfy the Lie algebra:
\begin{equation*}
  f^{abd} f^{dce} - f^{acd} f^{dbe} = f^{bcd} f^{ade}
  \quad \iff \quad
  [[T^a,T^b],T^c] + [[T^c,T^a],T^b] + [[T^b,T^c],T^a] = 0
\end{equation*}
Moreover, since the structure constant are real, the adjoint representation is always a real representation: the adjoint representation of $ \SUn{n_c} $ has degree $ n_c^2 - 1 $.

Representation are labelled by their Casimir operators. For any simple Lie algebra, given a representation $ \mathtt{r} $, a Casimir operator is defined as:
\begin{equation}
  T^a_\mathtt{r} T^a_\mathtt{r} = C_2(\mathtt{r}) \tens{I}_{n_{\mathtt{r}}}
  \label{eq:quad-cas}
\end{equation}
This is called the \bctxt{quadratic Casimir operator}, as it is associated to $ T^2 \equiv T^a T^a $ (a Casimir operator since $ [T^b, T^2] = i f^{bac} \{T^c,T^a\} = 0 $ by antisymmetry). For the fundamental and the adjoint representations $ \mathtt{n} $ and $ \mathtt{g} $ of $ \SUn{n_c} $, the quadratic Casimir operators are (\secref{sec:cas-op}):
\begin{equation}
  \caf \equiv C_2(\mathtt{n}) = \ttr \frac{n_c^2 - 1}{n_c}
  \qquad \qquad
  \caa \equiv C_2(\mathtt{g}) = 2\ttr n_c
  \label{eq:cas-sun}
\end{equation}
where $ \ttr $ (usually taken to be $ \ttr = \frac{1}{2} $) is the trace normalization of the generators in the fundamental representation:
\begin{equation}
  \tr (T^a_\mathtt{n} T^b_\mathtt{n}) = \ttr \delta^{ab}
  \label{eq:diag-norm}
\end{equation}

\subsection{Quantization}
\label{ssec:qcd-quant}

The quantization of a non-Abelian gauge theory requires careful treatment due to the presence of spurious non-physical degrees of freedom. These can be accounted for with the Faddeev--Popov method \cite{Faddeev-1967}, i.e. imposing a gauge fixing condition and introducing non-physical ghost fields, which serve as $ \virgolette{negative} $ degrees of freedom. Then, functional quantization can be performed naturally, resulting in the following Feynman rules for the Yang--Mills Lagrangian (excluding ghost fields):
\begin{align*}
  \begin{tikzpicture}[baseline = (r.base)]
    \begin{feynman}[inline = (r.base)]
      \vertex[dot] (a) {};
      \vertex (la) at ($(a) + (-0.3,0)$) {\(a\)};
      \vertex[right = 2cm of a, dot] (b) {};
      \vertex (lb) at ($(b) + (0.3,0)$) {\(b\)};
      \vertex[below = 0.25em of b] (r);
      \diagram* {
        a -- [fermion, momentum = \(p\)] (b),
      };
    \end{feynman}
  \end{tikzpicture}
  & \quad = \quad
  \frac{i}{\slashed{p} - m} \delta_{ab}
  &
  \begin{tikzpicture}[baseline = (r.base)]
    \begin{feynman}[inline = (r.base)]
      \vertex[dot] (a) {};
      \vertex (la) at ($(a) + (-0.5,0)$) {\(\mu,a\)};
      \vertex[right = 2cm of a, dot] (b) {};
      \vertex (lb) at ($(b) + (0.5,0)$) {\(\nu,b\)};
      \vertex[below = 0.25em of b] (r);
      \diagram* {
        a -- [gluon, momentum = \(k\)] (b),
      };
    \end{feynman}
  \end{tikzpicture}
  & \quad = \quad
  \frac{-i g_{\mu \nu}}{k^2} \delta_{ab}
  \\
  \begin{tikzpicture}[baseline = (r.base)]
    \begin{feynman}[inline = (r.base)]
      \vertex (a) {};
      \vertex (la) at ($(a) + (0,0.1)$) {\(\mu,a\)};
      \vertex[dot, below = 1.5cm of a] (v) {};
      \vertex[below = 0.8cm of v] (z) {};
      \vertex[left = 1.3cm of z] (b) {};
      \vertex (lb) at ($(b) + (-0.1,0.05)$) {\(b\)};
      \vertex[right = 1.3cm of z] (c) {};
      \vertex (lc) at ($(c) + (0.1,0)$) {\(c\)};
      \vertex[below = 0.25em of v] (r);
      \diagram* {
        (a) -- [gluon] (v),
        (b) -- [fermion] (v),
        (c) -- [anti fermion] (v),
      };
    \end{feynman}
  \end{tikzpicture}
  & \quad = \quad
  i g \gamma^\mu T^a \delta_{bc}
  &
  \begin{tikzpicture}[baseline = (r.base)]
    \begin{feynman}[inline = (r.base)]
      \vertex (a) {};
      \vertex (la) at ($(a) + (0,0.1)$) {\(\mu,a\)};
      \vertex[dot, below = 1.5cm of a] (v) {};
      \vertex[below = 0.8cm of v] (z) {};
      \vertex[left = 1.3cm of z] (b) {};
      \vertex (lb) at ($(b) + (-0.23,0.05)$) {\(\nu,b\)};
      \vertex[right = 1.3cm of z] (c) {};
      \vertex (lc) at ($(c) + (0.2,0)$) {\(\rho,c\)};
      \vertex[below = 0.25em of v] (r);
      \diagram* {
        (a) -- [gluon, momentum = \(p\)] (v),
        (b) -- [gluon, momentum = \(q\)] (v),
        (c) -- [gluon, momentum = \(k\)] (v),
      };
    \end{feynman}
  \end{tikzpicture}
  & \quad = \quad
  \begin{array}{l}
    g f^{abc} \,[\, g^{\mu \nu} \left( p - q \right)^\rho \\
    \quad\,\ \ +\, g^{\nu \rho} \left( q - k \right)^\mu \\
    \quad\,\ \ +\, g^{\rho \mu} \left( k - p \right)^\nu \,]
  \end{array}
\end{align*}
\begin{equation*}
  \begin{tikzpicture}[baseline = (r.base)]
    \begin{feynman}[inline = (r.base)]
      \vertex[dot] (v) {};

      \vertex[above = 1.3cm of v] (z1) {};
      \vertex[left = 1.3cm of z1] (a) {};
      \vertex[right = 1.3cm of z1] (b) {};

      \vertex[below = 1.3cm of v] (z2) {};
      \vertex[left = 1.3cm of z2] (c) {};
      \vertex[right = 1.3cm of z2] (d) {};

      \vertex (la) at ($(a) + (-0.23,-0.05)$) {\(\mu,a\)};
      \vertex (lb) at ($(b) + (0.2,0)$) {\(\nu,b\)};
      \vertex (lc) at ($(c) + (-0.2,-0.05)$) {\(\rho,c\)};
      \vertex (ld) at ($(d) + (0.23,0)$) {\(\sigma,d\)};

      \vertex[below = 0.25em of v] (r);

      \diagram* {
        (a) -- [gluon] (v),
        (b) -- [gluon] (v),
        (c) -- [gluon] (v),
        (d) -- [gluon] (v),
      };
    \end{feynman}
  \end{tikzpicture}
  \quad = \quad
  \begin{array}{l}
    - i g^2 \,[\, f^{abe} f^{cde} \left( g^{\mu \rho} g^{\nu \sigma} - g^{\mu \sigma} g^{\nu \rho} \right) \\
    \quad\,\ +\, f^{ace} f^{bde} \left( g^{\mu \nu} g^{\rho \sigma} - g^{\mu \sigma} g^{\nu \rho} \right) \\
    \quad\,\ +\, f^{ade} f^{bce} \left( g^{\mu \nu} g^{\rho \sigma} - g^{\mu \rho} g^{\nu \sigma} \right) \,]
  \end{array}
\end{equation*}
The peculiar nature of non-Abelian gauge theories is due to the possibility for gauge bosons to self-interact (last two vertices).

In the case of QCD, the quanta of the $ n $ spinor fields are the $ n $ quark falvours, while the $ n_c^2 - 1 $ gauge fields are gluon fields, and their colour charges are given by their respective $ \SUn{n_c} $ representations: $ n_c $ charges for the quarks (fundamental $ n $-dimensional representation) and $ n_c^2 - 1 $ charges for the gluons (adjoint representation).

\section{Renormalization scheme}
\label{sec:renorm}

The computation of NLO corrections to scattering processes in a generic QFT involves UV-divergent loop amplitudes. In order to obtain finite results from these divergences, a renormalization scheme must be implemented.

The generalized Catani's formula for IR singularities in virtual corrections is provided in \cite{Catani-2001} in a charge-unrenormalized (but mass-renormalized) way, i.e. still containing UV singularities too (recall \secref{sec:sing}), thus it is necessary to carry out the renormalization procedure explicitly. To this end, we formally state the renormalization scheme adopted in this work.

\subsection{Dimensional regularization}
\label{ssec:dim-reg}

In the evaluation of loop amplitudes, both UV- and IR-singularities are encountered. The most efficient way to simultaneously regularize both types of divergences is dimensional regularization \cite{dim-reg}.

In general, the dimensional regularization scheme consists in the analytic continuation of loop momenta to $ d = 4 - 2 \de $ dimensions, with $ \de \in \C : \Re\de < 0 $ for IR divergences and $ \Re\de > 0 $ for UV ones. This procedure turns loop integrals into meromorphic functions of $ \de \in \C $, allowing for the isolation of divergences as poles in $ \de $.

The dimensional regularization prescription leaves freedom in choosing the dimensionality of external momenta, as well as the number of polarizations of both external and internal particles, thus allowing for the definition of different regularization schemes. We choose to work with \bctxt{conventional dimensional regularization} (CDR), in which all momenta and polarization are analytically continued to $ d $ dimensions, as opposed to the 't Hooft--Veltman scheme (HV), in which only internal momenta and polarizations are.

When considering non-chiral gauge theories like QCD, CDR is the most natural choice, as the main difference between CDR and HV is the treatment of purely $ 4 $-dimensional objects, i.e. $ \gamma^5 $ and $ \epsilon_{\mu \nu \sigma \rho} $. In particular, in CDR both the Dirac algebra and Lorentz indices are analytically continued to $ d $ dimensions, leading to a mathematical inconsistency stemming from the fact that, when $ d \notin \N $, the following identities cannot hold simultaneously\footnotemark:
\begin{equation*}
  \{\gamma^5 , \gamma^\mu\} = 0 \quad \forall \mu = 0, 1, \dots, d-1
  \qquad \qquad
  \tr\{\gamma^5 \gamma^\mu \gamma^\nu \gamma^\rho \gamma^\sigma\} = -4 i \epsilon^{\mu \nu \rho \sigma}
\end{equation*}

\footnotetext{This inconsistency is the explicit manifestation of a more profound topological issue of analytically continuing the number of dimensions: the Levi-Civita symbol in $ d = 4 $ is linked to the Grassmann algebra $ \bigwedge(\R^{1,3}) $, and in particular to its top-form, but $ \bigwedge^k(\R^d) $ is only defined for $ d \in \N $, so the top exterior subspace $ \bigwedge^d(\R^{1,d-1}) $ is meaningless for $ d \notin \N $ and the Levi-Civita symbol cannot be analytically continued to $ d = 4 - 2 \de $ dimensions.}

The choice of CDR over HV is then clear: in QCD, the only pathological objects are encountered when considering chiral vertices (e.g. for pseudoscalar mesons) and electroweak interactions, and both can be handled via known prescriptions, e.g. the Breitenlohner-Maison/'t Hooft-Veltman (BMHV) scheme \cite{Breitenlohner-1977} or the Larin scheme \cite{Larin-1993}.

\subsection{Minimal subtraction}
\label{ssec:min-sub}

Once regularized, UV-divergences have to be removed via renormalization of fields and coupling constants. As a result of the renormalization procedure, a running coupling $ \rcr $ is introduced, and its definition in terms of the bare coupling $ \bc $ depends both on the regularization and the renormalization schemes.

In this work, we renormalize the coupling in a standard way (as in \cite{Catani-1998}) using the \bctxt{modified minimal-subtraction scheme} ($ \msb $), which directly subtracts UV-divergences from the coupling:
\begin{equation}
  \bc S_\de = \rcr \ren^{2\de} \left[ 1 - \frac{1}{2} \frac{\rcr}{2\pi} \frac{\beta_0}{\de} + \smo(\rc^2) \right]
  \label{eq:msb-def}
\end{equation}
where $ \ren $ is an arbitrary renormalization scale, $ S_\de $ is the typical phase-space volume factor in dimensional regularization:
\begin{equation}
  S_\de \equiv \left( 4\pi \right)^\de e^{-\eg \de}
\end{equation}
with $ \eg = 0.5772\dots $ the Euler-Mascheroni constant, and $ \beta_0 $ is the leading-order coefficient of the QCD $ \beta $-function \eref{eq:beta-func}:
\begin{equation}
  \beta_0 \defeq \frac{11}{6} \caa - \frac{2}{3} \ttr n_q
\end{equation}
where $ n_q $ is the number of active quark flavours at the considered energy scale. In this work $ n_q = n_f $ unless otherwise specified.

An important clarification about the dimensionality of $ \bc $ and $ \rc $ is needed, due to the presence of $ \ren^{2\de} $ in \eref{eq:msb-def}. In dimensional regularization, the action remains a dimensionless quantity, hence, given $ \act = \int \dd^dx \, \lag $ and that in natural units ($ c = \hbar = 1 $) all dimensions can be expressed as mass dimensions (since $ [T] = [L] = [M]^{-1} $), the QCD Lagrangian \eref{eq:qcd-lag} must have dimension $ [\lag] = d $, as $ [\dd^dx] = -d $. It is now trivial to verify the following dimensions:
\begin{equation*}
  [\Psi] = \frac{d - 1}{2}
  \qquad \qquad
  [A_\mu^a] = \frac{d - 2}{2}
  \qquad \qquad
  [g] = \frac{4 - d}{2} = \de
\end{equation*}
This shows that, in dimensional regularization, $ [\bc] = 2\de $. In order to work with dimensionless quantities, then, in \eref{eq:msb-def} we chose to extract the mass dimension from $ \rc $.

When dealing with scattering processes, a fundamental quantity is the amplitude of a process. In the Schrödinger picture, given a quantum system described by a Hamiltonian $ H $ and a Hilbert space $ \hilb $, the amplitude for the process $ \ket{a} \rightarrow \ket{b} $, where $ \ket{a} , \ket{b} \in \hilb $ is defined as:
\begin{equation}
  \ampl \defeq \braket{b | \smat(t , t_0) | a}
\end{equation}
where $ \Delta t \equiv t - t_0 $ is the time elapsed in the transition and the $ S $-matrix is defined as:
\begin{equation}
  \smat(t , t_0) \defeq e^{- i H (t - t_0)}
\end{equation}
For a $ n \rightarrow m $ scattering with defined initial- and final-state momenta, the amplitude can be written in the multi-particle phase as:
\begin{equation}
  \ampl_{n + m} \defeq \braket{\ve{p}_1, \dots, \ve{p}_n | \smat(+\infty , -\infty) | \ve{k}_1, \dots, \ve{k}_m} 
\end{equation}
where $ t, t_0 \rightarrow \pm \infty $ as to consider free-particle initial and final states (for a discussion on this adiabatic approximation, see Chapter 5 of \cite{Itzykson-1980}). The explicit expression of the amplitude can be computed from Feynman diagrams: for a complete discussion, see Chapter 4 of \cite{Peskin-1995}.

In general, we consider amplitudes $ \ampl_m $ involving $ m $ external QCD partons (gluons and quarks), with momenta $ \{p\} \equiv \{p_1, \dots, p_m\} $, and an arbitrary number of colorless particles (photons, leptons, ...). Dependence on the momenta and quantum numbers of colorless particles is always understood and not explicitly shown. The $ \msb $-renormalized amplitude has the following perturbative expansion in $ \rc $:
\begin{equation}
  \ampl_m(\rcr, \ren^2 ; \{p\}) = \left( \frac{\rcr}{2\pi} \right)^q \left[ \ampl_m^{(0)}(\ren^2 ; \{p\}) + \frac{\rcr}{2\pi} \ampl_m^{(1)}(\ren^2 ; \{p\}) + \smo(\rc^2) \right]
  \label{eq:mat-exp}
\end{equation}
where the overall power is, in general, $ q \in \frac{1}{2} \N_0 $. Note that, although UV-divergences have been removed by the renormalization procedure, these amplitudes are still IR-singular as $ \de \rightarrow 0 $.

\section{Colour-space formalism}
\label{sec:colour-space}

To handle the colour structure of QCD amplitudes, we adopt the colour-space formalism as in \cite{Catani-1997}.

The $ m $ external partons in the amplitude $ \ampl_m $ each carry two indices: a colour index and a spin index. Colour indices are denoted by $ c_1, \dots, c_m $: for gluons $ c_i \equiv a_i \in \{1, \dots, n_c^2 - 1\} $, as the field-strength tensor \eref{eq:field-str} transforms according to the adjoint representation of the gauge group, while for quarks $ c_i \equiv \alpha_i \in \{1, \dots, n_c\} $, as their Dirac fields transform according to the fundamental representation of the gauge group. Spin indices, on the other hand, are denoted by $ s_1, \dots, s_m $, and they need to take into account how helicities change in CDR: for gluons $ s_i \equiv \mu_i \in \{1, \dots, d\} $, while for quarks $ s_i \in \{1,2\} $.

Consider the $ m $-parton colour-space $ \hilb_c $ and helicity-space $ \hilb_s $, and introduce an orthonormal basis in each:
\begin{equation*}
  \{\ket{c_1, \dots, c_m}\} \in \hilb_c
  \qquad \qquad
  \{\ket{s_1, \dots, s_m}\} \in \hilb_s
\end{equation*}
Note that, as these are finite-dimensional Hilbert spaces, the non-canonical (basis-dependent) isomorphisms $ \hilb_c \leftrightarrow \hilb_c^* $ and $ \hilb_s \leftrightarrow \hilb_s^* $ are well-defined\footnotemark.

\footnotetext{Given a finite-dimensional $ \K $-vector space $ V $ and a basis $ \{v_i\}_{i = 1,\dots,n} \subset V $, with $ n = \dim_\K V $, then a basis $ \{\omega^1, \dots, \omega^n\} \subset V^* $ of $ V^* \defeq \Hom(V,\K) $ is defined by $ \omega^i(v_j) = \delta^i_j $, and the function $ \varphi : V \rightarrow V^* : v_i \mapsto \omega^i $ is a \textit{non-canonical isomorphism} $ V \leftrightarrow V^* $. \\
If $ V $ is infinite-dimensional, instead, given a basis $ \{v_i\}_{i \in \mathcal{I}} \subset V $, the above construction only allows to define linearly-independent subsets of $ V^* $, which are not granted to be bases.}

Then, to make the colour-helicity structure of the $ m $-parton amplitude explicit, we define it as an abstract vector in $ \hilb_c \otimes \hilb_s $, so that:
\begin{equation}
  \ampl_m^{\{c_1, \dots, c_m\}, \{s_1, \dots, s_m\}}(\{p_1, \dots, p_m\}) \equiv \braket{\{c_1, \dots, c_m\}, \{s_1, \dots, s_m\} | \ampl_m(\{p_1, \dots, p_m\})}
\end{equation}
with:
\begin{equation*}
  \ket{\{c_1, \dots, c_m\}, \{s_1, \dots, s_m\}} \equiv \ket{c_1, \dots, c_m} \otimes \ket{s_1, \dots, s_m}
\end{equation*}
Hence, it is clear that the squared amplitude summed over colours and helicities is:
\begin{equation}
  \abs{\ampl_m}^2 = \braket{\ampl_m | \ampl_m}
\end{equation}
To represent colour interactions at QCD vertices, we associate to each parton $ i $ a colour charge $ \tco_i = \{\tcc_i^a\}_{a = 1, \dots, n_c^2 - 1} $ related to the emission of a gluon. The action of $ \tco_i $ onto $ \hilb_c $ is defined by:
\begin{equation}
  \braket{c_1, \dots, c_i, \dots, c_m | \tcc_i^a | b_1, \dots, b_i, \dots, b_m} = \delta_{c_1, b_1} \dots \tcc^a_{c_i b_i} \dots \delta_{c_m, b_m}
  \label{eq:t-act}
\end{equation}
So, $ \{\tcc^a_{c_i b_i}\}_{a = 1, \dots, n_c^2 - 1} $ form a vector with respect to the colour index $ a $ of the emitted gluon, and they are matrices in different representations of $ \SUn{n_c} $, depending on the parton $ i $:
\begin{itemize}
  \item if $ i $ is a gluon, then $ \tcc^a_{cb} \equiv i f_{cab} $ (adjoint representation);
  \item if $ i $ is a final-state quark, then $ \tcc^a_{\alpha \beta} \equiv t^a_{\alpha \beta} $ (fundamental representation), while if it is a final-state antiquark $ \tcc^a_{\alpha \beta} \equiv - t^a_{\alpha \beta} $ (conjugate of fundamental representation);
  \item if $ i $ is an initial-state quark, by crossing-symmetry $ \tcc^a_{\alpha \beta} \equiv - t^a_{\alpha \beta}$, while if it is an initial-state antiquark $ \tcc^a_{\alpha \beta} \equiv t^a_{\alpha \beta} $.
\end{itemize}
The algebra of these QCD colour-charge operators is easily determined. First of all, we set:
\begin{equation}
  \tco_i \cdot \tco_j \equiv \sum_{a = 1}^{n_c^2 - 1} \tcc_i^a \tcc_i^b
  \label{eq:t-dot}
\end{equation}
Then, by the action \eref{eq:t-act}, it is clear that charges associated to different partons commute, i.e.:
\begin{equation}
  \tco_i \cdot \tco_j = \tco_j \cdot \tco_i
  \qquad
  \forall i \neq j \in \{1, \dots, m\}
\end{equation}
Moreover, by \ceref{eq:quad-cas}{eq:t-dot}:
\begin{equation}
  \tco_i^2 = C_i \id_{\hilb_c}
\end{equation}
with $ C_i \equiv \caf $ if $ i $ is a quark/antiquark and $ C_i \equiv \caa $ if it is a gluon, i.e. the quadratic Casimir operators \eref{eq:cas-sun}. Finally, as each vector $ \ket{\ampl_m} $ is a colour-singlet, colour conservation implies:
\begin{equation}
  \sum_{i = 1}^{m} \tco_i \ket{\ampl_m} = 0
  \label{eq:col-cons}
\end{equation}
This allows us to partially (or fully, if $ m = 2 $ or $ m = 3 $, as in Appendix A of \cite{Catani-1997}) factorize the colour-charge algebra in terms of quadratic Casimir operators.










