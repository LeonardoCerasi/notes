\selectlanguage{english}

\section{Renormalization scheme}

The computation of NLO corrections to scattering processes often involves diverging loop amplitudes. In order to obtain finite results from these divergences, a renormalization scheme must be implemented.

As the generalized Catani's formula for virtual corrections is provided in \cite{Catani-2001} in a charge-unrenormalized (but mass-renormalized) way, it is necessary to carry out the renormalization procedure explicitly. To this end, we formally state the renormalization scheme adopted in this work.

\subsection{Dimensional regularization}

In the evaluation of loop amplitudes, both UV- and IR-singularities are encountered. The most efficient way to simultaneously regularize both types of divergences is dimensional regularization, a regularization scheme first introduced by 't Hooft and Veltman in \cite{Thooft-1972}.

In general, the dimensional regularization scheme consists in the analytic continuation of loop momenta to $ d = 4 - 2 \de $ dimensions, with $ \de \in \C : \Re\de < 0 $. This procedure turns loop integrals into meromorphic\footnotemark functions of $ \de \in \C $, allowing for the isolation of divergences as poles in $ \de $.

\footnotetext{Given an open set $ D \subset \C $, then $ f : D \rightarrow \C $ is \textit{meromorphic} if it is holomorphic on $ D - P $, where $ P \subset D $ is a set of isolated points called \textit{poles}. Recall that a function $ f : D \rightarrow \C $ is \textit{holomorphic} on $ D $ if it is complex differentiable at every point in $ D $.}

The dimensional regularization prescription leaves freedom in choosing the dimensionality of external momenta, as well as the number of polarizations of both external and internal particles, thus allowing for the definition of different regularization schemes. We choose to work with \bctxt{conventional dimensional regularization} (CDR), in which all momenta and polarization are analytically continued to $ d $ dimensions, as opposed to the 't Hooft--Veltman scheme (HV), in which only internal momenta and polarizations are.

When considering non-chiral gauge theories like QCD, CDR is the most natural choice, as the main difference between CDR and HV is the treatment of purely $ 4 $-dimensional objects, i.e. $ \gamma^5 $ and $ \epsilon_{\mu \nu \sigma \rho} $. In particular, in CDR both the Dirac algebra and Lorentz indices are analytically continued to $ d $ dimensions, leading to a mathematical inconsistency when $ d \notin \N $.

\begin{observation}{Inconsistency in CDR}{cdr-inc}
  In $ 4 $-dimensional Minkowski space $ \R^{1,3} $ with metric signature $ \eta = (+,-,-,-) $, the Dirac algebra is defined as $ \dira \cong \clar{1,3} \otimes \C $ (complexification\footnotemark). This Clifford algebra admits a matrix representation with generators $ \{\gamma^\mu\}_{\mu = 0, 1, 2, 3} \subset \C^{4 \times 4} $ such that:
  \begin{equation}
    \{\gamma^\mu , \gamma^\nu\} = 2 \eta^{\mu \nu} \tens{I}_4
    \label{eq:dir-alg}
  \end{equation}
  As in CDR Lorentz indices too are $ d $-dimensional, the Dirac algebra becomes $ \clac{1,d-1} \simeq \clar{1,d-1} \otimes \C $, where usual Minkowski space $ \R^{1,3} $ is replaced by the pseudo-Euclidean space $ \R^{1,d-1} $ with metric signature $ \eta = (+,-,\dots,-) $. This structure, however, is ill-defined, as for $ d \in \N $ then $ \dim_\C \clac{1,d-1} = 2^d $, but an algebra of dimension $ 2^d : d \notin \N $ is meaningless.

  To solve this issue, in CDR spinor indices remain $ 4 $-dimensional, i.e. we consider a matrix representation generated by $ \{\gamma^\mu\}_{\mu = 0, \dots, d-1} \subset \C^{4 \times 4} $ and impose the formal relations \eref{eq:dir-alg}. Consistency is achieved by analytical continuation of trace identities: indeed, traces obtained by recursively applying \eref{eq:dir-alg} (like $ \tr\{\gamma^\mu \gamma^\nu\} = 4 \eta^{\mu \nu} $) are still valid in $ d $-dimensions, as the only dependence on dimension comes from contractions such as $ \tensor{\eta}{^\mu_\mu} = d $.

  A fatal inconsistency of CDR arises when considering $ \gamma^5 $. In $ \dira $, this matrix is defined as $ \gamma^5 \defeq \frac{i}{4!} \epsilon_{\mu \nu \rho \sigma} \gamma^\mu \gamma^\nu \gamma^\rho \gamma^\sigma = i \gamma^0 \gamma^1 \gamma^2 \gamma^3 $ and has the property $ \{\gamma^5 , \gamma^\mu\} = 0 \,\, \forall \mu = 0,1,2,3 $, which allows to prove the following identity:
  \begin{equation}
    \tr\{\gamma^5 \gamma^\mu \gamma^\nu \gamma^\rho \gamma^\sigma\} = -4 i \epsilon^{\mu \nu \rho \sigma}
    \label{eq:top-form-gamma-5}
  \end{equation}
  This construction cannot be generalized consistently to $ d \notin \N $. To show this, assume a $ d $-dimensional generalization $ \gamma^5 \in \C^{4 \times 4} : \{\gamma^5 , \gamma^\mu\} = 0 \,\,\forall \mu = 0, \dots, d-1 $, so that $ \gamma^5 \gamma^{\mu_1} \dots \gamma^{\mu_n} = \left( -1 \right)^n \gamma^{\mu_1} \dots \gamma^{\mu_n} \gamma^5 $. By the cyclicity of the trace, then:
  \begin{equation}
    \left( 1 - \left( -1 \right)^n \right) \tr\{\gamma^5 \gamma^{\mu_1} \dots \gamma^{\mu_n}\} = 0
  \end{equation}
  Consider $ n = d $: clearly $ 1 - \left( -1 \right)^d = 1 - e^{i \pi d} \neq 0 $ for $ d \notin \N $, so $ \tr\{\gamma^5 \gamma^{\mu_1} \dots \gamma^{\mu_d}\} = 0 $. This is an open contradiction to \eref{eq:top-form-gamma-5}, as analytic continuation should continuously preserve the top product of the algebra as $ \de \rightarrow 0 $.

  This contradiction is the explicit manifestation of a more profound topological issue of analytically continuing the number of dimensions: the Levi-Civita symbol in $ d = 4 $ is linked\footnotemark to the Grassmann algebra $ \bigwedge(\R^{1,3}) $, and in particular to its top-form, but $ \bigwedge^k(\R^d) $ is only defined for $ d \in \N $, so the top exterior subspace $ \bigwedge^d(\R^{1,d-1}) $ is meaningless for $ d \notin \N $ and the Levi-Civita symbol cannot be analytically continued to $ d = 4 - 2 \de $ dimensions.
\end{observation}

\footnotetext{Given an $ n $-dimensional vector space $ V(\K) $ with a quadratic form $ q $, associated linear form $ \, \omega $ and orthogonal basis $ \{e_i\}_{i = 1,\dots,n} $, and a unital associative $ \K $-algebra $ \mathcal{A} $, a \textit{Clifford mapping} is an injective $ \K $-linear map $ \rho : V \rightarrow \mathcal{A} : \mathit{1} \notin \rho(V) \land \rho(x)^2 = - q(x) \mathit{1} \,\,\forall x \in V $. If $ \rho(V) $ generates $ \mathcal{A} $, then $ (\mathcal{A},\rho) $ is a \textit{Clifford algebra} for $ (V,q) $, and is denoted by $ \cla{V} $. It can be shown with simple algebraic manipulation that $ \{\rho(x),\rho(y)\} = 2 \omega(x,y)\mathit{1} \,\, \forall x,y \in V $. $ \clar{m,n} $ denotes the Clifford algebra associated to $ \R^{m,n} $ with canonical pseudo-Euclidean quadratic form.}
\footnotetext{Given an $ n $-dimensional vector space $ V(\K) $, its \textit{Grassmann algebra} (or exterior algebra) is the $ \N_0 $-graded algebra $ \bigwedge(V) = \bigoplus_{k = 0}^n \bigwedge^k(V) $ of $ k $-forms. It can be shown that $ \dim_\K \bigwedge^k(V) = \binom{n}{k} $, hence the top exterior subspace $ \bigwedge^n(V) $ is $ 1 $-dimensional: indeed, given a basis $ \{e_i\}_{i = 1,\dots,n} \subset V $, it is $ \bigwedge^n(V) = \braket{e_1 \wedge \dots \wedge e_n} $, and the Levi-Civita symbol is normalized so that $ \epsilon^{1 \dots n} $ has the same sign of $ e_1 \wedge \dots \wedge e_n $.
}

As of \obsref{obs:cdr-inc}, it is now clear why we choose to adopt CDR: in QCD, the only pathological objects are encountered when considering chiral vertices (e.g. for pseudoscalar mesons) and electroweak interactions, and both can be handled via known prescriptions, e.g. the Breitenlohner-Maison/'t Hooft-Veltman (BMHV) scheme outlined in \cite{Breitenlohner-1977}.

\subsection{Minimal subtraction}

Once regularized, UV-divergences have to be removed via renormalization of fields and coupling constants. As a result of the renormalization procedure, a running coupling $ \rcr $ is introduced, and its definition in terms of the bare coupling $ \bc $ depends both on the regularization and the renormalization schemes.

In this work, we renormalize the coupling in a standard way (as in \cite{Catani-1998}) using the \bctxt{modified minimal-subtraction scheme} ($ \msb $), which directly subtracts UV-divergences from the coupling:
\begin{equation}
  \bc S_\de = \rcr \ren^{2\de} \left[ 1 - \frac{\rcr}{2\pi} \frac{\beta_0}{\de} + \smo(\rc^2) \right]
  \label{eq:msb-def}
\end{equation}
where $ \ren $ is an arbitrary renormalization scale, $ S_\de $ is the typical phase-space volume factor in dimensional regularization:
\begin{equation}
  S_\de \equiv \left( 4\pi \right)^\de e^{-\eg \de}
\end{equation}
with $ \eg = 0.5772\dots $ the Euler-Mascheroni constant, and $ \beta_0 $ is the first coefficient of the QCD $ \beta $-function:
\begin{equation}
  \beta_0 \defeq \frac{11}{6} \caa - \frac{2}{3} \ttr n_q
\end{equation}
where $ \caa $ and $ \ttr $ are linked to the gauge group $ \SUn{n_c} $ (see \secref{sec:colour-space})  and $ n_q $ is the number of active quark flavours at the considered energy scale\footnotemark.

\footnotetext{\label{quark-flav}Out of $ n = n_f + n_F $ total quark flavours ($ n_f $ massless and $ n_F $ massive quark flavours), we generally consider an energy scale such that $ n_q = n_f $, unless otherwise specified.}

\begin{observation}{Dimensionality of coupling}{}
  An important clarification about the dimensionality of $ \bc $ and $ \rc $ is needed, due to the presence of $ \ren^{2\de} $ in \eref{eq:msb-def}.

  Consider the QCD Lagrangian, i.e. a Yang-Mills Lagrangian with gauge group $ \SUn{n_c} $ and $ n $ quark species (see e.g. Chapter 15 of \cite{Weinberg-1996}, or \secref{ssec:gauge-th}):
  \begin{equation}
    \lag = - \frac{1}{4} F_{\mu \nu}^a F^{\mu \nu}_a + \bar{\Psi} \left( i \slashed{D} - m \right) \Psi
  \end{equation}
  with covariant derivative and field-strength tensor defined as:
  \begin{equation}
    D_\mu \defeq \pa_\mu - i g A_\mu^a T_a
    \qquad \qquad
    F_{\mu \nu}^a \defeq \pa_\mu A_\nu^a - \pa_\nu A_\mu^a + g f^{abc} A_\mu^b A_\nu^c
    \label{eq:cov-der-field-str}
  \end{equation}
  where $ \{T_a\}_{a = 1, \dots, n_c^2 - 1} \subset \C^{n \times n} $ is a Hermitian representation of $ \sua{n_c} $ and $ \Psi $ is a $ n $-spinor. Recall that the structure constants $ f^{abc} \in \R $ of $ \sua{n_c} $ are $ f^{abc} = \epsilon^{abc} $.

  In dimensional regularization, the action remains a dimensionless quantity, hence, given $ \act = \int \dd^dx \, \lag $ and that in natural units ($ c = \hbar = 1 $) all dimensions can be expressed as mass dimensions (since $ [T] = [L] = [M]^{-1} $), the Lagrangian must have dimension $ [\lag] = d $, as $ [\dd^dx] = -d $. It is now trivial to verify the following dimensions:
  \begin{equation*}
    [\Psi] = \frac{d - 1}{2}
    \qquad \qquad
    [A_\mu^a] = \frac{d - 2}{2}
    \qquad \qquad
    [g] = \frac{4 - d}{2} = \de
  \end{equation*}
  This shows that, in dimensional regularization, $ [\bc] = 2\de $. In order to work with dimensionless quantities, then, in \eref{eq:msb-def} we chose to extract the mass dimension from $ \rc $.
\end{observation}

In general, we consider amplitudes $ \mat_m $ involving $ m $ external QCD partons (gluons and quarks), with momenta $ \{p\} \equiv \{p_1, \dots, p_m\} $, and an arbitrary number of colorless particles (photons, leptons, ...). Dependence on the momenta and quantum numbers of colorless particles is always understood and not explicitly shown. The $ \msb $-renormalized amplitude has the following perturbative expansion in $ \rc $:
\begin{equation}
  \mat_m(\rcr, \ren^2 ; \{p\}) = \left( \frac{\rcr}{2\pi} \right)^q \left[ \mat_m^{(0)}(\ren^2 ; \{p\}) + \frac{\rcr}{2\pi} \mat_m^{(1)}(\ren^2 ; \{p\}) + \smo(\rc^2) \right]
\end{equation}
where the overall power is, in general, $ q \in \frac{1}{2} \N_0 $. Note that, although spoiled of UV-divergences, these amplitudes are still IR-singular as $ \de \rightarrow 0 $.

\section{Colour-space formalism}
\label{sec:colour-space}

We consider a generalized QCD with gauge group $ \SUn{n_c} $, with $ n_c $ colours and $ n = n_f + n_F $ quark flavours (see footnote \ref{quark-flav}). To handle the colour structure of QCD amplitudes, we adopt the colour-space formalism as in \cite{Catani-1997}.

\subsection{Gauge theories}
\label{ssec:gauge-th}

In order to better understand the colour-space formalism, it is useful to state how a general gauge theory is defined, and then analyze the specific case of a $ \SUn{n_c} $ gauge theory.

\subsubsection{Yang-Mills Lagrangian}

A quantum field theory can be built starting from its symmetry properties: in particular, specifying a group of local transformations, the \bctxt{gauge group}, under which the theory must be invariant. Historically, the idea of gauge theories was first explored by Yang and Mills in \cite{Yang-1954}, with the aim of studying isotopic gauge invariance for the nucleon, and then generalized by Utiyama in \cite{Utiyama-1956}. A modern treatment of gauge theories can be found in Chapter 15 of \cite{Peskin-1995}, which we follow for our discussion.

Consider $ n $ fermionic fields $ \{\psi_k(x)\}_{k = 1, \dots, n} $ and an $ n $-spinor $ \Psi(x) $ defined as:
\begin{equation}
  \Psi(x) =
  \begin{pmatrix}
    \psi_1(x) \\ \vdots \\ \psi_n(x)
  \end{pmatrix}
\end{equation}
As a gauge group, consider a $ d $-dimensional Lie group $ G $: in particular, take $ G $ to be a simply-connected Lie group, so that each element can be expressed via the exponential map, and compact too, so that its representations are unitary. Then, consider $ \{T^a\}_{a = 1, \dots, d} \subset \C^{n \times n} $ a representation of the associated Lie algebra $ \mathfrak{g} $, so that the action of $ G $ on $ \Psi $ can be expressed as:
\begin{equation}
  \Psi(x) \mapsto V(x) \Psi(x)
  \qquad \qquad
  V(x) \defeq \exp \left[ i \theta_a(x) T^a \right]
  \label{eq:gauge-trans}
\end{equation}
where the Lie parameters $ \{\theta_a(x)\}_{a = 1, \dots, d} \subset \mathcal{C}^\infty(\R^{1,3}) $ so define a local gauge transformation. The aim is to define a Lagrangian which is invariant under this transformation, i.e. the Lagrangian of a (local) gauge theory.

Simple terms invariant under global phase rotations, like the fermion mass term $ m \bar{\Psi} \Psi $, are of course invariant under \eref{eq:gauge-trans} too, but derivatives need a careful treatment: indeed, the limit-definition of a derivative involves fields at different spacetime points, which have different transformations according to \eref{eq:gauge-trans}. In order to define a derivative of \Psi, it is necessary to introduce a factor to subtract values of $ \Psi(x) $ in a meaningful way, so consider $ \tens{U}(y,x) \in \Un{n} : \tens{U}(x,x) = 1 $ and which transforms under the action of $ G $ as:
\begin{equation}
  \tens{U}(y,x) \mapsto V(y) \tens{U}(y,x) V\dg(x)
  \label{eq:fac-cov-der}
\end{equation}
By the unitarity of the representations of $ G $, it is clear that $ \tens{U}(y,x) \Psi(x) $ and $ \Psi(y) $ have the same transformation law, so they can be meaningfully subtracted. 
\begin{definition}{Covariant derivative}{}
  Given $ n^\mu \in \R^{1,3} $, the covariant derivative of a fermionic field $ \Psi(x) $ along $ n^\mu $ is defined as:
  \begin{equation}
    n^\mu D_\mu \Psi(x) \defeq \lim_{\varepsilon \rightarrow 0} \frac{1}{\varepsilon} \left[ \Psi(x + \varepsilon n) - \tens{U}(x + \varepsilon n, x) \Psi(x) \right]
    \label{eq:cov-der-def}
  \end{equation}
  where $ \tens{U}(y,x) $ is defined in \eref{eq:fac-cov-der}.
\end{definition}

To make this definition explicit, it is necessary to get an expression of $ \tens{U}(y,x) $ at infinitesimally-separted points. Given the unitarity of $ \tens{U}(y,x) $, it can be expressed through the generators $ \{T^a\}_{a = 1, \dots, d} $ as:
\begin{equation}
  \tens{U}(x + \varepsilon n, x) = \tens{I}_n + i g \varepsilon n^\mu A_\mu^a(x) T_a + \smo(\varepsilon^2)
  \label{eq:fac-cov-der-exp}
\end{equation}
where $ g \in \R $ is a constant. The new vector field $ A_\mu^a(x) $ (actually, $ d $ different vector fields) is a \bctxt{connection}, and it allows to express the covariant derivative as (directly from \eref{eq:cov-der-def}):
\begin{equation}
  D_\mu = \pa_\mu - i g A_\mu^a T_a
\end{equation}

\begin{proposition}{}{cov-der-gauge}
  The covariant derivative $ D_\mu \Psi $ transforms as $ \Psi $.
\end{proposition}

\begin{proofbox}
  \begin{proof}
    From \eeref{eq:fac-cov-der}{eq:fac-cov-der-exp} (recalling that $ \pa_\mu $ is anti-Hermitian):
    \begin{equation*}
      \begin{split}
        \tens{I}_n + i g \varepsilon n^\mu A_\mu^a T_a
        & \mapsto V(x + \varepsilon n) \left( \tens{I}_n + i g \varepsilon n^\mu A_\mu^a T_a \right) V\dg(x) \\
        & = \left[ \left( 1 + \varepsilon n^\mu \pa_\mu \right) V(x) \right] V\dg(x) + V(x) \left( i g \varepsilon n^\mu A_\mu^a T_a \right) V\dg(x) + \smo(\varepsilon^2) \\
        & = \tens{I}_n - \varepsilon n^\mu V(x) \pa_\mu V\dg(x) + V(x) \left( i g \varepsilon n^\mu A_\mu^a T_a \right) V\dg(x) + \smo(\varepsilon^2)
      \end{split}
    \end{equation*}
    Hence, the connection transforms as:
    \begin{equation*}
      A_\mu^a(x) T_a \mapsto V(x) \left[ A_\mu^a(x) T_a + \frac{i}{g} \pa_\mu \right] V\dg(x)
    \end{equation*}
    The derivative $ \pa_\mu V\dg(x) $ is non-trivial to compute, as $ G $ is in general non-Abelian, hence the exponent does not necessarily commute with its derivative. At $ \smo(\theta) $:
    \begin{equation*}
      \begin{split}
        A_\mu^a(x) T_a
        & \mapsto \left( \tens{I}_n + i \theta^b(x) T_b + \smo(\theta^2) \right) \left[ A_\mu^a(x) T_a + \frac{i}{g} \pa_\mu \right] \left( \tens{I}_n - i \theta^c(x) T_c + \smo(\alpha^2) \right) \\
        & = \left( \tens{I}_n + i \theta^b(x) T_b + \smo(\theta^2) \right) \left[ A_\mu^a(x) T_a - i A_\mu^a(x) \theta^c(x) T_a T_c + \frac{1}{g} \pa_\mu \theta^c(x) T_c + \smo(\theta^2) \right] \\
        & = A_\mu^a(x) T_a - i A_\mu^a(x) \theta^c(x) T_a T_c + i \theta^b(x) A_\mu^a(x) T_b T_a + \frac{1}{g} \pa_\mu \theta^c(x) T_c + \smo(\theta^2) \\
        & = A_\mu^a(x) T_a + f^{abc} A_\mu^a(x) \theta^b(x) T_c + \frac{1}{g} \pa_\mu \theta^a(x) T_a + \smo(\theta^2)
      \end{split}
    \end{equation*}
    \begin{equation*}
      \begin{split}
        \Longrightarrow \quad D_\mu \Psi
        & \mapsto \left[ \pa_\mu - i g A_\mu^a T_a - i g f^{abc} A_\mu^a \theta^b T_c - i \pa_\mu \theta^a T_a \right] \left( \tens{I}_n + i \theta^a T_a \right) \Psi + \smo(\theta^2) \\
        & = \big[ \pa_\mu + i \theta^a T_a \pa_\mu + i \pa_\mu \theta^a T_a - i g A_\mu^a T_a + g A_\mu^a \theta^b T_a T_b \\
        & \qquad \qquad \qquad \qquad \qquad \qquad \qquad \qquad - i g f^{abc} A_\mu^a \theta^b T_c - i \pa_\mu \theta^a T_a + \smo(\theta^2) \big] \Psi \\
        & = \left[ \pa_\mu + i \theta^a T_a \pa_\mu - i g A_\mu^a T_a + g A_\mu^a \theta^b T_a T_b - i g f^{abc} A_\mu^a \theta^b T_c + \smo(\theta^2) \right] \Psi
      \end{split}
    \end{equation*}
    Recognizing $ T_a T_b - i f^{abc} T_c = T_b T_a $ allows to write:
    \begin{equation*}
      \begin{split}
        D_\mu \Psi(x)
        & \mapsto \left[ \pa_\mu + i \theta^a(x) T_a \pa_\mu - i g A_\mu^a(x) T_a + g \theta^b(x) T_b A_\mu^a(x) T_a + \smo(\theta^2) \right] \Psi(x) \\
        & = \left[ \tens{I}_n + i \theta^a(x) T_a + \smo(\theta^2) \right] \left( \pa_\mu - i g A_\mu^a(x) T_a \right) \Psi(x) = V(x) D_\mu \Psi(x)
      \end{split}
    \end{equation*}
    which is the thesis.
  \end{proof}
\end{proofbox}

The gauge-invariant Lagrangian can thus be built using covariant derivatives (minimal coupling prescription), but there needs to be included a kinetic term for the connection, i.e. a gauge-invariant term depending on $ A_\mu^a(x) $ only.

\begin{lemma}{Field-strength tensor}{}
  The commutator of covariant derivatives reads:
  \begin{equation}
    [D_\mu , D_\nu] = - i g F_{\mu \nu}^a T_a
  \end{equation}
  with the \bclemma{field-strength tensor} defined as:
  \begin{equation}
    F_{\mu \nu}^a \defeq \pa_\mu A_\nu^a - \pa_\nu A_\mu^a + g f^{abc} A_\mu^b A_\nu^c
  \end{equation}
\end{lemma}

\begin{proofbox}
  \begin{proof}
    By direct computation:
    \begin{equation*}
      \begin{split}
        [D_\mu , D_\nu]
        & = [\pa_\mu , \pa_\nu] - i g [A_\mu^a , \pa_\nu] T_a - i g [\pa_\mu , A_\nu^a] T_a - g^2 A_\mu^b A_\nu^c [T_b , T_c] \\
        & = - i g \left( A_\mu^a \pa_\nu - \pa_\nu A_\mu^a - A_\mu^a \pa_\nu \right) T_a - i g \left( A_\nu^a \pa_\mu - \pa_\mu A_\nu^a - A_\nu^a \pa_\mu \right) T_a - i g^2 f^{bca} A_\mu^b A_\nu^b T_c \\
        & = - i g \left( \pa_\mu A_\nu^a - \pa_\nu A_\mu^a + g f^{bca} A_\mu^b A_\mu^c \right) T_a
      \end{split}
    \end{equation*}
    Using $ f^{bca} = f^{abc} $, as it is always possible to choose generators such that the structure constants are completely antisymmetric, yields the thesis.
  \end{proof}
\end{proofbox}

Note that the field-strength tensor is not itself a gauge-invariant quantity, as really there are $ d $ different field-strength tensors. However, it is straightforward to construct gauge-invariant combinations of $ F_{\mu \nu}^a $.

\begin{theorem}{Gauge invariance}{}
  Any globally-symmetric function of $ \Psi $, $ F_{\mu \nu}^a $ and their covariant derivatives is also locally-symmetric, i.e. gauge-invariant.
\end{theorem}

\begin{proofbox}
  \begin{proof}
    See Chapter 15 of \cite{Peskin-1995}.
  \end{proof}
\end{proofbox}

\begin{lemma}{Kinetic term}{}
  The following term is gauge-invariant:
  \begin{equation}
    \tr \{(F_{\mu \nu}^a T_a)^2\} = 2 F_{\mu \nu}^a F^{\mu \nu}_a
  \end{equation}
\end{lemma}

This allows defining the simplest non-Abelian gauge theory, \bctxt{Yang-Mills theory} without fermionic species:
\begin{equation}
  \lag_\text{YM} = - \frac{1}{4} F_{\mu \nu}^a F^{\mu \nu}_a
\end{equation}
To account for fermions interacting with the gauge field (i.e. the connection $ A_\mu^a(x) $), the Dirac Lagrangian with minimal coupling is added:
\begin{equation}
  \lag = - \frac{1}{4} F_{\mu \nu}^a F^{\mu \nu}_a + \bar{\Psi} \left( i \slashed{D} - m \right) \Psi
\end{equation}

\subsubsection{Gauge group \texorpdfstring{$ \SUn{n} $}{SU(n)}}

The $ \SUn{n} $ group is the group of unitary transformations of $ n $-dimensional complex vectors. Its (faithful) fundamental representation thus is:
\begin{equation*}
  \SUn{n} = \{\tens{U} \in \C^{n \times n} : \tens{U}\tens{U}\dg = \tens{U}\dg\tens{U} = \tens{I}_n \land \det{\tens{U}} = +1 \}
\end{equation*}
The generators of $ \SUn{n} $ can be found setting $ \tens{U} = \exp \left( i \theta_a T^a \right) = \tens{I}_n + i \theta_a T^a + \smo(\theta^2) $ and using $ \tens{U}\dg\tens{U} = \tens{I}_n $:
\begin{equation}
  T^a = T^{a\dagger}
  \label{eq:sun-herm}
\end{equation}
Moreover, by the Jacobi formula $ (\det A(t)) \frac{\dd}{\dd t} (\det A(t)) = \tr (A(t)^{-1} \frac{\dd}{\dd t} A(t)) $ evaluated at $ t = 0 $:
\begin{equation}
  \tr T^a = 0
  \label{eq:sun-trace}
\end{equation}
The traceless condition can be generalized to all semi-simple Lie algebras.
Therefore, the generators of $ \SUn{n} $ are $ \C^{n \times n} $ Hermitian traceless matrices: the dimension of $ \mathfrak{su}(n) $ then is $ n^2 - 1 $.

In general, the adjoint representation of a Lie group is given by representing its generators (i.e. the basis of the Lie algebra) with the structure constants of the Lie algebra:
\begin{equation}
  (T^b_\text{ad})_{ac} \equiv \bar{T}^b_{ac} = i f^{abc}
\end{equation}
In the case of $ \SUn{n} $, the structure constants are $ f^{abc} = \epsilon^{abc} $.

\begin{proposition}{Structure constants}{}
  The structure constants of a Lie algebra satisfy the Lie algebra.
\end{proposition}

\begin{proofbox}
  \begin{proof}
    As $ [T^a,T^b] = i f^{abc} T^c $, the Jacoby identity becomes (recalling that $ f^{abc} $ is totally antisymmetric):
    \begin{equation*}
      [[T^a,T^b],T^c] + [[T^b,T^c],T^a] + [[T^c,T^a],T^b] = 0
    \end{equation*}
    \begin{equation*}
      \iff f^{abd} f^{dce} + f^{bcd} f^{dae} + f^{cad} f^{dbe} = 0
    \end{equation*}
    The condition $ ([\bar{T}^a , \bar{T}^c])_{be} = i f^{acd} (\bar{T}^d)_{be} $ then gives:
    \begin{equation*}
      f^{bad} f^{dce} - f^{bcd} f^{dae} = f^{acd} f^{bde}
    \end{equation*}
    \begin{equation*}
      \iff f^{abd} f^{dce} + f^{bdc} f^{dae} + f^{cad} f^{dbe} = 0
    \end{equation*}
    These two expressions are equal, hence the thesis.
  \end{proof}
\end{proofbox}

Moreover, since the structure constant are real, the adjoint representation is always a real representation: the adjoint representation of $ \SUn{n} $ has degree $ n^2 - 1 $.\\
Representation are labelled by their Casimir operators. For any simple Lie algebra, given a representation $ \mathtt{r} $, a Casimir operator is defined as:
\begin{equation}
  T^a_\mathtt{r} T^a_\mathtt{r} = C_2(\mathtt{r}) \tens{I}_{n_{\mathtt{r}}}
  \label{eq:quad-cas}
\end{equation}
This is called the \bctxt{quadratic Casimir operator}, as it is associated to $ T^2 \equiv T^a T^a $ (a Casimir operator since $ [T^b, T^2] = i f^{bac} \{T^c,T^a\} = 0 $ by antisymmetry).

\begin{proposition}{Quadratic Casimir operator}{}
  For the fundamental and the adjoint representations $ \mathtt{n} $ and $ \mathtt{g} $ of $ \SUn{n} $, the quadratic Casimir operators are:
  \begin{equation}
    \caf \equiv C_2(\mathtt{n}) = \ttr \frac{n^2 - 1}{n}
    \qquad \qquad
    \caa \equiv C_2(\mathtt{g}) = 2\ttr n
  \end{equation}
  where $ \ttr $ is the trace normalization of the generators in the fundamental representation:
  \begin{equation}
    \tr (T^a_\mathtt{n} T^b_\mathtt{n}) = \ttr \delta^{ab}
    \label{eq:diag-norm}
  \end{equation}
\end{proposition}

\begin{proofbox}
  \begin{proof}
    First consider the fundamental representation $ \mathtt{n} $. Given $ \ttr \in \R $, it is always possible to choose the generators of $ \SUn{n} $ such that \eref{eq:diag-norm} holds.

    Contracting Eq. \ref{eq:quad-cas} with $ \delta^{ab} $ (with $ a,b = 1,\dots, n^2 - 1 $, as they label the basis of $ \mathfrak{su}(n) $) then:
    \begin{equation*}
      C_2(\mathtt{n}) n = \ttr (n^2 - 1)
    \end{equation*}
    To compute the Casimir operator for the adjoint representation, first consider the decomposition of the direct product of two representations:
    \begin{equation*}
      \mathtt{r}_1 \otimes \mathtt{r}_2 = \bigoplus_i \mathtt{r}_i
    \end{equation*}
    In this representation $ T^a_{\mathtt{r}_1 \otimes \mathtt{r}_2} = T^a_{\mathtt{r}_1} \otimes \id_{\mathtt{r}_2} + \id_{\mathtt{r}_1} \otimes T^a_{\mathtt{r}_2} $, and it acts on tensor objects $ \Xi_{pq} $ whose first index transforms according to $ \mathtt{r}_1 $ and second index according to $ \mathtt{r}_2 $. Recalling that $ \tr{T^a} = 0 $:
    \begin{equation*}
      \begin{split}
        \tr (T^a_{\mathtt{r}_1 \otimes \mathtt{r}_2})^2
        &= \tr ((T^a_{\mathtt{r}_1})^2 \otimes \id_{\mathtt{r}_2} + 2 T^a_{\mathtt{r}_1} \otimes T^a_{\mathtt{r}_2} + \id_{\mathtt{r}_1} \otimes (T^a_{\mathtt{r}_2})^2) \\
        &= \tr (C_2(\mathtt{r}_1) \id_{\mathtt{r}_1} \otimes \id_{\mathtt{r}_2}) + \tr (C_2(\mathtt{r}_2) \id_{\mathtt{r}_1} \otimes \id_{\mathtt{r}_2}) = (C_2(\mathtt{r}_1) + C_2(\mathtt{r}_2)) n_{\mathtt{r}_1} n_{\mathtt{r}_2}
      \end{split}
    \end{equation*}
    However, by the decomposition above:
    \begin{equation*}
      \tr (T^a_{\mathtt{r}_1 \otimes \mathtt{r}_2})^2 = \sum_i C_2(\mathtt{r}_i) n_{\mathtt{r}_i}
    \end{equation*}
    Consider $ \mathtt{n} \otimes \mathtt{n}^* $, where $ \mathtt{n}^* $ is the complex conjugate of the fundamental representation (for complex representations, $ \mathtt{r} $ and $ \mathtt{r}^* $ are generally inequivalent representations): then $ \Xi_{pq} $ contains a term proportional to the invariant $ \delta_{pq} $, while the other $ n^2 - 1 $  independent components transform as a general $ n \times n $ traceless tensor, i.e. under the adjoint representation of $ \SUn{n} $ (as of \eeref{eq:sun-herm}{eq:sun-trace}), thus $ \mathtt{n} \otimes \mathtt{n}^* = \ve{1} \oplus \mathtt{g} $ and the above identity becomes:
    \begin{equation*}
      (C_2(\ve{1}) + C_2(\mathtt{g})) (n^2 - 1) = (C_2(\mathtt{n}) + C_2(\mathtt{n}^*)) n^2
    \end{equation*}
    Using $ C_2(\ve{1}) = 0 $ (as all generators are trivially zero) and $ C_2(\mathtt{n}^*) = C_2(\mathtt{n}) $:
    \begin{equation*}
      C_2(\mathtt{g}) (n^2 - 1) = 2\ttr \frac{n^2 - 1}{n} n^2
    \end{equation*}
    which completes the proof.
  \end{proof}
\end{proofbox}

A usual choice for $ \ttr $ is $ \ttr = \frac{1}{2} $.

\subsection{QCD amplitudes}

The $ m $ external partons in the amplitude $ \mat_m $ each carry two indices: a colour index and a spin index. Colour indices are denoted by $ c_1, \dots, c_m $: for gluons $ c_i \equiv a_i \in \{1, \dots, n_c^2 - 1\} $, as the field-strength tensor \eref{eq:cov-der-field-str} transforms according to the adjoint representation of the gauge group, while for quarks $ c_i \equiv \alpha_i \in \{1, \dots, n_c\} $, as their Dirac fields transform according to the fundamental representation of the gauge group. Spin indices, on the other hand, are denoted by $ s_1, \dots, s_m $, and they need to take into account how helicities change in CDR: for gluons $ s_i \equiv \mu_i \in \{1, \dots, d\} $, while for quarks $ s_i \in \{1,2\} $.

Consider the $ m $-parton colour-space $ \hilb_c $ and helicity-space $ \hilb_s $, and introduce an orthonormal basis in each:
\begin{equation*}
  \{\ket{c_1, \dots, c_m}\} \in \hilb_c
  \qquad \qquad
  \{\ket{s_1, \dots, s_m}\} \in \hilb_s
\end{equation*}
Note that, being these finite-dimensional Hilbert spaces, the non-canonical (basis-dependent) isomorphisms $ \hilb_c \leftrightarrow \hilb_c^* $ and $ \hilb_s \leftrightarrow \hilb_s^* $ are well-defined\footnotemark.

\footnotetext{Given two $ \K $-vector spaces $ V,W $, the set of all $ \K $-linear functions $ V \rightarrow W $ is denoted by $ \Hom(V,W) $; for finite-dimensional spaces $ \dim_\K \Hom(V,W) = \dim_\K V \cdot \dim_\K W $. The \textit{dual space} is defined as $ V^* \defeq \Hom(V,\K) $. \\
If $ V $ is finite-dimensional, given a basis $ \{v_i\}_{i = 1,\dots,n} \subset V $, with $ n = \dim_\K V $, then a basis $ \{\omega^1, \dots, \omega^n\} \subset V^* $ is defined by $ \omega^i(v_j) = \delta^i_j $, and the function $ \varphi : V \rightarrow V^* : v_i \mapsto \omega^i $ is a \textit{non-canonical isomorphism} $ V \leftrightarrow V^* $. \\
If $ V $ is infinite-dimensional, instead, given a basis $ \{v_i\}_{i \in \mathcal{I}} \subset V $, the above construction only allows to define linearly-independent subsets of $ V^* $, which are not granted to be bases.}

Then, to explicit the colour-helicity structure of the $ m $-parton amplitude, we define it as an abstract vector in $ \hilb_c \otimes \hilb_s $, so that:
\begin{equation}
  \mat_m^{\{c_1, \dots, c_m\}, \{s_1, \dots, s_m\}}(\{p_1, \dots, p_m\}) \equiv \braket{\{c_1, \dots, c_m\}, \{s_1, \dots, s_m\} | \mat_m(\{p_1, \dots, p_m\})}
\end{equation}
with:
\begin{equation*}
  \ket{\{c_1, \dots, c_m\}, \{s_1, \dots, s_m\}} \equiv \ket{c_1, \dots, c_m} \otimes \ket{s_1, \dots, s_m}
\end{equation*}
Hence, it is clear that the squared amplitude summed over colours and helicities is:
\begin{equation}
  \abs{\mat_m}^2 = \braket{\mat_m | \mat_m}
\end{equation}
To represent colour interactions at QCD vertices, we associate to each parton $ i $ a colour charge $ \ve{T}_i = \{T_i^a\}_{a = 1, \dots, n_c^2 - 1} $ related to the emission of a gluon. The action of $ \ve{T}_i $ onto $ \hilb_c $ is defined by:
\begin{equation}
  \braket{c_1, \dots, c_i, \dots, c_m | T_i^a | b_1, \dots, b_i, \dots, b_m} = \delta_{c_1, b_1} \dots T^a_{c_i b_i} \dots \delta_{c_m, b_m}
  \label{eq:t-act}
\end{equation}
So, $ \{T^a_{c_i b_i}\}_{a = 1, \dots, n_c^2 - 1} $ form a vector with respect to the colour index $ a $ of the emitted gluon, and they are matrices in different representations of $ \SUn{n_c} $, depending on the parton $ i $:
\begin{itemize}
  \item if $ i $ is a gluon, then $ T^a_{cb} \equiv i f_{cab} $ (adjoint representation);
  \item if $ i $ is a final-state quark, then $ T^a_{\alpha \beta} \equiv t^a_{\alpha \beta} $ (fundamental representation), while if it is a final-state antiquark $ T^a_{\alpha \beta} \equiv - t^a_{\alpha \beta} $;
  \item if $ i $ is an initial-state quark, by crossing-symmetry $ T^a_{\alpha \beta} \equiv - t^a_{\alpha \beta}$, while if it is an initial-state antiquark $ T^a_{\alpha \beta} \equiv t^a_{\alpha \beta} $.
\end{itemize}

\begin{observation}{Colour-charge algebra}{}
  First of all, we set:
  \begin{equation}
    \ve{T}_i \cdot \ve{T}_j \equiv \sum_{a = 1}^{n_c^2 - 1} T_i^a T_i^b
    \label{eq:t-dot}
  \end{equation}
  Moreover, by the action \eref{eq:t-act}, it is clear that charges associated to different partons commute, i.e.:
  \begin{equation}
    \ve{T}_i \cdot \ve{T}_j = \ve{T}_j \cdot \ve{T}_i
    \qquad
    \forall i \neq j \in \{1, \dots, m\}
  \end{equation}
  Finally, by \eeref{eq:quad-cas}{eq:t-dot}:
  \begin{equation}
    \ve{T}_i^2 = C_i \id_{\hilb_c}
  \end{equation}
  with $ C_i \equiv \caf $ if $ i $ is a quark/antiquark and $ C_i \equiv \caa $ if it is a gluon.
\end{observation}

As each vector $ \ket{\mat_m} $ is a colour-singlet, colour conservation implies:
\begin{equation}
  \sum_{i = 1}^{m} \ve{T}_i \ket{\mat_m} = 0
\end{equation}
This allows to partially (or fully, if $ m = 2 $ or $ m = 3 $, as in Appendix A of \cite{Catani-1997}) factorize the colour-charge algebra in terms of quadratic Casimir operators.










