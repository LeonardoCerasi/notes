\selectlanguage{english}

The Standard Model of Particle Physics (SM) is, as of now, the most complete theoretical framework in subatomic physics, describing all known elementary particles and fundamental interactions \cite{Glashow-1961, Salam-1964, Weinberg-1967, Fritzsch-1972, Fritzsch-1973, Higgs-1964-1, Higgs-1964-2, Englert-1964, Guralnik-1964}, except for the very weak gravitational force. Over the last fifty years, the SM has been continuously tested via experiments, especially in the context of particle colliders, and its validity has been confirmed by the agreement of its predictions with experimental observations: this process culminated in 2012 with the discovery of the Higgs boson \cite{ATLAS-2012, CMS-2012} at the Large Hadron Collider (LHC) at CERN.

Despite its success, there is strong evidence for the existence of Physics Beyond the SM (BSM): the most prominent indications include the existence of dark matter and dark energy, the observed matter-antimatter asymmetry and the non-vanishing neutrino masses. Contrary to earlier expectations, though, since its first run in 2009 the LHC has not yet detected any new particle, nor any confirmation of BSM physics: on the contrary, the huge amount of data collected in its three runs (currently Run 3 is ongoing) puts increasingly stricter exclusion limits to BSM models \cite{CMS-ATLAS-SUSY, Bsekidt-2012, Ghosh-2025, Crivellin-2015}. As a consequence, the masses of hypothesized new particles become so large that, although still not excluded, their frequent production at the LHC is hardly possible.

The lack of any observation of BSM physics at the LHC has sparked a change in the research paradigm in High-Energy Particle Physics. Substantial further increase in the energy of colliding particles at the LHC (or anywhere else) is currently not feasible, hence it is clear that BSM physics searches based on the idea of detectable resonant-like structures on top of flat backgrounds has to be supplemented by new search strategies. Indeed, new particles can still be produced at the LHC, though in a way which does not allow for their direct detection: undetected light particles could be hidden in complex final states, while heavy particles could be virtually produced for extremely short periods of time, before disappearing back into the quantum vacuum. In the latter case, these virtual particles could affect measurable properties, prompting their indirect detection as deviations from SM predictions.

Given this shift of focus towards higher experimental precision in collider physics, it is clear that reliable theoretical predictions of hadron-collision processes is needed.
