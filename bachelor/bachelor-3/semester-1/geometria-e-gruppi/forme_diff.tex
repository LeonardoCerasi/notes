\selectlanguage{italian}

\section{Wedge product}\label{sec-wp}

\begin{definition}
	Si definisce 1\textit{-form} una contrazione di un tensore $ a_k $ di rango 1 con il differenziale:
	\begin{equation}
		\omega_1 \defeq a_k dx^k
		\label{eq:3.1}
	\end{equation}
\end{definition}

Le 1-form sono invarianti per cambi di SR:

\begin{equation}
	\tilde{\omega}_1 = \tilde{a}_k dy^k = \frac{\pa x^i}{\pa y^k} a_i \frac{\pa y^k}{\pa x^j} dx^j = \delta^i_j a_i dx^j = a_i dx^i = \omega_1
	\label{eq:3.2}
\end{equation}

È possibile definire anche le 0-form, ma esse coincidono con i campi scalari.

È possibile definire un prodotto antisimmetrico tra forme, il wedge product:
\begin{equation}
	\begin{split}
		dx \wedge dy = - dy \wedge dx \\
		dx \wedge dx = dy \wedge dy = 0
	\end{split}
	\label{eq:3.3}
\end{equation}

L'antisimmetria è necessaria affinché le coordinate trasformino bene sotto cambio di SR.\\
Geometricamente, il wedge product è associato ad un'area geometrica; ad esempio, in uno spazio a due dimensioni con cambio di SR $ (x,y) \rightarrow (u,v) $:

\begin{equation}
	du \wedge dv = \left( \frac{\pa u}{\pa x} dx + \frac{\pa u}{\pa v} dy \right) \wedge \left( \frac{\pa v}{\pa x} dx + \frac{\pa v}{\pa y} dy \right) = \underbrace{\left( \frac{\pa u}{\pa x} \frac{\pa v}{\pa y} - \frac{\pa v}{\pa x} \frac{\pa u}{\pa y} \right)}_{\det\tens{J}} dx \wedge dy
	\label{eq:3.4}
\end{equation}

che è proprio la legge di trasformazione delle aree ricavate dall'Analisi.\\
Generalizzando ad $ n $ dimensioni:

\begin{equation}
	dy_1 \wedge \dots \wedge dy_n = \left( \det\tens{J} \right) dy_1 \wedge \dots \wedge dy_n
	\label{eq:3.5}
\end{equation}

Questa equazione è valida in generale e non solo per sistemi ortogonali (in tal caso non ci sarebbero i wedge products).

\begin{definition}
	Si definisce 2\textit{-form} la contrazione di un tensore di rango 2 con un doppio wedge product:
	\begin{equation}
		\omega_2 \defeq \frac{1}{2} a_{ij} dx^i \wedge dx^j
		\label{eq:3.6}
	\end{equation}
\end{definition}

\begin{definition}\label{p-form}
	Si definisce $ p $\textit{-form} la contrazione di un tensore di rango $ p $:
	\begin{equation}
		\omega_p \defeq \frac{1}{p!} a_{i_1 \dots i_p} dx^{i_1} \wedge \dots \wedge dx^{i_p}
		\label{eq:3.7}
	\end{equation}
\end{definition}

\section{Derivata esteriore}

È possibile definire un'operatore differenziale tra forme, la derivata esteriore $ d $:

\begin{equation}
	d\omega_p = d\left( a_{i_1 \dots i_p} dx^{i_1} \wedge \dots \wedge dx^{i_p} \right) \defeq \pa_k a_{i_1 \dots i_p} dx^k \wedge dx^{i_1} \wedge \dots \wedge dx^{i_p} \equiv \omega_{p+1}
	\label{eq:3.8}
\end{equation}

Dunque, la derivata esteriore trasforma una $ p $-form in una $ (p+1) $-form.

\begin{example}
	Data un campo scalare (0-form) $ f $, il suo differenziale è $ df = \frac{\pa f}{\pa x^k} dx^k $.
\end{example}

\begin{example}
	Data una 1-form su $ \R^3 $ $ A = A_x dx + A_y dy + A_z dz $, si trova che la sua derivata esteriore è $ dA = \left( \nabla\times\ve{A} \right)_x dy \wedge dz + \left( \nabla\times\ve{A} \right)_y dz \wedge dx + \left( \nabla\times\ve{A} \right)_z dx \wedge dy $, dove si è definito $ \ve{A} = \left( A_x, A_y, A_z \right) $.
\end{example}

\begin{proposition}
	Il numero di componenti indipendenti di una $ p $-form su uno spazio di dimensione $ n $ è $ \binom{n}{p} $.
\end{proposition}

Ciò deriva dall'antisimmetria del wedge product, il quale può essere espresso tramite prodotti diretti:

\begin{equation}
	dx^i \wedge dx^j \defeq dx^i \otimes dx^j - dx^j \otimes dx^i
	\label{eq:3.9}
\end{equation}

In forma sintetica, si può scrivere:

\begin{equation}
	dx^{i_1} \wedge \dots \wedge dx^{i_p} \equiv dx^{[i_1} \wedge \dots \wedge dx^{i_p]}
	\label{eq:3.10}
\end{equation}

dove le parentesi quadre indicano il prodotto antisimmetrico (somma delle permutazioni pari e sottrazione di quelle dispari). Da ciò si vede il perché del fattore $ \frac{1}{p!} $ nella Def. \ref{p-form}: partendo da un tensore senza simmetrie, la costruzione della $ p $-form tramite il wedge product ne estrae la sola parte antisimmetrica.

\begin{example}
	Per una 2-form su $ \R^2 $: $ \omega_2 = \frac{1}{2} A_{ij} dx^i dx^j = \frac{1}{2}\left( A_{12} - A_{21} \right) dx \wedge dy $, dunque se $ A_{ij} $ è antisimmetrico ciò si riduce a $ \omega_2 = A_{12} dx \wedge dy $, mentre se esso è simmetrico si ha $ \omega_2 = 0 $.
\end{example}

\begin{proposition}\label{id-bianchi}
	Sotto le dovute ipotesi di regolarità, si ha $ d^2 \equiv 0 $.
\end{proposition}
\begin{proof}
	Data $ \omega_{p+1} = d\omega_p $:
	\begin{equation*}
		d^2 \omega_p = d\omega_{p+1} = \left( \pa_j \pa_k a_{i_1 \dots i_p} \right) dx^j \wedge dx^k \wedge dx^{i_1} \wedge \dots \wedge dx^{i_p}
	\end{equation*}
	Per il lemma di Schwarz $ \pa_j \pa_k f $ è simmetrico per $ f $ regolare, dunque per l'Eq. \ref{eq:3.9} si ha $ d^2 \omega_p = 0 $.
\end{proof}

$ d^2 = 0 $ è un modo elegante di scrivere l'identità di Bianchi: per rompere questa identità è necessaria una perturbazione dello spazio considerato, detta difetto topologico.\\
È possibile dare un'ulteriore caratterizzazione delle forme differenziali.

\begin{definition}
	Una $ p $-form $ \omega_p $ si dice chiusa se $ d\omega_p = 0 $.
\end{definition}

\begin{definition}
	Una $ p $-form $ \omega_p $ si dice esatta se esiste una $ (p-1) $-form $ \omega_{p-1} $ tale che $ \omega_p = d\omega_{p-1} $.
\end{definition}

\begin{proposition}
	Una forma esatta è anche chiusa.
\end{proposition}
\begin{proof}
	Direttamente dalla Prop. \ref{id-bianchi}.
\end{proof}

\begin{lemma}[di Poincaré]
	Su una varietà semplicemente connessa, una forma chiusa è anche esatta.
\end{lemma}

\begin{theorem}[di Stokes]\label{th-stokes}
	Data una $ p $-form $ \omega $ ed un dominio semplicemente connesso $ S $, si ha:
	\begin{equation}
		\int_{\pa S} \omega = \int_S d\omega
		\label{eq:3.11}
	\end{equation}
\end{theorem}

Si dimostra anche che la derivata esteriore anticommuta con i differenziali.

\begin{proposition}
	Data una $ p $-form $ \omega_p $ ed una generica forma $ \varsigma $, si ha:
	\begin{equation}
		d(\omega_p \wedge \varsigma) = d\omega_p \wedge \varsigma + (-1)^p \omega_p \wedge d\varsigma
		\label{eq:3.12}
	\end{equation}
\end{proposition}
\begin{proof}
	Nel caso di due 1-fom $ \omega = a_i dx^i $ e $ \varsigma = b_j dx^j $:
	\begin{equation*}
		\begin{split}
			d(\omega \wedge \varsigma) &= d\left( a_i  dx^i \wedge B_j dx^j \right) = \pa_k (a_i b_j) dx^k \wedge dx^i \wedge dx^j\\
						   &= (\pa_k a_i) b_j dx^k \wedge dx^i \wedge dx^j + a_i (\pa_k b_j) dx^k \wedge dx^i \wedge dx^j\\
						   &= (\pa_k a_i) b_j dx^k \wedge dx^i \wedge dx^j - a_i (\pa_k b_j) dx^i \wedge dx^k \wedge dx^j\\
						   &= \left( \pa_k a_i dx^k \wedge dx^i \right) \wedge b_j dx^j - a_i dx^i \wedge \left( \pa_k b_j dx^k \wedge dx^j \right)\\
						   &= d\omega \wedge \varsigma - \omega \wedge d\varsigma
		\end{split}
	\end{equation*}
\end{proof}

\section{Operatore di Hodge}

\subsection{Tensore di Levi-Civita}

Il simbolo di Levi-Civita tridimensionale (Def. \ref{def-lc}) è completamente antisimmetrico ed isotropico (Eq. \ref{eq:11}), ma è possibile renderlo un tensore di rango 3 tramite il determinante della metrica.

\begin{definition}
	Si definisce \textit{tensore di Levi-Civita} $ \varepsilon_{ijk} \defeq \sqg \,\epsilon_{ijk} $.
\end{definition}

Per trovare la legge di trasformazione del tensore di Levi-Civita, sono necessari alcuni risultati di Algebra Lineare.

\begin{theorem}[di Laplace]
	Data $ \tens{M}\in\R^{n\times n} $, si ha $ \det\tens{M} = \epsilon^{i_1 \dots i_n} M_{i_1 1} \dots M_{i_n n} $.
\end{theorem}
\begin{corollary}\label{cor-lap}
	$ \epsilon_{k_1 \dots k_n} \det\tens{M} = \epsilon^{i_1 \dots i_n} M_{i_1 k_1} \dots M_{i_n k_n} $.
\end{corollary}
\begin{corollary}
	$ \det \left( \tens{M}_1 \dots \tens{M}_k \right) = \left( \det\tens{M}_1 \right) \dots \left( \det\tens{M}_k \right) $.
\end{corollary}

Considerando un passaggio di RF $ x^i \mapsto y^i $:
\begin{equation*}
	\begin{split}
		\varepsilon'_{ijk} &= \underbrace{\sqrt{\tens{g}'} \epsilon'_{ijk} = \sqrt{\tens{g}'} \epsilon_{ijk}}_{\text{isotropia di }\epsilon_{ijk}} = \sqrt{\abs{ \det \left( \frac{\pa x^s}{\pa y^m} \frac{\pa x^t}{\pa y^n} g_{st} \right) }} \epsilon_{ijk}\\
				   &= \sqrt{\abs{ \det \left( \frac{\pa x^s}{\pa y^m} \right) \det \left( \frac{\pa x^t}{\pa y^n} \right) \det \left( g_{st} \right) }} \epsilon_{ijk} = \abs{\det \left( \frac{\pa x}{\pa y} \right)} \sqg \epsilon_{ijk} = \sigma \sqg \underbrace{\det \left( \frac{\pa x}{\pa y} \right) \epsilon_{ijk}}_{\text{Cor. \ref{cor-lap}}}\\
				   &= \sigma \sqg \frac{\pa x^s}{\pa y^i} \frac{\pa x^t}{\pa y^j} \frac{\pa x^u}{\pa y^k} \epsilon_{stu} = \sigma \frac{\pa x^s}{\pa y^i} \frac{\pa x^t}{\pa y^j} \frac{\pa x^u}{\pa y^k} \varepsilon_{stu}
	\end{split}
\end{equation*}
Essendo $ \varepsilon_{ijk} $ definito a partire da $ \epsilon_{ijk} $ (dunque da un prodotto antisimmetrico), esso è uno pseudotensore e di conseguenza la sua trasformazione dipende da $ \sigma $, il segno del determinante dello Jacobiano. Si ricordi che $ \det\tens{J} $ cambia di segno per ogni inversione nella trasformazione di RF.\\
Le componenti controvarianti del tensore di Levi-Civita sono:
\begin{equation}
	\varepsilon^{ijk} = g^{is} g^{jt} g^{ku} \varepsilon_{stu} = g^{is} g^{jt} g^{ku} \sqg\, \epsilon_{stu} = \sqg \det \left( g^{mn} \right) \epsilon_{ijk} = \frac{\sqg}{\det \left( g_{mn} \right)} \epsilon_{ijk} = \frac{\sigma}{\sqg} \epsilon_{ijk}
	\label{eq:3.16}
\end{equation}
dato che, per metriche generiche (non definite positive), $ \tens{g} \defeq \abs{\det \left( g_{mn} \right)} $.\\
Tutte queste proprietà di $ \varepsilon_{ijk} $ sono valide anche per il generico tensore di Levi-Civita $ n $-dimensionale, definito analogamente come:
\begin{equation}
	\varepsilon_{i_1 \dots i_n} \defeq \sqg \, \epsilon_{i_1 \dots i_n}
	\label{eq:3.17}
\end{equation}

\subsection{Hodge's star}

L'operatore di Hodge, anche detto duale di Hodge o Hodge's star, è un operatore che mappa una $ p $-form in una $ (n - p) $-form, dove $ n $ è la dimensione della varietà considerata.
\begin{definition}
	Data una varietà differenziale $ n $-dimensionale $ \mathcal{M} $, si definisce l'\textit{operatore di Hodge} $ * : \bigwedge^p \mathcal{M} \rightarrow \bigwedge^{n - p} \mathcal{M} $ tale che:
	\begin{equation}
		* \left( dx^{i_1} \wedge \dots \wedge dx^{i_p} \right) \defeq \frac{1}{(n - p)!} g^{i_1 k_1} \dots g^{i_p k_p} \varepsilon_{k_1 \dots k_p m_1 \dots m_{n - p}} dx^{m_1} \wedge \dots \wedge dx^{m_{n - p}}
		\label{eq:3.18}
	\end{equation}
\end{definition}

\begin{example}
	Su $ \R^3 $ si ha $ *1 = dx \wedge dy \wedge dz $, $ * \left( dx \wedge dy \wedge dz \right) = 1 $, $ *dx = dy \wedge dz $, $ * \left( dx \wedge dy \right) = dz $ e così via. Inoltre, si trovano due importanti proprietà:
	\begin{equation*}
		*df = * \left( \frac{\pa f}{\pa x}dx + \frac{\pa f}{\pa y}dy + \frac{\pa f}{\pa z}dz \right) = \frac{\pa f}{\pa x} dy \wedge dz + \frac{\pa f}{\pa y} dz \wedge dx + \frac{\pa f}{\pa z} dx \wedge dy
	\end{equation*}
	\begin{equation*}
		*\ve{F} = * \left( F_x dx + F_y dy + F_z dz \right) = F_x dy \wedge dz + F_y dz \wedge dx + F_z dx \wedge dy
	\end{equation*}
	Si vede dunque che:
	\begin{equation}
		d*df = \lap f \, dx \wedge dy \wedge dz
		\label{eq:3.19}
	\end{equation}
	\begin{equation}
		d*\ve{F} = \nabla\cdot\ve{F} \, dx \wedge dy \wedge dz
		\label{eq:3.20}
	\end{equation}
\end{example}
\begin{example}
	Su $ \R^{1,3} $ bisogna tener conto del segno negativo della coordinata temporale; ad esempio $ *1 = dt \wedge dx \wedge dy \wedge dz $, ma $ * \left( dt \wedge dx \wedge dy \wedge dz \right) = -1 $, inoltre per le 1-form:
	\begin{align*}
		*dt &= - dx \wedge dy \wedge dz & *dx &= - dy \wedge dz \wedge dt \\
		*dy &= - dz \wedge dx \wedge dt & *dz &= - dz \wedge dy \wedge dt
	\end{align*}
	Per le 2-form:
	\begin{align*}
		*(dx \wedge dt) &= dy \wedge dz & *(dx \wedge dy) &= - dz \wedge dt \\
		*(dy \wedge dt) &= dz \wedge dx & *(dy \wedge dz) &= - dx \wedge dt \\
		*(dz \wedge dt) &= dx \wedge dy & *(dz \wedge dx) &= - dy \wedge dt
	\end{align*}
	Per le 3-form:
	\begin{align*}
		*(dx \wedge dy \wedge dz) &= - dt & *(dy \wedge dz \wedge dt) &= -dx \\
		*(dz \wedge dx \wedge dt) &= - dy & *(dx \wedge dy \wedge dt) &= -dz
	\end{align*}
\end{example}










