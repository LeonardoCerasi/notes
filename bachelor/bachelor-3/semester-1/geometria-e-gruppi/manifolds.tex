\selectlanguage{italian}

In maniera informale, una varietà è un oggetto che su larga scala può avere proprietà di curvatura non banali, ma che localmente appare come uno spazio euclideo (piatto).\\
Si vede immediatamente una similitudine con il principio di equivalenza di Einstein: le leggi fisiche si riducono a quelle della relatività speciale in intorni abbastanza piccoli dello spaziotempo; dato che la gravità emerge dalle proprietà di curvatura dello spaziotempo, ciò è equivalente a dire che lo spaziotempo è descritto da una varietà differenziale.

\section{Varietà differenziali}

\begin{definition}
	Si definisce \textit{spazio euclideo} $ \R^n \defeq \{(x^1,\dots,x^n) : x^j \in \R\} $ dotato della metrica piatta definita positiva $ g_{ij} = \delta_{ij} $.
\end{definition}

L'identificazione locale di una varietà con $ \R^n $ non equivale ad un'uguaglianza di metriche, ma permette di definire in maniera equivalente le nozioni di funzioni, coordinate, operatori etc.

\begin{example}
	Esempi di varietà sono la $ n $-sfera $ \mathbb{S}^n $, ovvero il luogo dei punti in $ \R^{n+1} $ a distanza unitaria dall'origine, e l'$ n $-toro $ \mathbb{T}^n $, generato facendo coincidere i lati opposti di un $ n $-cubo.
\end{example}
\begin{example}
	Si definisce \textit{superficie di Riemann} una varietà differenziabile 2-dimensionale orientabile, compatta e senza confini; le superfici di Riemann possono essere caratterizzate dal loro genus $ g $, ovvero dal numero di buchi di tale varietà: condizione necessaria perché due superfici di Riemann siano omotope è che abbiano lo stesso genus.
\end{example}
\begin{example}
	Il gruppo di rotazioni continue in $ \R^n $ forma una varietà: in generale, i \textit{gruppi di Lie} sono varietà dotate di struttura di gruppo; ad esempio $ \text{SO}(2) \cong \mathbb{S}^1 $.
\end{example}

\subsection{Carte e atlanti}

La varietà nella sua interezza viene costruita attaccando in modo liscio le varie regioni localmente piatte che la compongono.

\begin{definition}
	Dati due insiemi $ M,N $, una \textit{mappa} $ \varphi : M \rightarrow N $ è una relazione che assegna ad ogni elemento di $ M $ al più un elemento di $ N $. $ M $ è detto \textit{dominio} e $ \varphi(M) $ \textit{immagine}.
\end{definition}

\begin{definition}
	Una mappa $ \varphi : M \rightarrow N $ si dice \textit{iniettiva} se ad ogni elemento dell'immagine corrisponde soltanto un elemento del dominio.
\end{definition}

\begin{definition}
	Una mappa $ \varphi : M \rightarrow N $ si dice \textit{suriettiva} se $ \varphi(M) = N $.
\end{definition}

\begin{definition}
	Una mappa $ \varphi : M \rightarrow N $ si dice \textit{biiettiva} se è iniettiva e suriettiva.
\end{definition}

\begin{definition}
	Data $ \varphi : M \rightarrow N $ biiettiva, si definisce la mappa inversa $ \varphi^{-1} : N \rightarrow M $ tale che $ \varphi \circ \varphi^{-1} = \text{id}_N $ e $ \varphi^{-1} \circ \varphi = \text{id}_M $.
\end{definition}

Si ricordi che una mappa tra spazi euclidei $ \varphi : \R^m \rightarrow \R^n $ è definita come:
\begin{equation}
	\varphi : (x^1,\dots,x^m) \mapsto (\varphi^1(x^1,\dots,x^m),\dots,\varphi^n(x^1,\dots,x^m))
	\label{eq:6.1}
\end{equation}

\begin{definition}
	Data una mappa $ \varphi : \R^m \rightarrow \R^n $, essa si dice \textit{di classe} $ \mathcal{C}^p $, $ p \ge 0 $ se per ognuna delle $ \varphi^i $ tutte le derivate $ n $-esime esistono continue, con $ n \le p $.
\end{definition}

\begin{definition}
	Una mappa $ \varphi \in \mathcal{C}^{\infty}(\R^m,\R^n) $ si dice \textit{liscia}.
\end{definition}

\begin{definition}
	Due insiemi $ M,N $ si dicono diffeomorfi, e si scrive $ M \cong N $, se esiste un diffeomorfismo tra loro, ovvero una mappa biiettiva $ \varphi \in \mathcal{C}^{\infty}(M,N) $ con inversa $ \varphi^{-1} \in \mathcal{C}^{\infty}(N,M) $.
\end{definition}

La nozione di diffeomorfismo permette di stabilire se due spazi rappresentano la stessa varietà: un diffeomorfismo è un isomorfismo tra varietà lisce.

\begin{example}
	$ \text{SO}(2) \cong \mathbb{S}^1 $.
\end{example}

\begin{definition}
	Dati $ \ve{x} \in \R^n $ e $ r \in \R^+ $, si definisce la \textit{palla aperta} centrata in $ \ve{x} $ di raggio $ r $ l'insieme $ \mathcal{B}_r(\ve{x}) \defeq \{ \ve{y} \in \R^n : \abs{\ve{y} - \ve{x}} < r \} $, con norma determinata dalla metrica.
\end{definition}

\begin{definition}
	$ V \subset \R^n $ si dice \textit{aperto} se $ \forall \ve{x} \in V \, \exists r \in \R^+ : \mathcal{B}_r(\ve{x}) \subset V $.
\end{definition}

Un aperto può quindi essere visto come un'unione di palle aperte, o anche come la parte interna di una o più superfici $ (n-1) $-dimensionali.

\begin{definition}
	Una \textit{carta} su un insieme $ \mathcal{M} $ consiste di un insieme $ U \subset \mathcal{M} $ ed una mappa $ \varphi : U \rightarrow \R^n $ tali per cui $ \varphi(U) \subset \R^n $ è aperto.
\end{definition}

$ \R^n $ è detto embedding space di $ \mathcal{M} $.

\begin{definition}
	Un \textit{atlante} su un insieme $ \mathcal{M} $ è un insieme di carte $ \mathcal{A} = \{U_{\alpha},\varphi_{\alpha}\} $ tali che la loro unione ricopra $ \mathcal{M} $ e siano componibili in maniera liscia: se $ U_{\alpha} \cap U_{\beta} \neq \emptyset $, allora le mappe $ \varphi_{\alpha} \circ \varphi_{\beta}^{-1} $ e $ \varphi_{\beta} \circ \varphi_{\alpha}^{-1} $ sono diffeomorfismi.
\end{definition}

Le mappe di transizione non sono altro che cambiamenti di coordinate sulle intersezioni.

\begin{definition}
	Si dice \textit{varietà differenziale} un insieme $ \mathcal{M} $ con un atlante massimale $ \mathcal{A} $ su di esso.
\end{definition}

Si richiede un atlante massimale per evitare che due varietà con due atlanti, uno massimale ed uno no, vengano identificate come diverse quando in realtà sono diffeomorfe.

\begin{theorem}[di Whitney]
	Una varietà $ n $-dimensionale può essere embedded in uno spazio al più $ 2n $-dimensionale.
\end{theorem}

\begin{example}
	La bottiglia di Klein è 2-dimensionale e non può essere embedded in $ \R^3 $, ma in $ \R^4 $ sì.
\end{example}

Bisogna considerare che la maggior parte delle varietà non può essere ricoperta da una sola carta.

\begin{example}
	Si consideri $ \mathbb{S}^1 $ con l'usuale scelta di coordinata $ \varphi : p \in \mathbb{S}^1 \mapsto \theta \in [0,2\pi] \subset \R $: se vengono inclusi $ 0 $ o $ 2\pi $, si ha che $ \varphi(\mathbb{S}^1) $ non è aperto in $ \R $, dunque sono necessarie due carte, una che escluda $ \theta = 0,2\pi $ ed una che escluda $ \theta = \pi $.
\end{example}

\begin{example}
	Si consideri $ \mathbb{S}^2 $ con proiezione stereografica (o di Mercator): dato che la proiezione della sfera $ (x^1)^2 + (x^2)^2 + (x^3)^2 = 1 $ su $ x^3 = -1 $ è definita a meno del polo nord, sono necessarie due carte: la proiezione stereografica su $ x^3 = -1 $, che esclude il polo nord, e quella su $ x^3 = 1 $, che esclude il polo sud. Definendo $ y^1,y^2 $ e $ z^1,z^2 $ le coordinate sui piani considerati, si trovano:
	\begin{equation*}
		\varphi_{-1} (x^1,x^2,x^3) = \left( \frac{2x^1}{1 - x^3}, \frac{2x^2}{1-x^3} \right) \equiv (y^1,y^2)
	\end{equation*}
	\begin{equation*}
		\varphi_1 (x^1,x^2,x^3) = \left( \frac{2x^1}{1 + x^3}, \frac{2x^2}{1 + x^3} \right) \equiv (z^1,z^2)
	\end{equation*}
	Queste si intersecano su $ -1 < x^3 < 1 $ e la mappa di transizione è:
	\begin{equation*}
		\varphi_1 \circ \varphi_{-1}^{-1} (y_1,y_2) = \left( \frac{4y^1}{(y^1)^2 + (y^2)^2}, \frac{4y^2}{(y^1)^2 + (y^2)^2} \right) \equiv (z^1,z^2)
	\end{equation*}
\end{example}

\section{Spazio tangente}

\begin{definition}
	Data una varietà $ \mathcal{M} $ ed un punto $ p \in \mathcal{M} $, si definisce lo \textit{spazio tangente} a $ \mathcal{M} $ in $ p $ lo spazio $ T_p\mathcal{M} $ di tutti i vettori tangenti alla varietà in quel punto.
\end{definition}

È possibile costruire $ T_p\mathcal{M} $ in maniera equivalente: si consideri una curva passante per $ p $, ovvero una mappa $ \varphi : I \subset \R \rightarrow \mathcal{M} : \lambda \mapsto \varphi(\lambda) $ tale che $ \exists \lambda_0 \in I : \varphi(\lambda_0) = p $. Se l'embedding space di $ \mathcal{M} $ è $ \R^n $, in ogni punto dell'immagine la curva determinerà un vettore $ \tau_{\varphi}(\lambda) $ con componenti $ \frac{dx^k}{d\lambda}(\lambda) $, detto vettore tangente: la mappa $ p \mapsto \tau_{\varphi}(\lambda_0) $ dipende dalla carta scelta.\\
Definendo $ \mathcal{F} $ lo spazio di funzioni lisce su $ \mathcal{M} $, ogni curva $ \varphi $ passante per $ p $ definisce un vettore $ \frac{df}{d\lambda} $ per ogni $ f \in \mathcal{F} $, detta derivata direzionale: la mappa $ f \mapsto \nabla_{\varphi} f(\lambda_0) $ non dipende dalla carta scelta.\\
Lo spazio tangente $ T_p\mathcal{M} $ può essere visto come lo spazio degli operatori derivata direzionale lungo curve passanti per $ p $.

\begin{definition}
	Data una varietà $ \mathcal{M} $ ed un punto $ p \in \mathcal{M} $, si definisce lo \textit{spazio cotangente} a $ \mathcal{M} $ in $ p $ lo spazio $ T^*_p\mathcal{M} $ duale di $ T_p\mathcal{M} $.
\end{definition}










