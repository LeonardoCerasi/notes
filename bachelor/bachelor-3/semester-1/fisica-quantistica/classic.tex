\selectlanguage{italian}

È possibile mostrare rigorosamente come le leggi della fisica classica emergano da quelle quantistiche. La principale difficoltà è determinata dal fatto che in fisica classica la meccanica consiste nel calcolare la traiettoria di un sistema date le sue condizioni iniziali: in particolare, nella formulazione hamiltoniana la traiettoria è determinata a partire da $ q_0 $ e $ p_0 $, mentre nella formulazione lagrangiana da $ q_0 $ e $ \dot{q}_0 $. Ciò in ambito quantistico non è possibile, a causa del principio d'indeterminazione.\\
È però possibile formulare la fisica classica in maniera compatibile col principio d'indeterminazione tramite il principio d'azione, il quale determina la traiettoria a partire da $ q_0 $ e $ q_1 $ (posizioni iniziale e finale). È inoltre possibile trovare un analogo classico della funzione d'onda nella teoria di Hamilton-Jacobi.

\section{Principio d'azione classico}

\begin{definition}
	Dato un sistema classico descritto da una lagrangiana $ \mathscr{L}(q,\dot{q},t) $ che si muove lungo una traiettoria $ q(t) $, detti $ q_0 \equiv q(t_0) $ e $ q_1 \equiv q(t_1) $, si definisce la sua \textit{azione} lungo tale traiettoria come:
	\begin{equation}
		\mathcal{S}(q_0,t_0 ; q_1,t_1) \defeq \int_{t_0}^{t_1} dt\, \mathscr{L}(q(t), \dot{q}(t), t)
		\label{eq:4.1}
	\end{equation}
\end{definition}

Il \textit{principio di minima azione} (o principio di Hamilton) afferma che, fissati $ (q_0,t_0) $ e $ (q_1,t_1) $, la traiettoria percorsa dal sistema è quella che estremizza l'azione, vista come un funzionale di $ q(t) $:
\begin{equation*}
	\begin{split}
		\delta \mathcal{S}
		&= 0 \\
		&= \int_{t_0}^{t_1} dt\, \delta \mathscr{L} (q(t), \dot{q}(t), t) = \int_{t_0}^{t_1} dt \left( \frac{\pa \mathscr{L}}{\pa q} \delta q + \frac{\pa \mathscr{L}}{\pa \dot{q}} \delta \dot{q} \right) \\
		&= \int_{t_0}^{t_1} dt \left( \frac{\pa \mathscr{L}}{\pa q} \delta q + \frac{\pa \mathscr{L}}{\pa \dot{q}} \frac{d}{dt} \delta q \right) = \int_{t_0}^{t_1} dt \left( \frac{\pa \mathscr{L}}{\pa q} - \frac{d}{dt} \frac{\pa \mathscr{L}}{\pa \dot{q}} \right) \delta q + \left[ \frac{\pa \mathscr{L}}{\pa \dot{q}} \delta q \right]_{t_0}^{t_1}
	\end{split}
\end{equation*}
Essendo gli estremi della traiettoria fissati, si ha $ \delta q(t_0) = \delta q(t_1) = 0 $, dunque dall'arbitrarietà di $ \delta q $ si ottengono le \textit{equazioni di Eulero-Lagrange}:
\begin{equation}
	\frac{\pa \mathscr{L}}{\pa q} - \frac{d}{dt} \frac{\pa \mathscr{L}}{\pa \dot{q}} = 0
	\label{eq:4.2}
\end{equation}
Una volta trovata la traiettoria e fissati $ (q_0,t_0) $, si vede immediatamente che una variazione $ \delta q(t) $ della traiettoria (tale che $ \delta q(t_0) = 0 $) determina una variazione dell'azione data da:
\begin{equation}
	\delta \mathcal{S}(t) = \frac{\pa \mathscr{L}}{\pa \dot{q}} \delta q(t)
	\label{eq:4.3}
\end{equation}
Ricordando che $ p \defeq \frac{\pa \mathscr{L}}{\pa \dot{q}} $, si trova:
\begin{equation}
	p(t) = \frac{\pa \mathcal{S}(q,t)}{\pa q}
	\label{eq:4.4}
\end{equation}

\begin{definition}
	Dato un sistema classico che si muove lungo una traiettoria $ q(t) $, si definisce la \textit{funzione principale di Hamilton} come l'azione valutata lungo la traiettoria, ovvero:
	\begin{equation}
		\mathcal{S}(q,t) = \mathcal{S}(q_0,t_0,q(t),t)
		\label{eq:4.5}
	\end{equation}
\end{definition}

Dato che l'impulso del sistema è il gradiente della funzione principale lungo la traiettoria, è naturale l'associazione di $ S(q,t) $ con la funzione d'onda quantistica.

\subsection{Teoria di Hamilton-Jacobi}

Il collegamento tra la fisica classica e quella quantistica è fornito dalla teoria di Hamilton-Jacobi.

\begin{theorem}[Hamilton-Jacobi]
	Dato un sistema classico unidimensionale descritto da un'hamiltoniana $ \mathcal{H}(q,p,t) $ che si muove lungo una traiettoria $ q(t) $, la funzione principale di Hamilton soddisfa l'\textit{equazione di Hamilton-Jacobi}:
	\begin{equation}
		\frac{\pa \mathcal{S}(q,t)}{\pa t} + \mathcal{H} \left( q, \frac{\pa \mathcal{S}}{\pa q}, t \right) = 0
		\label{eq:4.6}
	\end{equation}
\end{theorem}
\begin{proof}
	Si consideri la derivata totale nel tempo della funzione principale lungo la traiettoria:
	\begin{equation*}
		\begin{split}
			\frac{d \mathcal{S}(q,t)}{dt}
			&= \mathscr{L}(q,\dot{q},t) = p \dot{q} - \mathcal{H}(q,p,t) \\
			&= \frac{\pa \mathcal{S}(q,t)}{\pa t} + \frac{\pa \mathcal{S}(q,t)}{\pa q} \dot{q} = \frac{\pa \mathcal{S}(q,t)}{\pa t} + p \dot{q}
		\end{split}
		\quad \Rightarrow \quad
		\frac{\pa \mathcal{S}(q,t)}{\pa t} = - \mathcal{H} \left( q, \frac{\pa \mathcal{S}}{\pa q}, t \right)
	\end{equation*}
\end{proof}

\begin{example}
	Si consideri un oscillatore armonico unidimensionale, descritto dall'hamiltoniana $ \mathcal{H}(q,p) = \frac{p^2}{2m} + \frac{1}{2} m \omega^2 q^2 $; l'equazione di Hamilton-Jacobi in questo caso è:
	\begin{equation*}
		\frac{\pa \mathcal{S}}{\pa t} + \frac{1}{2m} \left( \frac{\pa \mathcal{S}}{\pa q} \right)^2 + \frac{1}{2} m \omega^2 q^2 = 0
	\end{equation*}
	Essendo il potenziale indipendente dal tempo, il sistema è invariante per traslazioni temporali, dunque lungo la traiettoria l'energia si conserva: fissata la traiettoria $ q(t) $, si ha $ \mathcal{H}(q,p) = E $. Di conseguenza:
	\begin{equation*}
		\frac{\pa \mathcal{S}}{\pa t} = -E
		\quad \Rightarrow \quad
		\mathcal{S}(q,t) = -E t + W(q)
	\end{equation*}
	dove $ W(q) $ è detta \textit{funzione caratteristica di Hamilton}. Per determinare quest'ultima, si risolve $ \mathcal{H}(q,p) = E $, che ora è una ODE:
	\begin{equation*}
		\frac{1}{2m} \left( \frac{d W(q)}{d q} \right)^2 + \frac{1}{2} m \omega^2 q^2 = E
		\quad \Rightarrow \quad
		W(q) = \pm \sqrt{2mE} \int_{q_0}^q d\xi \sqrt{1 - \frac{m \omega^2}{2E} \xi^2}
	\end{equation*}
\end{example}

Questo procedimento ha carattere generale.

\begin{proposition}
	Dato un sistema classico unidimensionale descritto da un'hamiltoniana indipendente dal tempo $ \mathcal{H}(q,p) = \frac{p^2}{2m} + V(q) $ che si muove lungo una traiettoria $ q(t) $ di energia $ E $, la funzione principale di Hamilton è data da $ \mathcal{S}(q,t) = -E t + W(q) $, dove la funzione caratteristica di Hamilton è determinata come:
	\begin{equation}
		W(q) = \pm \sqrt{2m} \int_{q_0}^q d\xi \sqrt{E - V(\xi)}
		\label{eq:4.7}
	\end{equation}
\end{proposition}

Una volta determinata la funzione principale sono noti anche i momenti canonici del sistema, dunque la traiettoria.\\
Il caso multidimensionale ($ q \equiv (q_1, \dots, q_f) $) è più complicato da trattare; in generale, l'equazione da risolvere per la funzione caratteristica avrà la forma:
\begin{equation}
	(\nabla W)^2 = 2m \left( E - V(q) \right)
	\label{eq:4.8}
\end{equation}
Dato che $ p \equiv (p_1, \dots, p_f) = \nabla W(q) $ e che la traiettoria è determinata da $ v_i = \frac{1}{m} p_i $, si può vedere la traiettoria percorsa dal sistema come quella determinata da un'onda che sospinge un oggetto, poiché essa segue il cammino di minima pendenza e la sua velocità è data dalla pendenza stessa.\\
Interpretando la velocità del sistema come la velocità di gruppo, si può vedere che essa è diversa dalla velocità di fase, ovvero quella a cui si muovono i fronti d'onda, ovvero le superfici con $ \mathcal{S}(q,t) $ costante. Assumendo che il sistema (unidimensionale) sia invariante per traslazioni temporali, dato che $ \mathcal{S}(q,t) = -E t + W(q) $, il fronte d'onda $ q_0(t) $ è determinato da:
\begin{equation*}
	\frac{d \mathcal{S}(q_0(t),t)}{dt} = 0
	\quad \Rightarrow \quad
	\frac{dW(q)}{dq} \bigg\vert_{q_0} \dot{q}_0 - E = 0
\end{equation*}
Essendo la velocità di fase $ v_f \equiv \dot{q}_0 $, si trova:
\begin{equation}
	v_f(t) = \pm \frac{E}{\sqrt{2m \left[ E - V(q_0(t)) \right]}}
	\label{eq:4.9}
\end{equation}
La velocità di gruppo è invece data dall'Eq. \ref{eq:4.4} come $ v_g = \frac{1}{m} \frac{\pa W}{\pa q} $:
\begin{equation}
	v_g(t) = \pm \sqrt{\frac{2}{m} \left[ E - V(q_0(t)) \right]}
	\label{eq:4.10}
\end{equation}

\section{Principio d'azione quantistico}

In ambito quantistico, parlando di traiettoria si intende solo fissare (ovvero misurare) $ q_0 \equiv q(t_0) $ e $ q_1 \equiv q(t_1) $. Si ricordi l'operatore di evoluzione temporale:
\begin{equation}
	\ket{\psi(t)} = \hat{S}(t,t_0) \ket{\psi(t_0)}
	\label{eq:4.11}
\end{equation}
La funzione d'onda in un generico punto $ q(t) $ è dunque:
\begin{equation}
	\psi(q,t) = \braket{q | \hat{S}(t,t_0) | \psi(t_0)}
	\label{eq:4.12}
\end{equation}

\begin{definition}
	Dato un sistema quantistico con operatore di evoluzione temporale $ \hat{S}(t,t_0) $, si definisce il suo \textit{propagatore} come l'autofunzione della posizione evoluta nel tempo:
	\begin{equation}
		K(q,t ; q_0,t_0) \defeq \braket{q | \hat{S}(t,t_0) | q_0}
		\label{eq:4.13}
	\end{equation}
\end{definition}

Il propagatore non è altro che l'elemento di matrice dell'operatore di evoluzione temporale tra autostati della posizione. Si noti che non necessariamente $ t > t_0 $, è ammesso anche $ t < t_0 $: l'evoluzione temporale quantistica è deterministica ed unitaria, dunque reversibile.

\begin{proposition}
	Dato un sistema quantistico con propagatore $ K(q,t ; q_0,t_0) $, si ha:
	\begin{equation}
		\psi(q,t) = \int dq'\, K(q,t ; q',t_0) \psi(q',t_0)
		\label{eq:4.14}
	\end{equation}
\end{proposition}
\begin{proof}
	Essendo $ \int dq \ket{q}\bra{q} = \tens{I} $, si vede banalmente dall'Eq. \ref{eq:4.12}.
\end{proof}

Dato che le autofunzioni della posizione sono $ \braket{q | q_0} = \delta(q - q_0) $, si ha la conferma che $ K(q,t ; q_0,t_0) $ è proprio la funzione d'onda dell'evoluto temporale dello stato iniziale $ \ket{q_0} $: da qui l'analogia con la funzione principale di Hamilton. Il propagatore contiene tutta la dinamica del sistema.

\begin{proposition}\label{prop-ass-conv}
	Il propagatore è associativo sotto convoluzione:
	\begin{equation}
		K(q,t ; q_0,t_0) = \int dq_1\, K(q,t ; q_1,t_1) K(q_1,t_1 ; q_0,t_0)
		\label{eq:4.15}
	\end{equation}
\end{proposition}
\begin{proof}
	Ricordando l'associatività dell'evoluzione temporale $ \hat{S}(t,t_0) = \hat{S}(t,t_1) \hat{S}(t_1,t_0) $:
	\begin{equation*}
		K(q,t ; q_0,t_0) = \braket{q | \hat{S}(t,t_0) | q_0} = \int dq_1 \braket{q | \hat{S}(t,t_1) | q_1} \braket{q_1 | \hat{S}(t_1,t_0) | q_0}
	\end{equation*}
\end{proof}

\begin{proposition}
	Dato un sistema quantistico descritto da un'hamiltoniana indipendente dal tempo $ \hat{\mathcal{H}} = \frac{\hat{p}^2}{2m} + \hat{V}(\hat{q}) $, considerando una traslazione temporale infinitesima $ dt \equiv \varepsilon $ si ha il propagatore:
	\begin{equation}
		K(q',t + \varepsilon ; q,t) = \sqrt{\frac{m}{2\pi i \varepsilon \hbar}} e^{i \frac{d\mathcal{S}(t)}{\hbar}}
		\label{eq:4.16}
	\end{equation}
	dove $ d\mathcal{S}(t) $ è l'elemento infinitesimo d'azione lungo l'evoluzione temporale del sistema.
\end{proposition}
\begin{proof}
	Per un'Hamiltoniana indipendente dal tempo l'operatore di evoluzione temporale è:
	\begin{equation*}
		\hat{S}(t,t_0) = e^{\frac{1}{i\hbar} (t - t_0) \hat{\mathcal{H}}}
	\end{equation*}
	Per calcolare esplicitamente il propagatore si sfrutta la risoluzione dell'identità sugli impulsi, il che permette di sostituire gli operatori con i rispettivi autovalori (si può mostrare formalmente):
	\begin{equation*}
		\begin{split}
			K(q',t+\varepsilon ; q,t)
			&= \braket{q' | \hat{S}(t+\varepsilon,t) | q} = \braket{q' | e^{\frac{\varepsilon}{i\hbar} \hat{\mathcal{H}}} | q} = \braket{q' | \left[ 1 + \frac{\varepsilon}{i\hbar} \left( \frac{\hat{p}^2}{2m} + \hat{V}(\hat{q}) \right) + o(\varepsilon) \right] | q} \\
			&= \int dp \braket{q' | p} \braket{p | \left[ 1 + \frac{\varepsilon}{i\hbar} \left( \frac{\hat{p}^2}{2m} + \hat{V}(\hat{q}) \right) + o(\varepsilon) \right] | q} \\
			&= \int dp \braket{q' | p} \left[ 1 + \frac{\varepsilon}{i\hbar} \left( \frac{p^2}{2m} + V(q) \right) + o(\varepsilon) \right] \braket{p | q} \\
			&= \int dp\, \frac{1}{\sqrt{2\pi\hbar}} e^{\frac{i}{\hbar} p q'} e^{- \frac{i}{\hbar} ( \frac{p^2}{2m} + V(q) ) \varepsilon} \frac{1}{\sqrt{2\pi\hbar}} e^{ - \frac{i}{\hbar} pq} = \int \frac{dp}{2\pi\hbar}\, e^{\frac{i}{\hbar} p (q' - q)} e^{- \frac{i}{\hbar} ( \frac{p^2}{2m} + V(q) ) \varepsilon}
		\end{split}
	\end{equation*}
	Questo integrale non contiene operatori ed è riconducibile ad un integrale gaussiano notando che $ t \mapsto t + \varepsilon \,\Rightarrow\, q \mapsto q + \varepsilon \dot{q} \equiv q' $:
	\begin{equation*}
		\begin{split}
			K(q',t+\varepsilon ; q,t)
			&= \int \frac{dp}{2\pi\hbar}\, e^{\frac{i}{\hbar} ( p\dot{q} - \frac{p^2}{2m} - V(q) ) \varepsilon} = e^{- \frac{i}{\hbar} V(q) \varepsilon} \int \frac{dp}{2\pi\hbar}\, e^{- \frac{i}{\hbar} \frac{1}{2m} (p^2 - 2m\hbar p\dot{q}) \varepsilon} \\
			&= \frac{1}{2\pi\hbar} e^{-\frac{i}{\hbar} V(q) \varepsilon} \int dp\, e^{-\frac{i \varepsilon}{2m\hbar} (p - m\dot{q})^2} e^{\frac{i m \varepsilon}{2\hbar} \dot{q}^2} \\
			&= \frac{1}{2\pi\hbar} e^{-\frac{i}{\hbar} (\frac{1}{2}m \dot{q}^2 - V(q)) \varepsilon} \sqrt{\frac{2m\hbar}{i\varepsilon}} \int d\lambda\, e^{-\lambda^2} = \sqrt{\frac{m}{2\pi i \varepsilon \hbar}} e^{\frac{i\varepsilon}{\hbar} (\frac{1}{2} m \dot{q}^2 - V(q))}
		\end{split}
	\end{equation*}
	La dimostrazione è completa notando che, fissata l'evoluzione temporale (la $ \virgolette{traiettoria} $):
	\begin{equation*}
		\varepsilon \left( \frac{1}{2} m \dot{q}^2 - V(q) \right) = dt\,\mathscr{L}(q,\dot{q}) = d\mathcal{S}(q,t) \equiv d\mathcal{S}(t)
	\end{equation*}
\end{proof}

Il propagatore è quindi dato da una fase pari alla variazione d'azione in unità di $ \hbar $. Il fattore di normalizzazione è fissato dal fatto che:
\begin{equation}
	\lim_{\varepsilon \rightarrow 0} K(q',t+\varepsilon ; q,t) = \braket{q',t | q,t} = \delta(q' - q)
	\label{eq:4.17}
\end{equation}
Infatti:
\begin{equation*}
	\lim_{\varepsilon \rightarrow 0} \sqrt{\frac{m}{2\pi i \varepsilon \hbar}} e^{\frac{i}{\hbar} \varepsilon ( \frac{1}{2} m \dot{q}^2 - V(q))} = \lim_{\varepsilon \rightarrow 0} \sqrt{\frac{m}{2\pi i \varepsilon \hbar}} e^{-\frac{m}{2i\hbar} \frac{(q' - q)^2}{\varepsilon}} = \delta(q' - q)
\end{equation*}
che è proprio la rappresentazione della delta di Dirac come limite di gaussiane.

\subsection{Path integral}

Dalla Prop. \ref{prop-ass-conv} si sviluppa l'idea di Feynman che porta ad una riformulazione della fisica quantistica: si può pensare l'evoluzione temporale di un sistema come una successione di evoluzioni temporali infinitesime. Dato un sistema quantistico con propagatore $ K(q,t ; q_0,t_0) $ e definito $ t_k \defeq t_0 + k \varepsilon $, con $ t = t_0 + \Delta t = t_0 + n \varepsilon $, si ha:
\begin{equation*}
	\begin{split}
		K(q,t ; q_0,t_0)
		&= \int dq_1 \dots dq_{n-1}\, K(q,t ; q_{n-1},t_{n-1}) \dots K(q_1,t_1 ; q_0,t_0) \\
		&= \int dq_1 \dots dq_{n-1} \left( \frac{m}{2\pi i \varepsilon \hbar} \right)^{\frac{n}{2}} e^{\frac{i}{\hbar} ( d\mathcal{S}(t_{n-1}) + \dots + d\mathcal{S}(t_0) )}
	\end{split}
\end{equation*}
Questo è un modo compatto di esprimere il principio quantistico di sovrapposizione, che detta come si compongono le ampiezze di transizione (il cui modulo quadro dà la probabilità di transizione): l'ampiezza di transizione da un certo stato iniziale ad un certo stato finale si calcola considerando tutti i possibili stati intermedi e sommando su tutte le traiettorie (i $ \virgolette{cammini} $ di Feynman). A differenza di un sistema classico, che percorre solo un insieme discreto di traiettorie (quelle che estremizzano l'azione), è come se il sistema quantistico percorresse tutte le traiettorie possibili.\\
Prendendo il limite per $ n \rightarrow \infty $ si ottiene un integrale funzionale, o integrale di Kac, il quale associa un numero ad un funzionale (in questo caso l'azione): dato che quest'ultimo è una mappa da uno spazio di funzioni a $ \R $ (o anche $ \C $), la misura d'integrazione di un integrale di Kac è il differenziale di una funzione.

\begin{proposition}
	Dato un sistema quantistico con azione $ \mathcal{S}[q(t)] $ e definito l'insieme di possibili traiettorie da $ q_0 $ a $ q $ come $ \mathscr{P} \defeq \{q(t) : q(t_0) = q_0 \land q(t) = q\} $ , il propagatore $ K(q,t ; q_0,t_0) $ si può esprimere come un integrale di Kac, detto \textit{path integral}:
	\begin{equation}
		K(q,t ; q_0,t_0) = \int_{\mathscr{P}} \mathcal{D}q(t)\, e^{\frac{i}{\hbar} \mathcal{S}[q(t)]}
		\label{eq:4.18}
	\end{equation}
	dove la misura $ \mathcal{D}q(t) $ è definita in modo da rispettare la condizione di normalizzazione Eq. \ref{eq:4.17}.
\end{proposition}

Quella del path integral è una formulazione alternativa della fisica quantistica che, con l'aggiunta della regola di Born (probabilità è modulo quadro della funzione d'onda), è completamente analoga a quella di Schrödinger: infatti, sia il principio di sovrapposizione che quello di ortonormalità degli stati fisici sono naturalmente soddisfatti dal path integral. Per quanto riguarda l'evoluzione temporale, si può mostrare che il path integral soddisfa l'equazione di Schrödinger.

\begin{proposition}
	Una funzione d'onda del tipo in Eq. \ref{eq:4.14}, con propagatore dato dall'Eq. \ref{eq:4.18}, soddisfa l'equazione di Schrödinger.
\end{proposition}
\begin{proof}
	L'evoluzione temporale della funzione d'onda è data dal path integral:
	\begin{equation*}
		\psi(q,t) = \int dq_0 \int_{\mathscr{P}} \mathcal{D}q(t)\, e^{\frac{i}{\hbar}\mathcal{S}[q(t)]} \psi(q_0,t_0)
	\end{equation*}
	Si consideri un'evoluzione temporale infinitesima, con $ t = t_0 + \varepsilon $ e $ q = q_0 + \delta $ (Eq. \ref{eq:4.16}):
	\begin{equation*}
		\begin{split}
			\psi(q,t)
			&= \int d\delta\, \sqrt{\frac{m}{2\pi i \varepsilon \hbar}} e^{\frac{i}{\hbar} (\frac{1}{2} m \frac{\delta^2}{\varepsilon^2} - V(q)) \varepsilon} \psi(q_0,t_0) = e^{-\frac{i}{\hbar} V(q) \varepsilon} \sqrt{\frac{m}{2\pi i \varepsilon \hbar}} \int d\delta\, e^{\frac{i}{\hbar} \frac{1}{2} m \frac{\delta^2}{\varepsilon}} \psi(q_0,t_0) \\
			&= e^{-\frac{i}{\hbar} V(q)\varepsilon} \sqrt{\frac{m}{2\pi i \varepsilon \hbar}} \int d\delta\, e^{\frac{i}{\hbar} \frac{1}{2} m \frac{\delta^2}{\varepsilon}} \left[ \psi(q,t_0) - \delta \frac{\pa}{\pa q} \psi(q,t_0) + \frac{1}{2} \delta^2 \frac{\pa^2}{\pa q^2} \psi(q,t_0) + o(\delta^2) \right]
		\end{split}
	\end{equation*}
	Si noti ora che l'integrale ha misura $ d\delta $, dunque il primo termine risulta in un integrale gaussiano, il secondo si annulla per parità dell'integranda (dispari su intervallo simmetrico) ed il termo è calcolato usando $ \int_{\R} dx\, x^2 \exp (-\alpha x^2) = \frac{1}{2\alpha} \sqrt{\frac{\pi}{\alpha}} $:
	\begin{equation*}
		\begin{split}
			\psi(q,t)
			&= e^{-\frac{i}{\hbar} V(q) \varepsilon} \sqrt{\frac{m}{2\pi i \varepsilon \hbar}} \left[ \sqrt{-\frac{2\pi \hbar \varepsilon}{i m}} \psi(q,t_0) + \frac{1}{2} \left( - \frac{1}{2} \frac{2\hbar \varepsilon}{im} \right) \sqrt{- \frac{2\pi\hbar \varepsilon}{im}} \frac{\pa^2}{\pa q^2} \psi(q,t_0) \right] \\
			&= e^{-\frac{i}{\hbar} V(q) \varepsilon} \left[ \psi(q,t_0) + \frac{i \varepsilon \hbar}{2m} \frac{\pa^2}{\pa q^2} \psi(q,t_0) \right] = \left[ 1 - \frac{i}{\hbar} V(q) \varepsilon + o(\varepsilon) \right] \left[ \psi(q,t_0) + \frac{i \varepsilon \hbar}{2m} \frac{\pa^2}{\pa q^2} \psi(q,t_0) \right] \\
			&= \left[ 1 - \frac{i}{\hbar} V(q) \varepsilon + o(\varepsilon) \right] \left[ \psi(q,t) - \varepsilon \frac{\pa}{\pa t} \psi(q,t) + \frac{i \varepsilon \hbar}{2m} \frac{\pa^2}{\pa q^2} \psi(q,t) + o(\varepsilon) \right] \\
			&= \psi(q,t) + \varepsilon \left[ - \frac{i}{\hbar} V(q) \psi(q,t) - \frac{\pa}{\pa t} \psi(q,t) + \frac{i\hbar}{2m} \frac{\pa^2}{\pa q^2} \psi(q,t) \right] + o(\varepsilon)
		\end{split}
	\end{equation*}
	Perché valga l'equazione, si deve avere proprio l'equazione di Schrödinger:
	\begin{equation*}
		i\hbar \frac{\pa}{\pa t} \psi(q,t) = -\frac{\hbar^2}{2m} \frac{\pa^2}{\pa q^2} \psi(q,t) + V(q) \psi(q,t)
	\end{equation*}
\end{proof}

È particolarmente agevole trovare il limite semiclassico per $ \hbar \rightarrow 0 $. Si noti che per un tempo complesso $ t \mapsto it $ si ha $ \mathcal{S} \mapsto i\mathcal{S} $, quindi in tal caso il peso di ciasun cammino è $ \exp\left( - \frac{1}{\hbar} \mathcal{S}[q(t)] \right) $: la probabilità è esponenzialmente smorzata dall'azione, dunque l'unico cammino ad avere effettiva rilevanza fisica è quello di minima azione, ovvero la traiettoria classica. Per continuazione analitica si estende il risultato ai tempi reali: si ritrova il principio di corrispondenza.\\
Anche senza invocare la continuazione analitica, si vede che la dipendenza del propagatore dall'azione è data da una fase: per $ \mathcal{S} $ grande rispetto ad $ \hbar $, si integra su una fase che oscilla in maniera estremamente veloce, risultando quindi in media nulla (questo è il fenomeno della decoerenza); d'altro canto, se $ \mathcal{S} $ è comparabile ad $ \hbar $, molti cammini contribuiscono al path integral e si hanno effetti di interferenza quantistica. Si vede dunque che, in generale, la fisica quantistica descrive sistemi con un basso numero di gradi di libertà: infatti, gli effetti quantistici si manifestano quando il valore numerico dell'azione, associato al volume occupato dal sistema nello spazio delle fasi e quindi al numero dei suoi gradi di libertà, è piccolo in unità di $ \hbar $.

\section{WKB approximation}

Dalla formulazione del path integral si ha che nel limite semiclassico la funzione d'onda è determinata da un solo cammino, ed inoltre essa è esprimibile come una pura fase $ \psi \sim \exp (-\frac{i}{\hbar} \mathcal{S}) $, dove l'azione è valutata lungo tale cammino. Ciò suggerisce di scrivere la funzione d'onda nel limite semiclassico come:
\begin{equation}
	\psi(\ve{x},t) = e^{\frac{i}{\hbar} \Theta(\ve{x},t)}
	\label{eq:4.19}
\end{equation}
dove $ \Theta(\ve{x},t) $ è un'opportuna funzione esprimibile come serie di potenze di $ \hbar $. Questa è nota come \textit{approssimazione WKB} (Wentzel, Kramers, Brillouin).

\begin{proposition}
	Per un sistema quantistico descritto dall'hamiltoniana $ \hat{\mathcal{H}} = \frac{\hat{p}^2}{2m} + \hat{V}(\hat{\ve{x}},t) $, la funzione $ \Theta(\hat{x},t) $ deve soddisfarre:
	\begin{equation}
		-\frac{\pa}{\pa t} \Theta(\ve{x},t) = \frac{1}{2m} \left( \nabla \Theta(\ve{x},t) \right)^2 + V(\ve{x},t) - \frac{i\hbar}{2m} \lap \Theta(\ve{x},t)
		\label{eq:4.20}
	\end{equation}
\end{proposition}
\begin{proof}
	Inserendo l'Ansatz Eq. \ref{eq:4.19} nell'equazione di Schrödinger:
	\begin{equation*}
		i\hbar e^{\frac{i}{\hbar}\Theta(\ve{x},t)} \frac{i}{\hbar} \frac{\pa}{\pa t} \Theta(\ve{x},t) = - \frac{\hbar^2}{2m} \nabla \left[ e^{\frac{i}{\hbar}\Theta(\ve{x},t)} \frac{i}{\hbar} \nabla \Theta(\ve{x},t) \right] + V(\ve{x},t) e^{\frac{i}{\hbar} \Theta(\ve{x},t)}
	\end{equation*}
	Svolgendo il gradiente e semplificando $ e^{\frac{i}{\hbar}\Theta(\ve{x},t)} $ si ottiene la tesi.
\end{proof}

L'approssimazione semiclassica di $ \Theta(\ve{x},t) $ si ottiene espandendo in serie di potenze di $ \hbar $:
\begin{equation}
	\Theta(\ve{x},t) = \mathcal{S}(\ve{x}.t) + \sum_{n = 1}^{\infty} \left( \frac{\hbar}{i} \right)^n S_n(\ve{x},t)
	\label{eq:4.21}
\end{equation}
Il termine di ordine zero è fissato dal principio di corrispondenza. Infatti, per ordine zero l'Eq. \ref{eq:4.20} diventa:
\begin{equation}
	- \frac{\pa}{\pa t} \mathcal{S}(\ve{x},t) = \frac{1}{2m} (\nabla \mathcal{S}(\ve{x},t))^2 + V(\ve{x},t)
	\label{eq:4.22}
\end{equation}
il che conferma l'identificazione di $ \mathcal{S}(\ve{x},t) $ con la funzione principale di Hamilton.

\subsection{Sistemi invarianti per traslazioni temporali}

Per potenziali invarianti per traslazioni temporali è noto dalla teoria classica di Hamilton-Jacobi che per un sistema di energia $ E $:
\begin{equation}
	\mathcal{S}(\ve{x},t) = -E t + \sigma_0(\ve{x})
	\label{eq:4.23}
\end{equation}
dove $ \sigma_0(\ve{x}) $ è la funzione caratteristica di Hamilton. Questa soddisfa l'equazione:
\begin{equation}
	E = \frac{1}{2m} \left( \nabla \sigma_0(\ve{x}) \right) + V(\ve{x})
	\quad \Rightarrow \quad
	\nabla \sigma_0(\ve{x}) = \pm \sqrt{2m \left[ E - V(\ve{x}) \right]}
	\label{eq:4.24}
\end{equation}
Nel caso unidimensionale la soluzione è data dall'Eq. \ref{eq:4.7}. Per quanto riguarda gli ordini superiori, si ricordi che, per un'hamiltoniana indipendente dal tempo, gli autostati sono stati stazionari:
\begin{equation}
	\psi(\ve{x},t) = e^{-\frac{i}{\hbar} E t} \psi(\ve{x})
	\label{eq:4.25}
\end{equation}
Ne segue immediatamente che la funzione $ \Theta(\ve{x},t) $ può essere espansa come:
\begin{equation}
	\Theta(\ve{x},t) = -E t + \sigma_0(\ve{x}) + \sum_{n = 1}^{\infty} \left( \frac{\hbar}{i} \right)^n \sigma_n(\ve{x})
	\label{eq:4.26}
\end{equation}
Tutti i contributi di ordine superiore sono indipendenti dal tempo. Al prim'ordine, l'Eq. \ref{eq:4.20} diventa:
\begin{equation}
	2 \nabla \sigma_0(\ve{x}) \cdot \nabla \sigma_1(\ve{x}) + \lap \sigma_0(\ve{x}) = \ve{0}
	\label{eq:4.27}
\end{equation}
Al second'ordine, invece:
\begin{equation}
	2 \nabla \sigma_0(\ve{x}) \cdot \sigma_2(\ve{x}) + \left( \nabla \sigma_1(\ve{x}) \right)^2 + \lap \sigma_1(\ve{x}) = \ve{0}
	\label{eq:4.28}
\end{equation}
In generale, $ \sigma_k $ può essere determinato a partire da ogni $ \sigma_j $ con $ j < k $, ai quali è legato da una equazione differenziale del prim'ordine.

\subsubsection{Sistema unidimensionale}

Nel caso unidimensionale indipendente dal tempo è possibile determinare esplicitamente la soluzione. Definendo l'\textit{impulso semiclassico} $ p(x) \defeq \frac{d}{dx} \sigma_0(x) $ (vedere Eq. \ref{eq:4.4}), l'Eq. \ref{eq:4.27} ha come soluzione:
\begin{equation*}
	\sigma_1(x) = -\frac{1}{2} \ln \abs{p(x)} + c_1
\end{equation*}
La soluzione semiclassica al prim'ordine può quindi essere scritta come:
\begin{equation*}
	\begin{split}
		\psi(x,t)
		&= \exp \left[ \frac{i}{\hbar} \left( -Et \pm \int_{x_0}^x d\xi\, p(\xi) - \frac{1}{2} \frac{\hbar}{i} \ln \abs{p(x)} + c_1 \right) \right] \\
		&= \frac{\mathcal{N}}{\sqrt{\abs{p(x)}}} e^{\frac{1}{i\hbar} Et} \exp \left[ \pm \int_{x_0}^x d\xi\, p(\xi) \right]
	\end{split}
\end{equation*}
dove $ \mathcal{N} $ è un fattore di normalizzazione. Si vede che ci sono due soluzioni linearmente indipendenti, corrispondenti a due stati di impulso semiclassico definito $ \pm p(x) $. La soluzione generale è quindi una sovrapposizione:
\begin{equation}
	\psi(x,t) = \frac{1}{\sqrt{\abs{p(x)}}} e^{\frac{1}{i\hbar} Et} \left[ \mathcal{A} \exp \left( \frac{i}{\hbar} \int_{x_0}^x d\xi\, p(\xi) \right) + \mathcal{B} \exp \left( - \frac{i}{\hbar} \int_{x_0}^x d\xi\, p(x) \right) \right]
	\label{eq:4.29}
\end{equation}
Questa viene di solito chiamata \textit{soluzione approssimata WKB}.
Tale soluzione ha un andamento diverso a seconda del valore dell'energia rispetto al potenziale. Infatti, dall'Eq. \ref{eq:4.24} si ha che:
\begin{equation*}
	p(x) = \pm \sqrt{2m \left[ E - V(x) \right]}
\end{equation*}
Di conseguenza, se $ E > V(x) $ l'impulso semiclassico è reale, dunque la soluzione è di tipo sinusoidale:
\begin{equation*}
	\psi(x,t) = \frac{\mathcal{N}}{\sqrt{\abs{p(x)}}} e^{\frac{i}{\hbar} Et} \sin \left( \frac{1}{\hbar} \int_{x_0}^x d\xi\, \sqrt{2m \left[ E - V(\xi) \right]} + \delta \right)
\end{equation*}
D'altro canto, se $ E < V(x) $ l'impulso semiclassico è immaginario, dunque la soluzione è data da due esponenziali.

\paragraph{Validità dell'approssimazione}

Dall'Eq. \ref{eq:4.20} è chiaro che le correzioni quantistiche all'equazione del moto classica sono date dal termine proporzionale ad $ \hbar $. L'approssimazione semiclassica richiede dunque che quest'ultimo sia piccolo rispetto al termine cinetico:
\begin{equation}
	\abs{\hbar \lap \Theta(\ve{x},t)} \ll \abs{(\nabla \Theta(\ve{x},t))^2}
	\label{eq:4.30}
\end{equation}
Nel caso unidimensionale, ciò si riduce, ricordando la lunghezza d'onda di de Broglie ($ \lambda = \frac{2\pi\hbar}{p} $):
\begin{equation*}
	\abs{\hbar \frac{dp}{dx}} \ll \abs{p^2}
	\quad \Rightarrow \quad
	\abs{\frac{d(\lambda / 2\pi)}{dx}} \ll 1
\end{equation*}
L'approssimazione è dunque valida quando la scala di variazione della lunghezza d'onda delle oscillazioni di $ \psi(x,t) $ è piccola rispetto alla lunghezza d'onda stessa in unità di $ 2\pi $, ovvero quando la lunghezza d'onda varia poco nel corso di un periodo d'oscillazione.\\
Dalla forma esplicita di $ p(x) $ si ricava una condizione sull'energia:
\begin{equation*}
	\frac{\hbar}{\sqrt{2m \left[ E - V(x) \right]}} \ll \frac{2 \abs{E - V(x)}}{V'(x)}
\end{equation*}
L'approssimazione è buona quando l'energia è molto maggiore o molto minore del potenziale nel punto, ovvero per stati molto legati o molto eccitati, mentre fallisce in prossimità dei punti d'inversione, quando $ E \approx V(x) $.

\begin{example}
	Si consideri una generica buca di potenziale unidimensionale (es: buca infinita di potenziale, oscillatore armonico, etc.): per un dato valore dell'energia $ E > V_{\text{min}} $ sono presenti due punti d'inversione, detti $ a $ e $ b $, soluzioni dell'equazione $ E - V(x) = 0 $. L'approssimazione semiclassica è valida nelle tre regioni che escludono gli intorni di $ a $ e $ b $: in questi ultimi, la soluzione può essere trovata considerando il potenziale lineare, dato che:
	\begin{equation*}
		V(x) = V(a) + V'(a) (x - a) + o(x - a)
	\end{equation*}
	ed idem con $ b $. La soluzione del problema del potenziale lineare è nota ed esattamente esprimibile in termini di funzioni di Airy: infatti, sulla base degli impulsi l'equazione di Schrödinger diventa:
	\begin{equation*}
		\left[ \frac{p^2}{2m} + i \lambda \frac{d}{dp} \right] \psi(p) = E \psi(p)
	\end{equation*}
	La risoluzione è banale e dalla conseguente trasformata di Fourier (per passare alla base delle coordinate) si ottengono appunto le funzioni di Airy.\\
	Con l'approssimazione semiclassica, invece, nelle regioni esterne della buca ($ x \ll b $ e $ x \gg a $, assumendo WLOG $ a > b $) si hanno soluzioni di tipo esponenziale, mentre in quella interna ($ b \ll x \ll a $) la soluzione è di tipo sinusoidale. Le relazioni tra le normalizzazioni e la fase della sinusoide sono date dalle condizioni di raccordo (continuità e differenziabilità); per un potenziale prima decrescente e poi crescente si trovano:
	\begin{equation*}
		\psi_b(x) =
		\begin{cases}
			\frac{\mathcal{N}}{\sqrt{\beta(x)}} \exp \left( - \int_x^b d\xi\, \frac{\beta(\xi)}{\hbar} \right) & x \ll b \\
			\frac{2\mathcal{N}}{\sqrt{p(x)}} \sin \left( \int_b^x d\xi\, \frac{p(\xi)}{\hbar} + \frac{\pi}{4} \right) & x \gg b
		\end{cases}
	\end{equation*}
	\begin{equation*}
		\psi_a(x) =
		\begin{cases}
			\frac{2\mathcal{N}'}{\sqrt{p(x)}} \sin \left( \int_x^a d\xi\, \frac{p(\xi)}{\hbar} + \frac{\pi}{4} \right) & x \ll a \\
			\frac{\mathcal{N}'}{\sqrt{\beta(x)}} \exp \left( \int_x^a d\xi\, \frac{\beta(\xi)}{\hbar} \right) & x \gg a
		\end{cases}
	\end{equation*}
	dove $ p(x) = i \beta(x) $ nelle regioni in cui $ E < V(x) $. Le soluzioni per $ x \gg b $ ed $ x \ll a $ devono essere uguali, poiché sono nella stessa regione, dunque si ha un'equazione del tipo $ \sin \alpha = \sin \beta $: questa ha come soluzioni $ \alpha = \beta $ o $ \alpha = \pi - \beta $, ma la prima non è possibile da soddisfare per ogni $ x $ nell'intervallo, dunque si deve avere che la somma degli argomenti delle sinusoidi sia $ \pi $. Inoltre, si nota che se la somma fosse $ 2\pi $ si avrebbe un segno di differenza tra le sinusoidi, il quale è assorbibile dalla normalizzazione. In generale, si ha:
	\begin{equation*}
		\mathcal{N}' = (-1)^{n + 1} \mathcal{N}
		\qquad \qquad
		\left( \int_b^x d\xi\, \frac{p(x)}{\hbar} + \frac{\pi}{4} \right) + \left( \int_x^a d\xi\, \frac{p(x)}{\hbar} + \frac{\pi}{4} \right) = n\pi
	\end{equation*}
	con $ n \in \N $. La condizione sugli integrali è una condizione di quantizzazione:
	\begin{equation*}
		\int_b^a d\xi\, p(\xi) = \left( n + \frac{1}{2} \right) \hbar \pi
	\end{equation*}
	Questa condizione di quantizzazione era stata già postulata da Bohr e Sommerfeld con considerazioni analoghe a quelle del modello di Bohr per l'atomo d'idrogeno (vedere Sec. \ref{bohr-hydrogen}), supponendo che ogni sistema hamiltoniano debba soddisfarre la \textit{condizione di Bohr-Sommerfeld}:
	\begin{equation}
		\oint p\,dq = 2\hbar \pi n
		\label{eq:4.31}
	\end{equation}
	Per $ n \gg 1 $ questa coincide con la condizione trovara nell'approssimazione semiclassica (si ha un fattore 2 poiché l'integrale è lungo un loop chiuso, dunque percorre l'intervallo due volte).\\
	Si vede inoltre il comportamento qualitativo dello spettro di una buca di potenziale generica: nella regione interna alla buca si ha $ \psi(x) \sim \sin (f(x) + \delta) $, con $ f(b) + \delta = \frac{\pi}{4} $ e $ f(a) + \delta = \frac{\pi}{4} + n\pi $. Ne segue che nello stato fondamentale il seno compie un semiperiodo, nel primo stato eccitato un periodo, nel secondo un periodo e mezzo, e così via: il numero di nodi della funzione d'onda cresce al crescere dell'energia e ad ogni stato eccitato si aggiunge un semiperiodo d'oscillazione aggiuntivo.
\end{example}










