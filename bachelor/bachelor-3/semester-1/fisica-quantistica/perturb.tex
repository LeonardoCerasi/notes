\selectlanguage{italian}

\section{Perturbazioni indipendenti dal tempo}

I metodi perturbativi indipendenti dal tempo vengono usati in tutti i casi in cui si cerca di determinare lo spettro di un operatore, ed in particolare un'hamiltoniana, che si può scrivere come la somma di un operatore, il cui spettro è noto, ed un termine di correzione indipendente dal tempo.\\
Si consideri un'hamiltoniana del tipo:
\begin{equation}
	\hat{\mathcal{H}} = \hat{\mathcal{H}}_0 + \varepsilon \hat{\mathcal{H}}'
	\label{eq:5.1}
\end{equation}
Si supponga che lo spettro si $ \hat{\mathcal{H}}_0 $ sia ortonormale e noto:
\begin{equation}
	\hat{\mathcal{H}} \ket{n_0} = E_n^{(0)} \ket{n_0}
	\qquad \qquad
	\braket{m_0 | n_0} = \delta_{mn}
	\label{eq:5.2}
\end{equation}
Lo spettro dell'hamiltoniana completa è invece:
\begin{equation}
	\hat{\mathcal{H}} \ket{n} = E_n \ket{n}
	\label{eq:5.3}
\end{equation}
L'idea del metodo perturbativo è determinare autofunzioni ed autovalori dell'hamiltoniana completa come serie di potenze di $ \varepsilon $:
\begin{equation}
	E_n = E_n^{(0)} + \sum_{k \ge 1} \varepsilon^k E_n^{(k)}
	\label{eq:5.4}
\end{equation}
\begin{equation}
	\ket{n} = \ket{n_0} + \sum_{k \ge 1} \varepsilon^k \ket{n_k}
	\label{eq:5.5}
\end{equation}
È importante osservare che, in generale, $ \varepsilon $ non è necessariamente piccolo e le due serie perturbative non è garantito che convergano.

\subsection{Spettro non-degenere}

Si consideri lo spettro dell'hamiltoniana non-perturbata in Eq. \ref{eq:5.2} non-degenere, dunque $ E_n^{(0)} \neq E_k^{(0)} $ per $ n \neq k $. Sostituendo gli sviluppi perturbativi nell'Eq. \ref{eq:5.3}:
\begin{equation*}
	\left( \hat{\mathcal{H}}_0 + \varepsilon \hat{\mathcal{H}}' \right) \left( \ket{n_0} + \varepsilon \ket{n_1} + \dots \right) = \left( E_n^{(0)} + \varepsilon E_n^{(1)} + \dots \right) \left( \ket{n_0} + \varepsilon \ket{n_1} + \dots \right)
\end{equation*}
Identificando i termini dello stesso ordine in $ \varepsilon $ si trova una sequenza di equazioni:
\begin{align*}
	\varepsilon^0 : \,\left( \hat{\mathcal{H}}_0 - E_n^{(0)} \right) \ket{n_0} &= 0 \\
	\varepsilon^1 : \,\left( \hat{\mathcal{H}}_0 - E_n^{(0)} \right) \ket{n_1} &= \left( E_n^{(1)} - \hat{\mathcal{H}}' \right) \ket{n_0} \\
	\varepsilon^2 : \,\left( \hat{\mathcal{H}}_0 - E_n^{(0)} \right) \ket{n_2} &= \left( E_n^{(1)} - \hat{\mathcal{H}} \right) \ket{n_1} + E_n^{(2)} \ket{n_0} \\
								 & \,\,\, \vdots
\end{align*}
In generale:
\begin{equation}
	\left( \hat{\mathcal{H}}_0 - E_n^{(0)} \right) \ket{n_k} = \left( E_n^{(1)} - \hat{\mathcal{H}}' \right) \ket{n_{k-1}} + \sum_{j = 2}^{k} E_n^{(j)} \ket{n_{k-j}}
	\label{eq:5.6}
\end{equation}

\begin{proposition}\label{pert-non-deg-spec}
	È possibile scegliere tutte le correzioni dell'autostato ortogonali all'autostato non-perturbato: $ \braket{n_0 | n_k} = 0 \,\forall k \in \N $.
\end{proposition}
\begin{proof}
	Dato $ \ket{n_k} $, si consideri $ \ket{\tilde{n}_k} \equiv \ket{n_k} + \lambda \ket{n_0} $:
	\begin{equation*}
		\left( \hat{\mathcal{H}}_0 - E_n^{(0)} \right) \ket{\tilde{n}_k}  = \left( \hat{\mathcal{H}}_0 - E_n^{(0)} \right) \ket{n_k} + \lambda \left( \hat{\mathcal{H}}_0 - E_n^{(0)} \right) \ket{n_0} = \left( \hat{\mathcal{H}}_0 - E_n^{(0)} \right) \ket{n_k}
	\end{equation*}
	Dunque l'Eq. \ref{eq:5.6} rimane inalterata per $ \ket{n_k} \mapsto \ket{n_k} + \lambda \ket{n_0} $. Pertanto, data la soluzione $ \ket{n_k} : \braket{n_0 | n_k} \neq 0 $, è sempre possibile trovarne una $ \ket{\tilde{n}_k} : \braket{n_0 | \tilde{n}_k} = 0 $  scegliendo $ \lambda_k = - \braket{n_0 | n_k} $.
\end{proof}

Data la Prop. \ref{pert-non-deg-spec}, si può estrarre la correzione al $ k $-esimo ordine dell'autovalore $ E_n^{(k)} $ dall'Eq. \ref{eq:5.6} semplicemente proiettando su $ \ket{n_0} $: così facendo, tutti i termini nel lato destro si annullano, eccetto quello con $ \ket{n_0} $, lasciando solo il lato sinistro. Si trova quindi:
\begin{equation}
	E_n^{(k)} = \braket{n_0 | \hat{\mathcal{H}}' | n_{k-1}}
	\label{eq:5.7}
\end{equation}
Si vede che la correzione all'ordine $ k $ è determinata da quella all'ordine $ k - 1 $. Per trovare la correzione al $ k $-esimo ordine dell'autostato, invece, è necessario applicare all'Eq. \ref{eq:5.6} l'inverso dell'operatore $ \hat{\mathcal{H}}_0 - E_n^{(0)} $; questo operatore, però, non è invertibile nello spazio degli autostati fisici, poiché $ \ket{n_0} $ ha autovalore nullo, essendo $ (\hat{\mathcal{H}}_0 - E_n^{(0)}) \ket{n_0} = 0 $. Il problema è risolto dalla Prop. \ref{pert-non-deg-spec}: tutte le correzioni $ \ket{n_k} $ appartengono al sottospazio ortogonale ad $ \ket{n_0} $, dunque l'operatore è invertibile in tale sottospazio. In generale si ha:
\begin{equation}
	\ket{n_j} = \sum_{k \neq n} \ket{k_0} \braket{k_0 | n_j}
	\label{eq:5.8}
\end{equation}
Nel sottospazio ortogonale ad $ \ket{n_0} $, dunque, l'operatore $ \hat{\mathcal{H}}_0 - E_n^{(0)} $ si scrive in serie come:
\begin{equation}
	\frac{1}{\hat{\mathcal{H}}_0 - E_n^{(0)}} = \sum_{k \neq n} \frac{1}{E_k^{(0)} - E_n^{(0)}} \ket{k_0} \bra{k_0}
	\label{eq:5.9}
\end{equation}
È possibile applicare questo operatore all'Eq. \ref{eq:5.6} per trovare la correzione dell'autostato a qualsiasi ordine. Per il prim'ordine:
\begin{equation*}
	\ket{n_1} = \sum_{k \neq n} \frac{1}{E_k^{(0)} - E_n^{(0)}} \ket{k_0} \braket{k_0 | \left( E_n^{(1)} - \hat{\mathcal{H}}' \right) | n_0} = \sum_{k \neq n} \frac{1}{E_k^{(0)} - E_n^{(0)}} \ket{k_0} \left[ \delta_{kn} E_n^{(1)} - \braket{k_0 | \hat{\mathcal{H}}' | n_0} \right]
\end{equation*}
La correzione al prim'ordine dell'autostato è dunque:
\begin{equation}
	\ket{n_1} = \sum_{k \neq n} \frac{\braket{k_0 | \hat{\mathcal{H}}' | n_0}}{E_n^{(0)} - E_k^{(0)}} \ket{k_0}
	\label{eq:5.10}
\end{equation}
La correzione al prim'ordine dell'autovalore, invece, si trova banalmente dall'Eq. \ref{eq:5.7}:
\begin{equation}
	E_n^{(1)} = \braket{n_0 | \hat{\mathcal{H}}' | n_0}
	\label{eq:5.11}
\end{equation}
Per trovare la correzione al second'ordine dell'autovalore dall'Eq. \ref{eq:5.7}:
\begin{equation*}
	E_n^{(2)} = \braket{n_0 | \hat{\mathcal{H}}' \sum_{k \neq n} \frac{\braket{k_0 | \hat{\mathcal{H}}' | n_0}}{E_n^{(0)} - E_k^{(0)}} | k_0} = \sum_{k \neq n} \frac{\braket{n_0 | \hat{\mathcal{H}}' | k_0} \braket{k_0 | \hat{\mathcal{H}}' | n_0}}{E_n^{(0)} - E_k^{(0)}}
\end{equation*}
da cui:
\begin{equation}
	E_n^{(2)} = \sum_{k \neq n} \frac{\abs{\braket{k_0 | \hat{\mathcal{H}}' | n_0}}^2}{E_n^{(0)} - E_k^{(0)}}
	\label{eq:5.12}
\end{equation}
Si nota che la correzione al second'ordine per lo stato fondamentale, ossia quello di minima energia, è sempre negativa, proprio come ci si aspetterebbe.
La correzione $ \ket{n_2} $ si trova invece come:
\begin{equation*}
	\begin{split}
		\ket{n_2}
		&= \sum_{k \neq n} \frac{1}{E_k^{(0)} - E_n^{(0)}} \ket{k_0} \bra{k_0} \left[ \left( E_n^{(1)} - \hat{\mathcal{H}}' \right) \ket{n_1} + E_n^{(2)} \ket{n_0} \right] \\
		&= \sum_{k \neq n} \frac{1}{E_k^{(0)} - E_n^{(0)}} \ket{k_0} \left[ E_n^{(1)} \braket{k_0 | n_1} - \braket{k_0 | \hat{\mathcal{H}}' | n_1} + \delta_{kn} E_n^{(2)} \right] \\
		&= \sum_{k \neq n} \frac{1}{E_n^{(0)} - E_k^{(0)}} \ket{k_0} \left[ \sum_{m \neq n} \frac{\braket{m_0 | \hat{\mathcal{H}}' | n_0}}{E_n^{(0)} - E_m^{(0)}} \braket{k_0 | \hat{\mathcal{H}}' | m_0} - E_n^{(1)} \sum_{m \neq n} \frac{\braket{m_0 | \hat{\mathcal{H}}' | n_0}}{E_n^{(0)} - E_m^{(0)}} \underbrace{\braket{k_0 | m_0}}_{\delta_{km}} \right]
	\end{split}
\end{equation*}
La correzione al second'ordine dell'autostato è quindi:
\begin{equation}
	\ket{n_2} = \sum_{k \neq n} \frac{1}{E_n^{(0)} - E_k^{(0)}} \left[ \sum_{m \neq n} \frac{\braket{m_0 | \hat{\mathcal{H}}' | n_0} \braket{k_0 | \hat{\mathcal{H}}' | m_0}}{E_n^{(0)} - E_m^{(0)}} - \frac{\braket{k_0 | \hat{\mathcal{H}}' | n_0} \braket{n_0 | \hat{\mathcal{H}}' | n_0}}{E_n^{(0)} - E_k^{(0)}} \right] \ket{k_0}
	\label{eq:5.13}
\end{equation}

\subsubsection{Spettro degenere}

Nel caso in cui lo spettro dell'hamiltoniana non-perturbata sia degenere, la trattazione diventa più complessa: in generale, la base di autostati di partenza e quella di arrivo saranno diverse.\\
Ad esempio, si prenda l'oscillatore armonico isotropo: due possibili basi sono $ \ket{n_1,n_2,n_3} $ e $ \ket{n,\ell,m} $, ma ce ne sono infinite altre. La perturbazione potrebbe selezionare una base particolare: ad esempio, una perturbazione lungo l'asse $ x $ seleziona la base $ \ket{n_1,n_2,n_3} $, mentre una perturbazione dipendente dal momento angolare seleziona la base $ \ket{n,\ell,m} $.\\
Si supponga che allo stesso autostato di energia $ E_n^{(0)} $ di $ \hat{\mathcal{H}} $ siano associati $ d $ autostati $ \ket{n^{(0)}_k} $, con $ k = 1, \dots, d $. Per effetto della perturbazione, le energie di questi stati possono in generale diventare diverse tra loro, riducendo o eliminando la degenerazione (o anche lasciandola invariata): gli stati perturbati $ \ket{n_k} = \ket{n^{(0)}_k} + \sum_{m \ge 1} \varepsilon^m \ket{n^{(m)}_k} $ avranno in generale energie $ E_{n,k} $ diverse tra loro.\\
Di conseguenza, qualsiasi combinazione lineare di $ \ket{n^{(0)}_k} $ è ancora autostato di $ \hat{\mathcal{H}}_0 $ con autostato $ E^{(0)}_n $, ma soltanto una base degenere precisa è quella a cui si riduce lo sviluppo perturbativo per $ \varepsilon \rightarrow 0 $. Per determinarla, si moltiplichi l'Eq. \ref{eq:5.6} al prim'ordine per $ \bra{n_j^{(0)}} $:
\begin{equation*}
	\braket{n^{(0)}_j | \left( \hat{\mathcal{H}}_0 - E^{(0)}_n \right) | n^{(1)}_k} = \braket{n^{(0)}_j | \left( E^{(1)}_{n,k} - \hat{\mathcal{H}}' \right) | n^{(0)}_k}
\end{equation*}
Ricordando che $ \bra{n^{(0)}_j} \hat{\mathcal{H}}_0 = \bra{n^{(0)}_j} E^{(0)}_n $ e $ \braket{n^{(0)}_j | n^{(0)}_k} = \delta_{jk} $ (base degenere ortonormale), si trova la condizione sulla base di autostati degeneri di $ \hat{\mathcal{H}}_0 $:
\begin{equation}
	\braket{n^{(0)}_j | \hat{\mathcal{H}}' | n^{(0)}_k} = E^{(1)}_{n,k} \delta_{jk}
	\label{eq:5.14}
\end{equation}
Bisogna dunque scegliere una base che diagonalizzi l'hamiltoniana perturbante (e, di conseguenza, quella perturbata completa): le correzioni al prim'ordine dell'energia sono gli autovalori così trovati, mentre gli autoket di ordine zero sono i rispettivi autovettori, ovvero la base diagonalizzante.\\
Se a seguito di questa procedura tutti gli autovalori $ E^{(1)}_{n,k} $ sono diversi tra loro, ovvero se la degenerazione è completamente eliminata, le correzioni ad ordini successivi sono calcolate nel caso non-degenere. Se invece la degenerazione è ancora presente, si ripete questa procedura nel sottospazio rimasto degenere dopo l'introduzione della perturbazione all'ordine precedente.

\section{Perturbazioni dipendenti dal tempo}

La teoria delle perturbazioni dipendenti dal tempo si usa per affrontare quei problemi (es.: fenomeni di diffusione) in cui è necessario valutare la probabilità che il sistema subisca una transizione da un certo stato iniziale ad un certo stato finale per effetto di un potenziale.

\subsection{Rappresentazione d'interazione}

È conveniente introdurre una nuova rappresentazione, intermedia tra quella di Heisenberg e quella di Schrödinger.
Si consideri un sistema descritto da una hamiltoniana del tipo:
\begin{equation}
	\hat{\mathcal{H}} = \hat{\mathcal{H}}_0 + \hat{V}(t)
	\label{eq:5.15}
\end{equation}
composta da un termine imperturbato ed una perturbazione dipendente dal tempo. La \textit{rappresentazione d'interazione} consiste nel trattare $ \hat{\mathcal{H}}_0 $ in rappresentazione di Heisenberg e $ \hat{V}(t) $ in rappresentazione di Schrödinger. Denotando con $ \ket{\psi(t)}_{\text{S}} $ gli stati in rappresentazione di Schrödinger, si definisce la rappresentazione d'interazione come:
\begin{equation}
	\ket{\psi(t)}_{\text{I}} \defeq e^{-\frac{1}{i\hbar} (t - t_0) \hat{\mathcal{H}}_0} \ket{\psi(t)}_{\text{S}} \equiv \hat{S}_0^{-1}(t,t_0) \ket{\psi(t)}_{\text{S}}
	\label{eq:5.16}
\end{equation}

\begin{proposition}
	Gli operatori in rappresentazione d'interazione sono legati a quelli in rappresentazione di Schrödinger da:
	\begin{equation}
		\hat{O}_{\text{I}} = \hat{S}_0^{-1}(t,t_0) \hat{O}_{\text{S}} \hat{S}_0(t,t_0)
		\label{eq:5.17}
	\end{equation}
\end{proposition}

\begin{proposition}
	La dipendenza temporale degli stati in rappresentazione d'interazione è data solo dal termine perturbativo:
	\begin{equation}
		\ket{\psi(t)}_{\text{I}} = \hat{S}_{\text{I}}(t,t_0) \ket{\psi(t_0)}_{\text{I}} \equiv \mathcal{T} \exp \left[ \frac{1}{i\hbar} \int_{t_0}^t dt'\, \hat{V}_{\text{I}}(t') \right] \ket{\psi(t_0)}_{\text{I}}
		\label{eq:5.18}
	\end{equation}
\end{proposition}
\begin{proof}
	Per calcolo esplicito:
	\begin{equation*}
		\begin{split}
			i\hbar \frac{\pa}{\pa t} \ket{\psi(t)}_{\text{I}}
			&= - e^{-\frac{1}{i\hbar} (t - t_0) \hat{\mathcal{H}}_0} \hat{\mathcal{H}}_0 \ket{\psi(t)}_{\text{S}} + i\hbar e^{-\frac{1}{i\hbar} (t - t_0) \hat{\mathcal{H}}_0} \frac{\pa}{\pa t} \ket{\psi(t)}_{\text{S}} \\
			&= - e^{-\frac{1}{i\hbar} (t - t_0) \hat{\mathcal{H}}_0} \hat{\mathcal{H}}_0 \ket{\psi(t)}_{\text{S}} + e^{-\frac{1}{i\hbar} (t - t_0) \hat{\mathcal{H}}_0} ( \hat{\mathcal{H}}_0 + \hat{V}(t) ) \ket{\psi(t)}_{\text{S}} \\
			&= e^{-\frac{1}{i\hbar} (t - t_0) \hat{\mathcal{H}}_0} \hat{V}(t) \ket{\psi(t)}_{\text{S}} = \hat{S}_0^{-1}(t,t_0) \hat{V}(t) \hat{S}_0(t,t_0) \ket{\psi(t)}_{\text{I}} = \hat{V}_{\text{I}}(t) \ket{\psi(t)}_{\text{I}}
		\end{split}
	\end{equation*}
	In generale $ \hat{V}(t) $ non commuta a tempi diversi, da cui la necessità del prodotto cronologico $ \mathcal{T} $.
\end{proof}

Si può verificare che l'evoluzione temporale determinata dalla rappresentazione di Schrödinger coincide con quella in rappresentazione d'interazione: in particolare, si ottengono le stesse probabilità per i risultati delle misure.

\begin{proposition}
	La rappresentazione d'interazione è equivalente a quella di Schrödinger.
\end{proposition}
\begin{proof}
	In rappresentazione di Schrödinger, l'ampiezza di probabilità che il sistema al tempo $ t $ si trovi in un autostato $ \ket{n} $ dell'operatore hermitiano $ \hat{O} $ è data da:
	\begin{equation*}
		\braket{n | \psi(t)}_{\text{S}} = \braket{n | \hat{S}(t,t_0) | \psi(t_0)}_{\text{S}}
	\end{equation*}
	In rappresentazione d'interazione, gli autostati di $ \hat{O} $ sono determinati come:
	\begin{equation*}
		\hat{O}_{\text{I}}(t) \ket{n(t)}_{\text{I}} = \lambda_n \ket{n(t)}_{\text{I}}
	\end{equation*}
	Dalle Eqq. \ref{eq:5.16}-\ref{eq:5.17}:
	\begin{equation*}
		\ket{n(t)}_{\text{I}} = \hat{S}^{-1}_0 (t,t_0) \ket{n}_{\text{S}} \equiv \ket{\tilde{n}(t)}
	\end{equation*}
	Si trova dunque:
	\begin{equation*}
		\braket{\tilde{n}(t) | \psi(t)}_{\text{I}} = \braket{n | \hat{S}_0 (t,t_0) \hat{S}^{-1}_0 (t,t_0) | \psi(t)}_{\text{S}} = \braket{n | \hat{S}(t,t_0) | \psi(t_0)}_{\text{S}}
	\end{equation*}
\end{proof}

\subsection{Sviluppo perturbativo dipendente dal tempo}

Si consideri lo spettro energetico dell'hamiltoniana non-perturbata, in generale degenere:
\begin{equation*}
	\hat{\mathcal{H}}_0 \ket{n} = E_n \ket{n}
\end{equation*}
Si supponga che, a seguito dell'interazione con la perturbazione $ \hat{V}(t) $, il sistema compia una transizione in $ \ket{m} $. L'ampiezza di probabilità della transizione è definita come:
\begin{equation}
	\mathscr{A}^{(\text{S})}_{nm}(t) \defeq \braket{m | \hat{S}(t,t_0) | n}
	\label{eq:5.19}
\end{equation}
Si ricordi che la probabilità è definita come $ \mathscr{P} \defeq \abs{\mathscr{A}}^2 $.

\begin{proposition}
	$ \mathscr{P}^{(\text{I})}_{nm} = \mathscr{P}^{(\text{S})}_{nm} $.
\end{proposition}
\begin{proof}
	Per calcolo esplicito, ricordando le Eqq. \ref{eq:5.16}-\ref{eq:5.17}:
	\begin{equation*}
		\begin{split}
			\mathscr{A}^{(\text{S})}_{nm}(t)
			&= \braket{\tilde{m} | \hat{S}_{\text{I}}(t,t_0) | \tilde{n}} = \braket{m | \hat{S}_0(t,t_o) \hat{S}_{\text{I}}(t,t_0) \hat{S}^{-1}_0 (t,t_o) | n} \\
			&= e^{\frac{1}{i\hbar} \left( E_m t - E_n t_0 \right)} \braket{m | \hat{S}_{\text{I}}(t,t_0) | n} = e^{\frac{1}{i\hbar} \left( E_m t - E_n t_0 \right)} \mathscr{A}^{(\text{I})}_{nm}(t)
		\end{split}
	\end{equation*}
\end{proof}

Si noti che in generale il tempo $ t_o $ da cui si definisce la rappresentazione d'interazione è diverso dal tempo $ t_0 $ in cui si trova lo stato iniziale.\\
Dato che la probabilità è indipendente dalla rappresentazione usata, si calcola $ \mathscr{A}^{(\text{I})}_{nm}(t) $:
\begin{equation*}
	\begin{split}
		\mathscr{A}^{(\text{I})}_{nm}(t)
		&= \braket{m | \id + \frac{1}{i\hbar} \int_{t_0}^t dt'\, \hat{V}_{\text{I}}(t') + \frac{1}{2} \frac{1}{(i\hbar)^2} \mathcal{T} \int_{t_0}^t dt' dt''\, \hat{V}_{\text{I}}(t') \hat{V}_{\text{I}}(t'') + \dots | n} \\
		&= \delta_{nm} + \frac{1}{i\hbar} \int_{t_0}^t dt' \braket{m | \hat{S}^{-1}_0 (t',t_o) \hat{V}(t') \hat{S}_0 (t',t_o) | n} + \dots
	\end{split}
\end{equation*}
Si può interpretare questa come una serie perturbativa, detta \textit{serie di Dyson}:
\begin{equation}
	\mathscr{A}^{(\text{I})}_{nm}(t) = \sum_{i = 0}^{\infty} \mathscr{A}^{(i)}_{nm}(t)
	\label{eq:5.20}
\end{equation}
Ricordando le proprietà del prodotto cronologico, si trovano i primi termini dello sviluppo:
\begin{equation}
	\mathscr{A}^{(0)}_{nm}(t) = \delta_{nm}
	\label{eq:5.21}
\end{equation}
\begin{equation}
	\mathscr{A}^{(1)}_{nm}(t) = \frac{1}{i\hbar} \int_{t_0}^t dt' \braket{m | \hat{V}(t') | n} e^{\frac{1}{i\hbar} \left( E_n - E_m \right) t'}
	\label{eq:5.22}
\end{equation}
\begin{equation}
	\mathscr{A}^{(2)}_{nm}(t) = \frac{1}{(i\hbar)^2} \int_{t_0}^t dt' \int_{t_0}^{t'} dt'' \sum_{k} \braket{m | \hat{V}(t') | k} e^{\frac{1}{i\hbar} \left( E_k - E_m \right) t'} \braket{k | \hat{V}(t'') | n} e^{\frac{1}{i\hbar} \left( E_n - E_k \right) t''}
	\label{eq:5.23}
\end{equation}
dove la sommatoria è sullo spettro di $ \hat{\mathcal{H}}_0 $.

\subsection{Regola aurea di Fermi}

Si consideri il caso in cui la perturbazione sia indipendente dal tempo, attiva solo per $ t > 0 $:
\begin{equation}
	V(t) = V \theta(t)
	\label{eq:5.24}
\end{equation}
dove $ \theta $ è la distribuzione di Heaviside. Ciò permette di trattare il caso in cui il sistema viene preparato in uno stato iniziale in una regione in cui il potenziale è trascurabile, tipico degli esperimenti di scattering. In questo caso, ponendo $ V_{mn} \equiv \braket{m | \hat{V} | n} $, la serie di Dyson al prim'ordine per $ n \neq m $ coincide con:
\begin{equation*}
	\mathscr{A}^{(1)}_{mn}(t) = \frac{1}{i\hbar} \int_0^t dt'\, e^{\frac{1}{i\hbar} \left( E_n - E_m \right) t'} V_{mn} = - \frac{e^{\frac{1}{i\hbar} \left( E_n - E_m \right) t} - 1}{E_m - E_n} V_{mn}
\end{equation*}
Ponendo $ E_n = \hbar \omega_n $, si trova:
\begin{equation*}
	\begin{split}
		\frac{1}{t} \mathscr{P}_{nm}
		&= \frac{1}{t} \frac{\abs{V_{nm}}^2}{(\omega_m - \omega_n)^2 \hbar^2} \abs{e^{it \left( \omega_m - \omega_n \right)} - 1}^2 \\
		&= \frac{1}{t} \frac{\abs{V_{nm}}^2}{(\omega_m - \omega_n)^2 \hbar^2} \abs{e^{\frac{it}{2} \left( \omega_m - \omega_n \right)} \left( e^{\frac{it}{2} \left( \omega_m - \omega_n \right)} - e^{-\frac{it}{2} \left( \omega_m - \omega_n \right)} \right)}^2 \\
		&= \frac{\abs{V_{nm}}^2}{(\omega_m - \omega_n)^2 \hbar^2} \frac{4}{t} \sin^2 \left[ \frac{t}{2} \left( \omega_m - \omega_n \right) \right]
	\end{split}
\end{equation*}
Di particolare interesse è il limite $ t \rightarrow \infty $: la particella incidente arriva da una regione lontana, dunque la perturbazione agisce per un tempo molto lungo rispetto alla scala di tempi tipici del potenziale stesso. La funzione $ x^{-2} \sin^2 (tx^2) $ diventa sempre più piccata nell'origine al crescere di $ t $, inoltre il suo integrale è costante:
\begin{equation*}
	\int_{-\infty}^{\infty} dx\, \frac{\sin^2 (tx^2)}{x^2} = t\pi
	\qquad \Rightarrow \qquad
	\lim_{t \rightarrow \infty} \frac{1}{\pi} \frac{\sin^2 (tx^2)}{tx^2} = \delta(x)
\end{equation*}
Nel limite per $ t \rightarrow \infty $ si ha dunque:
\begin{equation*}
	\frac{1}{t} \mathscr{P}_{nm} = \abs{V_{mn}}^2 \frac{\pi}{\hbar^2} \delta \left( \frac{\omega_m - \omega_n}{2} \right)
\end{equation*}
Ricordando che $ \delta(\alpha x) = \abs{\alpha}^{-1} \delta(x) $, si trova:
\begin{equation}
	\frac{1}{t} \mathscr{P}_{nm} = \frac{2\pi}{\hbar} \abs{V_{mn}}^2 \delta(E_m - E_n)
	\label{eq:5.25}
\end{equation}
Questa è nota come \textit{regola aurea di Fermi}: essa fornisce la probabilità di transizione per unità di tempo ed esprime la conservazione dell'energia nei fenomeni d'urto.

\subsection{Teoria d'urto}

L'osservabile fondamentale in teoria d'urto è la \textit{sezione d'urto} $ \sigma $: dato un flusso costante di particelle incidenti su un bersaglio, essa è il numero di particelle diffuse per unità di flusso entrante ed unità di tempo. La sezione d'urto può essere vista come una generalizzazione quantistica della sezione geometrica di un bersaglio.

\begin{example}
	Nel caso classico, dato un flusso costante di particelle $ j = \frac{dn}{dt ds} $ incidenti in un intervallo $ \Delta t $ su un bersaglio di area $ S $, il numero di particelle diffuse sarà $ N = j S \Delta t $, dunque la sezione geometrica è $ S = \frac{N}{j \Delta t} $.
\end{example}

Nel caso quantistico, dove l'interazione ha natura probabilistica (e non deterministica), è utile definire la \textit{sezione d'urto differenziale}, ovvero la sezione d'urto per unità di spazio delle fasi finale, indicata con $ \frac{d \sigma}{d \Omega} $, dove $ \Omega $ è un osservabile (o più osservabili) che caratterizzano la cinematica dello stato finale.
Dato un flusso $ j_a $ di particelle incidenti nello stato $ \ket{a} $, la sezione d'urto differenziale è definita come:
\begin{equation}
	d \sigma \defeq \lim_{t \rightarrow \infty} \frac{1}{j_a \Delta t} \abs{\braket{b | \hat{S}(t,-t) | a}}^2 db
	\label{eq:5.26}
\end{equation}
dove $ \Delta t = 2t $ e $ \ket{b} $ è lo stato in cui il sistema viene misurato al tempo $ t $. L'elemento di matrice nel limite dei grandi tempi viene detto \textit{elemento di matrice} $ S $:
\begin{equation}
	S_{ab} \equiv \lim_{t \rightarrow \infty} \abs{\braket{b | \hat{S}(t,-t) | a}}^2
	\label{eq:5.27}
\end{equation}
La matrice $ S $ è unitaria.

\subsubsection{Spazio delle fasi}

Si consideri la situazione tipica in cui stato iniziale e finale sono stati d'impulso definito:
\begin{equation*}
	\braket{\ve{x} | \ve{k}} = \psi_\ve{k}(\ve{x}) = \frac{1}{(2\pi)^{3/2}} e^{i \ve{k} \cdot \ve{x}} \,\, : \,\, \braket{\ve{k} | \ve{k}'} = \delta^{(3)}(\ve{k} - \ve{k}')
\end{equation*}
È conveniente passare in coordinate sferiche, dunque parametrizzando l'impulso col modulo $ k $ e gli angoli $ \vartheta_k $ e $ \varphi_k $; inoltre, essendo l'energia l'osservabile più comune negli esperimenti, conviene determinare il modulo $ k $ a partire dall'energia: si vuole dunque passare dagli stati $ \ket{\ve{k}} $ agli stati $ \ket{E, \Omega_k} $.\\
Assumento che per $ t \rightarrow \pm \infty $ l'Hamiltoniana sia quella della particella libera, dunque che il potenziale di scattering sia acceso e spento solo in un intervallo di tempo finito $ [-T, T] $, si ha che:
\begin{equation*}
	E = \frac{\hbar^2 k^2}{2m}
	\quad \Rightarrow \quad
	\delta (E - E') = \frac{2m}{\hbar^2} \delta(k^2 - k'^2) = \frac{m}{\hbar k} \delta(k - k')
\end{equation*}
Inoltre, essendo l'elemento infinitesimo dello spazio delle fasi $ d^3\ve{k} = k^2 dk d \Omega_k $ ($ d\Omega_k \equiv d \cos \vartheta_k d \varphi_k $):
\begin{equation*}
	\delta^{(3)}(\ve{k} - \ve{k}') = \frac{1}{k^2} \delta(k - k') \delta^{(2)}(\Omega_k - \Omega_{k'})
\end{equation*}
Si ha dunque:
\begin{equation*}
	\braket{E, \Omega_k | E', \Omega_{k'}} = \delta(E - E') \delta^{(2)}(\Omega_k - \Omega_{k'}) = \frac{m}k{\hbar} \delta(\ve{k} - \ve{k}')
\end{equation*}
Gli stati differiscono dunque solo per una normalizzazione:
\begin{equation}
	\ket{E, \Omega_k} = \sqrt{\frac{mk}{\hbar}} \ket{\ve{k}}
	\label{eq:5.28}
\end{equation}
Per quanto riguarda il flusso di particelle incidenti, a livello quantistico si parla di flusso di probabilità incidente, il quale è definito come:
\begin{equation}
	\ve{j} = - \frac{i\hbar}{2m} \left[ \psi^* \nabla \psi - \nabla \psi^* \psi \right]
	\label{eq:5.29}
\end{equation}
Usando $ \psi = \braket{\ve{x} | E, \Omega_k} $ e prendendo il modulo, si trova:
\begin{equation}
	j_k = \frac{k^2}{\hbar (2\pi)^3}
	\label{eq:5.30}
\end{equation}
La sezione d'urto differenziale diventa dunque:
\begin{equation*}
	d \sigma = \lim_{t \rightarrow \infty} \frac{(2\pi)^3 \hbar}{k^2} \frac{1}{\Delta t} \abs{\braket{E', \Omega_{k'} | \hat{S}(t,-t) | E, \Omega_k}}^2 dE' d \Omega_{k'}
\end{equation*}
Il rapporto $ \lim_{t \rightarrow \infty} S / \Delta t $ è la probabilità di transizione per unità di tempo, dunque dalla regola aurea di Fermi Eq. \ref{eq:5.25} si ottiene:
\begin{equation*}
	d \sigma = \frac{(2\pi)^4}{k^2} \abs{\braket{E', \Omega_{k'} | V | E, \Omega_k}}^2 \delta(E' - E) dE' d \Omega_{k'}
\end{equation*}
Integrando su tutti i possibili valori di $ E' $ si trova la sezione d'urto differenziale:
\begin{equation*}
	\frac{d \sigma}{d \Omega_{k'}} = \frac{(2\pi)^4}{k^2} \abs{\braket{E, \Omega_{k'} | V | E, \Omega_k}}^2
\end{equation*}
Questa è la sezione d'urto al primo ordine perturbativo, nota come \textit{approssimazione di Born}. Essendo $ E' = E $, si ha $ k' = k $ ed è possibile definire l'impulso trasferito $ \ve{q} \equiv \ve{k}' - \ve{k} $, così che:
\begin{equation}
	\frac{d \sigma}{d \Omega} = \frac{(2\pi)^4 m^2}{\hbar^4} \abs{f(\ve{q})}^2
	\label{eq:5.31}
\end{equation}
dove si è definito il \textit{fattore di forma} del bersaglio:
\begin{equation}
	f(\ve{q}) \defeq \frac{1}{(2\pi)^3} \int d^3 \ve{x}\, V(\ve{x}) e^{i \ve{q} \cdot \ve{x}}
	\label{eq:5.32}
\end{equation}
Questa non è altro che la trasformata di Fourier del potenziale che determina lo scattering.

\begin{example}
	Se si considera un potenziale centrale $ V = V(r) $, scrivendo $ q = 2k \sin \frac{\theta}{2} $, con $ \theta $ scattering angle, si trova:
	\begin{equation*}
		f(\ve{q}) = \frac{1}{(2\pi)^3} \int_0^\infty dr \int_{-1}^1 d \cos \vartheta \int_0^{2\pi} d \varphi\, r^2 V(r) e^{i qr \cos \theta} = \frac{1}{(2\pi)^2} \int_0^\infty dr\, r V(r) \frac{2 \sin (qr)}{q}
	\end{equation*}
	ovvero:
	\begin{equation*}
		\frac{d \sigma}{d \Omega} = \frac{4m^2}{q^2 \hbar^4} \abs{ \int_0^\infty dr\, r \sin(qr) V(r) }^2
	\end{equation*}
\end{example}
Ad esempio, per il potenziale di Yukawa $ V(r) \sim \frac{1}{r} e^{-r / \lambda} $ si trova $ \frac{d \sigma}{d \Omega} \sim q^{-2} (q^2 + \lambda^{-2})^{-1} $, che nel limite Coulombiano $ \lambda \rightarrow \infty $ si riduce alla sezione d'urto di Rutherford $ \frac{d \sigma}{d \Omega} \sim \left( \sin \frac{\theta}{2} \right)^{-4} $.










