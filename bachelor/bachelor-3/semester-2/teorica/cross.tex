\selectlanguage{english}

\section{Cross-sections and \texorpdfstring{$ S $}{S}-matrix}

The cross-section generalizes the notion of geometric area of an object. Consider a flux of particles (particles per unit area per unit time):
\begin{equation*}
  j = \frac{\dd n}{\dd S \dd t}
\end{equation*}
Then, the number of incident particles diffused by a disk of section $ S $ in a time $ \Delta t $ is:
\begin{equation*}
  N = j S \Delta t
\end{equation*}
The (classical) \bctxt{cross-section} of the disk is then defined as:
\begin{equation}
  \sigma \defeq \frac{N}{j \Delta t}
\end{equation}
From a quantum point of view, $ N $ is the probability of the transition $ \ket{\text{i}} \rightarrow \ket{\text{f}} $ between the initial and final states, thus:
\begin{equation*}
  \dd \sigma = \frac{1}{j \Delta t} \abs{\braket{\text{f}(f) | \text{i}}}^2 \dd f
\end{equation*}
where $ f $ is some quantity which the final state depends on.

\subsection{Classical and quantum definition}

\subsection{Phase-space integration}

\begin{proposition}{Scattering cross-section}{}
  Given a $ 2 \rightarrow n $ scattering process with well-defined initial momenta $ p_\mathcal{A} $ and $ p_\mathcal{B} $, then the \bcprop{differential cross-section} is:
  \begin{equation}
    \dd\sigma = \frac{1}{2E_\mathcal{A} 2E_\mathcal{B} \abs{\ve{v}_\mathcal{A} - \ve{v}_\mathcal{B}}} \prod_{k = 1}^n \frac{\dd^3p_k}{(2\pi)^3 2E_k} \abs{\mathcal{M}(\mathcal{A} \mathcal{B} \rightarrow \{f\})}^2 (2\pi)^4 \delta^{(4)}(p_\mathcal{A} + p_\mathcal{B} - \textstyle\sum_{i = 1}^n p_i)
    \label{eq:scatt-cr-sec}
  \end{equation}
  where $ \mathcal{M}(\mathcal{A} \mathcal{B} \rightarrow \{f\}) $ is the matrix element of the scattering process and $ \ve{v}_k \equiv \frac{\ve{p}_k}{E_k} $ is the velocity of the $ k^\text{th} $ particle.
\end{proposition}

This cross-section is differential in $ 3n - 4 $ variables, as the momentum-conserving $ \delta^{(4)} $ allows to perform $ 4 $ integrations. The integration measure is defined as the \bctxt{invariant} $ n $\bctxt{-body phase space}:
\begin{equation}
  \int \dd \Pi_n \equiv \prod_{k = 1}^n \int \frac{\dd^3p_k}{(2\pi)^3 2E_k} (2\pi)^4 \delta^{(4)}(P - \textstyle\sum_{i = 1}^n p_i)
\end{equation}
with $ P $ the total initial 4-momentum.

\section{Feynman diagrams}
