\documentclass[a4paper, 12pt]{book}
\usepackage[italian]{babel}

\usepackage[]{csvsimple}
\usepackage{float}

\usepackage{ragged2e}
\usepackage[left=25mm, right=25mm, top=15mm]{geometry}
\geometry{a4paper}
\usepackage{graphicx}
\usepackage{booktabs}
\usepackage{paralist}
\usepackage{subfig} 
\usepackage{fancyhdr}
\usepackage{amsmath}
\usepackage{amssymb}
\usepackage{amsfonts}
\usepackage{amsthm}
\usepackage{mathtools}
\usepackage{enumitem}
\usepackage{titlesec}
\usepackage{braket}
\usepackage{gensymb}
\usepackage{url}
\usepackage{hyperref}
\usepackage{csquotes}
\usepackage{multicol}
\usepackage{graphicx}
\usepackage{wrapfig}
\usepackage{caption}

\usepackage{esint}

\captionsetup{font=small}
\pagestyle{fancy}
\renewcommand{\headrulewidth}{0pt}
\lhead{}\chead{}\rhead{}
\lfoot{}\cfoot{\thepage}\rfoot{}
\usepackage{sectsty}
\usepackage[nottoc,notlof,notlot]{tocbibind}
\usepackage[titles,subfigure]{tocloft}
\renewcommand{\cftsecfont}{\rmfamily\mdseries\upshape}
\renewcommand{\cftsecpagefont}{\rmfamily\mdseries\upshape}

\let\oldsection\section% Store \section
\renewcommand{\section}{% Update \section
	\renewcommand{\theequation}{\thesection.\arabic{equation}}% Update equation number
	\oldsection}% Regular \section
\let\oldsubsection\subsection% Store \subsection
\renewcommand{\subsection}{% Update \subsection
	\renewcommand{\theequation}{\thesubsection.\arabic{equation}}% Update equation number
	\oldsubsection}% Regular \subsection

\newcommand{\abs}[1]{\left\lvert#1\right\rvert}
\newcommand{\norm}[1]{\left\lVert#1\right\rVert}

\newcommand{\g}{\text{g}}
\newcommand{\m}{\text{m}}
\newcommand{\cm}{\text{cm}}
\newcommand{\mm}{\text{mm}}
\newcommand{\s}{\text{s}}
\newcommand{\N}{\text{N}}
\newcommand{\Hz}{\text{Hz}}

\newcommand{\virgolette}[1]{``\text{#1}"}
\newcommand{\tildetext}{\raise.17ex\hbox{$\scriptstyle\mathtt{\sim}$}}

\renewcommand{\arraystretch}{1.2}

\addto\captionsenglish{\renewcommand{\figurename}{Fig.}}
\addto\captionsenglish{\renewcommand{\tablename}{Tab.}}

\DeclareCaptionLabelFormat{andtable}{#1~#2  \&  \tablename~\thetable}

\setlength{\parindent}{0pt}

\graphicspath{{./images/}}

\title{\Huge\textbf{Struttura della Materia} \\ \large Proff. G. Onida e R. Guerra, a.a. 2024-25}
\author{Leonardo Cerasi%
	\thanks{\scriptsize\href{mailto:leonardo.cerasi@studenti.unimi.it}{leo.cerasi@pm.me}}%
	, Lucrezia Bioni\\
	\small GitHub repository: \href{https://github.com/LeonardoCerasi/notes}{LeonardoCerasi/notes}}

\begin{document}

\frontmatter

\maketitle
\tableofcontents
\pagestyle{indice}

\chapter{Aknowledgments}
Le immagini presenti in questi appunti sono tratte dal libro \href{https://link-springer-com.pros1.lib.unimi.it/book/10.1007/978-3-319-14382-8}{Introduction to the Physics of Matter} del Prof. Manini.

\mainmatter

\chapter*{Introduzione}
\pagestyle{introd}
\addcontentsline{toc}{chapter}{Introduzione}
\markboth{Introduzione}{}
\selectlanguage{italian}

\paragraph{Scale di grandezza}

Nello studio della fisica dei nuclei e delle particelle subatomiche, le scale di grandezza tipiche sono estremamente piccole: la scala tipica delle dimensioni di un atomo è $ 1\ang = 10^{-10}\m  $, mentre quella del nucleo è di $ 4 $ ordini di grandezza minore ($ 10^{-14}\m = 10\fm $); per un singolo nucleone, invece, le dimensioni sono dell'ordine di $ 1\fm = 10^{-15}\m $, e il range tipico delle interazioni deboli è $ 10^{-18}\m $.\\
Per quanto riguarda la scala di energie, i processi atomici hanno energie solitamente attorno a $ 1\ev = 1.602\cdot10^{-19}\,\text{J} $, mentre quelli nucleari arrivano anche a $ 10\mev $; per le interazioni ad alte energie studiate dalla fisica particellare, i moderni accelleratori arrivano a scale di $ 1\tev $.\\
Per studiare la struttura dei costituenti della materia a vari livelli, è necessario utilizzare fasci di particelle (fotoni, elettroni, etc.) con determinate lunghezze d'onda (relazioni di de Broglie $ \lambda = \frac{h}{p} $), corrispondenti a determinate energie: per sondare i nuclei atomici sono necessari $ \lambda \sim 10\fm $ ed $ E \sim 1\mev $; per evidenziare la struttura a molti corpi del nucleo atomico servono $ \lambda \sim 1\fm $ ed $ E \sim 10\mev $; se si vogliono studiare gli stati eccitati dei singoli nucleoni occorrono $ \lambda \sim 10^{-3}\fm $ ed $ E \sim 1\gev $; infine, se si vuole mettere in luce la struttura composta da quark dei nucleoni, bisogna raggiungere $ \lambda < 10^{-4}\fm $ ed $ E > 200\gev $.

\paragraph{Interazioni fondamentali}

I vari costituenti della materia interagiscono tramite $ 4 $ interazioni fondamentali:
\begin{enumerate}
  \item interazione elettromagnetica: mediata dal fotone ($ m_{\gamma} = 0 $), con coupling constant $ \alpha \approx 1/137 $ e raggio d'azione infinito (essendo il fotone massless);
  \item interazione debole: mediata dai bosoni $ \w^{\pm} $ e $ \z^0 $ ($ m_W = 80.4\gev $, $ m_Z = 90.1\gev $), con coupling constant $ G_F \approx 1\cdot10^{-5} $) e raggio d'azione $ < 10^{-3}\fm $, dovuto al fatto che i bosoni bosoni $ \w^{\pm} $ e $ \z^0 $ sono molto pesanti e dunque, per il principio d'indeterminazione ($ \Delta E \Delta t \ge \frac{\hbar}{2} $), possono essere prodotti solo come particelle virtuali in processi di scattering per periodi di tempo estremamente brevi;
  \item interazione forte: mediata dai gluoni ($ m_g = 0 $), con coupling constant $ \alpha_s \approx 1 $ e raggio d'azione $ \approx 1\fm $, dovuto al fatto che i gluoni, sebbene massless, possono interagire tra loro;
  \item interazione gravitazionale: mediata dall'ipotetico gravitone ($ m_G = 0 $), con coupling constant $ G_N \approx 6\cdot10^{-39} $ e raggio d'azione infinito.
\end{enumerate}
Come si può notare, la gravità ha un'intensità di decine di ordini di grandezza inferiore alle altre interazioni fondamentali, per questo in ambito atomico, nucleare e particellare può essere trascurata.

\paragraph{Esperimenti}

A differenza della fisica atomica, che è descritta completamente dalla QED (Quantum Electrodynamics), la fisica nucleare non ha un'unica teoria coerente: la teoria fondamentale dell'interazione forte, la QCD (Quantum Chromodynamics), descrive le interazioni tra quark (mediate da gluoni), non quelle tra nucleoni (mediate da mesoni virtuali); inoltre, in ambito atomico le energie che entrano in gioco nei decadimenti ($ \sim 10\mev $) sono meno dello $ 0.1\% $ della massa del nucleo (espressa in unità naturali), dunque gli effetti relativistici possono essere ignorati, mentre per quanto riguarda i processi tra nucleoni le energie possono essere anche $ 100 $ volte la massa equivalente del protone, rendendo necessario l'utilizzo della meccanica quantistica relativistica; infine, bisogna considerare che sia il nucleo atomico che i nucleoni sono sistemi complessi a molti corpi, dunque, anche avendo una teoria dell'interazione tra singole coppie di costituenti, è estremamente difficile sviluppare modelli teorici per descrivere questi sistemi, e la trattazione è principalmente di natura fenomenologica, con tante teorie dei singoli processi che vengono sviluppate a partire dai dati sperimentali.\\
Gli esperimenti in fisica nucleare (utilizzati anche per studiare gli adroni in generale) sono principalmente di due tipi:
\begin{enumerate}
  \item scattering: un fascio di particelle, di cui si conoscono energia e momento lineare, è diretto verso l'oggetto bersagio da studiare e, attraverso le variazioni di quantità cinematiche misurabili, è possibile studiare le proprietà dell'interazioni e la struttura del bersaglio (risoluzione data dalla relazione di de Broglie);
  \item spettroscopia: nucleoni (o anche mesoni e barioni) vengono eccitati e si studiano i prodotti del decadimento di questi stati eccitati, inferendo le proprietà degli stati eccitati e le interazioni tra i prodotti di decadimento.
\end{enumerate}
Sia esperimenti di scattering che esperimenti spettroscopici necessitano di energie di ordini di grandezza simili.\\
Nel caso dello scattering è importante studiare la sezione d'urto d'interazione (cross section), ovvero la probabilità che avvenga una determinata reazione: in base all'angolo solido $ \Delta\Omega $ del rilevatore, alla cross section $ \frac{d\sigma}{s\Omega} $, all'intensità $ I_0 $ del fascio incidente e alla densità numerica di particelle $ n_0 $ che attraversano lo spessore $ dz $ del rilevatore, si può calcolare il numero di particelle rilevate in funzione dell'angolo d'emissione:
\begin{equation}
  \frac{dn(\theta)}{dt} = I_0 n_0 dz \frac{d\sigma}{d\Omega} \Delta\Omega
  \label{eq:1}
\end{equation}
La cross section è un'area geometrica (l'area effettiva di collisione) ed è solitamente misurata in barn: $ 1\barn = 100\fm^2 $; questa sezione d'urto è in realtà molto grande e misure più tipiche sono espresse in microbarn.

\thispagestyle{introd}


\part{Atomi e Molecole}
\pagestyle{body}

\chapter{Atomi Idrogenoidi}
\selectlanguage{italian}

\section{Principio variazionale di Ritz}

Per lo studio di un sistema quantistico, è utile dimostrare un principio variazionale sul valore di aspettazione dell'Hamiltoniana $ \mathcal{H} $ del sistema, ovvero sull'energia:
\begin{equation}
	E[\psi] \defeq \frac{\braket{\psi | \mathcal{H} | \psi}}{\braket{\psi | \psi}} \in \R
\end{equation}

\begin{proposition}{}{}
	Il valore di aspettazione di una Hamiltoniana su un suo autostato è stazionario.

	\tcblower

	\begin{proof}
		Prendendo una variazione infinitesima $ \ket{\psi + \delta\psi} $ ed usando $ \mathcal{H} \ket{\psi} = E[\psi] \ket{\psi} $:
		\begin{equation*}
			\begin{split}
				\delta E
				&= E[\psi + \delta\psi] - E[\psi] = \frac{\braket{\psi + \delta\psi | \mathcal{H} | \psi + \delta\psi}}{\braket{\psi + \delta\psi | \psi + \delta\psi}} - \frac{\braket{\psi | \mathcal{H} | \psi}}{\braket{\psi | \psi}} \\
				&\simeq \frac{\braket{\psi | \mathcal{H} | \psi} + \braket{\psi | \mathcal{H} | \delta\psi} + \braket{\delta\psi | \mathcal{H} | \psi}}{\braket{\psi | \psi} + \braket{\psi | \delta\psi} + \braket{\delta\psi | \psi}} - \frac{\braket{\psi | \mathcal{H} | \psi}}{\braket{\psi | \psi}} \\
				&= \frac{\braket{\psi | \mathcal{H} | \delta\psi} + \braket{\delta\psi | \mathcal{H} | \psi} - E[\psi] \braket{\psi | \delta\psi} - E[\psi] \braket{\delta\psi | \psi}}{\braket{\psi | \psi} + \braket{\psi | \delta\psi} + \braket{\delta\psi | \psi}} \\
				&= \frac{2\Re \braket{\delta\psi | (\mathcal{H} - E[\psi]) | \psi}}{\braket{\psi | \psi} + \braket{\psi | \delta\psi} + \braket{\delta\psi | \psi}} = 0
			\end{split}
		\end{equation*}
	\end{proof}
\end{proposition}

\begin{theorem}{Principio variazionale di Ritz}{ritz}
	Detto $ \ket{\psi_0} $ lo stato fondamentale di $ \mathcal{H} $, allora $ E[\psi] \ge E[\psi_0] \equiv E_0 \,\,\forall \ket{\psi} \in \mathscr{H} $.

	\tcblower
	\begin{proof}
		Data una base di autostati $ \{u_n\} : \mathcal{H} \ket{u_n} = E_n \ket{u_n} \land \braket{u_n | u_m} = \delta_{nm} $, dove $ u_0 $ è il ground state con $ E_0 \le E_1 \le E_2 \le \dots $, per il generico stato $ \ket{\psi} = \sum_n A_n \ket{u_n} $:
		\begin{equation*}
			E[\psi] - E_0
			= \frac{\abs{A_0}^2 E_0 + \sum_{i \neq 0} \abs{A_i}^2 E_i}{\abs{A_0}^2 + \sum_{i \neq 0} \abs{A_i}^2} - E_0 = \frac{\sum_{i \neq 0} \left( E_i - E_0 \right) \abs{A_i}^2}{\abs{A_0}^2 + \sum_{i \neq 0} \abs{A_i}^2} \ge 0
		\end{equation*}
	\end{proof}
\end{theorem}

Questo risultato è utile poiché permette di trovare il ground state minimizzando l'energia: se si parametrizza la funzione d'onda, il ground state sarà dato dal set di parametri per cui si ha il minimo dell'energia.\\
Una possibile applicazione è quella di ottimizzare i coefficienti di uno sviluppo di una funzione d'onda su una base di funzioni d'onda fissate: vedendo i coefficienti della combinazione lineare come parametri, minimizzando l'energia si ottiene un sistema lineare di equazioni la cui risoluzione fornisce i parametri che meglio approssimano la funzione d'onda reale. Questo è il caso, ad esempio, della Linear Combination of Atomic Orbital, nel quale si esprime la funzione d'onda di una molecola come combinazione lineare delle funzioni d'onda dei suoi atomi costituenti.\\
Questo metodo può essere ulteriormente generalizzato facendo variare parametricamente anche le funzioni d'onda di base sulle quali si effettua lo sviluppo (es.: metodo di Hartree-Fock).

\section{Soluzione analitica}

L'atomo a singolo elettrone (o idrogenoide) è uno dei pochi casi in cui l'equazione di Schrödinger può essere risolta analiticamente. Essendo il potenziale coulombiano un potenziale radiale a simmetria sferica, si può separare il problema in moto del centro di massa e moto radiale: essendo la massa nucleo $ M $ migliaia di volte quella dell'elettrone $ m_e $, il centro di massa può essere approssimato con la posizione stessa del nucleo. La correzione di massa ridotta diventa importante quando si considerano atomi esotici, come ad esempio l'idrogeno muonico (sistema legato protone-muone).\\
Concentrandosi sull'Hamiltoniana di moto relativo (quella del centro di massa è semplicemente l'Hamiltoniana di particella libera):
\begin{equation}
	\mathcal{H} = - \frac{\hbar^2}{2\mu} \nabla_\ve{r}^2 - \frac{Ze}{r}
	\label{eq:1-e-ham}
\end{equation}
dove $ \ve{r} \equiv \ve{R} - \ve{r}_e $ e $ \mu \equiv (M^{-1} + m_e^{-1})^{-1} $. In coordinate sferiche si trova:
\begin{equation*}
	\nabla^2 = \frac{2}{r} \frac{\pa}{\pa r} + \frac{\pa^2}{\pa r^2} - \frac{\hat{L}^2}{\hbar^2 r^2}
\end{equation*}
Dato che $ [\mathcal{H} , \hat{L}^2] = 0 $, si può cercare una soluzione $ \psi = \psi(r,\vartheta,\varphi) $ in funzione delle autofunzioni di $ \hat{L}^2 $: questi sono le armoniche sferiche $ Y_{\ell,m} : \hat{L}^2 Y_{\ell,m} = \hbar^2 \ell (\ell + 1) Y_{\ell,m} $, con $ \ell \in \N_0 $ e $ -\ell \le m \le \ell $. Scrivendo $ \psi(r,\vartheta,\varphi) = R(r) Y_{\ell,m}(\vartheta,\varphi) $ si ha:
\begin{equation}
	\left[ - \frac{\hbar^2}{2\mu} \left( \frac{2}{r} \frac{\dd}{\dd r} + \frac{\dd^2}{\dd r^2} \right) + V_\ell(r) \right] R(r) = E R(r)
	\qquad \qquad
	V_\ell(r) \equiv - \frac{Ze}{r} + \frac{\hbar^2 \ell(\ell + 1)}{2\mu r^2}
	\label{eq:1-e-rad-eq}
\end{equation}
Risolvendo questa equazione si trova che la funzione d'onda radiale dipende da un numero quantico, il \textit{numero quantico radiale} $ n_r $, tale per cui:
\begin{equation*}
	E_{n_r,\ell} = - \frac{Z^2 e^4 \mu}{2\hbar^2 (n_r + \ell + 1)^2}
\end{equation*}
Questo numero quantico rappresenta il numero di nodi nella funzione d'onda radiale $ R_{n_r,\ell}(r) $, nel caso di un potenziale radiale a simmetria sferica. Conviene definire un numero quantico equivalente, il \textit{numero quantico principale} $ n \equiv n_r + \ell + 1 $, così che:
\begin{equation}
	E_n = - \frac{Z^2 e^4 \mu}{2\hbar^2} \frac{1}{n^2} = - \frac{Z^2}{2} E_\text{Ha} \frac{\mu}{m_e} \frac{1}{n^2}
	\label{eq:1-e-en}
\end{equation}
Vale inoltre che $ 0 \le \ell \le n-1 $.

\paragraph{Degenerazione}

Dallo spettro energetico Eq. \ref{eq:1-e-en} si vede che l'energia non dipende né da $ m $ né da $ \ell $: nel primo caso si parla di \textit{degenerazione necessaria}, in quanto è una degenerazione dovuta alla simmetria sferica del problema e comporta $ d(\ell) = 2\ell + 1 $, mentre nel secondo caso di ha una \textit{degenerazione accidentale}, legata alla forma particolare del potenziale coulombiano. La presenza di quest'ultima degenerazione accidentale permette di definire il numero quantico principale, e si ha una degenerazione overall di $ d(n) = \sum_{\ell = 0}^{n-1} (2\ell + 1) = n^2 $.

\subsection{Funzione d'onda angolare}

L'armonica sferica $ Y_{\ell,m}(\vartheta,\varphi) $ contiene informazione esatta sul momento angolare totale e sulla sua proiezione sull'asse $ z $, in quanto $ \hat{L}^2 Y_{\ell,m} = \hbar^2 \ell (\ell + 1) $ e $ \hat{L}_z Y_{\ell,m} = \hbar m $, mentre quella sul momento angolare lungo le altre direzioni è di natura probabilistica, poiché $ [\hat{L}_x , \hat{L}_z],[\hat{L}_y , \hat{L}_z] \neq 0 $. Si noti, però, che grazie alla simmetria sferica la definizione di $ \ve{e}_z $ è arbitraria.\\
La forma generica di un'armonica sferica è data da:
\begin{equation}
	Y_{\ell,m}(\vartheta,\varphi) = \mathcal{N} e^{i m \varphi} P_\ell^{\abs{m}}(\cos \vartheta)
\end{equation}
dove $ P_\ell^{\abs{m}}(\cos \vartheta) $ è la funzione di Legendre. Si vede dunque che $ \abs{Y_{\ell,m}}^2 $ è indipendente da $ \varphi $, dunque la probabilità $ \abs{\braket{\vartheta,\varphi | n,\ell,m}}^2 $ dipende solo da $ \vartheta $. Inoltre, si trova che $ \ell - \abs{m} $ è pari al numero di nodi della funzione di Legendre, e che sotto operatore di parità $ \mathcal{P} Y_{\ell,m} = (-1)^\ell Y_{\ell,m} $.\\
Data la simmetria necessaria rispetto ad $ m $, si possono combinare orbitali con $ \pm m $ per ottenere degli orbitali reali: $ Y_{0,0} = \frac{1}{\sqrt{4\pi}} $ e $ Y_{\ell,0} \propto \cos{\vartheta} $ sono sempre reali, mentre ad esempio si definiscono gli orbitali reali $ \ch{p}_x , \ch{p}_y $ come:
\begin{equation*}
	\psi_{\ch{p}_x} = \frac{1}{\sqrt{2}} \left( Y_{1,-1} - Y_{1,1} \right)
	\qquad \qquad
	\psi_{\ch{p}_y} = \frac{i}{2} \left( Y_{1,-1} + Y_{1,1} \right)
\end{equation*}

\subsection{Funzione d'onda radiale}

Per quanto riguarda la funzione d'onda radiale $ R_{n_r,\ell}(r) $, essa ha la forma generica:
\begin{equation}
	R_{n,\ell}(r) = \mathcal{N} (kr)^\ell L_{n+\ell}^{2\ell + 1}(kr) e^{-\frac{1}{2} kr}
	\qquad
	k \equiv \frac{2Z}{a} \frac{1}{n}
	\label{eq:1-e-radial}
\end{equation}
dove $ L_{n+\ell}^{2\ell+1}(kr) $ è il polinomio di Laguerre, un polinomio di grado $ n - \ell - 1 $ (con termine noto non-nullo) che presenta un numero di nodi pari a $ n - \ell - 1 $, ed $ a \equiv \frac{m_e}{\mu} a_0 $ è la mass-rescaled atomic length. Si vedono subito i due limiti:
\begin{equation*}
	\lim_{r \rightarrow 0} R_{n,\ell}(r) \sim r^\ell
	\qquad \qquad
	\lim_{r \rightarrow \infty} R_{n,\ell}(r) \sim r^{n-1} e^{-\frac{1}{2} kr}
\end{equation*}
La distribuzione di probabilità spaziale è data dunque da:
\begin{equation}
	P_{n,\ell,m}(\ve{r}) \dd^3r = \abs{R_{n,\ell}(r)}^2 \abs{Y_{\ell,m}(\vartheta,\varphi)}^2 r^2 \sin{\vartheta} \dd r \dd \vartheta \dd \varphi
\end{equation}
La probabilità $ P_{n,0,0}(\ve{r}) $ ha un massimo assoluto in $ \ve{r} = \ve{0} $: la posizione più probabile per l'elettrone nello stato $ \ch{s} $ è il centro del potenziale, dove esso è più attrattivo. Per $ \ell \neq 0 $, invece, $ P_{n,\ell,m}(\ve{0}) = 0 $, e ciò evidenzia l'impossibilità per una particella dotata di momento angolare di cadere nel centro di un potenziale centrale.\\
Si osserva inoltre che, all'aumentare di $ Z $, il massimo di $ \abs{R_{n,\ell}(r)}^2 $ e dunque di $ P_{n,\ell,m}(\ve{r}) $ si sposta verso l'origine\footnote{In particolare $ R_{n,\ell}^{[Z]}(r) = Z^{3/2} R_{n,\ell}^{[1]}(rZ) $, quindi $ P_{n,\ell,m}^{[Z]}(r) = Z P^{[1]}(rZ) $.}: ciò implica che la distanza elettrone-nucleo è $ \propto Z^{-1} $, dunque, dato che $ V \propto Z/r $, si trova l'andamento dell'energia $ E \propto Z^2 $.\\
La funzione d'onda più semplice, quello dello stato $ 1\ch{s} $, si trova essere:
\begin{equation}
	\psi_{1,0,0}(\ve{r}) = \frac{1}{\sqrt{\pi}} \left( \frac{Z}{a} \right)^{3/2} e^{- Z r / a}
\end{equation}

\section{Spettro energetico}

Quando si parla di spettro d'eccitazione si intende lo spettro delle differenze di autovalori di energia: dati due stati $ \ket{i} , \ket{f} $, si ha $ \Delta E = E_f - E_i $. Dall'Eq. \ref{eq:1-e-en} si trova:
\begin{equation}
	\Delta E = - \frac{\mu}{m_e} \frac{E_\text{Ha}}{2} Z^2 \left( \frac{1}{n_f^2} - \frac{1}{n_i^2} \right)
	\label{eq:1-e-spectr}
\end{equation}
Queste sono quantità misurabili spettroscopicamente.

\subsection{Atomo di idrogeno}

Per l'atomo di idrogeno si ha il ground state $ E_1 = - \frac{1}{2} E_\text{Ha} \frac{\mu}{m_e} = - 13.5983\ev $. Le transizioni $ n_i \rightarrow n_f $ si raggruppano in serie, ciascuna caratterizzata dallo stesso stato finale $ n_f $ ad energia più bassa: per l'atomo di idrogeno, ciascuna serie è osservata in una presisa regione caratteristica dello spettro elettromagnetico, ed in particolare le serie di Lyman ($ n_f = 1 $) e Balmer ($ n_f = 2 $) non presentano alcun overlap con altre serie, dato che la distanza energetica tra $ E_1 \simeq - 13.60\ev $ o $ E_2 \simeq -3.39\ev $ ed il successivo stato eccitato supera l'intero range energetico tra quest'ultimo e l'ionization threshold ($ E = 0 $).\\
Ricordando che i fotoni nel visibile si trovano nel range $ 1.8\ev - 3\ev $ ($ 400\,\text{nm} - 700\,\text{nm} $), si trova che la serie di Lyman ($ 10\ev < E < 10\ev $) è nell'ultravioletto, la serie di Balmer ($ 1.8\ev < E < 3.2\ev $) è nel visibile e quella di Paschen ($ E < 1.5\ev $) nell'infrarosso.\\
Come da Eq. \ref{eq:1-e-spectr}, lo spettro atomico risente di un prefattore $ \mu / m_e $ che determina una debole dipendenza dalla massa del nucleo $ M $: di conseguenza, miscele di isotopi presenteranno delle duplicazioni di linee spettrali, sebbene energeticamente estremamente vicine. Inoltre, la dipendenza dell'energia da $ Z^2 $ può portare a sovrapposizioni parziali degli spettri di elementi diversi: as esempio, metà delle righe di metà delle serie dell'elio si sovrappongono a quelle dell'idrogeno.

\subsection{Modello di Bohr}

Gli spettri atomici furono osservati prima della formulazione della meccanica quantistica, dunque Bohr propose un modello per spiegarli.\\
Il modello di Bohr assume soltanto che gli elettroni si muovano attorno al nucleo in orbite circolari e che il loro momento angolare sia quantizzato in unità di $ \hbar $: $ mvr = n\hbar $, con $ n \in \N_0 $. La circolarità dell'orbita fa sì che la forza coulombiana sia di natura centripeta, ovvero:
\begin{equation*}
	\frac{mv^2}{r} = \frac{Ze^2}{r^2}
	\qquad \Rightarrow \qquad
	r = \frac{n^2 \hbar^2}{m_e Z e} = \frac{a_0}{Z} n^2
\end{equation*}
Si vede dunque che la distanza dell'elettrone dal nucleo è anch'essa quantizzata da $ n $, ed inoltre dipende da $ Z^{-1} $. L'energia cinetica e quella potenziale dell'elettrone sono:
\begin{equation*}
	T = \frac{1}{2} m v^2 = \frac{1}{2} \frac{Z^2 e^2}{a_0} \frac{1}{n^2} = \frac{1}{2} E_\text{Ha} \frac{Z^2}{n^2}
	\qquad \qquad
	U = - \frac{Z e^2}{r^2} = - E_\text{Ha} \frac{Z}{n^2}
\end{equation*}
Ciò è consistente col teorema del viriale\footnote{Il teorema del viriale stabilisce che, se $ U \propto r^\alpha $, allora $ \braket{T} = \frac{\alpha}{2} \braket{U} $.}, e per l'energia dell'elettrone si trova quindi:
\begin{equation*}
	E = T + U = - \frac{1}{2} E_\text{Ha} \frac{Z}{n^2} \equiv E_n
\end{equation*}
Si ha dunque un accordo perfetto con lo spettro energetico osservato. Risulta però errato lo spettro del momento angolare: ad esempio, nel ground state il modello di Bohr richiede un momento angolare pari a $ \hbar $, mentre un elettrone nello stato $ 1\ch{s} $ ha momento angolare nullo. Ciò è anche evidente considerando che l'elettrone ha una probabilità non-nulla di trovarsi in $ \ve{r} = \ve{0} $, ovvero all'interno del nucleo (es.: cattura elettronica), il che sarebbe impossibile se esso fosse dotato di momento angolare non-nullo.

\section{Momento magnetico e spin}

Il momento angolare di una particella carica in un'orbita periodica è associato ad un momento di dipolo magnetico. Da un punto di vista puramente classico, considerando una particella di massa $ m $ e carica $ q $ in un'orbita circolare di raggio $ r $ con velocità $ v $, il suo momento angolare è $ \ve{L} = \ve{r} \times \ve{p} = mrv \hat{\ve{n}} $; inoltre, si può associare ad essa una corrente $ i = \frac{q}{T} = \frac{qv}{2\pi r} $, così che il momento magnetico ad esso associato sia:
\begin{equation*}
	\boldsymbol{\mu} = i \pi r^2 \hat{\ve{n}} = \frac{q}{2} vr \hat{\ve{n}} = \frac{q}{2m} \ve{L}
\end{equation*}
Si dimostra che questa relazione è indipendente dalla forma dell'orbita e che vale, sotto forma operatoriale ($ \ve{L} \equiv \hbar \boldsymbol{\ell} $), anche in ambito quantistico:
\begin{equation}
	\boldsymbol{\mu} = g_\ell \mu_m \boldsymbol{\ell}
\end{equation}
dove $ \mu_m \equiv \frac{\hbar q}{2m} $ e $ g_\ell $ è detto coefficiente giromagnetico, che nel caso del momento angolare orbitale vale $ g_\ell = 1 $.

\begin{example}{Momento magnetico orbitale dell'elettrone}{}
	Nel caso dell'elettrone si ha:
	\begin{equation*}
		\boldsymbol{\mu} = - \mu_\text{B} \boldsymbol{\ell}
		\qquad \qquad
		\mu_\text{B} \equiv \frac{\hbar q_e}{2m_e} = 9.27401 \cdot 10^{-24} \,\text{J}\,\text{T}^{-1}
	\end{equation*}
	dove $ \mu_\text{B} $ è detto \textit{magnetone di Bohr}.
\end{example}

Ciò permette di misurare il momento angolare atomico facendo interagire il momento magnetico di un atomo con un campo magnetico esterno: in un campo magnetico uniforme, $ \boldsymbol{\mu} $ subirà una precessione attorno alla direzione di $ \ve{B} $ con frequenza $ \omega = \frac{q_e B}{2m_e} $ (frequenza di Larmor).

\subsection{Esperimento di Stern-Gerlach}

L'energia d'interazione di un momento magnetico con un campo magnetico è data da:
\begin{equation}
	\mathcal{H}_B = - \boldsymbol{\mu} \cdot \ve{B}
	\label{eq:ham-mag-int}
\end{equation}
In assenza di fattori che alterino l'angolo tra $ \boldsymbol{\mu} $ e $ \ve{B} $, questa energia è conservata nel tempo. In presenza di un campo magnetico non uniforme, sul momento magnetico agisce una forza:
\begin{equation}
	\ve{F} = - \nabla \mathcal{H}_B = \nabla (\boldsymbol{\mu} \cdot \ve{B})
	\label{eq:stern-gerlach}
\end{equation}
Assumendo che la componente $ z $ sia dominante, si ha $ F_z \simeq \mu_z \frac{\pa B_z}{\pa z} $.\\
Questa forza è alla base del funzionamento dell'apparato di Stern-Gerlach, che può essere visto come un dispositivo per misurare la componente $ z $ del momento magnetico atomico. Si consideri un fascio collimato di atomi neutri a velocità termiche emesso all'interno di una camera a vuoto: applicando un campo magnetico fortemente inomogeneo lungo l'asse $ z $, si va a deviare la traiettoria di ogni singolo atomo proporzionalmente al suo $ \mu_z $.\\
Classicamente, ci si aspetterebbe di osservare una distribuzione continua di atomi deviati, a seconda di vari valori di $ \mu_z $ distribuiti continuamente (assumendo $ \boldsymbol{\mu} $ distribuito casualmente nello spazio): si osserva, però, che essi si dispongono in due picchi discreti ben definiti, suggerendo dunque che $ \mu_z $ sia quantizzato. Questo risultato va a confermare la predizione quanto-meccanica della quantizzazione del momento angolare.\\
Effettuando l'esperimento di Stern-Gerlach con atomi di idrogeno, però si trovano comunque due picchi discreti: questo non è spiegabile tramite la quantizzazione del momento angolare orbitale, dato che $ \ch{H} $ ha $ \ell = 0 $, dunque $ L_z $ dovrebbe avere $ 2\ell + 1 = 1 $ possibili autovalori. Ciò suggerisce l'esistenza di un ulteriore grado di libertà negli atomi a singolo elettrone.

\subsection{Spin elettronico}

Il grado di libertà aggiuntivo sugegrito dall'esperimento di Stern-Gerlach è lo \textit{spin}, introdotto da Pauli come grado di libertà interno non-classico dell'elettrone. In particolare, lo spin può essere visto (in maniera imprecisa) come il momento angolare intrinseco dell'elettrone: matematicamente, esso si comporta come un momento angolare, ed in particolare $ \hat{S}^2 $ ha autovalori $ \hbar^2 s(s+1) $, mentre $ \hat{S}_z $ ha autovalori $ \hbar m_s $, con $ -s \le m_s \le s $.\\
Dall'esperimento di Stern-Gerlach con l'idrogeno si evince che $ 2s + 1 = 2 $, ovvero per l'elettrone $ s = \frac{1}{2} $: si indicano i due stati $ m_s = + \frac{1}{2} $ ed $ m_s = - \frac{1}{2} $ come $ \uparrow $ e $ \downarrow $. La funzionde d'onda totale dell'elettrone in un atomo idrogenoide sarà dunque:
\begin{equation}
	\psi_{n,\ell,m,m_s}(r,\vartheta,\varphi,\sigma) = R_{n,\ell}(r) Y_{\ell,m}(\vartheta,\varphi) \chi_{m_s}(\sigma)
\end{equation}
con $ \sigma \in \{-\frac{1}{2} , +\frac{1}{2}\} $ variabile relativo al grado di libertà di spin, così che $ \chi_{m_s}(\sigma) = \braket{\sigma | m_s} = \delta_{m_s , \sigma} $. Inoltre, si trova che la separazione dei due fasci in un apparato di Stern-Gerlach è compatibile con $ g_s = 2 $, che si trova essere vero per tutte le particelle fondamentali (structure-less): al momento magnetico orbitale va aggiunto un momento magnetico di spin.

\begin{proposition}{Momento magnetico totale}{}
	Detto $ \boldsymbol{j} = \boldsymbol{\ell} + \boldsymbol{s} $ il momento angolare totale, si ha:
	\begin{equation}
		\boldsymbol{\mu} = g_j \mu_m \boldsymbol{j}
		\qquad \qquad
		g_j = \frac{1}{2} \left( g_\ell + g_s \right) + \frac{1}{2} \frac{\ell (\ell + 1) - s (s + 1)}{j (j + 1)} \left( g_\ell - g_s \right)
	\end{equation}

	\tcblower

	\begin{proof}
		Per il momento magnetico totale si definisce\footnote{Questo è un caso specifico del teorema di Wigner-Eckardt: le quantità vettoriali medie di un oggetto a simmetria sferica liberamente rotante nello spazio sono proporzionali al suo momento angolare totale medio.}:
		\begin{equation*}
			\bs{\mu} = g_\ell \mu_m \bs{\ell} + g_s \mu_m \bs{s} \eqdef g_j \mu_m \bs{j}
		\end{equation*}
		Ricordando che $ \bs{\ell} \cdot \bs{s} = \frac{1}{2} ( \bs{j}^2 - \bs{\ell}^2 - \bs{s}^2 ) $, si ha:
		\begin{equation*}
			g_j J^2 = g_\ell \bs{\ell} \cdot \left( \bs{\ell} + \bs{s} \right) + g_s \bs{s} \cdot \left( \bs{\ell} + \bs{s} \right) = g_\ell \bs{\ell}^2 + g_s \bs{s}^2 + \left( g_\ell + g_s \right) \bs{\ell} \cdot \bs{s}
		\end{equation*}
		Promuovendo $ \bs{\ell} $, $ \bs{s} $ e $ \bs{j} $ ad operatori e sostituendoli con $ \braket{\hat{A}} \equiv \braket{j,m_j | \hat{A} | j,m_j} $:
		\begin{equation*}
			\begin{split}
				g_j j (j + 1)
				&= g_\ell \ell (\ell + 1) + g_s s (s + 1) + \frac{1}{2} \left( g_\ell + g_s \right) \left( j (j + 1) - \ell (\ell + 1) - s (s + 1) \right) \\
				&= \frac{1}{2} \left( g_\ell + g_s \right) j (j + 1) + \frac{1}{2} \left( \ell (\ell + 1) - s (s + 1) \right) \left( g_\ell - g_s \right)
			\end{split}
		\end{equation*}
	\end{proof}
\end{proposition}

\begin{definition}{Fattori di Landé}{}
	Nel caso dell'elettrone ($ g_\ell = 1 $, $ g_s = 2 $) si definisce il \textit{fattore di Landé} come:
	\begin{equation}
		g_j = \frac{3 j (j + 1) + s (s + 1) - \ell (\ell + 1)}{2 j (j + 1)}
		\label{eq:lande-g-factor}
	\end{equation}
\end{definition}

\section{Struttura fine}

Osservando sperimentalmente lo spettro dell'idrogeno, si osservano degli splitting delle linee spettrali molto piccoli ($ < 0.1 \,\text{meV} $) riconducibili a delle correzioni di natura relativistica.

\subsection{Interazione spin-orbita}

Considerando l'elettrone in orbita circolare attorno al nucleo con raggio $ \ve{r} $ e velocirà $ \ve{v} $, nel RF dell'elettrone il nucleo si muoverà con velocità $ -\ve{v} $, dunque gli sarà associata una corrente $ -Zq_e \ve{v} $. Per la legge di Biot-Savart, l'elettrone sarà quindi soggetto ad un campo magnetico:
\begin{equation*}
	\ve{B}(\ve{r}) = - \frac{1}{4\pi \epsilon_0 c^2} \frac{\ve{r} \times (-Zq_e \ve{v})}{r^3} = \frac{Zq_e}{4\pi \epsilon_0 c^2} \frac{\ve{r} \times \ve{v}}{r^3} = \frac{Zq_e}{4\pi \epsilon_0 c^2 m_e} r^{-3} \ve{L}
\end{equation*}
Si vede che questo è un effetto relativistico di ordine $ (v/c)^2 $.\\
Per descrivere l'interazione tra lo spin dell'elettrone e questo campo magnetico, bisogna apportare una correzione all'Eq. \ref{eq:ham-mag-int}, dovuta al fatto che l'RF dell'elettrone è accellerato, il che determina un fattore di $ \frac{1}{2} $. Si ottiene dunque:
\begin{equation*}
	\mathcal{H}_\text{s-o} = - \frac{1}{2} \boldsymbol{\mu}_s \cdot \ve{B}(\ve{r}) = \frac{1}{2} g_s \mu_\text{B} \bs{s} \cdot \left( \frac{Zq_e \hbar}{4\pi \epsilon_0 c^2 m_e} r^{-3} \bs{\ell} \right) = \frac{Ze^2 \hbar^2}{2m_e^2 c^2} r^{-3} \bs{s} \cdot \bs{\ell}
\end{equation*}
Questo operatore d'\textit{interazione spin-orbita} presenta dei termini off-diagonal piccoli ma non-nulli del tipo $ \braket{n,\ell,m,m_s | \mathcal{H}_\text{s-o} | n',\ell,m',m_s'} \neq 0 $ tra stati con stesso $ \ell $: gli stati con $ n' \neq n $ hanno energie non-relativistiche molto diverse, dunque le perturbazioni alla diagonale di $ \mathcal{H}_\text{s-o} $ da essi indotte possono essere ignorate. 

\begin{proposition}{Interazione spin-orbita}{}
	L'operatore d'interazione spin-orbita può essere scritto come:
	\begin{equation}
		\mathcal{H}_\text{s-o} = \sum_{n \in \N} \sum_{\ell = 1}^{n - 1} \xi_{n,\ell} \ket{n,\ell} \bra{n,\ell} \bs{s} \cdot \bs{\ell}
	\end{equation}
	dove il fattore $ \xi_{n,\ell} $ è dato da:
	\begin{equation}
		\xi_{n,\ell} = Z^4 \alpha^2 E_\text{Ha} \frac{(\mu / m_e)^3}{n^3 \ell (\ell + 1) (2\ell + 1)}
		\label{eq:1-e-int-spin-orb}
	\end{equation}

	\tcblower

	\begin{proof}
		L'operatore d'interazione spin-orbita può quindi essere riscritto come:
		\begin{equation*}
			\mathcal{H}_\text{s-o} = \frac{Ze^2\hbar^2}{2m_e^2c^2} \sum_{n,\ell} \ket{n,\ell} \braket{n,\ell | r^{-3} | n,\ell} \bra{n,\ell} \bs{s} \cdot \bs{\ell}
		\end{equation*}
		Per $ \ell > 0 $ si ha:
		\begin{equation*}
			\braket{n,\ell | r^{-3} | n,\ell} = \int_0^\infty r^{-3} \abs{R_{n,\ell}(r)}^2 r^2 \dd r = \left( \frac{Z}{a} \right)^3 \frac{2}{n^3 \ell (\ell + 1) (2\ell + 1)}
		\end{equation*}
		La tesi segue ricordando le definizioni di $ E_\text{Ha} $ ed $ \alpha $.
	\end{proof}
\end{proposition}

Con questa scrittura, si vedono immediatamente le caratteristiche dell'interazione spin-orbita:
\begin{itemize}
	\item $ \xi_{n,\ell} > 0 $, dunque $ \mathcal{H}_\text{s-o} $ favorisce le configurazioni con $ \ve{S} $ ed $ \ve{L} $ antiparalleli;
	\item $ \mathcal{H}_\text{s-o} \sim \alpha^2 $, dunque è una correzione relativistica di ordine $ (v/c)^2 $, ed è $ \alpha^2 \simeq 5.3 \cdot 10^{-5} $ volte più piccola delle tipiche energie orbitali (es.: primo stato su cui influisce ha $ \xi_{2\ch{p}} = 0.0301 \,\text{meV} $);
	\item $ \mathcal{H}_\text{s-o} \sim Z^4 $, dato che il campo generato dal nucleo va come $ \sim Z $ e la distanza nucleo-elettrone media va come $ \sim Z^{-1} $;
	\item $ \mathcal{H}_\text{s-o} \sim n^{-3} $, dato che la distanza nucleo-elettrone media va come $ \sim n $;
	\item $ \mathcal{H}_\text{s-o} \sim \ell^{-3} $, a causa dell'andamento $ R_{n,\ell} \sim r^\ell $ per $ r \rightarrow 0 $, regime in cui l'interazione spin-orbita è dominante.
\end{itemize}
Data la piccola energy scale, l'interazione spin-orbita può essere trattata perturbativamente al prim'ordine in $ \alpha^2 $. A tal fine, è utile ricordare gli elementi di matrice di $ \ve{s} \cdot \boldsymbol{\ell} $ nella coupled basis:
\begin{equation}
	\braket{\ell,s,j,m_j | \bs{s} \cdot \bs{\ell} | \ell,s,j',m_{j'}} = \frac{1}{2} \left( j(j + 1) - \ell (\ell + 1) - s(s + 1) \right) \delta_{j,j'} \delta_{m_j, m_{j'}}
	\label{eq:sl-coupled-basis}
\end{equation}
Nella coupled basis, dunque, l'interazione spin-orbita è diagonale e determina una correzione al prim'ordine dell'energia pari a:
\begin{equation}
	\Delta E^{(1)}_\text{s-o}(n,\ell,s,j) = Z^4 \alpha^2 E_\text{Ha} \left( \frac{\mu}{m_e} \right)^3 \frac{j(j + 1) - \ell(\ell + 1) - s(s + 1)}{2n^3 \ell (\ell + 1) (2\ell + 1)}
	\label{eq:spin-orbit-spectr}
\end{equation}
Si introduce la \textit{notazione spettroscopica} $ n ^{2s+1} [\ell] _j \equiv \ket{n,\ell,s,j, m_j} \,\,\forall m_j \in [-j,+j] $, la quale lascia indeterminato $ m_j $: l'interazione spin-orbita porta ad uno splitting degli orbitali $ ^{2s+1}[\ell] $ in dei multipletti $ ^{2s+1}[\ell]_j $ con $ j = \ell \pm \frac{1}{2} $, spiegando dunque lo sdoppiamento delle linee spettrali osservato sperimentalmente. Si trova facilmente che i livelli energetici splittati $ j \pm \frac{1}{2} $ hanno separazione enrgetica $ \Delta E_\text{s-o} = \xi_{n,\ell} \left( \ell + \frac{1}{2} \right) $.

\begin{example}{Splitting di orbitali p}{}
	Per un orbitale $ \ch{^2 P} $ si trova:
	\begin{equation*}
		\braket{1, \tfrac{1}{2}, j, m_j | \bs{s} \cdot \bs{\ell} | 1, \tfrac{1}{2}, j, m_j} =
		\begin{cases}
			-1 & j = \frac{1}{2} \\
			+ \frac{1}{2} & j = \frac{3}{2}
		\end{cases}
	\end{equation*}
	Questo viene dunque splittato in un doppietto $ \ch{^2P_{1/2}} $ (energia più bassa) ed un quartetto $ \ch{^2P_{3/2}} $ (energia più alta), energeticamente separati da $ \frac{3}{2} \xi_{n,1} $. Ad esempio, l'orbitale $ \ch{2^2P} $ dell'idrogeno subisce uno splitting di $ 45.2 \,\mu\text{eV} $.
\end{example}

\subsection{Correzione cinetica relativistica}

Un'ulteriore correzione relativistica di ordine $ (v/c)^2 $ deriva dall'energia cinetica, con $ p \ll \mu c $:
\begin{equation*}
	T = \sqrt{\mu^2 c^4 + p^2 c^2} - \mu c^2 = \mu c^2 \left[ 1 + \frac{1}{2} \frac{p^2}{\mu^2 c^2} - \frac{1}{8} \frac{p^4}{\mu^4 c^4} + \dots - 1 \right] = \frac{p^2}{2\mu} - \frac{p^4}{8 \mu^3 c^2} + \dots
\end{equation*}
Anche questa correzione può essere trattata come una perturbazione al prim'ordine.

\begin{proposition}{Correzione cinetica}{}
	Indipendentemente dalla base (coupled o uncoupled), si ha:
	\begin{equation}
		\braket{n,\ell | - \frac{p^4}{8 \mu^3 c^2} | n,\ell} = - \frac{Z^4 \alpha^2}{n^3} E_\text{Ha} \left( \frac{\mu}{m_e} \right)^3 \left( \frac{1}{2\ell + 1} - \frac{3}{8n} \right)
		\label{eq:kin-corr}
	\end{equation}

	\tcblower

	\begin{proof}
		Dall'Eq. \ref{eq:1-e-ham} si può riscrivere:
		\begin{equation*}
			\frac{p^4}{4\mu^2} = \left( \mathcal{H} + \frac{Ze^2}{r} \right)^2
		\end{equation*}
		Ciò permette di riscrivere l'elemento di matrice di $ p^2 $ in termini degli elementi di matrice di $ r^{-1} $ ed $ r^{-2} $; questi sono integrali radiali, dunque indipendenti dalla parte angolare (ovvero dalla scelta della base coupled o uncoupled).
	\end{proof}
\end{proposition}

La correzione relativistica di ordine $ \alpha^2 $ risulta essere quindi:
\begin{equation}
	\mathcal{H}_\text{rel} = \mathcal{H}_\text{s-o} - \frac{p^4}{8 \mu^3 c^2}
	\label{eq:ham-rel-corr}
\end{equation}

\begin{proposition}{Correzione relativistica}{}
	Al prim'ordine, la correzione all'energia determinata dalla perturbazione relativistica in Eq. \ref{eq:ham-rel-corr} è:
	\begin{equation}
		\Delta E_\text{rel}^{(1)}(n,j) = - \frac{Z^4 \alpha^2}{n^3} E_\text{Ha} \left( \frac{\mu}{m_e} \right)^3 \left( \frac{1}{2j + 1} - \frac{3}{8n} \right)
		\label{eq:rel-corr-spectr}
	\end{equation}

	\tcblower

	\begin{proof}
		Dalle Eqq. \ref{eq:spin-orbit-spectr}-\ref{eq:kin-corr}:
		\begin{equation*}
			\Delta E_\text{rel}^{(1)} = \frac{Z^4 \alpha^2}{n^3} E_\text{Ha} \left( \frac{\mu}{m_e} \right)^3 \left[ \frac{j(j+1) - \ell(\ell+1) - s(s+1)}{2\ell (\ell+1) (2\ell+1)} - \frac{1}{2\ell + 1} + \frac{3}{8n} \right]
		\end{equation*}
		Ricordando che per l'elettrone $ s = \frac{1}{2} $ e $ j = \ell \pm \frac{1}{2} $, si ha:
		\begin{equation*}
			\begin{split}
				\frac{j(j+1) - \ell(\ell+1) - s(s+1)}{2\ell (\ell+1) (2\ell+1)} - \frac{1}{2\ell + 1}
				&= \frac{1}{2\ell + 1} \left[ \frac{\pm \tfrac{1}{2} (2\ell + 1) - \tfrac{1}{2}}{2\ell (\ell + 1)} - 1 \right] \\
				(j = \ell + \tfrac{1}{2}) &= \frac{\ell - 2\ell^2 - 2\ell}{2\ell (\ell + 1) (2\ell + 1)} = - \frac{1}{2 (\ell + 1)} = - \frac{1}{2j + 1} \\
				(j = \ell - \tfrac{1}{2}) &= - \frac{2\ell^2 + 3\ell + 1}{2\ell (\ell + 1) (2\ell + 1)} = - \frac{1}{2\ell} = - \frac{1}{2j + 1}
			\end{split}
		\end{equation*}
		da cui la tesi.
	\end{proof}
\end{proposition}

Data questa correzione relativistica, gli autovalori d'energia all'ordine $ \alpha^2 $ sono (Eqq. \ref{eq:1-e-en}-\ref{eq:rel-corr-spectr}):
\begin{equation}
	E_{n,j} = - \frac{E_\text{Ha}}{2} \frac{\mu}{m_e} \frac{Z^2}{n^2} \left[ 1 + \left( Z \alpha \frac{\mu}{m_e} \right)^2 \frac{1}{n} \left( \frac{2}{2j + 1} - \frac{3}{4n} \right) \right]
\end{equation}
Si noti che, fissati $ n $ e $ j $, si ritrova la degenerazione accidentale in $ \ell $: risolvendo l'equazione di Dirac per un elettrone in un atomo idrogenoide, esatta a tutti gli ordini di $ \alpha $, si conferma la presenza di questa degenerazione, caratteristica del potenziale coulombiano ($ \ch{^2S}_{1/2} $ e $ \ch{^2P}_{1/2} $ sono degeneri).

\subsection{Lamb shift}

Come si può vedere in Fig. \ref{img:lamb}, in realtà la degenerazione accidentale in $ \ell $ viene rotta da fattori esterni: i principali sono la dimensione finita del nucleo atomico e le fluttuazioni di punto-zero del campo elettromagnetico. In ogni caso, lo splitting così determinato, denominato \textit{Lamb shift}, è estremamente piccolo e difficile da rilevare: seguendo l'esempio in Fig. \ref{img:lamb}, si vede che lo splitting tra $ \ch{^2S}_{1/2} $ e $ \ch{^2P}_{1/2} $ è dell'ordine di $ \sim 10\,\mu\text{eV} $.

\begin{figure}
	\centering
	\includegraphics[width = 0.70 \textwidth]{lamb-shift.png}
	\caption{Lamb shift, both theoretical (b) and experimental (a), in the Balmer $ \ch{H}\alpha $ line.}
	\label{img:lamb}
\end{figure}

\section{Struttura iperfine}

Al pari degli elettroni, anche i nuclei hanno un momento angolare di spin $ \ve{I} $, e per molte specie nucleari esso è non-nullo (es.: per il protone $ i = \tfrac{1}{2} $). Per il momento magnetico nucleare sussiste la relazione:
\begin{equation}
	\bs{\mu}_n = g_n \mu_\text{N} \bs{i}
\end{equation}
dove $ \mu_\text{N} \equiv \frac{\hbar q_e}{2m_p} $ è il \textit{magnetone nucleare}. Il valore di $ g_n $ dipende dalla struttura interna del nucleo: ad esempio, per il protone $ g_n \simeq 5.58569 $.\\
Sebbene il campo magnetico nucleare, a parità di distanza, risulti soppresso di $ m_e / m_p \simeq 1 / 1836 $ rispetto a quello elettronico, attraverso questi i momenti magnetici nucleare ed elettronico interagiscono. Dato il fattore $ r^\ell $ in Eq. \ref{eq:1-e-radial}, elettroni con $ \ell > 0 $ hanno probabilità basse di trovarsi vicino al nucleo, dunque l'interazione con essi può essere ignorata; per quanto riguarda gli orbitali $ \ch{S} $, il campo magnetico elettronico è interamente generato dallo spin elettronico $ \ve{S} $ e, analogamente all'interazione spin-orbita, l'interazione tra spin elettronico e spin nucleare è descritta da:
\begin{equation*}
	\mathcal{H}_\text{s-n} = - C \bs{\mu}_n \cdot \bs{\mu}_e = C g_n g_s \mu_\text{N} \mu_\text{B} \bs{i} \cdot \bs{s}
	\qquad \qquad
	C = \frac{2}{3} \frac{1}{4\pi \epsilon_0 c^2} \abs{R_{n,0}(0)}^2 = \frac{2}{3\pi \epsilon_0 c^2} \left( \frac{Z}{an} \right)^3
\end{equation*}
Ricordando che per l'elettrone $ g_s = 2 $, la coupling energy caratteristica risulta essere:
\begin{equation}
	\xi_\text{N} = \frac{4}{3} g_n \frac{Z^3 \alpha^2}{n^3} \frac{m_e}{m_n} E_\text{Ha} \simeq g_n \frac{Z^3}{n^3} \times 1.05\,\mu\text{eV}
\end{equation}
Secondo le regole del momento angolare, $ \ve{I} $ ed $ \ve{S} $ si accoppiano in un \textit{momento angolare atomico} $ \ve{F} = \ve{I} + \ve{S} $, così che per il fattore $ \bs{i} \cdot \bs{s} $ valga, nella coupled basis ($ F^2 $ ed $ F_z $ diagonali), una relazione analoga ad Eq. \ref{eq:sl-coupled-basis}. Ricordando che $ s = \tfrac{1}{2} $, l'interazione tra momenti magnetici nucleare ed elettronico determina uno splitting tra i livelli iperfini $ f = i \pm \tfrac{1}{2} $ pari energeticamente a $ \xi_\text{N} (i \pm \tfrac{1}{2}) $.

\begin{example}{Riga $ 21\,\text{cm} $ dell'idrogeno}{}
	Per $ \ch{^1H} $ si ha $ i = \tfrac{1}{2} $, quindi $ \braket{\bs{i} \cdot \bs{s}} = - \tfrac{3}{4}, \tfrac{1}{4} $ per $ f = 0,1 $ rispettivamente. Nel ground state $ n = 1 $, dunque i due stati iperfini $ f = 0 $ ed $ f = 1 $ risulteranno separati di $ \xi_\text{N} \simeq 5.88\,\mu\text{eV} $: la transizione tra questi livelli sarebbe proibità, poiché $ \Delta \ell = 0 $, ma ha una probabilità non-nulla di avvenire tramite emissione di un fotone con $ \lambda = 2\pi \hbar c \xi_\text{N} \simeq 21\,\text{cm} $ e frequenza $ \nu = \xi_\text{N} (2\pi \hbar)^{-1} \simeq 1.42\,\text{GHz} $. Questa riga spettrale ha valenza storica, in quanto la lunga vita media della transizione (poiché proibita) ne ha permesso la misura della frequenza molto precisa, tant'è che per un periodo è stata usata come standard per l'unità di tempo.
\end{example}

\begin{example}{Struttura iperfine del cesio-137}{}
	Il $ \ch{^{137}Cs} $ è un nuclide con tutte le shell complete ed un elettrone ottico esterno, dunque può essere trattato come un atomo idrogenoide con $ i = \tfrac{7}{2} $ (nuclide dispari-pari). Di conseguenza, per il momento angolare atomico sono possibili solo due valori, dato che $ 3 \le f \le 4 $: la transizioni tra di essi produce un fotone di frequenza $ \nu = 9.192631770 \,\text{GHz} $, e questa riga spettrale e al momento utilizzata come standard per l'unità di tempo.
\end{example}

\section{Transizioni elettroniche}

\subsection{Decadimento spontaneo}

Quando un atomo viene eccitato, esso tenderà a decadere spontaneamente: sperimentalmente, si osserva che non tutte le transizioni procedono allo stesso rate, e ciò può essere spiegato da un'analisi quanto-meccanica dell'interazione del sistema col campo elettromagnetico ambientale. Nell'\textit{approssimazione di dipolo elettrico}\footnotemark, si trova che la probabilità di decadimento radiativo nell'unità di tempo da uno stato iniziale $ \ket{\text{i}} $ ad uno finale $ \ket{\text{f}} $ è:
\begin{equation}
	\gamma_\text{if} = \frac{1}{3\pi \epsilon_0 \hbar^4 c^3} \mathcal{E}_\text{if}^3 \abs{\braket{\text{f} | \ve{d} | \text{i}}}^2
	\label{eq:electron-trans-prob}
\end{equation}
con $ \mathcal{E}_\text{if} \equiv \hbar \omega_\text{if} \defeq E_\text{i} - E_\text{f} $ e $ \ve{d} \equiv - q_e \ve{r} $ l'operatore dipolo elettrico. La dipendenda dall'elemento di matrice $ \braket{\text{f} | \ve{d} | \text{i}} $ impone delle selection rules: le transizioni proibite avvengono con probabilità estremamente inferiori, dato che sono associate a termini di ordine superiore (in $ \alpha $) nell'espansione in multipolo (dipolo magnetico, quadrupolo elettrico, ...).

\footnotetext{Approssimazione valida nel caso in cui la sorgente della radiazione abbia dimensioni molto minori della lunghezza d'onda della radiazione emessa. Essendo $ k = \frac{2\pi}{\lambda} $ e $ \lambda = \frac{hc}{E} $, si ha $ k = \frac{E}{\hbar c} $, dunque tale approssimazione è valida, per $ E \sim 10 - 30 \ev $, se $ k \sim (100 \ang)^{-1} $. Inoltre, si noti che:
\begin{equation}
	\abs{\braket{\text{f} | \ve{p} | \text{i}}}^2 = \abs{\braket{\text{f} | \frac{m}{i\hbar} [\ve{r} , \mathcal{H}] | \text{i}}}^2 = \frac{m^2}{\hbar^2} (E_\text{i} - E_\text{f})^2 \abs{\braket{\text{f} | \ve{r} | \text{i}}}^2 = \frac{m^2}{\hbar^2 q_e^2} \mathcal{E}_\text{if}^2 \abs{\braket{\text{f} | \ve{d} | \text{i}}}^2
	\label{eq:electric-dipole-approx}
\end{equation}}

\begin{theorem}{Electric-dipole selection rule}{}
	Nell'approssimazione di dipolo elettrico per un atomo idrogenoide, le transizioni spontanee ammesse soddisfano:
	\begin{equation}
		\Delta \ell = \pm 1
	\end{equation}

	\tcblower
	
	\begin{proof}
		Esplicitando l'elemento di matrice di $ \ve{d} \equiv -q_e \ve{r} $ in funzione di quello di $ \ve{r} $:
		\begin{equation*}
			\abs{\braket{\text{f} | \ve{r} | \text{i}}}^2 = \abs{\braket{\text{f} | r_x | \text{i}}}^2 + \abs{\braket{\text{f} | r_z | \text{i}}}^2 + \abs{\braket{\text{f} | r_z | \text{i}}}^2
		\end{equation*}
		In funzione delle armoniche sferiche:
		\begin{align*}
			r_x &= r \sin \vartheta \cos \varphi = r \sqrt{\frac{2\pi}{3}} \left[ Y_{1,-1}(\vartheta, \varphi) - Y_{1,1}(\vartheta, \varphi) \right] \\
			r_y &= r \sin \vartheta \sin \varphi = r i \sqrt{\frac{2\pi}{3}} \left[ Y_{1,-1}(\vartheta, \varphi) + Y_{1,1}(\vartheta, \varphi) \right] \\
			r_z &= r \sqrt{\frac{4\pi}{3}} Y_{1,0}(\vartheta, \varphi)
		\end{align*}
		Ne segue che:
		\begin{equation*}
			\abs{\braket{\text{f} | \ve{r} | \text{i}}}^2 = \abs{\braket{\text{f} | r | \text{i}}}^2 \frac{4\pi}{3} \left[ \abs{\braket{\text{f} | Y_{1,-1} | \text{i}}}^2 + \abs{\braket{\text{f} | Y_{1,0} | \text{i}}}^2 + \abs{\braket{\text{f} | Y_{1,1} | \text{i}}}^2 \right]
		\end{equation*}
		Esplicitando in termini della funzione d'onda elettronica:
		\begin{equation*}
			\begin{split}
				\abs{\braket{n_\text{f} , \ell_\text{f} , m_\text{f} | \ve{d} | n_\text{i} , \ell_\text{i} , m_\text{i}}}^2
				& = q_e^2 \abs{\int_0^\infty \dd r\, r^2 R_{n_\text{f} , \ell_\text{f}}(r) r R_{n_\text{i} , \ell_\text{i}}(r)}^2 \times \\
				& \times \frac{4\pi}{3} \sum_{m = -1,0,1} \abs{\int_0^\pi \dd \vartheta \, \sin \vartheta \int_0^{2\pi} \dd \varphi \, Y_{\ell_\text{f} , m_\text{f}}^*(\vartheta,\varphi) Y_{1,m}(\vartheta,\varphi) Y_{\ell_\text{i} , m_\text{i}}(\vartheta,\varphi)}^2
			\end{split}
		\end{equation*}
		L'integrale spaziale contribuisce soltanto allo smorzamento delle transizioni con $ \abs{n_\text{i} - n_\text{f}} $ grande, mentre la selection rule è imposta dall'integrale angolare: si noti che esso rappresenza l'overlap angolare tra $ Y_{\ell_\text{f} , m_\text{f}} $ e $ Y_{1,m} Y_{\ell_\text{i} , m_\text{i}} $, ma quest'ultimo può essere scomposto (secondo la decomposizione di Clebsch-Gordan) nella somma di stati con $ \ell = \abs{\ell_\text{i} - 1}, \ell_\text{i}, \ell_\text{i} + 1 $, dunque l'elemento di matrice è non-nullo sono per $ \ell_\text{f} = \ell $. Inoltre, si noti che l'integrale si annulla per $ \ell_\text{f} = \ell_\text{i} $: in tal caso, la parità dell'integranda sarebbe $ (-1)^{\ell_\text{i}} (-1)^1 (-1)^{\ell_\text{i}} = -1 $, dunque il suo integrale su tutto l'angolo solido è nullo. Gli unici valori possibili sono dunque $ \ell_\text{f} = \ell_\text{i} \pm 1 $, ovvero la selection rule cercata.
	\end{proof}
\end{theorem}

Essendo l'operatore dipolo elettrico associato a $ \ell = 1 $, si ha che $ m = -1,0,1 $; di conseguenza, si ottiene una selection rule anche sulla proiezione del momento angolare orbitale:
\begin{equation}
	\Delta m = 0, \pm 1
	\label{eq:1-e-el-dip-tr-m}
\end{equation}
Inoltre, dato che $ \ve{d} $ agisce come l'identità sullo spazio degli spin, si hanno:
\begin{equation}
	\Delta s = 0
	\qquad \qquad
	\Delta m_s = 0
	\label{eq:1-e-el-dip-tr-spin}
\end{equation}
È anche possibile esprimere la selection rule sulla base accoppiata, ottenendo:
\begin{equation}
	\Delta j = 0 , \pm 1
	\qquad \qquad
	\Delta m_j = 0 , \pm 1
\end{equation}
Una volta determinate le transizioni possibili, si può dare una stima del decay rate dall'Eq. \ref{eq:electron-trans-prob}, usando $ \mathcal{E}_\text{if} \simeq Z^2 E_\text{Ha} $ e $ \abs{\braket{\text{f} | \ve{d} | \text{i}}} = q_e a_0 / Z $:
\begin{equation}
	\gamma_\text{if} = \frac{\mathcal{E}_\text{if}^3 \abs{\braket{\text{f} | \ve{d} | \text{i}}}^2}{3\pi \epsilon_0 \hbar^4 c^3} \simeq \frac{Z^4 E_\text{Ha}^2}{\epsilon_0 \hbar^4 c^3} \hbar \omega_\text{if} \frac{q_e^2 a_0^2}{Z^2} \simeq \frac{e^2 Z^2}{(\hbar c)^3} e^4 \omega_\text{if} = Z^2 \alpha^3 \omega_\text{if}
	\label{eq:electric-dipole-decay-rate}
\end{equation}
dato che $ E_\text{Ha} a_0 = e^2 $ e $ \alpha = e^2 / (\hbar c) $. Per un atomo idrogenoide $ \omega_\text{if} \simeq Z^2 \cdot 10^{16} \,\text{Hz} $, dunque si trova un tempo di decadimento dell'ordine $ \gamma_\text{if}^{-1} \simeq Z^{-4} \,\text{ns} $.

\begin{example}{Doppietto giallo del sodio}{}
	L'atomo di sodio presenta 11 elettroni: 10 interni in shell complete ed 1 esterno in 3s. Come in tutti gli altri atomi alcalini, l'elettrone esterno si trova più lontano dal nucleo rispetto alle shell interne ed è chiamato \textit{elettrone ottico}, poiché tipicamente le sue transizioni cadono nel visibile. Il potenziale a cui è soggetto l'elettrone ottico può essere visto come Coulombiano con un $ Z_\text{eff} $ efficacie: a grandi distanze dal nucleo $ Z_\text{eff} \simeq Z - (Z-1) = 1 $, mentre avvicinandosi ad esso $ Z_\text{eff} = Z_\text{eff}(r) $, il che rompe la degenerazione accidentale (l'energia è $ E = E(n,\ell) $). \\
	Nel caso del sodio, il tipico doppietto giallo del suo spettro d'emissione è dovuto alla transizione ottica ($ \Delta \ell = \pm 1 $) più piccola possibile per l'elettrone ottico, ovvero $ \text{3p} \rightarrow \text{3s} $ ($ \Delta E \sim 2\ev $). In particolare, sono presenti due righe a causa dell'interazione spin-orbita ($ \Delta E \sim 2.1 \,\text{meV} $), poiché le transizioni possibili sono due: $ \ch{^3P}_{3/2} \rightarrow \ch{^3S}_{1/2} $ ($ \lambda = 589.0 \,\text{nm} $) e $ \ch{^3P}_{1/2} \rightarrow \ch{^3S}_{1/2} $ ($ \lambda = 589.6 \,\text{nm} $).
\end{example}

\subsection{Decadimento stimolato}

In generale, un sistema immerso in un campo elettromagnetico descritto dai potenziali $ \phi $ ed $ \ve{A} $ è descritto dall'Hamiltoniana:
\begin{equation}
	\mathcal{H} = \frac{1}{2m} \left( \ve{p} - \frac{q}{c} \ve{A} \right)^2 + q \phi
\end{equation}
Nel caso dell'atomo idrogenoide:
\begin{equation}
	\mathcal{H} = \frac{1}{2m_e} \left( \ve{p} + \frac{e}{c} \ve{A} \right)^2 + \frac{1}{2M} \left( \ve{P} - \frac{Ze}{c} \ve{A} \right)^2 - \frac{Ze}{\abs{\ve{r} - \ve{R}}}
\end{equation}
Trascurando $ \frac{1}{M} \ll \frac{1}{m_e} $, ponendo $ \ve{R} = \ve{0} $ e scegliendo un gauge $ [\nabla , \ve{A}] = 0 $:
\begin{equation*}
	\begin{split}
		\mathcal{H}
		& \simeq \frac{1}{2m_e} \left( \ve{p} + \frac{e}{c} \ve{A} \right)^2 - \frac{Ze}{r} = \frac{1}{2m_e} \left[ -\hbar^2 \lap - i \frac{\hbar e}{c} (\nabla \cdot \ve{A} + \ve{A} \cdot \nabla) + \frac{e^2}{c^2} \ve{A}^2 \right] - \frac{Ze}{r} \\
		& = \frac{\hbar^2}{2m_e} \left[ -\lap - 2i \frac{\alpha}{e} \ve{A} \cdot \nabla + \frac{\alpha^2}{e^2} \ve{A}^2 \right] - \frac{Ze}{r} = \underbrace{- \frac{\hbar^2}{2m_e} \lap - \frac{Ze}{r}}_{\mathcal{H}_0} \underbrace{- i \frac{\hbar^2}{m_e} \frac{\alpha}{e} \ve{A} \cdot \nabla}_{V} + \underbrace{\frac{\hbar^2}{2m_e} \frac{\alpha^2}{e^2} \ve{A}^2}_{\sim \, \alpha^2}
	\end{split}
\end{equation*}
L'ultimo termine può essere trascurato, dunque il problema può essere trattato perturbativamente al prim'ordine (nell'approssimazione di campo debole). Per semplificare la trattazione, si adottano le unità atomiche $ \hbar = m_e = e = 1 , c = \alpha^{-1} \simeq 137 $:
\begin{equation}
	\mathcal{H} \simeq - \frac{1}{2} \lap - \frac{Z}{r} - i \alpha \ve{A} \cdot \nabla
\end{equation}
La dipendenza temporale degli autostati di $ \mathcal{H} $ non è più banale, in quanto la perturbazione determina una dipendenza temporale:
\begin{equation}
	\psi(t,r) = \sum_{n \in \N} c_n(t) e^{-i E_n t} \psi_n(r)
	\label{eq:1-e-pert-autof}
\end{equation}
dove $ \psi_n(r) $ sono gli stati stazionari di $ \mathcal{H}_0 $ e al prim'ordine:
\begin{equation}
	c_n(t) \simeq c_n^{(0)} + c_n^{(1)}(t)
	\label{eq:1-e-pert-coeff}
\end{equation}

\begin{proposition}{Coefficienti perturbativi}{}
	Per un potenziale del tipo:
	\begin{equation}
		\ve{A}(t,\ve{r}) = \ve{A}_0 e^{i (\ve{k}_0 \cdot \ve{r} - \omega_0 t + \varphi_0)} + \text{c.c.} = 2 \ve{A}_0 \cos (\ve{k}_0 \cdot \ve{r} - \omega_0 t + \varphi_0)
	\end{equation}
	i coefficienti nello sviluppo perturbativo per una transizione tra due stati stazionari $ \ket{n} , \ket{m} $ sono:
	\begin{equation}
		\begin{split}
			c_n^{(1)}(t) = - \alpha t \ve{A}_0 \cdot \bigg[
			& \ve{M}_{nm}(\ve{k}_0) \exp \left( i \frac{\omega_{nm} - \omega_0}{2} t + i \varphi_0 \right) \sinc \left( \frac{\omega_{nm} - \omega_0}{2} t \right) + \\
			& \qquad - \ve{M}_{mn}^*(\ve{k}_0) \exp \left( i \frac{\omega_{nm} + \omega_0}{2} t + i \varphi_0 \right) \sinc \left( \frac{\omega_{nm} + \omega_0}{2} t \right) \bigg]
			\label{eq:1-e-big-coeff}
		\end{split}
	\end{equation}
	con $ \ve{M}_{nm}(\ve{k}_0) \equiv \braket{n | e^{i \ve{k}_0 \cdot \ve{r}} \nabla | m} $.

	\tcblower

	\begin{proof}
		L'equazione di Schrödinger per l'autofunzione \ref{eq:1-e-pert-autof} diventa:
		\begin{equation*}
			\sum_{m \in \N} c_m(t) e^{-i E_m t} \mathcal{H} \psi_m(r) = i \sum_{m \in \N} \left[ \frac{dc_m(t)}{dt} - i E_m c_m(t) \right] e^{-i E_m t} \psi_m(r)
		\end{equation*}
		Moltiplicando a sinistra per $ \psi_n^*(r) $ ed integrando su tutto lo spazio:
		\begin{equation*}
			\sum_{m \in \N} c_m(t) e^{-i E_m t} \left( \delta_{nm} + \braket{n | V | m} \right) = \sum_{m \in \N} \left[ i \frac{dc_m(t)}{dt} + E_m c_m(t) \right] e^{-i E_m t} \delta_{nm}
		\end{equation*}
		Semplificando, si trova:
		\begin{equation*}
			\frac{dc_n(t)}{dt} = -i \sum_{m \in \N} c_m(t) e^{i \mathcal{E}_{nm} t} \braket{n | V | m}
		\end{equation*}
		Dall'Eq. \ref{eq:1-e-pert-coeff}, dato che $ c_n^{(1)} \ll c_n^{(0)} $:
		\begin{equation*}
			\frac{dc_n^{(1)}(t)}{dt} = -i \sum_{k \in \N} c_k^{(0)}(t) e^{i \mathcal{E}_{nk} t} \braket{n | V | k}
		\end{equation*}
		Per uno stato iniziale stazionario $ \ket{m} $ si ha $ c_k^{(0)} = \delta_{km} $, dunque:
		\begin{equation*}
			\frac{dc_n^{(1)}(t)}{dt} = -i e^{i \mathcal{E}_{nm} t} \braket{n | V | m} = - e^{i \mathcal{E}_{nm} t} \alpha \ve{A}_0 \cdot \left[ \braket{n | e^{i\ve{k}_0 \cdot \ve{r}} \nabla | m} e^{-i \omega_0 t + \varphi_0} + \text{c.c.} \right]
		\end{equation*}
		Essendo $ \nabla $ anti-hermitiano, $ \ve{M}_{nm}^*(\ve{k}_0) = - \ve{M}_{mn}(\ve{k}_0) $, ovvero:
		\begin{equation*}
			\frac{dc_n^{(1)}(t)}{dt} = - \alpha \ve{A}_0 \cdot \left[ \ve{M}_{nm}(\ve{k}_0) e^{i (\omega_{nm} - \omega_0) t + i \varphi_0} - \ve{M}_{mn}^*(\ve{k}_0) e^{i (\omega_{nm} + \omega_0) t - i \varphi_0} \right]
		\end{equation*}
		Integrando:
		\begin{equation*}
			\int_0^t dt' e^{i \beta t'} = \frac{e^{i\beta t} - 1}{i \beta} = \frac{i (1 - \cos \beta t) + \sin \beta t}{\beta} = \frac{2}{\beta} e^{i \beta \frac{t}{2}} \sin \left( \beta \tfrac{t}{2} \right) = \frac{t e^{i\beta \frac{t}{2}} \sin \left( \beta \frac{t}{2} \right)}{\beta \frac{t}{2}}
		\end{equation*}
		Utilizzando questo risultato per risolvere l'ODE per $ c_n^{(1)}(t) $ si ottiene la tesi.
	\end{proof}
\end{proposition}

Si ottiene quindi che, per un sistema inizialmente in uno stato stazionario $ \ket{m} $:
\begin{equation}
	\psi(t,r) = \psi_m(r) + \sum_{n \in \N} c_{nm}(t) e^{- i E_n t} \psi_n(r)
\end{equation}
con $ c_{nm}(t) \equiv c_n^{(1)}(t) $ come in Eq. \ref{eq:1-e-big-coeff}. La probabilità di transizione da $ \ket{m} $ a $ \ket{n} $, con $ n \neq m $, sarà dunque $ P_{nm}(t) = \abs{c_{nm}(t)}^2 $.

\paragraph{Risonanza}

Essendo $ \omega_0 $ la pulsazione della radiazione incidente, in base al valore di $ \omega_{nm} $ (differenza tra i due livelli energetici) si hanno due possibili risonanze (a seconda di quale termine domina in Eq. \ref{eq:1-e-big-coeff}):
\begin{itemize}
	\item $ \omega_{nm} \approx \omega_0 $: domina il primo termine, ovvero si ha risonanza per il processo di assorbimento (eccitamento) con transizione da $ \omega_m $ a $ \omega_n = \omega_m + \omega_0 $;
	\item $ \omega_{nm} \approx - \omega_0 $: domina il secondo termine, ovvero si ha risonanza per il processo di emissione (decadimento) con transizione da $ \omega_m $ a $ \omega_n = \omega_m - \omega_0 $.
\end{itemize}
Inoltre, si noti che $ \sinc{x} \rightarrow 0 $ per $ x \rightarrow \infty $, dunque col passare del tempo le probabilità di transizione per $ n \neq m $ tendono ad annullarsi; ciò è legato alla natura della radiazione emessa: inizialmente essa è non-monocromatica, ma col passare del tempo tende a diventare sempre più monocromatica.

\paragraph{Golden rule di Fermi}

È possibile definire la probabilità di transizione per unità di tempo sia nel caso dell'assorbimento che in quello dell'emissione:
\begin{align*}
	P_\text{a}(t) &= \abs{c_{nm}(t)}^2 \simeq \alpha^2 t^2 A_0^2 \abs{\bs{\epsilon} \cdot \ve{M}_{nm}(\ve{k}_0)}^2 \sinc^2 \left( \tfrac{\omega_{nm} - \omega_0}{2} t \right) \sim 2\pi \alpha^2 t A_0^2 \abs{\bs{\epsilon} \cdot \ve{M}_{nm}(\ve{k}_0)}^2 \delta(\tfrac{\omega_{nm} - \omega_0}{2}) \\
	P_\text{e}(t) &= \abs{c_{nm}(t)}^2 \simeq \alpha^2 t^2 A_0^2 \abs{\bs{\epsilon} \cdot \ve{M}_{mn}^*(\ve{k}_0)}^2 \sinc^2 \left( \tfrac{\omega_{nm} + \omega_0}{2} t \right) \sim 2\pi \alpha^2 t A_0^2 \abs{\bs{\epsilon} \cdot \ve{M}_{mn}^*(\ve{k}_0)}^2 \delta(\tfrac{\omega_{nm} + \omega_0}{2})
\end{align*}
dove $ \bs{\epsilon} = \ve{A}_0 / A_0 $ è la polarizzazione della radiazione incidente e $ \delta(x) = \frac{1}{2\pi} \lim_{t \rightarrow \infty} t \sinc^2(xt) $. Definendo l'intensità della radiazione incidente come $ I_0 \equiv A_0^2 \omega_0^2 / (2\pi c) $, si ottiene la golden rule di Fermi:
\begin{align*}
	P_\text{a}(t) &\simeq 4\pi \alpha^2 t \abs{\bs{\epsilon} \cdot \ve{M}_{nm}(\ve{k}_0)}^2 \frac{I_0}{\omega_0^2} \delta(\omega_{nm} - \omega_0) \\
	P_\text{e}(t) &\simeq 4\pi \alpha^2 t \abs{\bs{\epsilon} \cdot \ve{M}_{mn}^*(\ve{k}_0)}^2 \frac{I_0}{\omega_0^2} \delta(\omega_{nm} + \omega_0)
\end{align*}
Si noti che l'unico termine distinto è la $ \delta $ di conservazione dell'energia, la quale può essere scritta in maniera unificata come $ \delta(\omega_\text{f} - \omega_\text{i}) $. Inoltre, si vede che la probabilità di transizione diverge alla risonanza: questo però non è un problema, in quanto non esistono sorgenti puramente monocromatiche, in quanto dovrebbero emettere per un tempo infinito; più realisitcamente, si ha una distribuzione d'intensità attorno a $ \omega_0 $, così da scrivere:
\begin{equation*}
	\frac{I_0}{\omega_0^2} \delta(\omega_{nm} \pm \omega_0) \longrightarrow \int_{-\infty}^{+\infty} d\omega\, \frac{I(\omega)}{\omega^2} \delta(\omega_{nm} \pm \omega_0) = \frac{I(\omega_{nm})}{\omega_{nm}^2}
\end{equation*}
Utilizzando l'approssimazione di dipolo elettrico Eq. \ref{eq:electric-dipole-approx}, si trova che il rate di transizione $ \gamma_\text{if} \equiv P_\text{if} / t $ può essere scritto come:
\begin{equation*}
	\gamma_\text{if} = 4\pi \alpha^2 \abs{\braket{\text{f} | e^{i \ve{k}_0 \cdot \ve{r}} \nabla | \text{i}}}^2 \frac{I(\omega_\text{if})}{\omega_\text{if}^2} \simeq 4\pi \alpha^2 \abs{\braket{\text{f} | \ve{p} | \text{i}}}^2 \frac{I(\omega_\text{if})}{\omega_\text{if}^2} = 4\pi \alpha^2 \omega_\text{if}^2 \abs{\braket{\text{f} | \ve{d} | \text{i}}}^2 \frac{I(\omega_\text{if})}{\omega_\text{if}^2}
\end{equation*}
Si trova dunque:
\begin{equation}
	\gamma_\text{if} = 4\pi \alpha^2 I(\omega_\text{if}) \abs{\braket{\text{f} | \ve{d} | \text{i}}}^2
\end{equation}
A differenza dell'emissione spontanea, che è $ \sim \mathcal{E}_\text{if}^3 $, quella stimolata è $ \sim \mathcal{E}_\text{if}^2 $, dunque l'emissione spontanea diventa più probabile di quella stimolata all'aumentare della differenza energetica.

\section{Campo magnetico esterno}

Maggiori informazioni su una specie atomica possono essere ricavate immergendo il campione in un campo magnetico uniforme e studiandone lo spettro. Il momento magnetico atomico totale può essere scritto come:
\begin{equation}
	\bs{\mu} = \bs{\mu}_\ell + \bs{\mu}_s = - \mu_\text{B} ( g_\ell \bs{\ell} + g_s \bs{s}) \simeq - \mu_\text{B} (\bs{\ell} + 2 \bs{s})
\end{equation}
L'Hamiltoniana di coupling con un campo magnetico esterno, WLOG $ \ve{B} = B \hat{\ve{e}}_z $, è:
\begin{equation}
	\mathcal{H}_\text{magn} = - \bs{\mu} \cdot \ve{B} = \mu_\text{B} B (\ell_z + 2s_z)
\end{equation}
Si vede che $ \mathcal{H}_\text{magn} $ è diagonalizzabile nella uncoupled basis $ \ket{\ell, m, s, m_s} $, mentre $ \mathcal{H}_\text{s-o} $ lo è nella coupled basis $ \ket{\ell, s, j, m_j} $: dato che $ [\mathcal{H}_\text{magn} , \mathcal{H}_\text{s-o}] \neq 0 $, essi non sono simultaneamente diagonalizzabili, ma vanno diagonalizzati di volta in volta in ogni sottospazio $ (2\ell + 1)(2s + 1) $-dimensionale a $ n,\ell,s $ fissati.
È però possibile studiare i casi limite per le energie caratteristiche $ \mu_\text{B} B $ e $ \xi $.

\subsection{Limite Paschen-Back}

Nel limite $ \mu_\text{B} B \gg \xi $ in cui domina il campo magnetico esterno, la diagonalizzazione è diretta poiché si usa la uncoupled basis:
\begin{equation}
	\Delta E_\text{magn}(m,m_s) = \braket{m,m_s | \mathcal{H}_\text{magn} | m,m_s} = \mu_\text{B} B (m + 2m_s)
\end{equation}
mentre la correzione dovuta a $ \mathcal{H}_\text{s-o} $ può essere trattata perturbativamente. \\
Il valore di $ B $ per cui si può effettuare questa approssimazione dipende dall'atomo considerato e dal livello energetico: ad esempio, considerando la shell $ \text{2p} $, per $ \ch{H} $ si ha $ B \gg 0.5 \,\text{T} $, mentre per $ \ch{He}^+ $ si ha $ B \gg 8 \,\text{T} $, a causa della dipendenza da $ Z^4 $ in Eq. \ref{eq:1-e-int-spin-orb}.

\subsection{Limite Zeeman}

Il limite $ \mu_\text{B} B \ll \xi $ in cui domina l'interazione spin-orbita è quello più comune e in esso la simmetria sferica subisce solo una debole perturbazione. Gli stati $ \ket{\ell,s,j,m_j} $ nella coupled basis sono dunque autostati approssimati di $ \mathcal{H}_\text{s-o} + \mathcal{H}_\text{magn} $, mentre la correzione al prim'ordine dell'energia è:
\begin{equation}
	\Delta E_\text{magn}(j,m_j) \simeq \braket{j,m_j | \mathcal{H}_\text{magn} | j,m_j} = g_j \mu_\text{B} B m_j
\end{equation}
dove $ g_j $ è il fattore di Landé (Eq. \ref{eq:lande-g-factor}).

\subsection{Linee spettrali}

Sperimentalmente, si confermano le considerazioni teoriche sia esatte che approssimate. Prendendo ad esempio un campione di $ \ch{H} $, applicando un campo magnetico sufficientemente potente si osserva una triplicazione delle linee spettrali ($ \Delta m = 0, \pm 1 $), in accordo con l'effetto Paschen-Back, mentre applicando un campo magnetico debole si osserva l'effetto Zeeman (Fig. \ref{zeeman-effect}).

\begin{figure}
	\centering
	\includegraphics[width = 0.85 \textwidth]{zeeman-effect.png}
	\caption{Zeeman split of the lowest Lyman line of $ \ch{H} $.}
	\label{zeeman-effect}
\end{figure}

Nel regime intermedio $ \mu_\text{B} B \sim \xi $, nessuna delle due basi riesce a dare una descrizione accurata dei livelli energetici. Ad esempio, Fig. \ref{mag-field-int} mostra lo splitting pattern dei 6 stati $ \ch{^2P} $ in funzione dell'intensità del campo magnetico.

\begin{figure}[!b]
	\centering
	\includegraphics[width = 0.50 \textwidth]{mag-field-int.png}
	\caption{Combined spin-orbit and magnetic splittings of $ \ch{^2P} $.}
	\label{mag-field-int}
\end{figure}












\chapter{Atomi a Più Elettroni}
\selectlanguage{italian}

La trattazione degli atomi a molti elettroni (Many-Electron Atoms) è resa non banale dall'interazione elettrone-elettrone, la quale rende impossibile la risoluzione esatta del problema\footnotemark. È dunque necessario adottare alcune semplificazioni.

\footnotetext{La funzione d'onda di ciascun elettrone dipende da tre variabili spaziali, dunque discretizzando ciascuna di esse in una griglia di 10 numeri reali, la descrizione di un sistema a $ N $ elettroni richiede di trattare $ (10^3)^N $ numeri reali: già per $ N = 4 $ ciò necessiterebbe di qualche Tb, mentre per $ N = 27 $ si raggiunge l'ordine di grandezza del numero totale di atomi nell'Universo. La trattazione numerica del problema è dunque impossibile.}

\section{Approssimazione a particelle indipendenti}

È possibile trovare soluzioni approssimate per MEAs costruendo il relativo spazio di Hilber a partire dagli stati single-particle.

\subsection{Particelle identiche}

Gli elettroni sono particelle indistinguibili tra loro, dunque è necessario ricordare le proprietà dei sistemi quantistici di particelle identiche. \\
Dato un sistema di $ N $ particelle identiche, ciascuna descritta da uno spazio di Hilbert $ \hilb $, il sistema totale sarà descritto da un sottospazio del prodotto diretto di tali spazi. In particolare, definendo l'operatore di scampio $ \pi_{ij} $ che scambia le particelle $ i \leftrightarrow j $, dato che $ \pi_{ij}^2 $ si ha che i suoi autovalori possibili sono $ \pm 1 $: stati $ \pi_{ij} \ket{\psi} = + \ket{\psi} $ sono detti \textit{stati bosonici}, sono simmetrici per scambio di particelle e descrivono sistemi di spin intero; stati $ \pi_{ij} \ket{\psi} = - \ket{\psi} $ sono detti \textit{stati fermionici}, sono antisimmetrici per scambio di particelle e descrivono sistemi di spin semi-intero. La definizione degli stati bosonici/fermionici a partire dagli stati single-particle è dunque data rispettivamente da:
\begin{equation}
	\ket{\alpha_1 , \dots , \alpha_N}^\text{(s)} \defeq \frac{1}{\sqrt{N!}} \sum_{\pi \in S^N} \ket{\alpha_{\pi(1)} , \dots , \alpha_{\pi(N)}}
\end{equation}
\begin{equation}
	\ket{\alpha_1 , \dots , \alpha_N}^\text{(a)} \defeq \frac{1}{\sqrt{N!}} \sum_{\pi \in S^N} (-1)^{\{\pi\}} \ket{\alpha_{\pi(1)} , \dots , \alpha_{\pi(N)}}
\end{equation}
dove $ \{\pi\} $ è il carattere della permutazione e $ \alpha_n $ indica il set completo di quantum numbers dell'$ n $-esima particella. Come si può vedere, il principio d'esclusione di Pauli discende banalmente da queste definizioni: un sistema bosonico non ha restrizioni sui quantum numbers dei singoli bosoni che lo compongono, mentre un sistema fermionico, dato il fattore $ (-1)^{\{\pi\}} $, risulta avere $ \ket{\psi} = 0 $ se si considerano due fermioni con gli stessi quantum numbers.

\begin{example}{Identicità degli atomi}{}
	Si consideri un atomo di $ \ch{^3He} $: esso è composto da 2 protoni, 1 neutrone e 2 elettroni, dunque overall è un sistema fermionico: se si scambiano tra loro due tali atomi, si ottiene un fattore $ (-1)\cdot(-1) = +1 $ per i protoni, idem per gli elettroni, e $ (-1) $ per i neutroni, risultando in un fattore totale $ (-1) $. \\
	D'altro canto, un atomo di $ \ch{^{238}U} $, composto da 92 protoni, 146 neutroni e 92 elettroni, è un sistema bosonico per un ragionamento analogo. \\
	In generale, atomi con $ A + Z $ pari sono bosoni, mentre atomi con $ A + Z $ dispari sono fermioni.
\end{example}

L'antisimmetria per scambio dei sistemi a molti elettroni ne condiziona fortemente al dinamica, in quanto la repulsione tra elettroni data dall'antisimettrizzazione della funzione d'onda è spesso più efficace della repulsione elettromagnetica tra di essi: senza antisimmetria, gli elettroni nel ground state occuperebbero tutti la shell $ \text{1s} $.

\subsubsection{Funzione d'onda fermionica}

Si consideri un sistema di $ N $ fermioni, ciascuno descritto da uno spazio di Hilber con ket base $ \ket{w_n} = \ket{\ve{r}_n , \sigma_n} $ di posizione e spin.

\begin{theorem}{Determinante di Slater}{}
	La base dello spazio di Hilber di un sistema di $ N $ fermioni è data dal \textit{determinante di Slater}:
	\begin{equation}
		\Psi_{\alpha_1 , \dots , \alpha_N}(w_1 , \dots , w_N) = \frac{1}{\sqrt{N!}}
		\begin{vmatrix}
			\psi_{\alpha_1}(w_1) & \dots & \psi_{\alpha_1}(w_N) \\
			\vdots & \ddots & \vdots \\
			\psi_{\alpha_N}(w_1) & \dots & \psi_{\alpha_N}(w_N)
		\end{vmatrix}
	\end{equation}

	\tcblower

	\begin{proof}
		Essendo lo spazio di Hilbert totale $ \hilb^\text{(a)} $, la funzione d'onda dello stato fermionico generico $ \ket{\alpha_1 , \dots , \alpha_N}^\text{(a)} $ sarà:
		\begin{equation*}
			\begin{split}
				\Psi_{\alpha_1 , \dots , \alpha_N}(w_1 , \dots , w_N)
				& = {^\text{(a)}\langle}w_1 , \dots , w_N | \alpha_1 , \dots , \alpha_N{\rangle^\text{(a)}} \\
				& = \frac{1}{N!} \sum_{\pi,\rho \in S^N} (-1)^{\{\pi\}} (-1)^{\{\rho\}} \braket{w_{\rho(1)} , \dots , w_{\rho(N)} | \alpha_{\pi(1)} , \dots , \alpha_{\pi(N)}} \\
				& = \sum_{\pi \in S^N} (-1)^{\{\pi\}} \frac{1}{N!} \sum_{\rho \in S^N} (-1)^{\{\rho\}} \psi_{\alpha_{\pi(1)}}(w_{\rho(1)}) \dots \psi_{\alpha_{\pi(N)}}(w_{\rho(N)}) \\
				& = \sum_{\pi \in S^n} (-1)^{\{\pi\}} \psi_{\alpha_{\pi(1)}}(w_1) \dots \psi_{\alpha_{\pi(N)}}(w_N)
			\end{split}
		\end{equation*}
		Questa è proprio la definizione di determinante della matrice $ A_{ij} = \psi_{\alpha_i}(w_j) $. Aggiungendo un fattore di normalizzazione $ (N!)^{-1/2} $ si ottiene la tesi\footnote{Ciò permette di usare come dominio d'integrazione tutto lo spazio di definizione delle $ w_n $, e non solo l'iper-triangolo $ w_1 > \dots > w_n $.}.
	\end{proof}
\end{theorem}

In questo modo diventa possibile trattare numericamente il problema\footnotemark. La complessità del problema viene relegata ai coefficienti dell'espansione dello stato generico su tale base:
\begin{equation}
	\ket{\Psi} = \sum_{\alpha_1 , \dots , \alpha_N} c_{\alpha_1 , \dots , \alpha_N} \ket{\alpha_1 , \dots , \alpha_N}^\text{(a)}
\end{equation}
con $ c_{\alpha_1 , \dots , \alpha_N} \in \C $. In questo modo, dunque, si ottiene lo stato del sistema totale a partire dagli stati single-particle: da qui l'\textit{approssimazione a particelle indipendenti}.

\footnotetext{I ket di base ottenuti tramite il determinante di Slater contengono una quantità d'informazione che, seguengo l'esempio della nota precedente, scala come $ N \cdot 10^3 $, dunque linearmente.}

\subsection{Elettroni non-interagenti}

Si consideri il potenziale d'interazione elettrone-elettrone $ V_{ee} $ completamente trascurabile: in tal caso, il problema ad $ N $ elettroni si fattorizza completamente, poiché ciascun elettrone si muove indipendentemente dagli altri nel potenziale $ V_{ne} $. Trascurando gli effetti relativistici, gli autostati dei singoli elettroni sono rappresentati dalla funzione d'onda idrogenoide\footnotemark: si ha dunque $ \alpha_i \equiv \{n_i, \ell_i, m_i, m_{s_i}\} $.

\footnotetext{Nel caso dei MEAs, si può ignorare la dimensione finita del nucleo, assumendo $ \mu \equiv m_e $ e $ a \equiv a_0 $.}

\begin{example}{Notazione spettroscopica}{}
	In notazione spettroscopica si perde l'informazione su $ m_i $ ed $ m_{s_i} $. Ad esempio:
	\begin{equation*}
		\ket{1,0,0,\uparrow ; 3,1,-1,\uparrow ; 3,1,0,\uparrow ; 3,1,1,\downarrow}^\text{(a)} \equiv \text{1s}^1 \text{3p}^3
	\end{equation*}
\end{example}

\subsubsection{Energia totale}

La binding energy di un atomo è definita come il lavoro necessario a separare l'atomo in un nucleo isolato e nei suoi $ N $ elettroni, tutti a riposo e all'infinito. La sua energia totale è invece $ E = - E_\text{bind} $. \\
Nel caso di elettroni non-interagenti, l'energia totale è semplicemente la somma delle loro singole energie (che sono negative).

\begin{example}{}{}
	L'energia dello stato $ \text{1s}^1 \text{3p}^3 $ è (dall'Eq. \ref{eq:1-e-en}:
	\begin{equation*}
		E[\text{1s}^1 \text{3p}^3] = E_1 + 3 E_3 = - \frac{1}{2} \left( \frac{1}{1^2} + 3 \cdot \frac{1}{3^2} \right) Z^2 E_\text{Ha} = - \frac{2}{3} Z^2 E_\text{Ha}
	\end{equation*}
	Gli stati ad energia minima per un sistema di $ N = 4 $ elettroni sono però quelli con $ 2 $ elettroni in $ n = 1 $ (massimo numero nell'unica shell con $ n = 1 $, ovvero $ \text{1s} $) e $ 2 $ elettroni in $ n = 2 $; ad esempio:
	\begin{equation*}
		E[\text{1s}^2 \text{2s}^2] = 2 E_1 + 2 E_2 = - \frac{5}{4} Z^2 E_\text{Ha}
	\end{equation*}
\end{example}

\subsubsection{Elettroni reali}

In sistemi reali, l'approssimazione $ V_{ee} \equiv 0 $ può risultare estremamente fallace: ad esempio, per un atomo neutro in cui $ N = Z $ si ha che $ V_{ee} $ è dello stesso ordine di grandezza, ma di segno opposto, di $ V_{ne} $, così da cancellarne gli effetti: in tal caso, trascurare $ V_{ee} $ porterebbe a risultati fisicamente insensati.

\section{Atomi a 2 elettroni}

Gli atomi polielettronici più semplici sono quelli con $ N = 2 $ (es.: $ \ch{He} $, $ \ch{Li}^+ $, $ \ch{Be}^{2+} $, ...). L'Hamiltoniana di questo sistema è:
\begin{equation*}
	\mathcal{H} = T + V_{ne} + V_{ee} = - \frac{\hbar^2}{2m_e} \lap_1 + - \frac{\hbar^2}{2m_e} \lap_2 - \frac{Ze^2}{r_1} - \frac{Ze^2}{r_2} + \frac{e^2}{\abs{\ve{r}_1 - \ve{r}_2}} = \mathcal{H}_1 + \mathcal{H}_2 + V_{ee}
\end{equation*}
Questa Hamiltoniana è resa non-fattorizzabile dal termine $ V_{ee} $, il quale può però essere trattato perturbativamente nel caso $ N = 2 $; per $ N \ge 3 $, invece, bisogna tener conto della schermatura del potenziale $ V_{ne} $ da parte di quello $ V_{ee} $. \\
La funzione d'onda per elettroni indipendenti ($ V_{ee} \equiv 0 $) in questo è:
\begin{equation*}
	\Psi_{\alpha_1 , \alpha_2}(w_1 , w_2) = \frac{1}{\sqrt{2}} \left[ \psi_{\alpha_1}(w_1) \psi_{\alpha_2}(w_2) - \psi_{\alpha_1}(w_2) \psi_{\alpha_2}(w_1) \right]
\end{equation*}
con:
\begin{equation*}
	\psi_{\alpha_i}(w_i) = R_{n_i, \ell_i}(r_i) Y_{\ell_i, m_i}(\vartheta_i, \varphi_i) \chi_{m_{s_i}}(\sigma_i) \equiv \psi_{n_i, \ell_i, m_i}(\ve{r}_i) \chi_{m_{s_i}}(\sigma_i)
\end{equation*}
Si nota però che gli stati $ \Psi_{\alpha_1, \alpha_2} $ così definiti non sono necessariamente autostati dello spin totale $ S^2 $ (con $ \bs{S} \defeq \bs{s}_1 + \bs{s}_2 $): è utile lavorare con autostati di $ S^2 $, poiché la perturbazione $ V_{ee} \equiv V_{ee} \otimes \id_\text{spin} $ agisce solo sullo spazio orbitale, dunque il suo elemento di matrice si annulla tra stati con $ S $ diverso. \\
Per ottenere tali autostati, è utile separare la parte spaziale della funzione d'onda da quella di spin: si ottengono così un singoletto $ S = 0 $ (con funzione d'onda di spin antisimmetrica) ed un tripletto simmetrico $ S = 1 $ (con funzione d'onda di spin simmetrica). Si definiscono i relativi spinori $ \mathcal{X}^{S,M_S} $:
\begin{align*}
	\mathcal{X}^{0,0}(\sigma_1, \sigma_2) &= \frac{1}{\sqrt{2}} \left[ \chi_\uparrow(\sigma_1) \chi_\downarrow(\sigma_2) - \chi_\uparrow(\sigma_2) \chi_\downarrow(\sigma_1) \right] \\
	\mathcal{X}^{1,-1}(\sigma_1, \sigma_2) &= \chi_\downarrow(\sigma_1) \chi_\downarrow(\sigma_2) \\
	\mathcal{X}^{1,0}(\sigma_1, \sigma_2) &= \frac{1}{\sqrt{2}} \left[ \chi_\uparrow(\sigma_1) \chi_\downarrow(\sigma_2) + \chi_\uparrow(\sigma_2) \chi_\downarrow(\sigma_1) \right] \\
	\mathcal{X}^{1,1}(\sigma_1, \sigma_2) &= \chi_\uparrow(\sigma_1) \chi_\downarrow(\sigma_2)
\end{align*}
Le parti spaziali devono essere coerentemente (anti)simmetrizzate, così da ottenere:
\begin{equation*}
	\Psi^{0,0}_{n_1, \ell_1, m_1 ; n_2, \ell_2, m_2}(w_1, w_2) = \frac{1}{\sqrt{2}} \left[ \psi_{n_1, \ell_1, m_1}(\ve{r}_1) \psi_{n_2, \ell_2, m_2}(\ve{r}_2) + \psi_{n_1, \ell_1, m_1}(\ve{r}_2) \psi_{n_2, \ell_2, m_2}(\ve{r}_1) \right] \mathcal{X}^{0,0}(\sigma_1, \sigma_2)
\end{equation*}
\begin{equation*}
	\Psi^{1,M_S}_{n_1, \ell_1, m_1 ; n_2, \ell_2, m_2}(w_1, w_2) = \frac{1}{\sqrt{2}} \left[ \psi_{n_1, \ell_1, m_1}(\ve{r}_1) \psi_{n_2, \ell_2, m_2}(\ve{r}_2) - \psi_{n_1, \ell_1, m_1}(\ve{r}_2) \psi_{n_2, \ell_2, m_2}(\ve{r}_1) \right] \mathcal{X}^{1,M_S}(\sigma_1, \sigma_2)
\end{equation*}
Gli stati dello spin-singlet non hanno condizioni sui quantum numbers orbitali, ma devono necessariamente avere $ m_{s_1} = - m_{s_2} $, mentre, al contrario, gli stati dello spin-triplet non hanno condizioni sui quantum numbers di spin, ma devono avere $ (n_1,\ell_1,m_1) \neq (n_2,\ell_2,m_2) $.

\subsection{Stati eccitati}

Trascurando l'interazione elettronica, le energie non-pertubate dipendono solo da $ n_1,n_2 $:
\begin{equation*}
	E_{n_1,n_2}^{(0)} = - \frac{1}{2} \left( \frac{1}{n_1^2} + \frac{1}{n_2^2} \right) Z^2 E_\text{Ha}
\end{equation*}
Si vede che il ground state è uno stato dello spin-singlet: infatti esso ha entrambi gli elettroni in $ \text{1s} $, dunque $ \uparrow\downarrow $ o $ \downarrow\uparrow $, ovvero descritti da $ \mathcal{X}^{0,0} $. \\
Ricordando le selection rules sullo spin per le transizioni di dipolo elettrico (Eq. \ref{eq:1-e-el-dip-tr-spin}), si vede che queste avvengono soltanto tra stati appartenenti entrambi al singoletto o entrambi al tripletto, come si può vedere in Fig. \ref{helium} nel caso dell'elio: storicamente, si pensava ci fossero due specie distinte di elio, l'orto-elio ($ S = 1 $) ed il para-elio ($ S = 0 $).

\begin{figure}
	\centering
	\includegraphics[width = 0.40 \textwidth]{ortho-para-he.png}
	\caption{Energy levels and electric-dipole transitions of atomic $ \ch{He} $.}
	\label{helium}
\end{figure}

\subsection{Elettroni interagenti}

Trattando in maniera perturbativa il potenziale d'interazione $ V_{ee} $, al prim'ordine si ha una correzione all'energia (nelle configurazioni para- o orto-) data da:
\begin{equation*}
	\begin{split}
		\Delta E_{k_1,k_2}^\text{(p,o)}
		& = \braket{\Psi_{k_1,k_2}^\text{(p,o)} | V_{ee} | \Psi_{k_1,k_2}^\text{(p,o)}} = \frac{e^2}{2} \int_{\R^6} \frac{d^3r_1 d^3r_2}{\abs{\ve{r}_1 - \ve{r}_2}}  \abs{\psi_{k_1}(\ve{r}_1) \psi_{k_2}(\ve{r}_2) \pm \psi_{k_1}(\ve{r}_2) \psi_{k_2}(\ve{r}_1)}^2 \\
		& = \frac{e^2}{2} \int_{\R^6} \frac{d^3r_1 d^3r_2}{\abs{\ve{r}_1 - \ve{r}_2}} \left[ \abs{\psi_{k_1}(\ve{r}_1)}^2 \abs{\psi_{k_2}(\ve{r}_2)}^2 + \abs{\psi_{k_1}(\ve{r}_2)}^2 \abs{\psi_{k_2}(\ve{r}_1)}^2 \right] \\
		& \quad \pm \frac{e^2}{2} \int_{\R^6} \frac{d^3r_1 d^3r_2}{\abs{\ve{r}_1 - \ve{r}_2}} \left[ \psi_{k_1}^*(\ve{r}_1) \psi_{k_2}^*(\ve{r}_2) \psi_{k_1}(\ve{r}_2) \psi_{k_2}(\ve{r}_1) + \psi_{k_1}(\ve{r}_1) \psi_{k_2}(\ve{r}_2) \psi_{k_1}^*(\ve{r}_2) \psi_{k_2}^*(\ve{r}_1) \right] \\
		& \equiv E_{k_1,k_2}^\text{(c)} \pm E_{k_1,k_2}^\text{(s)}
	\end{split}
\end{equation*}
dove $ k_i \equiv \{n_i,\ell_i,m_i\} $. $ E^\text{(c)} $ è l'energia dovuta all'interazione Coulombiana, mentre $ E^\text{(s)} $ è dovuta all'interazione di scambio: si dimostra che $ E^\text{(s)} \ge 0 $, dunque la configurazione para- ha un'energia superiore a quella orto- (come si vede in Fig. \ref{helium}). \\
Il fatto che la configurazione orto- sia energeticamente inferiore a quella para- può anche essere dedotto intuitivamente dalla simmetria della funzione d'onda: nel caso orto-, poiché la funzione d'onda spaziale è antisimmetrica, si ha $ \psi(\ve{r},\ve{r}) = 0 $, dunque è molto meno probabile che i due elettroni si trovino vicini rispetto al caso para-. Poiché $ V_{ee} \sim \abs{\ve{r}_1 - \ve{r}_2}^{-1} $, ciò significa che la correzione al prim'ordine dell'energia sarà maggiore nel caso para- rispetto a quello orto-. \\
Gli stati $ (n_1[\ell_1])(n_2[\ell_2]) $ subiscono dunque uno splitting energetico pari a $ 2E^\text{(s)}_{n_1,\ell_1 ; n_2,\ell_2} $ (la dipendenza da $ m $ c'è solo in presenza di campi elettromagnetici esterni che perturbano la simmetria sferica), noto come \textit{splitting da scambio}. Questo splitting è molto importante, poiché sta alla base della \textit{prima regola di Hund}: lo stato energeticamente più basso è sempre quello che massimizza lo spin\footnotemark.

\footnotetext{Ciò nel caso dell'atomo a 2 elettroni non ha alcun effetto sul ground state, poiché esso ha configurazione elettronica $ \text{1s}^2 $, dunque necessariamente $ S = 0 $.}












\chapter{Molecole}
\selectlanguage{italian}

Lo studio delle molecole, ed in particolare delle molecole diatomiche, permette di illustrare due concetti fondamentali nello studio della fisica della materia: la separazione adiabatica del moto elettronico da quello nucleare ed il legame chimico.

\section{Separazione adiabatica}

L'Hamiltoniana per un sistema generico di $ M $ nuclei ed $ N $ elettroni è:
\begin{equation}
	\mathcal{H} = T_n + T_e + V_{ne} + V_{nn} + V_{ee}
	\label{eq:mol-ham-tot}
\end{equation}
Sebbene la funzione d'onda totale del sistema $ \Psi(r,R) $ dipenda dalle coordinate $ r $ di tutti gli elettroni ed $ R $ di tutti i nuclei, è possibile semplificare il problema. Infatti, essendo le masse dei nuclei e quelle degli elettroni diverse di circa tre ordini di grandezza, si può considerare che anche le time-scales dei moti dei nuclei siano tre ordini di grandezza maggiori rispetto a quelle dei moti degli elettroni: c'è dunque un disaccoppiamento dei due moti\footnotemark, in quanto si può assumere che il moto nucleare non induca alcuna transizione sugli stati elettronici (poiché le energie dei moti nucleari sono troppo piccole per colmare i gap energetici tra stati elettronici), da cui il nome di \textit{separazione adiabatica} (o di Born-Oppenheimer).

\footnotetext{In maniera approssimativa, si può dire che gli elettroni percepiscono i nuclei come fermi, mentre i nuclei risentono solo degli effetti medi del moto elettronico.}

\begin{definition}{Fattorizzazione adiabatica}{}
	La funzione d'onda totale del sistema si fattorizza come:
	\begin{equation}
		\Psi(r,R) = \Phi(R) \psi_e(r,R)
	\end{equation}
	dove $ \Phi(R) $ è la \textit{funzione d'onda nucleare} e $ \psi_e(r,R) $ è la \textit{funzione d'onda elettronica}, la quale è soluzione dell'equazione d'onda elettronica:
	\begin{equation}
		[T_e + V_{ne}(r,R) + V_{ee}(r)] \psi_e^{(a)}(r,R) = E_e^{(a)}(R) \psi_e^{(a)}(r,R)
		\label{eq:mol-el-eq}
	\end{equation}
	con $ (a) $ set di autovalori.
\end{definition}

La funzione d'onda $ \psi_e^{(a)}(r,R) $ descrive un autostato elettronico per una geometria fissata dei nuclei: la dipendenza da $ R $ di $ \psi_e^{(a)}(r,R) $ ed $ E_e^{(a)}(R) $ è puramente parametrica.

\begin{proposition}{Equazione d'onda nucleare}{}
	La funzione d'onda nucleare $ \Phi(R) $ è soluzione dell'equazione d'onda nucleare:
	\begin{equation}
		[T_n + V_\text{ad}^{(a)}(R)] \Phi(R) = E_\text{tot} \Phi(R)
		\label{eq:mol-nucl-eq}
	\end{equation}
	dove il \textit{potenziale adiabatico totale} è:
	\begin{equation}
		V_\text{ad}^{(a)}(R) \equiv E_e^{(a)}(R) + V_{nn}(R)
		\label{eq:ad-pot}
	\end{equation}

	\tcblower

	\begin{proof}
		Partendo dall'Eq. \ref{eq:mol-ham-tot}:
		\begin{equation*}
			[T_n + T_e + V_{nn} + V_{ne} + V_{ee}] \Phi(R) \psi_e(r,R) = E_\text{tot} \Phi(R) \psi_e(r,R)
		\end{equation*}
		e notando che, essendo $ T_e \sim \lap_r $, si ha $ T_e [\Phi(R) \psi_e(r,R)] = \Phi(R) T_e \psi_e(r,R) $:
		\begin{equation*}
			- \sum_\alpha \frac{\hbar^2}{2M_\alpha} \lap_{R_\alpha} [\Phi(R) \psi_e(r,R)] + \Phi(R) \underbrace{[T_e + V_{nn} + V_{ne}] \psi_e(r,R)}_{\text{equazione elettronica}} + \Phi(R) V_{nn} \psi_e(r,R) = E_\text{tot} \Psi(r,R)
		\end{equation*}
		Il primo termine diventa:
		\begin{equation*}
			\begin{split}
				- \sum_\alpha \frac{\hbar^2}{2m_\alpha} \lap_{R_\alpha} [\Phi(R) \psi_e(r,R)]
				= & - \sum_\alpha \frac{\hbar^2}{2M_\alpha} [\Phi(R) \lap_{R_\alpha} \psi_e(r,R) + 2 \nabla_{R_\alpha} \Phi(R) \cdot \nabla_{R_\alpha} \psi_e(r,R)] \\
				& - \psi_e(r,R) \sum_\alpha \frac{\hbar^2}{2M_\alpha} \lap_{R_\alpha} \Phi(R)
			\end{split}
		\end{equation*}
		I primi due termini (non-adiabatici) possono essere ignorati in approssimazione adiabatica, così da rimanere solo col terzo:
		\begin{equation*}
			\psi_e^{(a)}(r,R) [T_n + E_e^{(a)}(R) + V_{nn}(R)] \Phi(R) = E_\text{tot} \Phi(R) \psi_e^{(a)}(r,R)
		\end{equation*}
		Moltiplicando per $ [\psi_e^{(a)}(r,R)]^* $ da sinistra ed integrando su tutte le $ r $ si ottiene infine la tesi.
	\end{proof}
\end{proposition}

L'equazione elettronica presenta, nel caso $ N > 1 $, le stesse complicazioni della trattazione dei MEAs, ed è ugualmente risolvibile col metodo HF (con l'ulteriore complicazione che ora bisogna considerare varie geometrie molecolari date dalle $ R $): una volta selezionato lo stato elettronico $ a $, esso segue adiabaticamente il moto nucleare, non transizionando ad altri stati $ a' \neq a $ (questa è un'approssimazione della realtà). \\
Per quanto riguarda l'equazione nucleare, essa esprime il moto dei nuclei all'interno del potenziale adiabatico totale, composto da un termine Coulombiano repulsivo ed un contributo elettronico attrattivo (è ciò che permette alle molecole di esistere come stati legati). $ V_\text{ad}^{(a)}(R) $, funzione di $ 3M $ variabili, presenta una simmetria roto-traslazionale: ciò suggerisce che esso dipenda dalle distanze relative tra i nuclei atomici. Un esempio di andamento di $ V_\text{ad}^{(a)}(R) $ è riportato in Fig. \ref{ad-pot} (per il ground state di una molecola diatomica): si vede che il potenziale ha un minimo $ R_\text{m} $ finito, attorno al quale avvengono le oscillazioni del moto nucleare a bassa temperatura.

\begin{figure}
	\centering
	\includegraphics[width = 0.40 \textwidth]{adiabatic-potential.png}
	\caption{Adiabatic potential for a diatomic molecule.}
	\label{ad-pot}
\end{figure}

\section{Legami}

L'andamento del potenziale adiabatico in Fig. \ref{ad-pot} è alquanto generale, anche per molecole con $ M > 2 $, e pone le basi per lo studio di strutture molecolari complesse.

\subsection{\texorpdfstring{$ \ch{H_2^+} $}{H2+}}

Si consideri ad esempio la più semplice molecola diatomica: $ \ch{H_2^+} $. In questo caso non è presente alcuna repulsione elettrone-elettrone, dunque sull'unico elettrone agisce esclusivamente $ V_{ne} = V_\text{L} + V_\text{R} $, dove $ \text{L},\text{R} $ si riferiscono rispettivamente al nucleo sinistro o destro (arbitrariamente scelto il verso dell'asse molecolare $ \hat{\ve{e}}_z $): questi possono essere assunti fissi e posti a $ \pm R_{12}/2 $ lungo $ \hat{\ve{e}}_z $, dove $ \ve{R}_{12} \equiv \abs{\ve{R}_\text{L} - \ve{R}_\text{R}} $.

\begin{figure}[!b]
	\centering
	\includegraphics[width = 0.40 \textwidth]{ne-potential-h2.png}
	\caption{$ V_{ne} $ potential for the $ \ch{H_2^^+} $ molecule.}
	\label{ne-h2}
\end{figure}

Con riferimento alla Fig. \ref{ne-h2}, si noti che per $ -R_{12}/2 < r_z < R_{12}/2 $ il potenziale $ V_{ne}(r,R_{12}) $ è circa doppiamente negativo rispetto al caso atomico, suggerendo che l'elettrone potrebbe abbassare la propria energia potenziale media avendo una distribuzione di probabilità piccata nel mezzo dei due nuclei. Per verificare se ciò permette la formazione di un legame, bisogna confrontare che $ V_\text{ad}(R_{12}) < V_\text{ad}(\infty) = E_\text{1s} = -\frac{1}{2} E_\text{Ha} $ (poiché in tale limite si hanno praticamente un protone ed un atomo di idrogeno infinitamente separati, e $ \mu/m_e \approx 1 $). A tal fine, si adotta un approccio variazionale per verificare il valore dell'energia tramite la LCAO (Linear Combination of Atomic Orbitals): si considerino i due ground state atomici $ \ket{\text{1s};\text{L}} \equiv \ket{\text{L}} $ e $ \ket{\text{1s};\text{R}} \equiv \ket{\text{R}} $, assumendo $ \braket{\text{L} | \text{R}} > 0 $, e, data la simmetria assiale (cilindrica attorno $ \hat{\ve{e}}_z $) del problema, si considerino le combinazioni lineari opportunamente normalizzate (plottate in Fig. \ref{gs-h2}):
\begin{equation}
	\ket{\text{S}} = [2(1 + \braket{\text{L} | \text{R}})]^{-1/2} (\ket{\text{L}} + \ket{\text{R}})
	\qquad \qquad
	\ket{\text{A}} = [2(1 - \braket{\text{L} | \text{R}})]^{-1/2} (\ket{\text{L}} - \ket{\text{R}})
	\label{eq:h2-kets}
\end{equation}

\begin{figure}[!b]
	\centering
	\includegraphics[width = 0.50 \textwidth]{symm-asymm-h2.png}
	\caption{Symmetric and antisymmetric LCAO for the ground state of $ \ch{H_2^+} $.}
	\label{gs-h2}
\end{figure}

Per il principio variazionale, per l'energia del ground state vale $ E_e^{(\text{gs})}(R_{12}) \le \braket{\text{S}/\text{A} | T_e + V_{ne} | \text{S}/\text{A}} \equiv \mathcal{E}_{\text{S},\text{A}} \,\,\forall R_{12} \in \R_+ $.

\begin{proposition}{}{}
	A grande distanza $ R_{12} \gg a_0 $ si ha:
	\begin{equation}
		\mathcal{E}_{\text{S},\text{A}} + \frac{E_\text{Ha}}{2} \simeq - \frac{e^2}{R_{12}} (1 \pm \braket{\text{L} | \text{R}})
		\label{eq:h2-symm-asymm}
	\end{equation}

	\tcblower

	\begin{proof}
		Per calcolo diretto:
		\begin{equation*}
			\begin{split}
				\mathcal{E}_\text{\text{S},\text{A}}
				& = \braket{\text{S}/\text{A} | T_e + V_{ne} | \text{S}/\text{A}} = \frac{\braket{\text{L} | T_e + V_{ne} | \text{L}} \pm \braket{\text{R} | T_e + V_{ne} | \text{L}} + (\text{L} \leftrightarrow \text{R})}{2 (1 \pm \braket{\text{L} | \text{R}})} \\
				& = \frac{\braket{\text{L} | T_e + V_\text{L} | \text{L}} + \braket{\text{L} | V_\text{R} | \text{L}} \pm \braket{\text{R} | T_e + V_\text{L} | \text{L}} \pm \braket{\text{R} | V_\text{R} | \text{L}}}{1 \pm \braket{\text{L} | \text{R}}} = - \frac{E_\text{Ha}}{2} + \frac{\braket{\text{L} | V_\text{R} | \text{L}} \pm \braket{\text{R} | V_\text{R} | \text{L}}}{1 \pm \braket{\text{L} | \text{R}}}
			\end{split}
		\end{equation*}
		essendo $ \ket{\text{L}} $ autostato di $ T_e + V_\text{L} $ con autovalore $ -E_\text{Ha}/2 $. I due elementi di matrice di $ V_\text{R} $ sono entrambi funzioni reali e negative di $ R_{12} $: il termine $ \braket{\text{L} | V_\text{R} | \text{L}} $ rappresenta l'attrazione che il nucleo destro esercita sull'elettrone orbitante attorno al nucleo sinistro, dunque per $ R_{12} $ abbastanza grande si ha:
		\begin{equation*}
			\braket{\text{L} | V_\text{R} | \text{L}} \simeq - \frac{e^2}{R_{12}}
		\end{equation*}
		Il termine $ \braket{\text{R} | V_\text{R} | \text{L}} $, d'altro canto, può essere approssimato considerando che la distribuzione $ \psi_\text{L}(\ve{r}) \psi_\text{R}(\ve{r}) $ è piccata nella regione assiale tra i due nuclei, al centro della quale $ V_\text{R} \simeq - \frac{e^2}{R_{12}/2} $, così che:
		\begin{equation*}
			\braket{\text{R} | V_\text{R} | V_\text{L}} \simeq - \frac{e^2}{R_{12}} 2\braket{\text{L} | \text{R}}
		\end{equation*}
		Si può quindi espandere l'espressione totale considerando $ \braket{\text{L} | \text{R}} $ sufficientemente piccolo:
		\begin{equation*}
			\mathcal{E}_{\text{S},\text{A}} + \frac{E_\text{Ha}}{2} \simeq - \frac{e^2}{R_{12}} \frac{1 \pm 2 \braket{\text{L} | \text{R}}}{1 \pm \braket{\text{L} | \text{R}}} \simeq - \frac{e^2}{R_{12}} (1 \pm \braket{\text{L} | \text{R}})(1 \mp \braket{\text{L} | \text{R}}) \simeq - \frac{e^2}{R_{12}} (1 \pm \braket{\text{L} | \text{R}})
		\end{equation*}
		che è il risultato cercato.
	\end{proof}
\end{proposition}

L'Eq. \ref{eq:h2-symm-asymm} mostra come $ \mathcal{E} + \frac{E_\text{Ha}}{2} + V_{nn} $, con $ V_{nn} = \frac{e^2}{R_{12}} $, sia negativo per $ \ket{\text{S}} $ e positivo per $ \ket{\text{A}} $: ciò significa che la repulsione tra i due nuclei è vinta solo dalla LCAO simmetrica, nel qual caso di forma effettivamente un \textit{legame}. \\
Si può anche considerare il risultato esatto per $ R_{12} $ generico (plottato in Fig. \ref{h2-en}):
\begin{equation}
	\mathcal{E}_{\text{S},\text{A}} = - \frac{E_\text{Ha}}{2} + \frac{\braket{\text{L} | V_\text{R} | \text{L}} \pm \braket{\text{R} | V_\text{R} | \text{L}}}{1 \pm \braket{\text{L} | \text{R}}}
\end{equation}

\begin{figure}[!b]
	\centering
	\includegraphics[width = 0.70 \textwidth]{h2-en.png}
	\caption{Total adiabatic potential for $ \ket{\text{S}} $ and $ \ket{\text{A}} $.}
	\label{h2-en}
\end{figure}

Si possono fare alcune osservazioni:
\begin{enumerate}
	\item $ V_\text{ad}^{(\text{S})}(R_{12}) $ presenta un minimo per $ R_{12} = R_\text{m} \simeq 2.5 a_0 $, e questa buca di potenziale è abbastanza profonda da legare i due protoni, pertanto $ \ket{\text{S}} $ è detto \textit{stato legante};
	\item $ V_\text{ad}^{(\text{A})}(R_{12}) $ è monotonamente decrescente, ovvero è un potenziale puramente repulsivo, dunque $ \ket{\text{A}} $ è detto \textit{stato anti-legante};
	\item per $ R_{12} < R_\text{m} $ entrambi i potenziali adiabatici divergono a causa di una singolarità in $ V_{nn} $, il quale non è più efficacemente schermato, ed anche perché gli elettroni sono costretti a muoversi in una buca di potenziale più stretta, dunque con un'energia cinetica maggiore (per il principio d'indeterminazione).
\end{enumerate}
In definitiva, il legame in $ \ch{H_2^+} $ è dovuto sia ad un abbassamento dell'energia cinetica elettronica (poiché l'elettrone si muove in una buca di potenziale più larga) e ad un abbassamento dell'energia potenziale del sistema (poiché l'elettrone scherma la repulsione tra i due nuclei tramite una distribuzione di probabilità piccata nel mezzo). \\
Questo modello variazionale è valido per $ \braket{\text{L} | \text{R}} \ll 1 $, ovvero $ R_{12} \gg a_0 $, ma diventa particolarmente inaccurato per $ R_{12} $ piccolo, poiché in tal caso nella LCAO vanno considerati anche gli stati diversi da $ \text{1s} $: il modello variazionale dà energia di legame molecolare $ V_\text{ad}^{(\text{S})}(\infty) - V_\text{ad}^{(\text{S})}(R_\text{m}) \simeq 1.76\ev $ con $ R_\text{m} = 2.50 a_0 $, mentre il valore sperimentale è $ 2.79\ev $ con $ R_\text{m} = 2.00 a_0 $.

\paragraph{Numeri quantici}

La simmetria di $ \ch{H_2^+} $ è cilindrica, dunque soltanto $ L_z $ può essere diagonalizzato\footnote{In coordinate cilindriche $ \ve{r} = (\rho \cos \varphi, \rho \sin \varphi, z) $ e $ \hat{L}_z \defeq -i\hbar (x \pa_y - y \pa_x) = -i \hbar \pa_\varphi $, quindi le sue autofunzioni sono $ f(\varphi) = \mathcal{N} e^{i m \varphi} $. La condizione di periodicità $ f(\varphi) = f(\varphi + 2\pi) $ impone $ m \in \Z $.}: l'unico numerico quantico $ \virgolette{buono} $ è $ m $. In base al valore di $ \abs{m} = 0, 1, 2, \dots $ si indicano gli stati con $ \sigma, \pi, \delta, \dots $; inoltre, gli stati anti-leganti acquistano un asterisco. Si noti che, a parte gli stati $ \sigma $, tutti questi stati sono doppiamente degeneri ($ \pm m $). Questa degenerazione è dovuta a una simmetria dell'Hamiltoniana. In particolare, in assenza di campi magnetici esterni, essa è simmetrica per $ \sigma_\text{v} : \varphi \mapsto -\varphi $, ovvero una riflessione rispetto al piano che contiene l'asse molecolare, la quale implica che l'energia sia $ E = E(\abs{m}) $. Nel caso di molecole diatomiche omonucleari, inoltre, si hanno due ulteriori simmetrie. La prima è la simmetria sotto riflessione rispetto al piano ortogonale all'asse molecolare, denominata $\sigma_\text{n}: z \mapsto -z$, e la seconda è una rotazione di $\pi$: $x \mapsto -x, y \mapsto -y$. La combinazione delle due costituisce l'inversione $ \sigma_\text{i} : \ve{r} \mapsto -\ve{r}$: in base all'autovalore di $ \sigma_\text{i} $, si distingue tra stati $ g $ (gerade, con $ +1 $) e $ u $ (ungerade, con $ -1 $).

\begin{example}{Stati $ \ket{\text{S}},\ket{\text{A}} $ in $ \ch{H_2^+} $}{}
	Gli stati $ \ket{\text{S}},\ket{\text{A}} $, definiti in Eq. \ref{eq:h2-kets}, sono LCAO di soli stati $ \text{1s} $, dunque sono stati $ \sigma $: in particolare, $ \ket{\text{S}} \equiv 1\sigma $ e $ \ket{\text{A}} \equiv 1\sigma^* $.
\end{example}



\subsubsection{Derivazione alternativa}

Si consideri la generica combinazione lineare $ \psi_e(r_1,r_2,R) = c_1 \psi_\text{1s}^{(1)}(r_1) + c_2 \psi_\text{1s}^{(2)}(r_2) $, con $ \ve{r}_{1,2} \equiv \ve{r} - \ve{R}_{1,2} $ e dipendenza parametrica da $ \ve{R} \equiv \ve{R}_1 - \ve{R}_2 $ (ricordare che $ \psi_\text{1s}(r) = \frac{1}{\sqrt{4\pi}} \frac{2}{a_0^{2/3}} \exp -r/a_0 $). Il metodo variazionale impone:
\begin{equation}
	\delta( \braket{\psi_e | \mathcal{H} | \psi_e} - E \braket{\psi_e | \psi_e}) = 0
	\label{eq:h2-var}
\end{equation}
Innanzitutto, si calcola:
\begin{equation*}
	\begin{split}
		\braket{\psi_e | \psi_e}
		& = \abs{c_1}^2 + \abs{c_2}^2 + c_1^* c_2 \int d^3r\, \psi_\text{1s}^{(1)*}(r_1) \psi_\text{1s}^{(2)}(r_2) + c_1 c_2^* \int d^3r\, \psi_\text{1s}^{(1)}(r_1) \psi_\text{1s}^{(2)*}(r_2) \\
		& \equiv \abs{c_1}^2 + \abs{c_2}^2 + c_1^* c_2 S_{12}(R) + c_1 c_2^* S_{12}^*(R)
	\end{split}
\end{equation*}
Inoltre, si trova:
\begin{equation*}
	\braket{\psi_e | \mathcal{H} | \psi_e} = \abs{c_1}^2 H_{11} + \abs{c_2}^2 H_{22} + c_1^* c_2 H_{12} + c_1 c_2^* H_{12}^*
\end{equation*}
con:
\begin{equation*}
	\begin{split}
		H_{11} = \int d^3r\, \psi_\text{1s}^{(1)*}(r_1) \left[ - \frac{\hbar^2 \lap}{2m_e} - \frac{e^2}{r_1} - \frac{e^2}{r_2} \right] \psi_\text{1s}^{(1)}(r_1)
		& = - \frac{E_\text{Ha}}{2} + \int d^3\, \psi_\text{1s}^{(1)*}(r_1) \left( -\frac{e^2}{r_2} \right) \psi_\text{1s}^{(1)}(r_1) \\
		& \equiv - \frac{E_\text{Ha}}{2} + J_1(R)
	\end{split}
\end{equation*}
\begin{equation*}
	\begin{split}
		H_{12} = \int d^3r\, \psi_\text{1s}^{(1)*}(r_1) \left[ - \frac{\hbar^2 \lap}{2m_e} - \frac{e^2}{r_1} - \frac{e^2}{r_2} \right] \psi_\text{1s}^{(2)}(r_2)
		& = - \frac{E_\text{Ha}}{2} S_{12}(R) + \int d^3r\, \psi_\text{1s}^{(1)*}(r_1) \left( - \frac{e^2}{r_1} \right) \psi_\text{1s}^{(2)}(r_2) \\
		\equiv - \frac{E_\text{Ha}}{2} S_{12}(R) + K_{12}(R)
	\end{split}
\end{equation*}
Per molecole omonucleari si trova $ H_{22} = H_{11} $, mentre in generale essi saranno diversi.
La variazione in Eq. \ref{eq:h2-var} avviene rispetto a $ c_1 $ e $ c_2 $, dunque imponendo le derivate parziali rispetto ad entrambi nulle si trova:
\begin{equation*}
	\begin{cases}
		c_1^* H_{11} + c_2^* H_{12}^* - E (c_1^* + c_2^* S_{12}^*) = 0 \\
		c_2^* H_{22} + c_1^* H_{12} - E (c_2^* - c_1^* S_{12}) = 0
	\end{cases}
	\qquad \iff \qquad
	\begin{vmatrix}
		H_{11} - E & H_{12}^* - E S_{12}^* \\
		H_{12} - E S_{12} & H_{22} - E
	\end{vmatrix}
	= 0
\end{equation*}
Questa matrice ha due autovalori (e due autovettori), che nel caso omonucleare sono:
\begin{equation*}
	E_\pm = \frac{H_{12} \mp H_{12}}{1 \mp S_{12}}
	\qquad \qquad
	\begin{pmatrix}
		c_1 \\ c_2
	\end{pmatrix}
	=
	\begin{pmatrix}
		1 \\ \pm 1
	\end{pmatrix}
\end{equation*}
In maniera esplicita:
\begin{equation}
	E_\pm(R) = - \frac{E_\text{Ha}}{2} + \frac{J_{1}(R) \mp K_{12}(R)}{1 \mp S_{12}(R)}
\end{equation}
che equivale all'Eq. \ref{eq:h2-symm-asymm}. Si vede che il legame non è dovuto all'integrale Coulombiano $ J_1(R) $, bensì all'integrale di risonanza $ K_{12}(R) $, il quale non ha un'interpretazione classica. Ovviamente tale modello è solo approssimato, ed una maggiore sovrapposizione coi dati sperimentali si potrebbe ottenere considerando altri stati oltre $ \text{1s} $ nella LCAO.

\subsection{Legami covalenti e ionici}

\subsubsection{Dimeri}

Il legame in $ \ch{H_2^+} $ non è un vero e proprio legame chimico, in quanto non c'è una coppia di elettroni che occupa lo stesso orbitale legante (legame covalente): questo è invece il caso di $ \ch{H_2} $, le cui autofunzioni per gli orbitali vanno determinate con metodi che considerino la repulsione elettrone-elettrone (come il metodo HF). Ad ogni modo, si trova un potenziale simile al potenziale adiabatico in Fig. \ref{ad-pot}; un tale potenziale ammette anche elettroni in stati eccitati: se entrambi si trovano nello stesso orbitale legante, allora la molecola esiste in uno stato legato, sebbene con binding energy minore rispetto a $ 1\sigma $. \\
Considerando atomi con $ Z $ via via maggiore, si vengono a formare molecole diatomiche omonucleari (o \textit{dimeri}) con un maggior numero di orbitali leganti/anti-leganti. Si evidenziano alcune proprietà:
\begin{enumerate}
	\item gli orbitali leganti sono sempre energeticamente inferiori rispetto ai corrispettivi anti-leganti;
	\item la forza di un legame covalente aumenta all'aumentare dell'ordine di legame, ovvero del numero di coppie di elettroni (di spin opposto, per il principio d'esclusione\footnotemark) presenti nell'orbitale legante ed assenti nel corrispettivo orbitale anti-legante (il più forte è $ \ch{N_2} $, con $ 10\ev $);
	\item nel riempimento della shell $ \text{2p} $, si ha un inversione per la quale gli orbitali $ 2\pi $ sono energeticamente inferiori a quelli $ 2\sigma $ (nonostante ci sia più overlap tra gli orbitali $ 2p_z $ (lungo l'asse molecolare) che formano $ 2\sigma $ rispetto agli orbitali $ 2p_x , 2p_y $ che formano $ 2\pi $); l'ordinamento $ \virgolette{normale} $ viene ristabilito da $ \ch{O_2} $ in poi.
\end{enumerate}

\footnotetext{Si noti che, considerando lo spin, gli orbitali $ \sigma $ possono contenere 2 elettroni, tutti gli altri orbitali 4 (poiché già doppiamente degeneri per $ \pm m $).}

\begin{figure}
	\centering
	\includegraphics[width = 0.50 \textwidth]{sigma-pi-b2.jpg}
	\caption{Bonding orbitals (only for valence electrons) for $ \ch{B_2} $.}
	\label{bond-b2}
\end{figure}

Prendendo il caso di $ \ch{B_2} $ (o anche $ \ch{O_2} $), gli ultimi due elettroni di valenza vanno disposti per l'Aufbau su un orbitale $ \pi $ (o $ \pi^* $), il quale può però ospitare quattro elettroni: di conseguenza, per la prima regola di Hund essi occupano l'orbitale con spin paralleli. \\
Vengono osservati anche dimeri come $ \ch{Be_2} $ e $ \ch{Ne_2} $ che, nonostante un ordine di legame nullo, manifestano un debole legame non chimico (legame di van der Waals).

\paragraph{Potenziali analitici}

È possibile formulare alcuni modelli analitici per approssimare il potenziale adiabatico. Un esempio è il \textit{potenziale di Lennard-Jones}:
\begin{equation}
	V_\text{LJ}(R) = 4 \varepsilon \left[ \left( \frac{\sigma}{R} \right)^{12} - \left( \frac{\sigma}{R} \right)^6 \right]
\end{equation}
con $ \varepsilon \sim 1-100 \,\text{meV} $ e $ \sigma \sim 1-3 \ang $. Questo potenziale descrive i dimeri di atomi i cui orbitali sono completamente pieni (ovvero i gas nobili): essi infatti non formano legami covalenti, ma legami dovuti a deboli fluttuazioni indotte sui dipoli elettrici\footnotemark (che altrimenti sarebbero nulli). Ponendo $ V_\text{LJ}' = 0 $ si trova la distanza d'equilibrio $ R_\text{m} = \sqrt[6]{2} \sigma $ e la profondità della buca di potenziale $ V_\text{LJ}(R_\text{m}) = -\varepsilon $ (si noti che questi valori valgono per soli dimeri, in quanto molecole poliatomiche hanno geometrie più complesse). \\
Per descrivere legami covalenti si utilizza invece il \textit{potenziale di Morse}:
\begin{equation}
	V_\text{M}(R) = \mathcal{E}_d \left[ 1 - e^{-\beta (R - R_\text{m})} \right]^2
\end{equation}
dove $ \mathcal{E}_d $ è l'energia di dissociazione della molecola e $ \beta $ è un parametro legato alla frequenza caratteristica delle oscillazioni attorno al minimo di potenziale da $ \kappa = 2 \mathcal{E}_d \beta^2 $. A differenza di $ V_\text{LJ}(R) $, il potenziale di Morse ha un valore finito per $ R = 0 $.

\footnotetext{Un dipolo istantaneo $ \bs{\mu}_1 $ in un atomo induce un dipolo istantaneo indotto $ \bs{\mu}_2 $ sull'atomo adiacente: questo dipolo indotto avrà sempre verso opposto al primo dipolo, risultando dunque in un'interazione attrattiva tra di essi che scala come $ R^{-6} $.}

\subsubsection{Molecole diatomiche eteronucleari}

Sebbene i dimeri godano di una particolare simmetria per riflessioni, esistono anche molecole diatomiche eteronucleari. In questi casi, c'è un'asimmetria nella distribuzione di carica, poiché gli elettroni saranno maggiormente attratti da uno dei due atomi: come si vede in Fig. \ref{etero}, gli orbitali leganti si trovano energeticamente più vicini agli orbitali dell'atomo che li ha energeticamente più bassi, fino a casi limite come $ \ch{HF} $ in cui gli orbitali coincidono energeticamente. Si distingue quindi tra legami covalenti polari e legami ionici: di quest'ultimi si può dare una descrizione approssimata come di un completo trasferimento di carica dall'atomo meno elettronegativo a quello più elettronegativo (come in $ \ch{HF} $).

\begin{figure}
	\centering
	\includegraphics[width = 0.50 \textwidth]{eteronuclear.png}
	\caption{Bonding orbitals for eteronuclear diatomic molecules.}
	\label{etero}
\end{figure}

\subsection{Classificazione}

Nonostante la generale tendenza degli atomi ad attrarsi a grandi distanze e respingersi a corte distanze (Fig. \ref{ad-pot}), ci sono grosse differenze di distanze di equilibrio e profondità della buca di potenziale per i vari meccanismi di legame. \\
Se uno (o entrambi) dei due atomi è un gas nobile, allora il legame è di tipo dipolo-dipolo indotto (van der Waals), con una distanze d'equilibrio $ R_\text{m} \sim 250-400 \,\text{pm} $ grande ed energia di legame nell'ordine dei $ \text{meV} $. I gas nobili mantengono la forma di gas monoatomico fino a temperature relativamente basse, per poi formare dimeri in grado di condensarsi e solidificarsi. \\
Al contrario, alcuni atomi con shell non complete ($ \ch{N} $, $ \ch{O} $, $ \ch{F} $) tendono a formare molecole diatomiche con legami covalenti corti e molto forti (binding energy $ \sim \text{eV} $), molecole mantenute anche nelle fasi liquide e solide a basse temperature. Per la maggior parte degli altri atomi, l'energia extra guadagnata formando legami chimici multipli rende energeticamente conveniente formare legami metallici estesi (o anche solidi covalenti), piuttosto che molecole diatomiche.\\
Nel caso di molecole eteronucleari fortemente polari si può arrivare alla formazione di legami ionici, con un transferimento (totale o parziale) di carica da un'atomo all'altro: al pari dei legami covalenti e metallici, le distanze d'equilibrio sono $ R_\text{m} \sim 80-250 \,\text{pm} $ e le binding energies $ \sim 1-10\ev $.

\section{Spettri molecolari}

Si passa ora allo studio del moto dei due nuclei di una molecola diatomica, governato dall'Eq. \ref{eq:mol-nucl-eq}. Si noti che il potenziale adiabatico è indipendente dalla posizione del centro di massa della molecola (invarianza per traslazioni) e dall'orientazione nello spazio dell'asse molecolare (invarianza per rotazioni): di conseguenza, si può separare il moto in moto del centro di massa, che risulta in uno spettro continuo (il moto termico randomico del centro di massa genera l'allargamento delle linee spettrali per effetto Doppler), e moto relativo, il quale gode di simmetria sferica rispetto a $ \ve{R}_{12} $. Per quanto riguarda ques'ultimo, tramite la separazione delle variabili si può separare l'equazione nucleare in tre equazioni, di cui due angolari standard ed una radiale con potenziale adiabatico. Le soluzioni delle equazioni angolari sono le armoniche sferiche $ Y_{\ell,m_\ell} $, dove $ \ell $ è il momento angolare molecolare ed $ m_\ell \in [-\ell,\ell] $ la sua proiezione lungo l'asse molecolare. Per quanto riguarda l'equazione radiale, invece, data dall'Eq. \ref{eq:1-e-rad-eq} con $ V(r) \equiv V_\text{ad}(r) $ (ed $ r \equiv R_{12} $), si può considerare che il potenziale adiabatico presenta un minimo in in un punto d'equilibrio finiro $ R_{12} = R_\text{m} > 0 $, dunque si può considerare che la molecola studiata si trovi in un intorno di $ R_\text{m} $, così da poter ignorare variazioni del termine centrifugo $ \frac{\hbar^2 \ell (\ell + 1)}{2\mu R_{12}^2} $ ($ R_{12}^{-2} $ varia poco in un intorno di $ R_\text{m} > 0 $) ed approssimare il moto radiale come indipendente da $ \ell $; espandendo il potenziale attorno ad $ R_\text{m} $:
\begin{equation*}
	V_\text{ad}(R_{12}) = V_\text{ad}(R_\text{m}) + \frac{1}{2} \frac{d^2 V_\text{ad}(R)}{dR^2}\bigg\vert_{R = R_\text{m}} (R_{12} - R_\text{m})^2 + \dots \equiv V_\text{ad}(R_\text{m}) + \frac{1}{2} \kappa (R_{12} - R_\text{m})^2 + \dots
\end{equation*}
Si può dunque approssimare il potenziale con un potenziale armonico ed il sistema con un oscillatore armonico; definendo il numero quantico $ v \in \N_0 $ come il numero di nodi della funzione d'onda radiale, si trova lo spettro radiale vibrazionale:
\begin{equation}
	E_\text{vib}(v) = \hbar \omega \left( v + \frac{1}{2} \right)
\end{equation}
con $ \omega \equiv \sqrt{\kappa / \mu} $. Grandezze tipiche sono $ \hbar \omega \sim 20-400 \,\text{meV} $. Il termine centrifugo, invece, determina uno spettro rotazionale:
\begin{equation}
	E_\text{rot}(\ell) = \frac{\hbar^2 \ell (\ell + 1)}{2\mu R_\text{m}^2} \equiv \frac{\ve{L}^2}{2I}
\end{equation}
Questa è la cosiddetta approssimazione del rotatore rigido, in cui si definisce il momento d'inerzia $ I $: ciò è possibile poiché si è assunto che $ R_{12} \approx R_\text{m} $ privo di variazioni in un intorno di $ R_\text{m} $. Grandezze tipiche sono nell'ordine dei $ \text{meV} $: l'energia rotazionale più alta è quella del $ \ch{H_2} $, pari a $ 7 \,\text{meV} $.

\subsection{Spettro rotazionale e roto-vibrazionale}

Se si considerano transizioni adiabatiche, ovverosia transizioni in cui gli elettroni rimangono nel loro ground state, le molecole soddisfano le solite selection rules di dipolo elettrico: $ \Delta \ell = \pm 1 $.
Nel caso delle molecole diatomiche, l'operatore di dipolo elettrico è determinato dalla separazione tra i due nuclei e dalla loro differenza di carica: nel caso di molecole omonucleari tale differenza è nulla, dunque per esse non si osservano transizioni di dipolo elettrico\footnote{Questo è il motivo per cui l'aria (composta principalmente da $ \ch{N_2} $ e $ \ch{O_2} $) è altamente trasparente ai raggi IR; la trasparenza nel visibile e nel vicino UV è invece associata al grosso gap (vari $ \text{eV} $) che separa il ground state elettronico ed il primo stato eccitato elettronico.}. Inoltre, l'operatore di dipolo elettrico dipende dalla distanza $ R_{12} $: in un intorno di $ R_\text{m} $, si può espandere come:
\begin{equation*}
	\ve{d}(R_{12}) = \ve{d}(R_\text{m}) + \frac{\dd \ve{d}(R)}{\dd R}\bigg\vert_{R = R_\text{m}} (R_{12} - R_\text{m}) + \dots
\end{equation*}
In questo modo, come prima si erano eliminate le anarmonicità meccaniche, ora si sono eliminate le anarmonicità elettriche. Il termine costante $ \ve{d}(R_\text{m}) $ dà luogo allo spettro rotazionale puro (poiché $ \braket{\text{f} | \ve{d}(R_\text{m}) | \text{f}} \neq 0 $ solo se $ v_\text{f} = v_\text{i} $), mentre il termine proporzionale a $ R_{12} - R_\text{m} $ determina lo spettro roto-vibrazionale\footnote{Più precisamente, il termine $ \sim (R_{12} - R_\text{m})^n $ ha elemento di matrice non-nullo solo se $ v_\text{f} - v_\text{i} = \pm n $.}. \\
Gli spettri rotazionali e vibrazionali delle molecole si osservano principalmente come spettri d'assorbimento, dato che il basso rate di emissione spontanea (decay rate $ \sim \mathcal{E}_\text{if}^3 $) determina una preponderanza di fenomeni di decadimento radiation-less (es.: collisioni tra molecole).

\subsubsection{Spettro rotazionale}

Gli spettri puramente rotazionali si osservano nella regione del lontano IR e sono associati a transizioni con $ \Delta v = 0 $ e $ \Delta \ell = \pm 1 $. La differenza energetica tra $ \ket{\ell_\text{i}} \equiv \ket{\ell} $ e $ \ket{\ell_\text{f}} \equiv \ket{\ell + 1} $ è:
\begin{equation}
	\Delta E_\text{rot}(\ell) = \frac{\hbar^2}{I} (\ell + 1)
\end{equation}
Di conseguenza, lo spettro rotazionale di un campione di molecole in vari stati rotazionali iniziali sarà composto da una serie di righe d'assorbimento equispaziate, con una separazione energetica $ \hbar^2 / I $ che è il doppio della tipica scala energetica rotazionale $ \hbar^2 / 2I $. Una misura di tale separazione permette di stimare $ I $ e, di conseguenza, la distanza interatomica d'equilibrio $ R_\text{m} $. \\
L'equispaziatura dei salti energetici rotazionali permette di definire una \textit{temperatura rotazionale}, determinata da tale scala energetica:
\begin{equation}
	\theta_\text{rot} \defeq \frac{\hbar^2}{I k_\text{B}}
\end{equation}
dove $ k_\text{B} = 8.617 \cdot 10^{-5} \ev/\text{K} $ è la costante di Boltzmann. La scala di temperatura tipica è $ \sim 12 \,\text{K} $ (per energie $ \sim 1 \,\text{meV} $). Questa temperatura va confrontata con quella a cui si svolge l'esperimento: se $ T \ll \theta_\text{rot} $ il sistema si troverà sostanzialmente nel suo ground state, mentre se $ T \gg \theta_\text{rot} $ saranno popolati con buona probabilità molti livelli eccitati.

\begin{figure}
	\centering
	\includegraphics[width = 0.80 \textwidth]{rot-spectr.png}
	\includegraphics[width = 0.80 \textwidth]{rot-vib-spectr.png}
	\caption{Observed purely-rotational (above) and roto-vibrational (below) absorption spectrum of gas-phase $ \ch{HCl} $.}
	\label{rot-vib-sp}
\end{figure}
\begin{figure}
	\centering
	\includegraphics[width = 0.40 \textwidth]{rot-vib-spectr-det.png}
	\caption{Scheme of electric-dipole transitions between rotational levels $ \ket{v = 0} \rightarrow \ket{v = 1} $.}
	\label{rot-vib-det}
\end{figure}

\subsubsection{Spettro roto-vibrazionale}

Gli spettri roto-vibrazionali si osservano nella regione del vicino IR e sono associati a transizioni $ \Delta v > 0 $ e $ \Delta \ell = \pm 1 $: la maggior parte dell'intensità è data dalle transizioni $ \Delta v = 1 $, mentre le overtone transitions $ \Delta v > 1 $ sono più deboli. \\
Come si vede in Figg. \ref{rot-vib-sp}-\ref{rot-vib-det}, una volta fissata una transizione con $ \Delta v = 1 $, essa è $ \virgolette{decorata} $ da varie transizioni $ \Delta \ell = \pm 1 $, essendo il campione composto di molecole in vari stati rotazionali iniziali: le transizioni con $ \Delta \ell = -1 $ formalo la P-branch, mentre quelle con $ \Delta \ell = +1 $ la R-branch. Si noti l'assenza di una transizione piccata in $ \hbar \omega $, la quale sarebbe associata ad una transizione dipolo-proibita $ \Delta \ell = 0 $. \\
In presenza di uno spettrometro a bassa risoluzione, non si riescono a distinguere le singole righe rotazionali e si osserva un'unica riga vibrazionale. \\
Analogamente allo spettro rotazionale, si può definire una \textit{temperatura vibrazionale}:
\begin{equation}
	\theta_\text{vib} \defeq \frac{\hbar \omega}{k_\text{B}}
\end{equation}
In questo caso le energie si attestano su $ \sim 0.5\ev $, dunque $ \theta_\text{vib} \sim 6000 \,\text{K} $.

\subsubsection{Molecole poliatomiche}

Nel caso di molecole poliatomiche, in generale il potenziale adiabatico dipenderà sia dalle distanze relative tra i vari nuclei che dagli angoli tra tali distanze, rendendo la trattazione del problema molto più complessa. In particolare, ci saranno vari modi normali d'oscillazione attorno al minimo multidimensionale di $ V_\text{ad} $, ciascuno modellabile da un oscillatore armonico indipendente.

\subsection{Eccitazioni elettroniche}

Fotoni nel range del visibile e UV possono provocare eccitazioni dello stato elettronico delle molecole: queste eccitazioni possono essere modellate dalla promozione di un elettrone da un orbitale molecolare occupato ad uno vuoto. Una transizione $ \psi_e^{(a)} \rightarrow \psi_e^{(b)} $ provoca un cambiamento di superficie potenziale adiabatica: data la time-scale estremamente piccola, i nuclei non hanno tempo di muoversi, dunque (come in Fig. \ref{elec-ex}) tale cambiamento è $ \virgolette{verticale} $ in $ R_{12} $. Dato che in generale $ R_\text{m}^{(a)} \neq R_\text{m}^{(b)} $, solitamente le transizioni elettroniche sono accompagnate da transizioni vibrazionali dovute alla variazione della geometria d'equilibrio.

\begin{figure}
	\centering
	\includegraphics[width = 0.50 \textwidth]{electr-exc.png}
	\caption{Adiabatic potential surfaces for $ \psi_e^{(a)} \rightarrow \psi_e^{(b)} $.}
	\label{elec-ex}
\end{figure}
\begin{figure}
	\centering
	\includegraphics[width = 0.50 \textwidth]{zero-point-eff.png}
	\caption{Quantum zero-point vibrational energy.}
	\label{zero-p}
\end{figure}

\subsubsection{Effetti di punto-zero}

Come si vede in Fig. \ref{zero-p}, la binding energy $ E_b $ di una molecola diatomica è leggermente inferiore alla profondità $ V_\text{ad}(\infty) - V_\text{ad}(R_\text{m}) $ della buca di potenziale adiabatico. I due valori coinciderebbero se le masse dei nuclei fossero infinite (o se essi fossero oggetti classici), ma per il principio d'indeterminazione è presente un'energia vibrazionale di punto-zero $ E_\text{vib}(0) = \frac{\hbar \omega}{2} $, la quale spiega la piccola differenza tra i due valori. \\
Sperimentalmente, si possono indagare questi effetti di punto-zero andando a variare la massa degli isotopi coinvolti, dato che $ \omega \propto \mu^{-1/2} $ mentre $ V_\text{ad}(R_{12}) $ non dipende da esso. Uno degli effetti di punto-zero più grandi si ha nel $ \ch{^4He_2} $: la buca di potenziale adiabatico è profonda circa $ 900 \,\mu\text{eV} $, ma il sistema è estremamente debolmente legato con $ E_b \simeq 0.1 \,\mu\text{eV} $, dunque $ E_\text{vib}(0) $ bilancia quasi completamente l'attrazione adiabatica. A confermare la dipendenza da $ \mu $, per il più leggero $ \ch{^3He_2} $ non si osservano stati legati (l'energia di punto-zero è più grande della buca di potenziale adiabatico).













\part{Solidi}
\pagestyle{body}

\chapter{Fisica Statistica}
\selectlanguage{italian}

La Fisica Statistica (quantistica) fornisce le relazioni tra le proprietà medie microscopiche di un sistema con le sua dinamica macroscopica. In particolare, si fanno due assunzioni fondamentali (per rendere la trattazione indipendente dalle condizioni iniziali del sistema):
\begin{enumerate}
	\item equilibrio: il sistema ha superato tutti i transienti e tutte le quantità collettive (es.: pressione) hanno delle piccole fluttuazioni attorno ad un valore ben definito;
	\item ergodicità: le medie temporali delle osservabili coincidono con le medie d'ensemble.
\end{enumerate}

\section{Operatore densità}

Si consideri un sistema quantistico di $ N $ corpi (con $ N \sim N_\text{A} \simeq 6.022 \cdot 10^{23} $). Detti $ \{\ket{\psi_i}\}_{i \in \mathcal{I}} $ i possibili stati in cui si può trovare il sistema e detta $ w_i \in [0,1] $ la probabilità che il sistema si trovi in $ \ket{\psi_i} $ (condizione di normalizzazione $ \sum_{i \in \mathcal{I}} w_i = 1 $), l'\textit{expectation value} di un'osservabile $ \hat{B} $ sarà:
\begin{equation}
	[B] \defeq \sum_{i \in \mathcal{I}} w_i \braket{\psi_i | \hat{B} | \psi_i}
\end{equation}
Data un set completo ortonormale $ \{\ket{b_j}\}_{j \in \mathcal{J}} $ di autostati di $ \hat{B} $, allora $ \hat{B} = \sum_{j \in \mathcal{J}} b_j \ket{b_j}\bra{b_j} $, ovvero:
\begin{equation}
	[B] = \sum_{i \in \mathcal{I}} \sum_{j \in \mathcal{J}} w_i b_j \abs{\braket{b_j | \psi_i}}^2
	\label{eq:b-exp-val}
\end{equation}

\begin{definition}{Operatore densità}{}
	Si definisce l'\textit{operatore statistico} (o operatore densità) come:
	\begin{equation}
		\hat{\rho} \defeq \sum_{i \in \mathcal{I}} w_i \ket{\psi_i}\bra{\psi_i}
	\end{equation}
\end{definition}

\begin{theorem}{Expectation value}{}
	L'expectation value di un'osservabile $ \hat{B} $ può essere espresso come:
	\begin{equation}
		[B] = \tr(\hat{\rho} \hat{B})
	\end{equation}

	\tcblower

	\begin{proof}
		Dall'Eq. \ref{eq:b-exp-val} (e ricordando che $ \{\ket{b_j}\}_{j \in \mathcal{J}} $ è base ortonormale di $ \hilb $):
		\begin{equation*}
			[B] = \sum_{i \in \mathcal{I}} \sum_{j \in \mathcal{J}} w_i b_j \braket{b_j | \psi_i} \braket{\psi_i | b_j} = \sum_{j \in \mathcal{J}} b_j \braket{b_j | \hat{\rho} | b_j} = \sum_{j \in \mathcal{J}} \braket{b_j | \hat{\rho} \hat{B} | b_j} \eqdef \tr(\hat{\rho} \hat{B})
		\end{equation*}
	\end{proof}
\end{theorem}

Un corollario banale è che $ \tr{\hat{\rho}} = 1 $ (con $ \hat{B} \equiv \id_\hilb $). Inoltre, si vede che per un sistema in uno stato puro, ovvero $ w_i = \delta_{i,i_0} $ per un certo $ i_0 \in \mathcal{I} $, allora $ \hat{\rho} = \ket{\psi_{i_0}}\bra{\psi_{i_0}} $ è un proiettore (dunque idempotente: $ \hat{\rho}^2 = \hat{\rho} $).

\section{Ensemble all'equilibrio}

\subsection{Ensemble microcanonico}

Si consideri un sistema isolato ad energia fissata $ E $ con una piccola incertezza $ \Delta E $: questo viene detto \textit{ensemble microcanonico}. Dall'ipotesi di ergodicità deriva che tutti gli stati amessi dalla conservazione dell'energia sono equiprobabili:
\begin{equation}
	w_i =
	\begin{cases}
		\frac{1}{\Omega} & E_i \in [E - \Delta E/2 , E + \Delta E/2] \\
		0 & E_i \notin [E - \Delta E/2 , E + \Delta E/2]
	\end{cases}
	\label{eq:microcanon-ens}
\end{equation}
dove $ \Omega = \Omega(E, \Delta E) $ è il numero di stati accessibili nello spazio delle fasi posta la condizione energetica.

\subsection{Ensemble canonico}

Si consideri ora un sistema $ \text{S} $ al'interno dell'universo $ \text{U} $ e sia $ \text{W} \equiv \text{U} - \text{S} $. Assumendo che $ \mathcal{H}_\text{U} = \mathcal{H}_\text{S} + \mathcal{H}_\text{W} + \mathcal{H}_\text{SW} \approx \mathcal{H}_\text{S} + \mathcal{H}_\text{W} $, ovvero che sia $ \text{S} $ che $ \text{U} $ siano considerabili isolati, allora questi saranno descritti dalla distribuzione microcanonica Eq. \ref{eq:microcanon-ens}.

\begin{theorem}{Distribuzione canonica}{}
	Dato uno stato possibile $ \ket{\psi_m} $ con energia $ E_m $ del sistema $ \text{S} $, la sua probabilità è data dalla \textit{distribuzione di Boltzmann}:
	\begin{equation}
		P_m^\text{S} = \frac{1}{Z} e^{- \beta E_m}
	\end{equation}
	dove $ \beta = \beta(T) $ e $ Z $ è la \textit{funzione di partizione} del sistema $ \text{S} $:
	\begin{equation}
		Z \defeq \sum_{i \in \mathcal{I}} e^{- \beta E_i} \equiv \tr e^{- \beta \mathcal{H}}
		\label{eq:part-func-states}
	\end{equation}

	\tcblower

	\begin{proof}
		Dall'ipotesi di interazione debole tra $ \text{S} $ e $ \text{W} $ si ha $ P^\text{U}(E) = P^\text{S}(E_m) P^\text{W}(E - E_m) $, dunque, essendo questi sistemi microcanonici:
		\begin{equation*}
			P_m^\text{S} = \frac{P^\text{U}(E)}{P^\text{W}(E-E_m)} = \frac{\Omega_\text{W}(E-E_m,\Delta E)}{\Omega_\text{U}(E,\Delta E)}
		\end{equation*}
		Assumendo $ E_m \ll E $ si può sviluppare in serie:
		\begin{equation*}
			\ln \Omega_\text{W}(E - E_m , \Delta E) = \ln \Omega_\text{W}(E , \Delta E) - \beta E_m + o(E_m^2)
			\qquad \qquad
			\beta \equiv \frac{\pa}{\pa x}\bigg\vert_{x = E} \ln \Omega_\text{W}(x , \Delta E)
		\end{equation*}
		Esponenziando e sostituendo nell'equazione precedente si ottiene:
		\begin{equation*}
			P_m^\text{S} = \frac{\Omega_\text{W}(E , \Delta E)}{\Omega_\text{U}(E, \Delta E)} e^{- \beta E_m} \equiv \frac{1}{Z} e^{- \beta E_m}
		\end{equation*}
		Il fattore iniziale non dipende da $ \ket{m} $, dunque è un fattore puramente di normalizzazione. Dalla condizione di normalizzazione $ \sum_{i \in \mathcal{I}} P_i^\text{S} = 1 $ si trova la tesi.
	\end{proof}
\end{theorem}

Si noti che nell'Eq. \ref{eq:part-func-states} la sommatoria è su tutti gli stati, includendo in particolare quelli degeneri. Si può passare ad una sommatoria sulle energie ammesse definendo la degenerazione del livello energetico $ E $ come $ g(E) $, così che:
\begin{equation}
	Z = \sum_E g(E) e^{- \beta E}
\end{equation}
Inoltre, sebbene $ P_m^\text{S} $ dipenda da tutti i possibili valori di energia, il rapporto di probabilità $ P_m^\text{S} / P_n^\text{S} = e^{-\beta (E_m - E_n)} $ dipende solo da $ \Delta E_{mn} \equiv E_m - E_n $. \\
L'operatore densità dell'ensemble canonico (di Gibbs) è diagonale nell'autobase di $ \mathcal{H} $:
\begin{equation}
	\hat{\rho}_\text{Gibbs} = \sum_{m \in \mathcal{I}} \frac{e^{-\beta E_m}}{Z} \ket{m}\bra{m} \equiv \frac{e^{-\beta \mathcal{H}}}{\tr e^{-\beta \mathcal{H}}}
	\label{eq:op-dens-gibbs}
\end{equation}

\begin{example}{Sottosistemi isolati}{}
	Si consideri $ \text{S} = \text{S}_1 \cup \text{S}_2 $, con $ \text{S}_1 $ ed $ \text{S}_2 $ isolati. Allora:
	\begin{equation*}
		P_{(m_1,m_2)}^\text{S} = P_{m_1}^{\text{S}_1} P_{m_2}^{\text{S}_2} = \frac{1}{Z_1 Z_2} e^{-\beta (E_{m_1} + E_{m_2})}
	\end{equation*}
	confermando che $ E_{(m_1,m_2)} = E_{m_1} + E_{m_2} $ ($ \beta_1 = \beta_2 = \beta $ all'equilibrio). Dunque la distribuzione di Boltzmann riproduce i risultati intuitivi. Si noti inoltre che la funzione di partizione è una funzione moltiplicativa ($ Z_{1,2} = Z_1 Z_2 $), così che il suo logaritmo sia additivo ($ \ln Z_{1,2} = \ln Z_1 + \ln Z_2 $).
\end{example}

\subsection{Termodinamica}

È possibile ricavare le principali quantità termodinamiche a partire dalla funzione di partizione e dall'operatore densità.

\begin{proposition}{Energia interna}{}
	L'energia interna media è:
	\begin{equation}
		U = - \frac{\pa}{\pa \beta} \ln Z
	\end{equation}

	\tcblower

	\begin{proof}
		Definendo $ U \equiv [\mathcal{H}] $:
		\begin{equation*}
			U = \tr(\hat{\rho}_\text{Gibbs} \mathcal{H}) = \sum_{m \in \mathcal{I}} P_m E_m = \frac{\sum_{m \in \mathcal{I}} E_m e^{-\beta E_m}}{\sum_{m \in \mathcal{I}} e^{- \beta E_m}} = - \frac{1}{Z} \frac{\pa Z}{\pa \beta} = - \frac{\pa}{\pa \beta} \ln Z
		\end{equation*}
	\end{proof}
\end{proposition}

Essendo $ \ln Z $ additivo su sistemi isolati, l'energia interna è correttamente una quantità termodinamica estensiva. Si vede inoltre che l'energia interna è una funzione non crescente di $ \beta $:
\begin{equation*}
	\frac{\pa U}{\pa \beta} = - \frac{\pa^2}{\pa \beta^2} Z = \frac{1}{Z^2} \left[ -Z \sum_{m \in \mathcal{I}} E_m^2 e^{-\beta E_m} + \left( \sum_{m \in \mathcal{I}} E_m e^{-\beta E_m} \right)^2 \right] = - [\mathcal{H}^2] + [\mathcal{H}]^2 = - [(\mathcal{H} - [\mathcal{H}])^2] \le 0
\end{equation*}

\begin{lemma}[before upper = {\tcbtitle}]{}{}
	\begin{equation}
		\beta = \frac{1}{k_\text{B} T}
	\end{equation}

	\tcblower

	\begin{proof}
		Si definisca l'energia libera $ F \equiv - \frac{1}{\beta} \ln Z $, così che $ U = \frac{\pa}{\pa \beta} (\beta F) $. D'altro canto, dalla Termodinamica si ha $ U - TS = F $, con $ S = - \frac{\pa F}{\pa T} $ l'entropia, dunque:
		\begin{equation*}
			\frac{F}{T} = \frac{U}{T} - S
			\quad \Rightarrow \quad
			U = \frac{\pa}{\pa(1/T)} \frac{F}{T} = \frac{\pa}{\pa \beta} (\beta F)
			\quad \Rightarrow \quad
			\beta \propto \frac{1}{T}
		\end{equation*}
		Nel limite classico si trova la corretta costante di proporzionalità.
	\end{proof}
\end{lemma}

\begin{proposition}{Calore specifico}{}
	Il calore specifico (a volume costante) è:
	\begin{equation}
		c_V = k_\text{B} \beta^2 \frac{\pa^2}{\pa \beta^2} \ln Z
	\end{equation}

	\tcblower

	\begin{proof}
		Ricordando che $ c_V \defeq \frac{\pa U}{\pa T} $ basta notare che $ \frac{\pa}{\pa T} = \frac{\pa \beta}{\pa T} \frac{\pa}{\pa \beta} = - \frac{1}{k_\text{B} T^2} \frac{\pa}{\pa \beta^2} $, ovvero:
		\begin{equation}
			\frac{\pa}{\pa T} = - k_\text{B} \beta^2 \frac{\pa}{\pa \beta}
		\end{equation}
	\end{proof}
\end{proposition}

È anche possibile dare una definizione più generale di entropia. Innanzitutto, dato che $ F = U - TS $, per l'ensemble canonico si trova:
\begin{equation}
	S = \frac{U}{T} - k_\text{B} \ln Z
	\label{eq:entropy-canon-ens}
\end{equation}
Questa può però essere generalizzata anche per sistemi non all'equilibrio.

\begin{definition}{Entropia}{}
	Dato un sistema con operatore densità $ \hat{\rho} $, si definisce l'\textit{entropia} come:
	\begin{equation}
		S \defeq - k_\text{B} \tr(\hat{\rho} \ln \hat{\rho})
	\end{equation}
\end{definition}

Sulla base degli autostati di $ \hat{\rho} $ (autostati $ \ket{\rho_m} $ con autovalori $ P_m $), l'entropia è:
\begin{equation}
	S = - k_\text{B} \sum_{m \in \mathcal{I}} P_m \ln P_m
\end{equation}

\begin{example}{Stato puro}{}
	Per uno stato puro $ P_m = \delta_{m,m_0} $, dunque correttamente $ S = -k_\text{B} \ln 1 = 0 $.
\end{example}

\begin{example}{Ensemble microcanonico}{}
	Per un ensemble microcanonico $ P_m = \frac{1}{\Omega} \,\,\forall m \in \mathcal{I} \,:\, \abs{\mathcal{I}} = \Omega $, dunque:
	\begin{equation*}
		S = - k_\text{B} \sum_{m = 1}^\Omega \frac{1}{\Omega} \ln \frac{1}{\Omega} = k_\text{B} \Omega \frac{1}{\Omega} \ln \Omega = k_\text{B} \ln \Omega
	\end{equation*}
	che è proprio l'equazione di Boltzmann.
\end{example}

\begin{example}{Ensemble canonico}{}
	Per l'ensemble canonico l'operatore densità è quello di Gibbs (Eq. \ref{eq:op-dens-gibbs}):
	\begin{equation*}
		S = - k_\text{B} \sum_{m \in \mathcal{I}} \frac{e^{-\beta E_m}}{Z} \ln \frac{e^{-\beta E_m}}{Z} = \frac{k_\text{B}}{Z} \sum_{m \in \mathcal{I}} e^{-\beta E_m} (\beta E_m + \ln Z) = k_\text{B} \beta [\mathcal{H}] + k_\text{B} \frac{Z \ln Z}{Z} = \frac{U}{T} + k_\text{B} \ln Z
	\end{equation*}
	che coincide con l'Eq. \ref{eq:entropy-canon-ens}. Si può dimostrare che $ \hat{\rho}_\text{Gibbs} $ è l'operatore densità che massimizza l'entropia per una data $ U = \tr(\hat{\rho}\mathcal{H}) $ fissata (oltre a $ N $ e $ V $): questo conferma il secondo principio della termodinamica, poiché un sistema generico descritto da $ \hat{\rho} $ tenderà all'ensemble canonico, ovverosia all'equilibrio, massimizzando l'entropia.
\end{example}

\section{Sistemi ideali}

Un sistema di $ N $ particelle si dice \textit{ideale} se si può separare:
\begin{equation}
	\mathcal{H} = \sum_{i = 1}^N \mathcal{H}_i \,:\, [\mathcal{H}_i , \mathcal{H}_j] = 0
\end{equation}
dove $ \mathcal{H}_i $ descrive soltanto i gradi di libertà della particella $ i $-esima. Lo spettro di questa Hamiltoniana è $ E_{\alpha_1, \dots, \alpha_N} = E_{\alpha_1} + \dots + E_{\alpha_N} $, ed inoltre si definisce il numero di occupazione $ n_k $ del livello energetico $ E_k $, $ I \in \mathcal{N} \subset \N $, così da poter scrivere la funzione di partizione del sistema come:
\begin{equation}
	Z = \sum_{\alpha_1, \dots, \alpha_N \in \mathcal{I}} \frac{\prod_{k \in \mathcal{I}} n_k!}{N!} \exp \left( - \beta \sum_{i = 1}^N E_{\alpha_i} \right)
\end{equation}
Per un sistema bosonico la sommatoria su $ \alpha_1, \dots, \alpha_N $ non ha condizioni, mentre per un sistema fermionico essa è solo su $ \alpha_1 \neq \dots \neq \alpha_N $ (ed in tal caso $ n_k \in \{0,1\} \,\,\forall k \in \mathcal{I} $, ovvero $ n_k! = 1 \,\,\forall k \in \mathcal{I} $). Questa funzione di partizione non è fattorizzabile in generale, a causa della produttoria per i bosoni e del principio di Pauli per i fermioni, ma diventa fattorizzabile in particolari regimi. \\
Si assuma che lo spettro enegetico del sistema non abbia limite superiore: a bassa temperatura, il fattore esponenziale nella distribuzione di Boltzmann tende a favorire stati con energia totale $ \sum_{i = 1}^N E_{\alpha_i} = \sum_{\alpha \in \mathcal{I}} n_\alpha E_\alpha $ più bassa; d'altro canto, ad alta temperatura, tutti gli stati single-particle con $ E_\alpha \lesssim k_\text{B} T $ hanno una probabilità pressoché eguale di essere occupati, e per $ T $ molto grande il numero di tali possibili stati è $ \gg N $, dunque la praticamente totalità degli stati a $ N $ particelle avranno soltanto stati con $ n_\alpha \in \{0,1\} $, indipendentemente dalla natura fermionica o bosonica del sistema. Ciò è riportato in Fig. \ref{boson-temp}. \\
Il limite classico (a $ Z $ separabile) può anche essere ottenuto agendo sulla densità del sistema: a parità di $ N $, sistemi con $ V $ maggiore avranno un maggior numero si stati accessibili: i gas ideali\footnote{I sistemi ideali possono essere solo in fase gassosa, poiché si ignora l'interazione tra particelle.} diventano classici o scaldandoli molto o rarefacendoli molto. Inoltre, a parità di densità e temperatura, un gas di particelle di massa inferiore avrà livelli energetici più vicini, dunque sarà più facilmente classico.

\begin{figure}
	\centering
	\includegraphics[width = 0.70 \textwidth]{ideal-high-temp.png}
	\caption{Occupancies of single-particle energy eigenstates for a bosonic system in the low-, intermediate- and high-temperature limits.}
	\label{boson-temp}
\end{figure}

\subsection{Limite ad alta temperatura}

Nel limite ad alta temperatura si può non tenere conto dell'occupazione degli stati, poiché il gas si comporta classicamente e $ n_\alpha \in \{0,1\} \,\,\forall \alpha \in \mathcal{I} $. Si ottiene dunque una funzione di partizione ad alta temperatura valida sia per fermioni che per bosoni:
\begin{equation}
	Z \simeq \frac{1}{N!} \sum_{\alpha_1, \dots, \alpha_N \in \mathcal{I}} \exp \left( -\beta \sum_{i = 1}^N E_{\alpha_i} \right) = \frac{1}{N!} \prod_{i = 1}^N Z_i \equiv \frac{Z_1^N}{N!}
\end{equation}
dove $ Z_1 $ è la funzione di partizione single-particle $ Z_1 \equiv \sum_{\alpha \in \mathcal{I}} \exp \left( -\beta E_\alpha \right) $. \\
Essendo in fase gassosa, il centro di massa di ciascuna particella si muove liberamente ed è dunque separabile dai gradi di libertà interni, ovvero $ \mathcal{H}_i = \mathcal{H}_{i,\text{tr}} + \mathcal{H}_{i,\text{int}} $ e $ Z_1 = Z_{1,\text{tr}} Z_{1,\text{int}} $: mentre $ Z_{1,\text{int}} $ dipende dallo specifico spettro di eccitazioni dei gradi di libertà interni, $ Z_{1,\text{tr}} $ è universale e dipende solo dalla massa della particella $ M $ e dal volume in cui si trova il sistema $ V $.

\subsubsection{Gradi di libertà traslazionali}

Si consideri una particella confinata in un cubo macroscopico di volume $ V = L^3 $ e si impongano delle condizioni al contorno periodiche, ovvero che la funzione d'onda è uguale su facce opposte del cubo. Le autofunzioni $ \psi_\ve{k}(\ve{r}) = L^{-3/2} \exp(i \ve{k} \cdot \ve{r}) $ possono avere solo valori quantizzati di $ \ve{k} $:
\begin{equation*}
	k_j = \frac{2\pi}{L} n_j \,\,,\,\, n_j \in \Z
\end{equation*}
L'energia cinetica traslazionale associata è:
\begin{equation}
	E_\ve{n} = \frac{\hbar^2 \ve{k}^2}{2M} = \frac{(2\pi\hbar)^2}{2M L^2} (n_x^2 + n_y^2 + n_z^2)
\end{equation}
Per $ L $ o $ M $ macroscopicamente grande (limite termodinamico) questo spettro diventa continuo:
\begin{equation*}
	\begin{split}
		Z_{1,\text{tr}}
		& = \sum_{\ve{n} \in \Z^3} e^{-\beta E_\ve{n}} = \sum_{\ve{n} \in \Z^3} \exp \left[ - \beta \frac{(2\pi\hbar)^2}{2M L^2} (n_x^2 + n_y^2 + n_z^2) \right] \\
		& \rightarrow \int_{\R^3} \dd n_x \dd n_y \dd n_z \exp \left[ -\beta \frac{(2\pi\hbar)^2}{2M L^2} (n_x^2 + n_y^2 + n_z^2) \right] = \left( \frac{L}{\hbar} \sqrt{\frac{M k_\text{B} T}{2\pi}} \right)^3
	\end{split}
\end{equation*}
Si può definire la \textit{lunghezza termica}:
\begin{equation}
	\Lambda \defeq \sqrt{\frac{2\pi\hbar^2}{M k_\text{B} T}}
\end{equation}
così da ottenere:
\begin{equation}
	Z_{1,\text{tr}} = \frac{V}{\Lambda^3}
\end{equation}
Si ha dunque:
\begin{equation*}
	Z = \frac{Z_1^N}{N!} = \frac{Z_{1,\text{tr}}^N}{N!} Z_{1,\text{int}}^N \equiv Z_\text{tr} Z_{1,\text{int}}^N
\end{equation*}

\begin{proposition}[before upper = {\tcbtitle}]{Energia interna traslazionale}{}
	\begin{equation}
		U_\text{tr} = \frac{3}{2} N k_\text{B} T
	\end{equation}

	\tcblower

	\begin{proof}
		\begin{equation*}
			\begin{split}
				U_\text{tr}
				& = - \frac{\pa}{\pa \beta} \ln Z_\text{tr} = - \frac{\pa}{\pa \beta} \ln \frac{V^N}{N! \Lambda^{3N}} = 3N \frac{\pa}{\pa \beta} \ln \Lambda = 3N \frac{\pa}{\pa \beta} \ln \sqrt{\beta} = \frac{3N}{2\beta} = \frac{3}{2} N k_\text{B} T
			\end{split}
		\end{equation*}
	\end{proof}
\end{proposition}

Ciò è in accordo con l'equipartizione dell'energia di Boltzmann.

\begin{theorem}{Equazione di stato dei gas perfetti}{}
	Per un gas perfetto vale l'equazione di stato:
	\begin{equation}
		P = \frac{N k_\text{B} T}{V}
	\end{equation}

	\tcblower

	\begin{proof}
		Innanzitutto, si trova l'energia libera nel limite ad alta temperatura:
		\begin{equation*}
			F \defeq - \frac{\ln Z}{\beta} \simeq - \frac{1}{\beta} \ln \frac{(Z_1)^N}{N!} = - \frac{1}{\beta} (N \ln Z_1 - \ln N!) \simeq - \frac{1}{\beta} (N \ln Z_1 - N \ln (N/e))
		\end{equation*}
		dove si è usata l'approssimazione di Stirling $ \ln N! \simeq N \ln (N/e) $. Dunque:
		\begin{equation}
			F = - N k_\text{B} T \ln \frac{e Z_1}{N}
		\end{equation}
		Per la pressione:
		\begin{equation*}
			\begin{split}
				P \defeq - \frac{\pa F}{\pa V} \bigg\vert_{T,N}
				& = - \frac{\pa}{\pa V} \bigg\vert_{T,N} \left[ - N k_\text{B} T \ln \frac{e Z_{1,\text{tr}} Z_{1,\text{int}}}{N} \right] = N k_\text{B} T \frac{\pa}{\pa V} \bigg\vert_{T,N} \left[ \ln \frac{eV}{N \Lambda^3} + \ln Z_{1,\text{int}} \right] \\
				& = N k_\text{B} T \frac{\pa}{\pa V}\bigg\vert_{T,N} \ln \frac{eV}{N \Lambda^3} = N k_\text{B} T \frac{1}{V}
			\end{split}
		\end{equation*}
	\end{proof}
\end{theorem}

Confrontando con $ pV = nRT $, essendo $ N = n N_\text{A} $, si trova $ k_\text{B} = R / N_\text{A} $.

\paragraph{Distribuzione energetica}

Per ricavare la distribuzione di Boltzmann delle energie cinetiche è necessario passare da un'integrale sugli stati $ \ve{n} $ ad un'integrale sull'energia $ E $. La densità degli stati nello spazio degli stati è:
\begin{equation}
	g(\ve{k}) = \frac{\dd n_x}{\dd k_x} \frac{\dd n_y}{\dd k_y} \frac{\dd n_z}{\dd k_z} = \frac{L^3}{8\pi^3} \equiv \frac{V}{8\pi^3}
\end{equation}
Si vede che gli stati sono distribuiti uniformemente. È necessario trovare la densità degli stati nello spazio delle energie tale per cui:
\begin{equation*}
	g(E) \dd E = g(\ve{k}) \dd^3\ve{k} = g(\ve{k}) 4\pi k^2 \dd k
\end{equation*}
Il differenziale $ \dd k $ si ottiene invertendo la relazione di dispersione $ E = E(k) $, che in questo caso dà:
\begin{equation*}
	k = \sqrt{\frac{2M E}{\hbar^2}}
	\quad \Rightarrow \quad
	\dd k = \sqrt{\frac{M}{2\hbar^2 E}} \dd E
\end{equation*}
Si ottiene dunque:
\begin{equation*}
	g(E) \dd E = \frac{V}{8\pi^3} 4 \pi \frac{2M E}{\hbar^2} \sqrt{\frac{M}{2\hbar^2 E}} \dd E = \frac{V M^{3/2}}{\sqrt{2} \pi^2 \hbar^3} \sqrt{E} \dd E
\end{equation*}
ovvero:
\begin{equation}
	g_\text{tr}(E) = \frac{V M^{3/2}}{\sqrt{2} \pi^2 \hbar^3} \sqrt{E}
\end{equation}
La distribuzione di probabilità che descrive l'energia di una singola particella si può trovare come:
\begin{equation*}
	\frac{\dd P(E)}{\dd E} = g_\text{tr}(E) \frac{e^{- \beta E}}{Z_{1,\text{tr}}} = \frac{V M^{3/2}}{\sqrt{2} \pi^2 \hbar^3} \sqrt{E} \frac{\Lambda^3}{V} e^{-\beta E}
\end{equation*}
Il risultato è indipendente dal volume e dalla massa della particella, ed è dunque universale (a parità di temperatura):
\begin{equation}
	\frac{\dd P(E)}{\dd E} = \frac{2}{\sqrt{\pi}} \beta^{3/2} \sqrt{E} e^{-\beta E}
	\label{eq:maxw-boltz-en}
\end{equation}
Si può ricavare anche una distribuzione delle velocità, trovando che ciascuna componente di $ \ve{v} = \frac{\ve{p}}{M} $ è distribuita gaussianamente come:
\begin{equation*}
	\frac{\dd P(v_j)}{\dd v_j} = \sqrt{\frac{\beta M}{2\pi}} \exp \left( -\beta \frac{M v_j^2}{2} \right)
\end{equation*}
La distribuzione di $ v \equiv \abs{\ve{v}} $ si trova dall'Eq. \ref{eq:maxw-boltz-en}:
\begin{equation*}
	\frac{\dd P(v)}{\dd v} = \frac{\dd P(E)}{\dd E} \frac{\dd E}{\dd v} = M v \frac{\dd P(E)}{\dd E}
\end{equation*}
Sostituendo $ E = \frac{1}{2} M v^2 $ si trova la \textit{distribuzione di Maxwell-Boltzmann}:
\begin{equation}
	\frac{\dd P(v)}{\dd v} = \sqrt{\frac{2 M^3}{\pi (k_\text{B} T)^3}} v^2 \exp \left( - \frac{M}{2 k_\text{B} T} v^2 \right)
\end{equation}












\chapter{Solidi Cristallini}
\selectlanguage{italian}

I sistemi macroscopici realizzano l'equilibrio termodinamico bilanciando la tendenza dell'energia interna a diminuire e quella dell'entropia ad aumentare: ad alte temperature prevale il contributo entropico, a basse temperature prevale quello energetico (nei solidi gli atomi hanno posizioni fisse, dunque l'entropia è più basse rispetto ai fluidi). Sperimentalmente, infatti, si trova che quasi tutti i materiali solidificano a temperature abbastanza basse: l'unica eccezione è l'elio, che rimane fluido fino a $ T \approx 0 \,\text{K} $, a meno che non si applichi una sufficiente pressione. \\
Così come la tendenza degli atomi a formare molecole (minimizzando $ V_\text{ad} $) viene misurata dalla binding energy molecolare, la tendenza di atomi/molecole a formare solidi viene misurata da una quantità analoga, l'\textit{energia di coesione}.

\section{Struttura microscopica}

Nello studio di solidi semplici, come ad esempio quelli formati da gas nobili, l'interazione tra atomi è semplificabile a coppie, quindi modellizzabile con il potenziale di Lennard-Jones:
\begin{equation}
	V_\text{ad}(\ve{R}_1 , \dots , \ve{R}_{N_n}) = \sum_{\alpha < \alpha'}^{N_n} V_2(\abs{\ve{R}_\alpha - \ve{R}_{\alpha'}})
	\qquad \qquad
	V_2(r) = 4\varepsilon \left[ \left( \frac{\sigma}{r} \right)^{12} - \left( \frac{\sigma}{r} \right)^6 \right]
\end{equation}
Per un coppia di atomi, la distanza d'equilibrio è $ R_\text{m} = \sqrt[6]{2} \sigma $ con potenziale minimo $ V_2(R_\text{m}) = -\varepsilon $. Inoltre, si vede che il potenziale decade come $ \sim r^{-6} $, quindi, contando che tipicamente il numero di atomi a distanza $ r $ da un dato atomo scala come $ \sim r^2 $, l'energia d'interazione totale a lunga distanza scala come $ r^{-4} $: si vede che ogni atomo interagisce significativamente soltanto con gli atomi ad esso vicini. \\
Si consideri un reticolo 1D di atomi: aggiungendo un terzo atomo ad una coppia di atomi, esso si posizionerà ad una distanza $ \approx R_\text{m} $ da essi (ignorando l'interazione tra secondi-vicini) ed abbasserà la buca di potenziale adiabatico a $ \approx -2\varepsilon $. Ciò è generalizzabile ad $ N_n $ atomi: essi si troveranno a distanza $ \approx R_\text{m} $ l'uno dall'altro e l'energia di coesione totale sarà $ \approx (N_n - 1) \varepsilon $ (circa $ \varepsilon $ per atomo, per $ N_n $ grande). Se si considera l'interazione con secondi- e terzi-vicini, si trova che la distanza d'equilibrio è leggermente minore rispetto a $ R_\text{m} $ e che l'energia di coesione è leggermente maggiore di $ (N_n - 1) \varepsilon $ (si veda Fig. \ref{lat-2}a). \\
Se invece si considera un reticolo 2D, il terzo atomo si posizionerà al vertice di un triangolo equilatero formato con i primi due. Aggiungendo altri atomi, si viene a formare un \textit{reticolo triangolare} (o esagonale, si veda Fig. \ref{lat-2}), in cui ciascun atomo a 6 primi-vicini, dunque l'energia di coesione sarà approssimativamente $ 3\varepsilon $ per atomo (6 legami condivisi da 2 atomi ciascuno). Si vede dunque che il modello d'interazione a coppie porta al principio di \textit{massima coordinazione}: gli atomi si dispongono in modo da massimizzare il numero di primi-vicini, così da massimizzare l'energia di coesione (minimizzando l'energia potenziale adiabatica). \\
Applicando il principio di massima coordinazione ad un reticolo 3D, si vede che 4 atomi massimizzano la loro coordinazione ponendosi ai vertici di un tetraedro regolare: gli atomi successivi si dispongono a formare o il \textit{face-centered cubic lattice} (reticolo fcc, Fig. \ref{lat-3}b) o l'\textit{hexagonal close-packed lattice} (reticolo hcp, Fig. \ref{lat-3}c). In entrambi questi reticoli ogni atomo ha 12 primi-vicini, dunque l'energia reticolare è $ 6\varepsilon $ per atomo; si vede, però, che entrambi sono formati da reticoli triangolari 2D sovrapposti: mentre nell'fcc si ripetono 3 orientazioni diverse, nell'hcp ce ne sono solo 2, dunque, quando si va a considerare l'interazione con secondi- e terzi-vicini, il reticolo fcc è leggermente favorito energeticamente rispetto all'hcp. \\
Confrontando i vari reticoli, si trovano delle distanze d'equilibrio leggermente diverse: in 1D $ R_\text{m} = \sqrt[6]{2} \sigma \simeq 1.22 \sigma $, in 2D $ R_\text{m} \simeq 1.11 \sigma $ ed in 3D $ R_\text{m} \simeq 1.09 \sigma $. Nella tavola periodica c'è una prevalenza a formare cristalli fcc ed hcp: ci sono alcuni bcc (body-centered cubic, simile all'fcc ma con un atomo al centro del cubo, al posto di quelli sulle facce) ed i casi del diamante (struttura a sé) e del polonio, che invece forma l'sc (simple cube). \\
È possibile formare strutture cristalline\footnote{Si noti che i vetri, al contrario dei cristalli, hanno una risposta plastica alle deformazioni: essi non sono solidi, ma liquidi con coefficiente di viscosità estremamente elevato, caratterizzati da livelli energetici estremamente vicini. Infatti, osservando i vedtri di edifici antichi, essi risulteranno ondulati, in quanto deformati nel tempo dalla gravità.} diverse da quelle descritte considerando un potenziale non necessariamente a 2 corpi (valido principalmente per gas nobili, che hanno interazioni deboli): potenziali a più corpi introducono effetti elettronici (quantistici) che danno direzionalità ai legami, fino ad arrivare ai solidi con legami metallici covalenti in cui le interazioni si estendono alla totalità del solido, impedendo di fattorizzare la funzione d'onda.

\begin{figure}[!h]
	\centering
	\includegraphics[width = 0.45 \textwidth]{lattice-1d.png}
	\qquad
	\includegraphics[width = 0.45 \textwidth]{lattice-2d.png}
	\caption{1D lattice and 2D triangular lattice of atoms.}
	\label{lat-2}
\end{figure}
\begin{figure}[!h]
	\centering
	\includegraphics[width = 0.70 \textwidth]{lattice-3d.png}
	\caption{3D lattices of atoms: (a) close packing of spheres, (b) fcc lattice, (c) hcp lattice.}
	\label{lat-3}
\end{figure}

\paragraph{Difetti}

Un cristallo, a qualunque temperatura finita, ha probabilità non-nulla di formare dei \textit{difetti}, come ad esempio un atomo mancante, un atomo aggiuntivo o un'impurità (atomo di tipo diverso). In particolare, essendo il numero di atomi in un cristallo estremamente grande, la concentrazione dei difetti in un cristallo all'equilibrio a bassa temperatura va come $ \sim e^{- \beta E} $ (statistica di Boltzmann), dove $ E \sim 1 \ev $ è l'energia tipica di formazione di un difetto. Termodinamicamente, è impossibile creare un cristallo con un'estensione finita privo di difetti. \\
La piccola differenza energetica tra reticoli fcc ed hcp permette la formazione di difetti estesi (al limite anche macroscopici) come dislocazioni, stacking faults e bordi di grano (vedere Figg. \ref{def-disl}-\ref{def-st-gr}). Una dislocazione avvience quando un layer a cui manca un atomo si collega ad un layer completo: queste configurazioni sono metastabili, e con un'energia finita, a temperatura sufficientemente elevata, è possibile spostare questi difetti all'interno del cristallo. Una stacking fault avviene invece quando si interrompe e viene modificata l'alternanza regolare dei layer, mentre un bordo di grano si verifica quando si sviluppa un cristallo a partire da due punti di duplicazione diversi e con orientazioni diverse, andando a creare un'interfaccia disordinata quando queste orientazioni si incontrano.

\begin{figure}[!h]
	\centering
	\includegraphics[width = 0.50 \textwidth]{def-disl.png}
	\caption{Dislocation in an sc lattice.}
	\label{def-disl}
\end{figure}
\begin{figure}[!h]
	\centering
	\includegraphics[width = 0.40 \textwidth]{def-st.png}
	\qquad \qquad
	\includegraphics[width = 0.40 \textwidth]{def-gr.png}
	\caption{Stacking fault and grain boundary in an fcc lattice.}
	\label{def-st-gr}
\end{figure}

\subsection{Reticoli cristallini}

Una delle proprietà fondamentali di un solido cristallino è l'equivalenza di varie sue regioni di spazio: infatti, si può assumere che la maggior parte degli atomi di un cristallo si trovino sufficientemente lontani dal suo bordo e da qualsiasi difetto per essere modellato come appartenente ad un cristallo perfetto ed infinitamente esteso. \\
Un cristallo perfetto infinito è un insieme di punti che gode di simmetria traslazionale discreta.

\begin{definition}{Reticolo di Bravais}{}
	Si dice \textit{reticolo di Bravais} $ n $-dimensionale un insieme di punti in $ \R^n $ che risulta invariante per traslazioni discrete:
	\begin{equation}
		\ve{r} \mapsto \ve{r}' = \ve{r} + \ve{R}
		\ , \
		\ve{R} = k_1 \ve{a}_1 + \dots k_n \ve{a}_n
	\end{equation}
	con $ \{k_j\}_{j = 1,\dots,n} \subset \Z $ e $ \{\ve{a}_j\}_{j = 1,\dots,n} \subset \R^n $ linearmente indipendenti.
\end{definition}

I generatori $ \{\ve{a}_j\}_{j = 1,\dots,n} $ vengono detti \textit{primitivi} se sono $ \virgolette{i più piccoli possibili} $ (non univocamente definiti) e formano una cella primitiva, ovverosia il volume minimo che contiene tutti i punti traslazionalmente non-equivalenti del reticolo (a meno di traslazioni) e che, ripetuto nello spazio tramite traslazioni discrete, riproduce il reticolo. Nel caso tridimensionale, il volume della cella primitiva è $ V_c = \abs{(\ve{a}_1 \cdot \ve{a}_2) \times \ve{a}_3} $. \\
La cella primitiva contiene tutta l'informazione del reticolo (che è una ripetizione di tale cella), dunque il suo studio è fondamentale. Il problema, nel definirla, è che non c'è un univoco set di vettori primitivi.

\begin{definition}{Cella di Wigner-Seitz}{}
	Dato punto $ \ve{R} $ in un reticolo di Bravais, si definisce la \textit{cella di Wigner-Seitz} l'insieme di punti $ \ve{r} $ più vicini ad $ \ve{R} $ rispetto ad un altro punto $ \ve{R}' $ del reticolo.
\end{definition}

\begin{proposition}{}
	La cella di Wigner-Seitz è una cella primitiva.
\end{proposition}

L'introduzione della cella di Wigner-Seitz elimina l'arbitrarietà nella scelta della cella primitiva.

\begin{figure}
	\centering
	\includegraphics[width = 0.50 \textwidth]{ws-2.png}
	\caption{Wigner-Seitx cell for a 2D lattice.}
	\label{ws-2}
\end{figure}
\begin{figure}
	\centering
	\includegraphics[width = 0.50 \textwidth]{ws-3.png}
	\quad
	\includegraphics[width = 0.30 \textwidth]{ws-g.png}
	\caption{Wigner-Seitx cell for the fcc lattice, the bcc lattic and graphene.}
	\label{ws-3}
\end{figure}

Si noti che, in generale, una cella primitiva può contenere più di un atomo (es.: $ \ch{Na Cl} $, grafite, etc.). Infatti, un cristallo è definito da un reticolo di Bravais e da una base, ovvero un insieme di uno o più atomi ripetuti in ogni punto del retivolo. È inoltre conveniente trattare le cosiddette \textit{celle convenzionali}, ovvero insiemi di una o più celle primitive che hanno forme preferibilmente cubiche (fcc, bcc o sc), per facilitare la trattazione. Bisogna tener conto che le celle convenzionali, essendo delle super-celle, contengono informazioni ridondanti rispetto alle celle primitive.

\subsubsection{Reticolo reciproco}

Si consideri una funzione $ f : \R \rightarrow \R $ periodica con periodo $ a \in \R $, ovvero tale per cui $ f(x -na) = f(x) \,\,\forall n \in \Z $. La sua antitrasformata di Fourier è:
\begin{equation}
	f(x) = \int_\R \frac{\dd k}{\sqrt{2\pi}} e^{ikx} \tilde{f}(k)
\end{equation}
I coefficienti di Fourier $ \tilde{f}(k) $ sono nulli, eccetto quelli con lo stesso periodo di $ f(x) $, ovvero quelli con $ e^{ikx} = e^{ik(x-a)} $: ciò equivale a $ e^{-ika} = 1 $, ovvero $ k = \ell \frac{2\pi}{a} $, con $ \ell \in \Z $. Questi valori vengono convenzionalmente indicati come:
\begin{equation}
	G = \frac{2\pi}{a} \ell
\end{equation}
e forniscono le frequenze discrete, nello spazio reciproco, che costituiscono la serie di Fourier di $ f(x) $:
\begin{equation}
	f(x) = \sum_G e^{iGx} \tilde{f}(G)
	\qquad \qquad
	\tilde{f}(G) = \frac{1}{a} \int_0^a \dd x\, e^{-i G x} f(x)
\end{equation}
I punti $ G $ nello spazio reciproco formano un reticolo, detto \textit{reticolo reciproco}, legato al reticolo nello spazio diretto da:
\begin{equation}
	e^{i G R} = e^{i 2\pi n \ell} = 1
\end{equation}
dove $ R = na $ sono le traslazioni del reticolo diretto. Si trova quindi $ \ell = \frac{1}{n} $. \\
Queste definizioni sono generalizzabili al caso 3D, in cui $ \ve{R} = n_1 \ve{a}_1 + n_2 \ve{a}_2 + n_3 \ve{a}_3 $. In questo caso, la condizione di reciprocità diventa:
\begin{equation}
	e^{i \ve{G} \cdot \ve{R}} = 1
\end{equation}
In 3D si trova che:
\begin{equation}
	\ve{G} = \ell_1 \ve{b}_1 + \ell_2 \ve{b}_2 + \ell_3 \ve{b}_3
\end{equation}
con $ \ell_1, \ell_2, \ell_3 \in \Z $ e:
\begin{equation}
	\ve{b}_1 = \frac{2\pi}{V_c} \ve{a}_2 \times \ve{a}_3
	\qquad
	\ve{b}_1 = \frac{2\pi}{V_c} \ve{a}_2 \times \ve{a}_3
	\qquad
	\ve{b}_1 = \frac{2\pi}{V_c} \ve{a}_2 \times \ve{a}_3
\end{equation}
Si noti che $ \ve{a}_i \cdot \ve{b}_j = 2\pi $, dunque $ \ve{G} \cdot \ve{R} = 2\pi (n_1 \ell_1 + n_2 \ell_2 + n_3 \ell_3) $. La serie di Fourier in 3D diventa:
\begin{equation}
	f(\ve{x}) = \sum_\ve{G} e^{i \ve{G} \cdot \ve{x}} \tilde{f}(\ve{G})
	\qquad \qquad
	\tilde{f}(\ve{G}) = \frac{1}{V_c} \int_{V_c} \dd^3x\, e^{-i \ve{G} \cdot \ve{x}} f(\ve{x})
\end{equation}
Il volume della cella primitiva del reticolo reciproco è $ \tilde{V}_c = (\ve{b}_1 \times \ve{b}_2) \cdot \ve{b}_3 = \frac{(2\pi)^3}{V_c} $: questa viene detta \textit{prima zona di Brillouin} (BZ).

\begin{example}{Reticoli reciproci}{}
	Il reticolo reciproco di un fcc con cubo di lato $ a $ è un bcc con cubo di lato $ 4\pi/a $ e viceversa, mentre il reticolo reciproco di un sc con cubo di lato $ a $ è un sc con cubo di lato $ 2\pi/a $.
\end{example}

Ciascun $ \ve{G} $ nel reticolo reciproco definisce un'onda piana $ e^{i \ve{G} \cdot \ve{x}} $ (nello spazio diretto) con la stessa periodicità del reticolo diretto. Se si considerano i fronti d'onda fissati, per esempio, da $ e^{i \ve{G} \cdot \ve{x}} = 1 $, questi formano una famiglia di piani paralleli tra loro e perpendicolari a $ \ve{G} $ (che ne dà la direzione di propagazione), separati tra loro da una distanza $ \lambda = 2\pi / \abs{\ve{G}} $. \\
Alcuni di questi piani passano per punti del reticolo reciproco; in particolare, tutti questi piani passano per punti del reticolo reciproco se gli indici interi $ (\ell_1 \, \ell_2 \, \ell_3) $ che definiscono $ \ve{G} $ non hanno divisori non-banali comuni: questi vengono detti \textit{indici di Miller} della famiglia di piani considerata. Se invece $ \ve{G} $ è determinato da $ (n\ell_1 \, n\ell_2 \, n\ell_3) $, con $ n \in \Z - \{0\} $, allora soltanto un piano ogni $ n $ passerà per punti del reticolo diretto. \\
Si dimostra che gli indici di Miller sono proporzionali ai reciproci delle intercette dei piani con le direzioni dei vettori primitivi del reticolo diretto.










\end{document}
