\section{Induzione Magnetica}

\subsection{Legge di Induzione di Faraday}

Svolgendo esperimenti su coppie di circuiti, uno dei quali percorso da corrente stazionaria, Faraday si accorse della presenza di un transiente, nel circuito scollegato, in corrispondenza dell'accensione e dello spegnimento del generatore di corrente: la sua intuizione fu che questo transiente fosse legato alla variazione del flusso del campo magnetico, la quale genera un campo elettrico non conservativo la cui circuitazione è una differenza di potenziale (la forza elettromotrice) lungo la spira:
\begin{equation}
	\frac{d\Phi_B}{dt} = -\mathcal{E}
	\label{eq:1}
\end{equation}

\subsubsection{Primo esperimento di Faraday}

Nel suo primo esperimento, Farady tenne ferma la bobina in cui scorre corrente e mosse la spira scollegata. \\ 
%
Consideriamo una spira quadrata di lato $ l $ parallela al piano $ xy $ e vincoliamola a muoversi lungo l'asse $ y $; consideriamo anche un campo magnetico non uniforme ortogonale alla spira, ovvero lungo l'asse $ z $: supponiamo che esso vari secondo $ \vec{B}(y) = B_0 \frac{\abs{y}}{L} \,\hat{e}_z $, con $ L $ lunghezza caratteristica del sistema. \\ 
%
Consideriamo un moto della spira dato da $ y(t) = vt $ e calcoliamo la f.e.m. dalla legge di Faraday:
\begin{equation}
	\begin{split}
		\E(t) &= -\frac{d}{dt} \oiint_{\text{spira}} \vec{B}\cdot d\vec{S} = -\frac{d}{dt} \int_{y(t)}^{y(t)+l} B(y')l\,dy' = -B_0 \frac{l}{L} \frac{d}{dt} \int_{vt}^{vt+l} y'dy' \\ 
		      &= -B_0 \frac{l}{2L} \frac{d}{dt} (2vt + l^2) = - B_0 \frac{l}{L} lv \qquad\qquad \vec{B}_1 \equiv \vec{B}(y + l), \, \vec{B}_3 \equiv \vec{B}(y) \\ 
		      &= (B_3 - B_1) lv
	\end{split}
	\label{eq:2}
\end{equation}
Si può analizzare l'esperimento anche con la forza di Lorentz, non conosciuta ai tempi di Faraday. \\ 
%
Sui lati della spira paralleli al moto (detti $ 2 $ e $ 4 $) si sviluppano forze ortogonali alla spira: se supponiamo la spira di spessore infinitesimo, ciò implica che le cariche non possono muoversi se non per un breve tratto, così che la forza di Lorentz viene immediatamente bilanciata dalla resistenza meccanica della struttura cristallina del conduttore e dai vincoli a cui è sottoposta la spira. D'altro canto, sui lati $ 1 $ e $ 3 $ la forza mette in moto le cariche, che si vanno a disporre lungo i lati $ 2 $ e $ 4 $ generando un campo elettrico che si oppone al moto delle cariche stesse: dato che il campo magnetico non è uniforme, le forze variano nel tempo e causano una continua ridistribuzione delle cariche. \\ 
%
La circuitazione di $ \vec{F} $ lungo la spira vale:
\begin{equation}
	\begin{split}
		\Gamma(\vec{F}) &= \oint_{\text{spira}} \vec{F}\cdot d\vec{l} = \int_1 \vec{F}\cdot d\vec{l} + \int_2 \vec{F}\cdot d\vec{l} + \int_3 \vec{F}\cdot d\vec{l} + \int_4 \vec{F}\cdot d\vec{l} \\ 
				&= \int_1 q (\vprod{v}{B})\cdot d\vec{l} + \int_3 q (\vprod{v}{B})\cdot d\vec{l} = qlv(B_3 - B_1)
	\end{split}
	\label{eq:3}
\end{equation}
Dato che a questo lavoro è associata una d.d.p. $ \E = \frac{1}{q} \Gamma(\vec{F}) $, otteniamo la stessa f.e.m. indotta \ref{eq:2}.

\subsubsection{Secondo esperimento di Faraday}

Nel suo secondo esperimento, Faraday tenne ferma la spira scollegata e mosse la bobina percorsa da corrente: nel sistema di riferimento $ \mathcal{S}' $ del laboratorio la spira è ferma, per cui non può agire la forza di Lorentz, quindi il meccanismo che genera la f.e.m. deve essere diverso da prima. \\ 
%
Supponendo che la velocità relativa sia direzionata lungo l'asse che congiunge i centri della spira e della bobina e che essa sia $ v \ll c $, nel sistema $ \mathcal{S} $ solidale alla bobina si ha $ \vec{B} = \gamma^{-1} \vec{B}' \approx \vec{B}' $, mentre in $ \mathcal{S}' $ vi è un campo elettrico $ \vec{E}' = \vprod{v}{B}' \approx \vprod{v}{B} $: quest'ultimo non è conservativo e la sua circuitazione lungo la spira vale $ \Gamma(\vec{E}) = l (E'_3 - E'_1) = lv (B'_3 - B'_1) $, alla quale è associata la f.e.m. $ \E = lv (B_3 - B_1) $. \\ 
%
Si noti che questa trattazione è di natura relativistica, ma vale solo per $ v \ll c $, poiché si assume che il valore dei campi ai due lati della spira sia calcolato simultaneamente nello stesso istante: la legge di Faraday non è un invariante relativistico. \\ 
%
Calcoliamo ora il flusso del campo magnetico, in questo caso senza assumere una sua particolare forma funzionale (e omettendo gli apici del sistema di riferimento per semplicità): il flusso infinitesimo è dato da $ d\Phi_B = B(y) \,dS = B(y) l \,dy = B(y) lv \,dt $, quindi si vede subito che:
\begin{equation}
	\frac{d\Phi_B}{dt} = lv (B_1 - B_3) = -\E
	\label{eq:4}
\end{equation}

\subsubsection{Spira con lato mobile}

Consideriamo ora una spira immersa in un campo magnetico uniforme $ \vec{B} = B \,\hat{e}_z $ ortogonale alla spira e con un lato che può scorrere sugli altri due: se lo mettiamo in moto con velocità $ \vec{v} = v \,\hat{e}_x $, il flusso magnetico concatenato alla spira varia nel tempo, anche se il campo è uniforme. \\ 
%
Nel lato mobile, le cariche sono sottoposte alla forza di Lorentz $ \vec{F} = -qvB \,\hat{e}_y $, dunque nella spira agisce una f.e.m. $ \E = \frac{1}{q} \oint \vec{F}\cdot d\vec{l} $: l'integrale è non-nullo solo lungo il lato in moto e vale $ \E = -lvB $ ($ \vec{F} $ e $ d\vec{l} $ sono discordi). D'altra parte, si ha $ \Phi_B(t) = lB x(t) = lB v t $, quindi si vede che la legge di Faraday è rispettata. \\ 
%
Data la f.e.m., si instaura necessariamente una corrente $ i = \E / R $ nel lato in moto (assumendolo ohmico), quindi vi è una dissipazione di energia $ P = \E^2 / R $ per effetto Joule: dato che la forza magnetica non compie lavoro, il lavoro necessario a questa dissipazione è fornito dall'esterno (ad esempio da noi) al sistema per mantenere un moto a velocità costante, in quanto il moto delle cariche è dissipativo. \\ 
%
Dato che la forza di Lorentz imprime una velocità lungo la barretta alle cariche, esse hanno una velocità totale $ \vec{u} $ in direzione diagonale rispetto alla barretta, scomponibile nella direzione parallela $ u_x = v $ e ed in quella ortogonale al moto $ u_y $. La forza effettiva agente sulle cariche $ \vec{F}_m $ è ortogonale ad $ \vec{u} $ e quindi non compie lavoro: $ F_{m,y} $ determina il moto delle cariche nella barretta, mentre $ F_{m,x} $ si oppone alla forza esterna che mantiene in moto la barretta. Questa forza effettiva deve essere bilanciata da una forza $ -\vec{F}_m $ poiché il sistema è in condizioni stazionarie: essa si scompone lungo $ y $ (parallelo alla barretta) in $ F_R $, corrispondente alla resistenza della struttura molecolare al moto delle cariche, ed in quella $ x $ in $ F_{\text{ext}} $, ovvero la forza esterna che mantiene la barretta in moto. \\ 
%
La potenza totale associata a $ \vec{F}_m $  è:
\begin{equation}
	P_m = P_{m,x} + P_{m,y} = -F_{m,x}u_x + F_{m,y}u_y = -qBu_yu_x + qBu_xu_y = 0
	\label{eq:5}
\end{equation}
come ci aspettavamo. \\ 
%
Possiamo quindi vedere che la forza magnetica non compie lavoro, ma fa da tramite fra la potenza spesa per applicare la forza $ \vec{F}_{\text{ext}} $ e la potenza dissipata per effetto Joule: l'energia spesa per mantenere la barretta in moto viene trasferita dalla forza magnetica alle cariche, le quali instaurano una corrente stazionaria e dissipano l'energia fornita tramite gli urti col reticolo cristallino del conduttore.

\subsubsection{Arbitrarietà delle superfici}

La trattazione svolta finora è stata fatta solo su circuiti piani e le relative superfici piane, ma possiamo dimostrare che il flusso magnetico non dipende dalla superficie scelta, purché delimitata sempre dallo stesso circuito. \\ 
%
Consideriamo un circuito $ C $ e due superifici da esso delimitate $ S_1 $ ed $ S_2 $: definendo $ S \equiv S_1 \cup S_2 $, che è una superficie chiusa, si ha:
\begin{equation}
	\begin{split}
		\Phi_S (\vec{B}) &= \oiint_S \vec{B}\cdot d\vec{S} = \iiint_V \dive\vec{B} \,dV = 0 \\ 
				 &= \oiint_{S_1} \vec{B}\cdot d\vec{S}_1 - \oiint_{S_2} \vec{B}\cdot d\vec{S}_2
	\end{split}
	\label{eq:6}
\end{equation}
dove si è usato che $ \dive\vec{B} = 0 $ e che, considerando la superficie chiusa, $ d\vec{S}_2 $ da direzione opposta a quando si integra sulla superficie aperta. Di conseguenza:
\begin{equation}
	\oiint_{S_1} \vec{B}\cdot d\vec{S}_1 = \oiint_{S_2} \vec{B}\cdot d\vec{S}_2
	\label{eq:7}
\end{equation}
ovvero il flusso non dipende dalla particolare superficie scelta. \\ 
%
Un'importante conseguenza è che, dato che $ \dive(\rot\vec{B}) = 0 $ (identità vettoriale), si ha che anche la superficie considerata nel teorema di Stokes è indifferente, purché $ \partial S = \gamma $.

\subsubsection{F.e.m. da movimento}

Consideriamo un generico circuito di forma qualunque $ \gamma $ in moto con velocità $ \vec{v} $ in un campo magnetico statico ma non uniforme e siamo $ S_{1,2} $ e $ \gamma_{1,2} $ le superfici e i relativi bordi che delimitano il circuito ai tempi $ t $ e $ t + dt $. \\ 
%
Data l'invarianza del flusso rispetto alla superficie scelta, possiamo calcolare $ \Phi_B(t+dt) $ considerando come bordo $ S_1 \cup \Delta S $, dove $ \Delta S $ è la superficie che unisce $	\gamma_1 $ e $ \gamma_2 $: dato che $ d\vec{S}_{\Delta S} = (\vec{v}dt)\times d\vec{l}_1 $, si ha:
\begin{equation}
	\begin{split}
		\Phi_B(t+dt) &= \oiint_{S_1 \cup \Delta S} \vec{B}\cdot d\vec{S} = \Phi_B(t) + \oiint_{\Delta S} \vec{B}\cdot d\vec{S}_{\Delta S} = \Phi_B(t) + dt \oint_{\gamma_1} \vec{B}\cdot (\vec{v}\times d\vec{l}_1) \\ 
			     &\approx \Phi_B(t) + \frac{d\Phi_B}{dt} dt
	\end{split}
	\label{eq:8}
\end{equation}
Quindi:
\begin{equation}
	\frac{d\Phi_B}{dt} = \oint_{\gamma_1} \vec{B} \cdot (\vec{v}\times d\vec{l}_1) = - \oint_{\gamma_1} (\vec{v}\times\vec{B}) \cdot d\vec{l}_1 = -\frac{1}{q} \oint_{\gamma_1} \vec{F}_m \cdot d\vec{l}_1 = -\E
	\label{eq:9}
\end{equation}
che è proprio la legge di Faraday.

\subsubsection{Terzo esperimento di Faraday}

Nel suo terzo esperimento, Faraday tenne ferme sia la bobina che la spira, ma fece variare la corrente passante per la bobina, generando un campo magnetico variabile: in questo caso, il fenomeno non è spiegabile tramite un cambio di sistema di riferimento, quindi è necessaria una nuova legge fisica. \\ 
%
Consideriamo una spira $ \gamma $ ed una superficie $ S : \partial S = \gamma $ immersa in un campo di induzione magnetica $ \vec{B}(t,\vec{r}) $; la legge di Faraday può essere sviluppata come:
\begin{equation}
	\begin{split}
		\E &= \oint_{\gamma} \vec{E} \cdot d\vec{l} = \oiint_S \rot\vec{E} \cdot d\vec{S} \\ 
		   &= -\frac{d\Phi_B}{dt} = -\frac{d}{dt} \oiint_S \vec{B}\cdot d\vec{S} = - \oiint_S \frac{\partial\vec{B}}{\pa t} \cdot d\vec{S}
	\end{split}
	\label{eq:10}
\end{equation}
Dunque otteniamo la terza equazione di Maxwell:
\begin{equation}
	\rot\vec{E} = -\frac{\pa\vec{B}}{\pa t}
	\label{eq:11}
\end{equation}
Dato il potenziale vettore $ \vec{A} : \vec{B} = \rot\vec{A} $, si ha:
\begin{equation}
	\rot\vec{E} = -\frac{\pa}{\pa t} (\rot\vec{A}) \quad\Longrightarrow\quad \rot\left(\vec{E} + \frac{\pa\vec{A}}{\pa t}\right) = 0 \quad\Longrightarrow\quad \vec{E} = -\nabla\phi - \frac{\pa\vec{A}}{\pa t}
	\label{eq:12}
\end{equation}
ovvero:
\begin{equation}
	\vec{E} = -\nabla\phi - \frac{\pa\vec{A}}{\pa t}
	\label{eq:13}
\end{equation}
Quindi il campo elettrico può essere espresso in funzione di un potenziale scalare e di uno vettore. \\ 
%
Riprendiamo il caso del circuito di forma generica in movimento, ma supponiamo questa volta che il campo magnetico sia anche variabile nel tempo:
\begin{equation}
	\begin{split}
		\frac{\Delta\Phi_B}{\Delta t} &= \frac{1}{\Delta t} \left[ \oiint_{S(t+\Delta t)} \vec{B}(\vec{r},t+\Delta t) \cdot d\vec{S} - \oiint_{S(t)} \vec{B}(\vec{r},t)\cdot d\vec{S} \right] \\ 
					      &= \frac{1}{\Delta t} \left[ \oiint_{S(t)} \left( \vec{B}(\vec{r},t) + \frac{\pa\vec{B}(\vec{r},t)}{\pa t} \Delta t \right) \cdot d\vec{S} + \oiint_{\Delta S} \left( \vec{B}(\vec{r},t) + \frac{\pa\vec{B}(\vec{r},t)}{\pa t} \Delta t \right) \cdot d\vec{S} \right] + \\ 
					      & \qquad\qquad\qquad\qquad\qquad\qquad\qquad\qquad\qquad\qquad\qquad\qquad\qquad\qquad -\frac{1}{\Delta t} \oiint_{S(t)} \vec{B}(\vec{r},t)\cdot d\vec{S} \\ 
					      &= \frac{1}{\Delta t} \left[ \oiint_{S(t)} \left( \frac{\pa\vec{B}(\vec{r},t)}{\pa t} \Delta t \right) \cdot d\vec{S} + \oiint_{\Delta S} \left( \vec{B}(\vec{r},t) + \frac{\pa\vec{B}(\vec{r},t)}{\pa t} \Delta t \right) \cdot \left((\vec{v}\Delta t)\times d\vec{l}\right) \right] \\ 
					      &= \oiint_{S(t)} \frac{\pa\vec{B}(\vec{r},t)}{\pa t} \cdot d\vec{S} + \oint_{\pa S(t)} \vec{B}(\vec{r},t)\cdot (\vec{v}\times d\vec{l})
	\end{split}
	\label{eq:14}
\end{equation}
dove abbiamo ignorato un termine proporzionale a $ \Delta t^2 $; passando al limite, otteniamo:
\begin{equation}
	\frac{d\Phi_B}{dt} = \oiint_{S(t)} \frac{\pa\vec{B}(\vec{r},t)}{\pa t} \cdot d\vec{S} - \oint_{\gamma(t)} \left(\vprod{v}{B}(\vec{r},t)\right)\cdot d\vec{l}
	\label{eq:15}
\end{equation}
Notiamo quindi due contributi alla f.e.m.: il primo termine, detto f.e.m. da induzione, associato alla f.e.m. indotta dal campo elettrico generato da un campo magnetico variabile nel tempo, ed il secondo, detto f.e.m. da movimento, legato al campo elettromotore $ \vprod{v}{B} $.

\subsection{Legge di Lenz}

Il segno negativo nella legge di Faraday non è una pura convenzione matematica, ma ha un significato fisico profondo, legato alla conservazione dell'energia. \\ 
%
Consideriamo una spira immersa in un campo magnetico uniforme $ \vec{B} $ e supponiamo di indurvi una variazione $ d\vec{B}_1 $ concorde a $ \vec{B} $: la corrispondente variazione di flusso magnetico genera una f.e.m. ed una corrente $ i_1 $; se invece induciamo una variazione $ d\vec{B}_2 $ opposta a $ \vec{B} $, avremo una f.e.m. indotta ed una corrente $ i_2 $ opposta ad $ i_1 $: la legge di Lenz afferma che la f.e.m. indotta ha segno tale per cui la corrente indotta genera un campo magnetico indotto che si oppone alla variazione di flusso magnetico. \\ 
%
Se riprendiamo il caso della spira con un lato mobile, vediamo che la variazione del flusso genera una corrente indotta $ i_{\text{ind}} $ che scorre in modo da generare un campo magnetico indotto $ \vec{B}_1 $ opposto a $ \vec{B} $, il cui flusso varia col moto della barretta: ciò genera un'ulteriore corrente indotta $ i_{\text{ind},1} $, opposta a $ i_{\text{int}} $, che causa una forza opposta al verso del moto: la legge di Lenz quindi assicura che la f.e.m. indotta si opponga alla variazione di flusso e rallenti la barretta; in caso contrario, si avrebbe una continua accelerazione della barretta ed una conseguente violazione della conservazione dell'energia. \\ 
%
Possiamo anche notare che, per il campo elettrico indotto, valgono delle equazioni formalmente identiche a quelle del campo magnetico: infatti esso è un campo non conservativo generato da una variazione del campo magnetico e non da cariche, quindi:
\begin{equation}
	\rot\vec{E}_{\text{ind}} = -\frac{\pa\vec{B}}{\pa t} \qquad\qquad \dive\vec{E}_{\text{ind}} = 0
	\label{eq:16}
\end{equation}
Per il teorema di Helmholtz, se $ \vec{E}_{\text{ind}} $ si annulla abbastanza velocemente all'infinito, allora la soluzione ha una forma completamente definita.

\subsubsection{Auto-induttanza e mutua induttanza}

Consideriamo una spira percorsa da corrente stazionaria: con la legge di Biot-Savart, possiamo calcolare il flusso magnetico attraverso di essa:
\begin{equation}
	\Phi_B = \oiint_S \vec{B}(\vec{r}) \cdot d\vec{S} = \oiint_S \left[ \frac{\mu_0 i}{4\pi} \oint_{\gamma} \frac{d\vec{l} \times (\vec{r} - \vec{r}\,')}{\abs{\vec{r}-\vec{r}\,'}^3} \right]\cdot d\vec{S} \equiv Li
	\label{eq:17}
\end{equation}
dove abbiamo definito l'auto-induttanza della spira:
\begin{equation}
	L \equiv \frac{\mu_0}{4\pi} \oiint_S \left[ \oint_{\gamma} \frac{d\vec{l} \times (\vec{r} - \vec{r}\,')}{\abs{\vec{r}-\vec{r}\,'}^3} \right] \cdot d\vec{S}
	\label{eq:18}
\end{equation}
che dipende solo dalle caratteristiche geometriche del della spira. In generale, essa è molto difficile da calcolare analiticamente e viene misurata sperimentalmente: la sua u.d.m. è l'Henry $ \text{H} = \text{T}\cdot\text{m}^2\cdot\text{A}^{-1} $. \\ 
%
Tenendo conto che $ \vec{B} = \rot\vec{A} $ e $ \vec{A}(\vec{r}) = \frac{\mu_0}{4\pi} i \oint_{\gamma} \frac{d\vec{l}}{\abs{\vec{r} - \vec{r}\,'}} $, usando il teorema di Stokes otteniamo la formula di Neumann per l'auto-induttanza:
\begin{equation}
	L = \frac{\mu_0}{4\pi} \oint_{\gamma} \oint_{\gamma} \frac{d\vec{l}\cdot d\vec{l}\,'}{\abs{\vec{r} - \vec{r}\,'}}
	\label{eq:19}
\end{equation}
Questa diverge per fili di raggio tendente a zero, dunque è necessario tener conto del raggio finito del filo. \\ 
%
Consideriamo ora due spire; il flusso magnetico generato dalla spira $ 1 $ sulla spira $ 2 $ è dato da:
\begin{equation}
	\Phi_2 = \frac{\mu_0}{4\pi} \oint_{\gamma_2} \vec{A}_1 \cdot d\vec{l}_2 = \frac{\mu_0 i_1}{4\pi} \oint_{\gamma_2} \oint_{\gamma_1} \frac{d\vec{l}_1 \cdot d\vec{l}_2}{\abs{\vec{r}_1 - \vec{r}_2}}
	\label{eq:20}
\end{equation}
Troviamo quindi la formula di Neumann per la mutua induttanza:
\begin{equation}
	M_{21} \equiv \frac{\mu_0 }{4\pi} \oint_{\gamma_2} \oint_{\gamma_1} \frac{d\vec{l}_1 \cdot d\vec{l}_2}{\abs{\vec{r}_1 - \vec{r}_2}}
	\label{eq:21}
\end{equation}
Si vede chiaramente che:
\begin{equation}
	M_{12} = M_{21}
	\label{eq:22}
\end{equation}
Tenendo conto anche delle auto-induttanze $ L_1 \equiv M_{11} $ e $ L_2 \equiv M_{22} $, otteniamo:
\begin{equation}
	\begin{split}
		\Phi_1 &= M_{11} i_1 + M_{12} i_2 \\ 
		\Phi_2 &= M_{21} i_1 + M_{22} i_2
	\end{split}
	\qquad\Longleftrightarrow\qquad \vec{\Phi}_B = \underbar{M} \cdot \vec{i}
	\label{eq:23}
\end{equation}
dove l'elemento della matrice di induttanza è dato da:
\begin{equation}
	M_{jk} = \frac{\mu_0}{4\pi} \oint_{\gamma_j} \oint_{\gamma_k} \frac{d\vec{l}_k \cdot d\vec{l}_j}{\abs{\vec{r}_k - \vec{r}_j}}
	\label{eq:24}
\end{equation}
%
Se invece supponiamo che nella spira $ 2 $ inizialmente non scorra corrente, mentre nella spira $ 1 $ essa vari nel tempo, avremo una forza elettromotrice indotta nella spira $ 2 $ data da:
\begin{equation}
	\E_{21} = - M_{21}\frac{di_1}{dt}
	\label{eq:25}
\end{equation}
La variazione di $ i_1 $ nel tempo genera una f.e.m. indotta anche nella spira $ 1 $, data da:
\begin{equation}
	\E_{1} = -L_1 \frac{di_1}{dt}
	\label{eq:26}
\end{equation}
Possiamo trascurare gli effetti che la variazione della corrente nella spira $ 2 $, ed il conseguente campo magnetico indotto, avrebbero sulla spira $ 1 $ (generando una corrente indotta opposta ad $ i_1 $) supponendo che essi siano del second'ordine rispetto agli effetti dovuti ad $ i_1 $.

\subsubsection{Generatore di corrente alternata}

Consideriamo una spira quadrata di lato $ l $ parallela al piano $ xy $ ed immersa in un campo magnetico uniforme $ \vec{B} = B \,\hat{e}_z $; supponiamo inoltre che la spira possa ruotare attorno al proprio asse (ad esempio lungo l'asse $ x $) con velocità angolare $ \omega $. Se ad un certo tempo $ t $ la normale alla spira forma un angolo $ \alpha(t) = \omega t $ col campo magnetico, si avrà che il flusso magnetico attraverso la spira sarà:
\begin{equation}
	\Phi_B (t) = Bl^2 \cos\omega t
	\label{eq:27}
\end{equation}
Di conseguenza, per la legge di Faraday, viene generata una f.e.m.:
\begin{equation}
	\E = Bl^2 \sin\omega t
	\label{eq:28}
\end{equation}
Dato che, per la legge di Lenz, questa f.e.m. indotta genera un campo magnetico indotto che si oppone alla rotazione della spira, sarà necessario continuare a fornirle energia per mantenerla in rotazione: essa sarà parzialmente convertita in energia elettrica e parzialmente dissipata dagli attriti meccanici.

\subsubsection{Il freno elettromagnetico}

Quando un conduttore si muove in un campo magnetico si sviluppa una f.e.m. indotta e le relative correnti superficiali indotte che, per la legge di Lenz, generano un campo magnetico che si oppone al campo magnetico esterno. \\ 
%
Ad esempio, consideriamo un foglio metallico che si muove lentamente in una regione con campo magnetico (vedi fig.): in essa, la forza di Lorentz fa muovere le cariche ortogonalmente a $ \vec{v} $, creando delle correnti che si chiudono scorrendo anche al di fuori della regione del campo magnetico; queste correnti generano un campo magnetico che si oppone alla variazione del campo magnetico esterno ed al moto della lastra. \\ 
%
%
%
\hbox{}\\ INSERITE LA FIGURA\\ \hbox{}\\ 
%
%
%
Una variante del freno magnetico si ha prendendo un disco conduttore in rotazione al posto di un foglio metallico: questa variante è utilizzata nelle ruote dei treni ad alta velocità, mentre la versione lineare nelle giostre a caduta o nelle montagne russe. \\ 
%
Un'altra applicazione della legge di Lenz è la frenata rigenerativa dei veicoli elettrici: quando al motore elettrico viene tolta l'alimentazione (ad esempio sollevando l'acceleratore) la sua rotazione viene sfruttata per produrre energia elettrica che viene re-immagazzinata nella batteria ed il cui campo magnetico indotto agisce frenando i veicolo.

\subsection{Circuiti con Induttanze}

\subsubsection{Circuito RL}

Consideriamo un circuito con un solenoide collegato ad una f.e.m. $ \E_0 $: il conduttore che forma avrà una resistenza complessiva $ R $, quindi possiamo schematizzare il circuito come una resistenza $ R $ in serie con un'induttanza $ L $. \\ 
Se nel circuito circola una corrente $ i(t) $, il flusso concatenato al solenoide $ \Phi_B(t) = Li(t) $ genera una f.e.m. indotta $ \E_{\text{ind}} $ che causa una corrente indotta $ i_{\text{ind}} $ opposta ad $ i(t) $. Ridefinendo $ i(t) $ così da includere anche la corrente indotta, si ha che il circuito è descritto da:
\begin{equation}
	\E_0 - L \frac{di(t)}{dt} = Ri(t)
	\label{eq:29}
\end{equation}
Dal punto di vista qualitativo, si può vedere come per tempi lunghi il sistema raggiunga una condizione stazionaria con corrente $ i_0 = \E_0 / R $; inoltre, dato che appena dopo l'accensione $ i \approx 0 $, si ha $ i(t) \approx \frac{\E_0}{L} t $. \\ 
%
Risolvendo analiticamente si vede che:
\begin{equation}
	i(t) = \frac{\E_0}{R} (1 - e^{-t/\tau}), \quad \tau \equiv \frac{L}{R}
	\label{eq:30}
\end{equation}
dove $ \tau $ è detta costante di tempo ($ [\tau] = s $). \\ 
%
Supponiamo ora di togliere tensione alla bobina e che, prima di questa azione, nel solenoide circoli una corrente $ i_0 $: senza alimentazione, la corrente tende a diminuire nel tempo e ciò genera una f.e.m. indotta che si oppone alla diminuzione di corrente:
\begin{equation}
	-L \frac{di(t)}{dt} = Ri(t) \qquad\Longrightarrow\qquad i(t) = i_0 e^{-t/\tau}
	\label{eq:31}
\end{equation}
%
Un esempio importante di circuito RL è il trasformatore. \\ 
%
Consideriamo due solenoidi coassiali con $ n_1 $ ed $ n_2 $ spire, molto lunghi e di raggio praticamente uguale: il campo di induzione magnetica sarà contenuto completamente all'interno dei due solenoidi. \\ 
%
Colleghiamo un generatore di f.e.m. $ V_P(t) $ al solenoide $ 1 $ (detto $ \virgolette{primario} $): se definiamo $ \Phi_B $ come il flusso magnetico concatenato ad una singola spira, si ha $ \Phi_{1,2} = n_{1,2} \Phi_B $ e $ \E_{1,2} = - n_{1,2} \frac{d\Phi_B}{dt} $, ovvero:
\begin{equation}
	\frac{\E_1}{\E_2} = \frac{n_1}{n_2}
	\label{eq:32}
\end{equation}
che è l'equazione fondamentale del trasformatore. \\ 
%
Se assumiamo la resistenza dei solenoidi come molto piccola, allora $ \E_1 \approx V_P(t) $ e $ V_S(t) = \frac{n_2}{n_1}V_P(t) $.

\subsubsection{Energia del campo magnetico}

La potenza erogata da un circuito RC è data da $ P_{\E}(t) = \E_0 i(t) = \frac{\E_0^2}{R} (1 - e^{-t/\tau}) $ mentre quella dissipata è $ P_R(t) = R i^2(t) = \frac{\E_0^2}{R} (1 - e^{-t/\tau})^2 $: la differenza tra potenza erogata e potenza dissipata è trasferita al campo magnetico generato dal solenoide e vale $ P_L(t) = \frac{\E_0^2}{R} e^{-t/\tau} (1 - e^{-t/\tau}) $. Possiamo dunque calcolare l'energia magnetica:
\begin{equation}
	W = \int_0^{\infty} P_L(t) \, dt = \frac{\E_0^2}{R} \int_0^{\infty} (e^{-t/\tau} - e^{-2t/\tau}) \, dt = \frac{\E_0^2}{R}\frac{\tau}{2} = \frac{1}{2}L i_0^2
	\label{eq:33}
\end{equation}
%
Se invece consideriamo il caso della scarica di un circuito RC, è presente la sola potenza dissipata $ P_R(t) = Ri_0^2 e^{-2t/\tau} $, quindi:
\begin{equation}
	W = \int_0^{\infty} P_R(t) \, dt = Ri_0^2 \int_0^{\infty} e^{-2t/\tau} dt = R i_0^2 \frac{\tau}{2} = \frac{1}{2} L i_0^2
	\label{eq:34}
\end{equation}
Questa energia, immagazzinata nel campo magnetico, viene quindi $ \virgolette{reuperata} $ e dissipata dalla resistenza. \\ 
%
Possiamo quindi definire un'energia potenziale associata al campo magnetico del solenoide:
\begin{equation}
	U_M \equiv \frac{1}{2} Li^2
	\label{eq:35}
\end{equation}
%
Possiamo anche slegare l'energia magnetica dalla corrente circolante nel circuito: ricordando la definizione di flusso magnetico per un'induttanza generica:
\begin{equation}
	\begin{split}
		U_M &= \frac{1}{2} \Phi_B i = \frac{1}{2} i \oiint_S \vec{B} \cdot d\vec{S} = \frac{1}{2} i \oint_{\gamma} \vec{A} \cdot d\vec{l} = \frac{1}{2} \iiint_V \sprod{A}{J} \,dV = \frac{1}{2\mu_0} \iiint_V \vec{A}\cdot (\rot\vec{B}) \,dV \\ 
		    &= \frac{1}{2\mu_0} \left[ \iiint_V \vec{B}\cdot (\rot\vec{A}) \,dV - \iiint_V \dive (\vprod{A}{B}) \,dV \right] \\ 
		    &= \frac{1}{2\mu_0} \left[ \iiint_V B^2 dV - \oiint_S (\vprod{A}{B}) \cdot d\vec{S} \right]
	\end{split}
	\label{eq:36}
\end{equation}
Dato che al di fuori del volume della spira $ \vec{J} = 0 $, l'integrale di volume può essere esteso a tutto lo spazio, quindi il secondo integrale tende ad annullarsi poiché l'integrando è proporzionale a $ r^{-1} $; otteniamo quindi l'espressione finale:
\begin{equation}
	U_M = \frac{1}{2\mu_0} \iiint_V B^2 dV
	\label{eq:37}
\end{equation}
%
Definendo la densità di energia magnetica $ \rho_M = \frac{1}{2\mu_0}B^2 $, possiamo notare l'analogia col campo elettrico, per il quale si ha $ \rho_E = \frac{\epsilon_0}{2}E^2 $. \\ 
%
Si può anche trovare un'utile relazione tra auto-induttanza e mutua induttanza: se consideriamo $ N $ spire accoppiate con coefficienti di mutua induttanza $ M_{jk} $, il flusso attraverso la $ j $-esima spira sarà $ \Phi_j  = \sum_{k=1}^{N} M_{jk}i_k $, quindi, dato che l'energia magnetica di una spira è $ U_M = \frac{1}{2}\Phi i $, si ha che l'energia magnetica totale sarà:
\begin{equation}
	U_M = \frac{1}{2} \displaystyle\sum_{j,k=1}^{N} M_{jk} i_j i_k
	\label{eq:38}
\end{equation}
Nel caso semplice di due spire, definendo $ M \equiv M_{12} = M_{21} $:
\begin{equation}
	U_M = \frac{1}{2} L_1 i_1^2 + M i_1 i_2 + \frac{1}{2} L_2 i_2^2
	\label{eq:39}
\end{equation}
Definendo $ x \equiv i_1 / i_2 $ si ha $ U_M = \frac{1}{2} i_2^2 (L_1 x^2 + 2Mx + L_2) $, che ha un minimo in $ x_m = -\frac{M}{L_1} $; dato che deve essere sempre $ U_M \ge 0 $, la condizione in $ x_m $ implica:
\begin{equation}
	M \le \sqrt{L_1 L_2}
	\label{eq:40}
\end{equation}
%
In generale è quindi possibile scrivere $ M = k \sqrt{L_1 L_2} $, con $ k \le 1 $, e ciò esprime il fatto che non tutte le linee di campo di una spira sono concatenate all'altra: più $ k $ è vicino ad $ 1 $, più sono le linee di campo concatenate. \\ 
Nel caso del trasformatore le linee di campo sono praticamente sovrapposte, dunque $ k \approx 1 $ e $ M \approx \sqrt{L_1 L_2} $.

\subsubsection{Circuito LC}

Consideriamo un circuito composto da un capacitore $ C $ ed un'induttanza $ L $, assumendo che la resistenza dei conduttori sia trascurabile. Supponiamo che a $ t = 0 $ il condensatore sia carico e che non scorra corrente: $ q(0) = q_{\text{max}} \equiv C V_{\text{max}} $, $ i(0) = 0 $. \\ 
%
Definendo $ V_C $ e $ V_L $ le cadute di potenziale ai capi del condensatore e dell'induttanza, per la seconda legge di Kirchhoff $ V_C + V_L = 0 $, ovvero:
\begin{equation}
	L \frac{di}{dt} + \frac{q}{C} = 0 \qquad\Longrightarrow\qquad \frac{d^2 i}{dt^2} = -\frac{1}{LC} i
	\label{eq:41}
\end{equation}
Definendo la frequenza $ \omega_0 = 1 / \sqrt{LC} $, la soluzione è $ i(t) = i_{\text{max}} \sin(\omega_0 t + \phi) $. La condizione $ i(0) = 0 $ implica $ \phi = 0 $, mentre, dato che $ V(t) = L \omega_0 i_{\text{max}} \cos\omega_0 t $, la condizione $ V(0) = V_{\text{max}} = q_{\text{max}} / C $ implica $ i_{\text{max}} = q_{\text{max}}\omega_0 $. \\ 
%
Ridefinendo $ q_0 \equiv q_{\text{max}} $, si ha:
\begin{equation}
	i(t) = q_0 \omega_0 \sin\omega_0 t
	\label{eq:42}
\end{equation}
Ricordando che $ U_E = \frac{1}{2} CV^2 $ e $ U_M = \frac{1}{2} Li^2 $, possiamo descrivere l'oscillazione del circuito LC: nell'istante iniziale il condensatore è completamente carico e tutta l'energia è immagazzinata nel campo elettrico; successivamente, inizia a scorrere corrente finché il condensatore non si scarica e tutta l'energia viene trasferita al campo magnetico, ma la corrente continua a scorrere e arriva a caricare di nuovo completamente il condensatore ma con segni delle cariche opposti, ritornando ad immagazzinare tutta l'energia nel campo elettrico; il semi-ciclo a questo punto si ripete ma con verso della corrente e dei campi invertito, ed al suo completamente si chiude il ciclo tornando nella condizione iniziale.

\subsubsection{Circuito RLC}

Consideriamo ora un circuito LC in cui però viene inserita anche una resistenza: ciò porterà ad una dissipazione di energia per effetto Joule, smorzando le oscillazioni. \\ 
Sempre per la seconda legge di Kirchhoff, si ha:
\begin{equation}
	L \frac{di}{dt} + \frac{q}{C} + Ri = 0 \qquad\Longrightarrow\qquad L \frac{d^2 i}{dt^2} + R \frac{di}{dt} + \frac{1}{C} i = 0
	\label{eq:43}
\end{equation}
Cerchiamo una soluzione del tipo $ i(t) = A e^{-\alpha t} \sin\omega t $:
\begin{equation}
	A e^{-\alpha t} \left[ \left( \frac{1}{C} - R\alpha + L(\alpha^2 - \omega^2)\right) \sin\omega t + \omega (R - 2L\alpha) \cos\omega t \right] = 0
	\label{eq:44}
\end{equation}
Affinché questa equazione sia soddisfatta, i coefficienti di $ \sin\omega t $ e $ \cos\omega t $ devono essere identicamente nulli:
\begin{equation}
	R - 2L\alpha = 0 \quad\Rightarrow\quad \alpha = \frac{R}{2L} \qquad \frac{1}{C} - R\alpha + L (\alpha^2 - \omega^2) = 0 \quad\Rightarrow\quad \omega^2 = \frac{1}{LC} - \frac{R^2}{4L^2}
	\label{eq:45}
\end{equation}
Una soluzione oscillatoria richiede $ \omega $ reale, pertanto, definendo $ \omega_0 \equiv 1 / \sqrt{LC} $, si ha la condizione sul valore della resistenza $ \omega_0 > R / 2L $: la presenza di una resistenza, oltre a smorzare l'oscillazione, attenua anche la sua frequenza.

\subsection{Legge di Ampère - Maxwell}

Se consideriamo il caso del circuito LC, in esso non è mai presente una corrente stazionaria o una carica costante nel tempo sulle armature del condensatore: dall'equazione di continuità $ \pa_t \rho = - \dive\vec{J} $, si ha $ \dive\vec{J} \neq 0 $; d'altro canto, dall'equazione di Ampère $ \rot\vec{B} = \mu_0 \vec{J} $ si ha $ \mu_0 \dive\vec{J} = \dive(\rot\vec{B}) = 0 $ (identità vettoriale $ \dive\rot = 0 $): questa apparente inconsistenza deriva dal fatto che all'equazione di Ampère manca un termine per tenere conto di correnti variabili. \\ 
%
Ci aspettiamo di trovare un problema analogo in un circuito RC: se consideriamo una superficie $ S : \pa S = \gamma $ che taglia il filo che collega il capacitore al resistore, avremo:
\begin{equation}
	\oint_{\gamma} \vec{B} \cdot d\vec{l} = \oiint_S (\rot\vec{B}) \cdot d\vec{S} = \mu_0 \oiint_S \vec{J}\cdot d\vec{S} \quad\Longrightarrow\quad \mu_0 i = \mu_0 J \pi^2 a
	\label{eq:46}
\end{equation}
dove $ a $ è la sezione del filo: in questo caso non sembra esserci nessuna incongruenza. \\ 
%
Se invece prendiamo una superficie $ S' $ delimitata dallo stesso contorno $ \gamma $, ma che stavolta, al posto di tagliare il filo, si chiude fra le armature del condensatore: dato che tra di esse non scorre corrente, $ S' $  non è attraversata da nessuna corrente, dunque $ \oiint_{S'} (\rot\vec{B}) \cdot d\vec{S}' = 0 $; d'altra parte, $ \gamma $ è comunque concatenato al filo, dunque $ \oint_{\gamma} \vec{B} \cdot d\vec{l} = \mu_0 i \neq 0 $, in apparente contraddizione col teorema di Stokes. \\ 
%
Per capire quale sia il termine mancante, consideriamo l'equazione di continuità e la legge di Gauss per il campo elettrico:
\begin{equation}
	\dive\vec{J} = - \frac{\pa\rho}{\pa t} = - \epsilon_0 \frac{\pa}{\pa t} \dive\vec{E} \qquad \Longrightarrow \qquad \dive \left(\vec{J} + \epsilon_0 \frac{\pa\vec{E}}{\pa t}\right) = 0
	\label{eq:47}
\end{equation}
Il termine che viene aggiunto alla densità di corrente è detto corrente di spostamento: non è una vera e propria corrente, nel senso che non corrisponde al movimento di cariche nel tempo, ma descrive il fatto che la presenza di un campo elettrico variabile genera un campo magnetico, a sua volta variabile. \\ 
Possiamo quindi scrivere la forma generale dell'equazione di Ampère, detta equazione di Ampère-Maxwell:
\begin{equation}
	\rot\vec{B} = \mu_0 \vec{J} + \frac{1}{c^2} \frac{\pa\vec{E}}{\pa t}
	\label{eq:48}
\end{equation}
%
Se applichiamo questa legge alla superficie $ S' $ considerata in precedenza, troviamo:
\begin{equation}
	\oiint_{S'} (\rot\vec{B}) \cdot d\vec{S}' = \mu_0 \epsilon_0 \oiint_{S'} \frac{\pa\vec{E}}{\pa t} \cdot d\vec{S}' = \mu_0 \epsilon_0 \oiint_{S'} \frac{1}{\epsilon_0} \frac{d\sigma}{dt} dS' = \mu_0 \frac{d\sigma}{dt} S' = \mu_0 \frac{dq}{dt} = \mu_0 i
	\label{eq:49}
\end{equation}
poiché tra le armature del condensatore $ E(t) = \frac{\sigma(t)}{\epsilon_0} $. Vediamo quindi che, contando anche la corrente di spostamento, non vi è più alcuna inconsistenza.
