\selectlanguage{english}

At early times, the thermodynamical properties of the universe were determined by local equilibrium. However, non-equilibrium dynamics is crucial for massive particles to acquire cosmological abundances and to understand the CMB.

\section{Hot Big Bang}
\label{sec:hot-big-bang}

To understand the thermal history of the universe, the key is to compare the \bctxt{rate of interactions} $ \Gamma $ with the \bctxt{rate of expansion} $ H $ (Hubble parameter). In particular, when $ \Gamma \gg H $, the timescale of particle interactions is much smaller that the expansion timescale:
\begin{equation}
  t_\Gamma \equiv \frac{1}{\Gamma} \ll \frac{1}{H} \equiv t_H
  \label{eq:thermal-eq-cond}
\end{equation}
Therefore, local thermal equilibrium is reached before the effect of the expansion becomes relevant. As the universe cools, however, the rate of interactions typically decreases faster than the expansion rate, and as $ t_\Gamma \sim t_H $ the particles decouple from the thermal bath. Note that different species have different interaction rates and decouple at different times.

Consider a generic $ 1 + 2 \lra 3 + 4 $ scattering process. In principle, each species has its own rate of interactions, e.g. $ \Gamma_1 = n_2 \sigma v_{1,2} $, where $ n_2 $ is the number density of the target species $ 2 $ and $ v_{1,2} $ is the average relative velocity between species $ 1 $ and $ 2 $, but at high energies it is expected that $ n_1 \sim n_2 \equiv n $, hence it is possible to define a single rate of interactions:
\begin{equation}
  \Gamma \equiv n \sigma v
  \label{eq:int-rate-def}
\end{equation}
where $ n $ is the number density of particles, $ v $ is their average velocity and $ \sigma $ is the interaction cross-section. For $ T \simeq 100 \gev $, all known particles are ultra-relativistic, so $ v \sim 1 $, and their masses can be ignored, leaving the temperature $ T $ as the only dimensionful scale: by dimensional analysis $ n \sim T^3 $. Moreover, interactions are mediated by gauge bosons, which are massless above the scale of electroweak symmetry breaking, leaving a similar dependence for the cross-section of strong and electroweak processes:
\begin{equation*}
  \sigma \sim \Bigg|\hspace{-0.4em}
    \begin{tikzpicture}[baseline = (r.base)]
      \begin{feynman}[inline = (r.base)]
        \vertex[dot] (v1) {};
        \vertex[dot] (v2) at ($ (v1) + (1,0) $) {};

        \vertex (a) at ($ (v1) + (-0.9,0.7) $) {};
        \vertex (b) at ($ (v1) + (-0.9,-0.7) $) {};
        \vertex (c) at ($ (v2) + (0.9,0.7) $) {};
        \vertex (d) at ($ (v2) + (0.9,-0.7) $) {};

        \vertex[below = 0.25em of v1] (r) {};

        \diagram* {
          (a) -- (v1) -- (b),
          (c) -- (v2) -- (d),
          (v1) -- [photon] (v2),
        };
      \end{feynman}
    \end{tikzpicture}
  \hspace{-0.4em}\Bigg|^2
  \sim \frac{\alpha^2}{T^2}
\end{equation*}
where $ \alpha \equiv g^2 / 4\pi $ is the generalized structure constant of the considered interaction. Then, dimensional analysis results into:
\begin{equation*}
  \Gamma = n \sigma v \sim T^3 \cdot \frac{\alpha^2}{T^2} \cdot 1 \sim \alpha^2 T
\end{equation*}
Now, recall \eref{eq:fried-hubble}, and in particular express it in terms of the reduced Planck mass:
\begin{equation*}
  \pl \equiv \sqrt{\frac{\hbar c}{8\pi G}} \simeq 2.4 \cdot 10^{18} \gev
  \quad \implies \quad
  H = \frac{\sqrt{\rho}}{3\pl}
\end{equation*}
The same argument as above gives $ \rho \sim T^4 $, thus $ H \sim T^2 / \pl $ and:
\begin{equation}
  \frac{\Gamma}{H} \sim \frac{\alpha^2 \pl}{T} \sim \frac{10^{16} \gev}{T}
\end{equation}
where $ \alpha \sim 0.01 $ has been used as numerical estimate. Therefore, the condition \eref{eq:thermal-eq-cond} is satisfied for $ 100 \gev \lesssim T \lesssim 10^{16} \gev $: in this range, all SM particles are in thermal equilibrium. This means that they exchange energy and momentum efficiently, thus reaching a state of maximum entropy: in this state, the number of particles per unit volume in phase space, i.e. the \emph{distribution function}, takes the form of the Bose--Einstein or the Fermi--Dirac distribution, depending on the statistics of the considered species.

On the contrary, when temperature drops below the mass of a particle species, that species becomes sub-relativistic and its distribution function acquires an exponential suppression of the form $ \sim e^{-m / T} $: this means that relativistic particles dominate the density and pressure of the primordial plasma. Furthermore, another consequence is that, had equilibrium persisted until today, the universe would be mostly photons, since any massive particle would be exponentially suppressed: this is not the case, due to the \bctxt{freeze-out} of massive particles caused by deviations from equilibrium (see \figref{fig:freeze-out} and \exref{ex:2-2-scattering}).

\begin{figure}
  \centering
  \includegraphics[width = 0.70 \textwidth]{freeze-out.png}
  \caption{Schematic illustration of freeze-out: at high temperature ($ T \gg m $) the particle abundance tracks its equilibrium value, while at low temperature ($ T \ll m $) particles freeze-out and maintain a density that is much larger than the Boltzmann-suppressed equilibrium abundance.}
  \label{fig:freeze-out}
\end{figure}

Below the scale of electroweak symmetry breaking, i.e. $ T \lesssim 100 \gev $, the gauge bosons of weak interactions receive masses through the Higgs mechanism ($ M_W \approx 80 \gev $ and $ M_Z \approx 90 \gev $), and the cross-section of weak processes becomes:
\begin{equation*}
  \sigma \sim \Bigg|\hspace{-0.4em}
    \begin{tikzpicture}[baseline = (r.base)]
      \begin{feynman}[inline = (r.base)]
        \vertex[dot] (v) {};

        \vertex (a) at ($ (v) + (-0.9,0.7) $) {};
        \vertex (b) at ($ (v) + (-0.9,-0.7) $) {};
        \vertex (c) at ($ (v) + (0.9,0.7) $) {};
        \vertex (d) at ($ (v) + (0.9,-0.7) $) {};

        \vertex[below = 0.25em of v1] (r) {};

        \diagram* {
          (a) -- (v) -- (b),
          (c) -- (v) -- (d),
        };
      \end{feynman}
    \end{tikzpicture}
  \hspace{-0.4em}\Bigg|^2
  \sim G_\text{F}^2 T^2
  \qquad \qquad
  G_\text{F} \sim \frac{\alpha}{M_W^2} \simeq 1.17 \cdot 10^{-5} \gev^{-2}
\end{equation*}
The strength of weak interactions now decreases as the temperature of the universe drops, and the ration $ \Gamma / H $ becomes:
\begin{equation}
  \frac{\Gamma}{H} \sim \frac{\alpha^2 \pl T^3}{M_W^4} \sim \left( \frac{T}{1 \mev} \right)^3
\end{equation}
The \bctxt{decoupling} of particles which interact with the primordial plasma only through weak interactions then happens at $ T_\text{dec} \sim 1 \mev $.

\newpage
\section{Equilibrium}

Observational evidence (mainly the perfect blackbody spectrum of the CMB) suggests that the early universe was in \emph{local thermal equilibrium}\footnotemark, a fact supported by the above analysis of the Standard Model at $ T \gtrsim 100 \gev $.

\footnotetext{Note that the universe can never be in perfect equilibrium, strictly speaking, since the FLRW metric does not have a time-like Killing vector. However, if the expansion is slow enough, particles have enough time to settle close to local equilibrium, and, since the universe is homogeneous, local values of thermodynamic quantities are also global values.}

\subsection{Equilibrium thermodynamics}

Consider a gas of weakly interacting particles. Quantistically, the momentum eigenstates of a particle in a volume $ V $ have a discrete spectrum: in particular, the density of states in momentum space $ \{\ve{p}\} $ is $ V / h^3 $, and the density of states in phase space $ \{\ve{x} , \ve{p}\} $ is $ 1 / h^3 $.

Considering a particle with $ g $ internal degrees of freedom (e.g. spin), in natural unit the density of states (which from now on is implicitly assumed to be in phase space) is $ g / (2\pi)^3 $. In order to obtain the number density, it is necessary to know how the particles are distributed amongst the momentum eigenstates: this information is encoded in the \emph{(phase space) distribution function} $ f(\ve{x} , \ve{p} , t) $. Homogeneity restricts $ f $ to be independent of $ \ve{x} $, while isotropy implies that the momentum-dependence is only through the magnitude $ p \equiv \norm{\ve{p}} $. The time dependence is typically left implicit, as it manifests itself in terms of the temperature dependence of the distribution function. Putting everything together, the number density of particles in real space is found to be:
\begin{equation}
  n(T) = \frac{g}{(2\pi)^3} \int \dd^3 p \, f(p,T)
  \label{eq:num-dens-def}
\end{equation}
The energy density is instead found by weighting the integration over momentum eigenstates with their energy. In particular, the assumption of the particles being weakly-interacting allows to approximate the energy as $ E(p) = \sqrt{m^2 + p^2} $, so that:
\begin{equation}
  \rho(T) = \frac{g}{(2\pi)^3} \int \dd^3 p \, f(p,T) E(p)
  \label{eq:en-dens-def}
\end{equation}
Pressure, on the other hand, requires more care.

\begin{proposition}[before upper = {\tcbtitle}]{Pressure of a weakly-interacting gas}{}
  \begin{equation}
    P = \frac{g}{(2\pi)^3} \int \dd^3 p \, f(p,T) \frac{p^2}{3E}
  \label{eq:press-def}
  \end{equation}
\end{proposition}

\begin{proofbox}
  \begin{proof}
    Consider a small area element $ \dd A $ with normal versor $ \hat{\ve{n}} $. All particles with velocity $ v \equiv \norm{\ve{v}} $, striking this area element between $ t $ and $ t + \dd t $, were located at $ t = 0 $ on a spherical shell of radius $ R = v t $ and width $ v \, \dd t $, hence a solid angle $ \dd \Omega $ of this shell defines a volume $ \dd V = R^2 v \, \dd t \dd \Omega $. Given \eref{eq:num-dens-def}, the number of particles in this volume is:
    \begin{equation*}
      \dd N = \frac{g}{(2\pi)^3} f(E,T) R^2 v \, \dd t \dd \Omega
    \end{equation*}
    Not all particles in $ \dd V $ reach the target, only those with velocities directed to the area element do. Taking into account the isotropy of the velocity distribution, the total number of these striking particles is:
    \begin{equation*}
      \dd N_A = \frac{\abs{\hat{\ve{v}} \cdot \hat{\ve{n}}} \dd A}{4 \pi R^2} \dd N = \frac{g}{(2\pi)^3} f(E,T) \frac{\abs{\hat{\ve{v}} \cdot \hat{\ve{n}}}}{4 \pi} \dd A \dd t \dd \Omega
    \end{equation*}
    where $ \hat{\ve{v}} \cdot \hat{\ve{n}} < 0 $ since the striking particles are directed towards the area element. Assuming elastic reflections, each particles transfers a momentum $ 2 \abs{\ve{p} \cdot \hat{\ve{n}}} $ to the target, hence the contribution of particles with velocity $ v $ to the pressure is:
    \begin{equation*}
      P(v) = \int \dd N_A \frac{2 \abs{\ve{p} \cdot \hat{\ve{n}}}}{\dd A \dd t} = \frac{g}{(2\pi)^3} f(E,T) \frac{p^2}{2\pi E} \int_0^{2\pi} \dd \varphi \int_{-1}^0 \dd \cos \theta \, \cos^2 \theta = \frac{g}{(2\pi)^3} f(E,T) \frac{p^2}{3E}
    \end{equation*}
    where $ v = p / E $ was used, and $ \theta $ was defined as $ \hat{\ve{v}} \cdot \hat{\ve{n}} = - \cos \theta < 0 $. Integrating over energy or momentum yields the thesis.
  \end{proof}
\end{proofbox}

The form of the distribution function depends on the thermodynamical state of the system. Assuming that the gas is in \emph{kinetic equilibrium}, i.e. if the particles can exchange energy and momentum efficiently, then the system is in a state of maximal entropy and the distribution function is either the Fermi--Dirac distribution (fermions) or the Bose--Einstein distribution (bosons):
\begin{equation}
  f(p,T) = \left[ e^\frac{E(p) - \mu(T)}{T} \pm 1 \right]^{-1}
\end{equation}
with the $ + $ sign for fermions and the $ - $ sign for bosons. At low temperatures, i.e. $ T < E - \mu $, both distributions reduce to the Maxwell--Boltzmann distribution:
\begin{equation}
  f(p,T) \approx e^{- \frac{E(p) - \mu(T)}{T}}
\end{equation}
The chemical potential $ \mu(T) $ characterizes the response of a system to a change in particle number: the second law of thermodynamics implies that, given a reaction, particles flow towards the side of the reaction with the lower total chemical potential, until \emph{chemical equilibrium} is achieved:
\begin{equation*}
  \sum_\text{reactants} \mu_i = \sum_\text{products} \mu_j
\end{equation*}
Note that, since the number of photons is not conserved (e.g. double Compton scattering $ e^- \gamma \lra e^- \gamma \gamma $), then $ \mu_\gamma = 0 $ and $ \mu_{\bar{X}} = - \mu_X $, due to particle-antiparticle annihilation $ X \bar{X} \lra \gamma \gamma $.

A system is in \bctxt{thermal equilibrium} if its species are both in kinetic and chemical equilibrium: these species then share a common temperature $ T_i = T $, which is often identified with the photon temperature $ T_\gamma $ (the ``temperature of the universe").

\subsection{Density and pressure}

At early times, the chemical potentials of all particles are so small that they can be neglected, thus \eeref{eq:num-dens-def}{eq:en-dens-def} can be written as:
\begin{equation}
  n(T) = \frac{g}{2\pi^2} T^3 I_\pm(x)
  \qquad \qquad
  \rho(T) = \frac{g}{2\pi^2} T^4 J_\pm(x)
  \label{eq:exact-rho}
\end{equation}
with $ x \equiv m / T $, $ \xi \equiv p / T $ and:
\begin{equation}
  I_\pm(x) \equiv \int_0^\infty \dd \xi \frac{\xi^2}{\exp \sqrt{\xi^2 + x^2} \pm 1}
  \qquad \qquad
  J_\pm(x) \equiv \int_0^\infty \dd \xi \frac{\xi^2 \sqrt{\xi^2 + x^2}}{\exp \sqrt{\xi^2 + x^2} \pm 1}
  \label{eq:integrals}
\end{equation}
These integrals need, in general, to be evaluated numerically, but analytic expressions are possible in the ultra-relativistic and sub-relativistic limits.

\begin{lemma}{Ultra-relativistic limit}{}
  In the limit of $ T \gg m $, i.e. $ x \ra 0 $:
  \begin{equation}
    n(T) = \frac{\zeta(3)}{\pi^2} g T^3
    \begin{cases}
      1 & \text{bosons} \\
      \tfrac{3}{4} & \text{fermions}
    \end{cases}
    \qquad \qquad
    \rho(T) = \frac{\pi^2}{30} g T^4
    \begin{cases}
      1 & \text{bosons} \\
      \tfrac{7}{8} & \text{fermions} \\
    \end{cases}
    \label{eq:ultra-rel}
  \end{equation}
\end{lemma}

\begin{proofbox}
  \begin{proof}
    For $ x \ra 0 $, the integrals in \eref{eq:integrals} reduce to:
    \begin{equation*}
      I_\pm(0) = \int_0^\infty \dd \xi \frac{\xi^2}{e^\xi \pm 1}
      \qquad \qquad
      J_\pm(0) = \int_0^\infty \dd \xi \frac{\xi^3}{e^\xi \pm 1}
    \end{equation*}
    These integrals can be solved using the standard integral:
    \begin{equation}
      \int_0^\infty \dd \xi \frac{\xi^n}{e^\xi - 1} = \zeta(n + 1) \Gamma(n + 1)
    \end{equation}
    In the bosonic case they are straightforward: $ I_-(0) = 2 \zeta(3) $ and $ J_-(0) = 6 \zeta(4) = \frac{\pi^4}{15} $. In the fermionic case, use instead:
    \begin{equation*}
      \frac{1}{e^\xi + 1} = \frac{1}{e^\xi - 1} - \frac{2}{e^{2\xi} - 1}
    \end{equation*}
    so that:
    \begin{equation*}
      I_+(0) = I_-(0) - 2 \left( \frac{1}{2} \right)^3 I_-(0) = \frac{3}{2} I_-(0)
    \end{equation*}
    \begin{equation*}
      J_+(0) = J_-(0) - 2 \left( \frac{1}{2} \right)^4 J_-(0) = \frac{7}{8} J_-(0)
    \end{equation*}
    which concludes the proof.
  \end{proof}
\end{proofbox}

From \eref{eq:press-def} with $ E \simeq p $ in the relativistic limit, the equation of state for a relativistic gas $ P = \frac{1}{3} \rho $ is recovered. Moreover, recalling \eref{eq:omega-def}, using $ T_0 = 2.73 \,\text{K} $ it is possible to compute the density parameter of radiation:
\begin{equation*}
  \rho_{\gamma,0} = \frac{\pi^2}{15} T_0^4 \simeq 4.6 \cdot 10^{-34} \,\text{g} \,\text{cm}^{-3}
  \quad \implies \quad
  \Omega_\gamma h^2 \simeq 2.5 \cdot 10^{-5}
\end{equation*}

\begin{lemma}{Sub-relativistic limit}{}
  In the limit $ T \ll m $, i.e. $ x \gg 1 $:
  \begin{equation}
    n(T) = g \left( \frac{m T}{2\pi} \right)^{3/2} e^{-m/T}
    \qquad \qquad
    \rho(T) = m n(T) + \frac{3}{2} n(T) T
    \label{eq:sub-rel}
  \end{equation}
\end{lemma}

\begin{proofbox}
  \begin{proof}
    For $ x \gg 1 $:
    \begin{equation*}
      I_\pm(x) \approx \int_0^\infty \dd \xi \, \xi^2 e^{-\sqrt{\xi^2 + x^2}}
    \end{equation*}
    Most of the contribution comes from the $ \xi \ll x $ regime, hence, at lowest order in $ \xi $:
    \begin{equation*}
      I_\pm(x) \approx \int_0^\infty \dd \xi \, \xi^2 \exp \left( x + \frac{\xi^2}{2x} \right) = (2x)^{3/2} e^{-x} \int_0^\infty \dd \xi \, \xi^2 e^{-\xi^2}
    \end{equation*}
    Recalling the standard integral:
    \begin{equation}
      \int_0^\infty \dd \xi \, \xi^n e^{-\xi^2} = \frac{1}{2} \Gamma\left( \frac{n + 1}{2} \right)
    \end{equation}
    with $ \Gamma(\frac{3}{2}) = \frac{\sqrt{\pi}}{2} $:
    \begin{equation*}
      I_\pm(x) = \sqrt{\frac{\pi}{2}} x^{3/2} e^{-x}
      \quad \implies \quad
      n(T) = g \left( \frac{m T}{2\pi} \right)^{3/2} e^{-m/T}
    \end{equation*}
    For the energy density, use the sub-relativistic expansion at lowest order in $ \xi $ in \eref{eq:integrals}:
    \begin{equation*}
      \begin{split}
        J_\pm(x)
        & \approx \int_0^\infty \dd \xi \, \xi^2 \left( x + \frac{\xi^2}{2x} \right) \exp \left( x + \frac{\xi^2}{2x} \right) = (2x)^{3/2} e^{-x} \int_0^\infty \dd \xi \, \xi^2 \left( x + \xi^2 \right) e^{-\xi^2} \\
        & = (2x)^{3/2} e^{-x} \frac{1}{2} \left[ x \Gamma \left( \frac{3}{2} \right) + \Gamma \left( \frac{5}{2} \right) \right] = \sqrt{\frac{\pi}{2}} x^{3/2} e^{-x} \left( x + \frac{3}{2} \right)
      \end{split}
    \end{equation*}
    Inserting $ x = m / T $ yields the thesis.
  \end{proof}
\end{proofbox}

In a similar way, from \eref{eq:press-def} it can be shown that $ P = n T $, which means that a non-relativistic gas behaves like pressureless dust since $ P = n T \ll n m = \rho $.

As expected, in the sub-relativistic limit $ T \ll m $ the number density, the energy density and the pressure of massive particle species fall exponentially (Boltzmann suppression). This is interpreted as the annihilation of particles with antiparticles: at high temperature (i.e. energy), matter-antimatter annihilation is balanced by pair production, but as $ T $ decreases below the mass of the particle the thermal energy is no longer sufficient for pair production and the number density decreases exponentially.

\subsection{Effective number of relativistic species}

From now on, the temperature of the universe is identified with the temperature of the photon gas: $ T \equiv T_\gamma $.

\begin{definition}{Effective number of relativistic degrees of freedom}{}
  The total radiation density in the universe is defined as:
  \begin{equation}
    \rho_\text{r} \defeq \sum_i \rho_i \equiv \frac{\pi^2}{30} \dof(T) T^4
    \label{eq:rad-en-dens-def}
  \end{equation}
  where the sum runs over all relativistic particle species. The \bcdef{effective number of relativistic degrees of freedom} is denoted by $ \dof(T) $.
\end{definition}

Note that $ \dof(T) $ receives different contributions from relativistic species in thermal equilibrium with the photons, i.e. with $ T = T_i \gg m_i $, and from decoupled relativistic species, i.e. with $ T \neq T_i \gg m_i $. In particular, from \eref{eq:ultra-rel}:
\begin{equation}
  \dof^\text{eq}(T) = \sum_\text{bosons} g_i + \frac{7}{8} \sum_\text{fermions} g_j
  \qquad \quad
  \dof^\text{dec}(T) = \sum_\text{bosons} g_i \left( \frac{T_i}{T} \right)^4 + \frac{7}{8} \sum_\text{fermions} g_j \left( \frac{T_j}{T} \right)^4
\end{equation}
When the temperature drops below the mass $ m_i $ of a particle species, it becomes non-relativistic and no longer contributes to $ \dof(T) $.

\begin{example}{$ \dof(T) $ above $ 100 \gev $}{}
  For $ T \gtrsim 100 \gev $, all SM particles are relativistic (as noted in \secref{sec:hot-big-bang}), hence all species contribute to $ \dof^\text{eq}(T) $. The bosons have:
  \begin{equation*}
    g_\text{b} = g_\gamma + g_g + g_{W,Z} + g_H = 2_\text{spin} + 8_\text{colour} \cdot 2_\text{spin} + 3_{\SUn{2}} \cdot 3_\text{spin} + 1_\text{spin} = 28
  \end{equation*}
  since massless vector bosons only have transverse polarization. The fermions, on the other hand, have:
  \begin{equation*}
    \begin{split}
      g_\text{f}
      & = g_q + g_{e,\mu,\tau} + g_\nu \\
      & = 6_\text{species} \cdot 2_\text{spin} \cdot 2_\text{anti} \cdot 3_\text{colour} + 3_\text{species} \cdot 2_\text{spin} \cdot 2_\text{anti} + 3_\text{species} \cdot 1_\text{spin} \cdot 2_\text{anti} = 90
    \end{split}
  \end{equation*}
  since neutrinos only have one helicity state (left-handed neutrinos and right-handed antineutrinos). The effective number of relativistic degrees of freedom then is:
  \begin{equation}
    \dof = g_\text{b} + \tfrac{7}{8} g_\text{f} = 106.75
  \end{equation}
  As the temperature drops, various particle species become non-relativistic and annihilate: the evolution of $ \dof(T) $ is plotted in \figref{fig:rel-dof}. At $ T \sim 150 \mev $, before the strange quarks have time to annihilate, the QCD phase transition takes place: quarks combine into baryons and mesons, which are all non-relativistic under $ 150 \mev $, except for the pions ($ \pi^\pm , \pi^0 $). The only relativistic species left are then pions, electrons, muons, neutrinos and photons: since the pions are spin-$ 0 $ particles, they contribute to $ g_\text{b} $ with $ g_\pi = 3_\text{species} \cdot 1_\text{spin} $, resulting in $ \dof = 2 + 3 + \frac{7}{8} \cdot (4 + 4 + 6) = 17.25 $.

  Moreover, it is important to note that the transition from relativistic to non-relativistic behaviour is not instantaneous: about $ 80\% $ of the particle-antiparticle annihilations takes place in the interval $ T = m \ra T = \frac{1}{6} m $.
\end{example}

\begin{figure}
  \centering
  \includegraphics[width = 0.8 \textwidth]{rel-dof.png}
  \caption{Evolution of $ \dof(T) $ with temperature, assuming Standard Model particles. The dotted line is the effective number of degrees of freedom in entropy $ \dofs(T) $.}
  \label{fig:rel-dof}
\end{figure}

\subsubsection{Entropy}

According to the second law of thermodynamics, the total entropy of the universe can only increase or stay constant. Since the fraction of baryons to photons is measured to be:
\begin{equation}
  \eta_\text{b} \equiv \frac{n_\text{b}}{n_\gamma} \simeq 5.5 \cdot 10^{-10} \frac{\Omega_\text{b} h^2}{0.020} \sim 10^{-9}
\end{equation}
the entropy of the universe is dominated by the entropy of the photon bath (at least as long as the universe is sufficiently uniform): any entropy production from non-equilibrium processes is then insignificant relative to the total entropy, and therefore the expansion of the universe can be treated as \emph{adiabatic}, so that the total entropy stays constant, even beyond equilibrium.

\begin{lemma}{Entropy at equilibrium}{}
  The entropy of a system at equilibrium stays constant.
\end{lemma}

\begin{proofbox}
  \begin{proof}
    Consider the second law of thermodynamics: $ T \dd S = \dd U + P \dd V $. Using $ U = \rho V $:
    \begin{equation*}
      \dd S = \frac{\dd [(\rho + P) V]}{T} - \frac{V \dd P}{T} = \frac{V}{T} \frac{\pa \rho}{\pa T} \dd T + \frac{\rho + P}{T} \dd V
    \end{equation*}
    Since the entropy function must exist, this differential form must be exact. By \tref{th:poincare}, an exact form is also closed, hence:
    \begin{equation}
      \frac{\pa}{\pa V} \left[ \frac{V}{T} \frac{\pa \rho}{\pa T} \right] = \frac{\pa}{\pa T} \left[ \frac{\rho + P}{T} \right]
      \quad \implies \quad
      \frac{\pa P}{\pa T} = \frac{\rho + P}{T}
      \label{eq:pt-thermo}
    \end{equation}
    Rearranging the second law of thermodynamics, then:
    \begin{equation*}
      \dd S = \frac{\dd [(\rho + P) V]}{T} - \frac{(\rho + P) V}{T^2} \dd T = \dd \left[ \frac{\rho + P}{T} V \right]
    \end{equation*}
    Tho show that entropy is conserved in equilibtrium, consider:
    \begin{equation*}
      \begin{split}
        \frac{\dd S}{\dd t} = \frac{\dd}{\dd t} \left[ \frac{\rho + P}{T} V \right] = \frac{V}{T} \left[ \dot{\rho} + \frac{\dot{V}}{V} (\rho + P) \right] + \frac{V}{T} \left[ \dot{P} - \frac{\rho + P}{T} \dot{T} \right]
      \end{split}
    \end{equation*}
    Since $ V \propto a^3 $, then $ \dot{V} / V = 3H $ and the first term vanishes by \eref{eq:r-p}, while the second vanishes by \eref{eq:pt-thermo}. Therefore, $ \dot{S} = 0 $ at equilibrium.
  \end{proof}
\end{proofbox}

It is convenient to define the \bctxt{entropy density}, which ignoring a constant integration constant is (by the above proof):
\begin{equation}
  s = \frac{\rho + P}{T}
\end{equation}
which, by adiabaticity (i.e. $ \dd Q = 0 \, \implies \, \dd S = 0 $), scales like $ s \propto a^{-3} $. Since for a relativistic gas $ P = \frac{1}{3} \rho $, recalling \eref{eq:ultra-rel} it is possible to express the entropy density in the ultra-relativistic limit as:
\begin{equation}
  s = \frac{2\pi^2}{45} g(T) T^3
\end{equation}

\begin{definition}{Effective number of relativistic degrees of freedom in entropy}{}
  The total entropy density in the universe is defined as:
  \begin{equation}
    s \defeq \sum_i s_i \equiv \frac{2\pi^2}{45} \dofs(T) T^3
    \label{eq:s-def}
  \end{equation}
  where the sum runs over all relativistic particle species. The \bcdef{effective number of degrees of freedom in entropy} is denoted by $ \dofs(T) $.
\end{definition}

As for $ \dof(T) $, there are two different contributions to $ \dofs(T) $ too: one from species in thermal equilibrium with the photon bath and one from decoupled species. Note that $ \dofs^\text{eq}(T) = \dof^\text{eq}(T) $, while, since $ s_i \propto T_i^3 $, for decoupled species:
\begin{equation}
  \dofs^\text{dec}(T) = \sum_\text{bosons} g_i \left( \frac{T_i}{T} \right)^3 + \frac{7}{8} \sum_\text{fermions} g_j \left( \frac{T_j}{T} \right)^3
\end{equation}
which is $ \dofs^\text{dec}(T) \neq \dof^\text{dec}(T) $. It follows that $ \dof(T) = \dofs(T) $ only when \emph{all} relativistic species are in thermal equilibrium with the photon bath: in the real universe, this is the case for until $ t \approx 1 \,\text{sec} $ (see \figref{fig:rel-dof}).
The conservation of entropy has two important consequences:
\begin{itemize}
  \item since $ s \propto a^{-3} $, if the number of particles in a comoving volume $ N_i $ is conserved (i.e. $ n_i \propto a^{-3} $), then $ n_i \propto s $, with $ N_i = n_i / s $ since it is a scale-invariant quantity. This is the case, for example, for the net baryon number after baryogenesis, which is conserved, hence $ N_\text{net} \equiv (n_\text{B} - n_{\bar{\text{B}}}) / s $ is constant;
  \item from the definition of $ s $, it follows that $ T \propto \dofs^{-1/3} a^{-1} $. Away from mass thresholds, $ \dofs $ is approximately constant and $ T \propto a^{-1} $ as expected, but when a particle species becomes non-relativistic the factor $ \dofs $ accounts for the transfer of its entropy to other relativistic species, making $ T $ change in the scaling as a consequence.
\end{itemize}

\subsubsection{Neutrino decoupling}

Neutrinos are coupled to the thermal bath only through weak interaction processes (e.g. $ \nu_e \bar{\nu}_e \lra e^+ e^- $), whose cross-section is estimated as $ \sigma \sim G_\text{F}^2 T^2 $. By \eref{eq:int-rate-def}, then, $ \Gamma \sim G_\text{F}^2 T^5 $, which allows to estimate the temperature for neutrino decoupling (with $ H \sim T^2 / \pl $):
\begin{equation*}
  \frac{\Gamma}{H} \sim \left( \frac{T}{1 \mev} \right)^3
  \quad \implies \quad
  T_\text{dec} \sim 1 \mev
\end{equation*}
A more accurate computation gives $ T_\text{dec} \sim 0.8 \mev $. After decoupling, neutrinos move freely along geodesics. Since $ p \propto a^{-1} $ (by \eref{eq:geodesic-momentum}), it is convenient to define the time-independent combination $ q \equiv a p $, so that the neutrino number density after decoupling can be written as:
\begin{equation*}
  n_\nu \propto a^{-3} \int \dd^3 q \left[ \exp \frac{q}{a T_\nu} + 1 \right]^{-1}
\end{equation*}
But, after decoupling, particle number conservation requires $ n_\nu \propto a^{-3} $, hence the neutrino temperature must scale as $ T_\nu \propto a^{-1} $. The neutrinos then follow a relativistic Fermi--Dirac distribution with decreasing temperature, even after they become non-relativistic at later times (when $ T_\nu \ll m_\mu $), due to the Louville theorem\footnotemark.

\footnotetext{The Liouville theorem states that, in a Hamiltonian system, the phase-space probability density (i.e. distribution function) is constant along the trajectories defined by the Hamilton equations of motion.}

\subsubsection{Electron-positron annihilation}

Shortly after the neutrinos decouple, the temperature drops below the electron mass ($ m_e \simeq 0.511 \mev $), so that electron-positron annihilation is no longer balanced by the $ \gamma \gamma \ra e^- e^+ $ pair production. The energy density and entropy of electrons and positrons are then transferred to the photons, but not to the decoupled neutrinos: the photons are thus heated relative to the neutrinos (see \figref{fig:neutrino-temp}).

\begin{figure}
  \centering
  \includegraphics[width = 0.70 \textwidth]{neutrino-temp.png}
  \caption{Difference in redshift between neutrino temperature $ T_\nu \propto a^{-1} $ and photon temperature $ T_\gamma \propto \dofs^{-1/3} a^{-1} $.}
    \label{fig:neutrino-temp}
\end{figure}

To quantify this effect, consider the change in the effective number of degrees of freedom in entropy. Neglecting neutrinos and other decoupled species\footnotemark:
\begin{equation*}
  \dofs^\text{eq} =
  \begin{cases}
    2_\gamma + \frac{7}{8} \cdot 4_e = \frac{11}{2} & T \gtrsim m_e \\
    2_\gamma = 2 & T < m_e
  \end{cases}
\end{equation*}

\footnotetext{Entropy is conserved separately for the thermal bath and the decoupled species, since they are non-interacting systems.}

Since, in equilibrium, $ \dofs^\text{eq} T^3 a^3 $ remains constant, the photon temperature $ a T_\gamma \equiv a T $ must increase by a factor of $ (11/4)^{1/3} $ after electron-positron annihilation, while $ a T_\nu $ remains the same. This means that, for $ T < m_e $:
\begin{equation}
  T_\nu = \sqrt[3]{\frac{4}{11}} T_\gamma
  \label{eq:t-nu-gamma}
\end{equation}
For $ T \ll m_e $, the only remaining relativistic species are photons and neutrinos, and the above equation implies that $ \dof \neq \dofs $:
\begin{equation*}
  \dof = 2_\gamma + \frac{7}{8} \cdot 2_\text{anti} \cdot N_\text{eff} \left( \frac{4}{11} \right)^{4/3} = 3.36
  \qquad \qquad
  \dofs = 2_\gamma + \frac{7}{8} \cdot 2_\text{anti} \cdot N_\text{eff} \left( \frac{4}{11} \right) = 3.94
\end{equation*}
where $ N_\text{eff} $ is the \bctxt{effective number of neutrino species}. If neutrino decoupling was instantaneous, then $ N_\text{eff} = 3 $; however, neutrino decoupling was not complete when $ e^+ e^- $ annihilation began, so a fraction of the released energy and entropy did leak to the neutrinos, resulting into $ N_\text{eff} = 3.046 $ (the current bound from the Planck satellite is $ N_\text{eff} = 3.04 \pm 0.18 $).

Note that \eref{eq:t-nu-gamma} holds until the present, thus allowing to estimate the temperature of the \bctxt{cosmic neutrino background} (C$ \nu $B): given the present CMB temperature $ T_\text{CMB} = 2.73 \,\text{K} = 0.24 \mmev $, it is $ T_\text{C$ \nu $B} = 1.95 \,\text{K} = 0.17 \mmev $. From \eref{eq:ultra-rel}, the number density of neutrinos is:
\begin{equation}
  n_\nu = \frac{3}{4} N_\text{eff} \cdot \frac{4}{11} n_\gamma
\end{equation}
which today corresponds to $ n_\nu / N_\text{eff} \simeq 112 \,\text{cm}^{-3} $ per flavour, as $ n_\gamma(2.73 \,\text{K}) \simeq 5 \,\text{K}^3 \simeq 415 \,\text{cm}^{-3} $. On the other hand, the present energy density of neutrinos depends on whether they are relativistic or non-relativistic today:
\begin{itemize}
  \item massless neutrinos: massless neutrinos are always relativistic, hence, by \eref{eq:ultra-rel}:
    \begin{equation*}
      \rho_\nu = \frac{7}{8} N_\text{eff} \left( \frac{4}{11} \right)^{4/3} \rho_\gamma \quad \implies \quad \Omega_\nu h^2 \simeq 1.7 \cdot 10^{-5}
    \end{equation*}
    were $ \Omega_\gamma h^2 \simeq 2.5 \cdot 10^{-5} $ was used;
  \item massive neutrinos: experiments on neutrino oscillations impose constraints on the sum of neutrino masses: $ 60 \mmev < \sum_i m_{\nu_i} < 1 \ev $. This means that neutrinos behave as radiation-like particles in the early universe ($ T \gg m_\nu $) and as matter-like particles in the late universe ($ T \ll m_\nu $), which means that $ \rho_{\nu,0} \approx \sum_i m_{\nu_i} n_{\nu_i , 0} $ (by \eref{eq:sub-rel}). Using the estimate $ n_{\nu_i} \simeq 112 \,\text{cm}^{-3} $:
    \begin{equation}
      \Omega_\nu h^2 \approx \frac{\sum_i m_{\nu_i}}{95 \ev}
    \end{equation}
    Hence neutrinos are a subdominant component, with $ 0.001 \lesssim \Omega_\nu \lesssim 0.02 $ (using $ h = 0.7 $).
\end{itemize}

\subsection{Thermal history}

In radiation era $ a \propto \sqrt{t} $, hence $ H = \frac{1}{2t} $ and, using \eref{eq:rad-en-dens-def}:
\begin{equation}
  \frac{1}{2t} = H \simeq \sqrt{\frac{\rho_\text{r}}{3\pl^2}} \simeq \frac{\pi}{3} \sqrt{\frac{\dof}{10}} \frac{T^2}{\pl}
  \quad \implies \quad
  \frac{T}{1 \mev} \simeq 1.5 \dof^{-1/4} \sqrt{\frac{1 \,\text{sec}}{t}}
  \label{eq:thermal-history}
\end{equation}
Therefore, $ 1 $ second after the Big Bang the universe was about $ 1 \mev $. It is also possible to state an approximated thermal history of the universe (recalling \eref{eq:redshift-a}):

\begin{center}
  \begin{tabular}{cccc}
    \hline
    \hspace{5em} Event \hspace{5em} & \hspace{1em} time $ t $ \hspace{1em} & \hspace{1em} redshift $ z $ \hspace{1em} & \hspace{1em} temperature $ T $ \hspace{1em} \\
    \hline
    \textbf{Singularity} & $ \ve{0} $ & $ \bs{\infty} $ & $ \bs{\infty} $ \\ [1ex]
    Inflation & $ \sim 10^{-35} \,\text{s} $ & $ \sim 10^{28} $ & $ \sim 10^{15} \gev $ \\ [1ex]
    Baryogenesis & $ \lesssim 20 \,\text{ps} $ & $ > 10^{15} $ & $ > 100 \gev $ \\ [1ex]
    EW phase transition & $ 20 \,\text{ps} $ & $ 10^{15} $ & $ 100 \gev $ \\ [1ex]
    QCD phase transition & $ 20 \,\mu\text{s} $ & $ 10^{12} $ & $ 150 \mev $ \\ [1ex]
    Dark matter freeze-out & ? & ? & ? \\ [1ex]
    Neutrino decoupling & $ 1 \,\text{s} $ & $ 6 \cdot 10^9 $ & $ 1.5 \mev $ \\ [1ex]
    $ e^- e^+ $ annihilation & $ 6 \,\text{s} $ & $ 2 \cdot 10^9 $ & $ 500 \kev $ \\ [1ex]
    Big Bang nucleosynthesis & $ 3 \,\text{min} $ & $ 4 \cdot 10^8 $ & $ 100 \kev $ \\ [1ex]
    Matter-radiation equality & $ 60 \,\text{kyr} $ & $ 3400 $ & $ 0.75 \ev $ \\ [1ex]
    Recombination & $ 260 - 380 \,\text{kyr} $ & $ 1100 - 1400 $ & $ 0.26 - 0.33 \ev $ \\ [1ex]
    Photon decoupling (CMB) & $ 380 \,\text{kyr} $ & $ 1100 $ & $ 0.26 \ev $ \\ [1ex]
    Reionization & $ 100 - 400 \,\text{Myr} $ & $ 10 - 30 $ & $ 2.6 - 7.0 \mmev $ \\ [1ex]
    $ \Lambda $-matter equality & $ 9 \,\text{Gyr} $ & $ 0.4 $ & $ 0.33 \mmev $ \\ [1ex]
    \textbf{Today} & $ \ve{13.8} \, \text{\textbf{Gyr}} $ & $ \ve{0} $ & $ \ve{0.24} \, \text{\textbf{meV}} $ \\
    \hline
  \end{tabular}
\end{center}

Note that the expansion of the universe during Inflation had a very different timescale with respect to today, although in both cases it is exponential ($ a \propto \exp H t $):
\begin{equation*}
  H_\text{I} \sim \frac{T_\text{I}^2}{\pl} \sim 10^{11} \gev
  \quad \implies \quad
  t_{H_\text{I}} \equiv \frac{1}{H_\text{I}} \sim 10^{-35} \,\text{sec}
\end{equation*}
\begin{equation*}
  H_0 \approx \frac{70 \,\text{km} \,\text{s}^{-1}}{1 \,\text{Mpc}}
  \quad \implies \quad
  t_{H_0} \equiv \frac{1}{H_0} \sim 4.5 \cdot 10^{17} \,\text{sec}
\end{equation*}
The expansion during Inflation was $ \sim 10^{52} $ times faster than it is today.

\newpage
\section{Beyond equilibrium}

\subsection{Boltzmann equation}

In the absence of interactions, the number of particles in a fixed physical volume $ V \propto a^3 $ is conserved, thus $ n_i \propto a^{-3} $ and:
\begin{equation*}
  \frac{\dot{n}_i}{n_i} = - 3 \frac{\dot{a}}{a}
  \quad \iff \quad
  \frac{1}{a^3} \frac{\dd (n_i a^3)}{\dd t} = 0
\end{equation*}
To include the effect of interactions, a collision term $ C_i(\{n_j\}) $ is added, which depends on the number density of all particle species:
\begin{equation}
  \frac{1}{a^3} \frac{\dd (n_i a^3)}{\dd t} = C_i(\{n_j\})
\end{equation}
This is the \bctxt{Boltzmann equation}, which describes the behaviour of the system beyond equilibrium. The form of the collision term depends on the specific interactions under consideration: interactions between three or more particles are very unlikely, so it is reasonable to assume only single-particle decays and two-particle scatterings/annihilations.

\begin{example}{$ 2 \lra 2 $ scattering}{2-2-scattering}
  Consider a generic process $ 1 + 2 \lra 3 + 4 $, and suppose to track the number density $ n_1 $ of species $ 1 $. The rate of change in the abundance of species $ 1 $ is given by the difference between the rates of annihilation (direct process) and production (inverse process) of the species, hence:
  \begin{equation*}
    \frac{1}{a^3} \frac{\dd (n_1 a^3)}{\dd t} = - \alpha n_1 n_2 + \beta n_3 n_4
  \end{equation*}
  The first term describes the direct process, and the parameter $ \alpha = \braket{\sigma v} $ is the \emph{thermally-averaged cross-section}\footnote{The angle brackets denote an average over the relative velocity $ v $ between particles $ 1 $ and $ 2 $.}. To determine the parameter $ \beta $, note that the collision term must vanish at (chemical) equilibrium, so that:
  \begin{equation*}
    \beta = \left( \frac{n_1 n_2}{n_3 n_4} \right)_\text{eq} \alpha
  \end{equation*}
  where the equilibrium number densities are given by \eref{eq:num-dens-def}. The Boltzmann equation then becomes:
  \begin{equation}
    \frac{1}{a^3} \frac{\dd (n_1 a^3)}{\dd t} = - \braket{\sigma v} \left[ n_1 n_2 - \left( \frac{n_1 n_2}{n_3 n_4} \right)_\text{eq} n_3 n_4 \right]
    \label{eq:boltzmann-2-2}
  \end{equation}
  Rewriting this equation in terms of the number of particles in a comoving volume $ N_i \equiv n_i / s $:
  \begin{equation}
    \frac{\dd \ln N_1}{\dd \ln a} = - \frac{\Gamma_1}{H} \left[ 1 - \left( \frac{N_1 N_2}{N_3 N_4} \right)_\text{eq} \frac{N_3 N_4}{N_1 N_2} \right]
  \end{equation}
  with the interaction rate $ \Gamma_1 \equiv n_2 \braket{\sigma v} $. The r.h.s. contains two different factors: the $ \Gamma_1 / H $ factor, describing the \bcex{interaction efficiency}, and the $ [1 - \dots] $ factor, describing the \bcex{deviation from equilibrium}. There are two main regimes for the reaction:
  \begin{itemize}
    \item $ \Gamma_1 \gg H $: the natural state of the system is chemical equilibrium since, given the large interaction rate, if $ N_1 \gg N_1^\text{eq} $ the r.h.s. is negative and $ N_1 $ reduces towards $ N_1^\text{eq} $, while if $ N_1 \ll N_1^\text{eq} $ the r.h.s. is positive and $ N_1 $ increases towards $ N_1^\text{eq} $. At equilibrium, then, the r.h.s. vanishes and particles assume their equilibrium abundances;
    \item $ \Gamma_1 < H $: when the reaction rate drops below the Hubble scale, the r.h.s. gets suppressed and the comoving density of particles approaches a constant \emph{relic density}, a phenomenon known as \bcex{freeze-out}.
  \end{itemize}
\end{example}

\subsection{Dark matter relics}

In this section, the hypothesis that dark matter is composed of weakly interacting massive particles (WIMPs) is assumed. WIMPs were in close contact with the rest of the cosmic plasma at high temperatures, but then experiences freeze-out at a critical temperature $ T_\text{f} $, which can be estimated using the Boltzmann equation.

\subsubsection{Dark matter freeze-out}

Some assumptions on the WIMP interactions have to be made. In particular, assume that a heavy dark matter particle $ X $ and its antiparticle $ \bar{X} $ can annihilate to produce two light (essentially massless) particle:
\begin{equation*}
  X + \bar{X} \lra \ell + \bar{\ell}
\end{equation*}
Moreover, assume the light particles to be tightly coupled to the cosmic plasma (e.g. if they are charged), so that they maintain their equilibrium densities $ n_\ell = n_\ell^\text{eq} $, and also that there are no initial asymmetries between $ X $ and $ \bar{X} $, i.e. $ n_X = n_{\bar{X}} $. \eref{eq:boltzmann-2-2} for the number of WIMPs in a comoving volume $ N_X \equiv n_X / s $ then is:
\begin{equation*}
  \frac{\dd N_X}{\dd t} = - s \braket{\sigma v} [N_X^2 - (N_X^\text{eq})^2]
\end{equation*}
Since the interesting dynamics takes place at $ T \sim M_X $, define $ x \equiv M_X / T $, so that:
\begin{equation*}
  \frac{\dd x}{\dd t} = \frac{\dd}{\dd t} \frac{M_X}{T} = - \frac{1}{T} \frac{\dd T}{\dd t} x \simeq H x
\end{equation*}
where $ T \propto a^{-1} $ was used (assuming $ \dofs \simeq \text{const.} \equiv \dofs(M_X) $) for the times relevant to freeze-out. Assuming radiation domination\footnotemark, i.e. $ H = H(M_X) / x^2 $, the Boltzmann equation reduces to the so-called \bctxt{Riccati equation}:
\begin{equation}
  \frac{\dd N_X}{\dd x} = - \frac{\lambda}{x^2} [N_X^2 - (N_X^\text{eq})^2]
\end{equation}
with:
\begin{equation}
  \lambda \equiv \frac{2\pi^2}{45} \dofs \frac{M_X^3 \braket{\sigma v}}{H(M_X)}
\end{equation}

\footnotetext{From \eref{eq:thermal-history} (with $ \dof \simeq \text{const.} $) $ H / T^2 = \text{const.} $, hence $ H / T^2 = H(M_X) / M_X^2 $, i.e. $ H = H(M_X) / x^2 $.}

This parameter can be treated as a constant in most WIMP theories, but, despite this, the Riccati equation has no analytic solutions. Numerical solutions for two different values of $ \lambda $ are pictured in \figref{fig:dm-freeze}: at high temperatures ($ x < 1 $) $ N_X \approx N_X^\text{eq} \simeq 1 $, while at low temperatures ($ x \gg 1 $) the equilibrium abundance is exponentially suppressed as $ N_X^\text{eq} \sim e^{-x} $. Ultimately, WIMPs will become so rare that they will not be able to find each other fast enough to maintain the equilibrium abundance: numerically, this freeze-out happens at $ x_\text{f} \sim 10 $, where the solution $ N_X $ to the Riccati equation starts to deviate significantly from the Boltzmann-suppressed equilibrium abundance $ N_X^\text{eq} $.

\begin{figure}
  \centering
  \includegraphics[width = 0.7 \textwidth]{dm-freeze.png}
  \caption{Abundance of WIMPs as the temperature drops below their mass.}
  \label{fig:dm-freeze}
\end{figure}

The final relic abundance $ N_X^\infty \equiv N_X(x = \infty) $ determines the freeze-out density of dark matter, and it can be estimated as a function of $ \lambda $. For $ x \gg x_\text{f} $ (after freeze-out), $ N_X $ will be much larger than $ N_X^\text{eq} $, hence the Riccati equation reduces to:
\begin{equation*}
  \frac{\dd N_X}{\dd x} \simeq - \frac{\lambda}{x^2} N_X^2
\end{equation*}
Integrating over the domain $ (x_\text{f} , \infty) $, the following solution is found:
\begin{equation*}
  \frac{1}{N_X^\infty} - \frac{1}{N_X^\text{f}} = \frac{\lambda}{x_\text{f}}
\end{equation*}
with $ N_X^\text{f} \equiv N_X(x_\text{f}) $. Typically $ N_X^\text{f} \gg N_X^\infty $ (see \figref{fig:dm-freeze}), hence an estimate for the relic abundance is:
\begin{equation}
  N_X^\infty \simeq \frac{x_\text{f}}{\lambda}
  \label{eq:dm-relic-estimate}
\end{equation}
This relation predicts that the relic abundance decreases as the interaction rate $ \lambda $ increases: this makes sense, since larger interaction rates maintain equilibrium longer, i.e. deeper in the Boltzmann-suppressed regime.

Note that the value of $ x_\text{f} \sim 10 $ is not terribly sensitive to the precise value of $ \lambda $: indeed, $ x_\text{f} \propto \abs{\log \lambda} $.

\subsubsection{WIMP miracle}

The freeze-out abundance of dark matter relics can be related to the dark matter density today:
\begin{equation*}
  \Omega_X \equiv \frac{\rho_{X,0}}{\rho_{\text{crit},0}} = \frac{M_X n_{X,0}}{3 \pl^2 H_0^2} = \frac{M_X N_{X,0} s_0}{3 \pl^2 H_0^2} = M_X N_X^\infty \frac{s_0}{3\pl^2 H_0^2}
\end{equation*}
as the number of WIMPs is conserved after freeze-out: $ N_{X,0} = N_X^\infty $. Inserting \ceref{eq:s-def}{eq:dm-relic-estimate}:
\begin{equation*}
  \Omega_X \simeq \frac{M_X x_\text{f}}{\lambda} \frac{2\pi^2}{45} \frac{\dofs(T_0) T_0^3}{3\pl^2 H_0^2} = \frac{H(M_X)}{M_X^2} \frac{x_\text{f}}{\braket{\sigma v}} \frac{\dofs(T_0)}{\dofs(M_X)} \frac{T_0^3}{3\pl^2 H_0^2}
\end{equation*}
Using \eref{eq:thermal-history} for $ H(M_X) $ gives:
\begin{equation}
  \Omega_X \simeq \frac{\pi}{9} \frac{x_\text{f}}{\braket{\sigma v}} \sqrt{\frac{\dof(M_X)}{10}} \frac{\dofs(T_0)}{\dofs(M_X)} \frac{T_0^3}{\pl^3 H_0^2}
\end{equation}
Substituting the measured value $ T_0 = 2.73 \,\text{K} $ and using $ \dofs(T_0) = 3.91 $ and $ \dofs(M_X) \simeq \dof(M_X) $:
\begin{equation*}
  \Omega_X h^2 \simeq \frac{x_\text{f}}{100} \sqrt{\frac{10}{\dof(M_X)}} \frac{10^{-8} \gev^{-2}}{\braket{\sigma v}}
\end{equation*}
This reproduces the observed dark matter density if $ \sqrt{\braket{\sigma v}} \sim 10^{-4} \gev^{-1} \sim 0.1 \sqrt{G_\text{F}} $: the fact that a thermal relic with a cross-section typical of the weak interaction gives the right dark matter abundance is known as the \emph{WIMP miracle}.

\subsection{Big Bang nucleosynthesis}

At $ T \sim 1 \mev $, photons, electrons and positrons are in equilibrium, neutrinos are about to decouple and baryons are non-relativistic, hence much fewer in number than relativistic particles. However, due to baryon number conservation, the total number of nucleons remains constant: weak nuclear reactions may convert neutrons and protons into each other, while strong nuclear reactions may build nuclei from them. This is known as \bctxt{Big Bang nucleosynthesis} (BBN).

In principle, BBN is a very complicated process involving many coupled Boltzmann equations. In practice, it is convenient to make two simplifications:
\begin{enumerate}
  \item only consider hydrogen and helium: no heavier element is produced at appreciable levels in BBN, hence one only needs to track $ \ch{^1H} \equiv \ch{H} $, $ \ch{^2H} \equiv \ch{D} $, $ \ch{^3H} \equiv \ch{T} $, $ \ch{^3He} $ and $ \ch{^4He} $;
  \item only consider protons and neutrons for $ T \gtrsim 0.1 \mev $: at these temperatures, only free protons and neutrons exist, while other ligh nuclei have yet to form.
\end{enumerate}

\subsubsection{Equilibrium abundances}

 To show the validity of these assumptions, recall that $ T \gtrsim 0.8 \mev $ protons and neutrons are coupled via weak interactions ($ \beta $-decay and inverse $ \beta $-decay):
\begin{equation*}
  n + \nu_e \lra p^+ + e^-
  \quad \qquad
  n + e^+ \lra p^+ + \bar\nu_e
\end{equation*}
Assume $ \mu_\nu , \mu_e \simeq 0 $, so that $ \mu_p = \mu_n $. Then, generalizing \eref{eq:sub-rel} to include the chemical potential:
\begin{equation}
  n_i^\text{eq} = g_i \left( \frac{m_i T}{2\pi} \right) e^\frac{\mu_i - m_i}{T}
  \quad \implies \quad
  \left( \frac{n_n}{n_p} \right)_\text{eq} = \left( \frac{m_n}{m_p} \right)^{3/2} e^{-(m_n - m_p) /T} \simeq e^{-Q/T}
  \label{eq:n-to-p}
\end{equation}
where $ Q \equiv m_n - m_p \simeq 1.30 \mev \ll m_n , m_p $. This shows that for $ T \gg 1 \mev $ there are as many neutrons as protons, while for $ T < 1 \mev $ the neutron fraction gets smaller\footnotemark.

\footnotetext{Were the up quark heavier than the down quark, then $ m_n < m_p $ and, with the decrease of temperature, protons would weakly decay into neutrons, producing a much different universe.}

Next, consider deuterium, which is produced by the following relation:
\begin{equation*}
  n + p^+ \lra D + \gamma
\end{equation*}
Since $ \mu_\gamma = 0 $, then $ \mu_{\ch{D}} = \mu_n + \mu_p $, recalling that $ g_{\ch{D}} = 3 $ and $ g_n = g_p = 2 $:
\begin{equation}
  \left( \frac{n_{\ch{D}}}{n_n n_p} \right)_\text{eq} = \left( \frac{m_{\ch{D}}}{m_n m_p} \frac{2\pi}{T} \right)^{3/2} e^{- (m_{\ch{D}} - m_n - m_p) / T}
\end{equation}
In the prefactor $ m_{\ch{D}} \approx 2m_n \approx 2m_p \simeq 1.9 \gev $, while the exponent can be expressed in terms of the binding energy of deuterium $ B_{\ch{D}} \equiv m_n + m_p - m_{\ch{D}} \simeq 2.22 \mev $. Hence, as long as chemical equilibrium holds:
\begin{equation*}
  \left( \frac{n_{\ch{D}}}{n_p} \right)_\text{eq} = n_n^\text{eq} \left( \frac{4\pi}{m_p T} \right)^{3/2} e^{B_{\ch{D}} / T}
\end{equation*}
For an order-of-magnitude estimate, approximate the neutron density with the baryon density, so that:
\begin{equation*}
  n_n \sim n_\text{b} = \eta_\text{b} n_\gamma = \eta_\text{b} \frac{2\zeta(3)}{\pi^2} T^3
  \quad \implies \quad
  \left( \frac{n_{\ch{D}}}{n_p} \right)_\text{eq} \sim \eta_\text{b} \left( \frac{T}{m_p} \right)^{3/2} e^{B_{\ch{D}} / T}
\end{equation*}
The smallness of $ \eta_\text{b} \sim 10^{-9} $ inhibits deuterium production until the temperature drops well below the binding energy $ B_{\ch{D}} $, and the same applies to all other nuclei: at $ T \gtrsim 0.1 \mev $, then, virtually all baryons are in the form of neutrons and protons. At this time, deuterium and helium are produced, but the reaction rates are too slow to produce any heavier nucleus.

\subsubsection{Neutron freeze-out and decay}

The primordial ration of neutrons is particularly important for BBN, since essentially all neutrons became incorporated into $ \ch{^4He} $. Defining the neutron fraction $ X_n \equiv n_n / (n_n + n_p) $, with \eref{eq:n-to-p}:
\begin{equation}
  X_n^\text{eq}(T) = \frac{e^{-Q/T}}{1 + e^{-Q/T}}
\end{equation}
Neutrons follow this equilibrium abundance until neutrinos decouple\footnotemark at $ T_\text{dec} \simeq 0.8 \mev $, when weak processes like the $ \beta $-decay effectively shut off. At this moment $ X_n^\text{eq}(T_\text{dec}) \simeq 0.17 \simeq \frac{1}{6} $, which can be taken as a rough estimate of the final freeze-out abundance, so that $ X_n^\infty \sim \frac{1}{6} $ (a more precise estimate with QFT gives $ X_n^\infty = 0.15 $).

\footnotetext{Note that $ Q \sim T_\text{dec} $ is just a coincidence, since the former is determined by the strong and electromagnetic interactions, while the latter by the weak interaction.}

At $ T \lesssim 0.2 \mev $, i.e. $ t \lesssim 100 \,\text{sec} $, the finite lifetime of the neutron becomes important, and indeed neutron decay must be accounted for:
\begin{equation}
  X_n(t) = X_n^\infty e^{-t/\tau_n} \sim \frac{1}{6} e^{-t/\tau_n}
  \label{eq:neutron-decay}
\end{equation}
where $ \tau_n = 886.7 \pm 0.8 \,\text{sec} $ is the neutron's lifetime.

\subsubsection{Helium fusion}

At this point, the universe is mostly made of protons and neutrons. Helium cannot form directly, since the density is too low and the time available is too short for reactions involving three or more nuclei to occur (at least at an appreciable rate). Heavier nuclei are instead formed sequentially, according to the following chain of reactions:
\begin{equation*}
  n + p^+ \lra \ch{D} + \gamma
  \quad \longrightarrow \quad
  D + p^+ \lra \ch{^3He} + \gamma
  \quad \longrightarrow \quad
  \ch{D} + \ch{^3He} \lra \ch{^4He} + p^+
\end{equation*}
Since deuterium is formed directly from neutrons and protons, it can follow its equilibrium abundance as long as there are enough free neutrons available. However, since $ B_{\ch{D}} $ is rather small, the deuterium abundance becomes large rather late (at $ T < 100 \kev $), so, although heavier nuclei have larger binding energy and hence would have larger equilibrium abundances, they cannot be formed until sufficient deuterium has become available, a phenomenon known as \bctxt{deuterium bottleneck}. Only when there is enough deuterium can helium be produced.

To get a rough estimate for the time of nucleosynthesis, first estimate the temperature $ T_\text{nuc} $ when the deuterium fraction in equilibrium would be of order $ 1 $, i.e. $ (n_{\ch{D}} / n_p)_\text{eq} \sim 1 $:
\begin{equation*}
  \eta_\text{b} \left( \frac{T_\text{nuc}}{m_p} \right)^{3/2} e^{B_{\ch{D}} / T} \sim 1
  \quad \implies \quad
  T_\text{nuc} \sim 0.06 \mev
\end{equation*}
From \eref{eq:thermal-history} with $ \dof = 3.38 $, the time of nucleosynthesis is found:
\begin{equation*}
  t_\text{nuc} \simeq 120 \,\text{sec} \left( \frac{0.1 \mev}{T_\text{nuc}} \right)^2 \sim 330 \,\text{sec}
\end{equation*}
A better estimate would be imposing $ (n_{\ch{D}} / n_p)_\text{eq} \sim 10^{-3} $, as suggested by \figref{fig:he-production}, resulting in $ t_\text{nuc} \sim 250 \,\text{sec} $: in both cases $ t_\text{nuc} \ll \tau_n $, so \eref{eq:neutron-decay} is not very sensitive to a precise estimate of $ t_\text{nuc} $. In both cases:
\begin{equation}
  X_n(t_\text{nuc}) \sim \frac{1}{8}
\end{equation}
Since $ B_{\ch{He}} > B_{\ch{D}} $, the Boltzmann factor $ e^{B/T} $ favours helium over deuterium, so helium is produced almost immediately after deuterium: virtually all remaining neutrons at $ t \sim t_\text{nuc} $ are then processed into $ \ch{^4He} $. Since two neutrons go into one nucleus of $ \ch{^4He} $, its final abundance is equal to half the neutron abundance at $ t_\text{nuc} $, hence:
\begin{equation}
  \frac{n_{\ch{He}}}{n_{\ch{H}}} = \frac{n_{\ch{He}}}{n_p} \simeq \frac{\frac{1}{2} X_n(t_\text{nuc})}{1 - X_n(t_\text{nuc})} \sim \frac{1}{16}
  \label{eq:he-h-ratio}
\end{equation}
This prediction is consistent with the observed helium in the universe, as shown in \figref{fig:he-abundance}. The abundances of other light elements are determined numerically solving the system of coupled Boltzmann equation, and the result is shown in \figref{fig:he-production} and \figref{fig:light}: all these theoretical abundances have quantitative agreement with observational data, with the exception of lithium (the so-called \emph{lithium problem}).

\begin{figure}
  \centering
  \includegraphics[width = 0.50 \textwidth]{he-abundance.png}
  \caption{Theoretical predictions (coloured) and observational constraints (grey) on today abundances of light nuclei.}
  \label{fig:he-abundance}
\end{figure}

\begin{figure}
  \centering
  \includegraphics[width = 0.70 \textwidth]{he-production.png}
  \caption{Numerical results for helium production in the early universe.}
  \label{fig:he-production}
\end{figure}

\begin{figure}
  \centering
  \includegraphics[width = 0.70 \textwidth]{light-abundances.png}
  \caption{Numerical results for the evolution of light element abundances.}
  \label{fig:light}
\end{figure}

\subsubsection{Constraints on BSM physics}

The result in \eref{eq:he-h-ratio} depends on several input parameters:
\begin{itemize}
  \item $ \dof $: during radiation era $ H \propto \dof^{1/2} T^2 $, hence variations in $ \dof $ affect the neutron freeze-out temperature $ T_\text{f} \sim T_\text{dec} $, since the neutrino decoupling temperature is determined by $ G_\text{F} T_\text{dec}^5 \sim \sqrt{G_\text{N} \dof} T_\text{dec}^2 $, i.e. $ T_\text{f} \sim T_\text{dec} \propto \dof^{1/6} $. An increase of $ \dof $ causes an increase of $ T_\text{f} $, which in turn increases the $ n_n / n_p $ ratio at freeze-out, hence increasing the final helium abundance;
  \item $ \tau_n $: a larger neutron lifetime would reduce the amount of neutron decay after freeze-out, thus increasing the final helium abundance;
  \item $ Q $: a larger mass difference between neutrons and protons would decrease the $ n_n / n_p $ ratio at freeze-out, so decreasing the final helium abundance;
  \item $ \eta_\text{b} $: the final helium abundance increases with an increase of $ \eta_\text{b} $, as nucleosynthesis starts earlier for larger baryon density (i.e. larger deuterium density);
  \item $ G_\text{N} $: increasing the strength of gravity would increase the neutron freeze-out temperature $ T_\text{f} \sim T_\text{dec} \propto G_\text{N}^{1/6} $, therefore increasing the final helium abundance;
  \item $ G_\text{F} $: increasing the strength of the weak nuclear force would decrease the neutron freeze-out temperature $ T_\text{f} \sim T_\text{dec} \propto G_\text{F}^{-2/3} $, thus decreasing the final helium abundance.
\end{itemize}

The dependence of BBN on SM parameters makes it a probe (and a constraint) of BSM physics.

\subsection{Recombination}
\label{ssec:recombination}

At $ T \gtrsim 1 \ev $, the universe still consisted of a plasma of free electrons and nuclei: photons were tightly coupled to electrons through Compton scattering, while electrons were strongly coupled to protons via Coulomb scattering. When temperature became low enough, electrons and nuclei combined to form neutral atoms: this phenomenon is known as \bctxt{Recombination}. As a consequence, the electron density fell sharply, thus causing a growth of the photon mean free path, which became larger than the horizon distance: photons decoupled from matter and the universe became transparent. Today, these photons are the cosmic microwave background.

\subsubsection{Saha equilibrium}

At $ T > 1 \ev $, baryons and photons are still in equilibrium through electromagnetic reactions, e.g.:
\begin{equation*}
  e^- + p^+ \lra \ch{H} + \gamma
\end{equation*}
Since $ T < m_e < m_p < m_{\ch{H}} $, from \eref{eq:sub-rel} the equilibrium abundances of the various species are:
\begin{equation}
  n_i^\text{eq} = g_i \left( \frac{m_i T}{2\pi} \right)^{3/2} \exp \frac{\mu_i - m_i}{T}
\end{equation}
with $ \mu_p + \mu_e = \mu_{\ch{H}} $ (since $ \mu_\gamma = 0 $). Then:
\begin{equation*}
  \left( \frac{n_{\ch{H}}}{n_e n_p} \right)_\text{eq} = \frac{g_{\ch{H}}}{g_e g_p} \left( \frac{m_{\ch{H}}}{m_e m_p} \frac{2\pi}{T} \right)^{3/2} \exp \frac{m_p + m_e - m_{\ch{H}}}{T}
\end{equation*}
Introducing the binding energy of hydrogen $ B_{\ch{H}} \equiv m_p + m_e - m_{\ch{H}} \simeq 13.6 \ev $ and the spin factors $ g_{\ch{H}} = 4 $, $ g_p = g_e = 2 $, and noting that $ n_e = n_p $ since the universe is not electrically charged (at least according to current observations), this equation becomes:
\begin{equation*}
  \left( \frac{n_{\ch{H}}}{n_e^2} \right)_\text{eq} = \left( \frac{2\pi}{m_e T} \right)^{3/2} e^{B_{\ch{H}} / T}
\end{equation*}
Now, define the \emph{free electron-to-baryon fraction} as $ X_e \equiv n_e / n_\text{b} $, where the baryon number can be written as $ n_\text{b} = \eta_\text{b} (2\zeta(3)/\pi^2) T^3 $, and note that protons are over $ 90\% $ (in number) of all nuclei, so that $ n_\text{b} \approx n_p + n_{\ch{H}} = n_e + n_{\ch{H}} $, so that:
\begin{equation}
  \frac{1 - X_e}{X_e^2} = \frac{n_{\ch{H}}}{n_e^2} n_\text{b}
\end{equation}
Putting everything together, the \bctxt{Saha equation} is found:
\begin{equation}
  \left( \frac{1 - X_e}{X_e^2} \right)_\text{eq} = \frac{2 \zeta(3)}{\pi^2} \eta_\text{b} \left( \frac{2\pi T}{m_e} \right)^{3/2} e^{B_{\ch{H}} / T}
\end{equation}
The solution of this equation as a function of redshift is plotted in \figref{fig:e-b-ratio}: although the Saha approximation correctly identifies the onset of Recombination, it is clearly insufficient to determine the relic densities of electrons after freeze-out.

\begin{figure}
  \centering
  \includegraphics[width = 0.70 \textwidth]{e-b-ratio.png}
  \caption{Free electron-to-baryon ratio as a function of redshift.}
  \label{fig:e-b-ratio}
\end{figure}

\subsubsection{Hydrogen recombination}

To study Recombination, define the recombination temperature $ T_\text{rec} $ as the temperature at which $ 90\% $ of electrons have combined with protons to form hydrogen, i.e.:
\begin{equation}
  X_e(T_\text{rec}) = 0.1
  \quad \implies \quad
  T_\text{rec} \approx 0.3 \ev \simeq 3600 \,\text{K}
\end{equation}
Note that $ T_\text{rec} \ll B_{\ch{H}} $: the reason is that there are many photons for each hydrogen atom ($ \eta_\text{b} \sim 10^{-9} $), hence, even when $ T < B_{\ch{H}} $, the high-energy tail of the photon distribution contains a sufficient number of photons with energy $ E > B_{\ch{H}} $ which can ionize the hydrogen atoms.

This estimate puts $ z_\text{rec} \simeq 1320 $, since $ T_\text{rec} = T_0 (1 + z_\text{rec}) $, which puts Recombination in the matter-dominated era which started at $ z_\text{eq} \simeq 3500 $. Then, using $ a(t) = (t/t_0)^{2/3} $, the time of Recombination is found:
\begin{equation}
  t_\text{rec} = \frac{t_0}{(1 + z_\text{rec})^{3/2}} \simeq 290'000 \,\text{yrs}
\end{equation}

\subsubsection{Photon decoupling}

At $ T = T_\text{rec} $ there are still free electrons which are coupled to the photon bath. Indeed, photons are coupled to matter mostly through their interactions with electrons:
\begin{equation*}
  e^- + \gamma \lra e^- + \gamma
\end{equation*}
with an interaction rate $ \Gamma_\gamma \approx n_e \sigma_\text{T} $ (as $ v \approx c = 1 $), where $ \sigma_\text{T} \simeq 2 \cdot 10^{-3} \mev^{-2} $ is the \emph{Thomson cross-section}. The interaction rate therefore decreases as the density of free electrons drops, and the decoupling of photons happens roughly when it becomes smaller than the expansion rate, i.e.:
\begin{equation}
  \Gamma_\gamma(T_\text{CMB}) \sim H(T_\text{CMB})
\end{equation}
Writing $ n_e = n_\text{b} X_e $, the interaction rate becomes:
\begin{equation*}
  \Gamma_\gamma(T) = n_\text{b} X_e(T) \sigma_\text{T} = \frac{2\zeta(3)}{\pi^2} \eta_\text{b} \sigma_\text{T} X_e(T) T^3
\end{equation*}
On the other hand, in matter era the $ \Omega_\text{m} $ term in \eref{eq:h-omega} dominates, so that:
\begin{equation*}
  H(T) \simeq H_0 \sqrt{\Omega_\text{m}} \left( \frac{T}{T_0} \right)^{3/2}
\end{equation*}
Combining this equation, $ T_\text{CMB} $ is found numerically as:
\begin{equation}
  X_e(T_\text{CMB}) T_\text{CMB}^{3/2} \sim \frac{\pi^2}{2\zeta(3)} \frac{H_0 \sqrt{\Omega_\text{m}}}{\eta_\text{b} \sigma_\text{T} T_0^{3/2}}
  \quad \implies \quad
  T_\text{CMB} \sim 0.27 \ev
\end{equation}
which corresponds to $ z_\text{CMB} \sim 1100 $ and $ t_\text{CMB} \sim 380 \,\text{kyrs} $. Note that, although $ T_\text{CMB} $ is not far from $ T_\text{rec} $, the ionization fraction decreases significantly between recombination and photon decoupling: $ X_e(T_\text{rec}) \simeq 0.1 \ra X_e(T_\text{CMB}) \simeq 0.01 $. This shows that a large degree of neutrality is necessary for the universe to become transparent to photon propagation: after decoupling, photons stream freely through the universe, and are observable today as the CMB.










