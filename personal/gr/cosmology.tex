
Of the few situations in which Einstein field equations sourced by matter \eref{eq:ef-eq} need to be solved directly, the one where $ T_{\mu \nu} $ plays a crucial role is Cosmology, the study of the universe as a whole.

\section{FLRW metric}

The key assumption of cosmology is that the universe is spatially homogeneous and isotropic. These conditions restrict the possible spatial geometries to only three:
\begin{itemize}
  \item Euclidean space $ \R^3 $, with vanishing curvature and metric:
    \begin{equation*}
      \dd s^2 = \dd r^2 + r^2 \left( \dd \vartheta^2 + \sin^2 \vartheta\, \dd \varphi^2 \right)
    \end{equation*}
  \item sphere $ \Sn{3} $, with uniform positive curvature and metric (implicit unitary radius):
    \begin{equation*}
      \dd s^2 = \frac{1}{1 - r^2} \dd r^2 + r^2 \left( \dd \vartheta^2 + \sin^2 \vartheta\, \dd \varphi^2 \right)
    \end{equation*}
  \item hyperboloid $ \hyp{3} $, with uniform negative curvature and metric:
    \begin{equation*}
      \dd s^2 = \frac{1}{1 + r^2} \dd r^2 + r^2 \left( \dd \vartheta^2 + \sin^2 \vartheta\, \dd \varphi^2 \right)
    \end{equation*}
\end{itemize}
The existence of these three symmetric spaces is analogous to the existence of three symmetric spacetimes as solutions of the vacuum field equations: dS and AdS have constant spacetime curvature, supplied by the cosmological constant, while $ \Sn{3} $ and $ \hyp{3} $ have constant spatial curvature. Indeed, the metric on $ \mathbb{S}^3 $ corresponds to the spatial part of the de Sitter metric \eref{eq:dS-metric}, while the metric on $ \hyp{3} $ corresponds to the spatial part of the anti-de Sitter metric \eref{eq:ads-metric}.

These spatial metrics are written in a unified form as:
\begin{equation}
  \dd s^2 = \gamma_{ij} \dd x^i \dd x^j = \frac{\dd r^2}{1 - kr^2} + r^2 \dd \Omega_2^2
\end{equation}
with $ k = +1,0,-1 $ on $ \Sn{3}, \R^3, \hyp{3} $ respectively. Cosmology studies spacetimes in which space expands as the universe evolves, thus the metric takes the form:
\begin{equation}
  \dd s^2 = - \dd t^2 + a^2(t) \gamma_{ij} \dd x^i \dd x^j
\end{equation}
This is the \bctxt{Friedmann--Lemaître--Robertson--Walker metric} and the dimensionless factor $ a(t) $ can be viewed as the size of the spatial dimensions.

\begin{example}{de Sitter metric}{}
  de Sitter metric in global coordinates \eref{eq:ds-global-coo} is a FLRW metric with $ k = +1 $.
\end{example}

\subsection{Curvature tensors}

To solve the field equations for FLRW metrics, first compute the Ricci tensor. Christoffel symbols are straightforward:
\begin{equation*}
  \Gamma^\mu_{00} = \Gamma^0_{i0} = 0
  \qquad
  \Gamma^0_{ij} = a\dot{a} \gamma_{ij}
  \qquad
  \Gamma^i_{0j} = \frac{\dot{a}}{a} \delta^i_j
  \qquad
  \Gamma^i_{jk} = \frac{1}{2} \gamma^{il} \left( \pa_j \gamma_{kl} + \pa_k \gamma_{jl} - \pa_l \gamma_{jk} \right)
\end{equation*}

\begin{proposition}{Ricci tensor}{}
  The Ricci tensor for a FLRW metric has non-vanishing components:
  \begin{equation}
    R_{00} = - 3 \frac{\ddot{a}}{a}
    \qquad
    R_{ij} = \left( \frac{\ddot{a}}{a} + 2 \left( \frac{\dot{a}}{a} \right)^2 + 2 \frac{k}{a^2} \right) g_{ij}
  \end{equation}
\end{proposition}

\begin{proofbox}
  \begin{proof}
    First, $ R_{0i} = 0 $ because there's no covariant 3-vector that it could possibly equal to. Then, contracting \eref{eq:riemann-comp} and recalling the only non-vanishing Christoffel symbols:
    \begin{equation*}
      R_{00} = - \pa_0 \Gamma^i_{i0} - \Gamma^j_{i0} \Gamma^i_{j0} = - 3 \frac{\dd}{\dd t} \left( \frac{\dot{a}}{a} \right) - 3 \left( \frac{\dot{a}}{a} \right)^2 = - 3 \frac{\ddot{a}}{a}
    \end{equation*}
    For the spatial components, consider the spatial metric in Cartesian coordinates:
    \begin{equation*}
      \gamma_{ij} = \delta_{ij} + \frac{k x_i x_j}{1 - k \ve{x} \cdot \ve{x}}
    \end{equation*}
    The Christoffel symbols depend on $ \pa \gamma $ and the Ricci tensor on $ \pa^2 \gamma $, thus to evaluate the latter at $ \ve{x} = 0 $ one only needs the metric up to quadratic order:
    \begin{equation*}
      \gamma_{ij} = \delta_{ij} + k x_i x_j + o(x^4)
      \quad \Rightarrow \quad
      \gamma^{ij} = \delta^{ij} - k x^i x^j + o(x^4)
      \quad \Rightarrow \quad
      \Gamma^i_{jk} = k x^i \delta_{jk} + o(x^3)
    \end{equation*}
    where $ i,j $ indices are raised or lowered by $ \delta^{ij} $. The Ricci tensor is then computed as:
    \begin{equation*}
      \begin{split}
        R_{ij}
        &= \pa_\rho \Gamma^\rho_{ij} - \pa_j \Gamma^\rho_{\rho i} + \Gamma^\lambda_{ij} \Gamma^\rho_{\rho \lambda} - \Gamma^\Gamma_{\rho i} \Gamma^\rho_{j \lambda} \\
        &= (\pa_0 \Gamma^0_{ij} + \pa_k \Gamma^k_{ij}) - \pa_j \Gamma^k_{ki} + (\Gamma^0_{ij} \Gamma^k_{k0} + \Gamma^k_{ij} \Gamma^l_{lk}) - (\Gamma^0_{ki} \Gamma^k_{j0} + \Gamma^k_{0i} \Gamma^0_{jk} + \Gamma^k_{li} \Gamma^l_{jk})
      \end{split}
    \end{equation*}
    Evaluating this expression at $ \ve{x} = \ve{0} $ allows to drop the $ \Gamma^k_{ij} \Gamma^l_{lk} $ term and to replace any undifferentiated $ \gamma_{ij} $ with $ \delta_{ij} $, so that:
    \begin{equation*}
      \begin{split}
        R_{ij} (\ve{x} = \ve{0})
        &= \left( \pa_0 (a\dot{a}) + 3k - k + 3 \dot{a}^2 - \dot{a}^2 - \dot{a}^2 \right) \delta_{ij} + o(x^2) \\
        &= \left( a\ddot{a} + 2 \dot{a}^2 + 2k \right) \delta_{ij} + o(x^2)
      \end{split}
    \end{equation*}
    Covariance implies that $ R_{ij} \sim \gamma_{ij} $, thus the general result is:
    \begin{equation*}
      R_{ij} = \left( a\ddot{a} + 2\dot{a}^2 + 2k \right) \gamma_{ij} = \frac{1}{a^2} \left( a\ddot{a} + 2\dot{a}^2 + 2k \right) g_{ij}
    \end{equation*}
    which completes the proof.
  \end{proof}
\end{proofbox}

The Ricci scalar is then easily computed:
\begin{equation}
  R = 6 \left( \frac{\ddot{a}}{a} + \left( \frac{\dot{a}}{a} \right)^2 + \frac{k}{a^2} \right)
\end{equation}
Finally, the non-vanishing components of the Einstein tensor are:
\begin{equation}
  G_{00} = 3 \left( \left( \frac{\dot{a}}{a} \right)^2 + \frac{k}{a^2} \right)
  \qquad
  G_{ij} = - \left( 2 \frac{\ddot{a}}{a} + \left( \frac{\dot{a}}{a} \right)^2 + \frac{k}{a^2} \right) g_{ij}
\end{equation}

\subsection{Friedmann equations}

There only remains to specify the matter content of the universe. By hypothesis, it is filled by a perfect fluid, so that the energy-momentum tensor is that in \eref{eq:em-tens-fluid}. Assuming that the fluid is at rest in the preferred frame of the universe, i.e. $ u^\mu = (1,0,0,0) $ in the FLRW coordinates, the Bianchi identity $ \na_\mu T^{\mu \nu} = 0 $ reads:
\begin{equation*}
  u^\mu \na_\mu \rho + (\rho + p) \na_\mu u^\mu = 0
\end{equation*}
Recalling that $ \na_\mu u^\mu = \pa_\mu u^\mu + \Gamma^\mu_{\mu \rho} u^\rho = \Gamma^i_{i0} u^0  = \dot{a} / a $, the \bctxt{continuity equation} is obtained:
\begin{equation}
  \dot{\rho} + \frac{3\dot{a}}{a} (\rho + p) = 0
\end{equation}
This equation expresses the conservation of energy in an expanding universe. The second constraint \eref{eq:euler-eq} is trivial for homogeneous isotropic fluids. There remains the equation of state, which for fluids of cosmological interest is simply:
\begin{equation}
  p = w \rho
\end{equation}
In particular, $ w = 0 $ describes pressureless dust, while $ w = \frac{1}{3} $ radiation. The continuity equation thus becomes:
\begin{equation*}
  \frac{\dot{\rho}}{\rho} = - 3 (1 + w) \frac{\dot{a}}{a}
\end{equation*}
This means that the energy density dillutes as the universe expands, for:
\begin{equation}
  \rho = \frac{\rho_0}{a^{3 (1 + w)}}
  \label{eq:flwr-state-eq}
\end{equation}
For pressureless dust $ \rho \sim a^{-3} $, which is the expected scaling of energy density with volume, while for radiation $ \rho \sim a^{-4} $, accounting for an extra $ a^{-1} $ factor due to redshift.

With this setting, the temporal component of the Einstein field equations becomes:
\begin{equation}
  \left( \frac{\dot{a}}{a} \right)^2 = \frac{8\pi G}{3} \rho + \frac{\Lambda}{3} - \frac{k}{a^2}
  \label{eq:friedmann-eq}
\end{equation}
This is the \bctxt{Friedmann equation}, and with \eref{eq:flwr-state-eq} it describe how the universe expands. The spatial component of the field equations, computed using the Friedmann equation too, reads:
\begin{equation}
  \frac{\ddot{a}}{a} - \frac{\Lambda}{3} = -\frac{4\pi G}{3} (\rho + 3p)
  \label{eq:ray-eq}
\end{equation}
This is the \bctxt{Raychaudhuri equation}, and it describes the acceleration of the expansion of the universe: it isn't independent of the Friedmann equation, as a time derivative of \eref{eq:friedmann-eq} yields \eref{eq:ray-eq}.

A particularly simple solution can be found by setting $ k = \Lambda = 0 $, i.e. considering a universe dominated only by a single homogeneous isotropic fluid with energy density \eref{eq:flwr-state-eq}:
\begin{equation*}
  \left( \frac{\dot{a}}{a} \right)^2 \sim \frac{1}{a^{3(1 + w)}}
  \quad \implies \quad
  a(t) = \left( \frac{t}{t_0} \right)^{2 / (3 + 3w)}
\end{equation*}
The solution $ w = \frac{1}{3} $, i.e. $ a(t) \sim t^{1/2} $, describes a radiation-dominated universe, which is a model for roughly the first 50'000 years of our Universe, while $ w = 0 $, i.e. $ a(t) \sim t^{2/3} $, describes a matter-dominated universe, which is a model for roughly the following 10 billion years of our Universe.










