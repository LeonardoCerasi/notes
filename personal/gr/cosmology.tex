\selectlanguage{english}

Of the few situations in which Einstein field equations sourced by matter \eref{eq:ef-eq} need to be solved directly, the one where $ T_{\mu \nu} $ plays a crucial role is Cosmology, the study of the universe.

\section{Geometry}

The key assumption of cosmology is that the universe is \emph{spatially homogeneous and isotropic}, as inferred from empirical observations (at least on large scales): this determines a foliation of spacetime into three-dimensional spatial sections $ \Sigma_t $ at various times $ t \in \R $, with each section being homogeneous and isotropic, i.e. maximally-symmetric.

\begin{lemma}{Maximally-symmetric 3-spaces}{max-sym}
  The three possible maximally-symmetric 3-spaces are:
  \begin{itemize}
    \item Euclidean space $ \R^3 $, with vanishing curvature and metric:
      \begin{equation*}
        \dd \ell^2 = \dd r^2 + r^2 \left( \dd \vartheta^2 + \sin^2 \vartheta\, \dd \varphi^2 \right)
      \end{equation*}
    \item the 3-sphere $ \Sn{3} $, with uniform positive curvature and metric (implicit unitary radius):
      \begin{equation*}
        \dd \ell^2 = \frac{1}{1 - r^2} \dd r^2 + r^2 \left( \dd \vartheta^2 + \sin^2 \vartheta\, \dd \varphi^2 \right)
      \end{equation*}
    \item the 3-hyperboloid $ \hyp{3} $, with uniform negative curvature and metric:
      \begin{equation*}
        \dd \ell^2 = \frac{1}{1 + r^2} \dd r^2 + r^2 \left( \dd \vartheta^2 + \sin^2 \vartheta\, \dd \varphi^2 \right)
      \end{equation*}
  \end{itemize}
  These spatial metrics can be unified as:
  \begin{equation}
    \dd \ell^2 = \gamma_{ij} \dd x^i \dd x^j = \frac{\dd r^2}{1 - kr^2} + r^2 \dd \Omega_2^2
  \end{equation}
  with $ k = +1,0,-1 $ on $ \Sn{3}, \R^3, \hyp{3} $ respectively. 
\end{lemma}

\begin{proofbox}
  \begin{proof}
    First, show that these spaces are homogeneous and isotropic:
    \begin{itemize}
      \item Euclidean space $ \R^3 $ is clearly invariant under both translations and rotations;
      \item the 3-sphere $ \Sn{3} $ can be embedded in $ \R^4 $ as:
        \begin{equation*}
          \dd \ell^2 = \dd \ve{x}^2 + \dd u^2
          \qquad \qquad
          \ve{x}^2 + u^2 = a^2
        \end{equation*}
        where $ a \in \R^+ $ is the radius of the 3-sphere. Then, homogeneity and isotropy are inherited from the invariance under 4-rotations of the embedding space;
      \item the 3-hyperboloid $ \hyp{3} $ can be embedded in $ \R^{1,3} $ as:
        \begin{equation*}
          \dd \ell^2 = \dd \ve{x}^2 - \dd u^2
          \qquad \qquad
          \ve{x}^2 - u^2 = - a^2
        \end{equation*}
        where $ a \in \R^+ $ is an arbitrary constant and $ u $ is a time coordinate. Then, homogeneity and isotropy are inherited from the invariance under pseudo-rotations, i.e. Lorentz transformations, of the embedding space.
    \end{itemize}
    In the last two cases, the coordinates can be rescaled as $ \ve{x} \mapsto a \ve{x} , u \mapsto a u $, so that the spaces can be written as:
    \begin{equation*}
      \dd \ell^2 = a^2 \left[ \dd \ve{x}^2 + k \dd u^2 \right]
      \qquad \qquad
      \ve{x}^2 + k u^2 = k
    \end{equation*}
    with $ k = + 1 $ for $ \Sn{3} $ and $ k = -1 $ for $ \hyp{3} $. The embedding condition gives $ \ve{x} \cdot \dd \ve{x} = - k u \, \dd u $, so that:
    \begin{equation*}
      \dd \ell^2 = a^2 \left[ \dd \ve{x}^2 + k \frac{\left( \ve{x} \cdot \dd \ve{x} \right)^2}{1 - k \ve{x}^2} \right]
    \end{equation*}
    which also includes Euclidean space for $ k = 0 $. Switching to spherical coordinates and using $ \ve{x} \cdot \dd \ve{x} = r \, \dd r $ the thesis is found.
  \end{proof}
\end{proofbox}

The existence of these three maximally-symmetric 3-spaces is analogous to the existence of three maximally-symmetric spacetimes as solutions of the vacuum field equations: dS and AdS have constant spacetime curvature, supplied by the cosmological constant, while $ \Sn{3} $ and $ \hyp{3} $ have constant spatial curvature. Indeed, the metric on $ \mathbb{S}^3 $ corresponds to the spatial part of the de Sitter metric \eref{eq:dS-metric}, while the metric on $ \hyp{3} $ corresponds to the spatial part of the anti-de Sitter metric \eref{eq:ads-metric}.

\subsection{FLRW metric}

Cosmology studies spacetimes in which space expands as the universe evolves, thus the metric takes the form:
\begin{equation}
  \dd s^2 = - \dd t^2 + a^2(t) \gamma_{ij} \dd x^i \dd x^j
\end{equation}
This is the \bctxt{Friedmann--Lemaître--Robertson--Walker metric} and the dimensionless factor $ a(t) $ can be viewed as the size of the spatial dimensions, called the \emph{scale factor}.

\begin{example}{de Sitter metric}{}
  de Sitter metric in global coordinates \eref{eq:ds-global-coo} is a FLRW metric with $ k = +1 $.
\end{example}

The FLRW metric has an important scale invariance:
\begin{equation}
  a \mapsto \lambda a
  \qquad
  r \mapsto \lambda^{-1} r
  \qquad
  k \mapsto \lambda^2 k
\end{equation}
This symmetry can be used to set $ a_0 \defeq a(t_0) \equiv 1 $, where $ t_0 $ is the present coordinate time. Therefore, the scale factor is dimensionless, while both $ r $ and $ k^{-1/2} $ have dimensions of a length.

\begin{figure}
  \centering
  \includegraphics[width = 0.80 \textwidth]{comoving.png}
  \caption{Comoving distance and physical distance between two fixed points: as the universe expands, the comoving distance remains constant, while the physical distance varies with $ a(t) $.}
  \label{fig:comoving}
\end{figure}

An important clarification must be made on $ r $: this is called a \emph{comoving coordinate}, while the physical results depend only on the \emph{physical coordinate} $ r_\text{phys} \defeq a(t) r $, as illustrated in \figref{fig:comoving}. Then, the physical velocity of an object clearly is:
\begin{equation}
  v_\text{phys} = v_\text{pec} + H r_\text{phys}
  \label{eq:hubble-flow}
\end{equation}
where the peculiar velocity $ v_\text{pec} = a(t) \dot{r} $ is the velocity measured by a comoving observer, i.e. an observer which follows the Hubble flow, and the \bctxt{Hubble flow} $ H r_\text{phys} $ is the contribution due to the expansion of the universe, with the \emph{Hubble parameter} defined as:
\begin{equation}
  H \defeq \frac{\dot{a}}{a}
  \label{eq:hubble-param}
\end{equation}
Note that the dot always represents a derivative with respect to coordinate time $ t $.

It is convenient to express the metric in terms of redefined time a radial coordinates, defined as:
\begin{equation}
  \dd \chi \equiv \frac{\dd r}{\sqrt{1 - k r^2}}
  \qquad \qquad
  \dd \eta \equiv \frac{\dd t}{a(t)}
  \label{eq:conformal-coordinates}
\end{equation}
where $ \eta $ is called \bctxt{conformal time}. The new radial coordinate allows to simplify the expression for $ g_{rr} $, while conformal time determines a factorization of the FLRW metric into a static metric and a time-dependent conformal factor:
\begin{equation}
  \dd s^2 = a^2(\eta) \left[ \dd \eta^2 - \left( \dd \chi^2 + S_k^2(\chi) \dd \Omega_2^2 \right) \right]
\end{equation}
with:
\begin{equation}
  S_k(\chi) = \frac{1}{\sqrt{k}}
  \begin{cases}
    \sinh (\sqrt{k} \chi) & k < 0 \\
    \sqrt{k} \chi & k = 0 \\
    \sin (\sqrt{k} \chi) & k > 0
  \end{cases}
\end{equation}

\newpage
\section{Kinematics}

\subsection{Particle motion}

In order to study the dynamics of particles in the FLRW metric from the geodesic equation, it is necessary to first compute the Christoffel symbols for this metric. It is straightforward to prove that the only non-vanishing Christoffel symbols are:
\begin{equation}
  \Gamma^0_{ij} = a\dot{a} \gamma_{ij}
  \qquad \qquad
  \Gamma^i_{0j} = \frac{\dot{a}}{a} \delta^i_j
  \qquad \qquad
  \Gamma^i_{jk} = \frac{1}{2} \gamma^{il} \left( \pa_j \gamma_{kl} + \pa_k \gamma_{jl} - \pa_l \gamma_{jk} \right)
  \label{eq:flrw-christoffel}
\end{equation}

\begin{proposition}{Geodesics and 4-momentum}{}
  A particle with 4-momentum $ p^\mu $ moves along geodesics described by:
  \begin{equation}
    p^\alpha \pa_\alpha p^\mu = - \Gamma^\mu_{\alpha \beta} p^\alpha p^\beta
  \end{equation}
\end{proposition}

\begin{proofbox}
  \begin{proof}
    Recall \eref{eq:geodesic-equation} and express it in terms of the 4-velocity $ u^\mu \equiv \frac{\dd x^\mu}{\dd \tau} $:
    \begin{equation*}
      \frac{\dd u^\mu}{\dd \tau} + \Gamma^\mu_{\nu \rho} u^\nu u^\rho = 0
    \end{equation*}
    But, using the chain rule:
    \begin{equation*}
      \frac{\dd u^\mu}{\dd \tau} = \frac{\dd x^\alpha}{\dd \tau} \frac{\pa u^\mu}{\pa x^\alpha} = u^\alpha \pa_\alpha u^\mu
    \end{equation*}
    Substituting $ p^\mu = m u^\mu $ yields the thesis (which is valid in the massless case too).
  \end{proof}
\end{proofbox}

Note that this is nothing but the condition for $ p^\mu $ to be parallely-transported along the geodesic, i.e. $ p^\alpha \na_\alpha p^\mu = 0 $. By the spatial homogeneity of the FLRW metric $ \pa_i p^\mu = 0 $, hence:
\begin{equation*}
  p^0 \dot{p}^\mu = - \Gamma^\mu_{\alpha \beta} p^\alpha p^\beta = - \left( 2 \Gamma^\mu_{0j} p^0 + \Gamma^\mu_{ij} p^i \right) p^j
\end{equation*}
First of all, a consequence of this equation is that massive particles which are at rest in the comoving frame, i.e. $ p^j = 0 $, remain at rest since $ \dot{p}^\mu = 0 $. Then, consider the $ \mu = 0 $ component of this equation and set $ p^0 \equiv E $:
\begin{equation*}
  E \dot{E} = - a \dot{a} \gamma_{ij} p^i p^j = - \frac{\dot{a}}{a} \ve{p}^2
\end{equation*}
where $ \ve{p}^2 \equiv - g_{ij} p^i p^j $ so that $ p^2 = E^2 - \ve{p}^2 = m^2 $. Thus, $ E \, \dd E = \ve{p} \cdot \dd \ve{p} \equiv \text{p} \, \dd \text{p} $, with $ \text{p} \equiv \norm{\ve{p}} $, and:
\begin{equation}
  \frac{\dot{\text{p}}}{\text{p}} = - \frac{\dot{a}}{a}
  \quad \implies \quad
  \text{p} \propto \frac{1}{a}
  \label{eq:geodesic-momentum}
\end{equation}
Thus, the physical 3-momentum of any particle decays with the expansion of the universe. In particular, for massless particles $ E = \text{p} $ decays with the expansion, while for massive particles:
\begin{equation*}
  \text{p} = \frac{m v}{\sqrt{1 - \ve{v}^2}} \propto \frac{1}{a}
\end{equation*}
where $ v^i = \dot{x}^i $ is the comoving peculiar velocity, i.e. the velocity with respect to the comoving frame. Hence, freely-falling particles converge onto the Hubble flow, as per \eref{eq:hubble-flow}.

\subsection{Redshift}

Since the physical properties of the universe are inferred from light signals, it is crucial to quantify how the wavelength of light gets redshifted by the expansions of the universe.

From a quantum point of view, the wavelenght of a light signal is linked to the photon's momentum by the de Broglie relation $ \lambda = h / p $. From \eref{eq:geodesic-momentum}, then, $ \lambda \propto a $, i.e.:
\begin{equation}
  \lambda_0 = \frac{a_0}{a(t)} \lambda(t)
  \label{eq:redshift}
\end{equation}
where $ t < t_0 $ is the time of emission of the photon. As $ a(t) < a_0 $, the light is redshifted.

The same result can be derived for classical EM waves. Consider a galaxy at fixed comoving distance $ d $ which, at a time $ \eta $, emits a signal of short conformal duration $ \Delta \eta $. The light signal arrives on Earth at $ \eta_0 = \eta + d $ with the same conformal duration, but the physical time interval is differrent:
\begin{equation*}
  \Delta t = a(\eta) \Delta \eta
  \qquad \qquad
  \Delta t_0 = a_0 \Delta \eta
\end{equation*}
Assuming $ \Delta t $ to be the period of the EM wave, then $ \lambda(t) = \Delta t $, hence \eref{eq:redshift} is recovered.

\begin{definition}{Redshift parameter}{}
  The \bcdef{redshift parameter} of a light signal emitted at $ t < t_0 $ with a wavelength $ \lambda(t) $ is
  \begin{equation}
    z \defeq \frac{\lambda_0 - \lambda(t)}{\lambda(t)}
  \end{equation}
  i.e. as the fractional shift in the photon's wavelength.
\end{definition}

By \eref{eq:redshift}, setting $ a_0 \equiv 1 $, the common definition of $ z $ is found:
\begin{equation}
  1 + z = \frac{1}{a(t)}
\end{equation}
For nearby sources, it is possible to expand $ a(t) $ as a power series:
\begin{equation*}
  a(t) = a_0 \left[ 1 + H_0 (t - t_0) + \dots \right]
\end{equation*}
where the \bctxt{Hubble constant} is defined as:
\begin{equation}
  H_0 \equiv \frac{\dot{a}(t_0)}{a(t_0)} = 100 h \, \text{km} \, \text{s}^{-1} \text{Mpc}^{-1}
  \qquad
  h \approx 0.67 \pm 0.01
\end{equation}
The dimensionless parameter $ h $ is introduced to keep track of how the uncertainties in the measurement of $ H_0 $ propagate in other physical quantities, since $ H_0 $ normalizes everything.
Inserting this expansion in the definition of $ z $ yields $ z = H_0 (t_0 - t) + \dots $, which, for close objects at distance $ d = t_0 - t $, becomes:
\begin{equation}
  z \simeq H_0 d
\end{equation}
Hence, the rate of redshift of nearby objects measures the current expansion rate of the universe.

\subsection{Distances}

For distant objects, it is important to precisely state which notion of distance is adopted.

\paragraph{Metric distance}

Rewrite the FLRW metric as:
\begin{equation*}
  \dd s^2 = \dd t^2 - a^2(t) \left[ \dd \chi^2 + S_k^2(\chi) \dd \Omega_2^2 \right]
  \qquad \qquad
  S_k(\chi) =
  \begin{cases}
    R_0 \sinh (\chi / R_0) & k = -1 \\
    \chi & k = 0 \\
    R_0 \sin (\chi / R_0) & k = + 1
  \end{cases}
\end{equation*}
where the length scale $ R_0 $ has been introduced thanks to the scale invariance of the metric to compensate for the imposition of $ a_0 \equiv 1 $. The distance multiplying the solid angle element is the \emph{metric distance}:
\begin{equation}
  d_\text{m} = S_k(\chi)
\end{equation}
In a flat universe, the metric distance is equal to the \emph{comoving distance} $ \chi $, which can be computed as a function of redshift:
\begin{equation}
  \chi(z) = \int_t^{t_0} \frac{\dd t'}{a(t')} = \int_0^z \frac{\dd z'}{H_0(z')}
\end{equation}
which depends on the matter content of the universe through $ H(z) $. It must be emphasized that neither the metric distance nor the comoving distance is physically observable.

\paragraph{Luminosity distance}

Type IA supernovae are called ``standard candles", since they are believed to be objects of known absolute luminosity $ L $, i.e. energy emitted per second: therefore, the observed flux $ F $, i.e. energy per second per receiving area, from a supernova explosion can then be used to infer its \emph{luminosity distance}. Considering a source at fixed vomoving distance $ \chi $, in a static Euclidean space the relation between $ F $ and $ L $ would be:
\begin{equation*}
  F = \frac{L}{4\pi \chi^2}
\end{equation*}
However, in a generic FLRW spacetime, this relation needs to be modified for three reasons:
\begin{enumerate}
  \item the proper area of a 3-sphere grows like $ 4\pi d_\text{m}^2 $, not like $ 4\pi \chi^2 $;
  \item the rate of received photons is lower than the rate of emitted photons by a factor of $ 1 / (1 + z) $;
  \item received photons have an energy that is lower than the energy of emitted photons by the same factor of $ 1 / (1 + z) $.
\end{enumerate}
Hence, the correct relation is:
\begin{equation}
  F = \frac{L}{4\pi d_L^2}
  \qquad \qquad
  d_L = (1 + z) d_\text{m}
\end{equation}
where $ d_L $ is the \emph{luminosity distance}.

\paragraph{Angular diameter distance}

Sometimes, it is possible to make use of ``standard rulers", i.e. objects of known physical size $ D $ (e.g. CMB fluctuations). Assuming that a standard ruler is at a fixed comoving distance $ \chi $, the \emph{angular diameter distance} can be defined as:
\begin{equation*}
  d_\text{A} = \frac{D}{\delta \theta}
\end{equation*}
where $ \delta \theta \ll 1 $ is the observed angular size of the object (small for all cosmological objects). In a generic FLRW spacetime, the relation between the physical transverse size of the object and its angular size in the sky instead is:
\begin{equation*}
  D = a(t) S_k(\chi) \delta \theta = \frac{d_\text{m}}{1 + z} \delta \theta
\end{equation*}
Hence, the angular diameter distance is:
\begin{equation}
  d_\text{A} = \frac{d_\text{m}}{1 + z}
\end{equation}
This is the measured distance between Earth and the object when the light was emitted, and in fact it is related to the luminosity distance by:
\begin{equation}
  d_\text{A} = \frac{d_L}{(1 + z)^2}
\end{equation}
The redshift dependence of the three distances $ d_\text{m} $, $ d_L $ and $ d_\text{A} $ is plotted in \figref{fig:redshift-distance}: note that the presence of a cosmological constant makes distances larger, a fact that has been crucial for the discovery of dark energy.

\begin{figure}
  \centering
  \includegraphics[width = 0.70 \textwidth]{redshift-distance.png}
  \caption{Distances measured in a flat universe, with matter only (dotted lines) and with $ 70\% $ dark energy (solid lines).}
  \label{fig:redshift-distance}
\end{figure}

\newpage
\section{Dynamics}

\subsection{Matter sources}

The requirements of spatial homogeneity and isotropy impose strict constraints on the energy-momentum tensor, so that it can only assume the form \eref{eq:em-tens-fluid} for a perfect fluid.

\subsubsection{Number density}

First, consider the number 4-current $ N^\mu $, whose components are the number density $ N^0 $ of ``particles" (where here particles have a generic nature, e.g. a particle could be a galaxy) and the flux of particles $ N^i $ in the direction $ x^i $.

\begin{definition}{Isotropy and homogeneity}{}
  A spacetime is \bcdef{isotropic} if the mean value of any 3-vector vanishes, while it is \bcdef{homogeneous} if the mean value of any 3-scalar\footnote{A 3-scalar is a scalar which is invariant under purely spatial coordinate transformations.} is a function of the time coordinate only.
\end{definition}

The isotropy and homogeneity conditions than impose that a comoving observer measures a number 4-current $ N^\mu = (n(t) , \ve{0}) $, where $ n(t) $ is the number of particles per proper volume as measured by the comoving observer. A general observer with a relative 4-velocity $ u^\mu $ with respect to the rest frame of the particle distribution measures instead:
\begin{equation}
  N^\mu = n u^\mu
\end{equation}
Indeed, for a comoving observer $ u^\mu = (1,\ve{0}) $, while in general $ u^\mu = \gamma (1,\ve{v}) $, which is the correctly boosted result.
Since the number of particles has to be conserved, in Minkowski spacetime $ \dot{n}^0 = - \pa_i N^i $, i.e. $ \pa_\mu N^\mu = 0 $, which generalizes to a curved spacetime as:
\begin{equation}
  \na_\mu N^\mu = 0
\end{equation}
In the comoving frame, then, recalling \eref{eq:flrw-christoffel}:
\begin{equation*}
  \frac{\dd n}{\dd t} + \Gamma^i_{i0} n = 0
  \quad \implies \quad
  \frac{\dot{n}}{n} = -3 \frac{\dot{a}}{a}
  \quad \implies \quad
  n(t) \propto a^{-3}(t)
\end{equation*}
which means that the number density is proportional to the inverse of the proper volume, as expected.

\subsubsection{Energy-momentum tensor}

In order to extend this analysis to the energy-momentum tensor $ T_{\mu \nu} $, decompose it into a 3-scalar $ T_{00} $, a 3-vector $ T_{i0} $ (recall that it is symmetric) and a 3-tensor $ T_{ij} $. By isotropy $ \pi_i \equiv T_{i0} = 0 $ (energy-momentum flow in the $ x^i $ direction), while around a point $ \ve{x} = \ve{0} $ it must be:
\begin{equation*}
  T_{ij}(\ve{x} \propto \ve{0}) \propto \delta_{ij} \propto g_{ij}(\ve{x} = \ve{0})
\end{equation*}
Since by homogeneity the proportionality constant can only depend on time, the energy-momentum tensor measured by a comoving observer takes the form:
\begin{equation*}
  T_{00} = \rho(t)
  \qquad \qquad
  \pi_i \equiv T_{i0} = 0
  \qquad \qquad
  T_{ij} = - P(t) g_{ij}(t,\ve{x})
\end{equation*}
or, with mixed indices, $ \tensor{T}{^\mu_\nu} = \diag \left( \rho , -P , -P , -P \right) $, which is precisely the energy-momentum tensor for a perfect fluid with energy density $ \rho(t) $ and pressure $ P(t) $ in its comoving frame. For a general observer with a 4-velocity $ u^\mu $ with respect to the rest frame of the fluid:
\begin{equation}
  \tensor{T}{^\mu_\nu} = \left( \rho + P \right) u^\mu u_\nu - P \delta^\mu_\nu
  \label{eq:flrw-en-mom-tens}
\end{equation}
To study the evolution of $ \rho(t) $ and $ P(t) $, recall that, in Minkowski spacetime, the fluid is subject to the continuity equation $ \dot{\rho} = -\pa_i \pi^i $ and to the Euler equation $ \dot{\pi}_i = \pa_i P $, which are combined into $ \pa_\mu \tensor{T}{^\mu_\nu} $. Generalizing to curved spacetime:
\begin{equation}
  \na_\mu \tensor{T}{^\mu_\nu} = 0
\end{equation}
i.e. the energy momentum tensor is covariantly-conserved. Recalling \eref{eq:flrw-christoffel}, it is easy to see that the evolution of the energy density is given by this generalized \bctxt{continuity equation}:
\begin{equation}
  \dot{\rho} + 3 \frac{\dot{a}}{a} \left( \rho + P \right) = 0
  \label{eq:r-p}
\end{equation}
which is equivalent to $ \dd U = - P \dd V $, with $ U \equiv \rho V $ and using $ V \propto a^3 $.

\subsubsection{Matter content}

Since the universe is filled with a mixture of different matter components, it is useful to classify the various matter sources by their equation of state $ \rho = \rho(P) $, which can be cast in the form:
\begin{equation}
  \rho = w P
\end{equation}
with $ w \in \R $. With this relation, \eref{eq:r-p} implies the scaling relation:
\begin{equation}
  \rho \propto a^{-3 (1 + w)}
\end{equation}

\paragraph{Matter}

Matter is characterized by $ w = 0 $, i.e. the pressure is negligible when compared to the energy density (to be precise, $ P = 0 $ for pressureless dust only). Then, the dilution of $ \rho \propto a^{-3} $ simply reflects the expansion of the volume $ V \propto a^3 $. This category has two elements:
\begin{itemize}
  \item \bctxt{baryons}, where in Cosmology this term is improperly used to denote both nuclei and electrons\footnotemark;
  \item \bctxt{dark matter}, which is the dominant form of matter in the universe, although its nature is unknown.
\end{itemize}

\footnotetext{Since the mass of the lightest nucleus $ \ch{^1H} \equiv p^+ $ is $ m_p \simeq 1836 m_e $, this convention is approximately justified, since the mass of baryonic matter is dominated by the mass of the nuclei.}

\paragraph{Radiation}

Radiation is a characterized by $ w = \frac{1}{3} $, which is the case, for instance, for a gas of relativistic particles, whose energy density is dominated by the kinetic energy (i.e. $ \text{p} \gg m $). In this case, the dilution $ \rho \propto a^{-4} $ includes an additional factor for the redshift of the energy $ E \propto a^{-1} $. This category includes:
\begin{itemize}
  \item \bctxt{photons}, which are always relativistic (since they are massless particles) and dominated the early universe;
  \item \bctxt{neutrinos}, which behaved like radiation for most of the history of the universe, and only recently started behaving like matter due to their small masses;
  \item \bctxt{gravitons}, which are believed to have been formed in the early universe as a background of gravitational waves.
\end{itemize}

\paragraph{Dark energy}

By experimental observations, the universe today seems to be dominated by a mysterious \emph{negative} pressure $ P = - \rho $, commonly referred to as dark energy. This mysterious form of energy has $ w = -1 $, hence its energy density does not dilute $ \rho \propto a^0 $, which means that energy has to be created as the universe expands\footnotemark.

\footnotetext{Recall \secref{ssec:energy-conservation}: since it is not possible to clearly state an energy conservation principle for a gravitational system, the only meaningfull equation is \eref{eq:r-p}, which is not violated by the creation of additional dark energy as the universe expands.}

A possible candidate as a form of dark energy is the vacuum energy predicted by Quantum Field Theory. Indeed, the ground state of the vacuum has the following energy-momentum tensor:
\begin{equation*}
  T^\text{vac}_{\mu \nu} = \rho_\text{vac} g_{\mu \nu}
\end{equation*}
which clearly implies $ P_\text{vac} = - \rho_\text{vac} $. However, the predicted $ \rho_\text{vac} $ is off of a factor of $ \sim 10^{120} $ with respect to the observed dark energy density, and this is called the ``cosmological constant problem".

\subsection{Curvature tensors}

To compute the Einstein tensor for the FLRW metric, recall the Christoffel symbols \eref{eq:flrw-christoffel}.

\begin{proposition}{Ricci tensor}{}
  The Ricci tensor for a FLRW metric has non-vanishing components:
  \begin{equation}
    R_{00} = - 3 \frac{\ddot{a}}{a}
    \qquad
    R_{ij} = \left[ \frac{\ddot{a}}{a} + 2 \left( \frac{\dot{a}}{a} \right)^2 + 2 \frac{k}{a^2} \right] g_{ij}
  \end{equation}
\end{proposition}

\begin{proofbox}
  \begin{proof}
    First, $ R_{0i} = 0 $ due to the isotropy of the FLRW metric. Then, contracting \eref{eq:riemann-comp} and recalling the only non-vanishing Christoffel symbols:
    \begin{equation*}
      R_{00} = - \pa_0 \Gamma^i_{i0} - \Gamma^j_{i0} \Gamma^i_{j0} = - 3 \frac{\dd}{\dd t} \left( \frac{\dot{a}}{a} \right) - 3 \left( \frac{\dot{a}}{a} \right)^2 = - 3 \frac{\ddot{a}}{a}
    \end{equation*}
    For the spatial components, consider the spatial metric in Cartesian coordinates (as derived in the proof of \lref{lemma:max-sym}):
    \begin{equation*}
      \gamma_{ij} = \delta_{ij} + \frac{k x_i x_j}{1 - k \ve{x} \cdot \ve{x}}
    \end{equation*}
    The Christoffel symbols depend on $ \pa \gamma $ and the Ricci tensor on $ \pa^2 \gamma $, thus to evaluate the latter at $ \ve{x} = 0 $ one only needs the metric up to quadratic order:
    \begin{equation*}
      \gamma_{ij} = \delta_{ij} + k x_i x_j + o(x^4)
      \quad \Rightarrow \quad
      \gamma^{ij} = \delta^{ij} - k x^i x^j + o(x^4)
      \quad \Rightarrow \quad
      \Gamma^i_{jk} = k x^i \delta_{jk} + o(x^3)
    \end{equation*}
    where $ i,j $ indices are raised or lowered by $ \delta^{ij} $. The Ricci tensor is then computed as:
    \begin{equation*}
      \begin{split}
        R_{ij}
        &= \pa_\rho \Gamma^\rho_{ij} - \pa_j \Gamma^\rho_{\rho i} + \Gamma^\lambda_{ij} \Gamma^\rho_{\rho \lambda} - \Gamma^\Gamma_{\rho i} \Gamma^\rho_{j \lambda} \\
        &= (\pa_0 \Gamma^0_{ij} + \pa_k \Gamma^k_{ij}) - \pa_j \Gamma^k_{ki} + (\Gamma^0_{ij} \Gamma^k_{k0} + \Gamma^k_{ij} \Gamma^l_{lk}) - (\Gamma^0_{ki} \Gamma^k_{j0} + \Gamma^k_{0i} \Gamma^0_{jk} + \Gamma^k_{li} \Gamma^l_{jk})
      \end{split}
    \end{equation*}
    Evaluating this expression at $ \ve{x} = \ve{0} $ allows to drop the $ \Gamma^k_{ij} \Gamma^l_{lk} $ term and to replace any undifferentiated $ \gamma_{ij} $ with $ \delta_{ij} $, so that:
    \begin{equation*}
      \begin{split}
        R_{ij} (\ve{x} = \ve{0})
        &= \left( \pa_0 (a\dot{a}) + 3k - k + 3 \dot{a}^2 - \dot{a}^2 - \dot{a}^2 \right) \delta_{ij} + o(x^2) \\
        &= \left( a\ddot{a} + 2 \dot{a}^2 + 2k \right) \delta_{ij} + o(x^2)
      \end{split}
    \end{equation*}
    Spatial homogeneity and isotropy imply that $ R_{ij} \propto \gamma_{ij} $, thus the general result is:
    \begin{equation*}
      R_{ij} = \left( a\ddot{a} + 2\dot{a}^2 + 2k \right) \gamma_{ij} = \frac{1}{a^2} \left( a\ddot{a} + 2\dot{a}^2 + 2k \right) g_{ij}
    \end{equation*}
    which is the thesis.
  \end{proof}
\end{proofbox}

The Ricci scalar is then easily computed:
\begin{equation}
  R = 6 \left[ \frac{\ddot{a}}{a} + \left( \frac{\dot{a}}{a} \right)^2 + \frac{k}{a^2} \right]
\end{equation}
Finally, the non-vanishing components of the Einstein tensor are:
\begin{equation}
  G_{00} = 3 \left[ \left( \frac{\dot{a}}{a} \right)^2 + \frac{k}{a^2} \right]
  \qquad \qquad
  G_{ij} = - \left[ 2 \frac{\ddot{a}}{a} + \left( \frac{\dot{a}}{a} \right)^2 + \frac{k}{a^2} \right] g_{ij}
  \label{eq:flrw-einstein}
\end{equation}

\subsection{Friedmann equations}

With \eref{eq:flrw-einstein}, the Einstein field equations \eref{eq:ef-eq} with the energy-momentum tensor \eref{eq:flrw-en-mom-tens} become (respectively the temporal and spatial components):
\begin{equation}
  \left( \frac{\dot{a}}{a} \right)^2 = \frac{8\pi G}{3} \rho - \frac{k}{a^2}
  \qquad \qquad
  \frac{\ddot{a}}{a} = - \frac{4\pi G}{3} \left( \rho + 3 P \right)
\end{equation}
where the cosmological constant has been suppressed since it is accounted for by dark energy as a form of matter content. Note that these equations are not independent, as the latter is the time derivative of the formed: nonetheless, these are called the \bctxt{Freedman equations}. The energy density $ \rho $ and the pressure $ P $ are to be understood as the sums of the contributions from the various forms of matter: the contribution from radiation $ \rho_r = \rho_\gamma + \rho_\nu $ (photons and neutrinos), the one from matter $ \rho_m = \rho_b + \rho_c $ (baryons and cold dark matter) and the vacuum energy contribution $ \rho_\Lambda $. Recalling the definition of the Hubble parameter \eref{eq:hubble-param}, the first Friedmann equation becomes:
\begin{equation}
  H^2 = \frac{8\pi G}{3} \rho - \frac{k}{a^2}
  \label{eq:fried-hubble}
\end{equation}
It is convenient to define dimensionless density parameters. To do so, observe that a flat universe corresponds to the \bctxt{critical density} today:
\begin{equation*}
  \rho_{\text{crit},0} = \frac{3H_0^2}{8\pi G} \simeq 1.9 \cdot 10^{-29} h^2 \,\text{g} \, \text{cm}^3
\end{equation*}
Then, the dimensionless density parameters are defined as:
\begin{equation}
  \Omega_a \equiv \frac{\rho_{a,0}}{\rho_{\text{crit},0}}
  \label{eq:omega-def}
\end{equation}
with $ a = r,m,\Lambda,\dots $ running over all possible matter species. \eref{eq:fried-hubble} can thus be rewritten as:
\begin{equation}
  H^2(a) = H_0^2 \left[ \Omega_r \left( \frac{a_0}{a} \right)^4 + \Omega_m \left( \frac{a_0}{a} \right)^3 + \Omega_k \left( \frac{a_0}{a} \right)^2 + \Omega_\Lambda \right]
\end{equation}
where a curvature density parameter was defined as:
\begin{equation}
  \Omega_k \equiv - \frac{k}{\left( a_0 H_0 \right)^2}
\end{equation}
Adopting the conventional normalization for the scale factor $ a_0 \equiv 1 $:
\begin{equation}
  \frac{H^2}{H_0^2} = \Omega_r a^{-4} + \Omega_m a^{-3} + \Omega_k a^{-2} + \Omega_\Lambda
\end{equation}

\subsubsection{\texorpdfstring{$ \bs{\Lambda} $}{L}CDM}

Empirical observations show that the universe is filled with radiation, matter and dark energy so that:
\begin{equation*}
  \abs{\Omega_k} \leq 0.01
  \qquad
  \Omega_r \simeq 9.4 \cdot 10^{-5}
  \qquad
  \Omega_m \simeq 0.32
  \qquad
  \Omega_\Lambda \simeq 0.68
\end{equation*}
In particular, matter splits into $ 5\% $ baryonic matter, i.e. with $ \Omega_b \simeq 0.05 $, and $ 27\% $ cold dark matter, i.e. with $ \Omega \simeq 0.27 $. Moreover, note that even today curvature only contribute for less than $ 1\% $ of cosmic energy, and, since $ \Omega_k \propto a^{-2} $ while $ \Omega_m \propto a^{-3} $ and $ \Omega_r \propto a^{-4} $, in the past it was completely negligible: hence, setting $ \Omega \equiv 0 $ is justified, i.e. $ k \equiv 0 $.

\subsubsection{Single-component universe}

Given the clearly distinct scalings of $ \Omega_r \propto a^{-4} $, $ \Omega_m \propto a^{-3} $ and $ \Omega_\Lambda \propto a^0 $, for most of its history the universe was dominated by a single component: first radiation, then matter and finally vacuum energy. This is illustrated in \figref{fig:scalings}.

\begin{figure}
  \centering
  \includegraphics[width = 0.80 \textwidth]{scalings.png}
  \caption{Evolution of energy densities in the universe.}
  \label{fig:scalings}
\end{figure}

For a flat single-component universe, the first Friedmann equation becomes:
\begin{equation}
  \frac{\dot{a}}{a} = H_0 \sqrt{\Omega_a} a^{- \frac{3}{2} \left( 1 + w_a \right)}
  \label{eq:a-sing-comp}
\end{equation}
where $ w_a $ characterizes the component. Note that $ \Omega_a \equiv 1 $, since this singl-component universe is flat. This equation can be integrated, obtaining the time-dependence of the scale factor:
\begin{equation}
  a(t) \propto
  \begin{cases}
    t^{\frac{2}{3 \left( 1 + w_a \right)}} & w_a \neq -1 \\
    e^{Ht} & w_a = -1
  \end{cases}
\end{equation}
Specializing for a radiation-dominated (RD), matter-dominated (MD) and $ \Lambda $-dominated ($ \Lambda $D) universe:
\begin{equation*}
  a(t) \propto
  \begin{cases}
    t^{1/2} & \text{RD} \\
    t^{2/3} & \text{MD} \\
    e^{Ht} & \Lambda\text{D}
  \end{cases}
  \quad \implies \quad
  a(\eta) \propto
  \begin{cases}
    \eta & \text{RD} \\
    \eta^2 & \text{MD} \\
    - \eta^{-1} & \Lambda\text{D}
  \end{cases}
\end{equation*}









