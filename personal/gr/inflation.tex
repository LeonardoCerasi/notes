\selectlanguage{english}

\section{Horizon problem}

\subsection{Particle horizon}

The propagation of lightrays is best studied using conformal time. Given the spatial isotropy of spacetime, it is always possible to define a coordinate system in which the lightrays moves radially, so that its motion is determined by a 2-dimensional line element:
\begin{equation*}
  \dd s^2 = a^2(\eta) \left[ \dd \eta^2 - \dd \chi^2 \right]
\end{equation*}
where, for a flat spacetime, $ \chi \equiv r $ as of \eref{eq:conformal-coordinates}. Since photons are massless, they move along null geodesics with $ \dd s^2 = 0 $, hence:
\begin{equation*}
  \Delta \chi (\eta) = \pm \Delta \eta
\end{equation*}
where the $ + $ describes outgoing photons and the $ - $ incoming photons. This means that the distance that time travels between $ \eta_1 $ and $ \eta_2 > \eta_1 $ is simply $ \Delta \eta = \eta_2 - \eta_1 $. Hence, assuming that the Big Bang started with a singularity\footnotemark at $ t_\text{i} \equiv 0 $, then the greatest distance from which an observer at time $ t $ is able to receive signals is:
\begin{equation}
  \chi_\text{p}(\eta) = \eta - \eta_\text{i} = \int_{t_\text{i}}^t \frac{\dd t}{a(t)}
\end{equation}
which is called the \bctxt{comoving particle horizon}. The particle horizon is obtained intersecting the past light-cone of the observer with the spacelike hypersurface $ \eta = \eta_\text{i} $, as shown in \figref{fig:particle-horizon}: causal influences on the observer can only come from within this region.

\footnotetext{Note that the Big Bang singularity is a moment in time, but \emph{not} point in space: indeed, in general it is pictured as an extended (possibly infinite) spacelike hypersurface.}

The comoving particle horizon $ \chi_\text{p} $ can be expressed in terms of the \emph{comoving Hubble radius} $ (aH)^{-1} $ as follows:
\begin{equation*}
  \chi_\text{p}(\eta) = \int_{a_\text{i}}^{a} \frac{\dd a}{a \dot{a}} = \int_{\ln a_\text{i}}^{\ln a} \frac{\dd \ln a}{a H}
\end{equation*}
where $ a_\text{i} $ corresponds to the Big Bang singularity. For a universe dominated by a single matter component with parameter $ w $, \eref{eq:a-sing-comp} determines the evolution of the comoving Hubble radius:
\begin{equation}
  (aH)^{-1} = H_0^{-1} a^{\frac{1}{2} \left( 1 + 3w \right)}
\end{equation}
Note that for ordinary matter sources the strong energy condition (SEC) $ 1 + 3w > 0 $ is satisfied, in which case the integral expression of $ \chi_\text{p} $ is dominated by the upper limit, while receiving vanishing contributions from earlier times. Performing the explicit integration:
\begin{equation}
  \chi_\text{p}(a) = \frac{2H_0^{-1}}{1 + 3 w} \left[ a^{\frac{1}{2} \left( 1 + 3 w \right)} - a_\text{i}^{\frac{1}{2} \left( 1 + 3 w \right)} \right] \equiv \eta - \eta_\text{i}
\end{equation}
Clearly, in presence of the SEC, $ \eta_\text{i} \ra 0 $ as $ a_\text{i} \ra 0 $, hence setting $ a_\text{i} \equiv 0 $ yields a finite comoving particle horizon:
\begin{equation}
  \chi_\text{p}(t) = \frac{2H_0^{-1}}{1 + 3 w} a^{\frac{1}{2} \left( 1 + 3 w \right)} = \frac{2}{1 + 3 w} (aH)^{-1}
\end{equation}
In the standard cosmology, then, $ \chi_\text{p} \sim (aH)^{-1} $, since the Big Bang model postulates $ a_\text{i} \ra 0 $.

\begin{figure}
  \centering
  \includegraphics[width = 0.70 \textwidth]{particle-horizon.png}
  \caption{Spacetime diagram in the $ \chi - \eta $ coordinate system illustrating the particle horizon.}
  \label{fig:particle-horizon}
\end{figure}

\subsection{CMB uniformity}

About $ 380'000 $ years after the Big Bang, the universe had cooled enough to allow the formation of the first hydrogen atoms: in this process, photons decoupled from the primordial plasma, a phenomenon known as Recombination (LINK TO SECTION) and observed today in the form of the cosmic microwave background (CMB).

The CMB is almost perfectly isotropic, with anisotropies in the order of $ 10^{-5} $, and this is a serious problem for the Big Bang model. Indeed, $ a_\text{CMB} \simeq 1100^{-1} $ and, since $ a(\eta) \sim \eta^2 $ for a matter-dominated universe, $ \eta_\text{CMB} \sim \sqrt{a_\text{CMB}} $, which determines a comoving particle horizon of $ \chi_\text{CMB} = \eta_\text{CMB} \sim 1100^{-1/2} \sim 1^\circ $: this means that any two points in the CMB that are separated by more than $ 1^\circ $ in the sky are causally-disconnected.

The fact that the $ \sim 10^4 $ causally-disconnected patches of the CMB are remarkably isotropic is known as the \emph{horizon problem}, since the comoving particle horizon in the Big Bang model is not sufficient to explain the isotropy of the CMB: there is not enough conformal time between the Big Bang and the CMB in the standard cosmology, as pictured in \figref{fig:horizon-problem}.

\begin{figure}[b]
  \centering
  \includegraphics[width = 0.60 \textwidth]{horizon-problem.png}
  \caption{The horizon problem in the standard Big Bang model.}
  \label{fig:horizon-problem}
\end{figure}

\subsection{SEC violation and Inflation}

To explain the uniformity of the CMB, more conformal time between the Big Bang and Recombination is needed. In order to do so, the SEC must be violated: if e.g. $ w = - \frac{1}{3} $, then from a certain point in time $ (aH)^{-1} $ remains constant going backwards, resulting in an infinite available conformal time for the various patches of the CMB to come into causal contact with each other.

However, the preferred model is one with $ w = -1 $ (constrained by other observations): in this case, the comoving Hubble radius expands going backwards in the past, as shown in \figref{fig:inflation}. Indeed:
\begin{equation*}
  \frac{\dd}{\dd t} (aH)^{-1} = \frac{1 + 3 w}{2} \frac{1}{a} < 0
\end{equation*}
which confirms the shrinking of the comoving Hubble radius, i.e. its expansion going backwards. Moreover:
\begin{equation*}
  \frac{\dd}{\dd t} (aH)^{-1} = \frac{\dd}{\dd t} \dot{a}^{-1} = - \frac{\ddot{a}}{\dot{a}^2}
\end{equation*}
which means that a shrinking comoving Hubble radius implies an accelerated expansion, as $ \ddot{a} > 0 $. For this reason, this is called an \emph{inflationary model}: in this model, the $ \eta = 0 $ surface is no longer the initial singulary of the Big Bang, but merely the end of the inflationary period, i.e. the transition between inflationary cosmology and standard cosmology, while the Big Bang singularity is pushed to negative conformal time:
\begin{equation*}
  \eta_\text{i} = \frac{2H_0^{-1}}{1 + 3 w} a_\text{i}^{\frac{1}{2} \left( 1 + 3 w \right)}
  \xrightarrow{\quad a_\text{i} \ra 0 \ , \ w < - \frac{1}{3} \quad} - \infty
\end{equation*}
Moreover, since $ w = -1 $ in the inflationary period, the energy density of the dominant form of matter $ \rho_\text{I} $ was (approximately) constant, resulting in a constant Hubble parameter $ H_\text{I} $ and an exponential evolution of the scale parameter: $ a(t) = e^{H_\text{I} t} $, from \eref{eq:a-sing-comp}. The inflationary metric is then (approximately) expressed as:
\begin{equation*}
  \dd s^2 = \dd t^2 - e^{2H_\text{I} t} \dd \ve{x}^2
\end{equation*}
which is a de Sitter metric. In fact, Inflation is considered a period of quasi-de Sitter expansion.

\begin{figure}
  \centering
  \includegraphics[width = 0.60 \textwidth]{inflation.png}
  \caption{Inflationary solution to the horizon problem.}
  \label{fig:inflation}
\end{figure}

\subsubsection{Estimates}

Measurements on the CMB allow to give some estimates on the inflationary period. In particular, in the power spectrum of the CMB only wavelength smaller than today's comoving Hubble horizon are observed, meaning that, at the start of inflation, the comoving Hubble radius had to be at least the same as today, if not larger, i.e. $ (a_\text{I} H_\text{I})^{-1} \gtrsim (a_0 H_0)^{-1} $.

On the other hand, the end of Inflation is conventionally taken to be around the GUT phase transition, at $ T_\text{GUT} \sim 10^{15} \gev $. Assuming that, after Inflation, the universe has been dominated by radiation only:
\begin{equation*}
  \frac{a_0 H_0}{a_\text{e} H_\text{e}} = \frac{H_0^{-1}}{H_0^{-1} a_\text{e}^{-1}} = a_\text{e} = \frac{T_{\text{CMB},0}}{T_\text{e}} \sim 10^{-28}
\end{equation*}
where $ T_{\text{CMB},0} \simeq 0.2 \mmev $ is the CMB temperature measured today. Then, since during Inflation $ w = -1 $, i.e. $ (aH)^{-1} \sim a^{-1} $, while after Inflation $ w = \frac{1}{3} $, i.e. $ (aH)^{-1} \sim a $, hence logarithmically $ (a_\text{I} H_\text{I})^{-1} \sim (a_0 H_0)^{-1} $ means that $ \left( \ln a_0 - \ln a_\text{e} \right) \sim 2 \left( \ln a_0 - \ln a_\text{I} \right) $, i.e. $ a_\text{I} \sim 10^{-56} $.

\newpage
\section{Inflationary model}

\subsection{Scalar field dynamics}

As a simple model for Inflation, consider a real scalar field $ \phi(t,\ve{x}) $, the \bctxt{inflaton}, subject to a potential $ V(\phi) $: if the energy-momentum carried by the inflaton field dominates the universe, then it determines the evolution of the FLRW background. From \eref{eq:em-tens-scalar} (recall the change of signature):
\begin{equation*}
  T_{\mu \nu} = \pa_\mu \phi \pa_\nu \phi - g_{\mu \nu} \left[ \frac{1}{2} \pa^\rho \phi \pa_\rho \phi - V(\phi) \right]
\end{equation*}
Consistency with the symmetries of the FLRW spacetime requires that $ \phi = \phi(t) $, hence, from $ \tensor{T}{^0_0} = \rho_\phi $, the energy density of the inflaton field is found to be:
\begin{equation}
  \rho_\phi = \frac{1}{2} \dot{\phi}^2 + V(\phi)
\end{equation}
which is simply the sum of the kinetic energy density and the potential energy density. Their difference is instead the pressure of the inflaton field, as inferred from $ \tensor{T}{^i_j} = - P_\phi \delta^i_j $:
\begin{equation}
  P_\phi = \frac{1}{2} \dot{\phi}^2 - V(\phi)
\end{equation}
Since inflation requires $ P_\phi < - \frac{1}{3} \rho_\phi $, for the inflaton the potential energy must dominate over the kinetic energy. Inserting $ \rho_\phi $ into \eref{eq:fried-hubble} (rewritten in terms of the reduced Planck mass\footnotemark):
\begin{equation}
  H^2 = \frac{1}{3\pl^2} \left[ \frac{1}{2} \dot{\phi}^2 + V(\phi) \right]
  \label{eq:inflaton-h2}
\end{equation}

\begin{proposition}{Klein--Gordon Equation}{}
  The equation of motion for the inflaton field is:
  \begin{equation}
    \ddot{\phi} + 3H \dot{\phi} + V_{,\phi} = 0
  \end{equation}
\end{proposition}

\begin{proofbox}
  \begin{proof}
    First, from \ceref{eq:r-p}{eq:fried-hubble}:
    \begin{equation*}
      2H\dot{H} = \frac{\dot{\rho}}{3\pl^2} = - \frac{H}{\pl^2} \left( \rho + P \right)
      \quad \implies \quad
      \dot{H} = - \frac{\rho + P}{2\pl^2}
    \end{equation*}
    Since $ \rho_\phi + P_\phi = \dot{\phi} $, inserting this equation in the time derivative of \eref{eq:inflaton-h2}:
    \begin{equation*}
      2H\dot{H} = \frac{1}{3\pl^2} \left[ \dot{\phi} \ddot{\phi} + V_{,\phi} \dot{\phi} \right]
      \quad \implies \quad
      \ddot{\phi} + 3H \dot{\phi} + V_{,\phi} = 0
    \end{equation*}
    which is the thesis.
  \end{proof}
\end{proofbox}

Note that the evolution of the inflaton field is sourced by the potential, which acts like a force $ V_{,\phi} $, while the expansion of the universe determines a \emph{friction} term $ 3 H \dot{\phi} $.

\footnotetext{In place of Newton's constant $ G $, it is useful to use the reduced Planck mass:
\begin{equation*}
  \pl \equiv \sqrt{\frac{\hbar c}{8\pi G}} \simeq 2.4 \cdot 10^{18} \,\text{GeV}
\end{equation*}
so that \eref{eq:fried-hubble} for a flat spacetime can be written as $ H^2 = \rho / (3\pl^2) $.}

\subsubsection{Slow-roll inflation}

Near a minimum $ \phi_0 $, the potential can be approximated by a harmonic potential:
\begin{equation*}
  V(\phi) \approx V_0 + \frac{1}{2} m^2 (\phi - \phi_0)^2
\end{equation*}
with $ V_0 \equiv V(\phi_0) $ and $ m^2 \equiv V_{,\phi\phi}(\phi_0) $. Since near the minimum $ \dot{\phi} \approx 0 $, the energy-momentum tensor can be written as:
\begin{equation*}
  \tensor{T}{^\mu_\nu} =
  \begin{bmatrix}
    \rho_\phi & 0 & 0 & 0 \\
    0 & P_\phi & 0 & 0 \\
    0 & 0 & P_\phi & 0 \\
    0 & 0 & 0 & P_\phi
  \end{bmatrix}
  \approx
  \begin{bmatrix}
    V_0 & 0 & 0 & 0 \\
    0 & -V_0 & 0 & 0 \\
    0 & 0 & -V_0 & 0 \\
    0 & 0 & 0 & -V_0
  \end{bmatrix}
\end{equation*}
Comparing to \eref{eq:lambda-emt}, it is clear that $ V_0 $ acts like a cosmological constant $ \Lambda $. Moreover, near the minimum $ w \approx -1 $ as sought.

However, with a harmonic potential the inflaton field does not significantly evolve over time, while in the inflationary cosmology Inflation must only occur between $ a_\text{I} $ and $ a_\text{e} $. To achieve this, a so-called ``slow-roll model" is introduced: the potential $ V(\phi) $ starts nearly constant, with a slightly negative slope, which imposes the condition $ T \gg V $ and results into $ w \approx -1 $. Then, the potential smoothly changes from linear to parabolic, reaching finally a minimum. An example of potential is drawn in \figref{fig:slow-roll}: the formulation of a precise model is futile, since it is not possible to empirically probe Inflation.

\begin{figure}
  \centering
  \includegraphics[width = 0.70 \textwidth]{slow-roll.png}
  \caption{Example of slow-roll potential: inflation occurs in the shaded parts of the potential.}
  \label{fig:slow-roll}
\end{figure}

\subsection{Reheating}

During Inflation, the energy density of the universe is dominated by the inflaton potential $ V(\phi) $. Then, as Inflation ends and the inflaton field gains kinetic energy, the energy of the inflaton sector has to be transferred to the Standard Model sector: this process is called \bctxt{Reheating}.

Setting WLOG $ V_0 \equiv 0 $, the homogeneous evolution of the inflaton field $ \phi = \phi(t) $ reads:
\begin{equation*}
  \ddot{\phi} = - m^2 \phi - 3H \dot{\phi}
\end{equation*}
The expansion timescale soon becomes much larger than the oscillation period, i.e. $ H^{-1} \gg m^{-1} $, hence the friction term can be neglected. Then, the continuity equation can be written as:
\begin{equation*}
  \dot{\rho}_\phi + 3H \rho_\phi = - 3H P_\phi = - \frac{3}{2} H \left[ m^2 \phi^2 - \dot{\phi}^2 \right]
\end{equation*}
Over a period of oscillation the term in parentheses averages to zero, hence the oscillating inflaton field behaves like pressureless matter with $ \rho_\phi \propto a^{-3} $, while the decrease of its energy density is reflected in a decrease of its oscillation amplitude.

After Reheating, the universe is dominated by radiation and its evolution is described by the standard cosmology.










