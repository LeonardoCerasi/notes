\selectlanguage{english}

\section{Differentiable manifolds}

\begin{definition}{Topological space}{}
  The \bcdef{topology} $ \mathcal{T} $ of a set $ X $ is a family of subsets of $ X $, i.e. $ \mathcal{T} \subseteq \pwst{X} $, defined as \bcdef{open sets}, with the following properties:
  \begin{enumerate}
    \item $ \emptyset,X \in \mathcal{T} $;
    \item $ O_{\alpha},O_{\beta} \in \mathcal{T} \implies O_{\alpha}\cap O_{\beta} \in \mathcal{T} $;
    \item $ \{O_{\alpha}\}_{\alpha \in \mathcal{I}} \subset \mathcal{T} \implies \bigcup_{\alpha \in \mathcal{I}} O_{\alpha} \in \mathcal{T} $.
  \end{enumerate}
  A \bcdef{topological space} $ M $ is a set of points, endowed with a topology $ \mathcal{T} $.
\end{definition}

Given a topological space $ (M,\mathcal{T}) $, $ O \in \mathcal{T} $ is a \bctxt{neighbourhood} of a point $ p \in M $ if $ p \in O $: then, $ (M,\mathcal{T}) $ is \bctxt{Hausdorff} if $ \forall p,q \in M \,\, \exists O_1, O_2 \in \mathcal{T} $ neighbourhoods of $ p $ and $ q $ respectively such that $ O_1 \cap O_2 = \emptyset $.

Topological spaces allow to introduce the concept of continuity: given two topological spaces $ (M_1, \mathcal{T}_1) $ and $ (M_2, \mathcal{T}_2) $, a map $ f : M_1 \ra M_2 $ is \bctxt{continuous} if $ O \in \mathcal{T}_2 \,\Rightarrow\, f^{-1}(O)\in \mathcal{T}_1 $.

\begin{definition}{Homeomorphism}{}
  Given two topological spaces $ (M_1, \mathcal{T}_1) $ and $ (M_2, \mathcal{T}_2) $, a map $ f : M_1 \rightarrow M_2 $ is a \bcdef{homeomorphism} if it is bijective and bicontinuous, i.e. both $ f $ and $ f^{-1} $ are continuous.
\end{definition}

\subsection{Definitions}

\begin{definition}{Differentiable manifold}{}
  An $ n $-dimensional \bcdef{differentiable manifold} $ \mathcal{M} $ is a Hausdorff topological space such that:
  \begin{enumerate}
    \item $ \mathcal{M} $ is locally homeomorphic to $ \R^n $, i.e. $ \forall p\in\mathcal{M} \,\, \exists O \in \mathcal{T}(\mathcal{M}) : p \in O \land \exists \varphi : O \rightarrow U \in \mathcal{T}(\R^n) $ homeomorphism;
    \item given $ O_{\alpha},O_{\beta} \in \mathcal{T}(\mathcal{M}) : O_{\alpha} \cap O_{\beta} \neq \emptyset $, the corresponding maps $ \varphi_{\alpha} : O_{\alpha} \rightarrow U_{\alpha}$ and $ \varphi_{\beta} : O_{\beta} \rightarrow U_{\beta} $ must be \textit{compatible}, i.e. $ \varphi_{\beta} \circ \varphi_{\alpha}^{-1} : \varphi_{\alpha}(O_{\alpha} \cap O_{\beta}) \rightarrow \varphi_{\beta}(O_{\alpha} \cap O_{\beta}) $ and its inverse must be smooth (of $ \mathcal{C}^{\infty} $ class).
  \end{enumerate}
\end{definition}

The maps $ \varphi_{\alpha} $ are called \bctxt{charts} and a collection of compatible charts is called an \bctxt{atlas}: a maximal atlas $ \mathcal{A} $ is an atlas such that $ \bigcup_{\alpha \in \mathcal{I}} O_{\alpha} = \mathcal{M} $. Two atlases are compatible if each chart of one atlas is compatible with every chart of the other: they define the same differentiable structure on the manifold.

Each chart $ \varphi_{\alpha} $ provides a coordinate system on $ O_{\alpha} $: $ \varphi_{\alpha}(p) = \left( x^1(p), \dots, x^{\mu}(p), \dots, x^n(p) \right) $. The transition functions $ \varphi_{\beta} \circ \varphi_{\alpha}^{-1} $ are therefore coordinate transformations on overlapping regions.

\begin{example}{Spheres}{}
  $ \mathbb{S}^n $ is a differentiable manifold for $ n \in \N $. In particular, to define a differentiable structure on $ \mathbb{S}^1 $, an atlas of two charts is needed: the standard parametrization $ \vartheta \in [0, 2\pi) $ is not a well-defined chart because $ [0,2\pi) $ is not an open set in the Euclidean topology of $ \R $, therefore the elimination of a point is necessary; usually, the two charts of the atlas are defined by $ \vartheta_1 \in (0,2\pi) $, excluding $ (1,0) $ (in the embedding space $ \R^2 $), and $ \vartheta_2 \in (-\pi,\pi) $, excluding $ (-1,0) $: they are evidently compatible, thus they form a maximal atlas.
\end{example}

In the remainder of these notes, $ \mathcal{M} $ is always taken to be an $ n $-dimensional differentiable manifold.

\subsection{Maps between manifolds}

Locally mapping $ \mathcal{M} $ to $ \R^n $ allows for the extension of the concepts of Analysis from $ \R^n $ to $ \mathcal{M} $.

\begin{definition}{Smooth maps}
  A function $ f : \mathcal{M} \rightarrow \R $ on a differentiable manifold $ (\mathcal{M},\mathcal{A}) $ is \bcdef{smooth} if $ f \circ \varphi_{\alpha}^{-1} : U_{\alpha} \rightarrow \R $ is smooth for all charts $ (U_{\alpha},\varphi_{\alpha}) \in \mathcal{A} $.

  A map $ f : \mathcal{M} \rightarrow \mathcal{N} $ between two differentiable manifolds $ (\mathcal{M},\mathcal{A}_1), (\mathcal{N},\mathcal{A}_2) $ is \bcdef{smooth} if $ \psi_{\alpha_2} \circ f \circ \varphi_{\alpha_1}^{-1} : U_{\alpha_1} \rightarrow V_{\alpha_2} $ is smooth for all charts $ (U_{\alpha_1},\varphi_{\alpha_1}) \in \mathcal{A}_1, (V_{\alpha_2},\varphi_{\alpha_2}) \in \mathcal{A}_2 $.
\end{definition}

A smooth homeomorphism $ f : \mathcal{M} \ra \mathcal{N} $ between two differentiable manifolds $ \mathcal{M} $ and $ \mathcal{N} $ is called a \bctxt{diffeomorphism}.

\begin{proposition}{Diffeomorphic manifolds}{}
  If $ \mathcal{M} $ and $ \mathcal{N} $ are \bcprop{diffeomorphic}, then $ \dim_{\R}\mathcal{M} = \dim_{\R}\mathcal{N} $.
\end{proposition}

\begin{example}{Diffenetiable structures}{}
  $ \Sn{7} $ can be covered by multiple incompatible atlases: the resulting manifolds are homeomorphic but not diffeomorphic.

  $ \R^n $ has a unique differentiable structure for all $ n \in \N $, except for $ n = 4 $: $ \R^4 $ can be covered by infinitely-many incompatible atlases.
\end{example}

\newpage
\section{Tangent spaces}

The notions of calculus can be defined on a differential manifold $ (\mathcal{M},\mathcal{A}) $ with the notion of tangent spaces. Indeed, the derivative of a function $ f : \mathcal{M} \rightarrow \R $ at a point $ p \in \mathcal{M} $, covered by the chart $ (\varphi,U) $, is defined as:
\begin{equation}
  \frac{\pa f}{\pa x^{\mu}}\bigg\vert_p \defeq \frac{\pa (f \circ \varphi^{-1})}{\pa x^{\mu}}\bigg\vert_{\varphi(p)}
  \label{eq:derivative-def}
\end{equation}

Evidently, this definition depends on the choise of coordinates $ x^{\mu} $, thus it depends on the chart.

\subsection{Tangent vectors}

\begin{definition}{Space of smooth functions}{}
  The set of all smooth functions on $ \mathcal{M} $ is denoted by $ \cm $.
\end{definition}

\begin{definition}{Tangent vector}{tangent-vector}
  A \bcdef{tangent vector} to $ \mathcal{M} $ in $ p \in \mathcal{M} $ is an operator $ X_p : \cm \rightarrow \R $ such that:
  \begin{enumerate}
    \item $ X_p(f + g) = X_p(f) + X_p(g) \,\, \forall f,g \in\cm $;
    \item $ X_p(f) = 0 $ for all constant functions $ f \in \cm $;
    \item $ X_p(fg) = X_p(f)g(p) + f(p)X_p(g) \,\, \forall f,g \in\cm $.
  \end{enumerate}
\end{definition}

Conditions 2. and 3. trivially imply that $ X_p(\alpha f) = \alpha X_p(f) \,\, \forall \alpha \in \R $, which means that $ X_p $ is a linear operator, i.e. $ X_p \in \Hom_\R(\cm , \R) $. Moreover, it is simple to check that $ \pa_{\mu}\vert_p $ satisfies the conditions of \dref{def:tangent-vector}.

\begin{theorem}{Tangent space}{}
  The set $ T_p\mathcal{M} $ of all tangent vectors at a point $ p\in\mathcal{M} $ forms an $ n $-dimensional space, called \bcth{tangent space}, and $ \{\pa_{\mu}\vert_p\}_{\mu = 1,\dots,n} $ is a basis of such space.
\end{theorem}

\begin{proofbox}
  \begin{proof}
    Defining $ f \circ \varphi^{-1} \equiv F : U \subset \mathcal{M} \rightarrow \R $, with $ f : \mathcal{M} \rightarrow \mathcal{M} $ and $ (\varphi,U) \in \mathcal{A} $, it can be shown that, in some neighbourhood of $ p $ (not necessarily $ U $), $ F $ cal always be written as:
    \begin{equation*}
      F(x) = F(x^{\mu}(p)) + \left( x^{\mu} - x^{\mu}(p) \right) F_{\mu}(x)
    \end{equation*}
    for some functions $ \{F_{\mu}\}_{\mu = 1 , \dots , n} $ (e.g. $ F(x) = F(0) + x \int_0^1 \dd t\,F(xt) $). Applying $ \pa_{\mu}\vert_{x(p)} $:
    \begin{equation*}
      \frac{\pa F}{\pa x^{\mu}}\bigg\vert_{x(p)} = F_{\mu}(x(p))
    \end{equation*}
    Defining $ f_{\mu} \equiv F_{\mu} \circ \varphi $, for any $ q \in \mathcal{M} $ in an appropriate neighbourhood of $ p $:
    \begin{equation*}
      f(q) = f(p) + \left( x^{\mu}(q) - x^{\mu}(p) \right) f_{\mu}(q)
    \end{equation*}
    Moreover, remembering Eq. \eref{eq:derivative-def}:
    \begin{equation*}
      f_{\mu}(p) = F_{\mu} \circ \varphi(p) = F_{\mu}(x(p)) = \frac{\pa F}{\pa x^{\mu}}\bigg\vert_{x(p)} = \frac{\pa f}{\pa x^{\mu}}\bigg\vert_p
    \end{equation*}
    Using these facts, the action of a tangent vector can be written explicitly:
    \begin{equation*}
      \begin{split}
        X_p(f)
        &= X_p\left( f(p) + \left( x^{\mu} - x^{\mu}(p) \right) f_{\mu} \right)\\
        &= X_p\left( f(p) \right) + X_p\left( \left( x^{\mu} - x^{\mu}(p) \right) \right) f_{\mu}(p) + \left( x^{\mu} - x^{\mu}(p) \right)(p) X_p\left( f_{\mu} \right)\\
        &= X_p\left( x^{\mu} \right) f_{\mu}(p)
      \end{split}
    \end{equation*}
    because $ f(p) $ is a constant and $ \left( x^{\mu} - x^{\mu}(p) \right)(p) = x^{\mu}(p) - x^{\mu}(p) = 0 $. Therefore, remembering the expression for $ f_{\mu}(p) $:
    \begin{equation*}
      X_p = X_p(x^{\mu}) \frac{\pa}{\pa x^{\mu}}\bigg\vert_p \equiv X^{\mu} \frac{\pa}{\pa x^{\mu}}\bigg\vert_p
    \end{equation*}
    Thus, $ T_p\mathcal{M} = \braket{\{\pa_{\mu}\vert_p\}_{\mu = 1 , \dots , n}} $. To check for linear independence, suppose $ \alpha = \alpha^{\mu} \pa_{\mu}\vert_p \equiv 0 $: acting on $ f = x^{\nu} $, it gives $ \alpha(f) = \alpha_{\mu} \pa_{\mu}(x^{\nu})\vert_p = \alpha_{\nu} = 0 $. This concludes the proof.
  \end{proof}
\end{proofbox}

\subsubsection{Changing coordinates}

Although $ \pa_{\mu}\vert_p $ depends on the choice of coordinates (it is a \bctxt{coordinate basis}), the existence of $ X_p $ is independent of that choice.
If two different charts $ (\varphi,U) $ and $ (\tilde{\varphi},V) $ intersect in a neighbourhood of $ p \in U \cap V $, the transition from $ x^{\mu} $ to $ y^{\mu} $ can be expressed as:
\begin{equation}
  X_p(f) = X^{\mu} \frac{\pa f}{\pa x^{\mu}}\bigg\vert_p = X^{\mu} \frac{\pa y^{\nu}}{\pa x^{\mu}}\bigg\vert_{\varphi(p)} \frac{\pa f}{\pa y^{\nu}}\bigg\vert_p
\end{equation}
This equation can have two interpretations, namely the alibi and the alias interpretation:
\begin{equation}
  \frac{\pa}{\pa x^{\mu}}\bigg\vert_p = \frac{\pa y^{\nu}}{\pa x^{\mu}}\bigg\vert_{\varphi(p)} \frac{\pa}{\pa y^{\nu}}\bigg\vert_p
  \qquad \qquad
  \tilde{X}^{\nu} = X^{\mu} \frac{\pa y^{\nu}}{\pa x^{\mu}}\bigg\vert_{\varphi(p)}
  \label{eq:vector-coord-basis}
\end{equation}
Components of vectors which transform this way are called \bctxt{contravariant}.

\subsubsection{Curves}

Consider a smooth curve on $ \mathcal{M} $, i.e. a smooth map $ \sigma : I \in \mathcal{T}(\R) \rightarrow \mathcal{M} $, WLOG parametrized as $ \sigma(t) : \sigma(0) = p \in \mathcal{M} $; with a given chart $ (\varphi,U) $, this curve becomes $ \varphi \circ \sigma : I \rightarrow \R^n $, parametrized by $ x^{\mu}(t) $.
The tangent vector to the curve in $ p $ is:
\begin{equation}
  X_p = \frac{dx^{\mu}(t)}{dt}\bigg\vert_{t=0} \frac{\pa}{\pa x^{\mu}}\bigg\vert_p
  \label{eq:tangent-curve}
\end{equation}
This operator, applied to a function $ f \in\cm $, computes the directional derivative of $ f $ along the curve. Every tangent vector can be written as in Eq. \eref{eq:tangent-curve}, therefore the tangent space can be seen as the space of all possible tangent vectors to curves passing through $ p $.

It must be noted that tangent spaces at different points are entirely different spaces: there is no way to directly compare vectors between them.

\subsection{Vector fields}

\begin{definition}{Vector fields}{}
  A \bcdef{vector field} $ X $ is a smooth map $ X : p \in \mathcal{M} \mapsto X_p \in T_p\mathcal{M} $. The space of all vector fields on $ \mathcal{M} $ is denoted by $ \xm $.
\end{definition}

Note that a vector field can also be viewed as a smooth map $ X : \cm \rightarrow \cm $, since $ (X(f))(p) = X_p(f) \in \R $.
Given a chart $ (\varphi,U) $, a vector field $ X $ can be expressed as:
\begin{equation}
  X = X^{\mu} \frac{\pa}{\pa x^{\mu}}
\end{equation}
with $ X^{\mu} \in \cm $. This expression is only defined on $ U $.

\subsubsection{Lie brakets}

Given two vector fields $ X,Y \in\xm $, their product is clearly not a vector field, as it does not satisfy Leibniz' rule:
\begin{equation*}
  XY(fg) = XY(f) g + Y(f) X(g) + X(f) Y(g) + f XY(g) \neq XY(f) g + f XY(g)
\end{equation*}
where $ XY(f) \equiv X(Y(f)) $.

\begin{definition}{Lie brackets}{}
  Given two vector fields $ X,Y \in\xm $, their commutator (or \bcdef{Lie bracket}) is defined as:
  \begin{equation}
    \left[ X,Y \right](f) = XY(f) - YX(f)
  \end{equation}
\end{definition}

With a given chart:
\begin{equation*}
  \begin{split}
    \left[ X,Y \right](f)
    &= X^{\mu} \frac{\pa}{\pa x^{\mu}} \left( Y^{\nu} \frac{\pa f}{\pa x^{\nu}} \right) - Y^{\mu} \frac{\pa}{\pa x^{\mu}} \left( X^{\nu} \frac{\pa f}{\pa x^{\nu}} \right)\\
    &= \left( X^{\mu} \frac{\pa Y^{\nu}}{\pa x^{\mu}} - Y^{\mu} \frac{\pa X^{\nu}}{\pa x^{\mu}} \right) \frac{\pa f}{\pa x^{\nu}}
  \end{split}
\end{equation*}
therefore:
\begin{equation}
  \left[ X,Y \right] = \left( X^{\mu} \frac{\pa Y^{\nu}}{\pa x^{\mu}} - Y^{\mu} \frac{\pa X^{\nu}}{\pa x^{\mu}} \right) \frac{\pa}{\pa x^{\nu}}
\end{equation}

\begin{theorem}{Jacobi identity}{}
  Given $ X,Y,Z \in\xm $, then:
  \begin{equation}
    \left[ X, \left[ Y,Z \right] \right] + \left[ Y, \left[ Z,X \right] \right] + \left[ Z, \left[ X,Y \right] \right] = 0
  \end{equation}
\end{theorem}

With Lie brackets, $ \xm $ can be given the structure of a Lie algebra.

\subsubsection{Integral curves}

\begin{definition}{Flow}{flow}
  A \bcdef{flow} on $ \mathcal{M} $ is a one-parameter family of diffeomorphisms $ \sigma_t : \mathcal{M} \rightarrow \mathcal{M} $, labelled by $ t\in\R $, with group structure: $ \sigma_0 = \id_{\mathcal{M}} $ and $ \sigma_s \circ \sigma_t = \sigma_{s+t} $, thus $ \sigma_{-t} = \sigma_t^{-1} $.
\end{definition}

Such flows give rise to streamlines on the manifold: these streamlines are required to be smooth.
Defining $ x^{\mu}(\sigma_t) \equiv x^{\mu}(t) $, a vector field can be defined by the tangent to the streamlines at each point on the manifold:
\begin{equation}
  X^{\mu}(x^{\mu}(t)) = \frac{\dd x^{\mu}(t)}{\dd t}
  \label{eq:tangent-flow}
\end{equation}
The inverse reasoning is also possible: given $ X \in \xm $, the streamlines defined by \eref{eq:tangent-flow} are the \bctxt{integral curves} of $ X $.

\begin{proposition}{Infinitesimal flow}{}
  The \bcprop{infinitesimal flow} generated by $ X \in\xm $ is:
  \begin{equation}
    x^{\mu}(t) = x^{\mu}(0) + tX^{\mu}(x(t)) + o(t)
    \label{eq:infinitesimal-flow}
  \end{equation}
\end{proposition}

A vector field which generates a flow defined for all $ t \in \R $ is called \bctxt{complete}.

\begin{theorem}{}
  If $ \mathcal{M} $ is compact, then all $ X \in \xm $ are complete.
\end{theorem}

\begin{example}{Integral curves on the $ 2 $-sphere}{}
  On $ \mathbb{S}^2 $, the flow generated by $ X = \pa_{\phi} $ is described by $ \dot{\phi} = 1, \dot{\theta} = 0 $, thus $ \theta(t) = \theta_0 $ and $ \phi(t) = \phi_0 + t $: the flow lines are lines of constant latitude.
\end{example}

\subsection{Lie derivative}

Defining calculus for vector fields requires a way to compare vectors of different tangent spaces.

\begin{definition}{Pull-back of functions}{}
  Given a diffeomorphism between two manifolds $ \varphi : \mathcal{M} \rightarrow \mathcal{N} $ and a function $ f : \mathcal{N} \rightarrow \R $, the \bcdef{pull-back} of $ f $ is the function $ \varphi^*f : \mathcal{M} \rightarrow \R $ such that $ \varphi^*f(p) = f(\varphi(p)) $.
\end{definition}

\begin{definition}{Push-forward of vector fields}{}
  Given a diffeomorphism between two manifolds $ \varphi : \mathcal{M} \rightarrow \mathcal{N} $ and a vector field $ X \in \xm $, the \bcdef{push-forward} of $ X $ is the vector field $ \varphi_*X \in \mathfrak{X}(\mathcal{N}) $ such that $ \varphi_*X(f) = X(\varphi^*f) $.
\end{definition}

This last equality must be evaluated at the appropriate points: $ [ \varphi_*X(f) ](\varphi(p)) = [ X(\varphi^*f) ](p) $.

With the appropriate charts on $ \mathcal{M} $ and $ \mathcal{N} $, the definitions above can be rewritten with coordinates:
\begin{equation}
  \varphi^*f(x) = f(y(x))
\end{equation}
\begin{equation}
  \varphi_*X(f) = X^{\mu} \frac{\pa f(y(x))}{\pa x^{\mu}} = X^{\mu} \frac{\pa y^{\alpha}}{\pa x^{\mu}} \frac{\pa f(y)}{\pa y^{\alpha}}
\end{equation}

The notions of pull-back and push-forward allow to compare tangent vectors at neighbouring points and, in particular, to define the derivative along a vector field.

\begin{definition}{Lie derivative of functions}{}
  Given a function $ f : \mathcal{M} \rightarrow \R $ and a vector field $ X \in\xm $, the derivative of $ f $ along $ X $, called \bcdef{Lie derivative}, is defined as:
  \begin{equation}
    \ld_X f(x) \defeq \lim_{t \rightarrow 0} \frac{f(\sigma_t(x)) - f(x)}{t} = \frac{df(\sigma_t(x))}{dt}\bigg\vert_{t=0}
  \end{equation}
  where $ \sigma_t $ is the flow generated by $ X $.
\end{definition}

\begin{lemma}[before upper = {\tcbtitle}]{}{}
  \begin{equation}
    \ld_X f = X(f)
  \end{equation}
\end{lemma}

\begin{proofbox}
  \begin{proof}
    $ \ld_X f = \frac{df(\sigma_t)}{dt} = \frac{\pa f}{\pa x^{\mu}} \frac{dx^{\mu}(t)}{dt} = X^{\mu} \frac{\pa f}{\pa x^{\mu}} = X(f) $.
  \end{proof}
\end{proofbox}

The Lie derivative can be extended to vector fields.

\begin{definition}{Lie derivative of vector fields}{}
  Given two vector fields $ X,Y \in\xm $, the \bcdef{Lie derivative} of $ Y $ along $ X $ is defined as:
  \begin{equation}
    \ld_X Y_p \defeq \lim_{t \rightarrow 0} \frac{((\sigma_{-t})_*Y)_p - Y_p}{t}
  \end{equation}
  where $ \sigma_t $ is the flow generated by $ X $.
\end{definition}

The use of the inverse flow $ \sigma_{-t} $ is necessary because to evaluate the vector field $ \ld_X Y $ at the point $ p \in \mathcal{M} $, the tangent vector $ Y_{\sigma_t(p)} \in T_{\sigma_t(p)}\mathcal{M} $ must be ``pushed-back" to $ T_p\mathcal{M} = T_{\sigma_0(p)}\mathcal{M} $.

With $ t \rightarrow 0 $, the infinitesimal flow $ \sigma_{-t} $ is, according to \eref{eq:infinitesimal-flow}, $ x^{\mu}(t) = x^{\mu}(0) - tX^{\mu} + o(t) $, therefore the Lie derivative of basis tangent vectors can be expressed as:
\begin{equation}
  (\sigma_{-t})_* \pa_{\mu} = \frac{\pa x^{\nu}(t)}{\pa x^{\mu}} \frac{\pa}{\pa x^{\nu}(t)} = \left( \delta^{\nu}_{\mu} - t \frac{\pa X^{\nu}}{\pa x^{\mu}} + o(t) \right) \pa_{\nu}(t)
  \quad\Longrightarrow\quad
  \ld_X \pa_{\mu} = - \frac{\pa X^{\nu}}{\pa x^{\mu}} \pa_{\nu}
  \label{eq:lie-der-basis-vector}
\end{equation}
Moreover, by the Jacobi identity it follows that:
\begin{equation}
  \ld_X \ld_Y Z - \ld_Y \ld_X Z = \ld_{\left[ X,Y \right]} Z
  \label{eq:prop-lie-algebra}
\end{equation}

\begin{lemma}[before upper = {\tcbtitle}]{}{}
  \begin{equation}
    \ld_X Y = \left[ X,Y \right]
  \end{equation}
\end{lemma}

\begin{proofbox}
  \begin{proof}
    $ \ld_X Y = \ld_X (Y^{\mu} \pa_{\mu}) = \left( \ld_X Y^{\mu} \right) \pa_{\mu} + Y^{\mu} \left( \ld_X \pa_{\mu} \right) = X^{\nu} \frac{\pa Y^{\mu}}{\pa x^{\nu}} \pa_{\mu} - Y^{\mu} \frac{\pa X^{\nu}}{\pa x^{\mu}} \pa_{\nu} = \left[ X,Y \right] $.
  \end{proof}
\end{proofbox}

\newpage
\section{Tensors}

\subsection{Dual Spaces}

\begin{definition}{Dual space}{}
  Given a vector space $ V $, its \bcdef{dual space} $ V^* $ is the space of all linear maps $ f : V \rightarrow \R $.
\end{definition}

Given a basis $ \{\ve{e}_{\mu}\}_{\mu = 1,\dots,n} $ of $ V $, its \bctxt{dual basis} $ \{\ve{f}^{\mu}\}_{\mu=1,\dots,n} $ of $ V^* $ can be defined by:
\begin{equation}
  \ve{f}^{\nu}(\ve{e}_{\mu}) = \delta^{\nu}_{\mu}
  \label{eq:dual-basis}
\end{equation}
A general vector in $ V $ can be written as $ X = X^{\mu} \ve{e}_{\mu} $, thus, according to \eref{eq:dual-basis}, $ X^{\mu} = \ve{f}^{\mu}(X) $.
Clearly, the map $ f : \ve{e}_{\mu} \mapsto \ve{f}^{\mu} $ is a non-canonical isomorphism between $ V $ and $ V^* $, hence $ \dim_{\R}V = \dim_{\R}V^* $.

\begin{proposition}[before upper = {\tcbtitle}]{Dual of the dual}{}
  \begin{equation}
    (V^*)^* \cong V
  \end{equation}
\end{proposition}

\begin{proofbox}
  \begin{proof}
    The natural isomorphism between $ (V^*)^* $ and $ V $ is basis-independent: suppose $ X \in V $ and $ \omega \in V^* $, so that $ \omega(X) \in \R $; $ X $ can be viewed as $ X \in (V^*)^* $ by setting $ X(\omega) \equiv \omega(X) $.
  \end{proof}
\end{proofbox}

\subsection{Cotangent vectors}

\begin{definition}{Cotangent space}{}
  Given a differentiable manifold $ (\mathcal{M},\mathcal{A}) $ and a point $ p \in \mathcal{M} $, the \bcdef{cotangent space} to $ \mathcal{M} $ at $ p $ is defined as $ T^*_p\mathcal{M} \defeq (T_p\mathcal{M})^* $.
\end{definition}

Elements of $ T^*_p\mathcal{M} $ are called cotangent vectors (or \bctxt{covectors}).

\begin{definition}{Covector field}{}
  A covector field (or \bcdef{1-form}) is a smooth map $ \omega : p \in \mathcal{M} \mapsto \omega_p \in T^*_p\mathcal{M} $. The space of all 1-forms on $ \mathcal{M} $ is denoted by $ \lm{1} $.
\end{definition}

Note that a 1-form can also be viewed as a smooth map $ \omega : \xm \rightarrow \cm $, since $ (\omega(X))(p) = \omega_p(X_p) \in \R $.

\begin{proposition}{}{}
  $ \{\dd x^{\mu}\}_{\mu = 1,\dots,n} $ is a basis of $ \lm{1} $, dual to the basis $ \{\pa_{\mu}\}_{\mu = 1,\dots,n} $ of $ \xm $.
\end{proposition}

\begin{proofbox}
  \begin{proof}
    Consider $ f \in \cm $ and define $ \dd f \in \lm{1} $ by $ \dd f(X) = X(f) $: taking $ f = x^{\mu} $ and $ X = \pa_{\mu} $, $ \dd f(X) = \pa_{\nu}(x^{\mu}) = \delta^{\mu}_{\nu} $, therefore $ \{\dd x^{\mu}\}_{\mu = 1,\dots,n} $ is the dual basis of $ \lm{1} $.
  \end{proof}
\end{proofbox}

This is also confirmed by $ \dd f = \frac{\pa f}{\pa x^{\mu}} \dd x^{\mu} $. These are coordinate bases: in fact, given two different charts $ (\varphi,U), (\tilde{\varphi},V) $:
\begin{equation}
  \dd y^{\mu} = \frac{\dd y^{\mu}}{\dd x^{\nu}}\dd x^{\nu}
\end{equation}
which is the inverse of \eref{eq:vector-coord-basis} (not evaluated at a specific point). This ensures that:
\begin{equation*}
  \dd y^{\mu}\left( \frac{\pa}{\pa y^{\nu}} \right) = \frac{\pa y^{\mu}}{\pa x^{\alpha}} \frac{\pa x^{\beta}}{\pa y^{\nu}} \dd x^{\alpha}\left( \frac{\pa}{\pa x^{\beta}} \right) = \frac{\pa y^{\mu}}{\pa x^{\alpha}} \frac{\pa x^{\alpha}}{\pa y^{\nu}} = \delta^{\mu}_{\nu}
\end{equation*}
A 1-form $ \omega \in \lm{1} $ can thus be expressed both as $ \omega = \omega_{\mu} \dd x^{\mu} = \tilde{\omega}_{\mu} \dd y^{\mu} $, with:
\begin{equation}
  \tilde{\omega}_{\omega} = \frac{\pa x^{\nu}}{\pa y^{\mu}} \omega_{\nu}
\end{equation}
Components of 1-forms which transform this way are called \bctxt{covariant}.

\begin{definition}{Pull-back of 1-forms}{}
  Given a diffeomorphism between two manifolds $ \varphi : \mathcal{M} \rightarrow \mathcal{N} $ and a 1-form $ \omega \in \Lambda^1(\mathcal{N}) $, the \bcdef{pull-back} of $ \omega $ is the 1-form $ \varphi^*\omega \in \lm{1} $ such that $ \varphi^*\omega(X) = \omega(\varphi_*X) $.
\end{definition}

With the appropriate charts on $ \mathcal{M} $ and $ \mathcal{N} $, the definition above can be rewritten with coordinates:
\begin{equation}
  \varphi^*\omega = \omega_{\alpha} \frac{\pa y^{\alpha}}{\pa x^{\mu}} \dd x^{\mu}
\end{equation}

\begin{definition}{Lie derivative of 1-forms}{}
  Given a vector field $ X \in \xm $ and a 1-form $ \omega \in \lm{1} $, the \bcdef{Lie derivative} of $ \omega $ along $ X $ is defined as:
  \begin{equation}
    \ld_X \omega_p \defeq \lim_{t \rightarrow 0} \frac{(\sigma_t^* \omega)_p - \omega_p}{t}
  \end{equation}
  where $ \sigma_t $ is the flow generated by $ X $.
\end{definition}

In contrast with the Lie derivative of a vector field, which pushes forward with $ \sigma_{-t} $ (i.e. pushes back), the Lie derivative of a 1-form pulls back with $ \sigma_t $: this results in the difference of a minus sign with respect to \eref{eq:lie-der-basis-vector}, giving:
\begin{equation}
  \ld_X \dd x^{\mu} = \frac{\pa X^{\mu}}{\pa x^{\nu}} \dd x^{\nu}
\end{equation}
Therefore, on a general 1-form $ \omega = \omega_{\mu} \dd x^{\mu} $:
\begin{equation}
  \ld_X \omega = \left( X^{\nu} \pa_{\nu} \omega_{\mu} + \omega_{\nu} \pa_{\mu} X^{\nu} \right) \dd x^{\mu}
  \label{eq:lie-derivative-1-forms}
\end{equation}

\subsection{Tensor fields}

\begin{definition}{Tensor}{}
  A \bcdef{tensor} of rank $ (r,s) $ at a $ p\in\mathcal{M} $ of a differentiable manifold $ (\mathcal{M},\mathcal{A}) $ is a multilinear map defined as:
  \begin{equation}
    T_p : \overbrace{T^*_p \mathcal{M} \times \dots \times T^*_p \mathcal{M}}^{r} \times \overbrace{T_p \mathcal{M} \times \dots \times T_p \mathcal{M}}^{s} \rightarrow \R
  \end{equation}
\end{definition}

For example, a cotangent vector $ \omega_p \in T^*_p \mathcal{M} $ is a tensor of rank $ (1,0) $, while a tangent vector $ X_p \in T_p \mathcal{M} $ is a tensor of rank $ (0,1) $.

\begin{definition}{Tensor field}{}
  A \bcdef{tensor field} of rank $ (r,s) $ is a smooth map $ T : p \in \mathcal{M} \mapsto T_p $ tensor of rank $ (r,s) $ at $ p $. It can also be viewed as a smooth map $ T : [\lm{1}]^r \times [\xm]^s \rightarrow \cm $.
\end{definition}

Given appropriate bases for vector fields $ \{\ve{e}_{\mu}\}_{\mu = 1,\dots,n} $ and 1-forms $ \{\ve{f}^{\mu}\}_{\mu = 1,\dots,n} $, the components of a tensor field are defined as:
\begin{equation}
  \tensor{T}{^{\mu_1}^{\dots}^{\mu_r}_{\nu_1}_{\dots}_{\nu_s}} \defeq T(\ve{f}^{\mu_1},\dots,\ve{f}^{\mu_r},\ve{e}_{\nu_1},\dots,\ve{e}_{\nu_s})
\end{equation}

\begin{proposition}{Components of tensor fields}{}
  On an $ n $-dimensional manifold, a $ (r,s) $ tensor field has $ n^{r+s} $ components, each being an element of $ \cm $.
\end{proposition}

Consider two general basis transformations, for vector fields and 1-forms, described by invertible matrices $ \mt{A} $ and $ \mt{B} $ such that $ \tilde{\ve{e}}_{\mu} = \tensor{A}{^\nu_\mu}\ve{e}_{\nu} $ and $ \tilde{\ve{f}}^{\mu} = \tensor{B}{^\mu_\nu}\ve{f}^{\nu} $, with necessary condition $ \tensor{A}{^\mu_\nu} \tensor{B}{^\rho_\mu} = \delta^{\rho}_{\nu} $ to ensure duality: this implies $ \mt{B} = \mt{A}^{-1} $, i.e. covectors transform inversely with respect to vectors. Thus:
\begin{equation}
  \tensor{\tilde{T}}{^{\mu_1}^{\dots}^{\mu_r}_{\nu_1}_{\dots}_{\nu_s}} = \tensor{B}{^{\mu_1}_{\rho_1}} \dots \tensor{B}{^{\mu_r}_{\rho_r}} \tensor{A}{^{k_1}_{\nu_1}} \dots \tensor{A}{^{k_s}_{\nu_s}} \tensor{T}{^{\rho_1}^{\dots}^{\rho_r}_{k_1}_{\dots}_{k_s}}
\end{equation}
If the considered basis are coordinate basis, then $ \tensor{A}{^\mu_\nu} = \frac{\pa x^{\mu}}{\pa y^{\nu}} $ and $ \tensor{B}{^\mu_\nu} = \frac{\pa y^{\mu}}{\pa x^{\nu}} $.

\subsection{Operations on tensors}

Algebraic addition and multiplication by functions are trivially defined on tensors of the same rank, and in fact the space of all $ (r,s) $ tensors at a point $ p \in \mathcal{M} $ is a vector space.

\begin{definition}{Tensor product}{}
  Given two tensor fields $ S $ of rank $ (p,q) $ and $ T $ of rank $ (r,s) $, their \bcdef{tensor product} is defined as:
  \begin{equation}
    \begin{split}
      &S \otimes T (\omega_1,\dots,\omega_p,\eta_1,\dots,\eta_r,X_1,\dots,X_q,Y_1,\dots,Y_s)\\
      &\qquad\qquad\qquad = S(\omega_1,\dots,\omega_p,X_1,\dots,X_q) T(\eta_1,\dots,\eta_r,Y_1,\dots,Y_s)
    \end{split}
  \end{equation}
  or, in components:
  \begin{equation}
    \tensor{(S \otimes T)}{^{\mu_1}^{\dots}^{\mu_p}^{\nu_1}^{\dots}^{\nu_r}_{\rho_1}_{\dots}_{\rho_q}_{\dots}_{\sigma_1}_{\dots}_{\sigma_s}} = \tensor{S}{^{\mu_1}^{\dots}^{\mu_p}_{\rho_1}_{\dots}_{\rho_q}} \tensor{T}{^{\nu_1}^{\dots}^{\nu_r}_{\sigma_1}_{\dots}_{\sigma_s}}
  \end{equation}
\end{definition}

It is also possible to contract tensor ($ (r,s) \mapsto (r-1,s-1) $): for example, given a rank $ (2,1) $ tensor, a rank $ (1,0) $ tensor can be defined as $ S(\omega) = T(\omega, \ve{f}^{\mu}, \ve{e}_{\mu}) $, with components $ \tensor{S}{^\mu} = \tensor{T}{^\nu^\mu_\mu} $; obviously, in general, $ \tensor{T}{^\nu^\mu_\mu} \neq \tensor{T}{^\mu^\nu_\mu} $.

It is convenient to introduce some notation: given an object $ T_{\mu_1 \dots \mu_n} $ dependent on some indices, its symmetric and antisymmetric parts are respectively defined as:
  \begin{equation}
    T_{(\mu_1 \dots \mu_n)} \defeq \frac{1}{n!} \sum_{\sigma \in S^n} T_{\sigma(\mu_1) \dots \sigma(\mu_n)}
  \end{equation}
  \begin{equation}
    T_{[\mu_1 \dots \mu_n]} \defeq \frac{1}{n!} \sum_{\sigma \in S^n} \sgn(\sigma) T_{\sigma(\mu_1) \dots \sigma(\mu_n)}
  \end{equation}
Conventionally, indices surrounded by $ |\cdot| $ are ignored, e.g. $ T_{[\mu|\nu|\rho]} = \frac{1}{2} \left( T_{\mu \nu \rho} - T_{\rho \nu \mu} \right) $.

As previously seen, vector fields are pushed forward and 1-form are pulled back: tensors will thus behave in a mixed way.

\begin{definition}{Push-forward of tensors}{}
  Given a diffeomorphism between two manifolds $ \varphi : \mathcal{M} \rightarrow \mathcal{N} $ and a $ (r,s) $ tensor field $ T $ on $ \mathcal{M} $, the \bcdef{push-forward} of $ T $ is the $ (r,s) $ tensor field $ \varphi_* T $ on $ \mathcal{N} $ such that, for $ \{\omega_j\}_{j = 1 , \dots , r} \ssq \Lambda^1(\mathcal{N}) $ and $ \{X_j\}_{j = 1 , \dots , s} \ssq \mathfrak{X}(\mathcal{N}) $:
  \begin{equation}
    \varphi_* T(\omega_1, \dots, \omega_r, X_1, \dots, X_s) = T(\varphi^* \omega_1, \dots, \varphi^* \omega_r, \varphi_*^{-1} X_1, \dots, \varphi_*^{-1} X_s)
  \end{equation}
\end{definition}

\begin{definition}{Lie derivative of tensors}{}
  Given a vector field $ X \in \xm $ and a $ (r,s) $ tensor field $ T $ on $ \mathcal{M} $, the \bcdef{Lie derivative} of $ T $ along $ X $ is defined as:
  \begin{equation}
    \ld_X T_p \defeq \lim_{t \rightarrow 0} \frac{((\sigma_{-t})_* T)_p - T_p}{t}
  \end{equation}
  where $ \sigma_t $ is the flow generated by $ X $.
\end{definition}

\newpage
\section{Differential forms}

\begin{definition}{$ p $-forms}{}
  A \bcdef{$ p $-form} totally anti-symmetric $ (0,p) $ tensor. The set of all $ p $-forms over a manifold $ \mathcal{M} $ is denoted as $ \lm{p} $.
\end{definition}

\begin{lemma}[before upper = {\tcbtitle}]{Independent components of $ p $-forms}{}
  \begin{equation}
    \dim_{\R} \lm{p} = \binom{n}{p}
  \end{equation}
\end{lemma}

The maximum degree of differential forms thus is $ p = n \equiv \dim_{\R}\mathcal{M} $: forms in $ \lm{n} $ are called \bctxt{top forms}.

\begin{definition}{Wedge product}{}
  Given $ \omega \in \lm{p}, \eta \in \lm{q} $, their \bcdef{wedge product} is a $ (p+q) $-form defined as:
  \begin{equation}
    (\omega \wedge \eta)_{\mu_1 \dots \mu_p \nu_1 \dots \nu_q} = \frac{(p + q)!}{p! q!} \omega_{[\mu_1 \dots \mu_p} \eta_{\nu_1 \dots \nu_q]}
  \end{equation}
\end{definition}

\begin{example}{1-forms to 2-form}{}
  Given $ \omega,\eta \in \lm{1} $, their wedge product is $ (\omega \wedge \eta)_{\mu \nu} = \omega_{\mu} \eta_{\nu} - \omega_{\nu} \eta_{\mu} $.
\end{example}

The wedge product is essentially a linear map $ \lm{p} \times \lm{q} \ra \lm{p + q} $. Moreover, it can be shown to be associative.

\begin{proposition}{Skew-symmetry}{}
  Given $ \omega \in\lm{p}, \eta \in\lm{q} $, then:
  \begin{equation}
    \omega \wedge \eta = (-1)^{pq} \eta \wedge \omega
  \end{equation}
\end{proposition}

Then, clearly, $ \omega \wedge \omega = 0 \,\,\forall \omega \in\lm{p} : p $ is odd.

\begin{proposition}{Bases for $ p $-forms}{}
  If $ \{\ve{f}^{\mu}\}_{\mu = 1,\dots,n} $ is a basis of $ \lm{1} $, then $ \{\ve{f}^{\mu_1} \wedge \dots \wedge \ve{f}^{\mu_p}\}_{\mu_1,\dots,\mu_p = 1,\dots,n} $ is a basis of $ \lm{p} $.
\end{proposition}

Locally $ \{\dd x^{\mu}\}_{\mu = 1,\dots,n} $ is a basis of $ T^*_p \mathcal{M} $, thus a general $ p $-form can be locally written as:
\begin{equation}
  \omega = \frac{1}{p!} \omega_{\mu_1 \dots \mu_p} \dd x^{\mu_1} \wedge \dots \wedge \dd x^{\mu_p}
\end{equation}

\begin{definition}{Exterior derivative}{}
  The \bcdef{exterior derivative} is a map $ \dd : \lm{p} \rightarrow \lm{p+1} $ defined as:
  \begin{equation}
    (\dd \omega)_{\mu_1 \dots \mu_{p+1}} = (p + 1) \pa_{[\mu_1} \omega_{\mu_2 \dots \mu_{p+1}]}
  \end{equation}
\end{definition}

In local coordinates:
\begin{equation}
  \dd \omega = \frac{1}{p!} \frac{\pa \omega_{\mu_1 \dots \mu_p}}{\pa x^{\nu}} \dd x^{\nu} \wedge \dd x^{\mu_1} \wedge \dots \wedge \dd x^{\mu_p}
\end{equation}

\begin{theorem}[before upper = {\tcbtitle}]{Poincaré's theorem}{}
  \begin{equation}
    \dd^2 = 0
  \end{equation}
\end{theorem}

\begin{proofbox}
  \begin{proof}
    Consequence of Schwarz lemma.
  \end{proof}
\end{proofbox}

\begin{proposition}{Product rule}{}
  Given $ \omega \in \lm{p}, \eta \in \lm{q} $, then:
  \begin{equation}
    \dd (\omega \wedge \eta) = \dd \omega \wedge \eta + (-1)^p \omega \wedge \dd \eta
    \label{eq:wedge-prod-rule}
  \end{equation}
\end{proposition}

\begin{proposition}{Pull-back and exterior derivative}{}
  Given a diffeomorphism between two manifolds $ \varphi: \mathcal{M} \rightarrow \mathcal{N} $ and $ \omega \in\lm{p} $, then $ \dd (\varphi^* \omega) = \varphi^* (\dd \omega) $.
\end{proposition}

\begin{propcorollary}{Lie derivative and exterior derivative}{}
  Given $ X \in \xm, \omega \in\lm{p} $, then $ \dd (\ld_X \omega) = \ld_X (\dd \omega) $.
\end{propcorollary}

\begin{definition}{Closed and exact forms}{}
  Given $ \omega \in \lm{p} $, it is:
  \begin{itemize}
    \item \bcdef{closed} if $ \dd \omega = 0 $;
    \item \bcdef{exact} if $ \exists \eta \in \lm{p - 1} : \omega = \dd \eta $.
  \end{itemize}
\end{definition}

By Poincaré's theorem, clearly $ \omega \in\lm{p} $ exact $ \implies \omega $ closed, while the converse is not true in general, but depends on the topology of the manifold: a full treatment of this matter led to the development of cohomology, as outlined in \secref{ssec:derahm}.

\begin{definition}{Interior product}{}
  Given a vector field $ X \in\xm $, the \bcdef{interior product} determined by $ X $ is the linear map $ \iota_X : \lm{p} \rightarrow \lm{p-1} $ defined as:
  \begin{equation}
    \iota_X \omega (Y_1, \dots, Y_{p-1}) \defeq \omega (X, Y_1, \dots, Y_{p-1})
  \end{equation}
\end{definition}

On 0-forms (i.e. scalar functions), the interior product is defined as $ \iota_X f \equiv 0 \,\, \forall X \in \xm $.

\begin{proposition}{Anti-commutation}{}
  Given $ X,Y \in\xm $, then $ \iota_X \iota_Y = - \iota_Y \iota_X $.
\end{proposition}

\begin{proofbox}
  \begin{proof}
    Consequence of the total anti-symmetry of $ p $-forms.
  \end{proof}
\end{proofbox}

\begin{proposition}{Distributivity}{}
  Given $ X \in\xm, \omega \in\lm{p}, \eta \in\lm{q} $, then:
  \begin{equation}
    \iota_X (\omega \wedge \eta) = \iota_X \omega \wedge \eta + (-1)^p \omega \wedge \iota_X \eta
  \end{equation}
\end{proposition}

\begin{theorem}{Cartan's theorem}{}
  Given a vector field $ X \in\xm $, then:
  \begin{equation}
    \ld_X = \dd \circ \iota_X + \iota_X \circ \dd
  \end{equation}
\end{theorem}

\begin{proofbox}
  \begin{proof}
    Consider $ \omega \in\lm{1} $:
    \begin{equation*}
      \iota_X (\dd \omega) = \iota_X \frac{1}{2} \left( \pa_{\mu} \omega_{\nu} - \pa_{\nu} \omega_{\mu} \right) \dd x^{\mu} \wedge \dd x^{\nu} = X^{\mu} \pa_{\mu} \omega_{\nu} \dd x^{\nu} - X^{\nu} \pa_{\mu} \omega_{\nu} \dd x^{\mu}
    \end{equation*}
    \begin{equation*}
      \dd (\iota_X \omega) = \dd (\omega_{\mu} X^{\mu}) = X^{\mu} \pa_{\nu} \omega_{\mu} \dd x^{\nu} + \omega_{\mu} \pa_{\nu} X^{\mu} \dd x^{\nu}
    \end{equation*}
    Thus, adding these expressions and recalling \eref{eq:lie-derivative-1-forms}:
    \begin{equation*}
      (\dd \iota_X + \iota_X \dd) \omega = \left( X^{\mu} \pa_{\mu} \omega_{\nu} + \omega_{\mu} \pa_{\nu} X^{\mu} \right) \dd x^{\nu} = \ld_X \omega
    \end{equation*}
    The $ p > 1 $ case is more complex.
  \end{proof}
\end{proofbox}

\subsection{de Rham cohomology}
\label{ssec:derahm}

While exact $ \implies $ closed, the converse is not true, in general: it depends on the topological properties of the manifold.

\begin{lemma}{Poincaré's lemma}{}
  If $ \mathcal{M} $ is simply connected, then $ \omega\in\lm{p} $ closed $ \implies \omega $ exact.
\end{lemma}

In general, it is always possible to choose a simply connected neighbourhood of a point $ p \in \mathcal{M} $, in which every closed form is exact, but that may not always be possible globally.

It is convenient to set the notation $ d_p \equiv d : \lm{p} \rightarrow \lm{p+1} $, so that the set of all closed $ p $-forms on $ \mathcal{M} $ is denoted by $ Z^p(\mathcal{M}) \defeq \ker d_p $, while the set of all exact $ p $-forms by $ B^p(\mathcal{M}) \defeq \ran d_{p-1} $.

\begin{definition}{Equivalent forms}{}
  Two closed $ p $-forms $ \omega,\omega' \in Z^P(\mathcal{M}) $ are said to be \bcdef{equivalent} if $ \omega = \omega' + \eta $ for some $ \eta \in B^p(\mathcal{M}) $.
\end{definition}

\begin{definition}{de Rahm cohomology}{}
  The \bcdef{$ p^{\text{th}} $ de Rham cohomolgy group} of a manifold $ \mathcal{M} $ is defined to be:
  \begin{equation}
    H^p(\mathcal{M}) \defeq Z^p(\mathcal{M}) / B^p(\mathcal{M})
  \end{equation}
  and its dimension is the $ p^\text{th} $ \bcdef{Betti number} of $ \mathcal{M} $:
  \begin{equation}
    B_p \defeq \dim_{\R} H^p(\mathcal{M})
  \end{equation}
\end{definition}

\begin{theorem}{Betti numbers}{}
  Given a differentiable manifold, its Betti numbers are always finite.
\end{theorem}

$ B_0 = 1 $ for any connected manifold: there exist constant functions, which are manifestly closed and not exact, due to the non-existence of ``$ (-1) $-forms". Higher Betti numbers are non-zero only if the manifold has some non-trivial topology.

\begin{definition}{Eulerv's character}{}
  The \bcdef{Euler's character} of a manifold $ \mathcal{M} $ is defined as:
  \begin{equation}
    \chi(\mathcal{M}) \defeq \sum_{p \in \N_0} (-1)^p B_p
  \end{equation}
\end{definition}

\begin{example}{Spheres and tori}{}
  The $ n $-sphere $ \mathbb{S}^n $ has only non-vanishing $ B_0 = B_n = 1 $, thus $ \chi(\mathbb{S}^n) = 1 + (-1)^n $, while the $ n $-torus $ \mathbb{T}^n $ has $ B_p = \binom{n}{p} $, hence $ \chi(\mathbb{T}^n) = 0 $.
\end{example}

\subsection{Integration}

\begin{definition}{Volume form}{}
  A \bcdef{volume form} on an $ n $-dimensional differentiable manifold $ \mathcal{M} $ is a nowhere-vanishing top form $ \volf $, i.e. locally $ \volf = v(x) \dd x^1 \wedge \dots \wedge \dd x^n : v(x) \neq 0 $.

  If such a form exists, the manifold is said to be \bcdef{orientable}, and its orientation is right/left-handed if respectively $ v(x) > 0 $ or $ v(x) < 0 $ locally on every subset of $ \mathcal{M} $.
\end{definition}

To ensure that the handedness of the manifold doesn't change on overlapping charts:
\begin{equation*}
  \volf = v(x) \frac{\pa x^1}{\pa y^{\mu_1}} \dd y^{\mu_1} \wedge \dots \wedge \frac{\pa x^n}{\pa x^{\mu_n}} \dd x^{\mu_n} = v(x) \det \left[ \frac{\pa x^{\mu}}{\pa y^{\nu}} \right] \dd y^1 \wedge \dots \wedge \dd y^n
\end{equation*}
It is therefore necessary that the two sets of coordinates on the overlapping region satisfy:
\begin{equation}
  \det \left[ \frac{\pa x^{\mu}}{\pa y^{\nu}} \right] > 0
\end{equation}
Non-orientable manifolds cannot be covered by overlapping charts satisfying this condition.

\begin{example}{Projective spaces}{}
  The real projective space $ \mathbb{RP}^n $ is orientable for odd $ n $ and non-orientable for even $ n $, while the complex projective space $ \mathbb{CP}^n $ is orientable for all $ n \in \N $.
\end{example}

\begin{definition}{Integration over manifolds}{}
  Given a function $ f : \mathcal{M} \rightarrow \R $ on an orientable manifold $ \mathcal{M} $ with volume form $ \volf $ and a chart $ (\varphi,U) $ on $ \mathcal{M} $ with coordinates $ \{x^{\mu}\}_{\mu = 1,\dots,n} $, the \bcdef{integral} of $ f $ on $ O = \varphi^{-1}(U) \subset \mathcal{M} $ is defined as:
  \begin{equation}
    \int_{O} f \, \volf \defeq \int_{U} \dd x_1 \dots \dd x_n \, f(x) v(x)
  \end{equation}
\end{definition}

It is clear that the volume form acts like a measure on the manifold. To integrate over the whole manifold, it must be divided up into different regions, each covered by a single chart.

\begin{definition}{Submanifold}{}
  A $ k $-dimensional manifold $ \Sigma $ is a \bcdef{submanifold} of an $ n $-dimensional manifold $ \mathcal{M} $, with $ n > k $, if there exists an injective map $ \varphi : \Sigma \rightarrow \mathcal{M} $ such that $ \varphi_* : T_p(\Sigma) \rightarrow T_{\varphi(p)}(\mathcal{M}) $ is injective.
\end{definition}

\begin{definition}{Integration over submanifolds}{}
  Given a $ k $-form $ \omega \in\lm{k} $, its integral over a $ k $-dimensional submanifold $ \Sigma $ of $ \mathcal{M} $ is defined as:
  \begin{equation}
    \int_{\varphi(\Sigma)} \omega \defeq \int_{\Sigma} \varphi^* \omega
  \end{equation}
\end{definition}

\begin{example}{Integration over a curve}{}
  Consider a 1-form $ \omega \in\lm{1} $ and a 1-dimensional submanifold $ \gamma $ of $ \mathcal{M} $ described by a curve $ \sigma : \gamma \rightarrow \mathcal{M}: x^{\mu} = \sigma^{\mu}(t) $: locally $ \omega = \omega_{\mu}(x) \dd x^{\mu} $, thus the integral of $ \omega $ on $ \gamma $ can be calculated as $ \int_{\sigma(\gamma)} \omega = \int_{\gamma} \sigma^* \omega = \int_{\gamma} \dd \tau \,\omega_{\mu}(x) \frac{\dd x^{\mu}}{\dd \tau} $.
\end{example}

\subsubsection{Stokes' theorem}

Integration can be generalized beyond smooth (i.e. differentiable) manifolds.

\begin{definition}{Manifold with boundary}{}
  An $ n $-dimensional \bcdef{manifold with boundary} is a Hausdorff topological space, equipped with a compatible maximal atlas, which is locally homeomorphic to $ \R^{n-1} \times [a,\infty) : a \in\R $.

  The \bcdef{boundary} $ \pa \mathcal{M} $ is the 1-dimensional submanifold determined parametrically by $ x^n = a $.
\end{definition}

\begin{theorem}{Stokes' theorem}{}
  Given an $ n $-dimensional manifold $ \mathcal{M} $ with boundary $ \pa \mathcal{M} $, then for any $ \omega \in\lm{n-1} $:
  \begin{equation}
    \int_{\mathcal{M}} d\omega = \int_{\pa \mathcal{M}} \omega
  \end{equation}
\end{theorem}

This important theorem unifies many different results.

\paragraph{Fundamental theorem of calculus}

Given the 1-dimensional manifold $ I = [a,b] \subset \R $, then for any $ 0 $-form (i.e. scalar function) $ \omega = \omega(x) $:
\begin{equation*}
  \int_{I} d\omega = \int_a^b \frac{\dd \omega}{\dd x} \dd x = \int_{\pa I} \omega = \omega(b) - \omega(a)
\end{equation*}

\paragraph{Green's theorem}

Given a 2-dimensional manifold with boundary $ S \subset \R^2 $ and a 1-form $ \omega = \omega_1 \dd x^1 + \omega_2 \dd x^2 $, then $ \dd \omega = (\pa_1 \omega_2 - \pa_2 \omega_1) \dd x^1 \wedge \dd x^2 $ and:
\begin{equation*}
  \int_{S} \dd \omega = \int_{S} \left( \frac{\pa \omega_2}{\pa x^1} - \frac{\pa \omega_1}{\pa x^2} \right) \dd x^1 \dd x^2 = \int_{\pa S} \omega = \int_{\pa S} \omega_1 \dd x^1 + \omega_2 \dd x^2
\end{equation*}

\paragraph{Gauss' theorem}

Given a 3-dimensional manifold with boundary $ V \subset \R^3 $ and a 2-form $ \omega = \omega_1 \dd x^2 \wedge \dd x^3 + \omega_2 \dd x^3 \wedge \dd x^1 + \omega_3 \dd x^1 \wedge \dd x^2 $, then $ \dd \omega = (\pa_1 \omega_1 + \pa_2 \omega_2 + \pa_3 \omega_3) \dd x^1 \wedge \dd x^2 \wedge \dd x^3 $ and:
\begin{equation*}
  \int_{V} \dd \omega = \int_{V} \left( \frac{\pa \omega_1}{\pa x^1} + \frac{\pa \omega_2}{\pa x^2} + \frac{\pa \omega_3}{\pa x^3} \right) = \int_{\pa V} \omega = \int_{\pa V} \omega_1 \dd x^2 \dd x^3 + \omega_2 \dd x^3 \dd x^1 + \omega_3 \dd x^1 \dd x^2
\end{equation*}










