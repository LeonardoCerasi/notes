\selectlanguage{english}

Observations of the CMB reveal that, at least until $ z_\text{CMB} $, the universe was extremely homogeneous, with temperature variations of the order $ \delta T / T \sim 10^{-5} $. However, to understand the formation and evolution of large-scale structures, it is necessary to introduce inhomogeneities: today these perturbations of the homogeneous background are huge (e.g. galaxies and voids), but up to $ z_\text{CMB} $ they can be treated with perturbation theory.

\begin{figure}
  \centering
  \includegraphics[width = 1.00 \textwidth]{cosmo_plot.pdf}
  \caption{Evolution of the comoving Hubble radius. Four examples of cosmological perturbations with different wavelengths are drawn.}
  \label{fig:hr}
\end{figure}

\figref{fig:hr} shows the evolution of the comoving Hubble radius with the key events of the history of the universe highlighted. Cosmic perturbations (assuming  a wake-like ansatz) are pictured: these perturbations evolve in a linear regime up to $ a_\text{CMB} $, which can be treated analytically, but then enter a non-linear regime which lasts to these days and has to be treated computationally.

\section{Newtonian perturbation theory}

Since the perturbations are assumed small, the metric perturbation with respect to the background metric can be treated in the Newtonian approximation.

\subsection{Flat background}

First, consider the evolution of cosmological perturbations in a flat universe. In the Newtonian approximation, the gravitational potential is the:
\begin{equation}
  \Phi(t,\ve{x}) = \delta \Phi(t,\ve{x})
\end{equation}
since a flat background has $ \bar \Phi(t) \equiv 0 $. The matter source too must be perturbed: assuming the background to be at rest, i.e. $ \bar{\ve{u}}(t) \equiv 0 $:
\begin{equation}
  \rho(t,\ve{x}) = \bar \rho(t) + \delta \rho(t,\ve{x})
  \qquad \qquad
  P(t,\ve{x}) = \bar P(t) + \delta P(t,\ve{x})
  \qquad \qquad
  \ve{u}(t,\ve{x}) = \delta \ve{u}(t,\ve{x})
\end{equation}
The equations that dictate the physical evolution of these perturbations are the continuity equation, the Euler equation and the Poisson equation:
\begin{equation}
  \dot{\rho} + \dive (\rho \ve{u}) = 0
  \label{eq:continuity}
\end{equation}
\begin{equation}
  \left( \pa_t + \ve{u} \cdot \grad \right) \ve{u} = - \frac{1}{\rho} \grad P - \grad \Phi
  \label{eq:euler}
\end{equation}
\begin{equation}
  \lap \Phi = 4\pi G \rho
  \label{eq:poisson}
\end{equation}

\subsubsection{Matter perturbations}

To study how matter perturbations evolve, consider the complete absence of gravity, i.e. $ \Phi(t,\ve{x}) \equiv 0 $, and a static background $ \bar \rho, \bar P = \text{const.} $, so that the (linear) continuity and Euler equations read:
\begin{equation*}
  \delta \dot \rho + \bar \rho \, \dive \delta \ve{u} = 0
  \qquad \qquad
  \bar \rho \, \delta \dot{\ve{u}} = - \grad \delta P
\end{equation*}
Introducing the density contrast $ \delta \equiv \delta \rho / \bar \rho $, these equations are rewritten as:
\begin{equation*}
  \dot{\delta} = - \dive \delta \ve{u}
  \qquad \qquad
  \delta \dot{\ve{u}} = - c_\text{s}^2 \grad \delta
\end{equation*}
where the \bctxt{sound velocity} is defined as:
\begin{equation}
  c_\text{s}^2 \equiv \frac{\delta p}{\delta \rho}
\end{equation}
Combining the above equation, a wave equation for the density contrast is found:
\begin{equation}
  \ddot{\delta} - c_\text{s}^2 \lap \delta = 0
\end{equation}
Therefore, perturbations evolve as sound waves with sound velocity $ c_\text{s} $. Since for cosmological matter sources $ P = w \rho $, the sound velocity is $ c_\text{s}^2 = w $: in matter ($ w = 0 $) no sound waves propagate since there is no pressure. The resulting dispersion relation is:
\begin{equation}
  \omega^2(k) = c_\text{s}^2 k^2
\end{equation}

\subsubsection{Gravitational perturbations}

Introducing gravitational perturbations in the universe, i.e. $ \delta \Phi(t,\ve{x}) \neq 0 $ (which evolves as $ \lap \delta \Phi = 4 \pi G \delta \rho $), the wave equation for the density contrast gains a source term:
\begin{equation}
  \ddot{\delta} - c_\text{s}^2 \lap \delta = 4 \pi G \bar{\rho} \, \delta
  \label{eq:wave-no-grav}
\end{equation}
The dispersion relation is then modified as:
\begin{equation}
  \omega^2(k) = c_\text{s}^2 k^2 - 4 \pi G \bar \rho
  \label{eq:jeans-no-grav}
\end{equation}
The source term then acts as a \emph{tachyonic mass term} (i.e. with $ m^2 < 0$), which introduces a characteristic scale of perturbations, the Jeans scale:
\begin{equation}
  k_\text{J}^2 \equiv \frac{4 \pi G \bar{\rho}}{c_\text{s}^2}
\end{equation}
For $ k > k_\text{J} $, i.e. for small-scale perturbations, $ \omega \in \R $ and the behavior of the perturbations is oscillating (sound waves), while for $ k < k_\text{J} $, i.e. large-scale perturbations, $ \omega \in \C $ and the perturbations experience an exponential evolution, which is known as \bctxt{Jeans' instability}: this is the expected collapse of large structures.

After the decoupling of photons (i.e. the emission of the CMB), matter starts behaving with $ w = 0 $, i.e. $ c_\text{s} = 0 $: since there are no waves, $ \omega \in \C \,\, \forall k \in \R^+_0 $ (from \eref{eq:jeans-no-grav}), i.e. there is a collapse and consequent formation of structures at all scales (stars, galaxies, galaxy clusters, \dots).

\subsection{FLRW background}

Now, consider the evolution of cosmological perturbations in a FLRW background, which introduces the distinction between comoving coordinates $ \ve{x} $ and physical coordinates $ \ve{r} \defeq a(t) \ve{x} $. As a consequence, the physical velocity is $ \ve{u} \equiv \dot{\ve{r}} = H \ve{r} + \ve{v} $, where $ \ve{v} \equiv a(t) \dot{\ve{x}} $ is the physical peculiar velocity.

\begin{lemma}[before upper = {\tcbtitle}]{Derivatives in a FLRW background}{}
  \begin{equation}
  \grad_r = a^{-1}(t) \grad_x
  \qquad \qquad
  \pa_t \vert_r = \pa_t \vert_x - H \ve{x} \cdot \grad_x
\end{equation}
\end{lemma}

\begin{proofbox}
  \begin{proof}
    Consider a function $ f(t,\ve{r}) = f(t,a(t)\ve{x}) $. Then, $ \grad_\ve{r} = a^{-1}(t) \grad_\ve{x} $ by definition, and:
    \begin{equation*}
      \left( \frac{\pa f}{\pa t} \right)_x = \frac{\pa}{\pa t} \bigg\vert_x f(t,a(t)\ve{x}) = \frac{\pa}{\pa t} \bigg\vert_r f(t,\ve{r}) + \frac{\pa (a(t) \ve{x})}{\pa t} \cdot \grad_r f(t,\ve{r}) = \left[ \frac{\pa}{\pa t} \bigg\vert_r + H \ve{x} \cdot \grad_x \right] f
    \end{equation*}
    which is the thesis.
  \end{proof}
\end{proofbox}

This shows that, while in a static spacetime $ \pa_t $ and $ \grad $ are independent, in an expanding spacetime this is no longer true. Then, assuming that the quantities in \eeref{eq:continuity}{eq:poisson} are physical quantities, they are modified when expressed in terms of comoving coordinates.

\begin{theorem}{Newtonian approximation in FLRW background}{}
  Cosmological perturbations in a FLRW background are governed by:
  \begin{equation}
    \dot{\rho} + 3H \rho + \frac{1}{a} \dive (\rho \ve{v}) = 0
  \end{equation}
  \begin{equation}
    \left( \frac{\pa}{\pa t} + \frac{\ve{v}}{a} \cdot \grad \right) \ve{u} = - \frac{1}{a \rho} \grad P - \frac{1}{a} \grad \Phi
  \end{equation}
  \begin{equation}
    \lap \Phi = 4 \pi G a^2 \rho
  \end{equation}
  where $ \grad \equiv \grad_x $ and $ \pa_t \equiv \pa_t \vert_x $.
\end{theorem}

\begin{proofbox}
  \begin{proof}
    First, consider the continuity equation \eref{eq:continuity}:
    \begin{equation*}
      \begin{split}
        0
        & = \left( \frac{\pa \rho}{\pa t} \right)_r + \grad_r \cdot (\rho \ve{u}) = \left( \frac{\pa}{\pa t} \bigg\vert_x - H \ve{x} \cdot \grad_x \right) \rho + \frac{1}{a} \grad_x \cdot (\rho \dot{a} \ve{x} + \rho \ve{v}) \\
        & \equiv \dot{\rho} - H \ve{x} \cdot \grad \rho + H \ve{x} \cdot \grad \rho + H \rho \dive \ve{x} + \frac{1}{a} \dive (\rho \ve{v}) = \dot{\rho} + 3 H \rho + \frac{1}{a} \dive (\rho \ve{v})
      \end{split}
    \end{equation*}
    Then, consider the Euler equation \eref{eq:euler}:
    \begin{equation*}
      \begin{split}
        \left( \frac{\pa}{\pa t} \bigg\vert_r + \ve{u} \cdot \grad_r \right) \ve{u} & = - \frac{1}{\rho} \grad_r P - \grad_r \Phi \\
        \left( \frac{\pa}{\pa t} - H \ve{x} \cdot \grad + H \ve{x} \cdot \grad + \frac{\ve{v}}{a} \cdot \grad \right) \ve{u} & = - \frac{1}{a \rho} \grad P - \frac{1}{a} \grad \Phi
      \end{split}
    \end{equation*}
    which is the thesis. Finally, the Poisson equation is trivially modified by $ \lap_r = a^{-2}(t) \lap $.
  \end{proof}
\end{proofbox}

Introducing the density contrast $ \delta \ll 1 $, these equations can be linearized\footnotemark. The continuity equation at zeroth order and at first order reads:
\begin{equation*}
  \dot{\bar \rho} + 3 H \bar \rho = 0
  \qquad \qquad
  \frac{\pa (\bar{\rho} \delta)}{\pa t} + 3 H \bar{\rho} \delta + \frac{1}{a} \dive (\bar{\rho} \ve{v}) = 0
\end{equation*}
Using the zeroth order equation for the evolution of the background density, the evolution of the density contrast is found to be:
\begin{equation}
  \dot \delta = - \frac{1}{a} \dive \ve{v}
  \label{eq:density-contrast}
\end{equation}
At linear order, the growth of density perturbations is sourced by the divergence of the fluid velocity. Then, the linearized Euler equation reads:
\begin{equation*}
  \frac{\pa (H \ve{r})}{\pa t} = - \frac{1}{a \bar{\rho}} \grad \bar P - \frac{1}{a} \grad \bar \Phi
  \qquad \quad
  \frac{\pa (H \ve{r} + \ve{v})}{\pa t} + \frac{\ve{v}}{a} H \cdot \grad \ve{r} = - \frac{1}{a \bar \rho} \grad (\bar P + \delta P) - \frac{1}{a} \grad (\bar \Phi + \delta \Pi)
\end{equation*}
which, using $ \ve{v} \cdot \grad \ve{x} = \ve{v} $, reduces to:
\begin{equation}
  \dot{\ve{v}} + H \ve{v} = - \frac{1}{a \bar \rho} \grad \delta P - \frac{1}{a} \grad \delta \Phi
  \label{eq:peculiar-velocity}
\end{equation}
In the absence of perturbations, this equation simply states that $ \ve{v} \propto a^{-1}(t) $. Now, combine \eeref{eq:density-contrast}{eq:peculiar-velocity} and substitute the Poisson equation $ \lap \delta \Phi = 4\pi G a^2 \bar \rho \, \delta $ to obtain:
\begin{equation}
  \ddot{\delta} + 2 H \dot{\delta} - \frac{c_\text{s}^2}{a^2} \lap \delta = 4\pi G \bar \rho \, \delta
  \label{eq:wave-newton}
\end{equation}

\footnotetext{Recall that, for the velocity field, the perturbation is the peculiar velocity $ \ve{v} $.}

This equation is equivalent to \eref{eq:wave-no-grav}, but with two important differences caused by the Hubble flow: a redshift of the sound velocity $ a^{-1}(t) c_\text{s} $ and the presence of friction term, which makes this a damped wave equation.

Note that the Jeans scale is still present, although now it is affected by the redshift due to the Hubble flow: if $ k > a k_\text{J} $, the perturbations behave like damped waves, while for $ k < a k_\text{J} $ the system presents Jeans' instability, which is no longer exponential due to the friction term (now it is a power-law instability).

For baryons, the sound velocity, and so the Jeans scale, depend on time. Before recombination, baryons are strongly coupled to photons, forming a relativistic photon-baryon fluid with $ c_\text{s} \approx 1 / \sqrt{3} $: the Jeans scale is then of the order of the comoving Hubble radius, suppressing the growth of baryonic fluctuations at subhorizon scales (oscillatory regime). After recombination, the baryonic fluid starts behaving like pressureless dust and baryonic fluctuations at all scales start to grow.

On the other hand, for cold dark matter (CDM) the sound speed is negligible at all times, so that subhorizon fluctuations can start to grow earlier.

\subsubsection{Matter era}

In matter era $ a \propto t^{2/3} $ and $ H(t) = \frac{2}{3t} $. The density contrast is just $ \delta \simeq \delta_\text{m} \equiv \delta \rho_\text{m} / \bar{\rho}_\text{m} $, since:
\begin{equation*}
  \delta = \frac{\delta \rho}{\bar \rho} \simeq \frac{\delta \rho_\text{r} + \delta \rho_\text{m}}{\bar{\rho}_\text{m}} \equiv \delta_\text{m} + \delta_\text{r} \frac{\bar{\rho}_\text{r}}{\bar{\rho}_\text{m}} \simeq \delta_\text{m}
\end{equation*}
By \eref{eq:fried-hubble}, then, $ 4\pi G \bar{\rho}_\text{m} = \frac{3}{2} H^2 = \frac{2}{3t^2} $, and \eref{eq:wave-newton} for subhorizon modes (i.e. $ \lambda < (aH)^{-1} $) becomes:
\begin{equation}
  \ddot{\delta}_\text{m} + \frac{4}{3t^2} \dot{\delta}_\text{m} - \frac{2}{3t^2} \delta_\text{m} = 0
\end{equation}
since $ c_\text{s}^2 \approx w = 0 $ in matter era. With a power-law ansatz $ \delta_\text{m} \propto t^\alpha $, two solutions are found:
\begin{equation}
  \delta_\text{m} \propto
  \begin{cases}
    t^{-1} \propto a^{-2/3}
    t^{2/3} \propto a \\
  \end{cases}
\end{equation}
The first solution is a \emph{decaying solution}, while the second one is a \emph{growing solution}. The growing solution is $ \delta \propto a $, which during matter era varies from $ a_\text{eq} \simeq 1/3401 $ to $ a_\Lambda \simeq 1/1.4 $, i.e. by $ \sim 10^3 $ times: assuming initial perturbations of order $ \delta_\text{m} \sim 10^{-5} $ (from CMB), at the end of matter era these perturbations have grown to $ \delta_\text{m} \sim 10^{-2} $, still not enough to explain the cosmic structures observed today. This is due to the fact that, after the CMB emission, cosmological perturbations enter the non-linear regime, and non-linear collapse is faster than the linear one.

Moreover, note that a smaller matter era determines a smaller growth of fluctuations: this means that the existence of dark matter (i.e. an additional dominant contribution to $ \Omega_\text{m} $ other than baryonic matter) is necessary to explain the current cosmic structures.

\subsubsection{Radiation era}

In radiation era $ a \propto t^{1/2} $ and $ H(t) = \frac{1}{2t} $. Now, the density contrast is $ \delta \simeq \delta_\text{r} + \delta_\text{m} \frac{\bar{\rho}_\text{m}}{\bar{\rho}_\text{r}} $: since the radiation fluid has a large sound velocity $ c_\text{s} \approx 1 / \sqrt{3} $, it is expected to oscillate over scales smaller than the comoving Hubble horizon, so that, averaging over many oscillation periods, $ \braket{\delta_\text{r}} = 0 $ and the time-average density contrast reduces to the matter contribution. Then, \eref{eq:wave-newton} becomes:
\begin{equation*}
  \ddot{\delta}_\text{m} + 2 H \dot{\delta}_\text{m} - 4\pi G \bar{\rho}_\text{m} \delta_\text{m} = 0
\end{equation*}
Since $ \dot{\delta}_\text{m} \sim H \delta_\text{m} $, the first two terms are $ \sim H^2 \sim \bar \rho_\text{r} \gg \bar \rho_\text{m} $, hence the last term is negligible and the evolution of $ \delta_\text{m} $ is given by:
\begin{equation}
  \ddot{\delta}_\text{m} + \frac{1}{t} \dot{\delta}_\text{m} = 0
\end{equation}
With a power-law ansatz, two solution are found:
\begin{equation}
  \delta_\text{m} \propto
  \begin{cases}
    \text{const.} \\
    \ln t \propto \ln a
  \end{cases}
\end{equation}
Therefore, matter perturbations grow slower in radiation era, compared to matter era (\emph{Mészáros effect}). On the other hand, superhorizon modes (i.e. $ \lambda > (aH)^{-1} $) cannot causally interact, and their evolution can be studied with a relativistic analysis.

To show how matter perturbations at different scales evolve, it is necessary to study their power spectrum $ \delta_\text{m}(k) $. In particular, Inflation predicts that, when perturbations enter the comoving Hubble horizon, they have the primordial power spectrum $ \delta_\text{m,initial}(k) \propto k^{-1} $ (\emph{Harrison--Zel'dovich spectrum}), and then they evolve according to a transfer function of the form (see \figref{fig:trans}):
\begin{equation}
  \mathcal{T}(k) \simeq
  \begin{cases}
    1 & k < k_\text{eq} \\
    \left( \frac{k_\text{eq}}{k} \right)^2 \ln \frac{k}{k_\text{eq}} & k > k_\text{eq}
  \end{cases}
\end{equation}
where $ k_\text{eq} \sim 60 $ is the wavenumber of perturbations entering the comoving Hubble horizon at radiation-matter equality ($ \lambda_\text{eq} \sim 100 \,\text{Mpc} $, approximately the size of a galaxy cluster). This means that large-scale perturbations grow less than small-scale perturbations, since they enter the Hubble radius later (see \figref{fig:hr}).

Combining these, it is possible to compute the predicted (linear) matter power spectrum for matter fluctuations observed today:
\begin{equation}
  \delta_\text{m}^2 \propto
  \begin{cases}
    k & k < k_\text{eq} \\
    k^{-3} \ln^2 \frac{k}{k_\text{eq}} & k > k_\text{eq}
  \end{cases}
\end{equation}
which has excellent agreement with experimental observations (\figref{fig:dm-power}), except for the high-$ k $ (i.e. small-$ \lambda $) regime, a discrepancy which is due to the limited applicability of the linear approximation to small-scale fluctuations. Indeed, the linear analysis tends to underestimate the formation of small-scale structure fuelled by the faster non-linear collapse.

\begin{figure}
  \centering
  \includegraphics[width = 0.45 \textwidth]{transfer-function.png}
  \caption{Transfer function for matter perturbation: the shape of $ \mathcal{T}(k) $ is a consequence of the suppressed growth of subhorizon modes during the radiation era.}
  \label{fig:trans}
\end{figure}

\begin{figure}
  \centering
  \includegraphics[width = 0.7 \textwidth]{dm-power.png}
  \caption{Measurements of the linear matter power spectrum.}
  \label{fig:dm-power}
\end{figure}

\subsubsection{Vacuum era}

Finally, in vacuum era $ a \propto e^{H t} $, with $ H \approx \text{const.} $ during the dark-matter-dominated era. Again $ 4\pi G \bar{\rho}_\text{m} \ll H^2 \propto \rho_\Lambda $ (the density of dark energy does not fluctuate), hence \eref{eq:wave-newton} reduces to:
\begin{equation}
  \ddot{\delta}_\text{m} + 2 H \dot{\delta}_\text{m} = 0
\end{equation}
which, with a power-law ansatz, has two solutions:
\begin{equation}
  \delta_\text{m} \propto
  \begin{cases}
    e^{-2Ht} \propto a^{-2} \\
    \text{const.}
  \end{cases}
\end{equation}
which shows that, during the vacuum era, the growth of cosmological perturbations (i.e. of structures) halts.

\newpage
\section{Relativistic perturbation theory}

The Newtonian analysis is inadequate on scales larger than the Hubble radius and for relativistic fluids (e.g. photons and neutrinos). Before photon decoupling, photons electrons and protons are strongly coupled by electromagnetic interactions, forming a single (relativistic) photon-baryon fluid: the matter perturbations then oscillate, since gravity makes perturbations collapse, causing an increase in radiation density that makes them re-expand. At photon decoupling, modes with different wavelengths are captured at different phases in their evolution, giving rise to a characteristic pattern in the CMB. After photon decoupling, the baryons lose the pressure support of the photons and fall into the gravitational potential wells created by dark matter, eventually forming starts and galaxies. These dark matter clusters only start to form after radiation-matter equality, since before that they are suppressed by the large radiation pressure of photons.

\subsection{Perturbations}

The Einstein field equations \eref{eq:ef-eq} couple perturbations in the stress-energy tensor to those in the metric, which then need to be studied simultaneously. In particular, working in conformal time, the perturbations with respect to the homogeneous background can be written as:
\begin{equation}
  g_{\mu \nu}(\eta , \ve{x}) = \bar g_{\mu \nu}(\eta) + \delta g_{\mu \nu}(\eta , \ve{x})
\end{equation}
\begin{equation}
  T_{\mu \nu}(\eta , \ve{x}) = \bar T_{\mu \nu}(\eta) + \delta T_{\mu \nu}(\eta , \ve{x})
\end{equation}
To avoid clutter, from now on the dependence of perturbations from $ (\eta , \ve{x}) $ is suppressed. Note that the unperturbed metric is assumed to be the flat FLRW metric, so that:
\begin{equation*}
  \dd \bar s^2 \equiv \bar g_{\mu \nu} \dd x^\mu \dd x^\nu = a^2(\eta) \left[ \dd \eta^2 - \dd \ve{x}^2 \right]
\end{equation*}

\subsubsection{Metric perturbations}

Since the metric tensor of a 4D manifold has $ 10 $ independent components, $ \delta g_{\mu \nu} $ can be written as:
\begin{equation}
  \dd s^2 \equiv g_{\mu \nu} \dd x^\mu \dd x^\nu = a^2(\eta) \left[ \left( 1 + 2A \right) \dd \eta^2 - 2 B_i \dd x^i \dd \eta - \left( \delta_{ij} + h_{ij} \right) \dd x^i \dd x^j \right]
  \label{eq:metric-pert}
\end{equation}
where $ A $ is a scalar, $ B_i $ is a spatial 3-vector and $ h_{ij} $ is a symmetric spatial 3-tensor. Then, by convention $ B^i = \delta^{ij} B_j $ and $ \tensor{h}{^i_j} = \delta^{ik} h_{kj} $.

\begin{lemma}{SVT decomposition of vectors}{}
  Given a 3-vector $ B_i $, it can be decomposed as:
  \begin{equation}
    B_i = \pa_i B + \hat B_i
  \end{equation}
  where $ B $ is a scalar and $ \hat B_i : \pa^i \hat B_i = 0 $ is a divergenceless 3-vector.
\end{lemma}

\begin{lemma}{SVT decomposition of symmetric tensors}{}
  Given a symmetric 3-tensor $ h_{ij} $, it can be decomposed as:
  \begin{equation}
    h_{ij} = 2C \delta_{ij} + 2 \pa_{\langle i} \pa_{j \rangle} E + 2 \pa_{(i} \hat E_{j)} + 2 \hat E_{ij}
  \end{equation}
  where $ C , E $ are scalars, $ \hat E_i : \pa^i \hat E_i = 0 $ is a divergenceless 3-vector, $ \hat E_{ij} : \pa^i \hat E_{ij} = 0 \,\,\land\,\, \tensor{\hat E}{^i_i} = 0 $ is a divergenceless traceless symmetric 3-tensor and:
  \begin{equation}
    \pa_{\langle i} \pa_{j \rangle} \defeq \pa_i \pa_j - \frac{1}{3} \delta_{ij} \lap
  \end{equation}
\end{lemma}

With the SVT decomposition, the $ 3 $ degrees of freedom of $ B_i $ are decomposed into $ 1 + 2 $ and the $ 6 $ of $ h_{ij} $ into $ 1 + 1 + 2 + 2 $, resulting in a complete decomposition of the $ 10 $ degrees of freedom of $ \delta g_{\mu \nu} $ into $ 4 + 4 + 2 $: scalars $ A,B,C,E $, vectors $ \hat B_i , \hat E_i $ and tensor $ \hat E_{ij} $.

The SVT decomposition is convenient since, at linear order, the Einstein equations for scalars, vectors and tensors do not mix and can therefore be treated separately. In particular, in the following vector perturbations are ignored, since they are not produced by Inflation, and even if they were, they would decay quickly with the expansion of the universe. On the other hand, $ \hat E_{ij} $ is a transverse traceless tensor, hence tensor perturbations are truly gravitational waves (\eref{eq:traceless}) and will be analyzed later.

\paragraph{Gauge fixing}

The remaining scalar perturbations can be further simplified with gauge invariance. Indeed, the perturbations in \eref{eq:metric-pert} are not uniquely defined, but depend on the choice of coordinates, i.e. on the choice of gauge: the chosen form of the perturbed metric has been determined by the implicit introduction of a specific time slicing of the spacetime and definition of specific coordinates on these time slices.
In general, choosing different coordinates introduces ``fictious perturbations", known as \emph{gauge modes}.

In order to avoid these gauge problems, there are two solutions: either define special combinations of the metric perturbations that do not transform under a change of coordinates, called \emph{Bardeen variables}, or fix a particular gauge. In the following, the \bctxt{Newtonian gauge} is used, where two of the four scalar perturbations are set to zero:
\begin{equation}
  B = E \equiv 0
\end{equation}
The perturbed metric then becomes:
\begin{equation}
  \dd s^2 = a^2(\eta) \left[ \left( 1 + 2 \Psi \right) \dd \eta^2 - \left( 1 - 2 \Phi \right) \delta_{ij} \dd x^i \dd x^j \right]
  \label{eq:perturbed-metric}
\end{equation}
where the Bardeen potentials $ \Psi \equiv A $ and $ \Phi \equiv - C $ have been introduced. For perturbations that decay at spatial infinity, the Newtonian gauge is unique (i.e. the gauge is fixed completely). Note that, in this gauge, time slices are orthogonal to the worldlines of observers at rest in these coordinates (since $ B = 0 $), and the induced geometry on time slices is isotropic (since $ E = 0 $), thus making the physics transparent.

The metric in the Newtonian gauge bears similarity to the weak-field limit in Minkwoski space \eref{eq:linear-newtonian}: this suggests that $ \Psi $ plays the role of a gravitational potential.

\subsubsection{Matter perturbations}

The perturbations of the stress-energy tensor can be written in term of the perturbed energy density $ \rho(\eta , \ve{x}) = \bar \rho(\eta) + \delta \rho(\eta , \ve{x}) $ and pressure $ P(\eta , \ve{x}) = \bar P(\eta) + \delta P(\eta , \ve{x}) $:
\begin{align}
  \tensor{\delta T}{^0_0} & = \delta \rho \\
  \tensor{\delta T}{^i_0} & = \left( \bar{\rho} + \bar{P} \right) v^i \\
  \tensor{\delta T}{^i_j} & = - \delta P \, \tensor{\delta}{^i_j} - \tensor{\Pi}{^i_j}
\end{align}
where $ v^i $ is a 3-vector, known as \emph{bulk velocity}, and $ \tensor{\Pi}{^i_j} $ is a symmetric traceless 3-tensor, known as \emph{anisotropic stress}. The $ 10 $ degrees of freedom of $ \delta \tensor{T}{^\mu_\nu} $ are then correctly decomposed as $ 1 + 1 + 3 + 5 $.
In the following analysis, the anisotropic stress is ignored: $ \tensor{\Pi}{^i_j} \equiv 0 $.

\subsection{Evolution}

\begin{lemma}{Christoffel symbols in perturbed FLRW}{}
  The Christoffel symbols associated to the metric \eref{eq:perturbed-metric} are:
  \begin{align}
    \Gamma^0_{00} & = \ham + \Psi' \\
    \Gamma^0_{i0} & = \pa_i \Psi \\
    \Gamma^i_{00} & = \delta^{ij} \pa_j \Psi \\
    \Gamma^0_{ij} & = \ham \delta_{ij} - \left[ \Phi' + 2\ham (\Phi + \Psi) \right] \delta_{ij} \\
    \Gamma^i_{j0} & = [\ham - \Phi'] \delta^i_j \\
    \Gamma^i_{jk} & = - 2 \delta^i_{(j} \pa_{k)} \Phi + \delta_{jk} \delta^{i\ell} \pa_\ell \Phi
  \end{align}
  where $ \ham \equiv \frac{a'}{a} $ and $ a' \defeq \frac{\dd a}{\dd \eta} $.
\end{lemma}

The equations of motion of matter perturbations follow from the covariant conservation of the stress-energy tensor: $ \na_\mu \tensor{T}{^\mu_\nu} = 0 $. The $ \nu = 0 $ component is the continuity equation, which at zeroth and linear order in perturbations is:
\begin{equation}
  \bar \rho' + 3 \ham (\bar \rho + \bar P) = 0
  \label{eq:continuity-friedmann}
\end{equation}
\begin{equation}
  \delta \rho' + 3 \ham (\delta \rho + \delta P) - 3 \Phi' (\bar \rho + \bar P) + \dive [(\bar \rho + \bar P) \ve{v}] = 0
  \label{eq:continuity-first}
\end{equation}
The first equation is just the continuity equation \eref{eq:r-p} for the background, while the second equation describes the evolution of the density perturbations: in particular, the first term is the universal dilution due to the expansion, the second term is a local dilution due to the ``local scale factor" $ (1 - \Phi) a $ in the spatial part of the metric, and the third term is the local divergence of the fluid flow. It is also possible to rewrite the linearized continuity equation in terms of the density contrast $ \delta \equiv \delta \rho / \bar \rho $.

\begin{proposition}[before upper = {\tcbtitle}]{Continuity equation for the density contrast}{}
  \begin{equation}
    \delta' = (1 + c_\text{s}^2) (\Phi' - \dive \ve{v})
  \end{equation}
\end{proposition}

\begin{proofbox}
  \begin{proof}
    Since the background is homogeneous, i.e. $ \bar \rho = \bar \rho (\eta) $ and $ \bar P = \bar P (\eta) $, \eref{eq:continuity-first} becomes:
    \begin{equation*}
      \delta \rho' + 3 (1 + c_\text{s}^2) (\ham \delta \rho - \Phi' \bar \rho) + (1 + c_\text{s}^2) \bar \rho \dive \ve{v} = 0
    \end{equation*}
    where the sound velocity $ c_\text{s}^2 \equiv \bar P / \bar \rho = w $ has been introduced. Then, recall \eref{eq:continuity-friedmann}, so that:
    \begin{equation*}
      \delta' = \frac{\delta \rho'}{\bar \rho} - \delta \frac{\bar \rho'}{\bar \rho} = \frac{\delta \rho'}{\bar \rho} + 3 (1 + c_\text{s}^2) \ham \delta
    \end{equation*}
    Inserting the above expression for $ \delta \rho' $ yields the thesis.
  \end{proof}
\end{proofbox}

The $ \nu = i $ component is the Euler equation, which reads (using \eref{eq:continuity-friedmann}):
\begin{equation}
  \ve{v}' + (1 - 3 c_\text{s}^2) \ham \ve{v} = - \frac{1}{\bar \rho + \bar P} \dive \delta P - \dive \Psi
\end{equation}
where $ c_\text{s}^2 \equiv \bar p' / \bar \rho' = w $. Note the equivalence with \eref{eq:euler} and the fact that, in absence of energy density and pressure, this equation is equivalent to $ v \propto a^{-1} $ as expected.

The equations of motion of metric perturbations follow instead from the Einstein field equations: $ \tensor{G}{^\mu_\nu} = 8\pi G \tensor{T}{^\mu_\nu} $. The $ (00) $-component at zeroth and linear order is:
\begin{equation}
  \ham^2 = \frac{8\pi G}{3} a^2 \bar \rho
  \label{eq:001}
\end{equation}
\begin{equation}
  \lap \Psi - 3 \ham (\Phi' + \ham \Psi) = 4\pi G a^2 \delta \rho
  \label{eq:002}
\end{equation}
The first equation is just the Friedmann equation \eref{eq:fried-hubble}, while the second is the relativistic generalization of a sourced Poisson equation for $ \Psi $, coupled to $ \Phi $: for subhorizon modes, i.e. with $ k \gg a H = \ham $ (since $ \pa_\eta = a \pa_t $), it reduces to the Poisson equation in the Newtonian limit $ \lap \Psi = 4\pi G a^2 \delta \rho $. Then, the $ (0i) $-component reads:
\begin{equation*}
  \pa_i (\Phi' + \ham \Phi) = -4\pi G a^2 (\bar \rho + \bar P) v_i
\end{equation*}
Applying a SVT decomposition to $ v^i = \pa^i v + \hat v^i $ and ignoring vector perturbations (as already stated), this equation reduces to a scalar equation:
\begin{equation}
  \Phi' + \ham \Phi = - 4\pi G a^2 (\bar \rho + \bar P) v
  \label{eq:0i}
\end{equation}
which is again a relativistic sourced Poisson equation for $ \Phi $. There only remain the spatial components, which are best studied in terms of their trace and traceless parts. The traceless component $ (ij - \frac{1}{3} ii) $ simply is:
\begin{equation}
  \Phi = \Psi
\end{equation}
while the trace component $ (ii) $ is:
\begin{equation}
  \Phi'' + 3\ham \Phi' + (2\ham' + \ham^2) \Phi = 4\pi G a^2 \delta P
  \label{eq:ii}
\end{equation}
Note that these equations are only valid with $ \tensor{\Pi}{^i_j} \equiv 0 $: if the anisotropic stress cannot be ignored, then $ \Phi \neq \Psi $ and the last equation has an additional source term.

\subsubsection{Metric evolution}

The evolution of the potential $ \Phi $ differs between radiation era and matter era.

\paragraph{Radiation era}

In radiation era $ \ham = \eta^{-1} $ (\eref{eq:a-matt}), so that $ 2\ham' + \ham^2 = - \ham^2 $. Moreover, since $ \delta P = \frac{1}{3} \delta \rho $, combining \eref{eq:002} and \eref{eq:ii} results into:
\begin{equation}
  \Phi'' + 4\ham \Phi' - \frac{1}{3} \lap \Phi = 0
\end{equation}
which is a damped wave equation with $ c_\text{s} = 1/\sqrt{3} $ (as expected from $ c_\text{s}^2 = w $). Now, insert the Fourier expansion of the metric perturbation:
\begin{equation}
  \Phi(\eta , \ve{x}) = \int_{\R^3} \frac{\dd^3 k}{(2\pi)^{3/2}} \Phi_\ve{k}(\eta) e^{\img \ve{k} \cdot \ve{x}}
  \label{eq:f-fourier}
\end{equation}
so that the wave equation becomes an ODE:
\begin{equation*}
  \Phi_\ve{k}'' + \frac{4}{\eta} \Phi_\ve{k}' + \frac{k^2}{3} \Phi_\ve{k} = 0
\end{equation*}
The solutions to this equation have a distinct behavior in the superhorizon and subhorizon regimes:
\begin{equation}
  \Phi_\ve{k}(\eta) \approx
  \begin{cases}
    \text{const.} & \eta \ll \frac{1}{k} \\
    - G \Phi_\text{k}(0) \frac{1}{(k \eta)^2} \cos \frac{k \eta}{\sqrt{3}} & \eta \gg \frac{1}{k}
  \end{cases}
  \label{eq:metric-pert-rad}
\end{equation}
This shows that outside the comoving Hubble horizon no linear evolution happens, and the metric perturbations only evolve with small non-linear terms. On the other hand, inside the comoving Hubble horizon, in radiation era the metric perturbations oscillate with a damped amplitude that decreases as $ a^{-2} $ (since $ a \propto \eta $).

\paragraph{Matter era}

In matter era $ \ham = 2 \eta^{-1} $, so that $ 2\ham' + \ham^2 = 0 $. Moreover, since $ w = 0 $, \eref{eq:ii} reduces to:
\begin{equation}
  \Phi'' + 3 \ham \Phi' = 0
\end{equation}
which has two solutions:
\begin{equation}
  \Phi(\eta) \propto
  \begin{cases}
    \text{const.} \\
    \eta^{-5} \propto a^{-5/2}
  \end{cases}
\end{equation}
In matter era, then, the growing solution is constant independently on the wavelength of the fluctuation.

Putting everything together, it is possible to outline the linear evolution of curvature perturbations. Modes with $ k > k_\text{eq} $, which enter the Hubble radius in radiation era, are affected by the damped oscillations described by \eref{eq:metric-pert-rad} before freezing in matter era. On the contrary, modes with $ k < k_\text{eq} $ enter the Hubble radius in matter era, thus freezing immediately (see \figref{fig:metric-pert}).

\begin{figure}
  \centering
  \includegraphics[width = 0.7 \textwidth]{metric-pert.png}
  \caption{Linear evolution of metric perturbations. The small suppression of modes $ k > k_\text{eq} $ is a minor effect due to the radiation-to-matter transition.}
  \label{fig:metric-pert}
\end{figure}

\subsubsection{Radiation evolution}

The evolution of perturbations in the radiation energy density, encapsulated in the radiation density contrast $ \delta_\text{r} \equiv \delta \rho_\text{r} / \bar \rho_\text{r} $, is of particular interest, since it predicts the shape of the CMB power spectrum (which is still described by the linear approximation): indeed, since $ \rho_\text{r} \propto T^4 $, then $ \delta \rho \propto 4 T^3 \delta T $ and:
\begin{equation}
  \delta_\text{r} = 4 \delta_T
\end{equation}
where $ \delta_T \equiv \delta T / \bar T $ is the temperature contrast, which is measured in CMB anisotropies. To study the evolution of $ \delta_\text{r} $, combine \eeref{eq:001}{eq:002}:
\begin{equation}
  \delta = \frac{2}{3} \frac{\lap \Phi}{\ham^2} - \frac{2 \Phi'}{\ham} - 2 \Phi
\end{equation}
After introducing the Fourier expansion \eref{eq:f-fourier}, this equation can be studied both in radiation era and in matter era.

\paragraph{Radiation era}

In radiation era $ \delta_\text{r} $ is dominant, i.e. $ \delta \approx \delta $, and $ \ham = \eta^{-1} $, so that:
\begin{equation*}
  \delta_\text{r} = - \frac{2}{3} (k\eta)^2 \Phi_\ve{k} - 2 \eta \Phi_\ve{k}' - 2 \Phi_\ve{k}
\end{equation*}
Recalling the expression \eref{eq:metric-pert-rad} for the metric perturbation, the solution for $ \delta_\text{r} $ is found:
\begin{equation}
  \delta_\text{r}(\eta) \approx
  \begin{cases}
    \text{const.} & \eta \ll \frac{1}{k} \\
    4 \Phi_\ve{k}(0) \cos \frac{k \eta}{\sqrt{3}} & \eta \gg \frac{1}{k}
  \end{cases}
\end{equation}
Note that superhorizon modes are still frozen in the linear regime, while subhorizon radiation fluctuations oscillate without damping (contrary to metric perturbations, which are damped).

\paragraph{Matter era}

In matter era $ \delta_\text{r} $ is subdominant, and its evolution has to be inferred from the continuity and the Euler equations: subhorizon modes still oscillate with the same frequency as in radiation era, however now with a decreased amplitude and a shift due to the gravitational potential of the dominant matter:
\begin{equation}
  \delta_\text{r}(\eta) \approx
  \begin{cases}
    \text{const.} & \eta \ll \frac{1}{k} \\
    -4 \Phi_\text{m} + A \cos \frac{k \eta}{\sqrt{3}} & \eta \gg \frac{1}{k}
  \end{cases}
\end{equation}
As seen in \figref{fig:hr}, radiation modes with $ k > k_\text{eq} $, i.e. those which enter the comoving Hubble horizon in matter era, do not evolve significantly before photon decoupling, while radiation modes with $ k < k_\text{eq} $ are affected by oscillations, hence arrive at the last-scattering surface with different phases. The power spectrum of the CMB is then expected to have a plateau at low $ k < k_\text{eq} $ (i.e. at high wavelengths): this is experimentally observed and is known as the \bctxt{Sachs--Wolfe plateau} (see \figref{fig:cmb-sw}).

\begin{figure}
  \centering
  \includegraphics[width = \textwidth]{cmb.png}
  \caption{Power spectrum of the CMB anisotropies.}
  \label{fig:cmb}
\end{figure}

\begin{figure}
  \centering
  \includegraphics[width = \textwidth]{cmb-sw.png}
  \caption{Illustration of the different contributions to the CMB power spectrum.}
  \label{fig:cmb-sw}
\end{figure}

\subsubsection{CMB power spectrum}

The radiation density contrast is not the only contribution to the temperature anisotropoies of the CMB. Indeed, the complete espression for $ \delta_T $ as a function of the direction of observation is:
\begin{equation}
  \delta_T(\hat{\ve{n}}) = \left( \frac{1}{4} \delta_\text{r} + \Psi \right)_* - \left( \hat{\ve{n}} \cdot \ve{v}_\text{b} \right)_* + \int_{\eta_*}^{\eta_0} \dd \eta \left( \Phi' + \Psi' \right)
\end{equation}
where $ \ve{v}_\text{b} $ is the bulk velocity of baryons relative to the observer RF, and the subscript $ * $ denotes the evaluation at $ \eta_* $, i.e. at last scattering. These three terms are separate contributions:
\begin{itemize}
  \item the first term is the \emph{Sachs--Wolfe term}, which combines the intrinsic temperature fluctuations associated to the radiation density contrast with the induced temperature perturbations arising from the gravitational redshift of the photons. With the sign convention adopted in \eref{eq:perturbed-metric}, $ \Psi < 0 $ corresponds to an overdensity (since, in the Newtonian limit, $ \Psi \propto -M $), hence $ \Psi_* < 0 $ leads as expected to a temperature decrement, since a photon loses energy when climbing out of a potential well;
  \item the second term is a \emph{Doppler term}, due to the relative motion of the last-scattering surface with respect to the observer (see \figref{fig:cmb-d}). This contribution is subdominant on superhorizon scales, while its oscillations are out of phase with the oscillations in the Sachs--Wolfe term, thus reducing the contrast between the peaks in the spectrum;
  \item the last term describes the additional gravitational redshift due to the evolution of the metric potentials along the line-of-sight, determining what is known as the \emph{integrated Sachs--Wolfe effect} (ISW). Since for most of the universe's history (namely, when it was matter dominated) these potential were constants, thus yielding no ISW effect. On the contrary, at early times the residual amount of radiation gives time-dependent gravitational potentials, resulting in an \emph{early ISW effect}: this contribution raises the height of the first peak in the power spectrum (see \figref{fig:cmb-sw}).
\end{itemize}

\begin{figure}
  \centering
  \includegraphics[width = 0.40 \textwidth]{cmb-doppler.png}
  \hspace{6em}
  \includegraphics[width = 0.30 \textwidth]{cmb_de.png}
  \caption{Doppler effect on the last-scattering surface.}
  \label{fig:cmb-d}
\end{figure}

The CMB power spectrum \figref{fig:cmb} imposes some constraints on the amount of matter, and in particularly of dark matter (since baryonic matter is observed), in the universe: indeed, the extension of the Sachs--Wolfe plateau is determined by $ k_\text{eq} $, and changing the amount of matter changed the time at which matter-radiation equality happens. Moreover, the second and third peaks of the power spectrum are sensible to the gravitational potential of matter: varying the amount of matter causes a variation in the height difference between the peaks.











