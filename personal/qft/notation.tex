\selectlanguage{english}

\textbf{\sffamily Conventions}

In these notes, the Lorentz--Minkowski metric $ \eta_{\mu \nu} $ has signature $ (+,-,-,-) $ and Greek indices generally run over spacetime coordinates, while Latin indices are general $ \N_0 $-indices defined in each context. Repeated indices are generally summed over, unless otherwise specified. The $ n $-dimensional Levi--Civita symbol $ \epsilon^{i_1 \dots i_n} $ is defined with the convention $ \epsilon^{0 1 \dots n} = +1 $.

All quantities are expressed in natural units, defined by $ \hbar \equiv c \equiv 1 $: in particular, the ``dimensions" of a quantity generally refers to its mass dimension.

The normal ordering and the time ordering of an expression are respectively denoted by the normal-ordering operator $ \normord $ and by the time-ordering operator $ \tempord $.

Given $ \alpha \in \C^{n \times n} $, with $ n \in \N_0 $, its complex conjugate is denoted as $ \alpha^* $, its transpose as $ \alpha\tsp $ and its Hermitian conjugate as $ \alpha\dg \defeq (\alpha^*)\tsp $. Given a Dirac spinor $ \Psi \in \C^4 $, its Dirac dual (or Dirac adjoint) is defined as $ \bar{\Psi} \defeq \Psi\dg \gamma^0 $.

The Landau symbol for a function $ f : D \ssq \R \ra \C $ is defined by the condition $ \exists M \in \R \,:\, \abs{\smo(f(x))} \leq M \abs{f(x)} \,\,\forall x \in D $.

\vspace{1em}

\textbf{\sffamily Mathematical notation}

The empty set is denoted by $ \emptyset $ and the power set of a set $ A $ by $ \pwst{A} \defeq \{B : B \ssq A\} $. The counting numbers are $ \N \equiv \{1,2,3,\dots\} $, and the natural numbers are defined by $ \N_0 \equiv \{0\} \cup \N $.

The imaginary unit is denoted by $ \img $ and the unit quaternions by $ \img , \jmg , \kmg $, so that $ \C(\R) = \lspan(1,\img) $ and $ \mathbb{H}(\R) = \lspan(1,\img,\jmg,\kmg) $.

The permutation group of $ n $ objects, i.e. the $ n^\text{th} $ symmetric group, is denoted by $ \sy{n} $.

Given two $ \K $-vector spaces $ V $ and $ W $, with $ \K $ a generic field, the space of all $ \K $-linear applications $ f : V \ra W $ is denoted by $ \Hom_\K(V,W) $: in particular, $ \Hom_\K(V) \equiv \End(V) $. The subset of $ \End(V) $ of all automorphisms of $ V $ is the automorphism group $ \Aut(V) $, which is a group under composition of morphisms.

Given a manifold $ \mathcal{M} $, the space of all smooth scalar functions on $ \mathcal{M} $ is denoted by $ \cm $, the space of all vector fields on $ \mathcal{M} $ by $ \xm $, the space of all $ p $-forms on $ \mathcal{M} $ by $ \lm{p} $ and the Grassmann algebra of $ \mathcal{M} $ by $ \bigwedge(\mathcal{M}) \defeq \bigoplus_{k = 0}^n \lm{k} $.

The exterior derivative is denoted by $ \dd $, the partial derivative by $ \pa $, the nabla operator by $ \grad \equiv \left( \pa_1 , \pa_2 , \pa_3 \right) $, the Laplacian operator by $ \lap \equiv \grad^2 $ and the D'Alembert operator by $ \Box \equiv \pa_0^2 - \lap $.

A list of ``important" Lie groups:

$ \GL{n,\K} \defeq \{\mt{A} \in \K^{n \times n} : \det \mt{A} \neq 0\} $ general linear group (Lie group for $ \K = \R , \C $)

$ \SL{n,\K} \defeq \{\mt{A} \in \GL{n,\K} : \det \mt{A} = 1\} $ special linear group (Lie group for $ \K = \R , \C $)

$ \On{n} \defeq \{\mt{A} \in \R^{n \times n} : \mt{A} \mt{A}\tsp = \mt{A}\tsp \mt{A} = \mt{I}_n\} $ orthogonal group

$ \SOn{n} \defeq \{\mt{A} \in \On{n} : \det \mt{A} = 1 $ special orthogonal group

$ \Un{n} \defeq \{\mt{A} \in \C^{n \times n} : \mt{A} \mt{A}\dg = \mt{A}\dg \mt{A} = \mt{I}_n\} $ unitary group

$ \SUn{n} \defeq \{\mt{A} \in \Un{n} : \det \mt{A} = 1 $ special unitary group

Given a Lie group $ G $, its associated Lie algebra is denoted by $ \mathfrak{g} $.

\hspace{1em}

\textbf{\sffamily Physical notation}

The Lorentz group is denoted by $ \lorg $, the Lorentz algebra by $ \lora $, the Poincaré group by $ \pog $, the Poincaré algebra by $ \poa $ and the Dirac algebra by $ \dira $.

Hilbert and Fock spaces are generally denoted respectively by $ \hilb $ and $ \fock $, the Hamiltonian and the Lagrangian of a system by $ H $ and $ L $, and the Hamiltonian density and the Lagrangian density of a filed theory by $ \ham : H = \int \dd^3x \, H $ and $ \lag : L = \int \dd^3x\, L $. The action functional is defined as $ \act \defeq \int \dd t\, L = \int \dd^4x\, \lag $.

The matrix element of a scattering process is generally denoted by $ \mat $, and its amplitude by $ \ampl \equiv \abs{\mat}^2 $.

The parity operator is denoted by $ \parity $, the charge-conjugation operator by $ \chargec $ and the time-reversal operator by $ \timer $.

For a general gauge theory, the covariant derivative is denoted by $ \covder_\mu $, the connection (or gauge field) by $ \con_\mu $ and the field-strength tensor by $ \fs_{\mu \nu} $. For the particular case of QED, these quantities are denoted respectively by $ D_\mu $, $ A_\mu $ and $ F_{\mu \nu} $.

The Feynman propagator in position and momentum space is denoted by $ \fpr(x,y) \equiv \fpr(x - y) $ and $ \fprp(p) $, the Dirac propagator by $ \dpr(x,y) \equiv \dpr(x - y) $ and $ \dprp(p) $, and the photon propagator by $ \gpr_{\mu \nu}(x,y) \equiv \gpr_{\mu \nu}(x - y) $ and $ \gprp_{\mu \nu}(p) $.










