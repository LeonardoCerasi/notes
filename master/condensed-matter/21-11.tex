\selectlanguage{english}

\section{Lattice vibrations}

Like for molecules, we expect to find multiple normal modes, since the crystal can be seen as a molecule with an extremely large number of atoms: in particular, since for a molecule with $ N $ atoms there are $ 3N - 6 $ normal frequencies, for a crystal we similarly expect $ \sim 3N $ normal frequencies.

\subsection{Linear chain}

Consider a monoatomic linear chain, and adopt the approximation where the only interactions are between neighbouring atoms and have a harmonic nature. Then, label each atom with $ s = 0 , 1 , \dots , N - 1 $ and assume that their equilibrium distance is $ a $, so that the total length of the chain is $ L = (N - 1) a $, and that their mass is $ M $. Defining $ u_s $ as the longitudinal displacement (analogous to stretching modes in molecules) from its equilibrium position of the atom $ s $, then its equation of motion is:
\begin{equation}
  C \left( u_{s + 1} - 2 u_s + u_{s - 1} \right) = M \ddot{u}_s
  \label{eq:linear-chain}
\end{equation}
where $ C \in \R^+ $ is the spring constant of the harmonic interaction between neighbouring atoms. The wave ansatz reads:
\begin{equation}
  u_s(t) = u_0 e^{\img \left( k s a - \omega t \right)}
\end{equation}
where we set the $ s = 0 $ atom at the origin. Inserting this ansatz into \eref{eq:linear-chain}:
\begin{equation*}
  - \omega^2 M = C \left( e^{\img k a} + e^{-\img k a} - 2 \right) = 2C \left( \cos k a - 1 \right)
\end{equation*}
which results in the \bctxt{dispersion relation} for vibrations in a crystal:
\begin{equation}
  \omega = 2 \sqrt{\frac{C}{M}} \abs{\sin \frac{k a}{2}}
\end{equation}
Contrary to waves in a continuous medium (EM waves in vacuum $ \omega = c k $, sound waves in air $ \omega = v_\text{s} k $), crystal vibrations propagate in a discrete medium, hence their dispersion relation is non-linear (see \figref{fig:acoustic}).

\begin{figure}
  \centering
  \includegraphics[width = 0.60 \textwidth]{acoustic-branch.png}
  \caption{Acoustic branches in a monoatomic crystal.}
  \label{fig:acoustic}
\end{figure}

Note that this analysis holds for all atoms in the linear chains, except for the endpoints: since $ N \sim \av $, these endpoints can be ignored, and the chain can be assumed to be infinite. To do so, it is necessary to impose \bctxt{perdiodic boundary conditions}, i.e.:
\begin{equation}
  u(L) = u(0)
  \qquad \qquad
  u'(L) = u'(0)
\end{equation}
These imply that $ \exp \img k L = 1 $, i.e. they make the normal frequencies discrete. It is trivial to see that two waves whose wave-vectors differ by a vector in the discrete lattice. i.e. $ k $ and $ k + 2 \pi n / a $, determine the same normal mode: all of the physical information is then contained in the interval $ k \in \left( - \frac{\pi}{a} , \frac{\pi}{a} \right] $, as pictured in \figref{fig:acoustic}.
The discrete normal frequencies then are:
\begin{equation}
  k_n = \frac{2\pi n}{L}
  \qquad
  n = 0, \pm 1 , \pm 2 , \dots , + \frac{N}{2}
\end{equation}
where $ N $ is assumed even. The total number of normal modes has been computed as:
\begin{equation*}
  \frac{2\pi}{a} \div \frac{2\pi}{L} = \frac{L}{a} = N
\end{equation*}
Note that the $ n = 0 $ normal mode is a translation, since $ k = 0 $ and $ \omega = 0 $. Moreover, we again see a reciprocity between real space and momentum space: the finiteness of the chain determines discrete normal frequencies, while its discreteness determines a finite number of normal frequencies.

Consider now the limit for $ \abs{k} \ll \frac{\pi}{a} $, i.e. $ \lambda \gg 2a $: since the wavelength is much larger than the separation between atoms, the oscillation is not affected by the discreteness of the medium, hence the dispersion curve is linear:
\begin{equation*}
  \lim_{k \ra 0} \omega = \sqrt{\frac{C}{M}} a k \equiv v_\text{s} k
\end{equation*}
where $ v_\text{s} $ is the ``sound wave" in the 1D crystal considered. In general, the \bctxt{group velocity}, i.e. the velocity in the transfer of mechanical energy in the crystal, is defined as:
\begin{equation}
  v_\text{g}(k) \defeq \frac{\dd \omega}{\dd k}
\end{equation}
For the 1D crystal:
\begin{equation*}
  v_\text{g}(k) = \sgn(k) \sqrt{\frac{C}{M}} \cos \frac{k a}{2}
\end{equation*}
Note that the group velocity vanishes at the boundary $ v_\text{g}(\pm \frac{\pi}{a}) = 0 $. To understand this, consider the displacement of the atoms:
\begin{equation*}
  u_s(t) = u_0 e^{\pm \img \pi s - \omega t} = (-1)^s u_0 e^{- \img \omega t}
\end{equation*}
which is a stationary wave.

\subsection{3D crystal}

In the case of a 3D crystal, it is still possible to have both longitudinal and transversal: however, contrary to the 1D crystal, the elastic constant of the oscillation is an ``effective" elastic constant, since each atom has multiple ``closest" atoms.

Consider an oscillation with wave-vector $ k $ along a principal crystal direction in an sc crystal, WLOG $ [1 \ 0 \ 0] $. Since this is a wave, it can undergo Bragg scattering, and in particular there could be the phenomenon of \bctxt{back-scattering}, i.e. the wave is reflected backward: from the Bragg relation \eref{eq:bragg} with $ n = 1 $ and $ \vartheta = \frac{\pi}{2} $ (reflection), since the distance between two atoms in the $ [1 \ 0 \ 0] $ direction is $ d = a $, it is trivial to find:
\begin{equation*}
  \lambda = 2a
  \quad \iff \quad
  k = \pm \frac{\pi}{a}
\end{equation*}
Now, note that if the wave is totally reflected, the total oscillation given by the sum of the two waves is a stationary wave, hence the same result as for the 1D crystal is recovered.

If instead the oscillation propagates along the $ [1 \ 1 \ 0] $ crystal direction, then $ d = \frac{a}{\sqrt{2}} $ and the back-scattering condition becomes:
\begin{equation*}
  \lambda = \sqrt{2} a
  \quad \iff \quad
  k = \pm \sqrt{2} \frac{\pi}{a}
\end{equation*}
In the general case, the back-scattering happens for all wave-vector which lie on the boundary of the primitive cell of the reciprocal lattice, which is called \bctxt{first Brilluoin zone} (1BZ): indeed, it is possible to show that, given a wave-vector $ \ve{k} $ on the boundary of the 1BZ and its diffracted wave-vector $ \ve{k}' $ according to Bragg diffraction, then $ \norm{\ve{k}' - \ve{k}} = \frac{2\pi}{a} $, i.e. a stationary wave.

\subsubsection{Vibrational bands}

\begin{example}{Biatomic linear chain}{}
  Consider a 1D crystal where two kinds of atoms, with masses $ M_1 $ and $ M_2 $, alternate: the primitive cell is then composed of a couple of nearby atoms, and define $ s \in \N_0 $ the index of the primitive cells in the chain (of periodicity $ a $).

  In the first-order harmonic approximation where only nearby atoms interact with an elastic constant $ C $, denoting the displacement of the mass $ M_1 $ in the $ s $ cell as $ u_s $ and that of the mass $ M_2 $ as $ v_s $, the equations of motion read:
  \begin{equation*}
    M_1 \ddot{u}_s = C \left( v_s - 2 u_s + v_{s - 1} \right)
    \qquad \qquad
    M_2 \ddot{v}_s = C \left( u_{s + 1} - 2 v_s + u_s \right)
  \end{equation*}
  The plane-wave ansatz now reads:
  \begin{equation*}
    u_s(t) = u_0 \exp \left( \img s k a - \img \omega t \right)
    \qquad \qquad
    v_s(t) = v_0 \exp \left( \img s k a - \img \omega t + \img \varphi \right) \equiv v_0' \exp \left( \img s k a - \img \omega t \right)
  \end{equation*}
  with $ u_0 , v_0 \in \R $ and $ v_0' \in \C $. Inserting these expressions in the equations of motion:
  \begin{equation*}
    - \omega^2 M_1 = C \left[ v_0' \left( 1 + e^{-\img k a} \right) - 2 u_0 \right]
    \qquad \qquad
    - \omega^2 M_2 = C \left[ u_0 \left( e^{\img k a} + 1 \right) - 2 v_0' \right]
  \end{equation*}
  which reduce to a linear system:
  \begin{equation*}
    \begin{bmatrix}
      2C - \omega^2 M_1 & -C \left( 1 + e^{-\img k a} \right) \\
      -C \left( 1 + e^{\img k a} \right) & 2C - \omega^2 M_2
    \end{bmatrix}
    \begin{pmatrix}
      u_0 \\ v_0'
    \end{pmatrix}
    = \ve{0}
    \quad \implies \quad
    \begin{vmatrix}
      2C - \omega^2 M_1 & -C \left( 1 + e^{-\img k a} \right) \\
      -C \left( 1 + e^{\img k a} \right) & 2C - \omega^2 M_2
    \end{vmatrix}
    = 0
  \end{equation*}
  The equation for $ \omega^2 $ then is:
  \begin{equation*}
    M_1 M_2 \omega^4 - 2 C \left( M_1 + M_2 \right) \omega^2 + 2 C^2 \left( 1 - \cos ka \right) = 0
  \end{equation*}
  which in general has two solutions for $ \omega^2 $. The interesting behavior is in the limits for $ k \ra 0 $ and $ k \ra \pm \frac{\pi}{a} $. For $ k \ra 0 $:
  \begin{equation*}
    M_1 M_2 \omega^2 - 2C \left( M_1 + M_2 \right) \omega^2 + C^2 (ka)^2 = 0
  \end{equation*}
  whose solutions are:
  \begin{equation*}
    \begin{split}
      \omega^2
      & = \frac{C \left( M_1 + M_2 \right) \pm \sqrt{C^2 \left( M_1 + M_2 \right)^2 - C^2 M_1 M_2 (ka)^2}}{M_1 M_2} \\
      & = \frac{C}{\mu} \left[ 1 \pm \sqrt{1 - \mu (ka)^2} \right] \approx \frac{C}{\mu} \left[ 1 \pm 1 \mp \frac{\mu}{2(M_1 + M_2)} (ka)^2 + \smo(k^3) \right]
    \end{split}
  \end{equation*}
  where $ \mu = M_1 M_2 / (M_1 + M_2) $ is the reduced mass of the primitive cell. Then, at first order:
  \begin{equation*}
    \omega_+ = \sqrt{\frac{2C}{\mu}} + \smo(k^2)
    \qquad \qquad
    \omega_- = \sqrt{\frac{C}{2(M_1 + M_2)}} a k + \smo(k^2) \equiv v_\text{s} k + \smo(k^2)
  \end{equation*}
  The $ \omega_- $ is analogous to that of the monoatomic linear chain, while $ \omega_+ $ is new. Substituting $ \omega_+ $ in the above linear system and setting $ k = 0 $:
  \begin{equation*}
    \begin{cases}
      \left( 1 - \frac{M_1}{\mu} \right) u_0 - v_0' = 0 \\
      - u_0 + \left( 1 - \frac{M_2}{\mu} \right) v_0' = 0
    \end{cases}
    \quad \iff \quad
    v_0' = - \frac{M_1}{M_2} u_0
  \end{equation*}
  This means that the two atoms in the primitive cell oscillate in phase-opposition, corresponding to a stationary wave: this justifies the dispersion curve being parallel to the $ k $-axis, since the group velocity of a stationary wave is $ v_\text{g} = 0 $. Note that, on the other hand, $ \omega_- = 0 $ at $ k = 0 $, which is not a vibration, but a translation ($ u_0 = v_0' $).

  Now, consider $ k \ra \pm \frac{\pi}{a} $:
  \begin{equation*}
    M_1 M_2 \omega^4 - 2C \left( M_1 + M_2 \right) \omega^2 + 4C^2 = 0
  \end{equation*}
  which has solutions:
  \begin{equation*}
    \omega^2 = \frac{C \left( M_1 + M_2 \right) \pm \sqrt{C^2 \left( M_1 + M_2 \right)^2 - 4C^2 M_1 M_2}}{M_1 M_2} = \frac{C \left( M_1 + M_2 \right) \pm C \left( M_1 - M_2 \right)}{M_1 M_2}
  \end{equation*}
  where $ M_1 > M_2 $ is assumed WLOG. The two solutions then read:
  \begin{equation*}
    \omega_+ = \frac{2C}{M_2} + \smo\left( \left( k \mp \frac{\pi}{a} \right)^2 \right)
    \qquad \qquad
    \omega_- = \frac{2C}{M_1} + \smo\left( \left( k \mp \frac{\pi}{a} \right)^2 \right)
  \end{equation*}
  To show that these correctly are stationary waves, insert these expressions in the above linear system with $ k = \pm \frac{\pi}{a} $: for $ \omega_+ $ it is $ u_0 = 0 $, and indeed the frequency only depends on $ M_2 $, and conversely for $ \omega_- $ it is $ v_0' = 0 $, and the frequency only depends on $ M_1 $.

  Note that $ \omega_+(0) > \omega_+(\pm \frac{\pi}{a}) $ and $ \omega_-(0) < \omega_-(\pm \frac{\pi}{a}) $. The full solutions $ \omega_+(k) $ and $ \omega_-(k) $ are plotted in \figref{fig:opt-mod} (where both the longitudinal and the two transversal normal modes are present): these are called bands, and in particular $ \omega_-(k) $ is an \bcex{acoustic band}, since it is linear for $ k \ra 0 $ and the atoms oscillate in phase, while $ \omega_+(k) $ is an \bcex{optical band}, since it is constant for $ k \ra 0 $ and the atoms oscillate in phase-opposition.
\end{example}

\begin{figure}
  \centering
  \includegraphics[width = 0.50 \textwidth]{opt-mod.png}
  \caption{Phonon dispersion curves for an $ n_d = 2 $ crystal.}
  \label{fig:opt-mod}
\end{figure}

In general, a crystal with $ n_d \in \N $ atoms in its primitive cell presents $ 3 $ acoustic bands and $ 3n_d - 3 $ optical bands, for a total of $ 3n_d $ bands.

To experimentally measure vibrational bands, consider an EM waves scattering on the sample crystal: the interaction with the EM waves makes electric charge oscillate, with positive charges and negative charges oscillating in opposite directions. To make the crystal vibrate, the EM wave $ \omega(\ve{k}) $ must resonate with the vibrational bands $ \Omega(\ve{K}) $, i.e. $ \ve{k} = \ve{K} $ and $ \Omega(\ve{k}) = \omega(\ve{k}) = \frac{c}{n} k $, where $ n $ is the refractive index of the crystal. This method only allows to study optical bands\footnotemark, since for acoustic bands $ v_\text{s} \ll \frac{c}{n} $. Quantistically, these resonance conditions correspond to an energy-momentum conservation condition for the transformation\footnotemark of a photon of energy $ \hbar \omega $ and momentum $ \hbar \ve{k} $ to a phonon of energy $ \hbar \Omega $ and momentum $ \hbar \ve{K} $.

\footnotetext{A scattering $ \gamma \ra \gamma + \varphi $ is also possible, but energetic photons are needed: IR photons generate resonances, visible photons generate scatterings. In general, three peaks are observed: one for the resonance and two smaller ones, one for $ \gamma \ra \gamma + \varphi $ at $ \omega - \Omega $ and one for $ \gamma + \varphi \ra \gamma $ at $ \omega + \Omega $ (Brilluoin anelastic scattering).
For even higher energies, e.g. X-rays, it is possible to combine Brilluoin scattering and Bragg scattering, so that the total energy-momentum conservation becomes:
\begin{equation*}
  \ve{k} - \ve{G} = \ve{k}' \pm \Omega
  \qquad \qquad
  \omega = \omega' \pm \Omega
\end{equation*}
since the crystal momentum $ \ve{G} $ does not determine an energy transfer between the EM wave and the crystal.}

\footnotetext{Acoustic bands can be studied using neutrons, since their dispersion relations $ E = \frac{\hbar^2 k^2}{2m_n} $ allows for an intersection even with acoustic bands, which have lower frequencies than optical bands.}

\subsection{Thermal properties}

For a perfect gas composed of $ N $ particles, the mean energy of the system is found by the \emph{equipartition principle} for energy:
\begin{equation*}
  \braket{E} = \frac{3}{2} N \kbt
\end{equation*}
Then, recalling that $ \dd Q = \dd E + P \dd V $, its specific heat and molar specific heat at constant volume are:
\begin{equation*}
  C_V \defeq \frac{\dd Q}{\dd T} = \frac{3}{2} N \kb
  \qquad \qquad
  c_V \defeq \frac{C_V}{n} = \frac{3}{2} R
\end{equation*}
with $ R \equiv \av \kb $.

The same classical treatment can be performed for a crystal: in this case, there are both kinetic and potential degrees of freedom, hence there is twice the total number of degrees of freedom, i.e. the expected classical result is $ c_V = 3 R $.

However, the crystal is a quantum system, hence the treatment must be extended to account for quantum effects. A crystal at $ T > 0 $ has all its atoms randomly oscillating: this random motion can be expressed as a linear combination of normal modes.

Suppose that we can filter a particular frequency $ \omega $: in general, the equation $ \omega(k) = \omega $ has multiple solutions for $ k $ in multiple bands, and they all depend on the crystal direction considered (since bands in different directions are generally different). Nonentheless, quantistically the energy only depends on $ \omega $ and it is quantized (for oscillations in a single direction) as:
\begin{equation}
  E_n = \left( n + \frac{1}{2} \right) \hbar \omega
  \label{eq:osc-quant}
\end{equation}
where the population of the $ n^\text{th} $ energy level follows a Boltzmann distribution:
\begin{equation}
  P(n) = A e^{- \frac{E_n}{\kbt}}
\end{equation}
Taking the ground state as reference $ E_0 \equiv 0 $, then:
\begin{equation*}
  \braket{E} = \frac{\sum_{n \in \N_0} n \hbar \omega P(n)}{\sum_{n \in \N_0} P(n)} = \frac{\sum_{n \in \N_0} n \hbar \omega e^{- n \frac{\hbar \omega}{\kbt}}}{\sum_{n \in \N_0} e^{- n \frac{\hbar \omega}{\kbt}}} = \hbar \omega \frac{\sum_{n \in \N_0} n e^{nx}}{\sum_{n \in \N_0} e^{nx}}
\end{equation*}
where $ x \equiv - \frac{\hbar \omega}{\kbt} $. Since $ e^x \in (0,1) $:
\begin{equation*}
  \braket{E} = \hbar \omega \left( \frac{1}{1 - e^x} \right)^{-1} \frac{\dd}{\dd x} \frac{1}{1 - e^x} = \frac{\hbar \omega}{e^{-x} - 1}
\end{equation*}
i.e.:
\begin{equation}
  \braket{E} = \frac{\hbar \omega}{e^\frac{\hbar \omega}{\kbt} - 1}
\end{equation}
Note that $ \lim_{T \ra \infty} \braket{E} = \kbt $: since there are three directions of oscillation (each with its own quantum number $ n_j $), the classical limit $ c_V = 3R $ is found. Moreover, writing $ \braket{E} = \hbar \omega n(\omega,T) $ from \eref{eq:osc-quant}, the \bctxt{phonon distribution} is found:
\begin{equation}
  n(\omega,T) = \frac{1}{e^\frac{\hbar \omega}{\kbt} - 1}
\end{equation}
where phonons are the quanta of vibrational energy. This is the Bose-Einstein distribution, hence phonons are bosons. At fixed temperature, the singularity at $ \omega = 0 $ means that lower level are more populated, since they require less energy; at fixed frequency, the distribution is asymptotically linear in $ T $ with angular coefficient $ \kb / (\hbar \omega) $, which means that lower frequencies are more easily populated since they require less energy.

Experimentally, the classical limit $ \lim_{T \ra \infty} c_V = 3R $ is confirmed. Moreover, we find that $ \lim_{T \ra 0} c_V \sim T^3 $: this is compatible with the quantum description, since for a sufficiently small temperature the energy will not be enough to excite the atoms to the first excited state, hence only the ground state will be populated and $ c_V = 0 $.

\subsubsection{Einstein model}

To study the limit of $ c_V(T) $ as $ T \ra 0 $, make the (erroneous) simplification that all atoms oscillate with the same frequency $ \omega_\text{E} $, thus ignoring the vibrational bands. Therefore:
\begin{equation}
  E_\text{E} = \frac{3N \hbar \omega}{e^\frac{\hbar \omega}{\kbt} - 1}
  \quad \implies \quad
  c_V = 3R \left( \frac{\hbar \omega}{\kbt} \right)^2 \frac{e^\frac{\hbar \omega}{\kbt}}{\left( e^\frac{\hbar \omega}{\kbt} - 1 \right)^2}
\end{equation}
In the low-temperature limit:
\begin{equation*}
  c_V \approx 3R \left( \frac{\hbar \omega}{\kbt} \right)^2 e^{- \frac{\hbar \omega}{\kbt}}
\end{equation*}
which correctly goes to $ c_V = 0 $, but it does not reproduce the experimental behavior.

\subsubsection{Debye model}

An improved model accounts for the various possible frequencies. The simplification adopted in this case is that these frequencies are continuously-distributed (i.e. we assume $ a \ra 0 $). Hence:
\begin{equation}
  E_\text{D} = \int_{\R^+} \dd \omega \, D(\omega) \frac{\hbar \omega}{e^\frac{\hbar \omega}{\kbt} - 1}
\end{equation}
where $ D(\omega) $ is the density of state, defined so that the number of states between $ \omega $ and $ \omega + \dd \omega $ is $ \dd N = D(\omega) \dd \omega $. To perform the computation, switch to the $ k $ variable, so that $ \dd N = D(\omega) \dd \omega = \tilde{D}(k) \dd k $: then, for the monoatomic linear chain of total length $ L $:
\begin{equation*}
  \tilde{D}(k) = \frac{\dd N}{\dd k} = \frac{1}{\frac{2\pi}{L}} = \frac{L}{2\pi}
\end{equation*}
since the spacing between discretized wave-vectors is $ \frac{2\pi}{L} $.

\paragraph{High-temperature limit}

In the high-temperature limit:
\begin{equation*}
  E_\text{D} \simeq \kbt \int_{\R^+} \dd \omega \, D(\omega) = 3N \kbt
\end{equation*}
since the total number of states is approximately $ 3N $ (for $ N \gg 1 $). This is the expected classical result.

\paragraph{Low-temperature limit}

In the low-temperature limit, on the other hand, due to the functional expression of the Bose--Einstein distribution, only low frequencies contribute to the integral, since high frequencies are exponentially suppressed: this means that only acoustic bands contribute to the integral, and we can set $ \omega \approx v_\text{s} k $. Supposing an isotropic crystal, so that $ v_\text{s} = \text{const.} $ in all directions, then near the origin in the $ k $-space (reciprocal space) the $ \omega = \text{const.} $ surfaces are spheres, i.e. $ k = \text{const.} $: this allows to directly count the number of states between $ \omega $ and $ \omega + \dd \omega $. The number of states inside the sphere of frequency $ \omega $ is simply given by:
\begin{equation}
  N(\omega) = \frac{\frac{4}{3} \pi k^3}{\left( \frac{2\pi}{L} \right)^3} = \frac{V}{6\pi^2 v_\text{s}^3} \omega^3
\end{equation}
where $ (2\pi)^3 / V $ is the volume of a single state in the reciprocal space. Then:
\begin{equation}
  D(\omega) = \frac{\dd N(\omega)}{\dd \omega} = \frac{V}{2\pi^2 v_\text{s}^3} \omega^2
\end{equation}
Since this only holds for low-frequencies, we introduce a cut-off frequency $ \omega_\text{D} $, called \bctxt{Debye frequency}, so that $ D(\omega) = 0 $ for $ \omega > \omega_\text{D} $ and:
\begin{equation}
  \int_0^{\omega_\text{D}} \dd \omega \, D(\omega) = 3N
  \quad \implies \quad
  \omega_\text{D} = v_\text{s} \sqrt[3]{6\pi^2 \frac{N}{V}}
\end{equation}
The energy then becomes:
\begin{equation*}
  E_\text{D} = \frac{V}{2\pi^2 v_\text{s}^3} \int_0^{\omega_\text{D}} \dd \omega \frac{\hbar \omega^3}{e^\frac{\hbar \omega}{\kbt} - 1} = \frac{3 \hbar V}{2\pi^2 v_\text{s}^3} \left( \frac{\kbt}{\hbar} \right)^4 \int_0^{x_\text{D}} \dd x \frac{x^3}{e^x - 1}
\end{equation*}
If $ T \ll \hbar \omega_\text{D} / \kb \equiv \theta_\text{D} $ (Debye temperature), then $ x_\text{D} \ra \infty $ and the integral is equal to $ \frac{\pi^4}{15} $:
\begin{equation}
  E_\text{D} \simeq \frac{\pi^2 \kb^4 V}{10 \hbar^3 v_\text{s}^3} T^4
\end{equation}
The specific heat then is:
\begin{equation}
  C_V = \frac{2}{5} \frac{\pi^2 \kb^4 V}{\hbar^3 v_\text{s}^3} T^3 = \frac{12 \pi^4}{5} N \kb \left( \frac{T}{\theta_\text{D}} \right)^3
\end{equation}
which is the expected classical result.










