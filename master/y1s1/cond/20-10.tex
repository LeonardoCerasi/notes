\selectlanguage{english}

\chapter{Lecture 20/10--24/10--27/10}

\section{Born--Oppenheimer approximation}

Consider a system of $ N_n $ nuclei and $ N_e $ electrons. The Hamiltonian of the system then is:
\begin{equation}
  \ham
  = - \frac{\hbar^2}{2} \sum_{\alpha = 1}^{N_n} \frac{\lap_\alpha}{M_\alpha} - \frac{\hbar^2}{2m_e} \sum_{i = 1}^{N_n} \lap_i
  + \frac{1}{2} \sum_{i \neq j = 1}^{N_n} \frac{e^2}{r_{ij}} + \frac{1}{2} \sum_{\alpha \neq \beta = 1}^{N_n} \frac{Z_\alpha Z_\beta e^2}{R_{\alpha \beta}} - \sum_{i = 1}^{N_e} \sum_{\alpha = 1}^{N_n} \frac{Z_\alpha e^2}{\norm{\ve{R}_\alpha - \ve{r}_i}}
\end{equation}
with $ r_{ij} \equiv \norm{\ve{r}_i - \ve{r}_j} $ and $ R_{\alpha \beta} \equiv \norm{\ve{R}_\alpha - \ve{R}_\beta} $. The total wave-function of the system describes both the nuclear and the electronic component, and it can be written as $ \Phi \equiv \Phi(\{\ve{R}_\alpha\} , \{\ve{r}_i\}) $.

Since $ M_\alpha \gg m_e \,\,\forall \alpha = 1, \dots, N_n $ (circa three orders of magnitudes), the timescales of nuclear processes are three orders of magnitude longer than the timescales of atomic (i.e. electronic) processes, hence we can approximate the two components of the system as factorized:
\begin{equation}
  \Phi(\{\ve{R}_\alpha\} , \{\ve{r}_i\}) = \Phi_n(\{\ve{R}_\alpha\}) \Phi_e(\{\ve{R}_\alpha\} , \{\ve{r}_i\})
\end{equation}
where we approximated the nuclear state to be independent of the electronic state, while the electronic state is supposed to react instantly to changes in the nuclear state. This is the \bctxt{Born--Oppenheimer approximation}, which is an adiabatic approximation\footnotemark.

\footnotetext{In general, an adiabatic approximation is one in which one or more components of a system are supposed to change slowly over time, compared to the other components.}

In the following, we adopt the shorthands $ R \equiv \{\ve{R}_\alpha\}_{\alpha = 1, \dots, N_n} $ and $ r \equiv \{\ve{r}_i\}_{i = 1, \dots, N_e} $, and we rename the various terms of the total Hamiltonian as:
\begin{equation*}
  \ham = T_n + T_e + V_{ee}(r) + V_{nn}(R) + V_{ne}(R,r)
\end{equation*}

\begin{theorem}{Atomic and nuclear wave-equations}{}
  In the Born--Oppenheimer approximation, the atomic and nuclear wave-functions respectively solve:
  \begin{equation}
    \left[ T_e + V_{ee}(r) + V_{ne}(R,r) \right] \Phi_e^{(g)}(R,r) = E_e^{(g)}(R) \Phi_e^{(g)}(R,r)
  \end{equation}
  \begin{equation}
    \left[ T_n + \vad^{(g)}(R) \right] \Phi_n(R) = E \Phi_n(R)
  \end{equation}
  where $ E $ is the total energy of the system and the \bcth{adiabatic potential}, dependent on the electronic state, is defined as:
  \begin{equation}
    \vad^{(g)}(R) \equiv E_e^{(g)}(R) + V_{nn}(R)
    \label{eq:adiabatic-potential-def}
  \end{equation}
\end{theorem}

\begin{proofbox}
  \begin{proof}
    The non-trivial terms of the total Hamiltonian are the operators $ T_e $ and $ T_n $, so we study their action of the total wave-function:
    \begin{equation*}
      T_e \Phi(R,r) = T_e \left[ \Phi_n(R) \Phi_e(R,r) \right] = \Phi_n(R) T_e \Phi_e(R,r)
    \end{equation*}
    \begin{equation*}
      \begin{split}
        T_n \Phi(R,r)
        & = T_n \left[ \Phi_n(R) \Phi_e(R,r) \right] \equiv - \frac{\hbar^2}{2} \sum_{\alpha = 1}^{N_n} \frac{\lap_\alpha}{M_\alpha} \left[ \Phi_n(R) \Phi_e(R,r) \right] \\
        & = - \frac{\hbar^2}{2} \sum_{\alpha = 1}^{N_n} \frac{1}{M_\alpha} \hspace{-0.1em} \left[ \Phi_e(R,r) \lap_\alpha \Phi_n(R) \hspace{-0.1em} + \Phi_n(R) \lap_\alpha \Phi_e(R,r) \hspace{-0.1em} + 2 \grad_\alpha \Phi_n(R) \hspace{-0.1em} \cdot \hspace{-0.1em} \grad_\alpha \Phi_e(R,r) \right] \\
        & = \Phi_e(R,r) T_n \Phi_n(R) - \sum_{\alpha = 1}^{N_n} \frac{\hbar^2}{2M_\alpha} \left[ \Phi_n(R) \lap \Phi_e(R,r) + 2 \grad_\alpha \Phi_n(R) \cdot \grad_\alpha \Phi_e(R,r) \right]
      \end{split}
    \end{equation*}

    WTS the summation is negligible in the Born--Oppenheimer approximation. To do so, first consider the operator $ \mathcal{O}_1 : \mathcal{O}_1 \Phi(R,r) \equiv \frac{\hbar^2}{M_\alpha} \grad_\alpha \Phi_n(R) \cdot \grad_\alpha \Phi_e(R,r) $ and compute its expectation value:
    \begin{equation*}
      \begin{split}
        \braket{\mathcal{O}_1}
        & = - \frac{\hbar^2}{M_\alpha} \int_{\R^{3(N_n + N_e)}} \dd^{3N_n} \hspace{-0.2em} R \, \dd^{3N_e} r \, \Phi_n(R)^* \Phi_e(R,r)^* \grad_\alpha \Phi_n(R) \cdot \grad_\alpha \Phi_e(R,r) \\
        & = - \frac{\hbar^2}{M_\alpha} \int_{\R^{3N_n}} \dd^{3N_n} \hspace{-0.2em} R \, \Phi_n(R)^* \grad_\alpha \Phi_n(R) \cdot \int_{\R^{3N_e}} \dd^{3N_e} r \, \Phi_e(R,r)^* \grad_\alpha \Phi_e(R,r) \\
        & \simeq - \frac{\hbar^2}{M_\alpha} \int_{\R^{3N_n}} \dd^{3N_n} \hspace{-0.2em} R \, \Phi_n(R)^* \grad_\alpha \Phi_n(R) \cdot \frac{1}{2} \grad_\alpha \int_{\R^{3N_e}} \dd^{3N_e} r \, \Phi_e(R,r)^* \Phi_e(R,r) = 0
      \end{split}
    \end{equation*}
    since the last integral is constant (assuming $ \Phi_e(R,r) $ to be normalized). The main approximation employed it that $ \Phi_e(R,r) \grad_\alpha \Phi_e(R,r)^* \simeq \Phi_e(R,r)^* \grad_\alpha \Phi_e(R,r) $: assuming this to be valid, we can suppress the second term in the summation, as its expectation value vanishes (this is another approximation).

    Now, focus on the first term in the summation. The dependence of $ \Phi_e $ on nuclear and atomic coordinates $ \{\ve{R}_\alpha\}_{\alpha = 1, \dots, N_n} , \{\ve{r}_i\}_{i = 1, \dots, N_e} $ is a relative dependence, i.e. it depends on relative distances of the kind $ \{\ve{R}_\alpha - \ve{r}_i\}_{\alpha = 1, \dots, N_n}^{i = 1, \dots, N_e} $, hence we see that:
    \begin{equation*}
      - \frac{\hbar^2}{2M_\alpha} \Phi_n(R) \lap_\alpha \Phi_e(R,r) = - \frac{\hbar^2}{2M_\alpha} \Phi_n(R) \sum_{i = 1}^{N_e} \lap_i \Phi_e(R,r) = \frac{m_e}{M_\alpha} T_e \Phi(R,r)
    \end{equation*}
    But $ m_e \lesssim 10^{-3} M_\alpha \,\,\forall \alpha = 1, \dots, N_n $, hence this term too is negligible. The Schrödinger equation then becomes:
    \begin{multline*}
      \Phi_e(R,r) T_n \Phi_n(R) + \Phi_n(R) T_e \Phi_e(R,r) + \left[ V_{ee}(r) + V_{ne}(R,r) + V_{nn}(R) \right] \Phi_n(R) \Phi_e(R,r) \\
      = E \Phi_n(R) \Phi_e(R,r)
    \end{multline*}
    which can be rewritten as:
    \begin{equation*}
      \left[ \ham_e(R,r) + \ham_n(R) \right] \Phi_n(R) \Phi_e(R,r) = E \Phi_n(R) \Phi_e(R,r)
    \end{equation*}
    \begin{equation*}
      \ham_e(R,r) \equiv T_e + V_{ee}(r) + V_{ne}(R,r)
      \qquad \qquad
      \ham_n(R) \equiv T_n + V_{nn}(R)
    \end{equation*}
    To solve for the electronic state, consider the nuclear state $ \Phi_n(R) $ fixed, i.e. $ \{\ve{R}_\alpha\}_{\alpha = 1, \dots, N_n} $ fixed positions:
    \begin{equation*}
      \ham_e(R,r) \Phi_e(R,r) = E_e(R) \Phi_e(R,r)
    \end{equation*}
    where the energy $ E_e(R) = E_e^{(g)}(R) $ of the electronic state $ g $ has a parametric dependence on the nuclear state through $ \{\ve{R}_\alpha\}_{\alpha = 1, \dots, N_n} $. Then, since $ \ham_n(R,r) \Phi_e(R) = \Phi_e(R) \ham_n(R,r) $ as shown above, it is clear that the evolution of the nuclear state is given by:
    \begin{equation*}
      \left[ \ham_n(R) + E_e^{(g)}(R) \right] \Phi_n(R) = E \Phi_n(R)
    \end{equation*}
    which is the thesis.
  \end{proof}
\end{proofbox}

An example of adiabatic potential for a two-electron atom is pictured in \figref{fig:adiabatic-2e}

\begin{figure}[h]
  \centering
  \includegraphics[width = 0.50 \textwidth]{adiabatic-potential.png}
  \caption{Adiabatic potential for a two-electron atom.}
  \label{fig:adiabatic-2e}
\end{figure}

We now focus on the nuclear motion: in particular, we denote the nuclear wave-function as $ \Phi_n(R) \equiv \Phi_\nu^{(g)}(R) $, where $ \nu $ are the quantum numbers relative to the nuclear state and $ g $ those relative to the electronic state. The Schrödinger equation then becomes:
\begin{equation}
  \left[ - \sum_{\alpha = 1}^{N_n} \frac{\hbar^2}{2M_\alpha} \lap_\alpha + \vad^{(g)}(R) \right] \Phi_\nu^{(g)}(R) = E_{g,\nu} \Phi_\nu^{(g)}(R)
\end{equation}

\subsection{Biatomic molecules}

In a biatomic molecule, the nuclear state can be described by a single quantity: indeed, given the positions $ \ve{R}_1 $ and $ \ve{R}_2 $ of the two nuclei, only the relative distance $ \ve{R} \equiv \ve{R}_1 - \ve{R}_2 $ is dynamically significant, since the center-of-mass motion of $ \ve{R}_\text{cm} $ is a free-particle motion which can be eliminated by working in the CM frame. In this frame:
\begin{equation*}
  \left[ - \frac{\hbar^2}{2\mu} \lap + \vad^{(g)}(R) \right] \Phi_\nu^{(g)}(\ve{R}) = E_{g,\nu} \Phi_\nu^{(g)}(\ve{R})
  \qquad \qquad
  \mu \equiv \frac{M_1 M_2}{M_1 + M_2}
\end{equation*}
where the adiabatic potential only depends on $ R \equiv \norm{\ve{R}} $, i.e. it is a central potential. Then, the solution can be factored as:
\begin{equation*}
  \Phi_\nu^{(g)}(\ve{R}) = \rf_\nu^{(g)}(R) Y_{K,M}(\vartheta , \varphi)
\end{equation*}
where $ K $ and $ M $ are the quantum numbers associated respectively to the orbital angular momentum and its projection on the zenithal axis. The radial Schrödinger equation is:
\begin{equation}
  \left[ - \frac{\hbar^2}{2\mu} \lap + \vad^{(g)}(R) + \frac{\hbar^2 K(K+1)}{2\mu R^2} \right] \rf_\nu^{(g)}(R) = E_{g,\nu} \rf_\nu^{(g)}(R)
  \label{eq:2-a-radial}
\end{equation}

Supposing that $ R \approx R_\text{eq} $, where $ R_\text{eq} $ is the equilibrium distance of $ \vad^{(g)}(R) $, then the third term can be seen as the rotational energy of a rigid body\footnotemark with $ I = \mu R_\text{eq}^2 $, since by definition $ \braket{\ve{L}^2} = \hbar^2 K(K+1) $:
\begin{equation}
  E_\text{rot}(K) = \mathcal{B} K(K+1)
  \label{eq:rotational-spectrum}
\end{equation}
where $ \mathcal{B} \equiv \frac{\hbar^2}{2I} \sim 1 \,\text{meV} - 10 \,\text{meV} $ is the rotational constant of the molecule. Keeping the electronic state fixed, as electronic transitions require higher energies, we can study purely-rotational transitions: these transitions have the same selection rules of electric-dipole transitions, i.e. $ \Delta K = \pm 1 $, hence the \bctxt{purely-rotational spectrum} is:
\begin{equation}
  \Delta E_\text{rot}(K) \equiv E_\text{rot}(K+1) - E_\text{rot}(K) = 2\mathcal{B} (K+1)
\end{equation}
Purely-rotational spectral lines are then equidistantly spaced by an energy $ 2 \mathcal{B} $ (typically in the IR or MW regions). The intensity $ I_K $ of the $ K^\text{th} $ line is proportional to the fraction $ n_K \equiv \frac{N_K}{N_\text{tot}} $ of molecules in the state $ K $, described by the Boltzmann distribution:
\begin{equation}
  n_K(T) = (2K+1) e^{- \frac{\mathcal{B} K(K+1)}{\kbt}}
\end{equation}
This distribution is linear at the origin and asintotically vanishing, hence it has a finite maximum.

\footnotetext{To be precise, $ I = \mu R_\text{eq}^2 $ is the moment of inertia of a body composed of two point-like masses $ M_1 $ and $ M_2 $, with reduced mass $ \mu $, which rotate around their center of mass at a distance $ R_\text{eq} $ from each other.}

\begin{lemma}[before upper = {\tcbtitle}]{Maximum of Boltzmann distribution}{}
  \begin{equation}
    K_\text{m}(T) = \frac{1}{2} \left[ \sqrt{\frac{2 \kbt}{\mathcal{B}}} - 1 \right]
  \end{equation}
\end{lemma}

\begin{proofbox}
  \begin{proof}
    To find the maximum of $ n_K(T) $ as a function of temperature, impose:
    \begin{equation*}
      \frac{\dd}{\dd K} n_K(T) = 0
      \quad \implies \quad
      \left[ 2 - \frac{\mathcal{B}}{\kbt} (2K+1)^2 \right] e^{- \frac{\mathcal{B} K(K+1)}{\kbt}} = 0
    \end{equation*}
    which is trivially solved.
  \end{proof}
\end{proofbox}

\subsection{Polyatomic molecules}

The purely-rotational spectrum of a polyatomic molecule can be treated analogously: in this case, however, the rotation no longer follows $ \ve{L} = I \bs{\omega} $, where $ I $ is the moment of inertial (a scalar), but is in general $ \ve{L} = \mt{I} \bs{\omega} $, where $ \mt{I} $ is the inertia tensor. The inertia tensor can be diagonalized with respect to the principal axis of rotations, which we identify WLOG with the $ x,y,z $-axis, so that the rotational energy is (promoting to operators):
\begin{equation}
  \ham_\text{rot} = \frac{L_x^2}{2I_x} + \frac{L_y^2}{2I_y} + \frac{L_z^2}{2I_z}
\end{equation}
The rotational Hamiltonian is in general non-diagonalizable, since the three components of $ \ve{L} $ do not commute, but we can analyze particular cases of interest.

\paragraph{Linear molecule}

Identifying the $ z $-axis with the axis of the molecule, then $ I_z = 0 $ and $ I_x = I_y \equiv I $, and consequently $ L_Z = 0 $, hence:
\begin{equation*}
  \ham_\text{rot} = \frac{L_x^2 + L_y^2}{2I} = \frac{\ve{L}^2}{2I}
\end{equation*}
which is precisely the result previously found.

\paragraph{Cylindrical molecule}

In presence of a cylindrical symmetry, identifying the $ z $-axis with the axis of the molecule, then $ I_x = I_y \equiv I_\perp $, but now $ I_z \neq 0 $, hence:
\begin{equation*}
  \ham_\text{rot} = \frac{L_x^2 + L_y^2}{2I_\perp} + \frac{L_z^2}{2I_z} = \frac{\ve{L}^2 - L_z^2}{2I_\perp} + \frac{L_z^2}{2I_z}
\end{equation*}
whose expectation value is:
\begin{equation*}
  E_\text{rot} = \frac{\hbar^2 K(K+1)}{2I_\perp} + \frac{\hbar^2 M^2}{2} \left( \frac{1}{I_z} - \frac{1}{I_\perp} \right)
\end{equation*}
The sign of the second term allows us to classify the various molecules:
\begin{itemize}
  \item if $ I_\perp < I_z $, the molecule is oblate (e.g. benzene $ \ch{C_6 H_6} $);
  \item if $ I_z < I_\perp $, the molecule is prolate (e.g. ammonia $ \ch{NH_3} $).
\end{itemize}
The purely-rotational spectrum of cylindrical molecule is more complex than that of linear molecules, since the $ M $-degeneracy is broken by the second term and the $ K^\text{th} $ level is split into $ M + 1 $ levels. Note that the electric-dipole selection rules in this case are $ \Delta K = \pm 1 $ and $ \Delta M = 0 $.

\paragraph{Spherical molecule}

The spherically-symmetric case is trivial, since $ I_x = I_y = I_z \equiv I $:
\begin{equation*}
  \ham_\text{rot} = \frac{L_x^2 + L_y^2 + L_z^2}{2I} = \frac{\ve{L}^2}{2I}
\end{equation*}
which is analogous to the linear case. However, a spherically-symmetric molecule has no electric dipoles (e.g. methane $ \ch{CH_4} $), hence it is harder to be studied via EM interactions.

\subsubsection{Electronic state}

Although most biatomic molecules have ground state with $ \Lambda = 0 $ (electronic angular momentum projected on the axis of the molecule), i.e. are $ \Sigma $ states, there are some molecules with ground state $ \Pi $, i.e. $ \Lambda = 1 $, like $ \ch{NO} $ and $ \ch{OH} $. In this case, then:
\begin{equation*}
  \ham_\text{rot} = \frac{L_x^2 + L_y^2}{2I} + \frac{L_z^2}{2I_e}
  \quad \implies \quad
  E_\text{rot} = \frac{\hbar^2 K(K+1)}{2I} + \frac{\hbar^2 \Lambda^2}{2} \left( \frac{1}{I_e} - \frac{1}{I} \right)
\end{equation*}
where $ I_e \sim m_e $ is the electronic moment of inertia. Note that, since $ m_e \ll \mu $, it is $ I_e \ll I $, hence the second term is positive; however, it is usually not observed, since transitions which change the quantum numbers $ \Lambda $ require much more energy ($ \sim 1 \ev $) than rotational transitions ($ \sim 1 \,\text{meV} $), and doing so would change the electronic state of the molecule and, as per \eref{eq:adiabatic-potential-def}, the geometry of the molecule.

\subsection{Molecular vibrations}

We now solve the radial Schrödinger equation \eref{eq:2-a-radial} for biatomic molecules. First, explicitly write the Laplacian in spherical coordinates (recalling that $ \rf_\nu^{(g)}(R) $ only depends on the radial coordinate):
\begin{equation}
  \left[ - \frac{\hbar^2}{2\mu} \frac{1}{R^2} \frac{\dd}{\dd R} \left( R^2 \frac{\dd}{\dd R} \right) + V_\text{eff}^{(g)}(R,K) \right] \rf_\nu^{(g)}(R) = E_{g,\nu} \rf_\nu^{(g)}(R)
\end{equation}
where we defined an effective potential to include the effect of the orbital angular momentum on the adiabatic potential. Now, focus on the first term:
\begin{equation*}
  \frac{1}{R^2} \frac{\dd}{\dd R} \left( R^2 \frac{\dd}{\dd R} \right) \rf(R) = \frac{1}{R} \left[ R \frac{\dd^2 \rf(R)}{\dd R^2} + 2 \frac{\dd \rf(R)}{\dd R} \right]
\end{equation*}
To simplify this expression, introduce an auxiliary function:
\begin{equation}
  u_\nu^{(g)}(R) \equiv R \rf_\nu^{(g)}(R)
\end{equation}
Then, trivially:
\begin{equation*}
  \frac{1}{R^2} \frac{\dd}{\dd R} \left( R^2 \frac{\dd}{\dd R} \right) \rf_\nu^{(g)}(R) = \frac{1}{R} \frac{\dd^2}{\dd R^2} u_\nu^{(g)}(R)
\end{equation*}
so that the radial equation can be recast in the simpler form:
\begin{equation}
  \left[ - \frac{\hbar^2}{2\mu} \frac{\dd^2}{\dd R^2} + V_\text{eff}^{(g)}(R,K) \right] u_\nu^{(g)}(R) = E_{g,\nu} u_\nu^{(g)}(R)
\end{equation}
To solve this equation, we employ a \bctxt{harmonic approximation}, assuming $ R \approx R_\text{eq} $:
\begin{equation*}
  \vad^{(g)}(R) \simeq \vad^{(g)} \left( R_\text{eq} \right) + \frac{1}{2} \frac{\dd^2 \vad^{(g)}(R)}{\dd R^2} \bigg\vert_{R = R_\text{eq}} \hspace{-0.2em} (R - R_\text{eq})^2
  \qquad \qquad
  \frac{\hbar^2 K(K+1)}{2\mu R^2} \simeq \frac{\hbar^2 K(K+1)}{2\mu R_\text{eq}^2}
\end{equation*}
The adiabatic potential then reduces to a harmonic potential with elastic constant:
\begin{equation}
  \kappa \equiv \frac{\dd^2 \vad^{(g)}(R)}{\dd R^2}\bigg\vert_{R = R_\text{eq}}
\end{equation}
and the system to a simple harmonic oscillator:
\begin{equation}
  \left[ - \frac{\hbar^2}{2\mu} \frac{\dd^2}{\dd R^2} + \frac{1}{2} \kappa \left( R - R_\text{eq} \right)^2 \right] u_\nu^{(g)}(R) = E_\text{vib}(n) u_\nu^{(g)}(R)
  \label{eq:vibrational-equation}
\end{equation}
where:
\begin{equation}
  E_\text{vib}(n) \equiv E_{g,\nu} - E_\text{rot}(K) - \vad^{(g)}(R_\text{eq})
\end{equation}
clearly is the vibrational energy of the molecule, since it is the difference between the total energy with respect to the equilibrium energy $ E_{g,\nu} - \vad^{(g)}(R_\text{eq}) $ and the rotational energy $ E_\text{rot}(K) $. The degrees of freedom (and the quantum numbers) of the system are therefore separated into electronic ($ g $), rotational ($ K , M $) and vibrational ($ n $) degrees of freedom.
The eigenfunctions of \eref{eq:vibrational-equation} are the Hermite functions, and its spectrum is:
\begin{equation}
  E_\text{vib}(n) = \left( n + \frac{1}{2} \right) \hbar \omega
  \qquad \qquad
  \omega \equiv \sqrt{\frac{\kappa}{\mu}}
\end{equation}
The molecules with the biggest vibrational energies are those with the lowest reduced mass, i.e. $ \ch{H_2} $ (with $ \hbar \omega \simeq 0.546 \ev $) and $ \ch{HCl} $ (with $ \hbar \omega \simeq 0.36 \ev $): clearly, then, $ E_\text{rot} \ll E_\text{vib} \ll E_e^{(g)} $, which justifies the separation of the degrees of freedom.

\subsubsection{Vibrational transitions}

As noted previously, homo-nuclear molecules are IR-inactive\footnotemark, since they have no electric dipoles, hence their interactions are mediated by higher-order terms in the multipole expansion of the EM field.

\footnotetext{Recall that IR transitions have energies $ \sim 0.1 \ev $.}

We then consider etero-nuclear molecules and analyze their electric-dipole transitions: it can be shown that, for vibrational transitions, the selection rule $ \Delta n = \pm 1 $ holds, i.e. $ \Delta E_\text{vib} = \hbar \omega $. Note that at standard temperature, since $ \kbt \simeq 26 \,\text{meV} $, contrary to rotational levels, only the ground state of the vibrational spectrum is populated.

\paragraph{Anarmonic correction}

It is possible to treat the cubic term in the expansion of $ \vad^{(g)}(R) $ at $ R = R_\text{eq} $ using perturbation theory, and the corrected energies are found to be:
\begin{equation}
  E_\text{vib}(n) = \left( n + \frac{1}{2} \right) \hbar \omega - a \left( n + \frac{1}{2} \right)^2 \hbar \omega
  \label{eq:vibrational-energy-anarmonic}
\end{equation}
with $ a \in \R^+ : a \ll 1 $, e.g. $ a \simeq 0.018 $ for $ \ch{HCl} $. Clearly, the correction becomes commensurable to the unperturbed energy for sufficiently large $ n \sim a^{-1} $, and approaching this asymptotic value the energy levels become increasingly less spaced: when these levels approximately overlap, no elastic force is present in the molecule, i.e. the molecule has dissociated into its two constituent atoms. This phenomenon can estimate\footnotemark the dissociation energy of the molecule.

\footnotetext{This is only an estimate as, since the correction term is comparable to unperturbed energy, the perturbative analysis is no longer valid, and one would have to include higher-order corrections.}

\begin{lemma}[before upper = {\tcbtitle}]{Dissociation energy}{}
  \begin{equation}
    E_\text{diss} \simeq \frac{\hbar \omega}{4a}
  \end{equation}
\end{lemma}

\begin{proofbox}
  \begin{proof}
    To find $ n_\text{diss} : E_\text{vib}(n_\text{diss} + 1) = E_\text{vib}(n_\text{diss}) $, impose:
    \begin{equation*}
      \frac{\dd E_\text{vib}(n)}{\dd n} = 0
      \quad \implies \quad
      \hbar \omega - 2 a \left( n + \frac{1}{2} \right) \hbar \omega = 0
      \quad \implies \quad
      n = \frac{1}{2a} - \frac{1}{2}
    \end{equation*}
    Then $ E_\text{diss} \simeq E_\text{vib}(n_\text{diss}) $.
  \end{proof}
\end{proofbox}

This result also allows to estimate the anarmonic coefficient from the dissociation energy. Moreover, it is possible to phenomenologically express the adiabatic potential in terms of the dissociation energy:
\begin{equation}
  \vad^{(g)}(R) = E_\text{diss} \left[ 1 - e^{-\beta \left( R - R_\text{eq} \right)} \right]^2 - E_\text{diss}
\end{equation}
which is known as \bctxt{Morse potential}. Note that this expression fails for $ R \rightarrow 0 $, since it does not reproduce the singularity of the adiabatic potential, but this region has little phenomenological interest. An interesting property of the Morse potential is that it allows for an analytic solution of the radial Schrödinger equation, with exact eigenenergies \eref{eq:vibrational-energy-anarmonic}.

The addition of an anarmonic correction also changes the selection rule for electric-dipole transitions: in this case $ \Delta n \in \Z - \{0\} $, but the peaks in the spectrum are progressively suppressed as $ \abs{\Delta n} $ becomes larger than $ 1 $.

\paragraph{Roto-vibrational spectrum}

\begin{figure}
  \centering
  \begin{subfigure}{0.70 \textwidth}
    \centering
    \includegraphics[width = 0.95 \textwidth]{rot-vib-spectr.png}
  \end{subfigure}%
  \begin{subfigure}{0.30 \textwidth}
    \centering
    \includegraphics[width = 0.95 \textwidth]{rot-vib-spectr-det.png}
  \end{subfigure}
  \caption{Observed roto-vibrational absorption spectrum of gas-phase $ \ch{HCl} $ and scheme of electric-dipole transitions between $ n = 0 $ and $ n = 1 $ rotational levels.}
  \label{fig:roto-vib}
\end{figure}

In the harmonic approximation, the absorption of IR radiation can only excite molecules from the $ n = 0 $ state to the $ n = 1 $ state. However, since $ E_\text{rot} \ll E_\text{vib} $, this absorption also induces multiple rotational transitions, hence both these state are split into multiple rotational levels with various values of $ K $, as shown in \figref{fig:roto-vib}. The roto-vibrational spectrum then is:
\begin{equation}
  \Delta E_\text{rot-vib}(K) = \hbar \omega \pm 2\mathcal{B} \left( K + 1 \right)
  \label{eq:roto-vibrational-spectrum}
\end{equation}
Note that the selection rule $ \Delta K = \pm 1 $ is valid if the electronic state is $ \Lambda = 0 $, i.e. for $ \Sigma $ states. If instead $ \Lambda \neq 0 $, then $ \Delta K = 0 , \pm 1 $, hence a central peak at $ \hbar \omega $ between the P and R branches is present, denoted as Q peak.

Including the anarmonic terms in the vibrational and rotational potentials, one finds the following corrections to the roto-vibrational energy:
\begin{equation}
  E_\text{rot-vib}^{(1)} = - a_1 \hbar \omega \left( n + \frac{1}{2} \right)^2 - a_2 \left( n + \frac{1}{2} \right) K (K+1) - a_3 K^2 (K+1)^2
\end{equation}
where $ a_1 $ is the previously discussed anarmonic coefficients and $ a_2 , a_3 \in \R^+ : a_2 , a_3 \ll \mathcal{B} $. In particular, the third term is called centrifugal distortion term: it reduces the energy of the molecule since, in classical analogy, it represents a distortion of the spring due to the rotation, increasing the moment of inertia. On the other hand, the second term intuitively represents the fact that, at an excited state $ n > 0 $, the equilibrium distance is increased, hence so is the moment of intertia. From a perturbative analysis, the centrifugal coefficients are:
\begin{equation}
  a_2 = \frac{6\mathcal{B}^2}{\hbar \omega}
  \qquad \qquad
  a_3 = \frac{4\mathcal{B}^3}{(\hbar \omega)^2}
\end{equation}

\paragraph{Electronic transitions}

\begin{figure}
  \centering
  \includegraphics[width = 0.50 \textwidth]{electr-exc.png}
  \caption{Electronic transition between two adiabatic potential surfaces.}
  \label{fig:elec-trans}
\end{figure}

Transitions between two electronic states, in the Born--Oppenheimer approximation, happen instantly as the radiation is absorbed by the molecule, hence it induces an instantaneous vertical shift between two adiabatic potential surfaces as in \figref{fig:elec-trans}: this is the \bctxt{Frank--Condon principle}. To study this transition, consider the general matrix element between two molecular states $ \ket{g,\nu} $ and $ \ket{g',\nu'} $:
\begin{equation*}
  \braket{g',\nu' | \bs{\mu_e} | g,\nu}
  = \int_{\R^{3(N_n + N_e)}} \dd^{3N_n}R \, \dd^{3N_e}r \, {\Phi_e^{(g')}}^*(R_\text{fin},r) {\Phi_n^{(\nu')}}^*(R) \left[ \bs{\mu}_e^{(e)} + \bs{\mu}_e^{(n)} \right] \Phi_e^{(g)}(R_\text{in},r) \Phi_n^{(\nu)}(R)
\end{equation*}
where we separated the electronic and nuclear electric dipoles. Their expressions are:
\begin{equation}
  \bs{\mu}_e^{(e)} = - e \sum_{i = 1}^{N_e} \ve{r}_i
  \qquad \qquad
  \bs{\mu}_e^{(n)} = + e \sum_{j = 1}^{N_n} \ve{R}_j
\end{equation}
By the Frank--Condon principle, $ R_\text{in} = R_\text{fin} $ at the instant of the transition, hence the two contributions can be written as:
\begin{equation*}
  \int_{\R^{3(N_n + N_e)}} \dd^{3N_n}R \, \dd^{3N_e}r \, {\Phi_e^{(g')}}^*(R_\text{in},r) {\Phi_n^{(\nu')}}^*(R) \bs{\mu}_e^{(n)} \Phi_e^{(g)}(R_\text{in},r) \Phi_n^{(\nu)}(R) = \braket{g' | g} \braket{\nu' | \bs{\mu}_e^{(n)} | \nu}
\end{equation*}
\begin{equation*}
  \int_{\R^{3(N_n + N_e)}} \dd^{3N_n}R \, \dd^{3N_e}r \, {\Phi_e^{(g')}}^*(R_\text{in},r) {\Phi_n^{(\nu')}}^*(R) \bs{\mu}_e^{(e)} \Phi_e^{(g)}(R_\text{in},r) \Phi_n^{(\nu)}(R) = \braket{\nu' | \nu} \braket{g' | \bs{\mu}_e^{(e)} | g}
\end{equation*}
Since the electronic eigenfunctions are orthonormal, $ \braket{g' | g} = \delta_{g',g} $, but in an electronic transition $ g' \neq g $, therefore the nuclear contribution vanishes. On the other hand, the electronic contribution has two terms:
\begin{itemize}
  \item $ \braket{g' | \bs{\mu}_e^{(e)} | g} $ is the matrix element for electric-dipole transitions between electronic states, which results in the usual selection rules $ \Delta \Lambda = 0 , \pm 1 $ and $ \Delta S = 0 $;
  \item $ \braket{\nu' | \nu} \neq 0 $ since the equilibrium point changes between the two adiabatic potential surfaces, hence the eigenfunctions associated to $ \ket{\nu'} $ are a different set of eigenfunctions when compared to those of $ \ket{\nu} $, as both $ I $ and $ \omega $ change with $ R_\text{eq} $. As a consequence, this term determines no selection rules, but simply determines the relative intensity of particular nuclear transitions (e.g. due to the form of Hermite functions, if $ R_\text{eq} $ shifts to the right, then $ \nu = 0 \ra \nu' > 0 $ is more intense than $ \nu = 0 \ra \nu' = 0 $).
\end{itemize}
This analysis shows that electronic transitions do not require etero-nuclear molecules, since nuclear dipoles play no role in them: these so-called \bctxt{vibronic transitions} can thus be used to study the vibrational spectrum of homo-nuclear molecules\footnotemark.

\footnotetext{Typically, since vibronic transitions require energies $ \sim 1\ev $, the detector is not capable to resolve rotational transitions, hence the roto-vibrational spectrum requires finer methods to be studied in homo-nuclear molecules.}










