\selectlanguage{english}

\begin{example}{Spectroscopic terms from level separation}{}
  The  separation between consecutive pairs of 3 atomic levels with spin-orbit coupling is in the ratio $ 3:5 $ with increasing energy. Determine the spectroscopic terms.

  \textit{Solution.} Since the levels are ordered in increasing energy, $ \xi > 0 $. Denoting the two energy intervals as $ \Delta E_1 $ and $ \Delta E_2 $, then $ \Delta E_2 / \Delta E_1 = 5 / 3 $ and, by Landé's spacing law:
  \begin{equation*}
    \frac{\xi \left( J_\text{min} + 2 \right)}{\xi \left( J_\text{min} + 1 \right)} = \frac{5}{3}
    \quad \implies \quad
    J_\text{min} = \frac{1}{2}
    \quad \implies \quad
    \begin{cases}
      \abs{L - S} = \frac{1}{2} \\
      L + S = \frac{5}{2}
    \end{cases}
  \end{equation*}
  where we set to zero the energy of the lowest state. The solutions are:
  \begin{equation*}
    (2L , 2S) \in \{ \left( 2 , 3 \right) , \left( 3 , 2 \right) \}
  \end{equation*}
  Since $ L \in \N_0 $, the only acceptable solution is $ L = 1 , S = \frac{3}{2} $, hence the spectroscopic terms are $ \ch{^4P_{\sfrac{1}{2} , \sfrac{3}{2} , \sfrac{5}{2}}} $.
\end{example}

\subsection{Atoms in a magnetic field}

When applying a magnetic field $ \ve{B} \neq \ve{0} $, the conjugate momentum shifts as $ \ve{P} \mapsto \ve{P} - q \ve{A} $, where $ q $ is the charge of the particle considered. Fixing the Coulomb gauge, the vector potential is uniquely defined by $ \rot \ve{A} = \ve{B} $ and $ \dive \ve{A} = 0 $: for a constant magnetic field:
\begin{equation}
  \ve{A}(\ve{r}) = \frac{1}{2} \ve{B} \times \ve{r}
\end{equation}

\begin{proposition}[before upper = {\tcbtitle}]{Mono-electron Hamiltonian in constant magnetic field}{}
  \begin{equation}
    \ham \simeq \ham_0 - \bs{\mu} \cdot \ve{B}
  \end{equation}
\end{proposition}

\begin{proofbox}
  \begin{proof}
    For a mono-electron atom, the Hamiltonian becomes:
    \begin{equation*}
      \ham = \frac{1}{2m} \left( \ve{P} + e \ve{A} \right)^2 - \frac{Z e^2}{r} = \frac{1}{2m} \left( -\hbar^2 \lap - 2i e \hbar \ve{A} \cdot \grad - i \hbar e \dive \ve{A} + e^2 \ve{A}^2 \right) - \frac{Z e^2}{r}
    \end{equation*}
    The third term in parentheses vanishes in the Coulomb gauge, while the last term is $ \sim \smo(B^2) $, hence we approximate it as negligible\footnotemark. Denoting the unperturbed Hamiltonian as $ \ham_0 $, then:
    \begin{equation*}
      \ham \simeq \ham_0 - \frac{i e \hbar}{2m} \left( \ve{B} \times \ve{r} \right) \cdot \grad = \ham_0 - \frac{i e \hbar}{2m} \left( \ve{r} \times \grad \right) \cdot \ve{B} = \ham_0 + \frac{e}{2m} \ve{L} \cdot \ve{B}
    \end{equation*}
    where we used $ (\ve{A} \times \ve{B}) \cdot \ve{C} = (\ve{B} \times \ve{C}) \cdot \ve{A} $. Recalling \eref{eq:mu-orbital-definition} completes the proof.
  \end{proof}
\end{proofbox}

\footnotetext{Indeed, a field $ B \sim 1 \,\text{T} $ is associated to an energy $ \bor B \sim 60 \,\mu\ev $, which can be treated as a perturbation.}

This expression holds in presence of spin too, summing the two magnetic moments:
\begin{equation}
  \bs{\mu} = \bs{\mu}_L + \bs{\mu}_S = - \frac{\bor}{\hbar} \left( \ve{L} + 2 \ve{S} \right)
\end{equation}
The total Hamiltonian is then found adding the spin-orbit coupling:
\begin{equation}
  \ham = \ham_0 + \frac{\bor}{\hbar} \left( \ve{L} + 2 \ve{S} \right) \cdot \ve{B} + \xi(\ve{r}) \ve{L} \cdot \ve{S}
\end{equation}
The two perturbation terms are not simultaneously-diagonalizable: the first term is diagonalized in the uncoupled basis, with quantum numbers $ \left( n  , \ell , s , m_\ell , m_s \right) $, while the second term is diagonalized in the coupled basis, with quantum numbers $ \left( n , \ell , s , j , m_j \right) $. In general, to diagonalized this Hamiltonian one would have to employ degenerate perturbation theory, since the presence of these perturbation terms breaks the degeneracy of the unperturbed Hamiltonian; however, limit cases are more interesting.

\paragraph{Atoms with $ \ve{S} \bs{= 0} $}

In this case there is no spin-orbit interaction, and the perturbation Hamiltonian is reduced to (assuming $ \ve{B} = B \ve{e}_z $):
\begin{equation*}
  \ham' = \frac{\bor}{\hbar} B L_z
  \quad \implies \quad
  \braket{\ham'} = \bor B m_L
\end{equation*}
The $ m_L $-degeneracy is thus broken, and each state is split into $ 2L + 1 $ equally-spaced states.

\begin{example}{Cadmium}{}
  Consider $ \ch{_{48}Cd} = [\ch{Kr}] 4\text{d}^{10} 5\text{s}^2 $: in this case $ L = S = J = 0 $, hence the ground state is $ \ch{^1S_0} $. Now, take the two excited states with valence electrons $ 5\text{s}^1 5\text{p}^1 $ and $ 5\text{s}^1 5\text{d}^1 $: given the selection rule (electric-dipole approximation) $ \Delta S = 0 $, these states have $ S = 0 $ like the ground state, hence they are $ \ch{^1P_1} $ and $ \ch{^1D_2} $. Applying a magnetic field, then, the ground state is unperturbed, the first excited state is split into three states and the second excited state is split into five states.

  Recall that, from the electric-dipole approximation, we can derive the selection rule $ \Delta m_L = 0 , \pm 1 $, where $ \Delta m_L = 0 $ is associated to a linear polarization of the EM wave along the $ z $-axis, while $ \Delta m_L = \pm 1 $ are circular polarizations in the $ xy $-plane. Historically, $ \Delta m_L = 0 $ transitions are called $ \pi $ transitions and $ \Delta m_L = \pm 1 $ ones $ \sigma $ transitions.

  Then, $ \ch{^1P_1} \ra \ch{^1S_0} $ is split into one $ \pi $ transition and two $ \sigma $ transitions, while $ \ch{^1D_2} \ra \ch{^1P_1} $ is split into three $ \pi $ transitions and six $ \sigma $ transitions: $ \pi $ transitions have unperturbed energies, while $ \sigma $ transitions have energies perturbed by $ \Delta E = \pm \bor B $.
\end{example}

This example shows that we cannot distinguish $ S = 0 $ atoms from one another by only applying a magnetic field, since their spectra are all equally perturbed: in particular, each line is split into three lines (corresponding to $ \Delta m_L = 0 , \pm 1 $), an effect historically known as \bctxt{normal Zeeman effect} (as it can be derived classically).

\paragraph{Atoms with $ \bs{S} \bs{\neq 0} $}

We can distinguish between two cases: the magnetic field dominates or the spin-orbit coupling dominates, which are respectively known as \bctxt{Paschen--Back effect} and \bctxt{anomalous Zeeman effect} (as it requires a quantum description, contrary to the normal Zeeman effect).

\subsubsection{Paschen--Back effect}
\label{ssec:paschen-back}

In this case, we approximate the perturbation Hamiltonian at first order as:
\begin{equation*}
  \ham' \simeq \frac{\bor}{\hbar} B \left( L_z + 2S_z \right)
  \quad \implies \quad
  \braket{\ham'} = \bor B \left( m_L + 2 m_S \right)
\end{equation*}
To include the spin-orbit coupling, we have to rewrite $ \ve{L} \cdot \ve{S} $ in a way best suited for the uncoupled basis, which has been fixed by the first-order approximation:
\begin{equation}
  \ve{L} \cdot \ve{S} = \frac{1}{2} \left( L_+ S_- + L_- S_+ \right) + L_z S_z
  \label{eq:spin-orbit-uncoupled}
\end{equation}
where we used \eref{eq:angular-momentum-plus-minus}. Clearly, then, on the uncoupled basis $ \braket{\ve{L} \cdot \ve{S}} = \hbar^2 m_L m_S $, so we can write the total pertubation energy as:
\begin{equation}
  \Delta E_\text{PB} = \bor B \left( m_L + 2 m_S \right) + \xi m_L m_S
  \label{eq:paschen-back}
\end{equation}
This expression suggests how to establish whether the magnetic field or the spin-orbit coupling dominates: the Paschen--Back effect takes place if $ \bor B \gg \xi $.

\begin{observation}{Classical treatment of Larmor precession}{larmor-precession}
  From a classical point of view, $ \ve{L} $ and $ \ve{S} $ are vectors which, in the presence of a magnetic field $ \ve{B} $, precede along the direction of $ \ve{B} $. Indeed, ignoring at first order the spin-orbit coupling, $ \ve{L} $ and $ \ve{S} $ are associated to independent magnetic moments $ \bs{\mu}_L $ and $ \bs{\mu}_S $, but in a magnetic field a magnetic moment $ \bs{\mu} $ precedes along the direction of $ \ve{B} $ according to the torque $ \bs{\tau} = \bs{\mu} \times \ve{B} $:
  \begin{equation*}
    \bs{\tau} = \frac{\dd \bs{\ell}}{\dd t} = \frac{\dd \bs{\ell}}{\dd \varphi} \frac{\dd \varphi}{\dd t}
    \quad \implies \quad
    \tau = \omega \frac{\dd \ell}{\dd \varphi}
  \end{equation*}
  Geometrically $ \dd \ell = \ell \sin \vartheta \dd \varphi $, where $ \vartheta $ is the angle between $ \bs{\ell} $ and $ \ve{B} $, hence:
  \begin{equation*}
    \mu B \sin \vartheta = \omega \ell \sin \vartheta
    \quad \implies \quad
    \omega = \frac{\mu B}{\ell}
  \end{equation*}
  Recalling \eref{eq:mu-orbital-definition} we find the angular velocity of the precession of $ \bs{\mu} $ (i.e. of $ \bs{\ell} $):
  \begin{equation}
    \omega = g_\ell \frac{\bor}{\hbar} B
  \end{equation}
  where we included the Landé factor for a general angular momentum. This phenomenon is known as \bcobs{Larmor precession}.

  Since $ g_L = 1 $ and $ g_S = 2 $, the orbital angular momentum and the spin precede at different velocities, hence their sum $ \ve{J} = \ve{L} + \ve{S} $ is no longer an integral of motion, which quantistically corresponds to $ j $ not being a good quantum number. However, note that Larmor precession leaves the components of $ \ve{L} $ and $ \ve{S} $ along $ \ve{B} $ unchanged: this is reflected in \eref{eq:spin-orbit-uncoupled}, since $ \braket{\ve{L} \cdot \ve{S}} = \hbar^2 m_L m_S $ which are still good quantum numbers. A classical way of putting this is that, over an extended time interval, $ \ve{L} $ and $ \ve{S} $ can be replaced with ``effective" angular momenta given by their mean values $ \ve{L}_\text{eff} \equiv L_z \ve{e}_z $ and $ \ve{S}_\text{eff} \equiv S_z \ve{e}_z $.
\end{observation}

\subsubsection{Anomalous Zeeman effect}
\label{ssec:anomalous-zeeman}

We now consider the case $ \xi \gg \bor B $ in which the spin-orbit interaction dominates, so we work in the coupled basis.

\begin{figure}
  \centering
  \includegraphics[width = 0.50 \textwidth]{anomalous-zeeman-effect.png}
  \caption{Classical view of the anomalous Zeeman effect}
  \label{fig:anomalous-zeeman-effect}
\end{figure}

As per \obsref{obs:larmor-precession}, since $ g_L = 1 $ and $ g_S = 2 $, the total angular momentum $ \ve{J} = \ve{L} + \ve{S} $ and its magnetic moment $ \bs{\mu}_J = \bs{\mu}_L + \bs{\mu}_S $ are not aligned (see \figref{fig:anomalous-zeeman-effect}). In absence of an external magnetic field, $ \ve{J} $ is a constant vector and $ \ve{L} $ and $ \ve{S} $ precede along its direction, since they are affected by each other's magnetic moment (i.e. generated magnetic field), and so do their magnetic moments and the total magnetic moment $ \bs{\mu}_J $. Then, adding an external magnetic field, $ \ve{J} $ precedes along its direction: since we are assuming $ \xi \gg \bor B $, the precession of $ \ve{L} $ , $ \ve{S} $ and $ \bs{\mu}_J $ along $ \ve{J} $ is much faster than the precession of $ \ve{J} $ along $ \ve{B} $, hence we can assume them to be independent, as represented in \figref{fig:anomalous-zeeman-effect}.

Given the greater angular frequency of the precession of $ \bs{\mu}_J $ along $ \ve{J} $, the only component of $ \bs{\mu}_J $ that interacts with $ \ve{B} $ is the mean component during the precession (denoted in \figref{fig:anomalous-zeeman-effect} as $ (\bs{\mu}_J)_J $), i.e. the projection of $ \bs{\mu}_J $ along $ \ve{J} $.

\begin{lemma}[before upper = {\tcbtitle}]{Energy perturbation for the anomalous Zeeman effect}{}
  \begin{equation}
    \braket{\bs{\mu}_J \cdot \ve{B}} = - \frac{\bor B}{\hbar} \frac{3 J(J+1) + S(S+1) - L(L+1)}{2 J(J+1)} m_J
  \end{equation}
\end{lemma}

\begin{proofbox}
  \begin{proof}
    Given the projection scheme outlined above and in \figref{fig:anomalous-zeeman-effect}, first project $ \bs{\mu}_J $ along $ \ve{J} $:
    \begin{equation*}
      (\bs{\mu}_J)_J \equiv \frac{\bs{\mu}_J \cdot \ve{J}}{\norm{\ve{J}}} \frac{\ve{J}}{\norm{\ve{J}}} = \frac{\left( \bs{\mu}_L + \bs{\mu}_S \right) \cdot \ve{J}}{J^2} \ve{J} = - \frac{\bor}{\hbar J^2} \left( \ve{L} + 2 \ve{S} \right) \cdot \left( \ve{L} + \ve{S} \right) \ve{J}
    \end{equation*}
    Then, project $ (\bs{\mu}_J)_J $ along $ \ve{B} = B \ve{e}_z $:
    \begin{equation*}
      (\bs{\mu}_J)_J \cdot \ve{B} = - \frac{\bor B}{\hbar J^2} \left( \ve{L} + 2 \ve{S} \right) \cdot \left( \ve{L} + \ve{S} \right) J_z
    \end{equation*}
    To solve the scalar product, recall that $ 2 \ve{L} \cdot \ve{S} = J^2 - L^2 - S^2 $:
    \begin{equation*}
      \left( \ve{L} + 2 \ve{S} \right) \cdot \left( \ve{L} + \ve{S} \right) = L^2 + 2 S^2 + 3 \ve{L} \cdot \ve{S} = \frac{3 J^2 + S^2 - L^2}{2}
    \end{equation*}
    Putting everything together:
    \begin{equation*}
      (\bs{\mu}_J)_J \cdot \ve{B} = - \frac{\bor B}{\hbar} \frac{3J^2 + S^2 - L^2}{2J^2} J_z
    \end{equation*}
    The expectation value is easily found.
  \end{proof}
\end{proofbox}

We define the Landé factor for $ J $ as:
\begin{equation}
  g_J \equiv \frac{3 J(J+1) + S(S+1) - L(L+1)}{2 J(J+1)}
\end{equation}
so that $ \bs{\mu}_J = g_J \frac{\bor}{\hbar} \ve{J} $ (truly this is $ (\bs{\mu}_J)_J $, but we identify the magnetic moment with its mean over many precession periods) the energy correction due to the anomalous Zeeman effect is:
\begin{equation}
  \Delta E_\text{AZ} = \frac{\xi}{2} \left[ J(J+1) - L(L+1) - S(S+1) \right] + g_J m_J \bor B
  \label{eq:anomalous-zeeman}
\end{equation}
Note that the $ g_J $ is not always between $ 1 = g_L $ and $ 2 = g_S $: for example, for $ (L , S , J) = (1 , \frac{3}{2} , \frac{1}{2}) $ it is $ g_J = \frac{8}{3} > 2 $, while for $ (L , S , J) = (3 , \frac{5}{2} , \frac{1}{2}) $ it is $ g_J = - \frac{2}{3} < 0 $.

\begin{example}{Paschen--Back effect in hydrogen}{}
  Determine the spectrum observed for transitions between $ n = 2 $ and $ n = 1 $ of hydrogen atoms in a $ B = 10 \,\text{T} $ magnetic field.

  \textit{Solution.} In this case $ \bor B = 0.58 \,\text{meV} $, while for $ \ch{^1H} $ the spin-orbit constant has the exact expression given by \eref{eq:spin-orbit-1-e}:
  \begin{equation*}
    \xi_{n,\ell} = \frac{\ryd Z^4 \alpha^2}{n^3 \ell \left( \ell + 1 \right) \left( \ell + \frac{1}{2} \right)}
  \end{equation*}
  The ground state $ \ch{^2S_{\sfrac{1}{2}}} $ has no spin-orbit interaction, while the possible $ n = 2 $ states are $ \ch{^2S_{\sfrac{1}{2}}} $ and $ \ch{^2P_{\sfrac{1}{2} , \sfrac{3}{2}}} $: in the latter case the spin-orbit constant is $ \xi_{2,1} = 0.03 \,\text{meV} $, hence the magnetic field induces a Paschen--Back effect.

  The transition $ \ch{^2S_{\sfrac{1}{2}}} \ra \ch{^2S_{\sfrac{1}{2}}} $ is dipole-forbidden, so consider $ \ch{^2P_{\sfrac{1}{2} , \sfrac{3}{2}}} \ra \ch{^2S_{\sfrac{1}{2}}} $: by \eref{eq:paschen-back}, $ \ch{^2P_{\sfrac{1}{2} , \sfrac{3}{2}}} $ splits into six levels, for $ m_L = -1,0,1 $ and $ m_S = -\frac{1}{2},\frac{1}{2} $, but $ m_L + 2m_S = 0 $ for $ (m_L , m_S) = (-1,\frac{1}{2}) , (1,-\frac{1}{2}) $, hence the non-degenerate split levels are five: on the other hand, $ \ch{^2S_{\sfrac{1}{2}}} $ splits into the two levels $ m_S = -\frac{1}{2} , \frac{1}{2} $. The relevant selection rule in this transition is $ \Delta m_S = 0 $: the possible transitions are then $ (m_L , \pm \frac{1}{2}) \ra (0 , \pm \frac{1}{2}) $, which, given the degeneracy above, produce five spectral lines.
\end{example}











