\selectlanguage{english}

\chapter{Lecture 29/09}

\begin{example}{Sodium transitions}{}
  Find the wavelength of the transitions $ 4\text{p} \ra 3\text{s} $ and $ 4\text{d} \ra 3\text{p} $ of sodium, knowing that $ \alpha_\text{s} = 1.35 $, $ \alpha_p = 0.86 $ and $ \alpha_d = 0.01 $.

  \textit{Solution.} Since $ \ch{Na} $ is an alkali atom, the energy gap between two levels can be written in terms of the respective quantum defects as:
  \begin{equation*}
    \Delta E = \ryd \left[ \frac{1}{(n_1 - \alpha_{\ell_1})^2} - \frac{1}{(n_2 - \alpha_{\ell_2})^2} \right]
  \end{equation*}
  Then, defining the \bcex{Rydberg wavelength} $ \lambda_\text{R} \equiv \frac{h c}{\ryd} \simeq 91.127 \,\text{nm} $, since $ \lambda = \frac{h c}{\Delta E} $ we get:
  \begin{equation*}
    \lambda_{4\text{p} \ra 3\text{s}} = 342.7 \,\text{nm}
    \qquad \qquad
    \lambda_{4\text{d} \ra 3\text{p}} = 585.9 \,\text{nm}
  \end{equation*}
  The first transition is UV, while the second transition is visible.
\end{example}

\begin{example}{Yellow doublet of sodium}{}
  The yellow doublet of sodium consists of two lines $ \lambda_1 = 589.6 \,\text{nm} $ and $ \lambda_2 = 589.0 \,\text{nm} $ associated to the transitions of the valence electron from the first excited state to the ground state. Compute the spin-orbit constant and the effective charge $ Z_\text{eff} $ that should be assumed in the spin-orbit relation for mono-electron atoms.

  \textit{Solution.} For heavier atoms, the kinetic correction and the Darwin term become smaller, and the relativistic corrections are determined by the spin-orbit term, which can be written as:
  \begin{equation*}
    \Delta E_\text{s-o} = \frac{\xi_{\ell , s}}{2} \left[ J \left( J + 1 \right) - L \left( L + 1 \right) - S \left( S + 1 \right) \right]
  \end{equation*}
  In particular, the ground state $ 3\text{S} $ remains unchanged, while the first excited state is split in $ 3\text{P}_{1/2} $ and $ 3\text{P}_{3/2} $: since $ \ch{Na} $ has a single valence electron, we expect $ \xi_{\ell , s} \ge 0 $ like for $ \ch{H} $, hence $ 3\text{P}_{1/2} $ is energetically closer to the ground state. Then, $ \lambda_1 $ is associated to $ 3\text{P}_{1/2} \ra 3\text{S}_{1/2} $ and $ \lambda_2 $ to $ 3\text{P}_{3/2} \ra 3\text{S}_{1/2} $, and:
  \begin{equation*}
    \frac{h c}{\lambda_2} - \frac{h c}{\lambda_1} = \frac{\xi}{2} \left[ \frac{3}{2} \left( \frac{3}{2} + 1 \right) - \frac{1}{2} \left( \frac{1}{2} + 1 \right) \right] = \frac{3}{2} \xi
    \quad \implies \quad
    \xi = \frac{2 h c}{3} \left( \frac{1}{\lambda_2} - \frac{1}{\lambda_1} \right) \approx 1.46 \,\text{meV}
  \end{equation*}
  The mono-electron atom relation is \eref{eq:spin-orbit-1-e}, hence:
  \begin{equation*}
    \xi_{\ell , s} = \frac{Z_\text{eff}^4 \ryd \alpha^2}{n^3 \ell \left( \ell + 1 \right) \left( \ell + \frac{1}{2} \right)}
    \quad \implies \quad
    Z_\text{eff} = \sqrt[4]{3^3 \cdot 1 \cdot 2 \cdot \frac{3}{2} \frac{\xi}{\ryd \alpha^2}} \approx 3.56
  \end{equation*}
\end{example}

\section{X-ray spectra}

For general multi-electron atoms, the energy eigenvalues for internal electrons can be empirically expressed by \bctxt{Mosley's law}:
\begin{equation}
  E_n \approx - \frac{\ryd \left( Z - \alpha_n \right)^2}{n^2}
\end{equation}
where $ \alpha_n $ are the screening constants determined by inner electrons (empirically $ \alpha_1 < \alpha_2 < \alpha_3 < \dots $).
Since chemical bonds only affect valence electrons, at the first order of approximation, the energies of internal electrons should remain unperturbed independently of the state of the sample.

Transition between core states, i.e. electronic configurations with different internal electrons, are described by this notation: a hole in a shell $ n = 1 , 2 , 3 , 4 , \dots $ is denoted by $ K , L , M , N , \dots $, while the difference $ \Delta n = 1 , 2 , 3 , \dots $ is denoted by $ \alpha , \beta , \gamma , \dots $. For example, the transition $ 1\text{s}^1 2\text{s}^2 2\text{p}^6 \dots \ra 1\text{s}^2 2 \text{s}^2 2 \text{p}^5 $ is denoted by $ K_\alpha \equiv L \ra K $, while $ 1\text{s}^2 2\text{s}^2 2\text{p}^5 3\text{s}^2 3\text{p}^6 4\text{s}^2 \ra 1\text{s}^2 2\text{s}^2 2\text{p}^6 3\text{s}^2 3\text{p}^6 4\text{s}^1 $ by $ L_\beta \equiv L \ra N $.

\section{Multi-electron atoms}

When accounting for multiple valence electrons, the potential is no longer spherically-symmetric. The corrections with respect to the unperturbed spherically-symmetric case are determined by the total angular momenta:
\begin{equation}
  \ve{S} \defeq \sum_i \bs{s}_i
  \qquad \qquad
  \ve{L} \defeq \sum_i \bs{\ell}_i
  \qquad \qquad
  \ve{J} = \ve{L} + \ve{S}
\end{equation}
This is the \emph{Russell-Saunders coupling}: another possible coupling is the \emph{jj coupling}, in which $ \ve{J} = \sum_i \bs{j}_i $, which is more effective for heavier atoms where the spin-orbit interaction dominates the potential. The total orbital $ \ve{L} $ gives rise to non-central correction terms, while the total spin $ \ve{S} $ imposes conditions on the possible states due to the Pauli exclusion principles, which are linked to the swap integral.

The spin-orbit interaction splits the atomic states into multiplets (since $ S \neq \frac{1}{2} $ in general). For a multiplet with fixed $ L $ and $ S $, the split levels are energetically separated by:
\begin{equation*}
  \Delta E(L , S) = \frac{\xi}{2} \left[ \left( J + 1 \right) \left( J + 2 \right) - J \left( J + 1 \right) \right] = \xi \left( J + 1 \right)
\end{equation*}
A measurement of the ordering of these split levels can then determine if $ \xi > 0 $ or $ \xi < 0 $, depending on weather the lowest state is that with lowest or highest total angular momentum.










