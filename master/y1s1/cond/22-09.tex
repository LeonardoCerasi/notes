\selectlanguage{english}

\chapter{Lecture 22/09}

\section{Mono-atomic atoms}

To describe a mono-electron atom, consider the Schrödinger equation for a single-particle quantum system:
\begin{equation}
  \ham \psi(\ve{r}) = E \psi(\ve{r})
\end{equation}
where the Hamiltonian can be written as:
\begin{equation}
  \ham = \frac{p^2}{2m} + V(\ve{r})
  \label{eq:1-particle-hamiltonian}
\end{equation}
A particularly interesting case is that of a spherically-symmetric potential $ V(\ve{r}) = V(r) $. This allows for a separation of variables of the kind $ \psi(\ve{r}) = R(r) Y(\vartheta , \varphi) $: the angular part is uniquely determined, as it is not affected by the potential, and is given by the \bctxt{spherical harmonics}:
\begin{equation}
  \psi_{n,\ell,m}(\ve{r}) = R_{n,\ell}(r) Y_{\ell,m}(\vartheta , \varphi)
  \label{eq:1-particle-wavefunction}
\end{equation}
A quantization of the wave-functions has appeared, with three quantum numbers $ n , \ell , m $: note that the energy eigenvalue $ E = E_{n,\ell} $ does not depend on $ m $. In the specific case of the mono-electron atom:
\begin{equation*}
  V(\ve{r}) = - \frac{Z e^2}{r}
\end{equation*}
and the radial solutions are given by the Laguerre polynomials, while the energy eigenvalues are the \bctxt{Bohr energies} (expressed in terms of the Rydberg energy):
\begin{equation}
  E_n = - \frac{Z^2 \ryd}{n^2}
  \qquad \qquad
  \ryd \equiv \frac{m e^4}{2 \hbar^2} \simeq 13.6 \ev
\end{equation}
These solutions contains several approximations: the mass of the electron is assumed negligible with respect to the mass of the nucleus (since $ m_e / m_p \sim 2000 $) and both the electron's spin and the relativistic effects of its motion (which have the same order of magnitude) have been ignored.

\subsection{Stern-Gerlach experiment}

The Stern-Gerlach experiment has proved the existence of the electron's spin. Consider the $ \ch{_{47}Ag} $ atom, which has the ground state configuration $ [\ch{Kr}] 4\text{d}^{10} 5\text{s}^1 $: full shells do not contribute to the orbital angular momentum, hence the ground state of $ \ch{_{47}Ag} $ has $ L = 0 $. As we can associate a magnetic moment\footnotemark to an electron orbiting around a nucleus, we can study the angular momentum of $ \ch{_{47}Ag} $ using a magnetic field: classically $ E = - \bs{\mu} \cdot \ve{B} $, but quantistically $ \bs{\mu} $ is an operator, so we have to consider its expectation value.

\footnotetext{By the correspondence principle of Quantum Mechanics, we can associate to the classical magnetic moment for an electron with orbital angular momentum $ \bs{\ell} $ a quantum analogous:
\begin{equation}
  \bs{\mu} = - \frac{e}{2m_e} \bs{\ell} \equiv - \frac{\bor}{\hbar} \bs{\ell}
  \qquad \qquad
  \bor \equiv \frac{e \hbar}{2m_e}
  \label{eq:mu-orbital-definition}
\end{equation}
where $ \bor \simeq 5.7884 \cdot 10^{-5} \ev/\text{T} $ is the \bctxt{Bohr magneton}.}

\begin{lemma}[before upper = {\tcbtitle}]{Expectation value of $ \bs{\mu} $}{}
  \begin{equation}
    \braket{\mu_x} = \braket{\mu_y} = 0
  \end{equation}
\end{lemma}

\begin{proofbox}
  \begin{proof}
    Since $ \braket{\bs{\mu}} \propto \braket{\bs{\ell}} $, WTS $ \braket{\ell_x} = \braket{\ell_y} = 0 $. Recall the action of the raising and lowering operators on the spherical harmonics:
    \begin{equation}
      \ell_\pm \ket{Y_{\ell , m}} = c_{\ell , m} \ket{Y_{\ell , m \pm 1}}
      \qquad \qquad
      c_{\ell , m} = \hbar \sqrt{\ell (\ell + 1) - m (m \pm 1)}
    \end{equation}
    with an abuse of notation.
    By direct calculation:
    \begin{equation*}
      \braket{\ell_\pm} \equiv \braket{Y_{\ell , m} | \ell_\pm | Y_{\ell , m}} = c_{\ell , m} \braket{Y_{\ell , m} | Y_{\ell , m \pm 1}} = 0
    \end{equation*}
    by the orthonormality of the spherical harmonics. Now, consider that:
    \begin{equation}
      \ell_x = \frac{\ell_+ + \ell_-}{2}
      \qquad \qquad
      \ell_y = \frac{\ell_+ - \ell_-}{2 \img}
      \label{eq:angular-momentum-plus-minus}
    \end{equation}
    This shows that $ \braket{\ell_x} = \braket{\ell_y} = 0 $.
  \end{proof}
\end{proofbox}

By carefully orienting the magnetic field, then, $ E = - \mu_z B_z $, thus the each atom is subjected to a force given by:
\begin{equation*}
  \ve{F} = \mu_z \grad B_z
\end{equation*}
The magnetic field is constructed in such a way that it is well-collimated with the axis of the experimental apparatus, hence $ \pa_x B_z = 0 $ and $ \pa_y B_y \approx 0 $, and the force to be measured is $ F_z = \mu_z \pa_z B_z $. Assuming that $ \pa_z B_z $ is constant and known for the apparatus, the atoms follow parabolic trajectories on the $ z $-axis dependent on $ \mu_z $.

Since $ L = 0 $ for the ground state of $ \ch{_{47}Ag} $, there should be only one observed outgoing trajectory corresponding to $ F_z = 0 $, but two parabolic trajectories are observed instead: this proves the existence of an additional intrinsic angular momentum in the valence electron, i.e. the spin, which has two quantized values, i.e. $ s = \pm \frac{1}{2} $. Moreover, this experiment also allows us to compute the \bctxt{Landé factor} for the spin\footnotemark, which is found to be $ g_s = 2 $, i.e.:
\begin{equation}
  \mu_s = - 2 \frac{\bor}{\hbar} \bs{s}
\end{equation}

\footnotetext{To be precise, the Landé factor for the spin is computed in QED to be:
  \begin{equation}
    g_s = 2 \left[ 1 + \frac{\alpha}{2\pi} + \smo(\alpha^2) \right]
    \qquad \qquad
    \alpha \equiv \frac{e^2}{4\pi \epsilon_0 \hbar c}
  \end{equation}
where $ \alpha \simeq \frac{1}{137} $ is the \bctxt{fine structure constant}.}

\subsection{Relativistic correction}

Relativistic correction are found writing $ E = T + V $, with $ T^2 = p^2 c^2 + m^2 c^4 $, and quantizing $ E \mapsto \img \hbar \pa_t $ and $ \ve{p} \mapsto -\img \hbar \grad $. By expanding $ T $ up to order $ p^4 $, the Hamiltonian of the system can be written as:
\begin{equation}
  \ham = \ham_0 + \ham_\text{rel}
\end{equation}
where $ \ham_0 $ is the Hamiltonian in \eref{eq:1-particle-hamiltonian} and $ \ham_\text{rel} $ includes the relativistic corrections:
\begin{equation}
  \ham_\text{rel} = - \frac{p^4}{8 m^3 c^2} + \frac{1}{2 m^2 c^2} \frac{1}{r} \frac{\dd V(r)}{\dd r} \bs{\ell} \cdot \bs{s} + \frac{\hbar^2}{8 m^2 c^2} \lap V(r)
\end{equation}
The first term is a kinetic correction, and it dominates for electrons closer to the nucleus (in particular s-electrons), while the third term (the \bctxt{Darwin term}) can be rewritten as $ \lap V(r) = e \lap \phi(r) = 4\pi e \rho(r) $, where $ \rho(r) $ is the charge distribution which generates the potential acting on the considered electron, i.e. $ \rho(r) = Z e \delta(r) $: since $ E[\psi] = \braket{\psi | \ham_\text{rel} | \psi} $, this last term is non-vanishing only for s-electrons, as only $ R_{n , 0}(r) $ is non-vanishing at $ r = 0 $.
On the other hand, the second term is the \bctxt{spin-orbit} term and is non-vanishing only for $ \ell \neq 0 $.

Denoting these terms by $ \ham_1 $, $ \ham_2 $ and $ \ham_3 $, we can compute the corresponding energy corrections:
\begin{equation}
  \Delta E_1 = \frac{Z^4 \ryd \alpha^2}{n^3} \left( \frac{3}{4n} - \frac{1}{\ell + \frac{1}{2}} \right)
\end{equation}
\begin{equation}
  \Delta E_2 = \frac{Z^4 \ryd \alpha^2}{n^3 \ell \left( \ell + 1 \right) \left( \ell + \frac{1}{2} \right)} \left[ j \left( j + 1 \right) - \ell \left( \ell + 1 \right) - s \left( s + 1 \right) \right]
  \label{eq:spin-orbit-1-e}
\end{equation}
\begin{equation}
  \Delta E_3 = \frac{Z^4 \ryd \alpha^2}{n^3} \delta_{\ell , 0}
\end{equation}
where $ \bs{j} = \bs{\ell} + \bs{s} $ is the total angular momentum of the electron. These three corrections have the same order of magnitude, since they are all proportional to $ Z^4 \ryd \alpha^2 / n^3 $, and can be combined in a single relativistic correction (using $ s = \frac{1}{2} $ for electrons):
\begin{equation}
  \Delta E_\text{rel} = - \frac{Z^4 \ryd \alpha^2}{n^4} \left( \frac{n}{j + \frac{1}{2}} - \frac{3}{4} \right)
\end{equation}
Note that the total angular momentum has quantized values $ \abs{\ell - s} \leq j \leq \ell + s $, hence the maximum value for $ j + \frac{1}{2} $ is $ \ell_\text{max} + 1 = n $ and the relativistic correction is always negative.

This relativistic correction introduces a dependence of the energy eigenvalue on $ j $, hence it induces a splitting of the previously-degenerate energy levels: this is the \bctxt{fine-structure splitting}.










