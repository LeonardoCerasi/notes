\selectlanguage{english}

\subsection{Lamb shift}

By \eref{eq:hydr-rel}, it is clear that states with same $ n $ and $ j $ quantum numbers are degenerate, even if their $ \ell $ quantum numbers differ: this is a feature of the Coulombian potential, thus only valid for hydrogenoid atoms.

However, there is still another small correction which breaks the degeneracy. Indeed, accounting for the finite dimensions of the nucleus and of the electron, a correction which depends on the distance of the electron from the nucleus is determined: this is called the \bctxt{Lamb shift}. In particular, this correction is greater (positively) for $ \text{s} $-electrons, e.g. the $ 2\text{S}_{1/2} $ and $ 2\text{P}_{1/2} $ are split by $ \sim 10 \mmev $.

\subsubsection{Electronic transitions}

Due to the Lamb shift, the lowest hydrogenoid states are, with increasing energies, $ 1\text{S}_{1/2} $, $ 2\text{P}_{1/2} $ and $ 2\text{S}_{1/2} $. To measure the Lamb shift, we could measure the energy difference between the spectral lines $ 2\text{S}_{1/2} \lra 1\text{S}_{1/2} $ and $ 2\text{S}_{1/2} \lra 2\text{P}_{1/2} $: however, the former is dipole-forbidden since $ \Delta \ell = 0 $, while the latter is suppressed by the small energy difference of the two states. Indeed, recall that the rate for \bctxt{spontaneous decay} (i.e. emission of radiation) from an initial state $ \ket{\text{i}} $ to a final state $ \ket{\text{f}} $ is:
\begin{equation}
  \gamma_\text{if} = \frac{\hbar}{2\pi m_e^2 c^3} \frac{e^2}{4\pi \epsilon_0} \omega_\text{if} \abs{\mat_\text{if}(\omega_\text{if})}^2
\end{equation}
It is also possible to compute the rate of \bctxt{stimulated decay} (or excitation) induced by an EM radiation of intensity $ I(\omega) $:
\begin{equation}
  \bar\gamma_\text{if} = \frac{4\pi^2}{m_e^2 c} \frac{e^2}{4\pi \epsilon_0} \frac{I(\omega_\text{if})}{\omega_\text{if}^2} \abs{\mat_\text{if}(\omega_\text{if})}^2
\end{equation}
Given the small energy split between the $ 2\text{P}_{1/2} $ and $ 2\text{S}_{1/2} $ states, with an EM radiation at the right frequency it is possible to excite both these states: the $ 2\text{P}_{1/2} $ state decays to the ground state $ 1\text{S}_{1/2} $ after $ \sim 10^{-9} \,\text{sec} $, while the $ 2\text{S}_{1/2} $ remains a metastable state, since the spontaneous decay to the ground state is dipole-forbidden and that to the $ 2\text{P}_{1/2} $ state is energetically suppressed.

\chapter{Multi-electron Atoms: Qualitative Treatment}

\section{Alkali metals}

The general Hamiltonian for an atom with $ N_e $ electrons is:
\begin{equation}
  \ham = \sum_{i = 1}^{N_e} \frac{p_i^2}{2m_e} - \sum_{i = 1}^{N_e} \frac{Z e^2}{r_i} + \frac{1}{2} \sum_{i \neq j = 1}^{N_e} \frac{e^2}{r_{ij}}
\end{equation}
with $ r_{ij} \equiv \abs{\ve{r}_i - \ve{r}_j} $. Clearly, the third term cannot be treated as a perturbation, since it can be larger than the unperturbed potential.

However, a class of atoms for which an approximate solution can be found is that of alkali atoms: indeed, these atoms only have one valence electron, hence the problem reduces to that of a hydrogenoid atoms, although with a different potential. To see this, note that a completely full shell has a spherically-symmetric charge distribution, since:
\begin{equation*}
  \sum_{m = -\ell}^\ell \abs{Y_{\ell,m}(\vartheta,\varphi)}^2 = \text{const.}
\end{equation*}
Since the $ Z-1 $ core electrons screen the Coulombian potential of the nucleus, the potential acting on the valence electron is intermediate between the fully-screened potential $ - e^2 / r $ (which dominates for $ r \ra \infty $) and the un-screened potential $ - Ze^2 / r $ (which dominates for $ r \ra 0 $). To express this, the energy eigenvalues for the screened potential are written as:
\begin{equation}
  E_{n,\ell} = - \frac{\ryd}{(n - \alpha_\ell)^2}
\end{equation}
where the \bctxt{quantum defect} $ \alpha_\ell $ encodes the non-Coulombian nature of the potential. Note the absence of a $ Z $-dependence. Quantum defects are experimentally measured: e.g. for $ \ch{Na} $ the relevant ones are $ \alpha_\text{s} = 1.35 $, $ \alpha_\text{p} = 0.83 $ and $ \alpha_\text{d} = 0.01 $, while the other ones are essentially zero (this signals the progressively increasing distance of orbitals from the nucleus).

It is possible to include the fine-structure splitting too: for non-hydrogenoid atoms, which are increasingly heavier, the kinetic term and the Darwin term rapidly become negligible, thus the relevant correction is due to the spin-orbit coupling. Recall that:
\begin{equation*}
  \ham_\text{s-o} = \frac{1}{2m_e^2 c^2} \frac{1}{r} \frac{\dd V(r)}{\dd r} \ve{L} \cdot \ve{S}
\end{equation*}
where $ \ve{L} $ and $ \ve{S} $ are the total orbital and spin angular momenta (\eref{eq:multi-e-ang}). Since the exact analytic expression of the potential is unknown, the spin-orbit correction is written as:
\begin{equation}
  \Delta E_\text{s-o} = \frac{\xi_{n,\ell}}{2} \left[ J (J+1) - L (L+1) - S (S+1) \right]
\end{equation}
where the spin-orbit factor $ \xi_{n,\ell} $ still depends on the $ n $ and $ \ell $ quantum numbers.










