\selectlanguage{english}

\chapter{Electronic Properties of Crystals}

Our main interest is in the description of metals. In the roughest approximations, the independent non-interacting electrons model, the valence electrons in the metal are assumed to be free and only bound to move inside the crystal: this approximation ignores the screening due to the interaction between electrons, since these are unaffected by any potential (except for the exclusion potential, since electrons are fermions).

This approximation best describes alkali metals, where the single valence electron can be assumed to be free in the metal, while it fails for more complex systems.

\section{Linear chain}

\subsection{Monodimensional case}

Consider a 1D linear chain, assuming it is infinite. The free-electron Schrödinger equation reads:
\begin{equation*}
  - \frac{\hbar^2}{2m_e} \frac{\dd^2 \psi}{\dd x^2} = E \psi
  \quad \implies \quad
  \psi \sim e^{-\img \left( k x - \frac{E}{\hbar} t \right)}
\end{equation*}
Denoting the distance between adjacent atoms with $ a $ and the total length of the chain with $ L $, the periodic boundary condition (so to have an infinite chain) impose:
\begin{equation}
  k_n = \frac{2\pi}{L} n
  \qquad \qquad
  n \in \Z - \{0\}
\end{equation}
analogously to the treatment of crystal vibrations. Inserting the functional expression for the wave-function in the Schrödinger equation, the dispersion curve for the free electron is recovered:
\begin{equation}
  E_n = \frac{\hbar^2 k_n^2}{2m_e} = \frac{2 \pi^2 \hbar^2}{m_e L^2} n^2
\end{equation}
Note that there is a double degeneracy: one due to $ E_n = E_{-n} $ and one due to spin. This means that the energy levels of these free electrons can be labelled with $ n \in \N $, and each state has degeneracy $ g = 4 $. Then, due to the Fermi--Dirac statistics, if there are $ N_e $ electrons, only the lowest $ N_e / 4 $ states are populated, and the energy of the highest state is called the \bctxt{Fermi energy}:
\begin{equation}
  E_\text{F} = \frac{\pi^2 \hbar^2}{8m_e} \left( \frac{N_e}{L} \right)^2
\end{equation}
Since $ N_e = n_v N_a $, where $ n_v $ is the number of valence electrons of the considered atomic species and $ N_a $ is the number of atoms in the linear chain, then $ N_a / L = 1 / a $ is the linear density of atoms.

It is useful to define a density of states $ D(E) $ in this case too. The number of states up to energy $ E $, ignoring the spin degeneracy, is:
\begin{equation*}
  E(N) = \frac{\pi^2 \hbar^2}{8m_e} \left( \frac{N}{L} \right)^2
  \quad \implies \quad
  N(E) = \frac{L}{\pi \hbar} \sqrt{2m_e E}
\end{equation*}
and the density of states is:
\begin{equation*}
  D(E) = \frac{\dd N}{\dd E} = \frac{L}{\pi \hbar} \sqrt{\frac{m}{2E}}
\end{equation*}

\subsection{Tridimensional case}

Generalizing to the 3D case:
\begin{equation*}
  - \frac{\hbar^2}{2m_e} \lap \psi = E \psi
  \quad \implies \quad
  \psi \sim e^{-\img \left( \ve{k} \cdot \ve{x} - \frac{E}{\hbar} t \right)}
\end{equation*}
and the periodic boundary conditions impose:
\begin{equation}
  k_i = \frac{2\pi}{L} n_i
  \qquad
  n_i \in \Z - \{0\}
\end{equation}
for $ i = x,y,z $. Then, the dispersion relation becomes:
\begin{equation}
  E = \frac{\hbar^2 k^2}{2m_e} = \frac{2 \pi^2 \hbar^2}{m_e L^2} \left( n_x^2 + n_y^2 + n_z^2 \right)
\end{equation}
In this case, the $ E = \text{const.} $ surfaces are \emph{always} spheres in reciprocal space (while for vibrations this was true only for small frequencies). Then:
\begin{equation}
  N(E) = 2 \frac{\frac{4}{3} \pi k^3}{\left( \frac{2\pi}{L} \right)^3} = \left( \frac{2m_e E}{\hbar^2} \right)^{3/2} \frac{V}{3\pi^2}
  \label{eq:ne-metal}
\end{equation}
and the density of states reads:
\begin{equation}
  D(E) = \frac{\dd N}{\dd E} = \frac{V}{2\pi^2} \left( \frac{2m_e}{\hbar^2} \right)^{3/2} \sqrt{E}
  \label{eq:de-metal}
\end{equation}
While in the 1D case $ D(E) $ decreases as $ E $ increases, since the gaps between levels become wider, in the 3D the opposite happens, as the number of states between $ E $ and $ E + \dd E $ increases as $ E $ increases. It is possible to show that in 2D the density of states is constant.

To determine the Fermi energy, we can impose $ N(E_\text{F}) = N_e $, since we already accounted for spin in the expression for $ N(E) $. Solving this equation yields:
\begin{equation}
  N_e = \frac{2V}{3\pi^2} k_\text{F}^3
  \quad \implies \quad
  k_\text{F} = \sqrt[3]{3\pi^2 n_e}
\end{equation}
where $ n_e \equiv N_e / V \sim a^{-1} $ is the electron number density.

%
%
%
% MANCA REGISTRAZIONE
%
%
%

\newpage
\section{Free electrons in metals}

\begin{proposition}{}{}
  For electrons in metals, which can be modelled as free electrons, the electronic contribution to the specific heat is:
  \begin{equation}
    C_V \simeq \frac{\pi^2}{2} N_e \kb \frac{T}{T_\text{F}}
  \end{equation}
  where $ T_\text{F} $ is the Fermi temperature of the electrons.
\end{proposition}

\begin{proofbox}
  \begin{proof}
    The total energy of the system is:
    \begin{equation*}
      U = \int_0^\infty \dd E \, D(E) f(E,T) E
    \end{equation*}
    where $ f(E,T) $ is the Fermi--Dirac distribution. Then, $ C_V = \frac{\dd U}{\dd T} $, so consider first the following integral:
    \begin{equation*}
      \int_0^\infty \dd E \, D(E) \frac{\pa f(E,T)}{\pa T} E_\text{F}
    \end{equation*}
    For metals $ T \ll T_\text{F} $, hence the variation with temperature of the distribution is, expressed as a function of energy, just two delta functions peaked at $ E = E_\text{F} $ (this is an approximation): one negative at $ E \ra E_\text{F}^- $, which represents the states which are devoid of electrons due to the increase in temperature, and one positive at $ E \ra E_\text{F}^+ $, which represents the states which are filled by those electrons. Then, since these delta functions peak the value of $ D(E) $ at $ D(E_\text{F}) $ and since they have the same area with opposite sign, the integral vanishes. This allows to rewrite:
    \begin{equation*}
      C_V = \int_0^\infty \dd E \, D(E) \frac{\pa f(E,T)}{\pa T} \left( E - E_\text{F} \right)
    \end{equation*}
    With the same reasoning, $ D(E) $ gets peaked at $ D(E_\text{F}) $ by the partial derivative. The latter can be written as:
    \begin{equation*}
      \frac{\pa f(E,T)}{\pa T} = \frac{\pa}{\pa T} \left[ e^{\left( E - E_\text{F} \right) / \kbt} + 1 \right]^{-1} = \frac{e^{\left( E - E_\text{F} \right) / \kbt}}{\left[ e^{\left( E - E_\text{F} \right) / \kbt} + 1 \right]^2} \frac{E - E_\text{F}}{\kbt^2} \equiv \frac{e^x}{\left[ e^x + 1 \right]^2} \frac{x}{T}
    \end{equation*}
    which results in:
    \begin{equation*}
      C_V = D(E_\text{F}) \kb^2 T \int_{-\frac{E_\text{F}}{\kbt}}^\infty \dd x \frac{x^2 e^x}{\left( e^x + 1 \right)^2} \simeq D(E_\text{F}) \kb^2 T \int_{-\infty}^{\infty} \dd x \frac{x^2 e^x}{\left( e^x + 1 \right)^2} = \frac{\pi^2}{3} D(E_\text{F}) \kb^2 T
    \end{equation*}
    since $ T \ll T_\text{F} $. By \eeref{eq:ne-metal}{eq:de-metal}:
    \begin{equation*}
      D(E) = \frac{3}{2} \frac{N(E)}{E}
      \quad \implies \quad
      D(E_\text{F}) = \frac{3}{2} \frac{N_e}{E_\text{F}}
      \quad \implies \quad
      C_V \simeq \frac{\pi^2}{2} N_e \kb \frac{T}{T_\text{F}}
    \end{equation*}
    which is the thesis.
  \end{proof}
\end{proofbox}

Then, the total specific heat of a metal at low temperature can be expressed as:
\begin{equation*}
  C_V(T) = \gamma T + A T^3
\end{equation*}
where the linear term is the electronic contribution, valid for $ T \ll T_\text{F} \sim 12'000 \,\text{K} $, and the cubic term is the lattice contribution. For insulators (non-metals) $ \gamma = 0 $.

\subsection{Electrical conductivity}

Electrical conductivity is a unique property of metals, since electrons in insulators are not free to oscillate. In presence of an EM field, an electron is subject to the Lorentz force, resulting in the following equation of motion:
\begin{equation}
  \hbar \frac{\dd \ve{k}}{\dd t} = - e \left( \ve{E} + \frac{\hbar}{m_e} \ve{k} \times \ve{B} \right)
  \label{eq:lorentz}
\end{equation}
recalling that $ m_e \ve{v} = \ve{p} = \hbar \ve{k} $. If $ \ve{B} = \ve{0} $, the solution is:
\begin{equation*}
  \ve{k}(t) = \ve{k}(0) - \frac{eE}{\hbar} t
\end{equation*}
Since the filled electronic states are inside the Fermi sphere $ k \leq k_\text{F} $ (this is exact at $ T = 0 $, while at $ T > 0 $ this is approximately true since small variations only happen near the boundary of the Fermi sphere), this means that an electric field shifts the Fermi sphere in an opposite direction with respect to itself.

This result does not agree with experiments: the measured current in the metal is $ i \propto v $, which means that if $ v $ increases linearly so must do $ i $, a phenomenon which is not observed. On the contrary, applying a constant electric field to a metal, a constant current and a constant potential difference are observed: this is known as \bctxt{Ohm's law}, which hints to the fact that the motion of the electrons in the metal is similar to the motion of a body in a viscous fluid (indeed, the free-electron model is only an approximation, and the electrons in the metal are not really free). Adding a viscous term to \eref{eq:lorentz}:
\begin{equation}
  \hbar \frac{\dd \ve{k}}{\dd t} = - e \ve{E} - \frac{\hbar \ve{k}}{\tau}
  \label{eq:lorentz-ohm}
\end{equation}
This viscous term $ - m_e \ve{v} / \tau $ represents the dissipative force determined by the random scatterings between electrons: $ \tau $ is then the mean time between collisions. To derive Ohm's law, impose $ \frac{\dd \ve{k}}{\dd t} = \ve{0} $, which has solution:
\begin{equation}
  \ve{v}_\text{d} = - \frac{e\tau}{m_e} \ve{E} \equiv - \mu \ve{E}
  \label{eq:drift-velocity}
\end{equation}
which is called \emph{drift velocity}, while $ \mu $ is the \emph{electrical mobility} of the metal. The drift velocity is the average velocity at which electrons move in the metal.

\subsubsection{Macroscopic analysis}

Consider a volume of section $ S $ and length $ \ell $: the total charge inside the volume is $ Q = N_q q $, with $ N_q =  n_q \ell S $ and $ n_q $ numeric density of the charges. The density current then is:
\begin{equation*}
  J = \frac{Q}{S t} = \frac{n_q q \ell S}{S \frac{\ell}{v_\text{d}}} = n_q q v_\text{d}
\end{equation*}
Inserting \eref{eq:drift-velocity}:
\begin{equation}
  \ve{J} = \frac{n_q q^2 \tau}{m_q} \ve{E}
\end{equation}
For a metal the charges are the electrons, i.e. $ q = -e $, but note that Ohm's law is insensitive to the sign of the charges. This relation allows to defined both resistivity $ \rho $ and conductivity $ \sigma \equiv \frac{1}{\rho} $:
\begin{equation}
  \sigma = \frac{n_q q^2 \tau}{m_q}
\end{equation}

\begin{example}{Copper}{}
  For $ \ch{Cu} $, resistivity is experimentally found to be $ \rho \simeq 1.80 \cdot 10^{-8} \, \Omega \, \text{m} $, while its density is $ d = 8930 \, \text{kg} \, \text{m}^3 $. Since copper has a single valence electron:
  \begin{equation*}
    n_e = \frac{N_e}{V} = \frac{N_e d}{M} = \frac{N_a d}{M} = \frac{d}{M_\text{Cu}}
  \end{equation*}
  Hence, the mean time between electronic collisions is:
  \begin{equation*}
    \tau = \frac{m_e M_\text{Cu}}{d e^2 \rho} \simeq 2.33 \cdot 10^{-14} \, \text{}
  \end{equation*}
  Then, it is possible to estimate the mean free path for the electrons in $ \ch{Cu} $:
  \begin{equation*}
    \lambda = \tau v_\text{F} = \tau \frac{\hbar}{m_e} k_\text{F} = \tau \frac{\hbar}{m_e} \sqrt[3]{3\pi^2 n_e}
  \end{equation*}
  The Fermi velocity, which is the maximum velocity of the electrons, is $ v_\text{F} \simeq 1.57 \cdot 10^6 \,\text{m} \, \text{s}^{-1} $, and the mean free path is $ \lambda \simeq 36.6 \, \text{nm} $, i.e. $ \lambda \gg a $: this justifies the free-electron approximation, since the electrons are effectively free in the primitive cell of the lattice, and it also confirms that the viscous effect on their motion is determined by electron-electron scattering, not by electron-nucleus scattering.

  The shift $ \Delta k $ of the Fermi sphere can be estimated from $ \hbar \Delta k = m_e v_\text{d} $:
  \begin{equation*}
    \Delta k = \frac{m_e}{\hbar} v_\text{d} = \frac{m_e}{\hbar} \frac{e}{m_e} \tau E = \frac{e \tau}{m_e} \rho J
  \end{equation*}
  For a current density $ J = 1 \, \text{A} \, \text{m}^2 $ the shift is $ \Delta k \simeq 0.64 \, \text{m}^{-1} $, which is extremely small if compared to $ k_\text{F} \simeq 1.34 \cdot 10^{10} \, \text{m}^{-1} $: since the only electrons which contribute to the current density are those affected by the shift, which is extremely close to the boundary of the Fermi sphere, this justifies the computation of the mean free path using the Fermi velocity. This implies that $ v_\text{d} \ll v_\text{F} $, which is true since $ v_\text{d} \simeq 7.4 \cdot 10^{-5} \,\text{m} \, \text{s}^{-1} $.
\end{example}

\subsection{Hall effect}

Now, consider \eref{eq:lorentz-ohm} with a magnetic field $ \ve{B} \neq \ve{0} $:
\begin{equation}
  m_e \frac{\dd \ve{v}}{\dd t} = -e \left( \ve{E} + \ve{v} \times \ve{B} \right) - \frac{m_e}{\tau} \ve{v}
\end{equation}
Assuming $ \ve{B} = B \hat{\ve{e}}_z $:
\begin{equation*}
  m_e \left( \frac{\dd}{\dd t} + \frac{1}{\tau} \right) v_x = -e \left( E_x + B v_y \right)
  \qquad \qquad
  m_e \left( \frac{\dd}{\dd t} + \frac{1}{\tau} \right) v_y = -e \left( E_y - B v_x \right)
\end{equation*}
\begin{equation*}
  m_e \left( \frac{\dd}{\dd t} + \frac{1}{\tau} \right) v_z = -e E_z
\end{equation*}
The EM field is assumed to be stationary, hence the electron reaches a constant velocity due to the viscous term. To find this drift velocity, it is useful to adopt the \bctxt{Hall geometry}: the $ z $-axis has been fixed by $ \ve{B} $, while the $ x $-axis if fixed by the current density $ \ve{J} = J \hat{\ve{e}}_x $, which is applied externally. In this geometry, it is intuitive to see that the drift velocity only has an $ x $-component: indeed, the current density puts the electrons in motion along $ \hat{\ve{e}}_x $, so they are subject to a Lorentz force along $ -\hat{\ve{e}}_y $ due to $ \ve{B} $, but this has only a transitory effect: indeed, due to the finite dimension of the metal sample we are considering (WLOG a parallelepiped), the electrons will accumulate on the side in the $ -\hat{\ve{e}}_y $, thus negatively charging it and positively charging the opposite side, and this generates a potential difference and an electric field which balance the magnetic Lorentz force. Then, setting $ v_x \equiv v_\text{d} $, the above system becomes:
\begin{equation*}
  m_e \frac{v_\text{d}}{\tau} = -e E_x
  \qquad \qquad
  0 = -e \left( E_y - v_\text{d} B \right)
  \qquad \qquad
  0 = 0
\end{equation*}
which has solutions:
\begin{equation*}
  v_\text{d} = - \frac{e \tau}{m_e} E_x
  \qquad \qquad
  E_y = - \frac{e \tau}{m_e} E_x B
\end{equation*}
Note that the drift velocity is the same as in the case without magnetic field.
Recalling that $ J = - n_e e v_\text{d} $, the transverse electric field can be written as:
\begin{equation}
  E_y = - \frac{1}{n_e e} J B
\end{equation}
Note that this quantity is indeed sensitive to the sign of the charges, as for general charges the constant is $ (n_q q)^{-1} $.

\subsection{Optical properties of metals}

Empirically, no transparent metal is observed.

\subsubsection{Electric fields in metals}

Consider a metal sample (WLOG a parallelepiped) and apply a stationary electric field $ \ve{E} $ perpendicular to one of its sides: in a transient the electrons accumulate on the side opposite to the direction of the electric field, hence at equilibrium the sides are electrically charged and inside the metal an electric field $ - \ve{E} $ is present, since in stationary conditions the electric field inside a conductive object vanishes.

Now, suppose that the external electric field is instantaneously turned off. To study how electrons return to their equilibrium conditions, notice that the two sides along $ \ve{E} $ have been charged with superficial charge densities $ \pm \sigma $: assuming that the electron distribution has been displaced by a displacement $ u $ and denoting the section of the metal in the direction of $ \ve{E} $ as $ S $, then $ Q = S u n_e e $ and $ \sigma \equiv \frac{Q}{S} = u n_e e $. The equation of motion for the displacement of the electron distribution is:
\begin{equation*}
  m_e \ddot{u} = -e E = -e \frac{\sigma}{\epsilon_0} = - \frac{ne^2}{\epsilon_0} u
\end{equation*}
where we are neglecting the viscous term $ - m_e \dot{u} / \tau $. The solution is a displacement which oscillates around the equilibrium configuration $ u = 0 $ with a frequency:
\begin{equation}
  \omega_\text{p} = \sqrt{\frac{n_e e^2}{m_e \epsilon_0}}
\end{equation}
which is called the \bctxt{plasma frequency} of the metal (a plasma is a gas with ions). Accounting for the viscous term makes the oscillation damped.

\subsubsection{EM waves on metals}

Consider the same setup as before, but now consider a time-dependent external electric field $ E(t) = E_0 e^{\img \omega t} $, so that $ u(t) = u_0 e^{\img \omega t} $ (now accounting for the viscous term would introduce a phase shift $ \varphi $). The equation of motion then becomes:
\begin{equation*}
  - m_e \omega^2 u_0 = -e E_0
  \quad \implies \quad
  u(t) = \frac{e}{m_e \omega^2} E(t)
\end{equation*}
Since the displacement of the electron distribution determines a shift of the positive-charge (holes) distribution too, an electric dipole is formed with electric-dipole moment $ p = q u = -e u $:
\begin{equation}
  p(t) = - \frac{e^2}{m_e \omega^2} E(t)
\end{equation}
The polarization is defined as $ P = \frac{1}{V} \sum_i p_i $, but since all the electrons have the same electric-dipole moment $ \sum_i p_i = N_e p = V n_e p $, i.e.:
\begin{equation}
  P(t) = - \frac{n_e e^2}{m_e \omega^2} E(t)
\end{equation}
Recall that the electric displacement is defined as $ \ve{D} = \epsilon_0 \epsilon_r \ve{E} = \epsilon_0 \ve{E} - \ve{P} $, hence the relative dielectric constant of the metal is found as:
\begin{equation}
  \epsilon_r = 1 + \frac{P}{\epsilon_0 E}
\end{equation}
Since the time dependence cancels in the fraction, the relative dielectric constant is only dependent on the incoming frequency, and it can be expressed in terms of the plasma frequency:
\begin{equation}
  \epsilon_r(\omega) = 1 - \frac{\omega_\text{p}^2}{\omega^2}
\end{equation}
The relative dielectric constant is related to the refractive index by $ n = \sqrt{\epsilon_r} $ (neglecting the relative magnetic permeability, which is assumed to be $ \mu_r = 1 $), so, assuming that the EM wave propagates in the $ x $-direction inside the metal, the complex exponential becomes:
\begin{equation*}
  \exp \left[ \img \left( k x - \omega t \right) \right] = \exp \left[ \img \omega \left( \frac{n}{c} x - t \right) \right]
\end{equation*}
where we used the dispersion relation $ \omega(k) = c k $. Note that the first term is real if $ \omega > \omega_\text{p} $, which results in the oscillatory behavior of EM waves, while it is complex if $ \omega < \omega_\text{p} $: this results in an exponential damping factor, which would mean that the medium is optically opaque.

For metals $ \hbar \omega_\text{p} \sim 10 \ev $ (in the UV region), which means that metals are opaque to optical radiation, as empirically observed.

\subsection{Electronic bands}

\subsubsection{Linear chain}

Consider a 1D metal modelled as a linear chain of atoms separated by $ a $. Since electrons can be considered free in a (ideal) metal, they are represented by waves $ \psi \propto e^{\img k x} $. Being interal waves of the linear chain, they must satisfy the back-scattering condition, so that $ k = \frac{\pi}{a} n $ with $ n \in \Z - \{0\} $. Then, consider the interference between the $ n = +1 $ and the $ n = -1 $ waves:
\begin{equation*}
  \psi^\pm \propto \frac{1}{\sqrt{2}} \left[ e^{\img \frac{\pi}{a} x} \pm e^{- \img \frac{\pi}{a} x} \right] \propto
  \begin{cases}
    \sqrt{2} \cos \frac{\pi}{a} x & (+) \\
    \sqrt{2} \sin \frac{\pi}{a} x & (-)
  \end{cases}
\end{equation*}
Recall that the charge density is $ \rho \propto \abs{\psi}^2 $: clearly, the $ + $ configuration is energeticaly favourable when compared to the $ - $ configuration, since $ \rho^+ $ peaks at $ na $, i.e. at the lattice positions where atoms are places and where the potential is deeper (more negative), as opposed to $ \rho^- $ which vanishes at $ na $. To be precise:
\begin{equation*}
  \rho^+(x) \propto 2 \cos^2 \frac{\pi}{a} x
  \qquad \qquad
  \rho^-(x) \propto 2 \sin^2 \frac{\pi}{a} x
\end{equation*}
Note that $ \rho_0 \propto \abs{\psi_0}^2 \propto \abs{e^{\pm \img \frac{\pi}{a} x}}^2 = 1 $, and defining its energy $ E_0(k) = \hbar^2 k^2 / (2m_e) $ it is clear that $ E^+ < E_0(\frac{\pi}{a}) < E^- $ (values on the boundary of the 1BZ): the energy transitions smoothly from the parabola $ E_0(k) $ inside the 1BZ to a function with horizontal tangent at $ k = \pm \frac{\pi}{a} $ and with value $ E(\pm \frac{\pi}{a}) = E^+ $. Then, for $ \frac{\pi}{a} < k < \frac{2\pi}{a} $, the energy has again the behavior of a parabola at the center, while becoming horizontal on the boundary: in particular, at $ k = \frac{\pi}{a} $ it has value $ E(\frac{\pi}{a}) = E^- $. It is clear that the boundary of the 1BZ presents a discontinuity, i.e. a \bctxt{gap}.

To study how wide this gap is, note that it is determined by the periodic potential determined by the atoms placed in the lattice positions, which perturbes the otherwise free electrons. Since this potential $ V(x) $ is periodic, it is possible to express it as a Fourier series:
\begin{equation}
  V(x) = \sum_{h \in \Z} v_h e^{\img \frac{2\pi}{a} h x}
\end{equation}
In particular, in the case of the linear chain $ V(x) $ is an even function, i.e. $ V(-x) = V(x) $, hence the Fourier series reduces to a cosine series:
\begin{equation}
  V(x) = \sum_{h \in \N_0} v_h \cos \frac{2\pi h}{a} x
\end{equation}
The normalized wavefunctions are:
\begin{equation}
  \psi^+_n = \sqrt{\frac{2}{a}} \cos \frac{\pi n}{a} x
  \qquad \qquad
  \psi^-_n = \sqrt{\frac{2}{a}} \sin \frac{\pi n}{a} x
\end{equation}
It is now possible to compute the $ n^\text{th} $ energy gap:
\begin{equation*}
  \begin{split}
    E_\text{g}^{(n)}
    & \equiv \braket{\psi^-_n | V(x) | \psi^-_n} - \braket{\psi^+_n | V(x) | \psi^+_n} \\
    & = \frac{2}{a} \sum_{h \in \N_0} v_h \int_0^a \dd x \, \cos \frac{2\pi h}{a} x \left[ \sin^2 \frac{\pi n}{a} x - \cos^2 \frac{\pi n}{a} x \right] \\
    & = - \frac{2}{a} \sum_{h \in \N_0} v_h \int_0^a \dd x \, \cos \frac{2\pi h}{a} x \, \cos \frac{2\pi n}{a} x = - \frac{2}{a} \sum_{h \in \N_0} v_h \frac{a}{2} \delta_{hn} = - v_n
  \end{split}
\end{equation*}
Note that $ v_h < 0 \,\, \forall n \in \N $, since the potential is attractive, hence $ E_\text{g}^{(n)} > 0 $ as expected. Qualitatively, this shows that free-electron states (i.e. atomic orbitals) subject to a periodic potential naturally quantize in \bctxt{energy bands} separated by gaps: this is an intermediate behavior between free electrons (continuous spectrum) and atomic orbitals (discrete spectrum).

\subsubsection{Tridimensional case}

The quantization into energy bands generalizes trivially to the tridimensional case. In particular, focusing on the 1BZ with $ - \frac{\pi}{a} < k < \frac{\pi}{a} $, it is possible to count how many states are present in the energy band, i.e. how many electrons it can contain:
\begin{equation}
  N = 2 \frac{V}{a^3} \equiv 2 N_\text{c}
\end{equation}
where the $ 2 $ comes from the spin degrees of freedom and $ N_\text{c} $ is the number of primitive cells (in real space). This number of available states is to be compared to the number of electrons: indeed, $ N_e = n_\text{v} N_\text{c} $, where $ n_\text{v} $ is the number of valence electrons in the primitive cell (e.g. $ n_\text{d} = 1 $ for alkali metals\footnotemark, $ n_\text{d} = 2 $ for alkali-earth metals, etc.).

\footnotetext{This justifies the treatment of alkali metals using the free-electron approximation: only half of the energy band is filled with electrons, i.e. the energy can be approximated with the free-electron parabola.}

\paragraph{Semiconductors}

A particular case is that of semiconductors, i.e. elements of the IV group (in particular $ \ch{C} $, $ \ch{Si} $ and $ \ch{Ge} $), which are characterized by large gaps between energy bands: for example, diamond has gaps of $ 5.4 \ev $, which is way larger than visible light, and in fact it is transparent, while $ \ch{Si} $ has gaps of $ 1.17 \ev $ (IR) and is opaque.

If a sample of IV-group metals is drugged with V-group (or higher) metals, the excess of electrons fills the second energy band, which becomes a conductive band, thus obtaining a semiconductor from what normally is an insulator (due to the large gaps). The same result is achieved with a III-group (or lower) drugging, where now the valence band itself becomes a conducting band (even if there are few available conducting states) since it is no longer full. In both cases, the energy band as seen by active electrons, i.e. those near the Fermi energy, can be approximated as a parabola:
\begin{equation}
  E(k) = \frac{\hbar^2 k^2}{2 m_\text{eff}}
\end{equation}
where $ m_\text{eff} \in \R $ is an effective mass (positive for V-group or higher and negative for III-group or lower). Therefore, all the results in the free-electron approximation are still valid, with the substitution $ m_e \mapsto \abs{m_\text{eff}} $ and $ q = -e \mapsto - \sgn (m_\text{eff}) e $: if $ m_\text{eff} < 0 $, the freed conducting states at the boundary of the energy band behave like positively-charged particles. For example, the value of the effective mass can be probed measuring the plasma frequency, while its sign can be determined with a Hall experiment; moreover, since there are few conducting states in semiconductors, the electric field generated by the Hall effect is stronger than in the metal case ($ E_y \propto n_q^{-1} $), at the cost of requiring a stronger current density due to the higher resistance of the medium.

The effective mass is a general concept. Given an energy band $ E(k) $, the effective mass is defined by the curvature of the band:
\begin{equation}
  m_\text{eff} \defeq \hbar^2 \left[ \frac{\dd^2 E(k)}{\dd k^2} \right]^{-1}
\end{equation}
Intuitively, this effective mass is determined by the potential which perturbes the free electrons, modifying their inertia.










