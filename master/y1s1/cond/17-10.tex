\selectlanguage{english}

\section{Many-electron atoms}

Given an atom with $ N_e $ electrons, the Hamiltonian of the system (in the massive approximation for the nucleus) is:
\begin{equation}
  \ham = \sum_{i = 1}^{N_e} \left( - \frac{\hbar^2}{2m_e} \lap_{\ve{r}_i} - \frac{Ze^2}{r_i} \right) + \sum_{i = 1}^{N_e} \sum_{j > i}^{N_e} \frac{e^2}{r_{ij}}
\end{equation}
with $ r_{ij} \equiv \norm{\ve{r}_i - \ve{r}_j} $. This Hamiltonian can be rewritten as $ \ham = \sum_{i = 1}^{N_e} \ham_i $, with:
\begin{equation}
  \ham_i = - \frac{\hbar^2}{2m_e} \lap_{\ve{r}_i} - \frac{Ze^2}{r_i} + \frac{1}{2} \sum_{j = 1}^{N_e} \frac{e^2}{r_{ij}}
\end{equation}
The third term cannot be treated as a perturbation; however, for valence electron we can approximate the potential generated by core electrons as a central potential, thus writing:
\begin{equation*}
  V_i(\{\ve{r}_k\}) = - \frac{Ze^2}{r_i} + \frac{1}{2} \sum_{j = 1}^{N_e} \frac{e^2}{r_{ij}} \equiv V_\text{c}(\ve{r}_i) + \ham_i'
\end{equation*}
where the potential $ \ham_i' $ generated by other valence electrons can now be treated as perturbation (independent-electrons approximation). Then, the unperturbed energy of the system is:
\begin{equation*}
  E_0 = \sum_{i = 1}^{N_e} E_i^{(0)}
  \qquad \qquad
  E_i^{(0)} \equiv E_{n_i , \ell_i} \,:\, (\ham_i + V_\text{c}) \psi_i = E_i^{(0)} \psi_i
\end{equation*}
where the wave-functions $ \psi_i $ are the single-electron wave-functions in \eref{eq:1-particle-wavefunction}. The total wave-function needs to be anti-symmetrized, so we set the notation $ \phi_\sigma(q_i) \equiv \psi_{n_i , \ell_i , m_{\ell_i}}(\ve{r}_i) \swf_{m_{s_i}} $ with $ \sigma \equiv \{n , \ell , m_\ell , m_s\} $: the total anti-symmetric wave-function is then expressed as a \bctxt{Slated determinant}:
\begin{equation}
  \Phi(\{q_k\}) = \frac{1}{\sqrt{N_e!}}
  \begin{vmatrix}
    \phi_{\sigma_1}(q_1) & \phi_{\sigma_2}(q_1) & \dots & \phi_{\sigma_{N_e}}(q_1) \\
    \phi_{\sigma_1}(q_2) & \phi_{\sigma_2}(q_2) & \dots & \phi_{\sigma_{N_e}}(q_2) \\
    \vdots & \vdots & \ddots & \vdots \\
    \phi_{\sigma_1}(q_{N_e}) & \phi_{\sigma_2}(q_{N_e}) & \dots & \phi_{\sigma_{N_e}}(q_{N_e}) \\
  \end{vmatrix}
\end{equation}
This wave-function is anti-symmetric under exchange of two electrons, given the alternating nature of the determinant under exchange of rows/columns. Moreover, if two electrons have the same quantum numbers, then the total wave-function vanishes (since the determinant of a matrix with two equal rows/columns is zero), in accordance with the Pauli exclusion principle.
Note that, in general, the Slater determinant is not an eigenfunction of the spin operator.

\begin{example}{Slater determinant for ground state of helium}{}
  Consider $ [\ch{He}] 1\text{s}^2 $: then $ \phi_\uparrow(q_i) = \psi_{1,0,0}(\ve{r}_i) \swf_i(\uparrow) \equiv \phi_i(\uparrow) $ and $ \phi_\downarrow(q_i) = \psi_{1,0,0}(\ve{r}_i) \swf_i(\downarrow) \equiv \phi_i(\downarrow) $, and the Slater determinant is:
  \begin{equation*}
    \begin{split}
      \Phi_{1,2} \equiv \Phi(q_1 , q_2)
      & = \frac{1}{\sqrt{2}} \left[ \phi_1(\uparrow) \phi_2(\downarrow) - \phi_1(\downarrow) \phi_2(\uparrow) \right] \\
      & = \psi_{1,0,0}(\ve{r}_1) \psi_{1,0,0}(\ve{r}_2) \frac{1}{\sqrt{2}} \left[ \swf_1(\uparrow) \swf_2(\downarrow) - \swf_1(\downarrow) \swf_2(\uparrow) \right] \equiv \Psi_{1,2} \swf_{1,2}
    \end{split}
  \end{equation*}
  where $ \psi_{1,2} $ is symmetric and $ \swf_{1,2} $ is anti-symmetric. This is precisely the singlet $ S = 0 $ state that is the ground state of helium.
\end{example}

\begin{example}{Slater determinant for excited state of helium}{}
  Write the three wave-functions for the $ 2\ch{^3S_1} $ lowest energy state of helium in terms of the Slater determinant of the possible single-electron spin-orbital functions.

  \textit{Solution.} The $ 2 $ in the notation $ 2\ch{^3S_1} $ means that the electronic configuration considered is $ [\ch{He}] = 1\text{s}^1 2\text{s}^1 $. The possible single-electron wave-functions are then:
  \begin{align*}
    \phi_{1,0,0,\uparrow}(\ve{r}_i) & \equiv \phi_i(1\uparrow)
                                    &
    \phi_{2,0,0,\uparrow}(\ve{r}_i) & \equiv \phi_i(2\uparrow)
    \\
    \phi_{1,0,0,\downarrow}(\ve{r}_i) & \equiv \phi_i(1\downarrow)
                                    &
    \phi_{2,0,0,\downarrow}(\ve{r}_i) & \equiv \phi_i(2\downarrow)
  \end{align*}
  There are four possible Slater determinants:
  \begin{equation*}
    \begin{split}
      \Phi_\text{a}
      & = \frac{1}{\sqrt{2}} \left[ \phi_1(1\uparrow) \phi_2(2\uparrow) - \phi_1(2\uparrow) \phi_2(1\uparrow) \right] \\
      & = \frac{1}{\sqrt{2}} \left[ \psi_{1\text{s}}(\ve{r}_1) \psi_{2\text{s}}(\ve{r}_2) - \psi_{2\text{s}}(\ve{r}_1) \psi_{1\text{s}}(\ve{r}_2) \right] \swf_1(\uparrow) \swf_2(\uparrow) \equiv \Psi_{1,2}^-(1\text{s} , 2\text{s}) \swf_{1,2}^+(\uparrow,\uparrow)
    \end{split}
  \end{equation*}
  \begin{equation*}
    \begin{split}
      \Phi_\text{b}
      & = \frac{1}{\sqrt{2}} \left[ \phi_1(1\downarrow) \phi_2(2\downarrow) - \phi_1(2\downarrow) \phi_2(1\downarrow) \right] \\
      & = \frac{1}{\sqrt{2}} \left[ \psi_{1\text{s}}(\ve{r}_1) \psi_{2\text{s}}(\ve{r}_2) - \psi_{2\text{s}}(\ve{r}_1) \psi_{1\text{s}}(\ve{r}_2) \right] \swf_1(\downarrow) \swf_2(\downarrow) \equiv \Psi_{1,2}^-(1\text{s} , 2\text{s}) \swf_{1,2}^+(\downarrow,\downarrow)
    \end{split}
  \end{equation*}
  \begin{equation*}
    \begin{split}
      \Phi_\text{c}
      & = \frac{1}{\sqrt{2}} \left[ \phi_1(1\uparrow) \phi_2(2\downarrow) - \phi_1(2\downarrow) \phi_2(1\uparrow) \right] \\
      & = \frac{1}{\sqrt{2}} \left[ \psi_{1\text{s}}(\ve{r}_1) \psi_{2\text{s}}(\ve{r}_2) \swf_1(\uparrow) \swf_2(\downarrow) - \psi_{2\text{s}}(\ve{r}_1) \psi_{1\text{s}}(\ve{r}_2) \swf_1(\downarrow) \swf_2(\uparrow) \right]
    \end{split}
  \end{equation*}
  \begin{equation*}
    \begin{split}
      \Phi_\text{d}
      & = \frac{1}{\sqrt{2}} \left[ \phi_1(1\downarrow) \phi_2(2\uparrow) - \phi_1(2\uparrow) \phi_2(1\downarrow) \right] \\
      & = \frac{1}{\sqrt{2}} \left[ \psi_{1\text{s}}(\ve{r}_1) \psi_{2\text{s}}(\ve{r}_2) \swf_1(\downarrow) \swf_2(\uparrow) - \psi_{2\text{s}}(\ve{r}_1) \psi_{1\text{s}}(\ve{r}_2) \swf_1(\uparrow) \swf_2(\downarrow) \right]
    \end{split}
  \end{equation*}
  $ \Phi_\text{a} $ and $ \Phi_\text{b} $ are correctly anti-symmetrized, and they correspond to the states with $ m_S = +1 $ and $ m_S = -1 $ of the $ 2\ch{^3S_1} $ triplet. On the other hand, $ \Phi_\text{c} $ and $ \Phi_\text{d} $ are not spin eigen-functions, hence we need to consider linear combinations of them:
  \begin{equation*}
    \begin{split}
      \Phi_{\text{c} + \text{d}}
      & \equiv \frac{1}{\sqrt{2}} \left[ \Phi_\text{c} + \Phi_\text{d} \right] \\
      & = \frac{1}{\sqrt{2}} \left[ \psi_{1\text{s}}(\ve{r}_1) \psi_{2\text{s}}(\ve{r}_2) - \psi_{2\text{s}}(\ve{r}_1) \psi_{1\text{s}}(\ve{r}_2) \right] \frac{1}{\sqrt{2}} \left[ \swf_1(\uparrow) \swf_2(\downarrow) + \swf_1(\downarrow) \swf_2(\uparrow) \right] \\
      & \equiv \Psi_{1,2}^-(1\text{s} , 2\text{s}) \swf_{1,2}^+(\uparrow,\downarrow)
    \end{split}
  \end{equation*}
  \begin{equation*}
    \begin{split}
      \Phi_{\text{c} - \text{d}}
      & \equiv \frac{1}{\sqrt{2}} \left[ \Phi_\text{c} - \Phi_\text{d} \right] \\
      & = \frac{1}{\sqrt{2}} \left[ \psi_{1\text{s}}(\ve{r}_1) \psi_{2\text{s}}(\ve{r}_2) + \psi_{2\text{s}}(\ve{r}_1) \psi_{1\text{s}}(\ve{r}_2) \right] \frac{1}{\sqrt{2}} \left[ \swf_1(\uparrow) \swf_2(\downarrow) - \swf_1(\downarrow) \swf_2(\uparrow) \right] \\
      & \equiv \Psi_{1,2}^+(1\text{s} , 2\text{s}) \swf_{1,2}^-(\uparrow,\downarrow)
    \end{split}
  \end{equation*}
  It is then clear that $ \Phi_{\text{c} + \text{d}} $ is the $ m_S = 0 $ state of the $ 2\ch{^3S_1} $ triplet, while $ \Phi_{\text{c} - \text{d}} $ is the singlet state $ 2\ch{^1S_0} $ with higher energy.

  To confirm that this identification of states is correct, consider the four total wave-function given as a linear combinations of $ \Phi_\text{a} $, $ \Phi_\text{b} $, $ \Phi_\text{c} $ and $ \Phi_\text{d} $:
  \begin{equation*}
    \Psi_j = \sum_{i = 1}^4 c_{ij} \Phi_i
    \qquad \qquad
    \ham' \ket{\Psi_j} = \Delta E_j \ket{\Psi_j}
  \end{equation*}
  where $ \ham' = \frac{e^2}{r_{12}} $ is the interaction Hamiltonian. The homogeneus system can be rewritten as:
  \begin{equation*}
    \sum_{i = 1}^4 c_{ij} \braket{\Phi_k | \ham' | \Phi_i} = c_{kj} \Delta E_j
    \quad \iff \quad
    \sum_{i = 1}^4 \left( \braket{\Phi_k | \ham' | \Phi_i} - \delta_{ki} \Delta E_j \right) c_{ij} = 0
  \end{equation*}
  This means that $ \ve{c}_j $ are eigenvectors of $ \mt{H} = [H_{ij}] \equiv [\braket{\Phi_k | \ham' | \Phi_i}] \in \C^{4 \times 4} $ with eigenvalues $ \Delta E_j $, hence we impose that $ \det(\mt{H} - \mt{I}_4 \Delta E) = 0 $:
  \begin{equation*}
    \begin{vmatrix}
      H_{11} - \Delta E & H_{12} & H_{13} & H_{14} \\
      H_{21} & H_{22} - \Delta E & H_{23} & H_{24} \\
      H_{31} & H_{32} & H_{33} - \Delta E & H_{34} \\
      H_{41} & H_{42} & H_{43} & H_{44} - \Delta E \\
    \end{vmatrix}
    = 0
  \end{equation*}
  Given the spin parts of $ \Phi_\text{a} $ and $ \Phi_\text{b} $, it is clear that $ H_{1j} \propto \delta_{1j} $ and $ H_{2j} \propto \delta_{2j} $, which, given that $ H_{ij} = H_{ji}^* $, reduces the matrix to a block-diagonal form:
  \begin{equation*}
    \begin{vmatrix}
      H_{11} - \Delta E & 0 & 0 & 0 \\
      0 & H_{22} - \Delta E & 0 & 0 \\
      0 & 0 & H_{33} - \Delta E & H_{34} \\
      0 & 0 & H_{43} & H_{44} - \Delta E \\
    \end{vmatrix}
    = 0
  \end{equation*}
  Explicitly computing the remaining elements with \pref{prop:helium-interaction}:
  \begin{equation*}
    H_{11} = H_{22} = \braket{\Psi_{1,2}^-(1\text{s} , 2\text{s}) | \ham' | \Psi_{1,2}^-(1\text{s} , 2\text{s})} = \mathcal{J}(1\text{s} , 2\text{s}) - \mathcal{K}(1\text{s} , 2\text{s})
  \end{equation*}
  \begin{equation*}
    H_{33} = H_{44} = \frac{1}{2} \left[ \braket{1\text{s} , 2\text{s} | \ham' | 1\text{s} , 2\text{s}} + \braket{2\text{s} , 1\text{s} | \ham' | 2\text{s} , 1\text{s}} \right] = \mathcal{J}(1\text{s} , 2\text{s})
  \end{equation*}
  \begin{equation*}
    H_{34} = H_{43} = \frac{1}{2} \left[ - \braket{1\text{s} , 2\text{s} | \ham' | 2\text{s} , 1\text{s}} - \braket{2\text{s} , 1\text{s} | \ham' | 1\text{s} , 2\text{s}} \right] = - \mathcal{K}(1\text{s} , 2\text{s})
  \end{equation*}
  Setting $ \mathcal{J}(1\text{s} , 2\text{s}) \equiv \mathcal{J} $ and $ \mathcal{K}(1\text{s} , 2\text{s}) \equiv \mathcal{K} $:
  \begin{equation*}
    \begin{vmatrix}
      \mathcal{J} - \mathcal{K} - \Delta E & 0 & 0 & 0 \\
      0 & \mathcal{J} - \mathcal{K} - \Delta E & 0 & 0 \\
      0 & 0 & \mathcal{J} - \Delta E & -\mathcal{K} \\
      0 & 0 & -\mathcal{K} & \mathcal{J} - \Delta E \\
    \end{vmatrix}
    = 0
  \end{equation*}
  The normalized eigenvectors are then trivially found:
  \begin{align*}
    \Delta E_1 &= \mathcal{J} - \mathcal{K}
               &
    \ve{c}_1 &= \left( 1,0,0,0 \right)
    \\
    \Delta E_2 &= \mathcal{J} - \mathcal{K}
               &
    \ve{c}_2 &= \left( 0,1,0,0 \right)
    \\
    \Delta E_3 &= \mathcal{J} - \mathcal{K}
               &
    \ve{c}_3 &= \left( 0,0,\frac{1}{\sqrt{2}},\frac{1}{\sqrt{2}} \right)
    \\
    \Delta E_4 &= \mathcal{J} + \mathcal{K}
               &
    \ve{c}_4 &= \left( 0,0,\frac{1}{\sqrt{2}},-\frac{1}{\sqrt{2}} \right)
  \end{align*}
  which confirm the above linear combinations and identifications.
\end{example}

\subsection{Hartee--Fock method}

In the independent-electrons approximation, the interaction potential can be written as:
\begin{equation}
  \ham_i = - \frac{\hbar^2}{2m_e} \lap_{\ve{r}_i} - \frac{Ze^2}{r_i} + \frac{1}{2} \sum_{j \neq i} \int \dd^3 r_j \, \abs{\psi_{\sigma_j}(\ve{r}_j)}^2 \frac{e^2}{\norm{\ve{r}_i - \ve{r}_j}}
  \label{eq:hartree-fock-def}
\end{equation}
In this case, the set of equations $ \ham_i \psi_i = E_i \psi_i $ must be solved iteratively, since $ \ham_i $ depends on the solutions $ \{\psi_k\}_{k = 1, \dots , N_e} $: to do so (numerically), we start with an arbitrary set of wave-functions, we solve the system of equations and then use these solutions as the starting point of another iteration of this procedure. After a number of iterations, the difference between two consecutive sets of wave-functions drops below a pre-set threshold, signaling that an acceptable solution has been found: this is the \bctxt{Hartree--Fock self-consistent method}.

\subsection{Hund's rules}

Writing $ V_i = V_\text{c} + \ham'_i $, we can derive Hund's rules for the ground state from the interaction term. As already seen, Hund's first rule stems from the exchange integral determined by $ \ham'_i $, while Hund's second rule is determined by the dependence on the orbital angular momenta of all the electrons of the third term in \eref{eq:hartree-fock-def} (in addition to being justified by classical intuition). Note that, being exchange integrals generally larger than the $ L $-dependent terms in the interaction potential, the first rule must be applied before applying the second rule.

Finally, Hund's third rule comes from the spin-orbit coupling, hence (for lighter atoms) it has less priority than the first two rules. Having already maximized $ S $ and $ L $, the maximization or minimization depends on the sign of the spin-orbit constant $ \xi $:
\begin{itemize}
  \item if $ \xi > 0 $, then $ J $ must be minimized;
  \item if $ \xi < 0 $, then $ J $ must be maximized.
\end{itemize}

\begin{example}{Cloride}{}
  Consider $ \ch{_{17}Cl} $, which has the ground state configuration $ [\ch{Cl}] = [\ch{Ne}] 3\text{s}^2 3\text{p}^5 $. To determine the spectroscopic term of the ground state, focus on the valence electrons $ 3\text{p}^5 $: the maximum (and only) value for the spin is $ S = \frac{1}{2} $, while for the orbital angular momentum is $ L = 1 $ (since the unpaired electron either has $ m_L = 0 $ or $ m_L = \pm 1 $). Then, regarding the spin-orbit interaction, the valence shell if full except for a hole: this can be modelled as an empty shell with a positively-charged electron, hence the spin-orbit constant changes sign (since the magnetic moment of a positively-charged electron would be opposite to that of a normal electron) and is $ \xi < 0 $. As a consequence, $ J $ must be maximized to $ J = \frac{3}{2} $, hence the ground state of $ \ch{_{17}Cl} $ is $ \ch{^2P_{\sfrac{3}{2}}} $.
\end{example}

This reasoning can be generalized to an alternative expression of Hund's third rule:
\begin{itemize}
  \item if the valence shell is less than half-full, then $ J $ must be minimized;
  \item if the valence shell is more than half-full, then $ J $ must be maximized.
\end{itemize}

\subsection{\texorpdfstring{$ j-j $}{j-j} coupling}

Hund's rule require that the spin-orbit coupling is weaker than the electromagnetic interaction between electrons: this results in the Russell--Saunders coupling for the total angular momentum, which is found as $ \ve{J} = \ve{S} + \ve{L} $.

For heavier atoms, on the other hand, the spin-orbit coupling dominates over the electromagnetic interaction, hence each $ \bs{\ell}_i $ and $ \bs{s}_i $ are coupled into $ \bs{j}_i = \bs{\ell}_i + \bs{s}_i $, which are then combined in $ \ve{J} = \sum_{i = 1}^{N_e} \bs{j}_i $: this is the $ j-j $ coupling. In the $ j-j $ coupling, spectroscopic terms no longer represent good quantum numbers: for two-electron atoms, we adopt the notation $ (j_1 , j_2)_J $.

For intermediate cases, a linear combination of states from one of these two bases needs to be considered: in this case, selection rules for electron-dipole transitions are no longer valid, and transitions between singlet and triplet (or even multiplet) states are possible.










