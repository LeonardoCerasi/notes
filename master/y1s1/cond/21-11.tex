\selectlanguage{english}

\chapter{Crystal vibrations}

Like for molecules, we expect to find multiple normal modes, since the crystal can be seen as a molecule with an extremely large number of atoms: in particular, since for a molecule with $ N $ atoms there are $ 3N - 6 $ normal frequencies, for a crystal we similarly expect $ \sim 3N $ normal frequencies.

\section{Linear chain}

Consider a monoatomic linear chain, and adopt the approximation where the only interactions are between neighbouring atoms and have a harmonic nature. Then, label each atom with $ s = 0 , 1 , \dots , N - 1 $ and assume that their equilibrium distance is $ a $, so that the total length of the chain is $ L = (N - 1) a $, and that their mass is $ M $. Defining $ u_s $ as the longitudinal displacement (analogous to stretching modes in molecules) from its equilibrium position of the atom $ s $, then its equation of motion is:
\begin{equation}
  C \left( u_{s + 1} - 2 u_s + u_{s - 1} \right) = M \ddot{u}_s
  \label{eq:linear-chain}
\end{equation}
where $ C \in \R^+ $ is the spring constant of the harmonic interaction between neighbouring atoms. The wave ansatz reads:
\begin{equation}
  u_s(t) = u_0 e^{\img \left( k s a - \omega t \right)}
\end{equation}
where we set the $ s = 0 $ atom at the origin. Inserting this ansatz into \eref{eq:linear-chain}:
\begin{equation*}
  - \omega^2 M = C \left( e^{\img k a} + e^{-\img k a} - 2 \right) = 2C \left( \cos k a - 1 \right)
\end{equation*}
which results in the \bctxt{dispersion relation} for vibrations in a crystal:
\begin{equation}
  \omega = 2 \sqrt{\frac{C}{M}} \abs{\sin \frac{k a}{2}}
\end{equation}
Contrary to waves in a continuous medium (EM waves in vacuum $ \omega = c k $, sound waves in air $ \omega = v_\text{s} k $), crystal vibrations propagate in a discrete medium, hence their dispersion relation is non-linear (see \figref{fig:acoustic}).

\begin{figure}
  \centering
  \includegraphics[width = 0.60 \textwidth]{acoustic-branch.png}
  \caption{Acoustic branches in a monoatomic crystal.}
  \label{fig:acoustic}
\end{figure}

Note that this analysis holds for all atoms in the linear chains, except for the endpoints: since $ N \sim \av $, these endpoints can be ignored, and the chain can be assumed to be infinite. To do so, it is necessary to impose \bctxt{perdiodic boundary conditions}, i.e.:
\begin{equation}
  u(L) = u(0)
  \qquad \qquad
  u'(L) = u'(0)
\end{equation}
These imply that $ \exp \img k L = 1 $, i.e. they make the normal frequencies discrete. It is trivial to see that two waves whose wave-vectors differ by a vector in the discrete lattice. i.e. $ k $ and $ k + 2 \pi n / a $, determine the same normal mode: all of the physical information is then contained in the interval $ k \in \left( - \frac{\pi}{a} , \frac{\pi}{a} \right] $, as pictured in \figref{fig:acoustic}.
The discrete normal frequencies then are:
\begin{equation}
  k_n = \frac{2\pi n}{L}
  \qquad
  n = 0, \pm 1 , \pm 2 , \dots , + \frac{N}{2}
\end{equation}
where $ N $ is assumed even. The total number of normal modes has been computed as:
\begin{equation*}
  \frac{2\pi}{a} \div \frac{2\pi}{L} = \frac{L}{a} = N
\end{equation*}
Note that the $ n = 0 $ normal mode is a translation, since $ k = 0 $ and $ \omega = 0 $. Moreover, we again see a reciprocity between real space and momentum space: the finiteness of the chain determines discrete normal frequencies, while its discreteness determines a finite number of normal frequencies.

Consider now the limit for $ \abs{k} \ll \frac{\pi}{a} $, i.e. $ \lambda \gg 2a $: since the wavelength is much larger than the separation between atoms, the oscillation is not affected by the discreteness of the medium, hence the dispersion curve is linear:
\begin{equation*}
  \lim_{k \ra 0} \omega = \sqrt{\frac{C}{M}} a k \equiv v_\text{s} k
\end{equation*}
where $ v_\text{s} $ is the ``sound wave" in the 1D crystal considered. In general, the \bctxt{group velocity}, i.e. the velocity in the transfer of mechanical energy in the crystal, is defined as:
\begin{equation}
  v_\text{g}(k) \defeq \frac{\dd \omega}{\dd k}
\end{equation}
For the 1D crystal:
\begin{equation*}
  v_\text{g}(k) = \sgn(k) \sqrt{\frac{C}{M}} \cos \frac{k a}{2}
\end{equation*}
Note that the group velocity vanishes at the boundary $ v_\text{g}(\pm \frac{\pi}{a}) = 0 $. To understand this, consider the displacement of the atoms:
\begin{equation*}
  u_s(t) = u_0 e^{\pm \img \pi s - \omega t} = (-1)^s u_0 e^{- \img \omega t}
\end{equation*}
which is a stationary wave.

\section{3D crystal}

In the case of a 3D crystal, it is still possible to have both longitudinal and transversal: however, contrary to the 1D crystal, the elastic constant of the oscillation is an ``effective" elastic constant, since each atom has multiple ``closest" atoms.

Consider an oscillation with wave-vector $ k $ along a principal crystal direction in an sc crystal, WLOG $ [1 \ 0 \ 0] $. Since this is a wave, it can undergo Bragg scattering, and in particular there could be the phenomenon of \bctxt{back-scattering}, i.e. the wave is reflected backward: from the Bragg relation \eref{eq:bragg} with $ n = 1 $ and $ \vartheta = \frac{\pi}{2} $ (reflection), since the distance between two atoms in the $ [1 \ 0 \ 0] $ direction is $ d = a $, it is trivial to find:
\begin{equation*}
  \lambda = 2a
  \quad \iff \quad
  k = \pm \frac{\pi}{a}
\end{equation*}
Now, note that if the wave is totally reflected, the total oscillation given by the sum of the two waves is a stationary wave, hence the same result as for the 1D crystal is recovered.

If instead the oscillation propagates along the $ [1 \ 1 \ 0] $ crystal direction, then $ d = \frac{a}{\sqrt{2}} $ and the back-scattering condition becomes:
\begin{equation*}
  \lambda = \sqrt{2} a
  \quad \iff \quad
  k = \pm \sqrt{2} \frac{\pi}{a}
\end{equation*}
In the general case, the back-scattering happens for all wave-vector which lie on the boundary of the primitive cell of the reciprocal lattice, which is called \bctxt{first Brilluoin zone} (1BZ): indeed, it is possible to show that, given a wave-vector $ \ve{k} $ on the boundary of the 1BZ and its diffracted wave-vector $ \ve{k}' $ according to Bragg diffraction, then $ \norm{\ve{k}' - \ve{k}} = \frac{2\pi}{a} $, i.e. a stationary wave.

\subsection{Vibrational bands}

\begin{example}{Biatomic linear chain}{}
  Consider a 1D crystal where two kinds of atoms, with masses $ M_1 $ and $ M_2 $, alternate: the primitive cell is then composed of a couple of nearby atoms, and define $ s \in \N_0 $ the index of the primitive cells in the chain (of periodicity $ a $).

  In the first-order harmonic approximation where only nearby atoms interact with an elastic constant $ C $, denoting the displacement of the mass $ M_1 $ in the $ s $ cell as $ u_s $ and that of the mass $ M_2 $ as $ v_s $, the equations of motion read:
  \begin{equation*}
    M_1 \ddot{u}_s = C \left( v_s - 2 u_s + v_{s - 1} \right)
    \qquad \qquad
    M_2 \ddot{v}_s = C \left( u_{s + 1} - 2 v_s + u_s \right)
  \end{equation*}
  The plane-wave ansatz now reads:
  \begin{equation*}
    u_s(t) = u_0 \exp \left( \img s k a - \img \omega t \right)
    \qquad \qquad
    v_s(t) = v_0 \exp \left( \img s k a - \img \omega t + \img \varphi \right) \equiv v_0' \exp \left( \img s k a - \img \omega t \right)
  \end{equation*}
  with $ u_0 , v_0 \in \R $ and $ v_0' \in \C $. Inserting these expressions in the equations of motion:
  \begin{equation*}
    - \omega^2 M_1 = C \left[ v_0' \left( 1 + e^{-\img k a} \right) - 2 u_0 \right]
    \qquad \qquad
    - \omega^2 M_2 = C \left[ u_0 \left( e^{\img k a} + 1 \right) - 2 v_0' \right]
  \end{equation*}
  which reduce to a linear system:
  \begin{equation*}
    \begin{bmatrix}
      2C - \omega^2 M_1 & -C \left( 1 + e^{-\img k a} \right) \\
      -C \left( 1 + e^{\img k a} \right) & 2C - \omega^2 M_2
    \end{bmatrix}
    \begin{pmatrix}
      u_0 \\ v_0'
    \end{pmatrix}
    = \ve{0}
    \quad \implies \quad
    \begin{vmatrix}
      2C - \omega^2 M_1 & -C \left( 1 + e^{-\img k a} \right) \\
      -C \left( 1 + e^{\img k a} \right) & 2C - \omega^2 M_2
    \end{vmatrix}
    = 0
  \end{equation*}
  The equation for $ \omega^2 $ then is:
  \begin{equation*}
    M_1 M_2 \omega^4 - 2 C \left( M_1 + M_2 \right) \omega^2 + 2 C^2 \left( 1 - \cos ka \right) = 0
  \end{equation*}
  which in general has two solutions for $ \omega^2 $. The interesting behaviour is in the limits for $ k \ra 0 $ and $ k \ra \pm \frac{\pi}{a} $. For $ k \ra 0 $:
  \begin{equation*}
    M_1 M_2 \omega^2 - 2C \left( M_1 + M_2 \right) \omega^2 + C^2 (ka)^2 = 0
  \end{equation*}
  whose solutions are:
  \begin{equation*}
    \begin{split}
      \omega^2
      & = \frac{C \left( M_1 + M_2 \right) \pm \sqrt{C^2 \left( M_1 + M_2 \right)^2 - C^2 M_1 M_2 (ka)^2}}{M_1 M_2} \\
      & = \frac{C}{\mu} \left[ 1 \pm \sqrt{1 - \mu (ka)^2} \right] \approx \frac{C}{\mu} \left[ 1 \pm 1 \mp \frac{\mu}{2(M_1 + M_2)} (ka)^2 + \smo(k^3) \right]
    \end{split}
  \end{equation*}
  where $ \mu = M_1 M_2 / (M_1 + M_2) $ is the reduced mass of the primitive cell. Then, at first order:
  \begin{equation*}
    \omega_+ = \sqrt{\frac{2C}{\mu}} + \smo(k^2)
    \qquad \qquad
    \omega_- = \sqrt{\frac{C}{2(M_1 + M_2)}} a k + \smo(k^2) \equiv v_\text{s} k + \smo(k^2)
  \end{equation*}
  The $ \omega_- $ is analogous to that of the monoatomic linear chain, while $ \omega_+ $ is new. Substituting $ \omega_+ $ in the above linear system and setting $ k = 0 $:
  \begin{equation*}
    \begin{cases}
      \left( 1 - \frac{M_1}{\mu} \right) u_0 - v_0' = 0 \\
      - u_0 + \left( 1 - \frac{M_2}{\mu} \right) v_0' = 0
    \end{cases}
    \quad \iff \quad
    v_0' = - \frac{M_1}{M_2} u_0
  \end{equation*}
  This means that the two atoms in the primitive cell oscillate in phase-opposition, corresponding to a stationary wave: this justifies the dispersion curve being parallel to the $ k $-axis, since the group velocity of a stationary wave is $ v_\text{g} = 0 $.
\end{example}










