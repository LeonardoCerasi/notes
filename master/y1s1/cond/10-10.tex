\selectlanguage{english}

\chapter{Lecture 10/10}

\section{Hyperfine interaction}

Like the electrons, the nucleus too has a total angular momentum $ I $, given by the momenta of the individual protons and neutrons. We only consider the angular momentum of the ground state, since nuclear reactions (which change the spin state of the nucleus) in general require high energies.

There are three possible cases for the nuclear spin in the ground state, depending on the number of protons $ Z $ and neutrons $ N $:
\begin{itemize}
  \item $ Z $, $ N $ odd: $ I $ is an integer and the nucleus is a boson;
  \item $ Z $ odd, $ N $ even (or vice versa): $ I $ is a half-integer and the nucleus is a fermion;
  \item $ Z $, $ N $ even: $ I = 0 $.
\end{itemize}
Although the nucleus is a composite body, the magnetic moment associated to the nuclear spin is still proportional to $ I $, so we write:
\begin{equation}
  \bs{\mu}_I = g_I \frac{\born}{\hbar} \ve{I}
  \qquad \qquad
  \born \equiv \frac{e \hbar}{2m_p}
  \label{eq:nuclear-magnetic-moment-def}
\end{equation}
where we introduced the Landé factor for the nuclear spin $ g_I $ (which is to be determined experimentally: e.g. $ g_I \simeq 5.59 $ for a proton and $ g_I \simeq -3.83 $ for a neutron) and the nuclear Bohr magneton $ \born $.

Analogously to the spin-orbit coupling, which leads to the fine interaction and the fine-structure splitting, the angular momenta $ J $ and $ I $ couple too, leading to the \bctxt{hyperfine interaction}:
\begin{equation}
  \ham_\text{HF} = \frac{\xi_\text{HF}}{\hbar^2} \ve{I} \cdot \ve{J}
\end{equation}
We can then define the total atomic angular momentum $ \ve{F} \equiv \ve{I} + \ve{J} $, so that:
\begin{equation}
  \Delta E_\text{HF} = \frac{\xi_\text{HF}}{2} \left[ F(F+1) - I(I+1) - J(J+1) \right]
  \label{eq:hyperfine-correction}
\end{equation}
To qualitatively describe the hyperfine constant $ \xi_\text{HF} $, rewrite the hyperfine interaction Hamiltonian as:
\begin{equation*}
  \ham_\text{HF} = - \bs{\mu}_I \cdot \ve{B}_J = - g_I \frac{\born}{\hbar^2} \ve{I} \cdot \ve{B}_J = g_I \frac{\born}{\hbar^2} \ve{I} \cdot \frac{\ve{J}}{\norm{\ve{J}}} B_J = \frac{g_I \born B_J}{\hbar^2 \sqrt{J(J+1)}} \ve{I} \cdot \ve{J}
\end{equation*}
where we used the fact that $ \ve{B}_J $ and $ \ve{J} $ are antiparallel. Then:
\begin{equation}
  \xi_\text{HF} = \frac{g_I \born B_J}{\sqrt{J(J+1)}}
\end{equation}
The magnetic field $ \ve{B}_J $ generated by $ \ve{J} $ at the position of the nucleus is difficult to be measured directly; nonetheless, we expect this magnetic field to be larger for electrons closer to nucleus: indeed, for a mono-electron atom, it is possible to prove the following expression:
\begin{equation}
  \xi_\text{HF} \simeq \frac{2}{3} \mu_0 g_J \bor g_I \born \abs{\psi(\ve{0})}^2
\end{equation}
where $ \psi(\ve{r}) $ is the electron's wave-function. In general, the hyperfine correction is significant for $ \text{s} $-electrons.

\begin{example}{Hyperfine interaction in hydrogen}{}
  For $ \ch{^1H} $ we have the explicit expression in terms of the Bohr radius:
  \begin{equation*}
    \abs{\psi(\ve{0})}^2 = \frac{1}{\pi \borr^3}
    \qquad \qquad
    \borr \simeq 52.92 \,\text{pm}
  \end{equation*}
  Hence, since $ g_I \simeq 5.59 $ for the proton, we can compute the hyperfine constant for the ground state $ \ch{^2S_{\sfrac{1}{2}}} $:
  \begin{equation*}
    \xi_\text{HF} \approx 5.9 \,\mu\text{eV}
  \end{equation*}
  Moreover, in this case $ I = \frac{1}{2} $ and $ J = \frac{1}{2} $, hence the atomic spin can be either $ F = 0 $ or $ F = 1 $: from \eref{eq:hyperfine-correction}, these two states are separated by $ 5.9 \,\mu\text{eV} $, i.e. the transition between $ F = 1 $ and $ F = 0 $ corresponds to a well known spectral line $ \lambda \approx 21.0 \,\text{cm} $, which is of astrophysical interest.
\end{example}

\begin{example}{Hyperfine interaction in generic mono-electron atoms}{}
  For $ n\text{s} $ states in generic mono-electron atoms, the following first-order approximation holds:
  \begin{equation*}
    \abs{\psi(\ve{0})}^2 \simeq \frac{Z^3}{\pi n^3 \borr^3}
  \end{equation*}
  For other orbitals, instead:
  \begin{equation*}
    \xi_\text{HF} \simeq \frac{\mu_0}{4\pi} g_I \bor \born \frac{Z^3}{\borr^3 n^3 J (J+1) (2L+1)}
  \end{equation*}
\end{example}

\subsection{Hyperfine structure in magnetic fields}

To study how the hyperfine strucure is affected by magnetic fields, consider first that $ \xi \gg \xi_\text{HF} $ and $ \bor B \gg \born B $: the total ordering is then given by the relation between $ \xi_\text{HF} $ and $ \bor B $, i.e. either weak field $ \xi_\text{HF} \gg \bor B $ or strong field $ \bor B \gg \xi_\text{HF} $.

\subsubsection{Weak magnetic field}

In the weak-field regime we adopt the coupled basis with $ \ve{F} = \ve{I} + \ve{J} $. The atomic magnetic moment is then:
\begin{equation}
  \bs{\mu}_F = \bs{\mu}_I + \bs{\mu}_J \approx \bs{\mu}_J
\end{equation}
since $ \mu_I \propto \born \ll \bor $. Then, following the discussion in \secref{ssec:anomalous-zeeman}, we project $ \bs{\mu}_F $ onto $ \ve{F} $, since this is the only component which does not vanish in the mean over many precession periods:
\begin{equation*}
  (\mu_F)_F \approx \bs{\mu}_J \cdot \frac{\ve{F}}{\norm{\ve{F}}} = - g_J \frac{\bor}{\hbar} \ve{J} \cdot \frac{\ve{J} + \ve{I}}{F} = - g_J \frac{\bor}{2 \hbar F} \left[ \ve{F}^2 + \ve{J}^2 - \ve{I}^2 \right]
\end{equation*}
Identifying $ \bs{\mu}_F $ with this projection (an ``effective" magnetic moment), the energy correction due to the magnetic field $ \ve{B} = B \ve{e}_z $ is:
\begin{equation*}
  \ham_B = - \bs{\mu}_F \cdot \ve{B} = - (\mu_F)_F \frac{\ve{F}}{\norm{\ve{F}}} \cdot \ve{B} = \frac{g_J \bor B}{2\hbar F^2} \left[ F^2 + J^2 - I^2 \right] F_z
\end{equation*}
that is:
\begin{equation}
  \Delta E_B = \frac{g_J \bor B}{2 F(F+1)} m_F \left[ F(F+1) + J(J+1) - I(I+1) \right]
\end{equation}
This expression can be rewritten as:
\begin{equation}
  \Delta E_B = g_F m_F \bor B
  \label{eq:zeeman-hyperfine}
\end{equation}
with the Landé factor for the atomic spin defined as:
\begin{equation}
  g_F \equiv g_J \frac{F(F+1) + J(J+1) - I(I+1)}{2F(F+1)}
\end{equation}
Each level is first split by the spin-orbit interaction, then by the hyperfine interaction and finally by the additional splitting in \eref{eq:zeeman-hyperfine}.

\subsubsection{Strong magnetic field}

In this case, as per \obsref{obs:larmor-precession}, the precessions of $ \ve{I} $ and $ \ve{J} $ along $ \ve{B} $ can be treated as independent, hence we work in the uncoupled basis. The derivation is analogous to that in \secref{ssec:paschen-back}:
\begin{equation}
  \Delta E_B = g_J m_J \bor B - g_I m_I \born B + \xi_\text{HF} m_I m_J
  \label{eq:paschen-back-hyperfine}
\end{equation}
where the negative sign before in the first terms stems from the definition \eref{eq:nuclear-magnetic-moment-def}, which lacks a negative sign (since the nucleus is positively charged). Note that this is the most common occurrence, given that $ \xi_\text{HF} $ is smaller than $ \xi $.

Now, each level is first split by the spin-orbit interaction, then by electronic Zeeman splitting (first term of \eref{eq:paschen-back-hyperfine}), and finally by the hyperfine interaction (third term of \eref{eq:paschen-back-hyperfine}), with the additional correction of the nuclear Zeeman energy (second term of \eref{eq:paschen-back-hyperfine}).

\begin{example}{Measure of $ \xi_\text{HF} $ for hydrogen}{}
  Consider the ground state $ \ch{^2S_{\sfrac{1}{2}}} $ of $ \ch{^1H} $, which has $ I = \frac{1}{2} $, and apply a strong magnetic field. This state has no spin-orbit interaction, hence, according to \figref{fig:hyperfine-hydrogen}, it first gets split into two $ m_J = + \frac{1}{2} $ and $ m_J = - \frac{1}{2} $ levels by the magnetic field, and then each of these levels is split into two $ m_I = + \frac{1}{2} $ and $ m_I = - \frac{1}{2} $ by the hyperfine interaction. We then have four levels $ (m_J , m_I) $, with energies (setting $ E_{\ch{^1S_{\sfrac{1}{2}}}} \equiv 0 $):
  \begin{align*}
    E_{(+,+)} &= \frac{1}{2} \left( g_J \bor - g_I \born \right) B + \frac{\xi_\text{HF}}{4}
              &
    E_{(+,-)} &= \frac{1}{2} \left( g_J \bor + g_I \born \right) B - \frac{\xi_\text{HF}}{4}
    \\
    E_{(-,+)} &= - \frac{1}{2} \left( g_J \bor + g_I \born \right) B - \frac{\xi_\text{HF}}{4}
              &
    E_{(-,-)} &= - \frac{1}{2} \left( g_J \bor - g_I \born \right) B + \frac{\xi_\text{HF}}{4}
  \end{align*}
  Note that no electric-dipole transitions are possible between these four states, since the orbital angular momentum of the electron is always zero, but two magnetic-dipole transitions are allowed: the selection rule in this case is $ \Delta S = \pm 1 $, and additionally $ \Delta I = 0 $ since we exclude nuclear transitions, hence $ (+,+) \lra (-,+) $ and $ (+,-) \lra (-,-) $ are possible (see \figref{fig:hyperfine-hydrogen}). The energies of these transitions then are:
  \begin{equation*}
    \Delta E_{(+,+) \ra (-,+)} = g_J \bor B + \frac{\xi_\text{HF}}{2}
    \qquad \qquad
    \Delta E_{(+,-) \ra (-,-)} = g_J \bor B - \frac{\xi_\text{HF}}{2}
  \end{equation*}
  The hyperfine constant $ \xi_\text{HF} $ is thus found by measuring the splitting in the ESR (electron spin resonance) spectrum of $ \ch{^1H} $.
\end{example}

\begin{figure}
  \centering
  \includegraphics[width = 0.70 \textwidth]{hyperfine-hydrogen.png}
  \caption{Hyperfine structure of hydrogen in a strong magnetic field.}
  \label{fig:hyperfine-hydrogen}
\end{figure}










