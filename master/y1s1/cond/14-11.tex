\selectlanguage{english}

\chapter{Lecture 17/11}

\section{Bravais lattices}

\begin{definition}{Crystal}{}
  A \bcdef{crystal} is a collection of atoms with a translational symmetry.
\end{definition}

The presence of a translational symmetry means that it is possible to define a \bctxt{periodic lattice} (or Bravais lattice) $ \rsscript{B} $ and a \bctxt{basis}, i.e. a set of atoms which repeats in each point of the lattice.

Given the periodicity of the lattice, it is possible to write the general element of the lattice as:
\begin{equation*}
  \ve{R} = n_1 \ve{R}_1 + n_2 \ve{R}_2 + n_3 \ve{R}_3
  \label{eq:bravais}
\end{equation*}
where $ n_1 , n_2 , n_3 \in \Q $ and $ \ve{R}_1 , \ve{R}_2 , \ve{R}_3 \in \R^3 $ LI are the principal directions or \emph{generators} of the lattice. In particular, if the generators are the ``smallest" possible (non-unique definition), they form a \bctxt{primitive cell}, i.e. the minimal volume which contains all translationally-unequivalent points of the lattice: clearly, the whole lattice is a repetition of the primitive cell under discrete translations.

\begin{proposition}{Wigner--Seitz cell}{}
  Given a point in a Bravais lattice $ \ve{R} \in \rsscript{B} $, the \bcprop{Wigner--Seitz cell} in $ \ve{R} $ is defined as the cell $ V_\text{WS} \defeq \{\ve{r} \in \R^3 : \norm{\ve{r} - \ve{R}} < \norm{\ve{r} - \ve{R}'} \,\, \forall \ve{R}' \in \rsscript{B} - \{\ve{R}\}\} $. This is a primitive cell of $ \rsscript{B} $.
\end{proposition}

The Wigner--Seitz cell eliminates the arbitrariness of the primitive cell. However, it is convenient to consider conventional (or unitary) cells, which are composed of one or more primitive cells to obtain a macro-cell with a simple shape (preferably cubic, like for the sc, bcc anf fcc lattices, see \figref{fig:ws-fcc-bcc}).

\begin{figure}
  \centering
  \begin{subfigure}{0.4835 \textwidth}
    \centering
    \includegraphics[width = \textwidth]{primitive.png}
  \end{subfigure}
  \begin{subfigure}{0.5065 \textwidth}
    \centering
    \includegraphics[width = \textwidth]{wigner-seitz.png}
  \end{subfigure}
  \caption{Arbitrary primitive cells and Wigner--Seitz cells for the fcc and bcc lattices.}
  \label{fig:ws-fcc-bcc}
\end{figure}

All of the physical information of the lattice is present in the Wigner--Seitz cell (or in a general primitve cell): nontheless, we study conventional cell for their simpler geometries, even though they present redundant information. In general, fixed an atom in the conventional cell as the origin, we denote the position of the other atoms as $ (n_1 , n_2 , n_3) $, with coefficients given by \eref{eq:bravais}; on the contrary, \bctxt{directions} are denoted by $ [h \ k \ \ell] $, with $ h , k , \ell \in \Z $ (with the same meaning as in \eref{eq:bravais}), while \bctxt{crystal planes} by $ (h \ k \ \ell) $, with Miller indices $ h , k , \ell \in \Z $ being the minimal integer multiples of the inverses of the plane's intercepts with the principal directions of the conventional cell. Note that both the notations for directions and crystal planes do not univocally define directions and crystal planes, but rather collections of equivalent (i.e. parallel) directions and crystal planes.

\begin{lemma}[before upper = {\tcbtitle}]{Directions and crystal planes}{}
  \begin{equation}
    [h \ k \ \ell] \perp (h \ k \ \ell) \quad \forall h , k , \ell \in \Z
  \end{equation}
\end{lemma}

\begin{example}{Crystal planes in cubic lattices}{}
  Consider a lattice with a cubic unitary cell with sides of length $ a $. Then, given a collection of crystal planes with Miller indices $ (h \ k \ \ell) $, the distance between two consecutive planes in this collection is:
  \begin{equation}
    d_{h,k,\ell} = \frac{a}{\sqrt{h^2 + k^2 + \ell^2}}
  \end{equation}
\end{example}

\subsection{Diffraction}












